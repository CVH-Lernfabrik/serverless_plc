
\section*{Kurzfassung}
%Üblicherweise enthalten wissenschaftliche Artikel ein Abstract --
%  % "--" ist ein Gedankenstrich. Er ist länger und dünner als ein
%  % Bindestrich und hat eine andere Bedeutung.
%typischerweise von 100 bis 150 Wörtern,
%ohne Bilder und Literaturzitate und in einem Absatz.
%In dieser Vorlage gliedert sich das Abstract in diese Erläuterung
%sowie ein „echtes” Abstract, das nun als ein zweiter Absatz folgt.
%  % Die Anführungszeichen um „echtes“ sind UTF-8-Anführungszeichen unten und oben.
%  % Dasselbe Ergebnis kann man auch mit "`echtes"' oder \glqq{}echtes\grqq{}
%  % erreichen, bei Verwendung des csquotes-Pakets auch mit \enquote{echtes}.
%
%\LaTeX\ ist ein Textsatzsystem,
%  % Hinter Macros wie z.B. "\LaTeX" werden Leerzeichen normalerweise ignoriert.
%  % Damit dort trotzdem ein Leerzeichen erscheint, ergänzen wir dieses explizit
%  % durch den Macro "\ " – ein Backslash, gefolgt von einem Leerzeichen.
%das sich vor allem für die Erstellung wissenschaftlicher Arbeiten etabliert hat.
%Diese Vorlage demonstriert einige wichtige Grundlagen von \LaTeX\
%anhand einer Beispiel-Bachelor-Arbeit. Dies geschieht sowohl im
%fertigen Text (also diesem) als auch im \LaTeX-Quelltext,
%  % also diesem, ;-)
%typischerweise in Gestalt von Kommentaren.

% Auf maximal einer Seite ist eine Kurzfassung der Arbeit anzugeben. Diese muss
% insbesondere die Motivation für die Arbeit und die zugrunde liegenden Fragestellungen,
% die gewählte Vorgehensweise und durchgeführten Aktivitäten sowie das Ergebnis der Arbeit enthalten.
% Das Abstract wird im Internet veröffentlicht.

% Am Markt sind zahlreiche Lösungen zur teleoperierten Steuerung von Fahrzeugen kommerziell erhältlich.
% Hierbei handelt es sich in der Regel um Systeme zur Steuerung von Sonder- oder Spezialfahrzeugen,
% wie etwa Schwertransport- oder Bergungsfahrzeuge. Diese Systeme verwenden unterschiedliche,
% meist drahtlose Punkt-zu-Punkt Verbindungen zur Übertragung der Steuersignale.
% Auch in der Filmindustrie existieren unterschiedliche Ansätze zur teleoperierten Steuerung
% von Fahrzeugen, etwa zur Durchführung von Fahrmanövern welche für Fahrzeuginsassen zu gefährlich wären.
