% % Imports nur für Referenzenauflösung während des Schreibens! Vorm Kompilieren auskommentieren!
% \bibliography{0_hauptdatei}
% % Mit \section{...} eröffnen wir einen neuen Abschnitt.
% Der Befehl setzt nicht nur den Text in einer größeren,
% fetten Schrift, sondern sorgt außerdem dafür, daß er im
% Inhaltsverzeichnis erscheint.
%
% Mit \label{...} erzeugen wir einen Bezeichner, mit dessen Hilfe
% wir später auf die Nummer des Abschnitts verweisen können (nämlich
% mit~\ref{...}).
%
% Das Kommentarzeichen hinter „Übersicht“ dient dazu, ein
% Leerzeichen zwischen „Übersicht“ und dem \label-Befehl
% zu vermeiden, das andernfalls sichtbar würde – z.B. im
% Inhaltsverzeichnis.
%

% % Imports nur für Referenzenauflösung während des Schreibens! Vorm Kompilieren auskommentieren!
% \bibliography{0_hauptdatei}
% % Mit \section{...} eröffnen wir einen neuen Abschnitt.
% Der Befehl setzt nicht nur den Text in einer größeren,
% fetten Schrift, sondern sorgt außerdem dafür, daß er im
% Inhaltsverzeichnis erscheint.
%
% Mit \label{...} erzeugen wir einen Bezeichner, mit dessen Hilfe
% wir später auf die Nummer des Abschnitts verweisen können (nämlich
% mit~\ref{...}).
%
% Das Kommentarzeichen hinter „Übersicht“ dient dazu, ein
% Leerzeichen zwischen „Übersicht“ und dem \label-Befehl
% zu vermeiden, das andernfalls sichtbar würde – z.B. im
% Inhaltsverzeichnis.
%

% % Imports nur für Referenzenauflösung während des Schreibens! Vorm Kompilieren auskommentieren!
% \bibliography{0_hauptdatei}
% % Mit \section{...} eröffnen wir einen neuen Abschnitt.
% Der Befehl setzt nicht nur den Text in einer größeren,
% fetten Schrift, sondern sorgt außerdem dafür, daß er im
% Inhaltsverzeichnis erscheint.
%
% Mit \label{...} erzeugen wir einen Bezeichner, mit dessen Hilfe
% wir später auf die Nummer des Abschnitts verweisen können (nämlich
% mit~\ref{...}).
%
% Das Kommentarzeichen hinter „Übersicht“ dient dazu, ein
% Leerzeichen zwischen „Übersicht“ und dem \label-Befehl
% zu vermeiden, das andernfalls sichtbar würde – z.B. im
% Inhaltsverzeichnis.
%

% % Imports nur für Referenzenauflösung während des Schreibens! Vorm Kompilieren auskommentieren!
% \bibliography{0_hauptdatei}
% \input{1_einleitung}
%\input{2_grundlagen}
%\input{3_konzeption}
%\input{4_implementierung}
%\input{5_tests}
%\input{6_zusammenfassung}
% % Ende Imports

\section{Einleitung und Motivation%
  \label{sec:1-einleitung}}
Ziel dieses Projektes ist die Integration eines OPC-Servers mit einer auf Linux
basierenden speicherprogrammierbaren Steuerung (SPS). Angeschlossen an diese SPS
ist jeweils ein digitales Ein-/\,bzw.~Ausgabemodul. Die von diesen bereitgestellten
Ein-/\, bzw.~Ausgänge (IO) sollen in der Datenstruktur des OPC-Servers abgebildet
und über diesen für OPC-Clients les-/\,und schreibar sein. Weiterhin sollen einige
Funktionen zur Überwachung und Steuerung der an die SPS angeschlossenen Aktoren
und Sensoren direkt im OPC-Server implementiert werden.
Hiermit stellt dieses Projekt eine der Grundlagen für ein übergeordnetes Projekt,
die cloudbasierte Steuerung eines miniaturisierten Produktions-Systems, dar.

Der hier verwendete OPC-Server ist Teil des sog. open62541 Projekts. Er ist in C
geschrieben und implementiert bereits einen großen Teil der im OPC-UA-Standard
spezifizierten Funktionen.
Als SPS findet ein Revolution Pi 3 der Firma Kunbus Verwendung. Dieser integriert
ein sog. Compute Module der Raspberry Pi Foundation in ein industrietaugliches
Gehäuse und erlaubt die Erweiterung mittels IO- oder Gateway-Modulen. Über diese
erfolgt die Kommunikation mit weiteren Komponenten der Automatisierungstechnik.

Motiviert ist dieses Projekt durch die Beobachtung, dass die Verbreitung offener
Standards sowie freier Software auch in der Automatisierungstechnik zunimmt.
Linux ist ein freies Betriebssystem, OPC-UA ein offen zugänglicher, aktiv gepflegter
und weit verbreiteter Standard. Der Raspberry Pi findet sowohl bei Hobby-Anwendern als
auch in den Bereichen Forschung und Entwicklung sowie bei industriellen Anwendern
Verwendung. Dieses Projekt stellt somit eine für unterschiedliche Anwender interessante
Entwicklung dar.

Im Anschluss an diese einleitende Übersicht im Abschnitt~\ref{sec:1-einleitung} folgt
die Darstellung der wichtigsten Grundlagen in Abschnitt~\ref{sec:2-grundlagen}.
Aufbauend auf diesen Grundlagen folgt die konzeptuelle Ausarbeitung im Abschnitt~\ref{sec:3-konzeption}.
Die Umsetzung wird im Abschnitt~\ref{sec:4-implementierung} erläutert.
Die Leistungsfähigkeit der Implementierung wird in Abschnitt~\ref{sec:5-tests} untersucht.
Eine Zusammenfassung und ein Ausblick schließen die Arbeit in
Abschnitt~\ref{sec:6-fazit} ab. Eventuell noch benötigte Anhänge
finden sich in den Anhängen [...] bis [...].

%% % Imports nur für Referenzenauflösung während des Schreibens! Vorm Kompilieren auskommentieren!
% \bibliography{0_hauptdatei}
% \input{1_einleitung}
% \input{2_grundlagen}
% \input{3_konzeption}
% \input{4_implementierung}
% \input{5_tests}
% \input{6_zusammenfassung}
% % Ende Imports

\section{Grundlagen%
  \label{sec:2-grundlagen}}

\subsection{Speicherprogrammierbare-Steuerung und Linux -- Revolution Pi%
     \label{sec:2-sps}}

\subsubsection{Kunbus RevolutionPi%
        \label{sec:2-revpi}}
Der RevolutionPi 3 ist eine speicherprogrammierbare Steuerung (SPS) des Herstellers
Kunbus GmbH. Kern dieser SPS ist das von der Raspberry Pi Foundation entwickelte
und vertriebene Raspberry Pi Compute Module 3. Dieses integriert ein Broadcom BCM2837
System-on-Chip (SoC) mit vier 1,2GHz Prozessorkernen, 1GB RAM, 4GB eMMC Anwendungsspeicher
und sonstige Peripherie in ein Modul im DDR2-SODIMM Formfaktor. Diese Spezifikationen
sind weitgehend identisch zu denen des ausgesprochen populären Raspberry Pi 3.
Der Revolution Pi profitiert daher von dem gleichen großen Angebot an Software
und Unterstützung wie der Raspberry Pi, ergänzt dessen Hardware jedoch um eine 24V
Spannungsversorgung, die Möglichkeit der Erweiterung durch mehrere industrietaugliche
Ein-/ Ausgabemodule und Gateways sowie ein Gehäuse zur Montage auf einer DIN-Schiene.
\begin{itemize}
  \item{Prozessor: BCM2837}
  \item{Taktfrequenz 1,2 GHz}
  \item{Anzahl Prozessorkerne: 4}
  \item{Arbeitsspeicher: 1 GByte}
  \item{eMMC Flash Speicher: 4 GByte}
  \item{Betriebssystem: Angepasstes Raspbian mit RT-Patch}
  \item{RTC mit 24h Pufferung über wartungsfreien Kondensator}
  \item{Treiber / API: Treiber schreibt zyklisch Prozessdaten in ein Prozessabbild, Zugriff auf Prozessabbild über Linux-Filesystem als API zu Fremdsoftware.}
  \item{Kommunikationsanschlüsse: 2 x USB 2.0 A (je 500 mA belastbar), 1 x Micro-USB, HDMI, Ethernet (RJ45) 10/100 Mbit/s}
  \item{Stromversorgung: min. 10,7 V, max. 28,8 V, maximal 10 Watt}
  \item{Zulässige Umgebungstemperatur: -40 bis +55 C}
  \item{Gehäuseabmessungen: (HxBxL) 96 mm x 22,5 mm x 110,5 mm (ohne gesteckte Stecker)}
  \item{ESD Schutz: 4 kV / 8 kV gemäß EN61131-2 und IEC 61000-6-2}
  \item{Surge / Burst Prüfungen: gemäß EN61131-2 und IEC 61000-6-2 eingekoppelt auf Versorgungsspannung, Ethernet und IO-Leitungen}
  \item{EMI Prüfungen: gemäß EN61131-2 und IEC 61000-6-2}
\end{itemize}

Kunbus bietet eine Auswahl an IO- und Gateway-Modulen zur Erweiterung des Revolution Pi an.
Gateways dienen der Kommunikation mit Systemen oder Komponenten der Automatisierungstechnik
über Protokolle wie PROFIBUS oder EtherCAT. IO-Module erlauben die Überwachung
und Steuerung von digitalen oder analogen Ein- und Ausgängen.

\subsubsection{Zugriff auf IO-Module%
        \label{sec:2-io}}
Der Zugriff auf die Ein- und Ausgänge der IO-Module erfolgt über ein Prozessabbild
und einen hierfür von Kunbus bereitgestellten Treiber, genannt piControl. Dieser
aktualisiert das Prozessabbild zyklisch. Die angestrebte Zykluszeit beträgt 5ms,
kann jedoch je nach Anzahl der angeschlossenen Module auch größer sein. Kunbus
garantiert bei drei IO-Modulen und zwei Gateway-Modulen eine Zykluszeit von 10 ms.
Jedes der IO-Module stellt ein eigenständiges eingebettetes System dar. Es verfügt
über einen Microcontroller, welcher die IOs bereitstellt und über einen RS485-Bus
mit dem Revolution Pi kommuniziert.
% https://revolution.kunbus.de/io-modul/

Lizenz: GPL
% https://github.com/RevolutionPi/piControl

\begin{lstlisting}[language={c},firstnumber={226},caption={Setzen der Scheduler-Priorität auf SCHED\_FIFO in revpi\_common.c\label{lst:2-sched_priority}}]
param.sched_priority = ktprio->prio;
ret = sched_setscheduler(child, SCHED_FIFO,
       &param);
\end{lstlisting}


\subsection{Echtzeit und Multithreading unter Linux -- preemptRT und posix%
     \label{sec:2-echtzeit}}


 Der Linux-Kernel verfügt über mehrere unterschiedliche Preemtion-Modelle:

\begin{itemize}
  \item No Forced Preemption (server):
  Ausgelegt auf maximal möglichen Durchsatz, lediglich Interrupts und
  System-Call-Returns bewirken Präemption.

  \item Voluntary Kernel Preemption (Desktop):
  Neben den implizit bevorrechtigten Interrupts und System-Call-Returns gibt es
  in diesem Modell weitere Abschnitte des Kernels in welchen Preämption explizit
  gestattet ist.

  \item Preemptible Kernel (Low-Latency Desktop):
  In diesem Modell ist der gesamte Kernel, mit Ausnahme sog.~kritischer Abschnitte
  präemptible. Nach jedem kritischen Abschnitt gibt es einen impliziten Präemptions-Punkt.

  \item Preemptible Kernel (Basic RT):
  Dieses Modell ist dem zuvor genannten sehr ähnlich, hier sind jedoch alle Interrupt-Handler
  als eigenständige Threads ausgeführt.

  \item Fully Preemptible Kernel (RT):
  Wie auch bei den beiden zuvor genannten Modellen ist hier der gesamte Kernel
  präemtible, die Anzahl und Dauer der nicht-präemtiblen kritischen Abschnitte
  ist auf ein notwendiges Minimum beschränkt. Alle Interrupt-Handler sind als
  eigenständige Threads ausgeführt, Spinlocks durch Sleeping-Spinlocks und Mutexe
  durch sog.~RT-Mutexe ersetzt.

\end{itemize}
\todo{Spinlocks und Mutexe sowie die RT-Varianten dieser erklären!}

Lediglich mit dem vollständig präemtiblen Kernel kann Echtzeit-Verhalten realisiert werden.

% https://wiki.linuxfoundation.org/realtime/documentation/technical_basics/preemption_models bzw kernel/Kconfig.preempt

\subsubsection{preemptRT%
        \label{sec:2-preemptRT}}
% https://wiki.linuxfoundation.org/realtime/documentation/technical_details/start
% https://wiki.linuxfoundation.org/realtime/documentation/technical_basics/start

Das dem PREEMPT RT Kernel zugrunde liegende Prinzip lässt sich in einer einfachen
Regel ausdrücken: Nur Code, welcher absolut nicht-präemtible sein darf, ist es
gestattet nicht-präemtible zu sein.
Das erklärte Ziel des PREEMPT\_RT Patches ist es folglich, die Menge des nicht-präemtiblen
Codes im Linux-Kernel auf das absolut notwendige Minimum zu reduzieren.

Dies wird durch Verwendung folgender Mechanismen erreicht:

\begin{itemize}
  \item Hochauflösende Timer
  \item Sleeping Spinlocks
  \item Threaded Interrupt Handlers
  \item rt\_mutex
  \item RCU
\end{itemize}


\subsubsection{posix%
        \label{sec:2-posix}}
Ist posix hier wirklich relevant? Debian bzw.~Raspbian sind weitgehend posix
kompatibel, aber wird es hier genutzt? -> JA, open62541 nutzt pthread.h
piControl nutzt kthread.h, und semaphore.h

\subsection{OPC-UA und open62541%
     \label{sec:2-opc}}

\subsubsection{OPC UA%
        \label{sec:2-opcua}}
Open Platform Communications (OPC) ist eine Familie von Standards zur herstellerunabhängigen
Kommunikation von Maschinen (M2M) in der Automatisierungstechnik. Die sog.~OPC Task Force, zu deren
Mitgliedern verschiedene große Firmen der Automatisierungsindustrie gehören, veröffentlichte
die OPC Specification Version 1.0 im August 1996.
Motiviert ist dieser offene Standard durch die Erkenntniss, dass die Anpassung der
zahlreichen Herstellerstandards an individuelle Infrastrukturen und Anlagen einen
großen Mehraufwand verursachen.
Die Wikipedia beschreibt das Anwendungsgebiet für OPC wie folgt:

\glqq{}OPC wird dort eingesetzt, wo Sensoren, Regler und Steuerungen verschiedener Hersteller
ein gemeinsames Netzwerk bilden. Ohne OPC benötigten zwei Geräte zum Datenaustausch
genaue Kenntnis über die Kommunikationsmöglichkeiten des Gegenübers. Erweiterungen
und Austausch gestalten sich entsprechend schwierig. Mit OPC genügt es, für jedes
Gerät genau einmal einen OPC-konformen Treiber zu schreiben. Idealerweise wird
dieser bereits vom Hersteller zur Verfügung gestellt. Ein OPC-Treiber lässt sich
ohne großen Anpassungsaufwand in beliebig große Steuer- und Überwachungssysteme
integrieren.

OPC unterteilt sich in verschiedene Unterstandards, die für den jeweiligen Anwendungsfall
unabhängig voneinander implementiert werden können. OPC lässt sich damit verwenden
für Echtzeitdaten (Überwachung), Datenarchivierung, Alarm-Meldungen und neuerdings
auch direkt zur Steuerung (Befehlsübermittlung).\grqq{}

OPC basiert in der ursprünglichen Spezifikation auf Microsofts DCOM-Spezifikation.
DCOM macht Funktionen und Objekte einer Anwendung anderen Anwendungen im Netzwerk
zugänglich. Der OPC-Standard definiert entsprechende DCOM-Objekte um mit anderen
OPC-Anwendungen Daten austauschen zu können. Die Verwendung von DCOM bindet Anwender
an Betriebssysteme von Microsoft. Die ursprüngliche OPC Spezifikation wird durch die
Entwicklung von OPC Unified Architecture (OPC UA) abgelöst.
OPC UA setzt auf einem eigenen Kommunikationionsstack auf, die Verwendung von DCOM
und damit die Bindung an Microsoft wurden aufgelöst.

Die OPC-UA-Architektur ist eine Service-orientierte Architektur (SOA), deren Struktur
aus mehreren Schichten besteht.

% Wikipedia
Das OPC-Informationsmodell ist nicht mehr nur eine Hierarchie aus Ordnern, Items
und Properties. Es ist ein sogenanntes Full-Mesh-Network aus Nodes, mit dem neben
den Nutzdaten eines Nodes auch Meta- und Diagnoseinformationen repräsentiert werden.
Ein Node ähnelt einem Objekt aus der objektorientierten Programmierung. Ein Node
kann Attribute besitzen, die gelesen werden können (Data Access (DA), Historical
Data Access (HDA)). Es ist möglich Methoden zu definieren und aufzurufen.
Eine Methode besitzt Aufrufargumente und Rückgabewerte. Sie wird durch ein Command
aufgerufen. Weiterhin werden Events unterstützt, die versendet werden können
(AE (Alarms \& Events), DA DataChange), um bestimmte Informationen zwischen Geräten
auszutauschen. Ein Event besitzt unter anderem einen Empfangszeitpunkt, eine Nachricht
und einen Schweregrad. Die o. g. Nodes werden sowohl für die Nutzdaten als auch
alle anderen Arten von Metadaten verwendet. Der damit modellierte OPC-Adressraum
beinhaltet nun auch ein Typmodell, mit dem sämtliche Datentypen spezifiziert werden.

% https://de.wikipedia.org/wiki/Open_Platform_Communications
% https://de.wikipedia.org/wiki/OPC_Unified_Architecture
% https://opcfoundation.org/developer-tools/specifications-unified-architecture
% Von Gerhard Gappmeier - ascolab GmbH, CC BY-SA 3.0, https://de.wikipedia.org/w/index.php?curid=1892069
\subsubsection{open62541%
        \label{sec:2-open62541}}
open62541 ist eine offene und freie Implementierung von OPC UA. Die in C geschriebene
Bibliothek stellt eine beständig zunehmende Anzahl der im OPC UA Standard definierten
Funktionen bereit. Sie kann sowohl zur Erstellung von OPC-Servern als auch -Clients
genutzt werden. Ergänzend zu der unter der Mozilla Public License v2.0 lizensierten
Bibliothek stellt das open62541 Projekt auch Beispielprogramme unter einer CC0 Lizenz
zur Verfügung.

Die Bibliothek eignet sich auch für die Entwicklung auf eingebetteten Systemen und
Microcontrollern. Je nach Umfang der gewünschten Funktionen und des OPC Informationsmodells
beträgt die Größe einer Server-Binary weniger als 100kb. %evtl. kürzen?

\todo{Nodes erklären! Evtl.~oben!}

Folgende Auswahl an Eigenschaften und Funktionen zeichnet die in dieser Arbeit verwendete
Version 0.3 von open62541 aus:
\begin{itemize}
  \item Kommunikationionsstack
  \begin{itemize}
      \item OPC UA Binär-Protokoll (HTTP oder SOAP werden gegenwärtig nicht unterstützt)
      \item Austauschbare Netzwerk-Schicht, welche die Verwendung eigener Netzwerk-APIs
      erlaubt.
      \item Verschlüsselte Kommunikationion
      \item Asynchrone Dienst-Anfragen im Client
  \end{itemize}
  \item Informationsmodell
  \begin{itemize}
    \item Unterstützung aller OPC UA Node-Typen, inkl.~Methoden
    \item Hinzufügen und Entfernen von Nodes und Referenzen zur Laufzeit.
    \item Vererbung und Instanziierung von Objekt- und Variablentypen
    \item Zugriffskontrolle auch für einzelne Nodes
  \end{itemize}
  \item Subscriptions
  \begin{itemize}
    \item Erlaubt die Überwachung (subscriptions / monitoreditems)
    \item Sehr geringer Ressourcenbedarf pro überwachtem Wert
  \end{itemize}
  \item Code-Generierung auf XML-Basis
  \begin{itemize}
    \item Erlaubt die Erstellung von Datentypen
    \item Erlaubt die Generierung des serverseitigen Informationsmodells
  \end{itemize}
\end{itemize}

% https://open62541.org/doc/0.3/


Mozilla Public License
CC0 Lizenz für Beispiele und Plugins

% https://open62541.org/doc/open62541-current.pdf
% https://open62541.org/

%% % Imports nur für Referenzenauflösung während des Schreibens! Vorm Kompilieren auskommentieren!
% \bibliography{0_hauptdatei}
% \input{1_einleitung}
% \input{2_grundlagen}
% \input{3_konzeption}
% \input{4_implementierung}
% \input{5_tests}
% \input{6_zusammenfassung}
% \input{anhang}
% % Ende Imports

\section{Systemkonzept%
  \label{sec:3-konzeption}}
Auf Basis der in Abschnitt \ref{sec:2-grundlagen} vorgestellten Möglichkeiten folgt nun die Ausarbeitung eines Konzepts.
In den folgenden Abschnitten soll näher auf zwei zentrale Aspekte eingegangen werden: Abschnitt~\ref{sec:3-anbindung} stellt Möglichkeiten zum Zugriff auf Variablen bzw.\,Werte im Prozessabbild des Revolution Pi vor; in Abschnitt~\ref{sec:3-integration} wird ein Konzept zur Bereitstellung dieser Variablen auf einem OPC-Server vorgestellt.

\subsection{Anbindung der IO an den OPC-Server%
     \label{sec:3-anbindung}}

Eine Webanwendung mit Bezeichnung PiCtory dient zur Konfiguration der I/O- und virtuellen Module des RevolutionPi. Die Konfiguration liegt im JSON-Format in der Datei \lstinline{/etc/revpi/config.rsc}. Der piControl-Treiber liest diese Datei beim Start. 
Der folgende Auszug aus der Manpage des piControl-Kernelmoduls beschreibt die von diesem zum Lesen und Schreiben einzelner Bits des Prozessabbildes bereitgestellten Funktionen~\citep[vgl.]{web-revpi-manpage}. Sie ist an dieser Stelle weitgehend ungekürzt zitiert, da sie die nutzbare Schnittstelle sehr kompakt beschreibt.

\begin{lstlisting}[breakindent=0pt, numbers=none, caption={Auszug aus der Revolution Pi Programmers Manual\label{lst:4-manpage}}]
KB_FIND_VARIABLE SPIVariable *argp
Find a variable in the process image by its name. A pointer to a structure of type SPIVariable must be passed as argument. [...]
The struct SPIVariable [...] is defined as 
typedef struct SPIVariableStr
{
    char strVarName[32]; // Variable name
    uint16_t i16uAddress; // Address of the byte in the process image
    uint8_t i8uBit; // 0-7 bit position, >= 8 whole byte
    uint16_t i16uLength; // length of the variable in bits.
    // Possible values are 1, 8, 16 and 32
} SPIVariable;

Set and get values of the process image
KB_GET_VALUE SPIValue *argp
[...]
KB_SET_VALUE SPIValue *argp
Write one bit or one byte to the process image [...].  This call is more efficient than the usual calls of seek and write because only one function call is necessary. If more than on application are writing bits in one output byte, this call is the only safe way to set a bit without overwriting the other bits because this call is doing a read-modify-write-cycle. 

The struct SPIValue used by this ioctl is defined as
typedef struct SPIValueStr
{
    uint16_t i16uAddress; // Address of the byte in the process image
    uint8_t i8uBit; // 0-7 bit position, >= 8 whole byte
    uint8_t i8uValue; // Value: 0/1 for bit access, whole byte otherwise
} SPIValue;
\end{lstlisting} 

Die oben beschriebenden Funtkionen \lstinline{KB_FIND_VARIABLE}, \lstinline{KB_GET_VALUE} und \lstinline{KB_SET_VALUE} ermöglichen einen einfachen und (lt.\,Manpage) effizienten Zugriff auf einzelne Bits des Prozessabbildes und damit auch auf die IO des RevolutionPi.
Der Zugriff des OPC-Servers auf das Prozessabbild soll daher mittels dieser Funktionen realisiert werden.
\lstinline{KB_FIND_VARIABLE} kann genutzt werden, um Adressen von Variablen im Prozessabbild mittels ihres Namens aufzulösen.
\lstinline{KB_GET_VALUE} und \lstinline{KB_SET_VALUE} ermöglichen den Zugriff auf die Werte dieser Variablen.


\subsection{Integration des OPC-Servers in das System%
     \label{sec:3-integration}}

open62541 bietet drei Möglichkeiten zum Abgleich von Variablen mit dem Prozessabbild~\citep[vgl.][Tutorials - Connecting a Variable with a Physical Process]{web-open62541}:
\begin{itemize}
    \item Manuelles oder zyklisches Aktualisieren
    \item Variable Value Callback
    \item Variable Datasource
\end{itemize}

Die zyklische Aktualisierung eines oder mehrerer Werte nimmt, abhängig von der Zykluszeit, viele Systemressourcen in Anspruch. Value Callbacks ermöglichen es, einen Variablenwert effizienter mit einer Ressource wie etwa einem Prozessabbild zu synchronisieren. An die Variable wird ein Callback angehängt, welches vor jedem Lesen und nach jedem Schreibvorgang ausgeführt wird.
Der Wert der Variablen wird weiterhin im Variablenknoten auf dem OPC-Server gespeichert, der Abgleich mit der verknüpften Ressource erfolgt durch die Callback-Methoden.

Sogenannte Datenquellen gehen noch einen Schritt weiter. Der Server leitet jede Lese- und Schreibanforderung direkt an eine Callback-Funktion weiter. Beim Lesen liefert der Rückruf eine Kopie des aktuellen Wertes. Die Datenquelle muss intern ein eigenes Speichermanagement implementieren.

Der Zugriff auf die Werte des Prozessabbildes erfolgt, wie in Abschnitt~\ref{sec:3-anbindung} beschrieben, über von piControl bereitgestellte Methoden. Um die durch open62541 gepflegte OPC-Datenstruktur und das durch piControl verwaltete Prozessabbild möglichst effektiv verknüpfen zu können, soll diese Interaktion mittels Datenquellen und den zugehörigen Callbacks implementiert werden.
%% % Imports nur für Referenzenauflösung während des Schreibens! Vorm Kompilieren auskommentieren!
% \bibliography{0_hauptdatei}
% \input{1_einleitung}
% \input{2_grundlagen}
% \input{3_konzeption}
% \input{4_implementierung}
% \input{5_tests}
% \input{6_zusammenfassung}
% \input{anhang}
% % Ende Imports

\section{Implementierung%
  \label{sec:4-implementierung}}
Das folgende Kapitel stellt in Auszügen die Implementierung des OPC-Servers sowie die Anbindung an die IO-Module
der SPS dar. Der Schwerpunkt liegt hierbei auf der Funktionsweise des piControl-Treibers und dessen Integration in das Projekt. Abschnitt~\ref{sec:4-picontrol} erklärt die zum Schreibens eines Bits verwendeten Funktionsaufrufe.
Zuvor soll jedoch in Abschnitt~\ref{sec:4-open62541} der Teil des OPC-Servers vorgestellt werden, welcher auf besagten Treiber zugreift. 

\subsection{Implementierung des OPC-Servers%
     \label{sec:4-open62541}}
Wie im vorangegangenen Abschnitt~\ref{sec:3-integration} begründet, soll die Verknüpfung zwischen dem Prozessabbild der SPS und den auf dem OPC-Server bereitgestellten Werten über sog.\,Datenquellen erfolgen. Hierzu ist zunächst eine Callback-Methode zu implementieren, welche bei einem Lese- oder Schreibzugriff auf eine Variable aufgerufen wird. Die Verknüpfung zwischen Callback-Methode und Variable muss manuell erfolgen.

\begin{lstlisting}[language={c},firstnumber=237,caption={Auszug der Methode \lstinline{linkDataSourceVariable} in \lstinline{variables.c}\label{lst:4-linkDataSourceVariable}}]
extern UA_StatusCode
 linkDataSourceVariable(UA_Server *server, UA_NodeId nodeId) {
     bool readonly = false;
     UA_DataSource dataSourceVariable;
     UA_StatusCode rc; |>\setcounter{lstnumber}{254}<|

     dataSourceVariable.read = readDataSourceVariable;
     if (!readonly)
        dataSourceVariable.write = writeDataSourceVariable;
     else
        dataSourceVariable.write = writeReadonlyDataSourceVariable;

     return UA_Server_setVariableNode_dataSource(server, nodeId, dataSourceVariable);
 }
\end{lstlisting}

\begin{figure}[h]
    \centering
    \includegraphics[width=0.42\textwidth]{doc/img/OPC_RevPiDO.pdf}
    \caption{Auszug des verwendeten Nodesets, hier Digitalausgang 1 des Versuchsaufbaus
      \label{fig:opc-do}}
\end{figure}

Die in Listing~\ref{lst:4-linkDataSourceVariable} abgebildete Methode \lstinline{linkDataSourceVariable()} erzeugt ein Struct vom Typ \lstinline{UA_DataSource}. In diesem werden dem Lesen und Schreiben einer OPC-Variablen entsprechende Callback-Methoden zugewiesen. Die Verknüpfung einer OPC-Variable, genauer ihrer NodeId, mit der zuvor definierten Datenquelle erfolgt über die von open62541 bereitgestellte Methode \lstinline{UA_Server_setVariableNode_dataSource()}. Vor dem Lesen und nach dem Schreiben dieser Variable werden von nun an die entsprechenden Callbacks aufgerufen.
     
\begin{lstlisting}[language={c},firstnumber=168,caption={Auszug des Callbacks \lstinline{writeDataSourceVariable} in \lstinline{variables.c}\label{lst:4-writeDataSourceVariable}}]  
extern UA_StatusCode
 writeDataSourceVariable(UA_Server *server,
            const UA_NodeId *sessionId, void *sessionContext,
            const UA_NodeId *nodeId, void *nodeContext,
            const UA_NumericRange *range, const UA_DataValue *dataValue) {

    UA_StatusCode retval  = UA_STATUSCODE_GOOD;
    UA_NodeId *nameNodeId = UA_malloc(sizeof(UA_NodeId));
    UA_QualifiedName nameQN = UA_QUALIFIEDNAME(1, "Name");
    UA_Variant nameVar;
    UA_Boolean bit;

    retval |= findSiblingByBrowsename(server, nodeId, &nameQN, nameNodeId);
    retval |= UA_Server_readValue(server, *nameNodeId, &nameVar);
    retval |= UA_Boolean_copy(dataValue->value.data, &bit);

    |>\tikzmarkin[set border color=martinired]{writeIO}<|PI_writeSingleIO(String_fromUA_String(nameVar.data), &bit, false);                                                 |>\tikzmarkend{writeIO}<|

    free(nameNodeId);
    return retval;
 }
\end{lstlisting}

Listing~\ref{lst:4-writeDataSourceVariable} zeigt die Callback-Methode, welche nach dem Schreiben einer Variablen auf dem OPC-Server aufgerufen wird.
Dieser Methode wird neben der NodeId der mit ihr verknüpften Variablen auch der Wert dieser in Form eines Zeigers auf ein Struct vom Typ \lstinline{UA_DataValue} übergeben.

Die Gestaltung des hier verwendeten Nodesets sieht vor, dass in einer OPC-Variablen \lstinline{"Name"} der Bezeichner des zu schreibenden Digitalausgangs hinterlegt ist, siehe Abbildung~\ref{fig:opc-do}. Dies erlaubt eine Rekonfiguration der Ein- und Ausgänge der SPS ohne Änderungen im Programmcode des OPC-Servers vornehmen zu müssen.
Es ist daher erforderlich, nach jedem Schreiben einer mit einem Digitalausgang verknüpften Variablen, hier \lstinline{"Value"}, dessen Bezeichner \lstinline{"Name"} abzufragen. 
Dies geschieht in den Zeilen 180 und 181.
Anschließend wird dieser Bezeichner sowie der zu schreibende Wert der Methode \lstinline{PI_writeSingleIO()} übergeben, welche wiederum die Interaktion mit piControl übernimmt (vgl. Abschnitt \ref{sec:4-picontrol}).
 
\subsection{Integration von piControl%
     \label{sec:4-picontrol}}
In Abschnitt~\ref{sec:2-io} wurde die Anbindung der IO-Module des Revolution Pi sowie die Funktionsweise von piControl aus Anwendersicht beschrieben. Die verfügbare Literatur beschränkt sich auch auf lediglich diese Sicht; eine weiterführende Dokumentation für Entwickler gibt es, neben der in Abschnitt~\ref{sec:3-anbindung} vorgestellten Manpage, nicht. 
In diesem Abschnitt soll daher der Quellcode von piControl sowie dessen Verwendung im Projekt genauer betrachtet werden.
Hierzu wird exemplarisch die in Abschnitt~\ref{sec:4-open62541} eingeführte Methode \lstinline{PI_writeSingleIO()} untersucht.
Diese Methode ermöglicht das Setzen eines einzelnen Bits im Prozessabbild der SPS, und damit das Schalten eines digitalen Ausgangs auf einem IO-Modul.
Die äquivalente Methode \lstinline{int piControlGetBitValue(SPIValue *pSpiValue)} zum Lesen eines Bits bzw. Eingangs funktioniert analog und soll daher an dieser Stelle nicht dediziert erörtert werden.

\begin{lstlisting}[language={c},firstnumber=97,
                   caption={Setzen eines phsikalischen, digitalen Ausgangs in \lstinline{revpi.c}
                   \label{lst:4-PI_writeSingleIO}}]
extern void PI_writeSingleIO(char *pszVariableName, bool *bit, bool verbose)
{
	int rc;
	SPIVariable sPiVariable;
	SPIValue sPIValue;

	strncpy(sPiVariable.strVarName, pszVariableName, sizeof(sPiVariable.strVarName));
	rc = piControlGetVariableInfo(&sPiVariable);
	if (rc < 0) {
		printf("Cannot find variable '%s'\n", pszVariableName);
		return;
	}

		sPIValue.i16uAddress = sPiVariable.i16uAddress;
		sPIValue.i8uBit = sPiVariable.i8uBit;
		sPIValue.i8uValue = *bit;
		rc = |>\tikzmarkin[set border color=martinired]{setBitValue}<|piControlSetBitValue(&sPIValue)|>\tikzmarkend{setBitValue}<|;
		if (rc < 0)
			printf("Set bit error %s\n", getWriteError(rc));
		else if (verbose)
			printf("Set bit %d on byte at offset %d. Value %d\n", sPIValue.i8uBit, sPIValue.i16uAddress,
			       sPIValue.i8uValue);
}
\end{lstlisting}

Der Programmcode in Listing~\ref{lst:4-PI_writeSingleIO} ist Teil des implementierten OPC-Servers. In diesem wird auf zwei Funktionen des piControl-Treibers zugegriffen. 
Beiden Methoden wird als Argument ein Zeiger auf ein Struct vom Typ \lstinline{SPIValue} übergeben. Der im Struct abgelegte Name wird mittels \lstinline{piControlGetVariableInfo(&sPIValue)} zu einer Adresse im Prozessabbild aufgelöst. Diese wird in \lstinline{sPIValue.i16uAdress} gespeichert. Der Wert der Variablen wird anschließend mittels \lstinline{piControlSetBitValue(&sPIValue)} an dieser Adresse in das Prozessabbild geschrieben.

\begin{lstlisting}[language={c},firstnumber=309,caption={Methode \lstinline{piControlSetBitValue} in \lstinline{piControlIf.c}\label{lst:4-piControlSetBitValue}}]
int |>\tikzmarkin[set border color=martiniblue]{setBitValueFcn}<|piControlSetBitValue(SPIValue *pSpiValue)|>\tikzmarkend{setBitValueFcn}<|
{
    piControlOpen();

    if (PiControlHandle_g < 0)
	    return -ENODEV;

    pSpiValue->i16uAddress += pSpiValue->i8uBit / 8;
    pSpiValue->i8uBit %= 8;

    if (|>\tikzmarkin[set border color=martinired]{ioctl}<|ioctl(PiControlHandle_g, KB_SET_VALUE, pSpiValue)|>\tikzmarkend{ioctl}<| < 0)
	    return errno;

    return 0;
}
\end{lstlisting}

Die in Listing~\ref{lst:4-piControlSetBitValue} dargestellte Methode \lstinline{piControlSetBitValue} ist lediglich eine Hüllfunktion (häufig auch als Wrapper-Funktion bezeichnet) für einen Aufruf des \lstinline{ioctl} Kernel-Moduls.
Folgende Parameter werden übergeben:
\lstinline{PiControlHandle_g} ist die Referenz auf die Geräte-Datei des piControl-Treibers. \lstinline{KB_SET_VALUE} ist das ioctl-Kommando zum Schreiben eines Bits in das Prozessabbild. Der Zeiger \lstinline{pSpiValue} verweist auf ein Struct des bereits vorgestellten Typs \lstinline{SPIValue}.

\begin{lstlisting}[language={c},firstnumber=80,caption={Methode \lstinline{piControlOpen} in \lstinline{piControlIf.c}\label{lst:4-piControlOpen}}]
void piControlOpen(void)
{
    /* open handle if needed */
    if (PiControlHandle_g < 0)
    {
	    |>\tikzmarkin[set border color=martiniblue]{PiControlHandle}<|PiControlHandle_g = open(PICONTROL_DEVICE, O_RDWR)|>\tikzmarkend{PiControlHandle}<|;
    }
}
\end{lstlisting}

Die in Listing~\ref{lst:4-piControlOpen} dargestellte Methode öffnet, sofern nicht bereits geschehen, die Geräte-Datei. Das Macro \lstinline{PICONTROL_DEVICE} verweist hierbei auf \lstinline{/dev/piControl0}.

\begin{lstlisting}[language={c},firstnumber=721,caption={Methode \lstinline{piControlIoctl} in \lstinline{piControlMain.c}\label{lst:4-piControlIoctl}}]
static long |>\tikzmarkin[set border color=martiniblue, below offset=0.9em]{piControlIoctl}<|piControlIoctl(struct file *file, unsigned int prg_nr, 
                           unsigned long usr_addr)                                      |>\tikzmarkend{piControlIoctl}<|
{
  int status = -EFAULT;
  tpiControlInst *priv;
  int timeout = 10000;	// ms

  if (prg_nr != KB_CONFIG_SEND && prg_nr != KB_CONFIG_START && !isRunning()) {
  	return -EAGAIN;
  }

  priv = (tpiControlInst *) file->private_data;

  if (prg_nr != KB_GET_LAST_MESSAGE) {
  	// clear old message
  	priv->pcErrorMessage[0] = 0;
  }

  switch (prg_nr) {|>\setcounter{lstnumber}{864}<|

    case |>\tikzmarkin[set border color=martiniblue]{KB_SET_VALUE}<|KB_SET_VALUE:|>\tikzmarkend{KB_SET_VALUE}<|
  		{
  			SPIValue *pValue = (SPIValue *) usr_addr;

  			if (!isRunning())
  				return -EFAULT;

  			if (pValue->i16uAddress >= KB_PI_LEN) {
  				status = -EFAULT;
  			} else {
  				INT8U i8uValue_l;
  				my_rt_mutex_lock(&piDev_g.lockPI);
  				i8uValue_l = piDev_g.ai8uPI[pValue->i16uAddress];

  				if (pValue->i8uBit >= 8) {
  					i8uValue_l = pValue->i8uValue;
  				} else {
  					if (pValue->i8uValue)
  						i8uValue_l |= (1 << pValue->i8uBit);
  					else
  						i8uValue_l &= ~(1 << pValue->i8uBit);
  				}

  				|>\tikzmarkin[set border color=martinired]{i8uValue}<|piDev_g.ai8uPI[pValue->i16uAddress] = i8uValue_l;|>\tikzmarkend{i8uValue}<|
  				rt_mutex_unlock(&piDev_g.lockPI);

  #ifdef VERBOSE
  				pr_info("piControlIoctl Addr=%u, bit=%u: %02x %02x\n", pValue->i16uAddress, pValue->i8uBit, pValue->i8uValue, i8uValue_l);
  #endif

  				status = 0;
  			}
  		}
  		break; |>\setcounter{lstnumber}{1314}<|

    default:
      pr_err("Invalid Ioctl");
      return (-EINVAL);
      break;

    }

    return status;
  }
\end{lstlisting}

Listing~\ref{lst:4-piControlIoctl} zeigt in Auszügen die ioctl-Methode des piControl Kernel-Treibers. Diese bekommt folgende Argumente übergeben: \lstinline{struct file *file} enthält den Verweis auf die Geräte-Datei, hier \lstinline{/dev/piControl0}. Der Wert von \lstinline{unsigned int prg_nr} beschreibt die Anfrage an den Treiber, in diesem Fall \lstinline{KB_SET_VALUE}. Das Argument \lstinline{unsigned long usr_addr} enthält einen typ-agnostischen Pointer. Dieser verweist auf einen Speicherbereich, in welchem die zur Bearbeitung der Anfrage notwendigen Daten abgelegt sind. Hier können auch vom Treiber empfangene Daten dem Anwendungsprogramm bereitgestellt werden. 

Die switch-case-Anweisung führt die über das Argument \lstinline{prg_nr} spezifizierte Aktion aus. Hier betrachten wir \lstinline{KB_SET_VALUE}:
Zunächst wird in Zeile 868 der übergebene Zeiger \lstinline{usr_addr} mittels explizitem Typecast zu einem Zeiger des Typs \lstinline{SPIValue *} konvertiert. Da dieser auf Daten im Userspace verweist, ist beim Zugriff durch den Kernel-Treiber besondere Vorsicht geboten.
In Zeile 877 wird mittels Mutex das Prozessabbild \lstinline{piDev_g} für den Zugriff durch andere Threads oder Prozesse gesperrt.
\lstinline{my_rt_mutex_lock} verweist hierbei auf die Funktion \lstinline{rt_mutex_lock} aus \lstinline{linux/sched.h}\footnote{Offenbar wurde hier auch eine alternative Implementierung vorgesehen, siehe revpi\_common.h}

In Zeile 889 wird das Byte \lstinline{i8uValue_l}, welches den zu schreibenden Wert enthält in das Prozessabbild übertragen. Anschließend wird die Mutex auf \lstinline{piDev_g} wieder entsperrt.
\newpage

\begin{lstlisting}[language={c},firstnumber=62,caption={Auszug des Struct \lstinline{spiControlDev} in \lstinline{piControlMain.h}\label{lst:4-spiControlDev}}]
|>\tikzmarkin[set border color=martiniblue]{spiControlDev}<|typedef struct spiControlDev|>\tikzmarkend{spiControlDev}<| {
	// device driver stuff
	int init_step;
	enum revpi_machine machine_type;
	void *machine;
	struct cdev cdev;	// Char device structure
	struct device *dev;
	struct thermal_zone_device *thermal_zone;

	|>\tikzmarkin[set border color=martiniblue]{processImage}<|// process image stuff
	INT8U ai8uPI[KB_PI_LEN];
	INT8U ai8uPIDefault|>\tikzmarkin[set border color=martinired]{KB_PI_LEN_0}<|[KB_PI_LEN]|>\tikzmarkend{KB_PI_LEN_0}<|;
	struct rt_mutex lockPI;        |>\tikzmarkend{processImage}<|
	bool stopIO;
	piDevices *devs; |>\setcounter{lstnumber}{94}<|
} tpiControlDev;
\end{lstlisting}

Das Prozessabbild ist als Byte-Array der Länge \lstinline{KB_PI_LEN} in Listing~\ref{lst:4-spiControlDev} definiert. Konfigurationsparameter wie \lstinline{KB_PI_LEN} oder die Zykluszeit für den Datenaustausch zwischen SPS und IO-Modulen sind im folgenden Listing~\ref{lst:4-process} definiert.

\begin{lstlisting}[language={c},firstnumber=119,caption={Konfigurationsparameter des Prozessabbildes in project.h\label{lst:4-process}}]
#define INTERVAL_PI_GATE (5*1000*1000)  // 5 ms piGateCommunication |>\setcounter{lstnumber}{128}<|

#define INTERVAL_IO_COM (5*1000*1000)  // 5 ms piIoComm |>\setcounter{lstnumber}{132}<|

#define KB_PD_LEN       512
|>\tikzmarkin[set border color=martiniblue]{KB_PI_LEN_1}<|#define KB_PI_LEN       4096|>\tikzmarkend{KB_PI_LEN_1}<|
\end{lstlisting}

Das zu setzende Bit wurde zu diesem Zeitpunkt erfolgreich in das Prozessabbild der SPS geschrieben.
Es stellt sich die Frage, wie dieses nun an das IO-Modul kommuniziert wird.
Die Kommunikation mit allen angebundenen Modulen ist ebenfalls Aufgabe des piControl-Treibers.

\begin{lstlisting}[language={c},firstnumber=256,caption={Auszug der Methode \lstinline{piIoThread} in \lstinline{revpi_core.c}\label{lst:4-piIoThread}}]
static int piIoThread(void *data)
{
	//TODO int value = 0;
	ktime_t time;
	ktime_t now;
	s64 tDiff;

	hrtimer_init(&piCore_g.ioTimer, CLOCK_MONOTONIC, HRTIMER_MODE_ABS);
	piCore_g.ioTimer.function = piIoTimer;

	pr_info("piIO thread started\n");

	now = hrtimer_cb_get_time(&piCore_g.ioTimer);

	PiBridgeMaster_Reset();

	while (!kthread_should_stop()) {
		if (|>\tikzmarkin[set border color=martinired]{PiBridgeMaster}<|PiBridgeMaster_Run()|>\tikzmarkend{PiBridgeMaster}<| < 0)
			break;
	}

	RevPiDevice_finish();

	pr_info("piIO exit\n");
	return 0;
}
\end{lstlisting}

Der Kernel-Thread \lstinline{piIoThread} ist verantwortlich für den zyklischen Datenaustausch mit den IO-Modulen. In diesem wird fortlaufend die Methode \lstinline{PiBridgeMaster_Run()} aufgerufen, siehe Listing~\ref{lst:4-piIoThread}.

\begin{lstlisting}[language={c},firstnumber=262,caption={Auszug der Methode \lstinline{PiBridgeMaster_Run(void)} in \lstinline{RevPiDevice.c}\label{lst:4-PiBridgeMaster_Run}}]
int PiBridgeMaster_Run(void)
{
	static kbUT_Timer tTimeoutTimer_s;
	static kbUT_Timer tConfigTimeoutTimer_s;
	static int error_cnt;
	static INT8U last_led;
	static unsigned long last_update;
	int ret = 0;
	int i;

	my_rt_mutex_lock(&piCore_g.lockBridgeState);
	if (piCore_g.eBridgeState != piBridgeStop) {
		switch (eRunStatus_s) { |>\setcounter{lstnumber}{514}<|
		    case enPiBridgeMasterStatus_EndOfConfig:|>\setcounter{lstnumber}{621}<|
		    if (|>\tikzmarkin[set border color=martinired]{RevPiDevice}<|RevPiDevice_run()|>\tikzmarkend{RevPiDevice}<|) {
				// an error occured, check error limits |>\setcounter{lstnumber}{641}<|
			} else {
				ret = 1;
			}
			piCore_g.image.drv.i16uRS485ErrorCnt = RevPiDevice_getErrCnt();
			break;
\end{lstlisting}

Die in Listing~\ref{lst:4-PiBridgeMaster_Run} dargestellte Methode ist eine sog. State-Machine. Ist die Konfiguration der IO-Module erfolgreich abgeschlossen, so führt sie bei Aufruf lediglich die Methode \lstinline{RevPiDevice_run()} aus.

\begin{lstlisting}[language={c},firstnumber=140,caption={Auszug der Methode \lstinline{RevPiDevice_run(void)} in \lstinline{RevPiDevice.c}\label{lst:4-RevPiDevice_run}}]
int RevPiDevice_run(void)
{
	INT8U i8uDevice = 0;
	INT32U r;
	int retval = 0;

	RevPiDevices_s.i16uErrorCnt = 0;

	for (i8uDevice = 0; i8uDevice < RevPiDevice_getDevCnt(); i8uDevice++) {
		if (RevPiDevice_getDev(i8uDevice)->i8uActive) {
			switch (RevPiDevice_getDev(i8uDevice)->sId.i16uModulType) {
			case KUNBUS_FW_DESCR_TYP_PI_DIO_14:
			case KUNBUS_FW_DESCR_TYP_PI_DI_16:
			case KUNBUS_FW_DESCR_TYP_PI_DO_16:
				r = |>\tikzmarkin[set border color=martinired]{sendCyclicTelegram}<|piDIOComm_sendCyclicTelegram(i8uDevice)|>\tikzmarkend{sendCyclicTelegram}\setcounter{lstnumber}{166} <|;

				break; |>\setcounter{lstnumber}{216}<|
			}
		}
	} |>\setcounter{lstnumber}{227}<|
	return retval;
}
\end{lstlisting}

Diese iteriert wie in Listing~\ref{lst:4-RevPiDevice_run} abgebildete durch alle gegenwärtig in der SPS konfigurierten Module. Ist das aktuelle Modul als aktiv markiert, so wird anhand eines sog. Firmware-Descriptors entschieden, welche Methode für die Ansteuerung des Moduls aufzurufen ist.

\begin{lstlisting}[language={c},firstnumber=161,caption={Auszug der Methode \lstinline{piDIOComm_sendCyclicTelegram} in \lstinline{piDIOComm.c}\label{lst:4-sendCyclicTelegram}}]
INT32U piDIOComm_sendCyclicTelegram(INT8U i8uDevice_p)
{
	INT32U i32uRv_l = 0;
	SIOGeneric sRequest_l;
	SIOGeneric sResponse_l;
	INT8U len_l, data_out[18], i, p, data_in[70];
	INT8U i8uAddress;
	int ret; |>\setcounter{lstnumber}{239}<|
	
    |>\tikzmarkin[set border color=martinired]{piIoComm}<|ret = piIoComm_send((INT8U *) & sRequest_l, IOPROTOCOL_HEADER_LENGTH + len_l + 1);  |>\tikzmarkend{piIoComm}\setcounter{lstnumber}{298}<|
}
\end{lstlisting}

Im Falle des hier verwendeten DO-Moduls wird die in Listing~\ref{lst:4-sendCyclicTelegram} abgebildete Methode \lstinline{piDIOComm_sendCyclicTelegram()} aufgerufen. Dieser wird ein Zeiger auf das zu schreibende Gerät übergeben. 
Zunächst wird das Prozessabbild mittels eines proprietären, jedoch im Quellcode offen nachvollziehbaren Protokolls in ein \lstinline{sRequest_l} genanntes Byte-Array umgewandelt. Dieser Schritt ist in Listing~\ref{lst:4-sendCyclicTelegram} nicht abgebildet. Anschließend wird \lstinline{piIoComm_send()} ein Zeiger auf die so generierte Schreib-Anfrage übergeben.

\begin{lstlisting}[language={c},firstnumber=220,caption={Auszug der Methode \lstinline{piIOComm_send} in \lstinline{piIOComm.c}\label{lst:4-piIOComm_send}}]
int piIoComm_send(INT8U * buf_p, INT16U i16uLen_p)
{
	ssize_t write_l = 0;
	INT16U i16uSent_l = 0;|>\setcounter{lstnumber}{249}<|

	while (i16uSent_l < i16uLen_p) {
		write_l = vfs_write(piIoComm_fd_m, buf_p + i16uSent_l, i16uLen_p - i16uSent_l, &piIoComm_fd_m->f_pos);
		if (write_l < 0) {
			pr_info_serial("write error %d\n", (int)write_l);
			return -1;
		} 
		i16uSent_l += write_l;|>\setcounter{lstnumber}{263}<|
	}
	clear();
	vfs_fsync(piIoComm_fd_m, 1);
	return 0;
}
\end{lstlisting}

Listing~\ref{lst:4-piIOComm_send} zeigt die Implementierung von \lstinline{piIoComm_send()}. Diese Methode ist für das Schreiben der oben generierten Anfrage auf die seriellen Schnittstelle verantwortlich. Realisiert wird dies mittels der Methode \lstinline{vfs_write()}. Diese ist in \lstinline{<linux/fs.h>} definiert. Sie ermöglicht das Schreiben einer Datei im Userspace aus dem Kernel heraus. Geschrieben wird hier die Datei mit dem Deskriptor \lstinline{piIoComm_fd_m}.
Da die Funktion \lstinline{vfs_write()} durch andere Kernel-Tasks unterbrochen werden kann, ist nicht gewährleistet, dass die gesamte Anfrage mit nur einem Aufruf geschrieben wird. Die oben abgebildete while-Schleife stellt das vollständige Senden der Anfrage sicher.

\begin{lstlisting}[language={c},firstnumber=157,caption={Auszug der Methode \lstinline{piIOComm_open_serial} in \lstinline{piIOComm.c}\label{lst:4-piIOComm_open_serial}}]
int piIoComm_open_serial(void)
{   |>\setcounter{lstnumber}{167}<|
	struct file *fd;	/* Filedeskriptor */
	struct termios newtio;	/* Schnittstellenoptionen */

	|>\tikzmarkin[set border color=martiniblue]{fd}<|/* Port oeffnen - read/write, kein "controlling tty", 
	    Status von DCD ignorieren */
	fd = filp_open(|>\tikzmarkin[set border color=martinired]{tty}<|REV_PI_TTY_DEVICE|>\tikzmarkend{tty}<|, O_RDWR | O_NOCTTY, 0); |>\setcounter{lstnumber}{208}<|
	
	piIoComm_fd_m = fd;                                                      |>\tikzmarkend{fd}\setcounter{lstnumber}{217}<|

	return 0;
}
\end{lstlisting}

Der zum Schreiben auf die serielle Schnittstelle verwendete Datei-Deskriptor wird von der in Listing~\ref{lst:4-piIOComm_open_serial} abgebildeten Methode \lstinline{piIoComm_open_serial()} generiert. 

\begin{lstlisting}[language={c},firstnumber=45,caption={Definition der seriellen Schnittstelle in \lstinline{piIOComm.h}\label{lst:4-REV_PI_TTY_DEVICE}}]
#define REV_PI_TTY_DEVICE	"/dev/ttyAMA0"
\end{lstlisting}

Das in Listing~\ref{lst:4-REV_PI_TTY_DEVICE} definierte Macro verweist auf eine der seriellen Schnittstellen des RaspberryPi.
Die Implementierung des zugehörigen Schnittstellentreibers soll hier nicht weiter untersucht werden. Somit ist an dieser Stelle die Kette vom Setzen einer Variablen auf dem OPC-Server bis hin zur Aktualisierung des Prozessabbilds der IO-Module geschlossen.

% \begin{lstlisting}[language={c},firstnumber={226},caption={Setzen der Scheduler-Priorität auf SCHED\_FIFO in 
% revpi\_common.c\label{lst:2-sched_priority}}]
% param.sched_priority = ktprio->prio;
% ret = sched_setscheduler(child, SCHED_FIFO, &param);
% \end{lstlisting}
%% % Imports nur für Referenzenauflösung während des Schreibens! Vorm Kompilieren auskommentieren!
% \bibliography{0_hauptdatei}
% \input{1_einleitung}
% \input{2_grundlagen}
% \input{3_konzeption}
% \input{4_implementierung}
% \input{5_tests}
% \input{6_zusammenfassung}
% % Ende Imports

\section{Test des OPC-Servers im Gesamtsystem%
  \label{sec:5-tests}}

%% % Imports nur für Referenzenauflösung während des schreibens! Vorm Kompilieren auskommentieren!
% \bibliography{0_hauptdatei}
% \input{1_einleitung}
% \input{2_grundlagen}
% \input{3_konzeption}
% \input{4_implementierung}
% \input{5_tests}
% \input{6_zusammenfassung}
% % Ende Imports

\section{Zusammenfassung und Ausblick%
  \label{sec:6-fazit}}
Der folgende Abschnitt~\ref{sec:6-zusammenfassung} fasst die gewonnenen Erkenntnisse und den Stand der Implementierung zusammen.
Den Abschluss dieser Arbeit bildet der Ausblick in Abschnitt~\ref{sec:6-ausblick}.

\subsection{Zusammenfassung%
     \label{sec:6-zusammenfassung}}

\subsection{Ausblick%
     \label{sec:6-ausblick}}

% % Ende Imports

\section{Einleitung und Motivation%
  \label{sec:1-einleitung}}
Ziel dieses Projektes ist die Integration eines OPC-Servers mit einer auf Linux
basierenden speicherprogrammierbaren Steuerung (SPS). Angeschlossen an diese SPS
ist jeweils ein digitales Ein-/\,bzw.~Ausgabemodul. Die von diesen bereitgestellten
Ein-/\, bzw.~Ausgänge (IO) sollen in der Datenstruktur des OPC-Servers abgebildet
und über diesen für OPC-Clients les-/\,und schreibar sein. Weiterhin sollen einige
Funktionen zur Überwachung und Steuerung der an die SPS angeschlossenen Aktoren
und Sensoren direkt im OPC-Server implementiert werden.
Hiermit stellt dieses Projekt eine der Grundlagen für ein übergeordnetes Projekt,
die cloudbasierte Steuerung eines miniaturisierten Produktions-Systems, dar.

Der hier verwendete OPC-Server ist Teil des sog. open62541 Projekts. Er ist in C
geschrieben und implementiert bereits einen großen Teil der im OPC-UA-Standard
spezifizierten Funktionen.
Als SPS findet ein Revolution Pi 3 der Firma Kunbus Verwendung. Dieser integriert
ein sog. Compute Module der Raspberry Pi Foundation in ein industrietaugliches
Gehäuse und erlaubt die Erweiterung mittels IO- oder Gateway-Modulen. Über diese
erfolgt die Kommunikation mit weiteren Komponenten der Automatisierungstechnik.

Motiviert ist dieses Projekt durch die Beobachtung, dass die Verbreitung offener
Standards sowie freier Software auch in der Automatisierungstechnik zunimmt.
Linux ist ein freies Betriebssystem, OPC-UA ein offen zugänglicher, aktiv gepflegter
und weit verbreiteter Standard. Der Raspberry Pi findet sowohl bei Hobby-Anwendern als
auch in den Bereichen Forschung und Entwicklung sowie bei industriellen Anwendern
Verwendung. Dieses Projekt stellt somit eine für unterschiedliche Anwender interessante
Entwicklung dar.

Im Anschluss an diese einleitende Übersicht im Abschnitt~\ref{sec:1-einleitung} folgt
die Darstellung der wichtigsten Grundlagen in Abschnitt~\ref{sec:2-grundlagen}.
Aufbauend auf diesen Grundlagen folgt die konzeptuelle Ausarbeitung im Abschnitt~\ref{sec:3-konzeption}.
Die Umsetzung wird im Abschnitt~\ref{sec:4-implementierung} erläutert.
Die Leistungsfähigkeit der Implementierung wird in Abschnitt~\ref{sec:5-tests} untersucht.
Eine Zusammenfassung und ein Ausblick schließen die Arbeit in
Abschnitt~\ref{sec:6-fazit} ab. Eventuell noch benötigte Anhänge
finden sich in den Anhängen [...] bis [...].

%% % Imports nur für Referenzenauflösung während des Schreibens! Vorm Kompilieren auskommentieren!
% \bibliography{0_hauptdatei}
% % Mit \section{...} eröffnen wir einen neuen Abschnitt.
% Der Befehl setzt nicht nur den Text in einer größeren,
% fetten Schrift, sondern sorgt außerdem dafür, daß er im
% Inhaltsverzeichnis erscheint.
%
% Mit \label{...} erzeugen wir einen Bezeichner, mit dessen Hilfe
% wir später auf die Nummer des Abschnitts verweisen können (nämlich
% mit~\ref{...}).
%
% Das Kommentarzeichen hinter „Übersicht“ dient dazu, ein
% Leerzeichen zwischen „Übersicht“ und dem \label-Befehl
% zu vermeiden, das andernfalls sichtbar würde – z.B. im
% Inhaltsverzeichnis.
%

% % Imports nur für Referenzenauflösung während des Schreibens! Vorm Kompilieren auskommentieren!
% \bibliography{0_hauptdatei}
% \input{1_einleitung}
%\input{2_grundlagen}
%\input{3_konzeption}
%\input{4_implementierung}
%\input{5_tests}
%\input{6_zusammenfassung}
% % Ende Imports

\section{Einleitung und Motivation%
  \label{sec:1-einleitung}}
Ziel dieses Projektes ist die Integration eines OPC-Servers mit einer auf Linux
basierenden speicherprogrammierbaren Steuerung (SPS). Angeschlossen an diese SPS
ist jeweils ein digitales Ein-/\,bzw.~Ausgabemodul. Die von diesen bereitgestellten
Ein-/\, bzw.~Ausgänge (IO) sollen in der Datenstruktur des OPC-Servers abgebildet
und über diesen für OPC-Clients les-/\,und schreibar sein. Weiterhin sollen einige
Funktionen zur Überwachung und Steuerung der an die SPS angeschlossenen Aktoren
und Sensoren direkt im OPC-Server implementiert werden.
Hiermit stellt dieses Projekt eine der Grundlagen für ein übergeordnetes Projekt,
die cloudbasierte Steuerung eines miniaturisierten Produktions-Systems, dar.

Der hier verwendete OPC-Server ist Teil des sog. open62541 Projekts. Er ist in C
geschrieben und implementiert bereits einen großen Teil der im OPC-UA-Standard
spezifizierten Funktionen.
Als SPS findet ein Revolution Pi 3 der Firma Kunbus Verwendung. Dieser integriert
ein sog. Compute Module der Raspberry Pi Foundation in ein industrietaugliches
Gehäuse und erlaubt die Erweiterung mittels IO- oder Gateway-Modulen. Über diese
erfolgt die Kommunikation mit weiteren Komponenten der Automatisierungstechnik.

Motiviert ist dieses Projekt durch die Beobachtung, dass die Verbreitung offener
Standards sowie freier Software auch in der Automatisierungstechnik zunimmt.
Linux ist ein freies Betriebssystem, OPC-UA ein offen zugänglicher, aktiv gepflegter
und weit verbreiteter Standard. Der Raspberry Pi findet sowohl bei Hobby-Anwendern als
auch in den Bereichen Forschung und Entwicklung sowie bei industriellen Anwendern
Verwendung. Dieses Projekt stellt somit eine für unterschiedliche Anwender interessante
Entwicklung dar.

Im Anschluss an diese einleitende Übersicht im Abschnitt~\ref{sec:1-einleitung} folgt
die Darstellung der wichtigsten Grundlagen in Abschnitt~\ref{sec:2-grundlagen}.
Aufbauend auf diesen Grundlagen folgt die konzeptuelle Ausarbeitung im Abschnitt~\ref{sec:3-konzeption}.
Die Umsetzung wird im Abschnitt~\ref{sec:4-implementierung} erläutert.
Die Leistungsfähigkeit der Implementierung wird in Abschnitt~\ref{sec:5-tests} untersucht.
Eine Zusammenfassung und ein Ausblick schließen die Arbeit in
Abschnitt~\ref{sec:6-fazit} ab. Eventuell noch benötigte Anhänge
finden sich in den Anhängen [...] bis [...].

% % % Imports nur für Referenzenauflösung während des Schreibens! Vorm Kompilieren auskommentieren!
% \bibliography{0_hauptdatei}
% \input{1_einleitung}
% \input{2_grundlagen}
% \input{3_konzeption}
% \input{4_implementierung}
% \input{5_tests}
% \input{6_zusammenfassung}
% % Ende Imports

\section{Grundlagen%
  \label{sec:2-grundlagen}}

\subsection{Speicherprogrammierbare-Steuerung und Linux -- Revolution Pi%
     \label{sec:2-sps}}

\subsubsection{Kunbus RevolutionPi%
        \label{sec:2-revpi}}
Der RevolutionPi 3 ist eine speicherprogrammierbare Steuerung (SPS) des Herstellers
Kunbus GmbH. Kern dieser SPS ist das von der Raspberry Pi Foundation entwickelte
und vertriebene Raspberry Pi Compute Module 3. Dieses integriert ein Broadcom BCM2837
System-on-Chip (SoC) mit vier 1,2GHz Prozessorkernen, 1GB RAM, 4GB eMMC Anwendungsspeicher
und sonstige Peripherie in ein Modul im DDR2-SODIMM Formfaktor. Diese Spezifikationen
sind weitgehend identisch zu denen des ausgesprochen populären Raspberry Pi 3.
Der Revolution Pi profitiert daher von dem gleichen großen Angebot an Software
und Unterstützung wie der Raspberry Pi, ergänzt dessen Hardware jedoch um eine 24V
Spannungsversorgung, die Möglichkeit der Erweiterung durch mehrere industrietaugliche
Ein-/ Ausgabemodule und Gateways sowie ein Gehäuse zur Montage auf einer DIN-Schiene.
\begin{itemize}
  \item{Prozessor: BCM2837}
  \item{Taktfrequenz 1,2 GHz}
  \item{Anzahl Prozessorkerne: 4}
  \item{Arbeitsspeicher: 1 GByte}
  \item{eMMC Flash Speicher: 4 GByte}
  \item{Betriebssystem: Angepasstes Raspbian mit RT-Patch}
  \item{RTC mit 24h Pufferung über wartungsfreien Kondensator}
  \item{Treiber / API: Treiber schreibt zyklisch Prozessdaten in ein Prozessabbild, Zugriff auf Prozessabbild über Linux-Filesystem als API zu Fremdsoftware.}
  \item{Kommunikationsanschlüsse: 2 x USB 2.0 A (je 500 mA belastbar), 1 x Micro-USB, HDMI, Ethernet (RJ45) 10/100 Mbit/s}
  \item{Stromversorgung: min. 10,7 V, max. 28,8 V, maximal 10 Watt}
  \item{Zulässige Umgebungstemperatur: -40 bis +55 C}
  \item{Gehäuseabmessungen: (HxBxL) 96 mm x 22,5 mm x 110,5 mm (ohne gesteckte Stecker)}
  \item{ESD Schutz: 4 kV / 8 kV gemäß EN61131-2 und IEC 61000-6-2}
  \item{Surge / Burst Prüfungen: gemäß EN61131-2 und IEC 61000-6-2 eingekoppelt auf Versorgungsspannung, Ethernet und IO-Leitungen}
  \item{EMI Prüfungen: gemäß EN61131-2 und IEC 61000-6-2}
\end{itemize}

Kunbus bietet eine Auswahl an IO- und Gateway-Modulen zur Erweiterung des Revolution Pi an.
Gateways dienen der Kommunikation mit Systemen oder Komponenten der Automatisierungstechnik
über Protokolle wie PROFIBUS oder EtherCAT. IO-Module erlauben die Überwachung
und Steuerung von digitalen oder analogen Ein- und Ausgängen.

\subsubsection{Zugriff auf IO-Module%
        \label{sec:2-io}}
Der Zugriff auf die Ein- und Ausgänge der IO-Module erfolgt über ein Prozessabbild
und einen hierfür von Kunbus bereitgestellten Treiber, genannt piControl. Dieser
aktualisiert das Prozessabbild zyklisch. Die angestrebte Zykluszeit beträgt 5ms,
kann jedoch je nach Anzahl der angeschlossenen Module auch größer sein. Kunbus
garantiert bei drei IO-Modulen und zwei Gateway-Modulen eine Zykluszeit von 10 ms.
Jedes der IO-Module stellt ein eigenständiges eingebettetes System dar. Es verfügt
über einen Microcontroller, welcher die IOs bereitstellt und über einen RS485-Bus
mit dem Revolution Pi kommuniziert.
% https://revolution.kunbus.de/io-modul/

Lizenz: GPL
% https://github.com/RevolutionPi/piControl

\begin{lstlisting}[language={c},firstnumber={226},caption={Setzen der Scheduler-Priorität auf SCHED\_FIFO in revpi\_common.c\label{lst:2-sched_priority}}]
param.sched_priority = ktprio->prio;
ret = sched_setscheduler(child, SCHED_FIFO,
       &param);
\end{lstlisting}


\subsection{Echtzeit und Multithreading unter Linux -- preemptRT und posix%
     \label{sec:2-echtzeit}}


 Der Linux-Kernel verfügt über mehrere unterschiedliche Preemtion-Modelle:

\begin{itemize}
  \item No Forced Preemption (server):
  Ausgelegt auf maximal möglichen Durchsatz, lediglich Interrupts und
  System-Call-Returns bewirken Präemption.

  \item Voluntary Kernel Preemption (Desktop):
  Neben den implizit bevorrechtigten Interrupts und System-Call-Returns gibt es
  in diesem Modell weitere Abschnitte des Kernels in welchen Preämption explizit
  gestattet ist.

  \item Preemptible Kernel (Low-Latency Desktop):
  In diesem Modell ist der gesamte Kernel, mit Ausnahme sog.~kritischer Abschnitte
  präemptible. Nach jedem kritischen Abschnitt gibt es einen impliziten Präemptions-Punkt.

  \item Preemptible Kernel (Basic RT):
  Dieses Modell ist dem zuvor genannten sehr ähnlich, hier sind jedoch alle Interrupt-Handler
  als eigenständige Threads ausgeführt.

  \item Fully Preemptible Kernel (RT):
  Wie auch bei den beiden zuvor genannten Modellen ist hier der gesamte Kernel
  präemtible, die Anzahl und Dauer der nicht-präemtiblen kritischen Abschnitte
  ist auf ein notwendiges Minimum beschränkt. Alle Interrupt-Handler sind als
  eigenständige Threads ausgeführt, Spinlocks durch Sleeping-Spinlocks und Mutexe
  durch sog.~RT-Mutexe ersetzt.

\end{itemize}
\todo{Spinlocks und Mutexe sowie die RT-Varianten dieser erklären!}

Lediglich mit dem vollständig präemtiblen Kernel kann Echtzeit-Verhalten realisiert werden.

% https://wiki.linuxfoundation.org/realtime/documentation/technical_basics/preemption_models bzw kernel/Kconfig.preempt

\subsubsection{preemptRT%
        \label{sec:2-preemptRT}}
% https://wiki.linuxfoundation.org/realtime/documentation/technical_details/start
% https://wiki.linuxfoundation.org/realtime/documentation/technical_basics/start

Das dem PREEMPT RT Kernel zugrunde liegende Prinzip lässt sich in einer einfachen
Regel ausdrücken: Nur Code, welcher absolut nicht-präemtible sein darf, ist es
gestattet nicht-präemtible zu sein.
Das erklärte Ziel des PREEMPT\_RT Patches ist es folglich, die Menge des nicht-präemtiblen
Codes im Linux-Kernel auf das absolut notwendige Minimum zu reduzieren.

Dies wird durch Verwendung folgender Mechanismen erreicht:

\begin{itemize}
  \item Hochauflösende Timer
  \item Sleeping Spinlocks
  \item Threaded Interrupt Handlers
  \item rt\_mutex
  \item RCU
\end{itemize}


\subsubsection{posix%
        \label{sec:2-posix}}
Ist posix hier wirklich relevant? Debian bzw.~Raspbian sind weitgehend posix
kompatibel, aber wird es hier genutzt? -> JA, open62541 nutzt pthread.h
piControl nutzt kthread.h, und semaphore.h

\subsection{OPC-UA und open62541%
     \label{sec:2-opc}}

\subsubsection{OPC UA%
        \label{sec:2-opcua}}
Open Platform Communications (OPC) ist eine Familie von Standards zur herstellerunabhängigen
Kommunikation von Maschinen (M2M) in der Automatisierungstechnik. Die sog.~OPC Task Force, zu deren
Mitgliedern verschiedene große Firmen der Automatisierungsindustrie gehören, veröffentlichte
die OPC Specification Version 1.0 im August 1996.
Motiviert ist dieser offene Standard durch die Erkenntniss, dass die Anpassung der
zahlreichen Herstellerstandards an individuelle Infrastrukturen und Anlagen einen
großen Mehraufwand verursachen.
Die Wikipedia beschreibt das Anwendungsgebiet für OPC wie folgt:

\glqq{}OPC wird dort eingesetzt, wo Sensoren, Regler und Steuerungen verschiedener Hersteller
ein gemeinsames Netzwerk bilden. Ohne OPC benötigten zwei Geräte zum Datenaustausch
genaue Kenntnis über die Kommunikationsmöglichkeiten des Gegenübers. Erweiterungen
und Austausch gestalten sich entsprechend schwierig. Mit OPC genügt es, für jedes
Gerät genau einmal einen OPC-konformen Treiber zu schreiben. Idealerweise wird
dieser bereits vom Hersteller zur Verfügung gestellt. Ein OPC-Treiber lässt sich
ohne großen Anpassungsaufwand in beliebig große Steuer- und Überwachungssysteme
integrieren.

OPC unterteilt sich in verschiedene Unterstandards, die für den jeweiligen Anwendungsfall
unabhängig voneinander implementiert werden können. OPC lässt sich damit verwenden
für Echtzeitdaten (Überwachung), Datenarchivierung, Alarm-Meldungen und neuerdings
auch direkt zur Steuerung (Befehlsübermittlung).\grqq{}

OPC basiert in der ursprünglichen Spezifikation auf Microsofts DCOM-Spezifikation.
DCOM macht Funktionen und Objekte einer Anwendung anderen Anwendungen im Netzwerk
zugänglich. Der OPC-Standard definiert entsprechende DCOM-Objekte um mit anderen
OPC-Anwendungen Daten austauschen zu können. Die Verwendung von DCOM bindet Anwender
an Betriebssysteme von Microsoft. Die ursprüngliche OPC Spezifikation wird durch die
Entwicklung von OPC Unified Architecture (OPC UA) abgelöst.
OPC UA setzt auf einem eigenen Kommunikationionsstack auf, die Verwendung von DCOM
und damit die Bindung an Microsoft wurden aufgelöst.

Die OPC-UA-Architektur ist eine Service-orientierte Architektur (SOA), deren Struktur
aus mehreren Schichten besteht.

% Wikipedia
Das OPC-Informationsmodell ist nicht mehr nur eine Hierarchie aus Ordnern, Items
und Properties. Es ist ein sogenanntes Full-Mesh-Network aus Nodes, mit dem neben
den Nutzdaten eines Nodes auch Meta- und Diagnoseinformationen repräsentiert werden.
Ein Node ähnelt einem Objekt aus der objektorientierten Programmierung. Ein Node
kann Attribute besitzen, die gelesen werden können (Data Access (DA), Historical
Data Access (HDA)). Es ist möglich Methoden zu definieren und aufzurufen.
Eine Methode besitzt Aufrufargumente und Rückgabewerte. Sie wird durch ein Command
aufgerufen. Weiterhin werden Events unterstützt, die versendet werden können
(AE (Alarms \& Events), DA DataChange), um bestimmte Informationen zwischen Geräten
auszutauschen. Ein Event besitzt unter anderem einen Empfangszeitpunkt, eine Nachricht
und einen Schweregrad. Die o. g. Nodes werden sowohl für die Nutzdaten als auch
alle anderen Arten von Metadaten verwendet. Der damit modellierte OPC-Adressraum
beinhaltet nun auch ein Typmodell, mit dem sämtliche Datentypen spezifiziert werden.

% https://de.wikipedia.org/wiki/Open_Platform_Communications
% https://de.wikipedia.org/wiki/OPC_Unified_Architecture
% https://opcfoundation.org/developer-tools/specifications-unified-architecture
% Von Gerhard Gappmeier - ascolab GmbH, CC BY-SA 3.0, https://de.wikipedia.org/w/index.php?curid=1892069
\subsubsection{open62541%
        \label{sec:2-open62541}}
open62541 ist eine offene und freie Implementierung von OPC UA. Die in C geschriebene
Bibliothek stellt eine beständig zunehmende Anzahl der im OPC UA Standard definierten
Funktionen bereit. Sie kann sowohl zur Erstellung von OPC-Servern als auch -Clients
genutzt werden. Ergänzend zu der unter der Mozilla Public License v2.0 lizensierten
Bibliothek stellt das open62541 Projekt auch Beispielprogramme unter einer CC0 Lizenz
zur Verfügung.

Die Bibliothek eignet sich auch für die Entwicklung auf eingebetteten Systemen und
Microcontrollern. Je nach Umfang der gewünschten Funktionen und des OPC Informationsmodells
beträgt die Größe einer Server-Binary weniger als 100kb. %evtl. kürzen?

\todo{Nodes erklären! Evtl.~oben!}

Folgende Auswahl an Eigenschaften und Funktionen zeichnet die in dieser Arbeit verwendete
Version 0.3 von open62541 aus:
\begin{itemize}
  \item Kommunikationionsstack
  \begin{itemize}
      \item OPC UA Binär-Protokoll (HTTP oder SOAP werden gegenwärtig nicht unterstützt)
      \item Austauschbare Netzwerk-Schicht, welche die Verwendung eigener Netzwerk-APIs
      erlaubt.
      \item Verschlüsselte Kommunikationion
      \item Asynchrone Dienst-Anfragen im Client
  \end{itemize}
  \item Informationsmodell
  \begin{itemize}
    \item Unterstützung aller OPC UA Node-Typen, inkl.~Methoden
    \item Hinzufügen und Entfernen von Nodes und Referenzen zur Laufzeit.
    \item Vererbung und Instanziierung von Objekt- und Variablentypen
    \item Zugriffskontrolle auch für einzelne Nodes
  \end{itemize}
  \item Subscriptions
  \begin{itemize}
    \item Erlaubt die Überwachung (subscriptions / monitoreditems)
    \item Sehr geringer Ressourcenbedarf pro überwachtem Wert
  \end{itemize}
  \item Code-Generierung auf XML-Basis
  \begin{itemize}
    \item Erlaubt die Erstellung von Datentypen
    \item Erlaubt die Generierung des serverseitigen Informationsmodells
  \end{itemize}
\end{itemize}

% https://open62541.org/doc/0.3/


Mozilla Public License
CC0 Lizenz für Beispiele und Plugins

% https://open62541.org/doc/open62541-current.pdf
% https://open62541.org/

% % % Imports nur für Referenzenauflösung während des Schreibens! Vorm Kompilieren auskommentieren!
% \bibliography{0_hauptdatei}
% \input{1_einleitung}
% \input{2_grundlagen}
% \input{3_konzeption}
% \input{4_implementierung}
% \input{5_tests}
% \input{6_zusammenfassung}
% \input{anhang}
% % Ende Imports

\section{Systemkonzept%
  \label{sec:3-konzeption}}
Auf Basis der in Abschnitt \ref{sec:2-grundlagen} vorgestellten Möglichkeiten folgt nun die Ausarbeitung eines Konzepts.
In den folgenden Abschnitten soll näher auf zwei zentrale Aspekte eingegangen werden: Abschnitt~\ref{sec:3-anbindung} stellt Möglichkeiten zum Zugriff auf Variablen bzw.\,Werte im Prozessabbild des Revolution Pi vor; in Abschnitt~\ref{sec:3-integration} wird ein Konzept zur Bereitstellung dieser Variablen auf einem OPC-Server vorgestellt.

\subsection{Anbindung der IO an den OPC-Server%
     \label{sec:3-anbindung}}

Eine Webanwendung mit Bezeichnung PiCtory dient zur Konfiguration der I/O- und virtuellen Module des RevolutionPi. Die Konfiguration liegt im JSON-Format in der Datei \lstinline{/etc/revpi/config.rsc}. Der piControl-Treiber liest diese Datei beim Start. 
Der folgende Auszug aus der Manpage des piControl-Kernelmoduls beschreibt die von diesem zum Lesen und Schreiben einzelner Bits des Prozessabbildes bereitgestellten Funktionen~\citep[vgl.]{web-revpi-manpage}. Sie ist an dieser Stelle weitgehend ungekürzt zitiert, da sie die nutzbare Schnittstelle sehr kompakt beschreibt.

\begin{lstlisting}[breakindent=0pt, numbers=none, caption={Auszug aus der Revolution Pi Programmers Manual\label{lst:4-manpage}}]
KB_FIND_VARIABLE SPIVariable *argp
Find a variable in the process image by its name. A pointer to a structure of type SPIVariable must be passed as argument. [...]
The struct SPIVariable [...] is defined as 
typedef struct SPIVariableStr
{
    char strVarName[32]; // Variable name
    uint16_t i16uAddress; // Address of the byte in the process image
    uint8_t i8uBit; // 0-7 bit position, >= 8 whole byte
    uint16_t i16uLength; // length of the variable in bits.
    // Possible values are 1, 8, 16 and 32
} SPIVariable;

Set and get values of the process image
KB_GET_VALUE SPIValue *argp
[...]
KB_SET_VALUE SPIValue *argp
Write one bit or one byte to the process image [...].  This call is more efficient than the usual calls of seek and write because only one function call is necessary. If more than on application are writing bits in one output byte, this call is the only safe way to set a bit without overwriting the other bits because this call is doing a read-modify-write-cycle. 

The struct SPIValue used by this ioctl is defined as
typedef struct SPIValueStr
{
    uint16_t i16uAddress; // Address of the byte in the process image
    uint8_t i8uBit; // 0-7 bit position, >= 8 whole byte
    uint8_t i8uValue; // Value: 0/1 for bit access, whole byte otherwise
} SPIValue;
\end{lstlisting} 

Die oben beschriebenden Funtkionen \lstinline{KB_FIND_VARIABLE}, \lstinline{KB_GET_VALUE} und \lstinline{KB_SET_VALUE} ermöglichen einen einfachen und (lt.\,Manpage) effizienten Zugriff auf einzelne Bits des Prozessabbildes und damit auch auf die IO des RevolutionPi.
Der Zugriff des OPC-Servers auf das Prozessabbild soll daher mittels dieser Funktionen realisiert werden.
\lstinline{KB_FIND_VARIABLE} kann genutzt werden, um Adressen von Variablen im Prozessabbild mittels ihres Namens aufzulösen.
\lstinline{KB_GET_VALUE} und \lstinline{KB_SET_VALUE} ermöglichen den Zugriff auf die Werte dieser Variablen.


\subsection{Integration des OPC-Servers in das System%
     \label{sec:3-integration}}

open62541 bietet drei Möglichkeiten zum Abgleich von Variablen mit dem Prozessabbild~\citep[vgl.][Tutorials - Connecting a Variable with a Physical Process]{web-open62541}:
\begin{itemize}
    \item Manuelles oder zyklisches Aktualisieren
    \item Variable Value Callback
    \item Variable Datasource
\end{itemize}

Die zyklische Aktualisierung eines oder mehrerer Werte nimmt, abhängig von der Zykluszeit, viele Systemressourcen in Anspruch. Value Callbacks ermöglichen es, einen Variablenwert effizienter mit einer Ressource wie etwa einem Prozessabbild zu synchronisieren. An die Variable wird ein Callback angehängt, welches vor jedem Lesen und nach jedem Schreibvorgang ausgeführt wird.
Der Wert der Variablen wird weiterhin im Variablenknoten auf dem OPC-Server gespeichert, der Abgleich mit der verknüpften Ressource erfolgt durch die Callback-Methoden.

Sogenannte Datenquellen gehen noch einen Schritt weiter. Der Server leitet jede Lese- und Schreibanforderung direkt an eine Callback-Funktion weiter. Beim Lesen liefert der Rückruf eine Kopie des aktuellen Wertes. Die Datenquelle muss intern ein eigenes Speichermanagement implementieren.

Der Zugriff auf die Werte des Prozessabbildes erfolgt, wie in Abschnitt~\ref{sec:3-anbindung} beschrieben, über von piControl bereitgestellte Methoden. Um die durch open62541 gepflegte OPC-Datenstruktur und das durch piControl verwaltete Prozessabbild möglichst effektiv verknüpfen zu können, soll diese Interaktion mittels Datenquellen und den zugehörigen Callbacks implementiert werden.
% % % Imports nur für Referenzenauflösung während des Schreibens! Vorm Kompilieren auskommentieren!
% \bibliography{0_hauptdatei}
% \input{1_einleitung}
% \input{2_grundlagen}
% \input{3_konzeption}
% \input{4_implementierung}
% \input{5_tests}
% \input{6_zusammenfassung}
% \input{anhang}
% % Ende Imports

\section{Implementierung%
  \label{sec:4-implementierung}}
Das folgende Kapitel stellt in Auszügen die Implementierung des OPC-Servers sowie die Anbindung an die IO-Module
der SPS dar. Der Schwerpunkt liegt hierbei auf der Funktionsweise des piControl-Treibers und dessen Integration in das Projekt. Abschnitt~\ref{sec:4-picontrol} erklärt die zum Schreibens eines Bits verwendeten Funktionsaufrufe.
Zuvor soll jedoch in Abschnitt~\ref{sec:4-open62541} der Teil des OPC-Servers vorgestellt werden, welcher auf besagten Treiber zugreift. 

\subsection{Implementierung des OPC-Servers%
     \label{sec:4-open62541}}
Wie im vorangegangenen Abschnitt~\ref{sec:3-integration} begründet, soll die Verknüpfung zwischen dem Prozessabbild der SPS und den auf dem OPC-Server bereitgestellten Werten über sog.\,Datenquellen erfolgen. Hierzu ist zunächst eine Callback-Methode zu implementieren, welche bei einem Lese- oder Schreibzugriff auf eine Variable aufgerufen wird. Die Verknüpfung zwischen Callback-Methode und Variable muss manuell erfolgen.

\begin{lstlisting}[language={c},firstnumber=237,caption={Auszug der Methode \lstinline{linkDataSourceVariable} in \lstinline{variables.c}\label{lst:4-linkDataSourceVariable}}]
extern UA_StatusCode
 linkDataSourceVariable(UA_Server *server, UA_NodeId nodeId) {
     bool readonly = false;
     UA_DataSource dataSourceVariable;
     UA_StatusCode rc; |>\setcounter{lstnumber}{254}<|

     dataSourceVariable.read = readDataSourceVariable;
     if (!readonly)
        dataSourceVariable.write = writeDataSourceVariable;
     else
        dataSourceVariable.write = writeReadonlyDataSourceVariable;

     return UA_Server_setVariableNode_dataSource(server, nodeId, dataSourceVariable);
 }
\end{lstlisting}

\begin{figure}[h]
    \centering
    \includegraphics[width=0.42\textwidth]{doc/img/OPC_RevPiDO.pdf}
    \caption{Auszug des verwendeten Nodesets, hier Digitalausgang 1 des Versuchsaufbaus
      \label{fig:opc-do}}
\end{figure}

Die in Listing~\ref{lst:4-linkDataSourceVariable} abgebildete Methode \lstinline{linkDataSourceVariable()} erzeugt ein Struct vom Typ \lstinline{UA_DataSource}. In diesem werden dem Lesen und Schreiben einer OPC-Variablen entsprechende Callback-Methoden zugewiesen. Die Verknüpfung einer OPC-Variable, genauer ihrer NodeId, mit der zuvor definierten Datenquelle erfolgt über die von open62541 bereitgestellte Methode \lstinline{UA_Server_setVariableNode_dataSource()}. Vor dem Lesen und nach dem Schreiben dieser Variable werden von nun an die entsprechenden Callbacks aufgerufen.
     
\begin{lstlisting}[language={c},firstnumber=168,caption={Auszug des Callbacks \lstinline{writeDataSourceVariable} in \lstinline{variables.c}\label{lst:4-writeDataSourceVariable}}]  
extern UA_StatusCode
 writeDataSourceVariable(UA_Server *server,
            const UA_NodeId *sessionId, void *sessionContext,
            const UA_NodeId *nodeId, void *nodeContext,
            const UA_NumericRange *range, const UA_DataValue *dataValue) {

    UA_StatusCode retval  = UA_STATUSCODE_GOOD;
    UA_NodeId *nameNodeId = UA_malloc(sizeof(UA_NodeId));
    UA_QualifiedName nameQN = UA_QUALIFIEDNAME(1, "Name");
    UA_Variant nameVar;
    UA_Boolean bit;

    retval |= findSiblingByBrowsename(server, nodeId, &nameQN, nameNodeId);
    retval |= UA_Server_readValue(server, *nameNodeId, &nameVar);
    retval |= UA_Boolean_copy(dataValue->value.data, &bit);

    |>\tikzmarkin[set border color=martinired]{writeIO}<|PI_writeSingleIO(String_fromUA_String(nameVar.data), &bit, false);                                                 |>\tikzmarkend{writeIO}<|

    free(nameNodeId);
    return retval;
 }
\end{lstlisting}

Listing~\ref{lst:4-writeDataSourceVariable} zeigt die Callback-Methode, welche nach dem Schreiben einer Variablen auf dem OPC-Server aufgerufen wird.
Dieser Methode wird neben der NodeId der mit ihr verknüpften Variablen auch der Wert dieser in Form eines Zeigers auf ein Struct vom Typ \lstinline{UA_DataValue} übergeben.

Die Gestaltung des hier verwendeten Nodesets sieht vor, dass in einer OPC-Variablen \lstinline{"Name"} der Bezeichner des zu schreibenden Digitalausgangs hinterlegt ist, siehe Abbildung~\ref{fig:opc-do}. Dies erlaubt eine Rekonfiguration der Ein- und Ausgänge der SPS ohne Änderungen im Programmcode des OPC-Servers vornehmen zu müssen.
Es ist daher erforderlich, nach jedem Schreiben einer mit einem Digitalausgang verknüpften Variablen, hier \lstinline{"Value"}, dessen Bezeichner \lstinline{"Name"} abzufragen. 
Dies geschieht in den Zeilen 180 und 181.
Anschließend wird dieser Bezeichner sowie der zu schreibende Wert der Methode \lstinline{PI_writeSingleIO()} übergeben, welche wiederum die Interaktion mit piControl übernimmt (vgl. Abschnitt \ref{sec:4-picontrol}).
 
\subsection{Integration von piControl%
     \label{sec:4-picontrol}}
In Abschnitt~\ref{sec:2-io} wurde die Anbindung der IO-Module des Revolution Pi sowie die Funktionsweise von piControl aus Anwendersicht beschrieben. Die verfügbare Literatur beschränkt sich auch auf lediglich diese Sicht; eine weiterführende Dokumentation für Entwickler gibt es, neben der in Abschnitt~\ref{sec:3-anbindung} vorgestellten Manpage, nicht. 
In diesem Abschnitt soll daher der Quellcode von piControl sowie dessen Verwendung im Projekt genauer betrachtet werden.
Hierzu wird exemplarisch die in Abschnitt~\ref{sec:4-open62541} eingeführte Methode \lstinline{PI_writeSingleIO()} untersucht.
Diese Methode ermöglicht das Setzen eines einzelnen Bits im Prozessabbild der SPS, und damit das Schalten eines digitalen Ausgangs auf einem IO-Modul.
Die äquivalente Methode \lstinline{int piControlGetBitValue(SPIValue *pSpiValue)} zum Lesen eines Bits bzw. Eingangs funktioniert analog und soll daher an dieser Stelle nicht dediziert erörtert werden.

\begin{lstlisting}[language={c},firstnumber=97,
                   caption={Setzen eines phsikalischen, digitalen Ausgangs in \lstinline{revpi.c}
                   \label{lst:4-PI_writeSingleIO}}]
extern void PI_writeSingleIO(char *pszVariableName, bool *bit, bool verbose)
{
	int rc;
	SPIVariable sPiVariable;
	SPIValue sPIValue;

	strncpy(sPiVariable.strVarName, pszVariableName, sizeof(sPiVariable.strVarName));
	rc = piControlGetVariableInfo(&sPiVariable);
	if (rc < 0) {
		printf("Cannot find variable '%s'\n", pszVariableName);
		return;
	}

		sPIValue.i16uAddress = sPiVariable.i16uAddress;
		sPIValue.i8uBit = sPiVariable.i8uBit;
		sPIValue.i8uValue = *bit;
		rc = |>\tikzmarkin[set border color=martinired]{setBitValue}<|piControlSetBitValue(&sPIValue)|>\tikzmarkend{setBitValue}<|;
		if (rc < 0)
			printf("Set bit error %s\n", getWriteError(rc));
		else if (verbose)
			printf("Set bit %d on byte at offset %d. Value %d\n", sPIValue.i8uBit, sPIValue.i16uAddress,
			       sPIValue.i8uValue);
}
\end{lstlisting}

Der Programmcode in Listing~\ref{lst:4-PI_writeSingleIO} ist Teil des implementierten OPC-Servers. In diesem wird auf zwei Funktionen des piControl-Treibers zugegriffen. 
Beiden Methoden wird als Argument ein Zeiger auf ein Struct vom Typ \lstinline{SPIValue} übergeben. Der im Struct abgelegte Name wird mittels \lstinline{piControlGetVariableInfo(&sPIValue)} zu einer Adresse im Prozessabbild aufgelöst. Diese wird in \lstinline{sPIValue.i16uAdress} gespeichert. Der Wert der Variablen wird anschließend mittels \lstinline{piControlSetBitValue(&sPIValue)} an dieser Adresse in das Prozessabbild geschrieben.

\begin{lstlisting}[language={c},firstnumber=309,caption={Methode \lstinline{piControlSetBitValue} in \lstinline{piControlIf.c}\label{lst:4-piControlSetBitValue}}]
int |>\tikzmarkin[set border color=martiniblue]{setBitValueFcn}<|piControlSetBitValue(SPIValue *pSpiValue)|>\tikzmarkend{setBitValueFcn}<|
{
    piControlOpen();

    if (PiControlHandle_g < 0)
	    return -ENODEV;

    pSpiValue->i16uAddress += pSpiValue->i8uBit / 8;
    pSpiValue->i8uBit %= 8;

    if (|>\tikzmarkin[set border color=martinired]{ioctl}<|ioctl(PiControlHandle_g, KB_SET_VALUE, pSpiValue)|>\tikzmarkend{ioctl}<| < 0)
	    return errno;

    return 0;
}
\end{lstlisting}

Die in Listing~\ref{lst:4-piControlSetBitValue} dargestellte Methode \lstinline{piControlSetBitValue} ist lediglich eine Hüllfunktion (häufig auch als Wrapper-Funktion bezeichnet) für einen Aufruf des \lstinline{ioctl} Kernel-Moduls.
Folgende Parameter werden übergeben:
\lstinline{PiControlHandle_g} ist die Referenz auf die Geräte-Datei des piControl-Treibers. \lstinline{KB_SET_VALUE} ist das ioctl-Kommando zum Schreiben eines Bits in das Prozessabbild. Der Zeiger \lstinline{pSpiValue} verweist auf ein Struct des bereits vorgestellten Typs \lstinline{SPIValue}.

\begin{lstlisting}[language={c},firstnumber=80,caption={Methode \lstinline{piControlOpen} in \lstinline{piControlIf.c}\label{lst:4-piControlOpen}}]
void piControlOpen(void)
{
    /* open handle if needed */
    if (PiControlHandle_g < 0)
    {
	    |>\tikzmarkin[set border color=martiniblue]{PiControlHandle}<|PiControlHandle_g = open(PICONTROL_DEVICE, O_RDWR)|>\tikzmarkend{PiControlHandle}<|;
    }
}
\end{lstlisting}

Die in Listing~\ref{lst:4-piControlOpen} dargestellte Methode öffnet, sofern nicht bereits geschehen, die Geräte-Datei. Das Macro \lstinline{PICONTROL_DEVICE} verweist hierbei auf \lstinline{/dev/piControl0}.

\begin{lstlisting}[language={c},firstnumber=721,caption={Methode \lstinline{piControlIoctl} in \lstinline{piControlMain.c}\label{lst:4-piControlIoctl}}]
static long |>\tikzmarkin[set border color=martiniblue, below offset=0.9em]{piControlIoctl}<|piControlIoctl(struct file *file, unsigned int prg_nr, 
                           unsigned long usr_addr)                                      |>\tikzmarkend{piControlIoctl}<|
{
  int status = -EFAULT;
  tpiControlInst *priv;
  int timeout = 10000;	// ms

  if (prg_nr != KB_CONFIG_SEND && prg_nr != KB_CONFIG_START && !isRunning()) {
  	return -EAGAIN;
  }

  priv = (tpiControlInst *) file->private_data;

  if (prg_nr != KB_GET_LAST_MESSAGE) {
  	// clear old message
  	priv->pcErrorMessage[0] = 0;
  }

  switch (prg_nr) {|>\setcounter{lstnumber}{864}<|

    case |>\tikzmarkin[set border color=martiniblue]{KB_SET_VALUE}<|KB_SET_VALUE:|>\tikzmarkend{KB_SET_VALUE}<|
  		{
  			SPIValue *pValue = (SPIValue *) usr_addr;

  			if (!isRunning())
  				return -EFAULT;

  			if (pValue->i16uAddress >= KB_PI_LEN) {
  				status = -EFAULT;
  			} else {
  				INT8U i8uValue_l;
  				my_rt_mutex_lock(&piDev_g.lockPI);
  				i8uValue_l = piDev_g.ai8uPI[pValue->i16uAddress];

  				if (pValue->i8uBit >= 8) {
  					i8uValue_l = pValue->i8uValue;
  				} else {
  					if (pValue->i8uValue)
  						i8uValue_l |= (1 << pValue->i8uBit);
  					else
  						i8uValue_l &= ~(1 << pValue->i8uBit);
  				}

  				|>\tikzmarkin[set border color=martinired]{i8uValue}<|piDev_g.ai8uPI[pValue->i16uAddress] = i8uValue_l;|>\tikzmarkend{i8uValue}<|
  				rt_mutex_unlock(&piDev_g.lockPI);

  #ifdef VERBOSE
  				pr_info("piControlIoctl Addr=%u, bit=%u: %02x %02x\n", pValue->i16uAddress, pValue->i8uBit, pValue->i8uValue, i8uValue_l);
  #endif

  				status = 0;
  			}
  		}
  		break; |>\setcounter{lstnumber}{1314}<|

    default:
      pr_err("Invalid Ioctl");
      return (-EINVAL);
      break;

    }

    return status;
  }
\end{lstlisting}

Listing~\ref{lst:4-piControlIoctl} zeigt in Auszügen die ioctl-Methode des piControl Kernel-Treibers. Diese bekommt folgende Argumente übergeben: \lstinline{struct file *file} enthält den Verweis auf die Geräte-Datei, hier \lstinline{/dev/piControl0}. Der Wert von \lstinline{unsigned int prg_nr} beschreibt die Anfrage an den Treiber, in diesem Fall \lstinline{KB_SET_VALUE}. Das Argument \lstinline{unsigned long usr_addr} enthält einen typ-agnostischen Pointer. Dieser verweist auf einen Speicherbereich, in welchem die zur Bearbeitung der Anfrage notwendigen Daten abgelegt sind. Hier können auch vom Treiber empfangene Daten dem Anwendungsprogramm bereitgestellt werden. 

Die switch-case-Anweisung führt die über das Argument \lstinline{prg_nr} spezifizierte Aktion aus. Hier betrachten wir \lstinline{KB_SET_VALUE}:
Zunächst wird in Zeile 868 der übergebene Zeiger \lstinline{usr_addr} mittels explizitem Typecast zu einem Zeiger des Typs \lstinline{SPIValue *} konvertiert. Da dieser auf Daten im Userspace verweist, ist beim Zugriff durch den Kernel-Treiber besondere Vorsicht geboten.
In Zeile 877 wird mittels Mutex das Prozessabbild \lstinline{piDev_g} für den Zugriff durch andere Threads oder Prozesse gesperrt.
\lstinline{my_rt_mutex_lock} verweist hierbei auf die Funktion \lstinline{rt_mutex_lock} aus \lstinline{linux/sched.h}\footnote{Offenbar wurde hier auch eine alternative Implementierung vorgesehen, siehe revpi\_common.h}

In Zeile 889 wird das Byte \lstinline{i8uValue_l}, welches den zu schreibenden Wert enthält in das Prozessabbild übertragen. Anschließend wird die Mutex auf \lstinline{piDev_g} wieder entsperrt.
\newpage

\begin{lstlisting}[language={c},firstnumber=62,caption={Auszug des Struct \lstinline{spiControlDev} in \lstinline{piControlMain.h}\label{lst:4-spiControlDev}}]
|>\tikzmarkin[set border color=martiniblue]{spiControlDev}<|typedef struct spiControlDev|>\tikzmarkend{spiControlDev}<| {
	// device driver stuff
	int init_step;
	enum revpi_machine machine_type;
	void *machine;
	struct cdev cdev;	// Char device structure
	struct device *dev;
	struct thermal_zone_device *thermal_zone;

	|>\tikzmarkin[set border color=martiniblue]{processImage}<|// process image stuff
	INT8U ai8uPI[KB_PI_LEN];
	INT8U ai8uPIDefault|>\tikzmarkin[set border color=martinired]{KB_PI_LEN_0}<|[KB_PI_LEN]|>\tikzmarkend{KB_PI_LEN_0}<|;
	struct rt_mutex lockPI;        |>\tikzmarkend{processImage}<|
	bool stopIO;
	piDevices *devs; |>\setcounter{lstnumber}{94}<|
} tpiControlDev;
\end{lstlisting}

Das Prozessabbild ist als Byte-Array der Länge \lstinline{KB_PI_LEN} in Listing~\ref{lst:4-spiControlDev} definiert. Konfigurationsparameter wie \lstinline{KB_PI_LEN} oder die Zykluszeit für den Datenaustausch zwischen SPS und IO-Modulen sind im folgenden Listing~\ref{lst:4-process} definiert.

\begin{lstlisting}[language={c},firstnumber=119,caption={Konfigurationsparameter des Prozessabbildes in project.h\label{lst:4-process}}]
#define INTERVAL_PI_GATE (5*1000*1000)  // 5 ms piGateCommunication |>\setcounter{lstnumber}{128}<|

#define INTERVAL_IO_COM (5*1000*1000)  // 5 ms piIoComm |>\setcounter{lstnumber}{132}<|

#define KB_PD_LEN       512
|>\tikzmarkin[set border color=martiniblue]{KB_PI_LEN_1}<|#define KB_PI_LEN       4096|>\tikzmarkend{KB_PI_LEN_1}<|
\end{lstlisting}

Das zu setzende Bit wurde zu diesem Zeitpunkt erfolgreich in das Prozessabbild der SPS geschrieben.
Es stellt sich die Frage, wie dieses nun an das IO-Modul kommuniziert wird.
Die Kommunikation mit allen angebundenen Modulen ist ebenfalls Aufgabe des piControl-Treibers.

\begin{lstlisting}[language={c},firstnumber=256,caption={Auszug der Methode \lstinline{piIoThread} in \lstinline{revpi_core.c}\label{lst:4-piIoThread}}]
static int piIoThread(void *data)
{
	//TODO int value = 0;
	ktime_t time;
	ktime_t now;
	s64 tDiff;

	hrtimer_init(&piCore_g.ioTimer, CLOCK_MONOTONIC, HRTIMER_MODE_ABS);
	piCore_g.ioTimer.function = piIoTimer;

	pr_info("piIO thread started\n");

	now = hrtimer_cb_get_time(&piCore_g.ioTimer);

	PiBridgeMaster_Reset();

	while (!kthread_should_stop()) {
		if (|>\tikzmarkin[set border color=martinired]{PiBridgeMaster}<|PiBridgeMaster_Run()|>\tikzmarkend{PiBridgeMaster}<| < 0)
			break;
	}

	RevPiDevice_finish();

	pr_info("piIO exit\n");
	return 0;
}
\end{lstlisting}

Der Kernel-Thread \lstinline{piIoThread} ist verantwortlich für den zyklischen Datenaustausch mit den IO-Modulen. In diesem wird fortlaufend die Methode \lstinline{PiBridgeMaster_Run()} aufgerufen, siehe Listing~\ref{lst:4-piIoThread}.

\begin{lstlisting}[language={c},firstnumber=262,caption={Auszug der Methode \lstinline{PiBridgeMaster_Run(void)} in \lstinline{RevPiDevice.c}\label{lst:4-PiBridgeMaster_Run}}]
int PiBridgeMaster_Run(void)
{
	static kbUT_Timer tTimeoutTimer_s;
	static kbUT_Timer tConfigTimeoutTimer_s;
	static int error_cnt;
	static INT8U last_led;
	static unsigned long last_update;
	int ret = 0;
	int i;

	my_rt_mutex_lock(&piCore_g.lockBridgeState);
	if (piCore_g.eBridgeState != piBridgeStop) {
		switch (eRunStatus_s) { |>\setcounter{lstnumber}{514}<|
		    case enPiBridgeMasterStatus_EndOfConfig:|>\setcounter{lstnumber}{621}<|
		    if (|>\tikzmarkin[set border color=martinired]{RevPiDevice}<|RevPiDevice_run()|>\tikzmarkend{RevPiDevice}<|) {
				// an error occured, check error limits |>\setcounter{lstnumber}{641}<|
			} else {
				ret = 1;
			}
			piCore_g.image.drv.i16uRS485ErrorCnt = RevPiDevice_getErrCnt();
			break;
\end{lstlisting}

Die in Listing~\ref{lst:4-PiBridgeMaster_Run} dargestellte Methode ist eine sog. State-Machine. Ist die Konfiguration der IO-Module erfolgreich abgeschlossen, so führt sie bei Aufruf lediglich die Methode \lstinline{RevPiDevice_run()} aus.

\begin{lstlisting}[language={c},firstnumber=140,caption={Auszug der Methode \lstinline{RevPiDevice_run(void)} in \lstinline{RevPiDevice.c}\label{lst:4-RevPiDevice_run}}]
int RevPiDevice_run(void)
{
	INT8U i8uDevice = 0;
	INT32U r;
	int retval = 0;

	RevPiDevices_s.i16uErrorCnt = 0;

	for (i8uDevice = 0; i8uDevice < RevPiDevice_getDevCnt(); i8uDevice++) {
		if (RevPiDevice_getDev(i8uDevice)->i8uActive) {
			switch (RevPiDevice_getDev(i8uDevice)->sId.i16uModulType) {
			case KUNBUS_FW_DESCR_TYP_PI_DIO_14:
			case KUNBUS_FW_DESCR_TYP_PI_DI_16:
			case KUNBUS_FW_DESCR_TYP_PI_DO_16:
				r = |>\tikzmarkin[set border color=martinired]{sendCyclicTelegram}<|piDIOComm_sendCyclicTelegram(i8uDevice)|>\tikzmarkend{sendCyclicTelegram}\setcounter{lstnumber}{166} <|;

				break; |>\setcounter{lstnumber}{216}<|
			}
		}
	} |>\setcounter{lstnumber}{227}<|
	return retval;
}
\end{lstlisting}

Diese iteriert wie in Listing~\ref{lst:4-RevPiDevice_run} abgebildete durch alle gegenwärtig in der SPS konfigurierten Module. Ist das aktuelle Modul als aktiv markiert, so wird anhand eines sog. Firmware-Descriptors entschieden, welche Methode für die Ansteuerung des Moduls aufzurufen ist.

\begin{lstlisting}[language={c},firstnumber=161,caption={Auszug der Methode \lstinline{piDIOComm_sendCyclicTelegram} in \lstinline{piDIOComm.c}\label{lst:4-sendCyclicTelegram}}]
INT32U piDIOComm_sendCyclicTelegram(INT8U i8uDevice_p)
{
	INT32U i32uRv_l = 0;
	SIOGeneric sRequest_l;
	SIOGeneric sResponse_l;
	INT8U len_l, data_out[18], i, p, data_in[70];
	INT8U i8uAddress;
	int ret; |>\setcounter{lstnumber}{239}<|
	
    |>\tikzmarkin[set border color=martinired]{piIoComm}<|ret = piIoComm_send((INT8U *) & sRequest_l, IOPROTOCOL_HEADER_LENGTH + len_l + 1);  |>\tikzmarkend{piIoComm}\setcounter{lstnumber}{298}<|
}
\end{lstlisting}

Im Falle des hier verwendeten DO-Moduls wird die in Listing~\ref{lst:4-sendCyclicTelegram} abgebildete Methode \lstinline{piDIOComm_sendCyclicTelegram()} aufgerufen. Dieser wird ein Zeiger auf das zu schreibende Gerät übergeben. 
Zunächst wird das Prozessabbild mittels eines proprietären, jedoch im Quellcode offen nachvollziehbaren Protokolls in ein \lstinline{sRequest_l} genanntes Byte-Array umgewandelt. Dieser Schritt ist in Listing~\ref{lst:4-sendCyclicTelegram} nicht abgebildet. Anschließend wird \lstinline{piIoComm_send()} ein Zeiger auf die so generierte Schreib-Anfrage übergeben.

\begin{lstlisting}[language={c},firstnumber=220,caption={Auszug der Methode \lstinline{piIOComm_send} in \lstinline{piIOComm.c}\label{lst:4-piIOComm_send}}]
int piIoComm_send(INT8U * buf_p, INT16U i16uLen_p)
{
	ssize_t write_l = 0;
	INT16U i16uSent_l = 0;|>\setcounter{lstnumber}{249}<|

	while (i16uSent_l < i16uLen_p) {
		write_l = vfs_write(piIoComm_fd_m, buf_p + i16uSent_l, i16uLen_p - i16uSent_l, &piIoComm_fd_m->f_pos);
		if (write_l < 0) {
			pr_info_serial("write error %d\n", (int)write_l);
			return -1;
		} 
		i16uSent_l += write_l;|>\setcounter{lstnumber}{263}<|
	}
	clear();
	vfs_fsync(piIoComm_fd_m, 1);
	return 0;
}
\end{lstlisting}

Listing~\ref{lst:4-piIOComm_send} zeigt die Implementierung von \lstinline{piIoComm_send()}. Diese Methode ist für das Schreiben der oben generierten Anfrage auf die seriellen Schnittstelle verantwortlich. Realisiert wird dies mittels der Methode \lstinline{vfs_write()}. Diese ist in \lstinline{<linux/fs.h>} definiert. Sie ermöglicht das Schreiben einer Datei im Userspace aus dem Kernel heraus. Geschrieben wird hier die Datei mit dem Deskriptor \lstinline{piIoComm_fd_m}.
Da die Funktion \lstinline{vfs_write()} durch andere Kernel-Tasks unterbrochen werden kann, ist nicht gewährleistet, dass die gesamte Anfrage mit nur einem Aufruf geschrieben wird. Die oben abgebildete while-Schleife stellt das vollständige Senden der Anfrage sicher.

\begin{lstlisting}[language={c},firstnumber=157,caption={Auszug der Methode \lstinline{piIOComm_open_serial} in \lstinline{piIOComm.c}\label{lst:4-piIOComm_open_serial}}]
int piIoComm_open_serial(void)
{   |>\setcounter{lstnumber}{167}<|
	struct file *fd;	/* Filedeskriptor */
	struct termios newtio;	/* Schnittstellenoptionen */

	|>\tikzmarkin[set border color=martiniblue]{fd}<|/* Port oeffnen - read/write, kein "controlling tty", 
	    Status von DCD ignorieren */
	fd = filp_open(|>\tikzmarkin[set border color=martinired]{tty}<|REV_PI_TTY_DEVICE|>\tikzmarkend{tty}<|, O_RDWR | O_NOCTTY, 0); |>\setcounter{lstnumber}{208}<|
	
	piIoComm_fd_m = fd;                                                      |>\tikzmarkend{fd}\setcounter{lstnumber}{217}<|

	return 0;
}
\end{lstlisting}

Der zum Schreiben auf die serielle Schnittstelle verwendete Datei-Deskriptor wird von der in Listing~\ref{lst:4-piIOComm_open_serial} abgebildeten Methode \lstinline{piIoComm_open_serial()} generiert. 

\begin{lstlisting}[language={c},firstnumber=45,caption={Definition der seriellen Schnittstelle in \lstinline{piIOComm.h}\label{lst:4-REV_PI_TTY_DEVICE}}]
#define REV_PI_TTY_DEVICE	"/dev/ttyAMA0"
\end{lstlisting}

Das in Listing~\ref{lst:4-REV_PI_TTY_DEVICE} definierte Macro verweist auf eine der seriellen Schnittstellen des RaspberryPi.
Die Implementierung des zugehörigen Schnittstellentreibers soll hier nicht weiter untersucht werden. Somit ist an dieser Stelle die Kette vom Setzen einer Variablen auf dem OPC-Server bis hin zur Aktualisierung des Prozessabbilds der IO-Module geschlossen.

% \begin{lstlisting}[language={c},firstnumber={226},caption={Setzen der Scheduler-Priorität auf SCHED\_FIFO in 
% revpi\_common.c\label{lst:2-sched_priority}}]
% param.sched_priority = ktprio->prio;
% ret = sched_setscheduler(child, SCHED_FIFO, &param);
% \end{lstlisting}
% % % Imports nur für Referenzenauflösung während des Schreibens! Vorm Kompilieren auskommentieren!
% \bibliography{0_hauptdatei}
% \input{1_einleitung}
% \input{2_grundlagen}
% \input{3_konzeption}
% \input{4_implementierung}
% \input{5_tests}
% \input{6_zusammenfassung}
% % Ende Imports

\section{Test des OPC-Servers im Gesamtsystem%
  \label{sec:5-tests}}

% % % Imports nur für Referenzenauflösung während des schreibens! Vorm Kompilieren auskommentieren!
% \bibliography{0_hauptdatei}
% \input{1_einleitung}
% \input{2_grundlagen}
% \input{3_konzeption}
% \input{4_implementierung}
% \input{5_tests}
% \input{6_zusammenfassung}
% % Ende Imports

\section{Zusammenfassung und Ausblick%
  \label{sec:6-fazit}}
Der folgende Abschnitt~\ref{sec:6-zusammenfassung} fasst die gewonnenen Erkenntnisse und den Stand der Implementierung zusammen.
Den Abschluss dieser Arbeit bildet der Ausblick in Abschnitt~\ref{sec:6-ausblick}.

\subsection{Zusammenfassung%
     \label{sec:6-zusammenfassung}}

\subsection{Ausblick%
     \label{sec:6-ausblick}}

% % Ende Imports

\section{Grundlagen%
  \label{sec:2-grundlagen}}

\subsection{Speicherprogrammierbare-Steuerung und Linux -- Revolution Pi%
     \label{sec:2-sps}}

\subsubsection{Kunbus RevolutionPi%
        \label{sec:2-revpi}}
Der RevolutionPi 3 ist eine speicherprogrammierbare Steuerung (SPS) des Herstellers
Kunbus GmbH. Kern dieser SPS ist das von der Raspberry Pi Foundation entwickelte
und vertriebene Raspberry Pi Compute Module 3. Dieses integriert ein Broadcom BCM2837
System-on-Chip (SoC) mit vier 1,2GHz Prozessorkernen, 1GB RAM, 4GB eMMC Anwendungsspeicher
und sonstige Peripherie in ein Modul im DDR2-SODIMM Formfaktor. Diese Spezifikationen
sind weitgehend identisch zu denen des ausgesprochen populären Raspberry Pi 3.
Der Revolution Pi profitiert daher von dem gleichen großen Angebot an Software
und Unterstützung wie der Raspberry Pi, ergänzt dessen Hardware jedoch um eine 24V
Spannungsversorgung, die Möglichkeit der Erweiterung durch mehrere industrietaugliche
Ein-/ Ausgabemodule und Gateways sowie ein Gehäuse zur Montage auf einer DIN-Schiene.
\begin{itemize}
  \item{Prozessor: BCM2837}
  \item{Taktfrequenz 1,2 GHz}
  \item{Anzahl Prozessorkerne: 4}
  \item{Arbeitsspeicher: 1 GByte}
  \item{eMMC Flash Speicher: 4 GByte}
  \item{Betriebssystem: Angepasstes Raspbian mit RT-Patch}
  \item{RTC mit 24h Pufferung über wartungsfreien Kondensator}
  \item{Treiber / API: Treiber schreibt zyklisch Prozessdaten in ein Prozessabbild, Zugriff auf Prozessabbild über Linux-Filesystem als API zu Fremdsoftware.}
  \item{Kommunikationsanschlüsse: 2 x USB 2.0 A (je 500 mA belastbar), 1 x Micro-USB, HDMI, Ethernet (RJ45) 10/100 Mbit/s}
  \item{Stromversorgung: min. 10,7 V, max. 28,8 V, maximal 10 Watt}
  \item{Zulässige Umgebungstemperatur: -40 bis +55 C}
  \item{Gehäuseabmessungen: (HxBxL) 96 mm x 22,5 mm x 110,5 mm (ohne gesteckte Stecker)}
  \item{ESD Schutz: 4 kV / 8 kV gemäß EN61131-2 und IEC 61000-6-2}
  \item{Surge / Burst Prüfungen: gemäß EN61131-2 und IEC 61000-6-2 eingekoppelt auf Versorgungsspannung, Ethernet und IO-Leitungen}
  \item{EMI Prüfungen: gemäß EN61131-2 und IEC 61000-6-2}
\end{itemize}

Kunbus bietet eine Auswahl an IO- und Gateway-Modulen zur Erweiterung des Revolution Pi an.
Gateways dienen der Kommunikation mit Systemen oder Komponenten der Automatisierungstechnik
über Protokolle wie PROFIBUS oder EtherCAT. IO-Module erlauben die Überwachung
und Steuerung von digitalen oder analogen Ein- und Ausgängen.

\subsubsection{Zugriff auf IO-Module%
        \label{sec:2-io}}
Der Zugriff auf die Ein- und Ausgänge der IO-Module erfolgt über ein Prozessabbild
und einen hierfür von Kunbus bereitgestellten Treiber, genannt piControl. Dieser
aktualisiert das Prozessabbild zyklisch. Die angestrebte Zykluszeit beträgt 5ms,
kann jedoch je nach Anzahl der angeschlossenen Module auch größer sein. Kunbus
garantiert bei drei IO-Modulen und zwei Gateway-Modulen eine Zykluszeit von 10 ms.
Jedes der IO-Module stellt ein eigenständiges eingebettetes System dar. Es verfügt
über einen Microcontroller, welcher die IOs bereitstellt und über einen RS485-Bus
mit dem Revolution Pi kommuniziert.
% https://revolution.kunbus.de/io-modul/

Lizenz: GPL
% https://github.com/RevolutionPi/piControl

\begin{lstlisting}[language={c},firstnumber={226},caption={Setzen der Scheduler-Priorität auf SCHED\_FIFO in revpi\_common.c\label{lst:2-sched_priority}}]
param.sched_priority = ktprio->prio;
ret = sched_setscheduler(child, SCHED_FIFO,
       &param);
\end{lstlisting}


\subsection{Echtzeit und Multithreading unter Linux -- preemptRT und posix%
     \label{sec:2-echtzeit}}


 Der Linux-Kernel verfügt über mehrere unterschiedliche Preemtion-Modelle:

\begin{itemize}
  \item No Forced Preemption (server):
  Ausgelegt auf maximal möglichen Durchsatz, lediglich Interrupts und
  System-Call-Returns bewirken Präemption.

  \item Voluntary Kernel Preemption (Desktop):
  Neben den implizit bevorrechtigten Interrupts und System-Call-Returns gibt es
  in diesem Modell weitere Abschnitte des Kernels in welchen Preämption explizit
  gestattet ist.

  \item Preemptible Kernel (Low-Latency Desktop):
  In diesem Modell ist der gesamte Kernel, mit Ausnahme sog.~kritischer Abschnitte
  präemptible. Nach jedem kritischen Abschnitt gibt es einen impliziten Präemptions-Punkt.

  \item Preemptible Kernel (Basic RT):
  Dieses Modell ist dem zuvor genannten sehr ähnlich, hier sind jedoch alle Interrupt-Handler
  als eigenständige Threads ausgeführt.

  \item Fully Preemptible Kernel (RT):
  Wie auch bei den beiden zuvor genannten Modellen ist hier der gesamte Kernel
  präemtible, die Anzahl und Dauer der nicht-präemtiblen kritischen Abschnitte
  ist auf ein notwendiges Minimum beschränkt. Alle Interrupt-Handler sind als
  eigenständige Threads ausgeführt, Spinlocks durch Sleeping-Spinlocks und Mutexe
  durch sog.~RT-Mutexe ersetzt.

\end{itemize}
\todo{Spinlocks und Mutexe sowie die RT-Varianten dieser erklären!}

Lediglich mit dem vollständig präemtiblen Kernel kann Echtzeit-Verhalten realisiert werden.

% https://wiki.linuxfoundation.org/realtime/documentation/technical_basics/preemption_models bzw kernel/Kconfig.preempt

\subsubsection{preemptRT%
        \label{sec:2-preemptRT}}
% https://wiki.linuxfoundation.org/realtime/documentation/technical_details/start
% https://wiki.linuxfoundation.org/realtime/documentation/technical_basics/start

Das dem PREEMPT RT Kernel zugrunde liegende Prinzip lässt sich in einer einfachen
Regel ausdrücken: Nur Code, welcher absolut nicht-präemtible sein darf, ist es
gestattet nicht-präemtible zu sein.
Das erklärte Ziel des PREEMPT\_RT Patches ist es folglich, die Menge des nicht-präemtiblen
Codes im Linux-Kernel auf das absolut notwendige Minimum zu reduzieren.

Dies wird durch Verwendung folgender Mechanismen erreicht:

\begin{itemize}
  \item Hochauflösende Timer
  \item Sleeping Spinlocks
  \item Threaded Interrupt Handlers
  \item rt\_mutex
  \item RCU
\end{itemize}


\subsubsection{posix%
        \label{sec:2-posix}}
Ist posix hier wirklich relevant? Debian bzw.~Raspbian sind weitgehend posix
kompatibel, aber wird es hier genutzt? -> JA, open62541 nutzt pthread.h
piControl nutzt kthread.h, und semaphore.h

\subsection{OPC-UA und open62541%
     \label{sec:2-opc}}

\subsubsection{OPC UA%
        \label{sec:2-opcua}}
Open Platform Communications (OPC) ist eine Familie von Standards zur herstellerunabhängigen
Kommunikation von Maschinen (M2M) in der Automatisierungstechnik. Die sog.~OPC Task Force, zu deren
Mitgliedern verschiedene große Firmen der Automatisierungsindustrie gehören, veröffentlichte
die OPC Specification Version 1.0 im August 1996.
Motiviert ist dieser offene Standard durch die Erkenntniss, dass die Anpassung der
zahlreichen Herstellerstandards an individuelle Infrastrukturen und Anlagen einen
großen Mehraufwand verursachen.
Die Wikipedia beschreibt das Anwendungsgebiet für OPC wie folgt:

\glqq{}OPC wird dort eingesetzt, wo Sensoren, Regler und Steuerungen verschiedener Hersteller
ein gemeinsames Netzwerk bilden. Ohne OPC benötigten zwei Geräte zum Datenaustausch
genaue Kenntnis über die Kommunikationsmöglichkeiten des Gegenübers. Erweiterungen
und Austausch gestalten sich entsprechend schwierig. Mit OPC genügt es, für jedes
Gerät genau einmal einen OPC-konformen Treiber zu schreiben. Idealerweise wird
dieser bereits vom Hersteller zur Verfügung gestellt. Ein OPC-Treiber lässt sich
ohne großen Anpassungsaufwand in beliebig große Steuer- und Überwachungssysteme
integrieren.

OPC unterteilt sich in verschiedene Unterstandards, die für den jeweiligen Anwendungsfall
unabhängig voneinander implementiert werden können. OPC lässt sich damit verwenden
für Echtzeitdaten (Überwachung), Datenarchivierung, Alarm-Meldungen und neuerdings
auch direkt zur Steuerung (Befehlsübermittlung).\grqq{}

OPC basiert in der ursprünglichen Spezifikation auf Microsofts DCOM-Spezifikation.
DCOM macht Funktionen und Objekte einer Anwendung anderen Anwendungen im Netzwerk
zugänglich. Der OPC-Standard definiert entsprechende DCOM-Objekte um mit anderen
OPC-Anwendungen Daten austauschen zu können. Die Verwendung von DCOM bindet Anwender
an Betriebssysteme von Microsoft. Die ursprüngliche OPC Spezifikation wird durch die
Entwicklung von OPC Unified Architecture (OPC UA) abgelöst.
OPC UA setzt auf einem eigenen Kommunikationionsstack auf, die Verwendung von DCOM
und damit die Bindung an Microsoft wurden aufgelöst.

Die OPC-UA-Architektur ist eine Service-orientierte Architektur (SOA), deren Struktur
aus mehreren Schichten besteht.

% Wikipedia
Das OPC-Informationsmodell ist nicht mehr nur eine Hierarchie aus Ordnern, Items
und Properties. Es ist ein sogenanntes Full-Mesh-Network aus Nodes, mit dem neben
den Nutzdaten eines Nodes auch Meta- und Diagnoseinformationen repräsentiert werden.
Ein Node ähnelt einem Objekt aus der objektorientierten Programmierung. Ein Node
kann Attribute besitzen, die gelesen werden können (Data Access (DA), Historical
Data Access (HDA)). Es ist möglich Methoden zu definieren und aufzurufen.
Eine Methode besitzt Aufrufargumente und Rückgabewerte. Sie wird durch ein Command
aufgerufen. Weiterhin werden Events unterstützt, die versendet werden können
(AE (Alarms \& Events), DA DataChange), um bestimmte Informationen zwischen Geräten
auszutauschen. Ein Event besitzt unter anderem einen Empfangszeitpunkt, eine Nachricht
und einen Schweregrad. Die o. g. Nodes werden sowohl für die Nutzdaten als auch
alle anderen Arten von Metadaten verwendet. Der damit modellierte OPC-Adressraum
beinhaltet nun auch ein Typmodell, mit dem sämtliche Datentypen spezifiziert werden.

% https://de.wikipedia.org/wiki/Open_Platform_Communications
% https://de.wikipedia.org/wiki/OPC_Unified_Architecture
% https://opcfoundation.org/developer-tools/specifications-unified-architecture
% Von Gerhard Gappmeier - ascolab GmbH, CC BY-SA 3.0, https://de.wikipedia.org/w/index.php?curid=1892069
\subsubsection{open62541%
        \label{sec:2-open62541}}
open62541 ist eine offene und freie Implementierung von OPC UA. Die in C geschriebene
Bibliothek stellt eine beständig zunehmende Anzahl der im OPC UA Standard definierten
Funktionen bereit. Sie kann sowohl zur Erstellung von OPC-Servern als auch -Clients
genutzt werden. Ergänzend zu der unter der Mozilla Public License v2.0 lizensierten
Bibliothek stellt das open62541 Projekt auch Beispielprogramme unter einer CC0 Lizenz
zur Verfügung.

Die Bibliothek eignet sich auch für die Entwicklung auf eingebetteten Systemen und
Microcontrollern. Je nach Umfang der gewünschten Funktionen und des OPC Informationsmodells
beträgt die Größe einer Server-Binary weniger als 100kb. %evtl. kürzen?

\todo{Nodes erklären! Evtl.~oben!}

Folgende Auswahl an Eigenschaften und Funktionen zeichnet die in dieser Arbeit verwendete
Version 0.3 von open62541 aus:
\begin{itemize}
  \item Kommunikationionsstack
  \begin{itemize}
      \item OPC UA Binär-Protokoll (HTTP oder SOAP werden gegenwärtig nicht unterstützt)
      \item Austauschbare Netzwerk-Schicht, welche die Verwendung eigener Netzwerk-APIs
      erlaubt.
      \item Verschlüsselte Kommunikationion
      \item Asynchrone Dienst-Anfragen im Client
  \end{itemize}
  \item Informationsmodell
  \begin{itemize}
    \item Unterstützung aller OPC UA Node-Typen, inkl.~Methoden
    \item Hinzufügen und Entfernen von Nodes und Referenzen zur Laufzeit.
    \item Vererbung und Instanziierung von Objekt- und Variablentypen
    \item Zugriffskontrolle auch für einzelne Nodes
  \end{itemize}
  \item Subscriptions
  \begin{itemize}
    \item Erlaubt die Überwachung (subscriptions / monitoreditems)
    \item Sehr geringer Ressourcenbedarf pro überwachtem Wert
  \end{itemize}
  \item Code-Generierung auf XML-Basis
  \begin{itemize}
    \item Erlaubt die Erstellung von Datentypen
    \item Erlaubt die Generierung des serverseitigen Informationsmodells
  \end{itemize}
\end{itemize}

% https://open62541.org/doc/0.3/


Mozilla Public License
CC0 Lizenz für Beispiele und Plugins

% https://open62541.org/doc/open62541-current.pdf
% https://open62541.org/

%% % Imports nur für Referenzenauflösung während des Schreibens! Vorm Kompilieren auskommentieren!
% \bibliography{0_hauptdatei}
% % Mit \section{...} eröffnen wir einen neuen Abschnitt.
% Der Befehl setzt nicht nur den Text in einer größeren,
% fetten Schrift, sondern sorgt außerdem dafür, daß er im
% Inhaltsverzeichnis erscheint.
%
% Mit \label{...} erzeugen wir einen Bezeichner, mit dessen Hilfe
% wir später auf die Nummer des Abschnitts verweisen können (nämlich
% mit~\ref{...}).
%
% Das Kommentarzeichen hinter „Übersicht“ dient dazu, ein
% Leerzeichen zwischen „Übersicht“ und dem \label-Befehl
% zu vermeiden, das andernfalls sichtbar würde – z.B. im
% Inhaltsverzeichnis.
%

% % Imports nur für Referenzenauflösung während des Schreibens! Vorm Kompilieren auskommentieren!
% \bibliography{0_hauptdatei}
% \input{1_einleitung}
%\input{2_grundlagen}
%\input{3_konzeption}
%\input{4_implementierung}
%\input{5_tests}
%\input{6_zusammenfassung}
% % Ende Imports

\section{Einleitung und Motivation%
  \label{sec:1-einleitung}}
Ziel dieses Projektes ist die Integration eines OPC-Servers mit einer auf Linux
basierenden speicherprogrammierbaren Steuerung (SPS). Angeschlossen an diese SPS
ist jeweils ein digitales Ein-/\,bzw.~Ausgabemodul. Die von diesen bereitgestellten
Ein-/\, bzw.~Ausgänge (IO) sollen in der Datenstruktur des OPC-Servers abgebildet
und über diesen für OPC-Clients les-/\,und schreibar sein. Weiterhin sollen einige
Funktionen zur Überwachung und Steuerung der an die SPS angeschlossenen Aktoren
und Sensoren direkt im OPC-Server implementiert werden.
Hiermit stellt dieses Projekt eine der Grundlagen für ein übergeordnetes Projekt,
die cloudbasierte Steuerung eines miniaturisierten Produktions-Systems, dar.

Der hier verwendete OPC-Server ist Teil des sog. open62541 Projekts. Er ist in C
geschrieben und implementiert bereits einen großen Teil der im OPC-UA-Standard
spezifizierten Funktionen.
Als SPS findet ein Revolution Pi 3 der Firma Kunbus Verwendung. Dieser integriert
ein sog. Compute Module der Raspberry Pi Foundation in ein industrietaugliches
Gehäuse und erlaubt die Erweiterung mittels IO- oder Gateway-Modulen. Über diese
erfolgt die Kommunikation mit weiteren Komponenten der Automatisierungstechnik.

Motiviert ist dieses Projekt durch die Beobachtung, dass die Verbreitung offener
Standards sowie freier Software auch in der Automatisierungstechnik zunimmt.
Linux ist ein freies Betriebssystem, OPC-UA ein offen zugänglicher, aktiv gepflegter
und weit verbreiteter Standard. Der Raspberry Pi findet sowohl bei Hobby-Anwendern als
auch in den Bereichen Forschung und Entwicklung sowie bei industriellen Anwendern
Verwendung. Dieses Projekt stellt somit eine für unterschiedliche Anwender interessante
Entwicklung dar.

Im Anschluss an diese einleitende Übersicht im Abschnitt~\ref{sec:1-einleitung} folgt
die Darstellung der wichtigsten Grundlagen in Abschnitt~\ref{sec:2-grundlagen}.
Aufbauend auf diesen Grundlagen folgt die konzeptuelle Ausarbeitung im Abschnitt~\ref{sec:3-konzeption}.
Die Umsetzung wird im Abschnitt~\ref{sec:4-implementierung} erläutert.
Die Leistungsfähigkeit der Implementierung wird in Abschnitt~\ref{sec:5-tests} untersucht.
Eine Zusammenfassung und ein Ausblick schließen die Arbeit in
Abschnitt~\ref{sec:6-fazit} ab. Eventuell noch benötigte Anhänge
finden sich in den Anhängen [...] bis [...].

% % % Imports nur für Referenzenauflösung während des Schreibens! Vorm Kompilieren auskommentieren!
% \bibliography{0_hauptdatei}
% \input{1_einleitung}
% \input{2_grundlagen}
% \input{3_konzeption}
% \input{4_implementierung}
% \input{5_tests}
% \input{6_zusammenfassung}
% % Ende Imports

\section{Grundlagen%
  \label{sec:2-grundlagen}}

\subsection{Speicherprogrammierbare-Steuerung und Linux -- Revolution Pi%
     \label{sec:2-sps}}

\subsubsection{Kunbus RevolutionPi%
        \label{sec:2-revpi}}
Der RevolutionPi 3 ist eine speicherprogrammierbare Steuerung (SPS) des Herstellers
Kunbus GmbH. Kern dieser SPS ist das von der Raspberry Pi Foundation entwickelte
und vertriebene Raspberry Pi Compute Module 3. Dieses integriert ein Broadcom BCM2837
System-on-Chip (SoC) mit vier 1,2GHz Prozessorkernen, 1GB RAM, 4GB eMMC Anwendungsspeicher
und sonstige Peripherie in ein Modul im DDR2-SODIMM Formfaktor. Diese Spezifikationen
sind weitgehend identisch zu denen des ausgesprochen populären Raspberry Pi 3.
Der Revolution Pi profitiert daher von dem gleichen großen Angebot an Software
und Unterstützung wie der Raspberry Pi, ergänzt dessen Hardware jedoch um eine 24V
Spannungsversorgung, die Möglichkeit der Erweiterung durch mehrere industrietaugliche
Ein-/ Ausgabemodule und Gateways sowie ein Gehäuse zur Montage auf einer DIN-Schiene.
\begin{itemize}
  \item{Prozessor: BCM2837}
  \item{Taktfrequenz 1,2 GHz}
  \item{Anzahl Prozessorkerne: 4}
  \item{Arbeitsspeicher: 1 GByte}
  \item{eMMC Flash Speicher: 4 GByte}
  \item{Betriebssystem: Angepasstes Raspbian mit RT-Patch}
  \item{RTC mit 24h Pufferung über wartungsfreien Kondensator}
  \item{Treiber / API: Treiber schreibt zyklisch Prozessdaten in ein Prozessabbild, Zugriff auf Prozessabbild über Linux-Filesystem als API zu Fremdsoftware.}
  \item{Kommunikationsanschlüsse: 2 x USB 2.0 A (je 500 mA belastbar), 1 x Micro-USB, HDMI, Ethernet (RJ45) 10/100 Mbit/s}
  \item{Stromversorgung: min. 10,7 V, max. 28,8 V, maximal 10 Watt}
  \item{Zulässige Umgebungstemperatur: -40 bis +55 C}
  \item{Gehäuseabmessungen: (HxBxL) 96 mm x 22,5 mm x 110,5 mm (ohne gesteckte Stecker)}
  \item{ESD Schutz: 4 kV / 8 kV gemäß EN61131-2 und IEC 61000-6-2}
  \item{Surge / Burst Prüfungen: gemäß EN61131-2 und IEC 61000-6-2 eingekoppelt auf Versorgungsspannung, Ethernet und IO-Leitungen}
  \item{EMI Prüfungen: gemäß EN61131-2 und IEC 61000-6-2}
\end{itemize}

Kunbus bietet eine Auswahl an IO- und Gateway-Modulen zur Erweiterung des Revolution Pi an.
Gateways dienen der Kommunikation mit Systemen oder Komponenten der Automatisierungstechnik
über Protokolle wie PROFIBUS oder EtherCAT. IO-Module erlauben die Überwachung
und Steuerung von digitalen oder analogen Ein- und Ausgängen.

\subsubsection{Zugriff auf IO-Module%
        \label{sec:2-io}}
Der Zugriff auf die Ein- und Ausgänge der IO-Module erfolgt über ein Prozessabbild
und einen hierfür von Kunbus bereitgestellten Treiber, genannt piControl. Dieser
aktualisiert das Prozessabbild zyklisch. Die angestrebte Zykluszeit beträgt 5ms,
kann jedoch je nach Anzahl der angeschlossenen Module auch größer sein. Kunbus
garantiert bei drei IO-Modulen und zwei Gateway-Modulen eine Zykluszeit von 10 ms.
Jedes der IO-Module stellt ein eigenständiges eingebettetes System dar. Es verfügt
über einen Microcontroller, welcher die IOs bereitstellt und über einen RS485-Bus
mit dem Revolution Pi kommuniziert.
% https://revolution.kunbus.de/io-modul/

Lizenz: GPL
% https://github.com/RevolutionPi/piControl

\begin{lstlisting}[language={c},firstnumber={226},caption={Setzen der Scheduler-Priorität auf SCHED\_FIFO in revpi\_common.c\label{lst:2-sched_priority}}]
param.sched_priority = ktprio->prio;
ret = sched_setscheduler(child, SCHED_FIFO,
       &param);
\end{lstlisting}


\subsection{Echtzeit und Multithreading unter Linux -- preemptRT und posix%
     \label{sec:2-echtzeit}}


 Der Linux-Kernel verfügt über mehrere unterschiedliche Preemtion-Modelle:

\begin{itemize}
  \item No Forced Preemption (server):
  Ausgelegt auf maximal möglichen Durchsatz, lediglich Interrupts und
  System-Call-Returns bewirken Präemption.

  \item Voluntary Kernel Preemption (Desktop):
  Neben den implizit bevorrechtigten Interrupts und System-Call-Returns gibt es
  in diesem Modell weitere Abschnitte des Kernels in welchen Preämption explizit
  gestattet ist.

  \item Preemptible Kernel (Low-Latency Desktop):
  In diesem Modell ist der gesamte Kernel, mit Ausnahme sog.~kritischer Abschnitte
  präemptible. Nach jedem kritischen Abschnitt gibt es einen impliziten Präemptions-Punkt.

  \item Preemptible Kernel (Basic RT):
  Dieses Modell ist dem zuvor genannten sehr ähnlich, hier sind jedoch alle Interrupt-Handler
  als eigenständige Threads ausgeführt.

  \item Fully Preemptible Kernel (RT):
  Wie auch bei den beiden zuvor genannten Modellen ist hier der gesamte Kernel
  präemtible, die Anzahl und Dauer der nicht-präemtiblen kritischen Abschnitte
  ist auf ein notwendiges Minimum beschränkt. Alle Interrupt-Handler sind als
  eigenständige Threads ausgeführt, Spinlocks durch Sleeping-Spinlocks und Mutexe
  durch sog.~RT-Mutexe ersetzt.

\end{itemize}
\todo{Spinlocks und Mutexe sowie die RT-Varianten dieser erklären!}

Lediglich mit dem vollständig präemtiblen Kernel kann Echtzeit-Verhalten realisiert werden.

% https://wiki.linuxfoundation.org/realtime/documentation/technical_basics/preemption_models bzw kernel/Kconfig.preempt

\subsubsection{preemptRT%
        \label{sec:2-preemptRT}}
% https://wiki.linuxfoundation.org/realtime/documentation/technical_details/start
% https://wiki.linuxfoundation.org/realtime/documentation/technical_basics/start

Das dem PREEMPT RT Kernel zugrunde liegende Prinzip lässt sich in einer einfachen
Regel ausdrücken: Nur Code, welcher absolut nicht-präemtible sein darf, ist es
gestattet nicht-präemtible zu sein.
Das erklärte Ziel des PREEMPT\_RT Patches ist es folglich, die Menge des nicht-präemtiblen
Codes im Linux-Kernel auf das absolut notwendige Minimum zu reduzieren.

Dies wird durch Verwendung folgender Mechanismen erreicht:

\begin{itemize}
  \item Hochauflösende Timer
  \item Sleeping Spinlocks
  \item Threaded Interrupt Handlers
  \item rt\_mutex
  \item RCU
\end{itemize}


\subsubsection{posix%
        \label{sec:2-posix}}
Ist posix hier wirklich relevant? Debian bzw.~Raspbian sind weitgehend posix
kompatibel, aber wird es hier genutzt? -> JA, open62541 nutzt pthread.h
piControl nutzt kthread.h, und semaphore.h

\subsection{OPC-UA und open62541%
     \label{sec:2-opc}}

\subsubsection{OPC UA%
        \label{sec:2-opcua}}
Open Platform Communications (OPC) ist eine Familie von Standards zur herstellerunabhängigen
Kommunikation von Maschinen (M2M) in der Automatisierungstechnik. Die sog.~OPC Task Force, zu deren
Mitgliedern verschiedene große Firmen der Automatisierungsindustrie gehören, veröffentlichte
die OPC Specification Version 1.0 im August 1996.
Motiviert ist dieser offene Standard durch die Erkenntniss, dass die Anpassung der
zahlreichen Herstellerstandards an individuelle Infrastrukturen und Anlagen einen
großen Mehraufwand verursachen.
Die Wikipedia beschreibt das Anwendungsgebiet für OPC wie folgt:

\glqq{}OPC wird dort eingesetzt, wo Sensoren, Regler und Steuerungen verschiedener Hersteller
ein gemeinsames Netzwerk bilden. Ohne OPC benötigten zwei Geräte zum Datenaustausch
genaue Kenntnis über die Kommunikationsmöglichkeiten des Gegenübers. Erweiterungen
und Austausch gestalten sich entsprechend schwierig. Mit OPC genügt es, für jedes
Gerät genau einmal einen OPC-konformen Treiber zu schreiben. Idealerweise wird
dieser bereits vom Hersteller zur Verfügung gestellt. Ein OPC-Treiber lässt sich
ohne großen Anpassungsaufwand in beliebig große Steuer- und Überwachungssysteme
integrieren.

OPC unterteilt sich in verschiedene Unterstandards, die für den jeweiligen Anwendungsfall
unabhängig voneinander implementiert werden können. OPC lässt sich damit verwenden
für Echtzeitdaten (Überwachung), Datenarchivierung, Alarm-Meldungen und neuerdings
auch direkt zur Steuerung (Befehlsübermittlung).\grqq{}

OPC basiert in der ursprünglichen Spezifikation auf Microsofts DCOM-Spezifikation.
DCOM macht Funktionen und Objekte einer Anwendung anderen Anwendungen im Netzwerk
zugänglich. Der OPC-Standard definiert entsprechende DCOM-Objekte um mit anderen
OPC-Anwendungen Daten austauschen zu können. Die Verwendung von DCOM bindet Anwender
an Betriebssysteme von Microsoft. Die ursprüngliche OPC Spezifikation wird durch die
Entwicklung von OPC Unified Architecture (OPC UA) abgelöst.
OPC UA setzt auf einem eigenen Kommunikationionsstack auf, die Verwendung von DCOM
und damit die Bindung an Microsoft wurden aufgelöst.

Die OPC-UA-Architektur ist eine Service-orientierte Architektur (SOA), deren Struktur
aus mehreren Schichten besteht.

% Wikipedia
Das OPC-Informationsmodell ist nicht mehr nur eine Hierarchie aus Ordnern, Items
und Properties. Es ist ein sogenanntes Full-Mesh-Network aus Nodes, mit dem neben
den Nutzdaten eines Nodes auch Meta- und Diagnoseinformationen repräsentiert werden.
Ein Node ähnelt einem Objekt aus der objektorientierten Programmierung. Ein Node
kann Attribute besitzen, die gelesen werden können (Data Access (DA), Historical
Data Access (HDA)). Es ist möglich Methoden zu definieren und aufzurufen.
Eine Methode besitzt Aufrufargumente und Rückgabewerte. Sie wird durch ein Command
aufgerufen. Weiterhin werden Events unterstützt, die versendet werden können
(AE (Alarms \& Events), DA DataChange), um bestimmte Informationen zwischen Geräten
auszutauschen. Ein Event besitzt unter anderem einen Empfangszeitpunkt, eine Nachricht
und einen Schweregrad. Die o. g. Nodes werden sowohl für die Nutzdaten als auch
alle anderen Arten von Metadaten verwendet. Der damit modellierte OPC-Adressraum
beinhaltet nun auch ein Typmodell, mit dem sämtliche Datentypen spezifiziert werden.

% https://de.wikipedia.org/wiki/Open_Platform_Communications
% https://de.wikipedia.org/wiki/OPC_Unified_Architecture
% https://opcfoundation.org/developer-tools/specifications-unified-architecture
% Von Gerhard Gappmeier - ascolab GmbH, CC BY-SA 3.0, https://de.wikipedia.org/w/index.php?curid=1892069
\subsubsection{open62541%
        \label{sec:2-open62541}}
open62541 ist eine offene und freie Implementierung von OPC UA. Die in C geschriebene
Bibliothek stellt eine beständig zunehmende Anzahl der im OPC UA Standard definierten
Funktionen bereit. Sie kann sowohl zur Erstellung von OPC-Servern als auch -Clients
genutzt werden. Ergänzend zu der unter der Mozilla Public License v2.0 lizensierten
Bibliothek stellt das open62541 Projekt auch Beispielprogramme unter einer CC0 Lizenz
zur Verfügung.

Die Bibliothek eignet sich auch für die Entwicklung auf eingebetteten Systemen und
Microcontrollern. Je nach Umfang der gewünschten Funktionen und des OPC Informationsmodells
beträgt die Größe einer Server-Binary weniger als 100kb. %evtl. kürzen?

\todo{Nodes erklären! Evtl.~oben!}

Folgende Auswahl an Eigenschaften und Funktionen zeichnet die in dieser Arbeit verwendete
Version 0.3 von open62541 aus:
\begin{itemize}
  \item Kommunikationionsstack
  \begin{itemize}
      \item OPC UA Binär-Protokoll (HTTP oder SOAP werden gegenwärtig nicht unterstützt)
      \item Austauschbare Netzwerk-Schicht, welche die Verwendung eigener Netzwerk-APIs
      erlaubt.
      \item Verschlüsselte Kommunikationion
      \item Asynchrone Dienst-Anfragen im Client
  \end{itemize}
  \item Informationsmodell
  \begin{itemize}
    \item Unterstützung aller OPC UA Node-Typen, inkl.~Methoden
    \item Hinzufügen und Entfernen von Nodes und Referenzen zur Laufzeit.
    \item Vererbung und Instanziierung von Objekt- und Variablentypen
    \item Zugriffskontrolle auch für einzelne Nodes
  \end{itemize}
  \item Subscriptions
  \begin{itemize}
    \item Erlaubt die Überwachung (subscriptions / monitoreditems)
    \item Sehr geringer Ressourcenbedarf pro überwachtem Wert
  \end{itemize}
  \item Code-Generierung auf XML-Basis
  \begin{itemize}
    \item Erlaubt die Erstellung von Datentypen
    \item Erlaubt die Generierung des serverseitigen Informationsmodells
  \end{itemize}
\end{itemize}

% https://open62541.org/doc/0.3/


Mozilla Public License
CC0 Lizenz für Beispiele und Plugins

% https://open62541.org/doc/open62541-current.pdf
% https://open62541.org/

% % % Imports nur für Referenzenauflösung während des Schreibens! Vorm Kompilieren auskommentieren!
% \bibliography{0_hauptdatei}
% \input{1_einleitung}
% \input{2_grundlagen}
% \input{3_konzeption}
% \input{4_implementierung}
% \input{5_tests}
% \input{6_zusammenfassung}
% \input{anhang}
% % Ende Imports

\section{Systemkonzept%
  \label{sec:3-konzeption}}
Auf Basis der in Abschnitt \ref{sec:2-grundlagen} vorgestellten Möglichkeiten folgt nun die Ausarbeitung eines Konzepts.
In den folgenden Abschnitten soll näher auf zwei zentrale Aspekte eingegangen werden: Abschnitt~\ref{sec:3-anbindung} stellt Möglichkeiten zum Zugriff auf Variablen bzw.\,Werte im Prozessabbild des Revolution Pi vor; in Abschnitt~\ref{sec:3-integration} wird ein Konzept zur Bereitstellung dieser Variablen auf einem OPC-Server vorgestellt.

\subsection{Anbindung der IO an den OPC-Server%
     \label{sec:3-anbindung}}

Eine Webanwendung mit Bezeichnung PiCtory dient zur Konfiguration der I/O- und virtuellen Module des RevolutionPi. Die Konfiguration liegt im JSON-Format in der Datei \lstinline{/etc/revpi/config.rsc}. Der piControl-Treiber liest diese Datei beim Start. 
Der folgende Auszug aus der Manpage des piControl-Kernelmoduls beschreibt die von diesem zum Lesen und Schreiben einzelner Bits des Prozessabbildes bereitgestellten Funktionen~\citep[vgl.]{web-revpi-manpage}. Sie ist an dieser Stelle weitgehend ungekürzt zitiert, da sie die nutzbare Schnittstelle sehr kompakt beschreibt.

\begin{lstlisting}[breakindent=0pt, numbers=none, caption={Auszug aus der Revolution Pi Programmers Manual\label{lst:4-manpage}}]
KB_FIND_VARIABLE SPIVariable *argp
Find a variable in the process image by its name. A pointer to a structure of type SPIVariable must be passed as argument. [...]
The struct SPIVariable [...] is defined as 
typedef struct SPIVariableStr
{
    char strVarName[32]; // Variable name
    uint16_t i16uAddress; // Address of the byte in the process image
    uint8_t i8uBit; // 0-7 bit position, >= 8 whole byte
    uint16_t i16uLength; // length of the variable in bits.
    // Possible values are 1, 8, 16 and 32
} SPIVariable;

Set and get values of the process image
KB_GET_VALUE SPIValue *argp
[...]
KB_SET_VALUE SPIValue *argp
Write one bit or one byte to the process image [...].  This call is more efficient than the usual calls of seek and write because only one function call is necessary. If more than on application are writing bits in one output byte, this call is the only safe way to set a bit without overwriting the other bits because this call is doing a read-modify-write-cycle. 

The struct SPIValue used by this ioctl is defined as
typedef struct SPIValueStr
{
    uint16_t i16uAddress; // Address of the byte in the process image
    uint8_t i8uBit; // 0-7 bit position, >= 8 whole byte
    uint8_t i8uValue; // Value: 0/1 for bit access, whole byte otherwise
} SPIValue;
\end{lstlisting} 

Die oben beschriebenden Funtkionen \lstinline{KB_FIND_VARIABLE}, \lstinline{KB_GET_VALUE} und \lstinline{KB_SET_VALUE} ermöglichen einen einfachen und (lt.\,Manpage) effizienten Zugriff auf einzelne Bits des Prozessabbildes und damit auch auf die IO des RevolutionPi.
Der Zugriff des OPC-Servers auf das Prozessabbild soll daher mittels dieser Funktionen realisiert werden.
\lstinline{KB_FIND_VARIABLE} kann genutzt werden, um Adressen von Variablen im Prozessabbild mittels ihres Namens aufzulösen.
\lstinline{KB_GET_VALUE} und \lstinline{KB_SET_VALUE} ermöglichen den Zugriff auf die Werte dieser Variablen.


\subsection{Integration des OPC-Servers in das System%
     \label{sec:3-integration}}

open62541 bietet drei Möglichkeiten zum Abgleich von Variablen mit dem Prozessabbild~\citep[vgl.][Tutorials - Connecting a Variable with a Physical Process]{web-open62541}:
\begin{itemize}
    \item Manuelles oder zyklisches Aktualisieren
    \item Variable Value Callback
    \item Variable Datasource
\end{itemize}

Die zyklische Aktualisierung eines oder mehrerer Werte nimmt, abhängig von der Zykluszeit, viele Systemressourcen in Anspruch. Value Callbacks ermöglichen es, einen Variablenwert effizienter mit einer Ressource wie etwa einem Prozessabbild zu synchronisieren. An die Variable wird ein Callback angehängt, welches vor jedem Lesen und nach jedem Schreibvorgang ausgeführt wird.
Der Wert der Variablen wird weiterhin im Variablenknoten auf dem OPC-Server gespeichert, der Abgleich mit der verknüpften Ressource erfolgt durch die Callback-Methoden.

Sogenannte Datenquellen gehen noch einen Schritt weiter. Der Server leitet jede Lese- und Schreibanforderung direkt an eine Callback-Funktion weiter. Beim Lesen liefert der Rückruf eine Kopie des aktuellen Wertes. Die Datenquelle muss intern ein eigenes Speichermanagement implementieren.

Der Zugriff auf die Werte des Prozessabbildes erfolgt, wie in Abschnitt~\ref{sec:3-anbindung} beschrieben, über von piControl bereitgestellte Methoden. Um die durch open62541 gepflegte OPC-Datenstruktur und das durch piControl verwaltete Prozessabbild möglichst effektiv verknüpfen zu können, soll diese Interaktion mittels Datenquellen und den zugehörigen Callbacks implementiert werden.
% % % Imports nur für Referenzenauflösung während des Schreibens! Vorm Kompilieren auskommentieren!
% \bibliography{0_hauptdatei}
% \input{1_einleitung}
% \input{2_grundlagen}
% \input{3_konzeption}
% \input{4_implementierung}
% \input{5_tests}
% \input{6_zusammenfassung}
% \input{anhang}
% % Ende Imports

\section{Implementierung%
  \label{sec:4-implementierung}}
Das folgende Kapitel stellt in Auszügen die Implementierung des OPC-Servers sowie die Anbindung an die IO-Module
der SPS dar. Der Schwerpunkt liegt hierbei auf der Funktionsweise des piControl-Treibers und dessen Integration in das Projekt. Abschnitt~\ref{sec:4-picontrol} erklärt die zum Schreibens eines Bits verwendeten Funktionsaufrufe.
Zuvor soll jedoch in Abschnitt~\ref{sec:4-open62541} der Teil des OPC-Servers vorgestellt werden, welcher auf besagten Treiber zugreift. 

\subsection{Implementierung des OPC-Servers%
     \label{sec:4-open62541}}
Wie im vorangegangenen Abschnitt~\ref{sec:3-integration} begründet, soll die Verknüpfung zwischen dem Prozessabbild der SPS und den auf dem OPC-Server bereitgestellten Werten über sog.\,Datenquellen erfolgen. Hierzu ist zunächst eine Callback-Methode zu implementieren, welche bei einem Lese- oder Schreibzugriff auf eine Variable aufgerufen wird. Die Verknüpfung zwischen Callback-Methode und Variable muss manuell erfolgen.

\begin{lstlisting}[language={c},firstnumber=237,caption={Auszug der Methode \lstinline{linkDataSourceVariable} in \lstinline{variables.c}\label{lst:4-linkDataSourceVariable}}]
extern UA_StatusCode
 linkDataSourceVariable(UA_Server *server, UA_NodeId nodeId) {
     bool readonly = false;
     UA_DataSource dataSourceVariable;
     UA_StatusCode rc; |>\setcounter{lstnumber}{254}<|

     dataSourceVariable.read = readDataSourceVariable;
     if (!readonly)
        dataSourceVariable.write = writeDataSourceVariable;
     else
        dataSourceVariable.write = writeReadonlyDataSourceVariable;

     return UA_Server_setVariableNode_dataSource(server, nodeId, dataSourceVariable);
 }
\end{lstlisting}

\begin{figure}[h]
    \centering
    \includegraphics[width=0.42\textwidth]{doc/img/OPC_RevPiDO.pdf}
    \caption{Auszug des verwendeten Nodesets, hier Digitalausgang 1 des Versuchsaufbaus
      \label{fig:opc-do}}
\end{figure}

Die in Listing~\ref{lst:4-linkDataSourceVariable} abgebildete Methode \lstinline{linkDataSourceVariable()} erzeugt ein Struct vom Typ \lstinline{UA_DataSource}. In diesem werden dem Lesen und Schreiben einer OPC-Variablen entsprechende Callback-Methoden zugewiesen. Die Verknüpfung einer OPC-Variable, genauer ihrer NodeId, mit der zuvor definierten Datenquelle erfolgt über die von open62541 bereitgestellte Methode \lstinline{UA_Server_setVariableNode_dataSource()}. Vor dem Lesen und nach dem Schreiben dieser Variable werden von nun an die entsprechenden Callbacks aufgerufen.
     
\begin{lstlisting}[language={c},firstnumber=168,caption={Auszug des Callbacks \lstinline{writeDataSourceVariable} in \lstinline{variables.c}\label{lst:4-writeDataSourceVariable}}]  
extern UA_StatusCode
 writeDataSourceVariable(UA_Server *server,
            const UA_NodeId *sessionId, void *sessionContext,
            const UA_NodeId *nodeId, void *nodeContext,
            const UA_NumericRange *range, const UA_DataValue *dataValue) {

    UA_StatusCode retval  = UA_STATUSCODE_GOOD;
    UA_NodeId *nameNodeId = UA_malloc(sizeof(UA_NodeId));
    UA_QualifiedName nameQN = UA_QUALIFIEDNAME(1, "Name");
    UA_Variant nameVar;
    UA_Boolean bit;

    retval |= findSiblingByBrowsename(server, nodeId, &nameQN, nameNodeId);
    retval |= UA_Server_readValue(server, *nameNodeId, &nameVar);
    retval |= UA_Boolean_copy(dataValue->value.data, &bit);

    |>\tikzmarkin[set border color=martinired]{writeIO}<|PI_writeSingleIO(String_fromUA_String(nameVar.data), &bit, false);                                                 |>\tikzmarkend{writeIO}<|

    free(nameNodeId);
    return retval;
 }
\end{lstlisting}

Listing~\ref{lst:4-writeDataSourceVariable} zeigt die Callback-Methode, welche nach dem Schreiben einer Variablen auf dem OPC-Server aufgerufen wird.
Dieser Methode wird neben der NodeId der mit ihr verknüpften Variablen auch der Wert dieser in Form eines Zeigers auf ein Struct vom Typ \lstinline{UA_DataValue} übergeben.

Die Gestaltung des hier verwendeten Nodesets sieht vor, dass in einer OPC-Variablen \lstinline{"Name"} der Bezeichner des zu schreibenden Digitalausgangs hinterlegt ist, siehe Abbildung~\ref{fig:opc-do}. Dies erlaubt eine Rekonfiguration der Ein- und Ausgänge der SPS ohne Änderungen im Programmcode des OPC-Servers vornehmen zu müssen.
Es ist daher erforderlich, nach jedem Schreiben einer mit einem Digitalausgang verknüpften Variablen, hier \lstinline{"Value"}, dessen Bezeichner \lstinline{"Name"} abzufragen. 
Dies geschieht in den Zeilen 180 und 181.
Anschließend wird dieser Bezeichner sowie der zu schreibende Wert der Methode \lstinline{PI_writeSingleIO()} übergeben, welche wiederum die Interaktion mit piControl übernimmt (vgl. Abschnitt \ref{sec:4-picontrol}).
 
\subsection{Integration von piControl%
     \label{sec:4-picontrol}}
In Abschnitt~\ref{sec:2-io} wurde die Anbindung der IO-Module des Revolution Pi sowie die Funktionsweise von piControl aus Anwendersicht beschrieben. Die verfügbare Literatur beschränkt sich auch auf lediglich diese Sicht; eine weiterführende Dokumentation für Entwickler gibt es, neben der in Abschnitt~\ref{sec:3-anbindung} vorgestellten Manpage, nicht. 
In diesem Abschnitt soll daher der Quellcode von piControl sowie dessen Verwendung im Projekt genauer betrachtet werden.
Hierzu wird exemplarisch die in Abschnitt~\ref{sec:4-open62541} eingeführte Methode \lstinline{PI_writeSingleIO()} untersucht.
Diese Methode ermöglicht das Setzen eines einzelnen Bits im Prozessabbild der SPS, und damit das Schalten eines digitalen Ausgangs auf einem IO-Modul.
Die äquivalente Methode \lstinline{int piControlGetBitValue(SPIValue *pSpiValue)} zum Lesen eines Bits bzw. Eingangs funktioniert analog und soll daher an dieser Stelle nicht dediziert erörtert werden.

\begin{lstlisting}[language={c},firstnumber=97,
                   caption={Setzen eines phsikalischen, digitalen Ausgangs in \lstinline{revpi.c}
                   \label{lst:4-PI_writeSingleIO}}]
extern void PI_writeSingleIO(char *pszVariableName, bool *bit, bool verbose)
{
	int rc;
	SPIVariable sPiVariable;
	SPIValue sPIValue;

	strncpy(sPiVariable.strVarName, pszVariableName, sizeof(sPiVariable.strVarName));
	rc = piControlGetVariableInfo(&sPiVariable);
	if (rc < 0) {
		printf("Cannot find variable '%s'\n", pszVariableName);
		return;
	}

		sPIValue.i16uAddress = sPiVariable.i16uAddress;
		sPIValue.i8uBit = sPiVariable.i8uBit;
		sPIValue.i8uValue = *bit;
		rc = |>\tikzmarkin[set border color=martinired]{setBitValue}<|piControlSetBitValue(&sPIValue)|>\tikzmarkend{setBitValue}<|;
		if (rc < 0)
			printf("Set bit error %s\n", getWriteError(rc));
		else if (verbose)
			printf("Set bit %d on byte at offset %d. Value %d\n", sPIValue.i8uBit, sPIValue.i16uAddress,
			       sPIValue.i8uValue);
}
\end{lstlisting}

Der Programmcode in Listing~\ref{lst:4-PI_writeSingleIO} ist Teil des implementierten OPC-Servers. In diesem wird auf zwei Funktionen des piControl-Treibers zugegriffen. 
Beiden Methoden wird als Argument ein Zeiger auf ein Struct vom Typ \lstinline{SPIValue} übergeben. Der im Struct abgelegte Name wird mittels \lstinline{piControlGetVariableInfo(&sPIValue)} zu einer Adresse im Prozessabbild aufgelöst. Diese wird in \lstinline{sPIValue.i16uAdress} gespeichert. Der Wert der Variablen wird anschließend mittels \lstinline{piControlSetBitValue(&sPIValue)} an dieser Adresse in das Prozessabbild geschrieben.

\begin{lstlisting}[language={c},firstnumber=309,caption={Methode \lstinline{piControlSetBitValue} in \lstinline{piControlIf.c}\label{lst:4-piControlSetBitValue}}]
int |>\tikzmarkin[set border color=martiniblue]{setBitValueFcn}<|piControlSetBitValue(SPIValue *pSpiValue)|>\tikzmarkend{setBitValueFcn}<|
{
    piControlOpen();

    if (PiControlHandle_g < 0)
	    return -ENODEV;

    pSpiValue->i16uAddress += pSpiValue->i8uBit / 8;
    pSpiValue->i8uBit %= 8;

    if (|>\tikzmarkin[set border color=martinired]{ioctl}<|ioctl(PiControlHandle_g, KB_SET_VALUE, pSpiValue)|>\tikzmarkend{ioctl}<| < 0)
	    return errno;

    return 0;
}
\end{lstlisting}

Die in Listing~\ref{lst:4-piControlSetBitValue} dargestellte Methode \lstinline{piControlSetBitValue} ist lediglich eine Hüllfunktion (häufig auch als Wrapper-Funktion bezeichnet) für einen Aufruf des \lstinline{ioctl} Kernel-Moduls.
Folgende Parameter werden übergeben:
\lstinline{PiControlHandle_g} ist die Referenz auf die Geräte-Datei des piControl-Treibers. \lstinline{KB_SET_VALUE} ist das ioctl-Kommando zum Schreiben eines Bits in das Prozessabbild. Der Zeiger \lstinline{pSpiValue} verweist auf ein Struct des bereits vorgestellten Typs \lstinline{SPIValue}.

\begin{lstlisting}[language={c},firstnumber=80,caption={Methode \lstinline{piControlOpen} in \lstinline{piControlIf.c}\label{lst:4-piControlOpen}}]
void piControlOpen(void)
{
    /* open handle if needed */
    if (PiControlHandle_g < 0)
    {
	    |>\tikzmarkin[set border color=martiniblue]{PiControlHandle}<|PiControlHandle_g = open(PICONTROL_DEVICE, O_RDWR)|>\tikzmarkend{PiControlHandle}<|;
    }
}
\end{lstlisting}

Die in Listing~\ref{lst:4-piControlOpen} dargestellte Methode öffnet, sofern nicht bereits geschehen, die Geräte-Datei. Das Macro \lstinline{PICONTROL_DEVICE} verweist hierbei auf \lstinline{/dev/piControl0}.

\begin{lstlisting}[language={c},firstnumber=721,caption={Methode \lstinline{piControlIoctl} in \lstinline{piControlMain.c}\label{lst:4-piControlIoctl}}]
static long |>\tikzmarkin[set border color=martiniblue, below offset=0.9em]{piControlIoctl}<|piControlIoctl(struct file *file, unsigned int prg_nr, 
                           unsigned long usr_addr)                                      |>\tikzmarkend{piControlIoctl}<|
{
  int status = -EFAULT;
  tpiControlInst *priv;
  int timeout = 10000;	// ms

  if (prg_nr != KB_CONFIG_SEND && prg_nr != KB_CONFIG_START && !isRunning()) {
  	return -EAGAIN;
  }

  priv = (tpiControlInst *) file->private_data;

  if (prg_nr != KB_GET_LAST_MESSAGE) {
  	// clear old message
  	priv->pcErrorMessage[0] = 0;
  }

  switch (prg_nr) {|>\setcounter{lstnumber}{864}<|

    case |>\tikzmarkin[set border color=martiniblue]{KB_SET_VALUE}<|KB_SET_VALUE:|>\tikzmarkend{KB_SET_VALUE}<|
  		{
  			SPIValue *pValue = (SPIValue *) usr_addr;

  			if (!isRunning())
  				return -EFAULT;

  			if (pValue->i16uAddress >= KB_PI_LEN) {
  				status = -EFAULT;
  			} else {
  				INT8U i8uValue_l;
  				my_rt_mutex_lock(&piDev_g.lockPI);
  				i8uValue_l = piDev_g.ai8uPI[pValue->i16uAddress];

  				if (pValue->i8uBit >= 8) {
  					i8uValue_l = pValue->i8uValue;
  				} else {
  					if (pValue->i8uValue)
  						i8uValue_l |= (1 << pValue->i8uBit);
  					else
  						i8uValue_l &= ~(1 << pValue->i8uBit);
  				}

  				|>\tikzmarkin[set border color=martinired]{i8uValue}<|piDev_g.ai8uPI[pValue->i16uAddress] = i8uValue_l;|>\tikzmarkend{i8uValue}<|
  				rt_mutex_unlock(&piDev_g.lockPI);

  #ifdef VERBOSE
  				pr_info("piControlIoctl Addr=%u, bit=%u: %02x %02x\n", pValue->i16uAddress, pValue->i8uBit, pValue->i8uValue, i8uValue_l);
  #endif

  				status = 0;
  			}
  		}
  		break; |>\setcounter{lstnumber}{1314}<|

    default:
      pr_err("Invalid Ioctl");
      return (-EINVAL);
      break;

    }

    return status;
  }
\end{lstlisting}

Listing~\ref{lst:4-piControlIoctl} zeigt in Auszügen die ioctl-Methode des piControl Kernel-Treibers. Diese bekommt folgende Argumente übergeben: \lstinline{struct file *file} enthält den Verweis auf die Geräte-Datei, hier \lstinline{/dev/piControl0}. Der Wert von \lstinline{unsigned int prg_nr} beschreibt die Anfrage an den Treiber, in diesem Fall \lstinline{KB_SET_VALUE}. Das Argument \lstinline{unsigned long usr_addr} enthält einen typ-agnostischen Pointer. Dieser verweist auf einen Speicherbereich, in welchem die zur Bearbeitung der Anfrage notwendigen Daten abgelegt sind. Hier können auch vom Treiber empfangene Daten dem Anwendungsprogramm bereitgestellt werden. 

Die switch-case-Anweisung führt die über das Argument \lstinline{prg_nr} spezifizierte Aktion aus. Hier betrachten wir \lstinline{KB_SET_VALUE}:
Zunächst wird in Zeile 868 der übergebene Zeiger \lstinline{usr_addr} mittels explizitem Typecast zu einem Zeiger des Typs \lstinline{SPIValue *} konvertiert. Da dieser auf Daten im Userspace verweist, ist beim Zugriff durch den Kernel-Treiber besondere Vorsicht geboten.
In Zeile 877 wird mittels Mutex das Prozessabbild \lstinline{piDev_g} für den Zugriff durch andere Threads oder Prozesse gesperrt.
\lstinline{my_rt_mutex_lock} verweist hierbei auf die Funktion \lstinline{rt_mutex_lock} aus \lstinline{linux/sched.h}\footnote{Offenbar wurde hier auch eine alternative Implementierung vorgesehen, siehe revpi\_common.h}

In Zeile 889 wird das Byte \lstinline{i8uValue_l}, welches den zu schreibenden Wert enthält in das Prozessabbild übertragen. Anschließend wird die Mutex auf \lstinline{piDev_g} wieder entsperrt.
\newpage

\begin{lstlisting}[language={c},firstnumber=62,caption={Auszug des Struct \lstinline{spiControlDev} in \lstinline{piControlMain.h}\label{lst:4-spiControlDev}}]
|>\tikzmarkin[set border color=martiniblue]{spiControlDev}<|typedef struct spiControlDev|>\tikzmarkend{spiControlDev}<| {
	// device driver stuff
	int init_step;
	enum revpi_machine machine_type;
	void *machine;
	struct cdev cdev;	// Char device structure
	struct device *dev;
	struct thermal_zone_device *thermal_zone;

	|>\tikzmarkin[set border color=martiniblue]{processImage}<|// process image stuff
	INT8U ai8uPI[KB_PI_LEN];
	INT8U ai8uPIDefault|>\tikzmarkin[set border color=martinired]{KB_PI_LEN_0}<|[KB_PI_LEN]|>\tikzmarkend{KB_PI_LEN_0}<|;
	struct rt_mutex lockPI;        |>\tikzmarkend{processImage}<|
	bool stopIO;
	piDevices *devs; |>\setcounter{lstnumber}{94}<|
} tpiControlDev;
\end{lstlisting}

Das Prozessabbild ist als Byte-Array der Länge \lstinline{KB_PI_LEN} in Listing~\ref{lst:4-spiControlDev} definiert. Konfigurationsparameter wie \lstinline{KB_PI_LEN} oder die Zykluszeit für den Datenaustausch zwischen SPS und IO-Modulen sind im folgenden Listing~\ref{lst:4-process} definiert.

\begin{lstlisting}[language={c},firstnumber=119,caption={Konfigurationsparameter des Prozessabbildes in project.h\label{lst:4-process}}]
#define INTERVAL_PI_GATE (5*1000*1000)  // 5 ms piGateCommunication |>\setcounter{lstnumber}{128}<|

#define INTERVAL_IO_COM (5*1000*1000)  // 5 ms piIoComm |>\setcounter{lstnumber}{132}<|

#define KB_PD_LEN       512
|>\tikzmarkin[set border color=martiniblue]{KB_PI_LEN_1}<|#define KB_PI_LEN       4096|>\tikzmarkend{KB_PI_LEN_1}<|
\end{lstlisting}

Das zu setzende Bit wurde zu diesem Zeitpunkt erfolgreich in das Prozessabbild der SPS geschrieben.
Es stellt sich die Frage, wie dieses nun an das IO-Modul kommuniziert wird.
Die Kommunikation mit allen angebundenen Modulen ist ebenfalls Aufgabe des piControl-Treibers.

\begin{lstlisting}[language={c},firstnumber=256,caption={Auszug der Methode \lstinline{piIoThread} in \lstinline{revpi_core.c}\label{lst:4-piIoThread}}]
static int piIoThread(void *data)
{
	//TODO int value = 0;
	ktime_t time;
	ktime_t now;
	s64 tDiff;

	hrtimer_init(&piCore_g.ioTimer, CLOCK_MONOTONIC, HRTIMER_MODE_ABS);
	piCore_g.ioTimer.function = piIoTimer;

	pr_info("piIO thread started\n");

	now = hrtimer_cb_get_time(&piCore_g.ioTimer);

	PiBridgeMaster_Reset();

	while (!kthread_should_stop()) {
		if (|>\tikzmarkin[set border color=martinired]{PiBridgeMaster}<|PiBridgeMaster_Run()|>\tikzmarkend{PiBridgeMaster}<| < 0)
			break;
	}

	RevPiDevice_finish();

	pr_info("piIO exit\n");
	return 0;
}
\end{lstlisting}

Der Kernel-Thread \lstinline{piIoThread} ist verantwortlich für den zyklischen Datenaustausch mit den IO-Modulen. In diesem wird fortlaufend die Methode \lstinline{PiBridgeMaster_Run()} aufgerufen, siehe Listing~\ref{lst:4-piIoThread}.

\begin{lstlisting}[language={c},firstnumber=262,caption={Auszug der Methode \lstinline{PiBridgeMaster_Run(void)} in \lstinline{RevPiDevice.c}\label{lst:4-PiBridgeMaster_Run}}]
int PiBridgeMaster_Run(void)
{
	static kbUT_Timer tTimeoutTimer_s;
	static kbUT_Timer tConfigTimeoutTimer_s;
	static int error_cnt;
	static INT8U last_led;
	static unsigned long last_update;
	int ret = 0;
	int i;

	my_rt_mutex_lock(&piCore_g.lockBridgeState);
	if (piCore_g.eBridgeState != piBridgeStop) {
		switch (eRunStatus_s) { |>\setcounter{lstnumber}{514}<|
		    case enPiBridgeMasterStatus_EndOfConfig:|>\setcounter{lstnumber}{621}<|
		    if (|>\tikzmarkin[set border color=martinired]{RevPiDevice}<|RevPiDevice_run()|>\tikzmarkend{RevPiDevice}<|) {
				// an error occured, check error limits |>\setcounter{lstnumber}{641}<|
			} else {
				ret = 1;
			}
			piCore_g.image.drv.i16uRS485ErrorCnt = RevPiDevice_getErrCnt();
			break;
\end{lstlisting}

Die in Listing~\ref{lst:4-PiBridgeMaster_Run} dargestellte Methode ist eine sog. State-Machine. Ist die Konfiguration der IO-Module erfolgreich abgeschlossen, so führt sie bei Aufruf lediglich die Methode \lstinline{RevPiDevice_run()} aus.

\begin{lstlisting}[language={c},firstnumber=140,caption={Auszug der Methode \lstinline{RevPiDevice_run(void)} in \lstinline{RevPiDevice.c}\label{lst:4-RevPiDevice_run}}]
int RevPiDevice_run(void)
{
	INT8U i8uDevice = 0;
	INT32U r;
	int retval = 0;

	RevPiDevices_s.i16uErrorCnt = 0;

	for (i8uDevice = 0; i8uDevice < RevPiDevice_getDevCnt(); i8uDevice++) {
		if (RevPiDevice_getDev(i8uDevice)->i8uActive) {
			switch (RevPiDevice_getDev(i8uDevice)->sId.i16uModulType) {
			case KUNBUS_FW_DESCR_TYP_PI_DIO_14:
			case KUNBUS_FW_DESCR_TYP_PI_DI_16:
			case KUNBUS_FW_DESCR_TYP_PI_DO_16:
				r = |>\tikzmarkin[set border color=martinired]{sendCyclicTelegram}<|piDIOComm_sendCyclicTelegram(i8uDevice)|>\tikzmarkend{sendCyclicTelegram}\setcounter{lstnumber}{166} <|;

				break; |>\setcounter{lstnumber}{216}<|
			}
		}
	} |>\setcounter{lstnumber}{227}<|
	return retval;
}
\end{lstlisting}

Diese iteriert wie in Listing~\ref{lst:4-RevPiDevice_run} abgebildete durch alle gegenwärtig in der SPS konfigurierten Module. Ist das aktuelle Modul als aktiv markiert, so wird anhand eines sog. Firmware-Descriptors entschieden, welche Methode für die Ansteuerung des Moduls aufzurufen ist.

\begin{lstlisting}[language={c},firstnumber=161,caption={Auszug der Methode \lstinline{piDIOComm_sendCyclicTelegram} in \lstinline{piDIOComm.c}\label{lst:4-sendCyclicTelegram}}]
INT32U piDIOComm_sendCyclicTelegram(INT8U i8uDevice_p)
{
	INT32U i32uRv_l = 0;
	SIOGeneric sRequest_l;
	SIOGeneric sResponse_l;
	INT8U len_l, data_out[18], i, p, data_in[70];
	INT8U i8uAddress;
	int ret; |>\setcounter{lstnumber}{239}<|
	
    |>\tikzmarkin[set border color=martinired]{piIoComm}<|ret = piIoComm_send((INT8U *) & sRequest_l, IOPROTOCOL_HEADER_LENGTH + len_l + 1);  |>\tikzmarkend{piIoComm}\setcounter{lstnumber}{298}<|
}
\end{lstlisting}

Im Falle des hier verwendeten DO-Moduls wird die in Listing~\ref{lst:4-sendCyclicTelegram} abgebildete Methode \lstinline{piDIOComm_sendCyclicTelegram()} aufgerufen. Dieser wird ein Zeiger auf das zu schreibende Gerät übergeben. 
Zunächst wird das Prozessabbild mittels eines proprietären, jedoch im Quellcode offen nachvollziehbaren Protokolls in ein \lstinline{sRequest_l} genanntes Byte-Array umgewandelt. Dieser Schritt ist in Listing~\ref{lst:4-sendCyclicTelegram} nicht abgebildet. Anschließend wird \lstinline{piIoComm_send()} ein Zeiger auf die so generierte Schreib-Anfrage übergeben.

\begin{lstlisting}[language={c},firstnumber=220,caption={Auszug der Methode \lstinline{piIOComm_send} in \lstinline{piIOComm.c}\label{lst:4-piIOComm_send}}]
int piIoComm_send(INT8U * buf_p, INT16U i16uLen_p)
{
	ssize_t write_l = 0;
	INT16U i16uSent_l = 0;|>\setcounter{lstnumber}{249}<|

	while (i16uSent_l < i16uLen_p) {
		write_l = vfs_write(piIoComm_fd_m, buf_p + i16uSent_l, i16uLen_p - i16uSent_l, &piIoComm_fd_m->f_pos);
		if (write_l < 0) {
			pr_info_serial("write error %d\n", (int)write_l);
			return -1;
		} 
		i16uSent_l += write_l;|>\setcounter{lstnumber}{263}<|
	}
	clear();
	vfs_fsync(piIoComm_fd_m, 1);
	return 0;
}
\end{lstlisting}

Listing~\ref{lst:4-piIOComm_send} zeigt die Implementierung von \lstinline{piIoComm_send()}. Diese Methode ist für das Schreiben der oben generierten Anfrage auf die seriellen Schnittstelle verantwortlich. Realisiert wird dies mittels der Methode \lstinline{vfs_write()}. Diese ist in \lstinline{<linux/fs.h>} definiert. Sie ermöglicht das Schreiben einer Datei im Userspace aus dem Kernel heraus. Geschrieben wird hier die Datei mit dem Deskriptor \lstinline{piIoComm_fd_m}.
Da die Funktion \lstinline{vfs_write()} durch andere Kernel-Tasks unterbrochen werden kann, ist nicht gewährleistet, dass die gesamte Anfrage mit nur einem Aufruf geschrieben wird. Die oben abgebildete while-Schleife stellt das vollständige Senden der Anfrage sicher.

\begin{lstlisting}[language={c},firstnumber=157,caption={Auszug der Methode \lstinline{piIOComm_open_serial} in \lstinline{piIOComm.c}\label{lst:4-piIOComm_open_serial}}]
int piIoComm_open_serial(void)
{   |>\setcounter{lstnumber}{167}<|
	struct file *fd;	/* Filedeskriptor */
	struct termios newtio;	/* Schnittstellenoptionen */

	|>\tikzmarkin[set border color=martiniblue]{fd}<|/* Port oeffnen - read/write, kein "controlling tty", 
	    Status von DCD ignorieren */
	fd = filp_open(|>\tikzmarkin[set border color=martinired]{tty}<|REV_PI_TTY_DEVICE|>\tikzmarkend{tty}<|, O_RDWR | O_NOCTTY, 0); |>\setcounter{lstnumber}{208}<|
	
	piIoComm_fd_m = fd;                                                      |>\tikzmarkend{fd}\setcounter{lstnumber}{217}<|

	return 0;
}
\end{lstlisting}

Der zum Schreiben auf die serielle Schnittstelle verwendete Datei-Deskriptor wird von der in Listing~\ref{lst:4-piIOComm_open_serial} abgebildeten Methode \lstinline{piIoComm_open_serial()} generiert. 

\begin{lstlisting}[language={c},firstnumber=45,caption={Definition der seriellen Schnittstelle in \lstinline{piIOComm.h}\label{lst:4-REV_PI_TTY_DEVICE}}]
#define REV_PI_TTY_DEVICE	"/dev/ttyAMA0"
\end{lstlisting}

Das in Listing~\ref{lst:4-REV_PI_TTY_DEVICE} definierte Macro verweist auf eine der seriellen Schnittstellen des RaspberryPi.
Die Implementierung des zugehörigen Schnittstellentreibers soll hier nicht weiter untersucht werden. Somit ist an dieser Stelle die Kette vom Setzen einer Variablen auf dem OPC-Server bis hin zur Aktualisierung des Prozessabbilds der IO-Module geschlossen.

% \begin{lstlisting}[language={c},firstnumber={226},caption={Setzen der Scheduler-Priorität auf SCHED\_FIFO in 
% revpi\_common.c\label{lst:2-sched_priority}}]
% param.sched_priority = ktprio->prio;
% ret = sched_setscheduler(child, SCHED_FIFO, &param);
% \end{lstlisting}
% % % Imports nur für Referenzenauflösung während des Schreibens! Vorm Kompilieren auskommentieren!
% \bibliography{0_hauptdatei}
% \input{1_einleitung}
% \input{2_grundlagen}
% \input{3_konzeption}
% \input{4_implementierung}
% \input{5_tests}
% \input{6_zusammenfassung}
% % Ende Imports

\section{Test des OPC-Servers im Gesamtsystem%
  \label{sec:5-tests}}

% % % Imports nur für Referenzenauflösung während des schreibens! Vorm Kompilieren auskommentieren!
% \bibliography{0_hauptdatei}
% \input{1_einleitung}
% \input{2_grundlagen}
% \input{3_konzeption}
% \input{4_implementierung}
% \input{5_tests}
% \input{6_zusammenfassung}
% % Ende Imports

\section{Zusammenfassung und Ausblick%
  \label{sec:6-fazit}}
Der folgende Abschnitt~\ref{sec:6-zusammenfassung} fasst die gewonnenen Erkenntnisse und den Stand der Implementierung zusammen.
Den Abschluss dieser Arbeit bildet der Ausblick in Abschnitt~\ref{sec:6-ausblick}.

\subsection{Zusammenfassung%
     \label{sec:6-zusammenfassung}}

\subsection{Ausblick%
     \label{sec:6-ausblick}}

% \input{anhang}
% % Ende Imports

\section{Systemkonzept%
  \label{sec:3-konzeption}}
Auf Basis der in Abschnitt \ref{sec:2-grundlagen} vorgestellten Möglichkeiten folgt nun die Ausarbeitung eines Konzepts.
In den folgenden Abschnitten soll näher auf zwei zentrale Aspekte eingegangen werden: Abschnitt~\ref{sec:3-anbindung} stellt Möglichkeiten zum Zugriff auf Variablen bzw.\,Werte im Prozessabbild des Revolution Pi vor; in Abschnitt~\ref{sec:3-integration} wird ein Konzept zur Bereitstellung dieser Variablen auf einem OPC-Server vorgestellt.

\subsection{Anbindung der IO an den OPC-Server%
     \label{sec:3-anbindung}}

Eine Webanwendung mit Bezeichnung PiCtory dient zur Konfiguration der I/O- und virtuellen Module des RevolutionPi. Die Konfiguration liegt im JSON-Format in der Datei \lstinline{/etc/revpi/config.rsc}. Der piControl-Treiber liest diese Datei beim Start. 
Der folgende Auszug aus der Manpage des piControl-Kernelmoduls beschreibt die von diesem zum Lesen und Schreiben einzelner Bits des Prozessabbildes bereitgestellten Funktionen~\citep[vgl.]{web-revpi-manpage}. Sie ist an dieser Stelle weitgehend ungekürzt zitiert, da sie die nutzbare Schnittstelle sehr kompakt beschreibt.

\begin{lstlisting}[breakindent=0pt, numbers=none, caption={Auszug aus der Revolution Pi Programmers Manual\label{lst:4-manpage}}]
KB_FIND_VARIABLE SPIVariable *argp
Find a variable in the process image by its name. A pointer to a structure of type SPIVariable must be passed as argument. [...]
The struct SPIVariable [...] is defined as 
typedef struct SPIVariableStr
{
    char strVarName[32]; // Variable name
    uint16_t i16uAddress; // Address of the byte in the process image
    uint8_t i8uBit; // 0-7 bit position, >= 8 whole byte
    uint16_t i16uLength; // length of the variable in bits.
    // Possible values are 1, 8, 16 and 32
} SPIVariable;

Set and get values of the process image
KB_GET_VALUE SPIValue *argp
[...]
KB_SET_VALUE SPIValue *argp
Write one bit or one byte to the process image [...].  This call is more efficient than the usual calls of seek and write because only one function call is necessary. If more than on application are writing bits in one output byte, this call is the only safe way to set a bit without overwriting the other bits because this call is doing a read-modify-write-cycle. 

The struct SPIValue used by this ioctl is defined as
typedef struct SPIValueStr
{
    uint16_t i16uAddress; // Address of the byte in the process image
    uint8_t i8uBit; // 0-7 bit position, >= 8 whole byte
    uint8_t i8uValue; // Value: 0/1 for bit access, whole byte otherwise
} SPIValue;
\end{lstlisting} 

Die oben beschriebenden Funtkionen \lstinline{KB_FIND_VARIABLE}, \lstinline{KB_GET_VALUE} und \lstinline{KB_SET_VALUE} ermöglichen einen einfachen und (lt.\,Manpage) effizienten Zugriff auf einzelne Bits des Prozessabbildes und damit auch auf die IO des RevolutionPi.
Der Zugriff des OPC-Servers auf das Prozessabbild soll daher mittels dieser Funktionen realisiert werden.
\lstinline{KB_FIND_VARIABLE} kann genutzt werden, um Adressen von Variablen im Prozessabbild mittels ihres Namens aufzulösen.
\lstinline{KB_GET_VALUE} und \lstinline{KB_SET_VALUE} ermöglichen den Zugriff auf die Werte dieser Variablen.


\subsection{Integration des OPC-Servers in das System%
     \label{sec:3-integration}}

open62541 bietet drei Möglichkeiten zum Abgleich von Variablen mit dem Prozessabbild~\citep[vgl.][Tutorials - Connecting a Variable with a Physical Process]{web-open62541}:
\begin{itemize}
    \item Manuelles oder zyklisches Aktualisieren
    \item Variable Value Callback
    \item Variable Datasource
\end{itemize}

Die zyklische Aktualisierung eines oder mehrerer Werte nimmt, abhängig von der Zykluszeit, viele Systemressourcen in Anspruch. Value Callbacks ermöglichen es, einen Variablenwert effizienter mit einer Ressource wie etwa einem Prozessabbild zu synchronisieren. An die Variable wird ein Callback angehängt, welches vor jedem Lesen und nach jedem Schreibvorgang ausgeführt wird.
Der Wert der Variablen wird weiterhin im Variablenknoten auf dem OPC-Server gespeichert, der Abgleich mit der verknüpften Ressource erfolgt durch die Callback-Methoden.

Sogenannte Datenquellen gehen noch einen Schritt weiter. Der Server leitet jede Lese- und Schreibanforderung direkt an eine Callback-Funktion weiter. Beim Lesen liefert der Rückruf eine Kopie des aktuellen Wertes. Die Datenquelle muss intern ein eigenes Speichermanagement implementieren.

Der Zugriff auf die Werte des Prozessabbildes erfolgt, wie in Abschnitt~\ref{sec:3-anbindung} beschrieben, über von piControl bereitgestellte Methoden. Um die durch open62541 gepflegte OPC-Datenstruktur und das durch piControl verwaltete Prozessabbild möglichst effektiv verknüpfen zu können, soll diese Interaktion mittels Datenquellen und den zugehörigen Callbacks implementiert werden.
%% % Imports nur für Referenzenauflösung während des Schreibens! Vorm Kompilieren auskommentieren!
% \bibliography{0_hauptdatei}
% % Mit \section{...} eröffnen wir einen neuen Abschnitt.
% Der Befehl setzt nicht nur den Text in einer größeren,
% fetten Schrift, sondern sorgt außerdem dafür, daß er im
% Inhaltsverzeichnis erscheint.
%
% Mit \label{...} erzeugen wir einen Bezeichner, mit dessen Hilfe
% wir später auf die Nummer des Abschnitts verweisen können (nämlich
% mit~\ref{...}).
%
% Das Kommentarzeichen hinter „Übersicht“ dient dazu, ein
% Leerzeichen zwischen „Übersicht“ und dem \label-Befehl
% zu vermeiden, das andernfalls sichtbar würde – z.B. im
% Inhaltsverzeichnis.
%

% % Imports nur für Referenzenauflösung während des Schreibens! Vorm Kompilieren auskommentieren!
% \bibliography{0_hauptdatei}
% \input{1_einleitung}
%\input{2_grundlagen}
%\input{3_konzeption}
%\input{4_implementierung}
%\input{5_tests}
%\input{6_zusammenfassung}
% % Ende Imports

\section{Einleitung und Motivation%
  \label{sec:1-einleitung}}
Ziel dieses Projektes ist die Integration eines OPC-Servers mit einer auf Linux
basierenden speicherprogrammierbaren Steuerung (SPS). Angeschlossen an diese SPS
ist jeweils ein digitales Ein-/\,bzw.~Ausgabemodul. Die von diesen bereitgestellten
Ein-/\, bzw.~Ausgänge (IO) sollen in der Datenstruktur des OPC-Servers abgebildet
und über diesen für OPC-Clients les-/\,und schreibar sein. Weiterhin sollen einige
Funktionen zur Überwachung und Steuerung der an die SPS angeschlossenen Aktoren
und Sensoren direkt im OPC-Server implementiert werden.
Hiermit stellt dieses Projekt eine der Grundlagen für ein übergeordnetes Projekt,
die cloudbasierte Steuerung eines miniaturisierten Produktions-Systems, dar.

Der hier verwendete OPC-Server ist Teil des sog. open62541 Projekts. Er ist in C
geschrieben und implementiert bereits einen großen Teil der im OPC-UA-Standard
spezifizierten Funktionen.
Als SPS findet ein Revolution Pi 3 der Firma Kunbus Verwendung. Dieser integriert
ein sog. Compute Module der Raspberry Pi Foundation in ein industrietaugliches
Gehäuse und erlaubt die Erweiterung mittels IO- oder Gateway-Modulen. Über diese
erfolgt die Kommunikation mit weiteren Komponenten der Automatisierungstechnik.

Motiviert ist dieses Projekt durch die Beobachtung, dass die Verbreitung offener
Standards sowie freier Software auch in der Automatisierungstechnik zunimmt.
Linux ist ein freies Betriebssystem, OPC-UA ein offen zugänglicher, aktiv gepflegter
und weit verbreiteter Standard. Der Raspberry Pi findet sowohl bei Hobby-Anwendern als
auch in den Bereichen Forschung und Entwicklung sowie bei industriellen Anwendern
Verwendung. Dieses Projekt stellt somit eine für unterschiedliche Anwender interessante
Entwicklung dar.

Im Anschluss an diese einleitende Übersicht im Abschnitt~\ref{sec:1-einleitung} folgt
die Darstellung der wichtigsten Grundlagen in Abschnitt~\ref{sec:2-grundlagen}.
Aufbauend auf diesen Grundlagen folgt die konzeptuelle Ausarbeitung im Abschnitt~\ref{sec:3-konzeption}.
Die Umsetzung wird im Abschnitt~\ref{sec:4-implementierung} erläutert.
Die Leistungsfähigkeit der Implementierung wird in Abschnitt~\ref{sec:5-tests} untersucht.
Eine Zusammenfassung und ein Ausblick schließen die Arbeit in
Abschnitt~\ref{sec:6-fazit} ab. Eventuell noch benötigte Anhänge
finden sich in den Anhängen [...] bis [...].

% % % Imports nur für Referenzenauflösung während des Schreibens! Vorm Kompilieren auskommentieren!
% \bibliography{0_hauptdatei}
% \input{1_einleitung}
% \input{2_grundlagen}
% \input{3_konzeption}
% \input{4_implementierung}
% \input{5_tests}
% \input{6_zusammenfassung}
% % Ende Imports

\section{Grundlagen%
  \label{sec:2-grundlagen}}

\subsection{Speicherprogrammierbare-Steuerung und Linux -- Revolution Pi%
     \label{sec:2-sps}}

\subsubsection{Kunbus RevolutionPi%
        \label{sec:2-revpi}}
Der RevolutionPi 3 ist eine speicherprogrammierbare Steuerung (SPS) des Herstellers
Kunbus GmbH. Kern dieser SPS ist das von der Raspberry Pi Foundation entwickelte
und vertriebene Raspberry Pi Compute Module 3. Dieses integriert ein Broadcom BCM2837
System-on-Chip (SoC) mit vier 1,2GHz Prozessorkernen, 1GB RAM, 4GB eMMC Anwendungsspeicher
und sonstige Peripherie in ein Modul im DDR2-SODIMM Formfaktor. Diese Spezifikationen
sind weitgehend identisch zu denen des ausgesprochen populären Raspberry Pi 3.
Der Revolution Pi profitiert daher von dem gleichen großen Angebot an Software
und Unterstützung wie der Raspberry Pi, ergänzt dessen Hardware jedoch um eine 24V
Spannungsversorgung, die Möglichkeit der Erweiterung durch mehrere industrietaugliche
Ein-/ Ausgabemodule und Gateways sowie ein Gehäuse zur Montage auf einer DIN-Schiene.
\begin{itemize}
  \item{Prozessor: BCM2837}
  \item{Taktfrequenz 1,2 GHz}
  \item{Anzahl Prozessorkerne: 4}
  \item{Arbeitsspeicher: 1 GByte}
  \item{eMMC Flash Speicher: 4 GByte}
  \item{Betriebssystem: Angepasstes Raspbian mit RT-Patch}
  \item{RTC mit 24h Pufferung über wartungsfreien Kondensator}
  \item{Treiber / API: Treiber schreibt zyklisch Prozessdaten in ein Prozessabbild, Zugriff auf Prozessabbild über Linux-Filesystem als API zu Fremdsoftware.}
  \item{Kommunikationsanschlüsse: 2 x USB 2.0 A (je 500 mA belastbar), 1 x Micro-USB, HDMI, Ethernet (RJ45) 10/100 Mbit/s}
  \item{Stromversorgung: min. 10,7 V, max. 28,8 V, maximal 10 Watt}
  \item{Zulässige Umgebungstemperatur: -40 bis +55 C}
  \item{Gehäuseabmessungen: (HxBxL) 96 mm x 22,5 mm x 110,5 mm (ohne gesteckte Stecker)}
  \item{ESD Schutz: 4 kV / 8 kV gemäß EN61131-2 und IEC 61000-6-2}
  \item{Surge / Burst Prüfungen: gemäß EN61131-2 und IEC 61000-6-2 eingekoppelt auf Versorgungsspannung, Ethernet und IO-Leitungen}
  \item{EMI Prüfungen: gemäß EN61131-2 und IEC 61000-6-2}
\end{itemize}

Kunbus bietet eine Auswahl an IO- und Gateway-Modulen zur Erweiterung des Revolution Pi an.
Gateways dienen der Kommunikation mit Systemen oder Komponenten der Automatisierungstechnik
über Protokolle wie PROFIBUS oder EtherCAT. IO-Module erlauben die Überwachung
und Steuerung von digitalen oder analogen Ein- und Ausgängen.

\subsubsection{Zugriff auf IO-Module%
        \label{sec:2-io}}
Der Zugriff auf die Ein- und Ausgänge der IO-Module erfolgt über ein Prozessabbild
und einen hierfür von Kunbus bereitgestellten Treiber, genannt piControl. Dieser
aktualisiert das Prozessabbild zyklisch. Die angestrebte Zykluszeit beträgt 5ms,
kann jedoch je nach Anzahl der angeschlossenen Module auch größer sein. Kunbus
garantiert bei drei IO-Modulen und zwei Gateway-Modulen eine Zykluszeit von 10 ms.
Jedes der IO-Module stellt ein eigenständiges eingebettetes System dar. Es verfügt
über einen Microcontroller, welcher die IOs bereitstellt und über einen RS485-Bus
mit dem Revolution Pi kommuniziert.
% https://revolution.kunbus.de/io-modul/

Lizenz: GPL
% https://github.com/RevolutionPi/piControl

\begin{lstlisting}[language={c},firstnumber={226},caption={Setzen der Scheduler-Priorität auf SCHED\_FIFO in revpi\_common.c\label{lst:2-sched_priority}}]
param.sched_priority = ktprio->prio;
ret = sched_setscheduler(child, SCHED_FIFO,
       &param);
\end{lstlisting}


\subsection{Echtzeit und Multithreading unter Linux -- preemptRT und posix%
     \label{sec:2-echtzeit}}


 Der Linux-Kernel verfügt über mehrere unterschiedliche Preemtion-Modelle:

\begin{itemize}
  \item No Forced Preemption (server):
  Ausgelegt auf maximal möglichen Durchsatz, lediglich Interrupts und
  System-Call-Returns bewirken Präemption.

  \item Voluntary Kernel Preemption (Desktop):
  Neben den implizit bevorrechtigten Interrupts und System-Call-Returns gibt es
  in diesem Modell weitere Abschnitte des Kernels in welchen Preämption explizit
  gestattet ist.

  \item Preemptible Kernel (Low-Latency Desktop):
  In diesem Modell ist der gesamte Kernel, mit Ausnahme sog.~kritischer Abschnitte
  präemptible. Nach jedem kritischen Abschnitt gibt es einen impliziten Präemptions-Punkt.

  \item Preemptible Kernel (Basic RT):
  Dieses Modell ist dem zuvor genannten sehr ähnlich, hier sind jedoch alle Interrupt-Handler
  als eigenständige Threads ausgeführt.

  \item Fully Preemptible Kernel (RT):
  Wie auch bei den beiden zuvor genannten Modellen ist hier der gesamte Kernel
  präemtible, die Anzahl und Dauer der nicht-präemtiblen kritischen Abschnitte
  ist auf ein notwendiges Minimum beschränkt. Alle Interrupt-Handler sind als
  eigenständige Threads ausgeführt, Spinlocks durch Sleeping-Spinlocks und Mutexe
  durch sog.~RT-Mutexe ersetzt.

\end{itemize}
\todo{Spinlocks und Mutexe sowie die RT-Varianten dieser erklären!}

Lediglich mit dem vollständig präemtiblen Kernel kann Echtzeit-Verhalten realisiert werden.

% https://wiki.linuxfoundation.org/realtime/documentation/technical_basics/preemption_models bzw kernel/Kconfig.preempt

\subsubsection{preemptRT%
        \label{sec:2-preemptRT}}
% https://wiki.linuxfoundation.org/realtime/documentation/technical_details/start
% https://wiki.linuxfoundation.org/realtime/documentation/technical_basics/start

Das dem PREEMPT RT Kernel zugrunde liegende Prinzip lässt sich in einer einfachen
Regel ausdrücken: Nur Code, welcher absolut nicht-präemtible sein darf, ist es
gestattet nicht-präemtible zu sein.
Das erklärte Ziel des PREEMPT\_RT Patches ist es folglich, die Menge des nicht-präemtiblen
Codes im Linux-Kernel auf das absolut notwendige Minimum zu reduzieren.

Dies wird durch Verwendung folgender Mechanismen erreicht:

\begin{itemize}
  \item Hochauflösende Timer
  \item Sleeping Spinlocks
  \item Threaded Interrupt Handlers
  \item rt\_mutex
  \item RCU
\end{itemize}


\subsubsection{posix%
        \label{sec:2-posix}}
Ist posix hier wirklich relevant? Debian bzw.~Raspbian sind weitgehend posix
kompatibel, aber wird es hier genutzt? -> JA, open62541 nutzt pthread.h
piControl nutzt kthread.h, und semaphore.h

\subsection{OPC-UA und open62541%
     \label{sec:2-opc}}

\subsubsection{OPC UA%
        \label{sec:2-opcua}}
Open Platform Communications (OPC) ist eine Familie von Standards zur herstellerunabhängigen
Kommunikation von Maschinen (M2M) in der Automatisierungstechnik. Die sog.~OPC Task Force, zu deren
Mitgliedern verschiedene große Firmen der Automatisierungsindustrie gehören, veröffentlichte
die OPC Specification Version 1.0 im August 1996.
Motiviert ist dieser offene Standard durch die Erkenntniss, dass die Anpassung der
zahlreichen Herstellerstandards an individuelle Infrastrukturen und Anlagen einen
großen Mehraufwand verursachen.
Die Wikipedia beschreibt das Anwendungsgebiet für OPC wie folgt:

\glqq{}OPC wird dort eingesetzt, wo Sensoren, Regler und Steuerungen verschiedener Hersteller
ein gemeinsames Netzwerk bilden. Ohne OPC benötigten zwei Geräte zum Datenaustausch
genaue Kenntnis über die Kommunikationsmöglichkeiten des Gegenübers. Erweiterungen
und Austausch gestalten sich entsprechend schwierig. Mit OPC genügt es, für jedes
Gerät genau einmal einen OPC-konformen Treiber zu schreiben. Idealerweise wird
dieser bereits vom Hersteller zur Verfügung gestellt. Ein OPC-Treiber lässt sich
ohne großen Anpassungsaufwand in beliebig große Steuer- und Überwachungssysteme
integrieren.

OPC unterteilt sich in verschiedene Unterstandards, die für den jeweiligen Anwendungsfall
unabhängig voneinander implementiert werden können. OPC lässt sich damit verwenden
für Echtzeitdaten (Überwachung), Datenarchivierung, Alarm-Meldungen und neuerdings
auch direkt zur Steuerung (Befehlsübermittlung).\grqq{}

OPC basiert in der ursprünglichen Spezifikation auf Microsofts DCOM-Spezifikation.
DCOM macht Funktionen und Objekte einer Anwendung anderen Anwendungen im Netzwerk
zugänglich. Der OPC-Standard definiert entsprechende DCOM-Objekte um mit anderen
OPC-Anwendungen Daten austauschen zu können. Die Verwendung von DCOM bindet Anwender
an Betriebssysteme von Microsoft. Die ursprüngliche OPC Spezifikation wird durch die
Entwicklung von OPC Unified Architecture (OPC UA) abgelöst.
OPC UA setzt auf einem eigenen Kommunikationionsstack auf, die Verwendung von DCOM
und damit die Bindung an Microsoft wurden aufgelöst.

Die OPC-UA-Architektur ist eine Service-orientierte Architektur (SOA), deren Struktur
aus mehreren Schichten besteht.

% Wikipedia
Das OPC-Informationsmodell ist nicht mehr nur eine Hierarchie aus Ordnern, Items
und Properties. Es ist ein sogenanntes Full-Mesh-Network aus Nodes, mit dem neben
den Nutzdaten eines Nodes auch Meta- und Diagnoseinformationen repräsentiert werden.
Ein Node ähnelt einem Objekt aus der objektorientierten Programmierung. Ein Node
kann Attribute besitzen, die gelesen werden können (Data Access (DA), Historical
Data Access (HDA)). Es ist möglich Methoden zu definieren und aufzurufen.
Eine Methode besitzt Aufrufargumente und Rückgabewerte. Sie wird durch ein Command
aufgerufen. Weiterhin werden Events unterstützt, die versendet werden können
(AE (Alarms \& Events), DA DataChange), um bestimmte Informationen zwischen Geräten
auszutauschen. Ein Event besitzt unter anderem einen Empfangszeitpunkt, eine Nachricht
und einen Schweregrad. Die o. g. Nodes werden sowohl für die Nutzdaten als auch
alle anderen Arten von Metadaten verwendet. Der damit modellierte OPC-Adressraum
beinhaltet nun auch ein Typmodell, mit dem sämtliche Datentypen spezifiziert werden.

% https://de.wikipedia.org/wiki/Open_Platform_Communications
% https://de.wikipedia.org/wiki/OPC_Unified_Architecture
% https://opcfoundation.org/developer-tools/specifications-unified-architecture
% Von Gerhard Gappmeier - ascolab GmbH, CC BY-SA 3.0, https://de.wikipedia.org/w/index.php?curid=1892069
\subsubsection{open62541%
        \label{sec:2-open62541}}
open62541 ist eine offene und freie Implementierung von OPC UA. Die in C geschriebene
Bibliothek stellt eine beständig zunehmende Anzahl der im OPC UA Standard definierten
Funktionen bereit. Sie kann sowohl zur Erstellung von OPC-Servern als auch -Clients
genutzt werden. Ergänzend zu der unter der Mozilla Public License v2.0 lizensierten
Bibliothek stellt das open62541 Projekt auch Beispielprogramme unter einer CC0 Lizenz
zur Verfügung.

Die Bibliothek eignet sich auch für die Entwicklung auf eingebetteten Systemen und
Microcontrollern. Je nach Umfang der gewünschten Funktionen und des OPC Informationsmodells
beträgt die Größe einer Server-Binary weniger als 100kb. %evtl. kürzen?

\todo{Nodes erklären! Evtl.~oben!}

Folgende Auswahl an Eigenschaften und Funktionen zeichnet die in dieser Arbeit verwendete
Version 0.3 von open62541 aus:
\begin{itemize}
  \item Kommunikationionsstack
  \begin{itemize}
      \item OPC UA Binär-Protokoll (HTTP oder SOAP werden gegenwärtig nicht unterstützt)
      \item Austauschbare Netzwerk-Schicht, welche die Verwendung eigener Netzwerk-APIs
      erlaubt.
      \item Verschlüsselte Kommunikationion
      \item Asynchrone Dienst-Anfragen im Client
  \end{itemize}
  \item Informationsmodell
  \begin{itemize}
    \item Unterstützung aller OPC UA Node-Typen, inkl.~Methoden
    \item Hinzufügen und Entfernen von Nodes und Referenzen zur Laufzeit.
    \item Vererbung und Instanziierung von Objekt- und Variablentypen
    \item Zugriffskontrolle auch für einzelne Nodes
  \end{itemize}
  \item Subscriptions
  \begin{itemize}
    \item Erlaubt die Überwachung (subscriptions / monitoreditems)
    \item Sehr geringer Ressourcenbedarf pro überwachtem Wert
  \end{itemize}
  \item Code-Generierung auf XML-Basis
  \begin{itemize}
    \item Erlaubt die Erstellung von Datentypen
    \item Erlaubt die Generierung des serverseitigen Informationsmodells
  \end{itemize}
\end{itemize}

% https://open62541.org/doc/0.3/


Mozilla Public License
CC0 Lizenz für Beispiele und Plugins

% https://open62541.org/doc/open62541-current.pdf
% https://open62541.org/

% % % Imports nur für Referenzenauflösung während des Schreibens! Vorm Kompilieren auskommentieren!
% \bibliography{0_hauptdatei}
% \input{1_einleitung}
% \input{2_grundlagen}
% \input{3_konzeption}
% \input{4_implementierung}
% \input{5_tests}
% \input{6_zusammenfassung}
% \input{anhang}
% % Ende Imports

\section{Systemkonzept%
  \label{sec:3-konzeption}}
Auf Basis der in Abschnitt \ref{sec:2-grundlagen} vorgestellten Möglichkeiten folgt nun die Ausarbeitung eines Konzepts.
In den folgenden Abschnitten soll näher auf zwei zentrale Aspekte eingegangen werden: Abschnitt~\ref{sec:3-anbindung} stellt Möglichkeiten zum Zugriff auf Variablen bzw.\,Werte im Prozessabbild des Revolution Pi vor; in Abschnitt~\ref{sec:3-integration} wird ein Konzept zur Bereitstellung dieser Variablen auf einem OPC-Server vorgestellt.

\subsection{Anbindung der IO an den OPC-Server%
     \label{sec:3-anbindung}}

Eine Webanwendung mit Bezeichnung PiCtory dient zur Konfiguration der I/O- und virtuellen Module des RevolutionPi. Die Konfiguration liegt im JSON-Format in der Datei \lstinline{/etc/revpi/config.rsc}. Der piControl-Treiber liest diese Datei beim Start. 
Der folgende Auszug aus der Manpage des piControl-Kernelmoduls beschreibt die von diesem zum Lesen und Schreiben einzelner Bits des Prozessabbildes bereitgestellten Funktionen~\citep[vgl.]{web-revpi-manpage}. Sie ist an dieser Stelle weitgehend ungekürzt zitiert, da sie die nutzbare Schnittstelle sehr kompakt beschreibt.

\begin{lstlisting}[breakindent=0pt, numbers=none, caption={Auszug aus der Revolution Pi Programmers Manual\label{lst:4-manpage}}]
KB_FIND_VARIABLE SPIVariable *argp
Find a variable in the process image by its name. A pointer to a structure of type SPIVariable must be passed as argument. [...]
The struct SPIVariable [...] is defined as 
typedef struct SPIVariableStr
{
    char strVarName[32]; // Variable name
    uint16_t i16uAddress; // Address of the byte in the process image
    uint8_t i8uBit; // 0-7 bit position, >= 8 whole byte
    uint16_t i16uLength; // length of the variable in bits.
    // Possible values are 1, 8, 16 and 32
} SPIVariable;

Set and get values of the process image
KB_GET_VALUE SPIValue *argp
[...]
KB_SET_VALUE SPIValue *argp
Write one bit or one byte to the process image [...].  This call is more efficient than the usual calls of seek and write because only one function call is necessary. If more than on application are writing bits in one output byte, this call is the only safe way to set a bit without overwriting the other bits because this call is doing a read-modify-write-cycle. 

The struct SPIValue used by this ioctl is defined as
typedef struct SPIValueStr
{
    uint16_t i16uAddress; // Address of the byte in the process image
    uint8_t i8uBit; // 0-7 bit position, >= 8 whole byte
    uint8_t i8uValue; // Value: 0/1 for bit access, whole byte otherwise
} SPIValue;
\end{lstlisting} 

Die oben beschriebenden Funtkionen \lstinline{KB_FIND_VARIABLE}, \lstinline{KB_GET_VALUE} und \lstinline{KB_SET_VALUE} ermöglichen einen einfachen und (lt.\,Manpage) effizienten Zugriff auf einzelne Bits des Prozessabbildes und damit auch auf die IO des RevolutionPi.
Der Zugriff des OPC-Servers auf das Prozessabbild soll daher mittels dieser Funktionen realisiert werden.
\lstinline{KB_FIND_VARIABLE} kann genutzt werden, um Adressen von Variablen im Prozessabbild mittels ihres Namens aufzulösen.
\lstinline{KB_GET_VALUE} und \lstinline{KB_SET_VALUE} ermöglichen den Zugriff auf die Werte dieser Variablen.


\subsection{Integration des OPC-Servers in das System%
     \label{sec:3-integration}}

open62541 bietet drei Möglichkeiten zum Abgleich von Variablen mit dem Prozessabbild~\citep[vgl.][Tutorials - Connecting a Variable with a Physical Process]{web-open62541}:
\begin{itemize}
    \item Manuelles oder zyklisches Aktualisieren
    \item Variable Value Callback
    \item Variable Datasource
\end{itemize}

Die zyklische Aktualisierung eines oder mehrerer Werte nimmt, abhängig von der Zykluszeit, viele Systemressourcen in Anspruch. Value Callbacks ermöglichen es, einen Variablenwert effizienter mit einer Ressource wie etwa einem Prozessabbild zu synchronisieren. An die Variable wird ein Callback angehängt, welches vor jedem Lesen und nach jedem Schreibvorgang ausgeführt wird.
Der Wert der Variablen wird weiterhin im Variablenknoten auf dem OPC-Server gespeichert, der Abgleich mit der verknüpften Ressource erfolgt durch die Callback-Methoden.

Sogenannte Datenquellen gehen noch einen Schritt weiter. Der Server leitet jede Lese- und Schreibanforderung direkt an eine Callback-Funktion weiter. Beim Lesen liefert der Rückruf eine Kopie des aktuellen Wertes. Die Datenquelle muss intern ein eigenes Speichermanagement implementieren.

Der Zugriff auf die Werte des Prozessabbildes erfolgt, wie in Abschnitt~\ref{sec:3-anbindung} beschrieben, über von piControl bereitgestellte Methoden. Um die durch open62541 gepflegte OPC-Datenstruktur und das durch piControl verwaltete Prozessabbild möglichst effektiv verknüpfen zu können, soll diese Interaktion mittels Datenquellen und den zugehörigen Callbacks implementiert werden.
% % % Imports nur für Referenzenauflösung während des Schreibens! Vorm Kompilieren auskommentieren!
% \bibliography{0_hauptdatei}
% \input{1_einleitung}
% \input{2_grundlagen}
% \input{3_konzeption}
% \input{4_implementierung}
% \input{5_tests}
% \input{6_zusammenfassung}
% \input{anhang}
% % Ende Imports

\section{Implementierung%
  \label{sec:4-implementierung}}
Das folgende Kapitel stellt in Auszügen die Implementierung des OPC-Servers sowie die Anbindung an die IO-Module
der SPS dar. Der Schwerpunkt liegt hierbei auf der Funktionsweise des piControl-Treibers und dessen Integration in das Projekt. Abschnitt~\ref{sec:4-picontrol} erklärt die zum Schreibens eines Bits verwendeten Funktionsaufrufe.
Zuvor soll jedoch in Abschnitt~\ref{sec:4-open62541} der Teil des OPC-Servers vorgestellt werden, welcher auf besagten Treiber zugreift. 

\subsection{Implementierung des OPC-Servers%
     \label{sec:4-open62541}}
Wie im vorangegangenen Abschnitt~\ref{sec:3-integration} begründet, soll die Verknüpfung zwischen dem Prozessabbild der SPS und den auf dem OPC-Server bereitgestellten Werten über sog.\,Datenquellen erfolgen. Hierzu ist zunächst eine Callback-Methode zu implementieren, welche bei einem Lese- oder Schreibzugriff auf eine Variable aufgerufen wird. Die Verknüpfung zwischen Callback-Methode und Variable muss manuell erfolgen.

\begin{lstlisting}[language={c},firstnumber=237,caption={Auszug der Methode \lstinline{linkDataSourceVariable} in \lstinline{variables.c}\label{lst:4-linkDataSourceVariable}}]
extern UA_StatusCode
 linkDataSourceVariable(UA_Server *server, UA_NodeId nodeId) {
     bool readonly = false;
     UA_DataSource dataSourceVariable;
     UA_StatusCode rc; |>\setcounter{lstnumber}{254}<|

     dataSourceVariable.read = readDataSourceVariable;
     if (!readonly)
        dataSourceVariable.write = writeDataSourceVariable;
     else
        dataSourceVariable.write = writeReadonlyDataSourceVariable;

     return UA_Server_setVariableNode_dataSource(server, nodeId, dataSourceVariable);
 }
\end{lstlisting}

\begin{figure}[h]
    \centering
    \includegraphics[width=0.42\textwidth]{doc/img/OPC_RevPiDO.pdf}
    \caption{Auszug des verwendeten Nodesets, hier Digitalausgang 1 des Versuchsaufbaus
      \label{fig:opc-do}}
\end{figure}

Die in Listing~\ref{lst:4-linkDataSourceVariable} abgebildete Methode \lstinline{linkDataSourceVariable()} erzeugt ein Struct vom Typ \lstinline{UA_DataSource}. In diesem werden dem Lesen und Schreiben einer OPC-Variablen entsprechende Callback-Methoden zugewiesen. Die Verknüpfung einer OPC-Variable, genauer ihrer NodeId, mit der zuvor definierten Datenquelle erfolgt über die von open62541 bereitgestellte Methode \lstinline{UA_Server_setVariableNode_dataSource()}. Vor dem Lesen und nach dem Schreiben dieser Variable werden von nun an die entsprechenden Callbacks aufgerufen.
     
\begin{lstlisting}[language={c},firstnumber=168,caption={Auszug des Callbacks \lstinline{writeDataSourceVariable} in \lstinline{variables.c}\label{lst:4-writeDataSourceVariable}}]  
extern UA_StatusCode
 writeDataSourceVariable(UA_Server *server,
            const UA_NodeId *sessionId, void *sessionContext,
            const UA_NodeId *nodeId, void *nodeContext,
            const UA_NumericRange *range, const UA_DataValue *dataValue) {

    UA_StatusCode retval  = UA_STATUSCODE_GOOD;
    UA_NodeId *nameNodeId = UA_malloc(sizeof(UA_NodeId));
    UA_QualifiedName nameQN = UA_QUALIFIEDNAME(1, "Name");
    UA_Variant nameVar;
    UA_Boolean bit;

    retval |= findSiblingByBrowsename(server, nodeId, &nameQN, nameNodeId);
    retval |= UA_Server_readValue(server, *nameNodeId, &nameVar);
    retval |= UA_Boolean_copy(dataValue->value.data, &bit);

    |>\tikzmarkin[set border color=martinired]{writeIO}<|PI_writeSingleIO(String_fromUA_String(nameVar.data), &bit, false);                                                 |>\tikzmarkend{writeIO}<|

    free(nameNodeId);
    return retval;
 }
\end{lstlisting}

Listing~\ref{lst:4-writeDataSourceVariable} zeigt die Callback-Methode, welche nach dem Schreiben einer Variablen auf dem OPC-Server aufgerufen wird.
Dieser Methode wird neben der NodeId der mit ihr verknüpften Variablen auch der Wert dieser in Form eines Zeigers auf ein Struct vom Typ \lstinline{UA_DataValue} übergeben.

Die Gestaltung des hier verwendeten Nodesets sieht vor, dass in einer OPC-Variablen \lstinline{"Name"} der Bezeichner des zu schreibenden Digitalausgangs hinterlegt ist, siehe Abbildung~\ref{fig:opc-do}. Dies erlaubt eine Rekonfiguration der Ein- und Ausgänge der SPS ohne Änderungen im Programmcode des OPC-Servers vornehmen zu müssen.
Es ist daher erforderlich, nach jedem Schreiben einer mit einem Digitalausgang verknüpften Variablen, hier \lstinline{"Value"}, dessen Bezeichner \lstinline{"Name"} abzufragen. 
Dies geschieht in den Zeilen 180 und 181.
Anschließend wird dieser Bezeichner sowie der zu schreibende Wert der Methode \lstinline{PI_writeSingleIO()} übergeben, welche wiederum die Interaktion mit piControl übernimmt (vgl. Abschnitt \ref{sec:4-picontrol}).
 
\subsection{Integration von piControl%
     \label{sec:4-picontrol}}
In Abschnitt~\ref{sec:2-io} wurde die Anbindung der IO-Module des Revolution Pi sowie die Funktionsweise von piControl aus Anwendersicht beschrieben. Die verfügbare Literatur beschränkt sich auch auf lediglich diese Sicht; eine weiterführende Dokumentation für Entwickler gibt es, neben der in Abschnitt~\ref{sec:3-anbindung} vorgestellten Manpage, nicht. 
In diesem Abschnitt soll daher der Quellcode von piControl sowie dessen Verwendung im Projekt genauer betrachtet werden.
Hierzu wird exemplarisch die in Abschnitt~\ref{sec:4-open62541} eingeführte Methode \lstinline{PI_writeSingleIO()} untersucht.
Diese Methode ermöglicht das Setzen eines einzelnen Bits im Prozessabbild der SPS, und damit das Schalten eines digitalen Ausgangs auf einem IO-Modul.
Die äquivalente Methode \lstinline{int piControlGetBitValue(SPIValue *pSpiValue)} zum Lesen eines Bits bzw. Eingangs funktioniert analog und soll daher an dieser Stelle nicht dediziert erörtert werden.

\begin{lstlisting}[language={c},firstnumber=97,
                   caption={Setzen eines phsikalischen, digitalen Ausgangs in \lstinline{revpi.c}
                   \label{lst:4-PI_writeSingleIO}}]
extern void PI_writeSingleIO(char *pszVariableName, bool *bit, bool verbose)
{
	int rc;
	SPIVariable sPiVariable;
	SPIValue sPIValue;

	strncpy(sPiVariable.strVarName, pszVariableName, sizeof(sPiVariable.strVarName));
	rc = piControlGetVariableInfo(&sPiVariable);
	if (rc < 0) {
		printf("Cannot find variable '%s'\n", pszVariableName);
		return;
	}

		sPIValue.i16uAddress = sPiVariable.i16uAddress;
		sPIValue.i8uBit = sPiVariable.i8uBit;
		sPIValue.i8uValue = *bit;
		rc = |>\tikzmarkin[set border color=martinired]{setBitValue}<|piControlSetBitValue(&sPIValue)|>\tikzmarkend{setBitValue}<|;
		if (rc < 0)
			printf("Set bit error %s\n", getWriteError(rc));
		else if (verbose)
			printf("Set bit %d on byte at offset %d. Value %d\n", sPIValue.i8uBit, sPIValue.i16uAddress,
			       sPIValue.i8uValue);
}
\end{lstlisting}

Der Programmcode in Listing~\ref{lst:4-PI_writeSingleIO} ist Teil des implementierten OPC-Servers. In diesem wird auf zwei Funktionen des piControl-Treibers zugegriffen. 
Beiden Methoden wird als Argument ein Zeiger auf ein Struct vom Typ \lstinline{SPIValue} übergeben. Der im Struct abgelegte Name wird mittels \lstinline{piControlGetVariableInfo(&sPIValue)} zu einer Adresse im Prozessabbild aufgelöst. Diese wird in \lstinline{sPIValue.i16uAdress} gespeichert. Der Wert der Variablen wird anschließend mittels \lstinline{piControlSetBitValue(&sPIValue)} an dieser Adresse in das Prozessabbild geschrieben.

\begin{lstlisting}[language={c},firstnumber=309,caption={Methode \lstinline{piControlSetBitValue} in \lstinline{piControlIf.c}\label{lst:4-piControlSetBitValue}}]
int |>\tikzmarkin[set border color=martiniblue]{setBitValueFcn}<|piControlSetBitValue(SPIValue *pSpiValue)|>\tikzmarkend{setBitValueFcn}<|
{
    piControlOpen();

    if (PiControlHandle_g < 0)
	    return -ENODEV;

    pSpiValue->i16uAddress += pSpiValue->i8uBit / 8;
    pSpiValue->i8uBit %= 8;

    if (|>\tikzmarkin[set border color=martinired]{ioctl}<|ioctl(PiControlHandle_g, KB_SET_VALUE, pSpiValue)|>\tikzmarkend{ioctl}<| < 0)
	    return errno;

    return 0;
}
\end{lstlisting}

Die in Listing~\ref{lst:4-piControlSetBitValue} dargestellte Methode \lstinline{piControlSetBitValue} ist lediglich eine Hüllfunktion (häufig auch als Wrapper-Funktion bezeichnet) für einen Aufruf des \lstinline{ioctl} Kernel-Moduls.
Folgende Parameter werden übergeben:
\lstinline{PiControlHandle_g} ist die Referenz auf die Geräte-Datei des piControl-Treibers. \lstinline{KB_SET_VALUE} ist das ioctl-Kommando zum Schreiben eines Bits in das Prozessabbild. Der Zeiger \lstinline{pSpiValue} verweist auf ein Struct des bereits vorgestellten Typs \lstinline{SPIValue}.

\begin{lstlisting}[language={c},firstnumber=80,caption={Methode \lstinline{piControlOpen} in \lstinline{piControlIf.c}\label{lst:4-piControlOpen}}]
void piControlOpen(void)
{
    /* open handle if needed */
    if (PiControlHandle_g < 0)
    {
	    |>\tikzmarkin[set border color=martiniblue]{PiControlHandle}<|PiControlHandle_g = open(PICONTROL_DEVICE, O_RDWR)|>\tikzmarkend{PiControlHandle}<|;
    }
}
\end{lstlisting}

Die in Listing~\ref{lst:4-piControlOpen} dargestellte Methode öffnet, sofern nicht bereits geschehen, die Geräte-Datei. Das Macro \lstinline{PICONTROL_DEVICE} verweist hierbei auf \lstinline{/dev/piControl0}.

\begin{lstlisting}[language={c},firstnumber=721,caption={Methode \lstinline{piControlIoctl} in \lstinline{piControlMain.c}\label{lst:4-piControlIoctl}}]
static long |>\tikzmarkin[set border color=martiniblue, below offset=0.9em]{piControlIoctl}<|piControlIoctl(struct file *file, unsigned int prg_nr, 
                           unsigned long usr_addr)                                      |>\tikzmarkend{piControlIoctl}<|
{
  int status = -EFAULT;
  tpiControlInst *priv;
  int timeout = 10000;	// ms

  if (prg_nr != KB_CONFIG_SEND && prg_nr != KB_CONFIG_START && !isRunning()) {
  	return -EAGAIN;
  }

  priv = (tpiControlInst *) file->private_data;

  if (prg_nr != KB_GET_LAST_MESSAGE) {
  	// clear old message
  	priv->pcErrorMessage[0] = 0;
  }

  switch (prg_nr) {|>\setcounter{lstnumber}{864}<|

    case |>\tikzmarkin[set border color=martiniblue]{KB_SET_VALUE}<|KB_SET_VALUE:|>\tikzmarkend{KB_SET_VALUE}<|
  		{
  			SPIValue *pValue = (SPIValue *) usr_addr;

  			if (!isRunning())
  				return -EFAULT;

  			if (pValue->i16uAddress >= KB_PI_LEN) {
  				status = -EFAULT;
  			} else {
  				INT8U i8uValue_l;
  				my_rt_mutex_lock(&piDev_g.lockPI);
  				i8uValue_l = piDev_g.ai8uPI[pValue->i16uAddress];

  				if (pValue->i8uBit >= 8) {
  					i8uValue_l = pValue->i8uValue;
  				} else {
  					if (pValue->i8uValue)
  						i8uValue_l |= (1 << pValue->i8uBit);
  					else
  						i8uValue_l &= ~(1 << pValue->i8uBit);
  				}

  				|>\tikzmarkin[set border color=martinired]{i8uValue}<|piDev_g.ai8uPI[pValue->i16uAddress] = i8uValue_l;|>\tikzmarkend{i8uValue}<|
  				rt_mutex_unlock(&piDev_g.lockPI);

  #ifdef VERBOSE
  				pr_info("piControlIoctl Addr=%u, bit=%u: %02x %02x\n", pValue->i16uAddress, pValue->i8uBit, pValue->i8uValue, i8uValue_l);
  #endif

  				status = 0;
  			}
  		}
  		break; |>\setcounter{lstnumber}{1314}<|

    default:
      pr_err("Invalid Ioctl");
      return (-EINVAL);
      break;

    }

    return status;
  }
\end{lstlisting}

Listing~\ref{lst:4-piControlIoctl} zeigt in Auszügen die ioctl-Methode des piControl Kernel-Treibers. Diese bekommt folgende Argumente übergeben: \lstinline{struct file *file} enthält den Verweis auf die Geräte-Datei, hier \lstinline{/dev/piControl0}. Der Wert von \lstinline{unsigned int prg_nr} beschreibt die Anfrage an den Treiber, in diesem Fall \lstinline{KB_SET_VALUE}. Das Argument \lstinline{unsigned long usr_addr} enthält einen typ-agnostischen Pointer. Dieser verweist auf einen Speicherbereich, in welchem die zur Bearbeitung der Anfrage notwendigen Daten abgelegt sind. Hier können auch vom Treiber empfangene Daten dem Anwendungsprogramm bereitgestellt werden. 

Die switch-case-Anweisung führt die über das Argument \lstinline{prg_nr} spezifizierte Aktion aus. Hier betrachten wir \lstinline{KB_SET_VALUE}:
Zunächst wird in Zeile 868 der übergebene Zeiger \lstinline{usr_addr} mittels explizitem Typecast zu einem Zeiger des Typs \lstinline{SPIValue *} konvertiert. Da dieser auf Daten im Userspace verweist, ist beim Zugriff durch den Kernel-Treiber besondere Vorsicht geboten.
In Zeile 877 wird mittels Mutex das Prozessabbild \lstinline{piDev_g} für den Zugriff durch andere Threads oder Prozesse gesperrt.
\lstinline{my_rt_mutex_lock} verweist hierbei auf die Funktion \lstinline{rt_mutex_lock} aus \lstinline{linux/sched.h}\footnote{Offenbar wurde hier auch eine alternative Implementierung vorgesehen, siehe revpi\_common.h}

In Zeile 889 wird das Byte \lstinline{i8uValue_l}, welches den zu schreibenden Wert enthält in das Prozessabbild übertragen. Anschließend wird die Mutex auf \lstinline{piDev_g} wieder entsperrt.
\newpage

\begin{lstlisting}[language={c},firstnumber=62,caption={Auszug des Struct \lstinline{spiControlDev} in \lstinline{piControlMain.h}\label{lst:4-spiControlDev}}]
|>\tikzmarkin[set border color=martiniblue]{spiControlDev}<|typedef struct spiControlDev|>\tikzmarkend{spiControlDev}<| {
	// device driver stuff
	int init_step;
	enum revpi_machine machine_type;
	void *machine;
	struct cdev cdev;	// Char device structure
	struct device *dev;
	struct thermal_zone_device *thermal_zone;

	|>\tikzmarkin[set border color=martiniblue]{processImage}<|// process image stuff
	INT8U ai8uPI[KB_PI_LEN];
	INT8U ai8uPIDefault|>\tikzmarkin[set border color=martinired]{KB_PI_LEN_0}<|[KB_PI_LEN]|>\tikzmarkend{KB_PI_LEN_0}<|;
	struct rt_mutex lockPI;        |>\tikzmarkend{processImage}<|
	bool stopIO;
	piDevices *devs; |>\setcounter{lstnumber}{94}<|
} tpiControlDev;
\end{lstlisting}

Das Prozessabbild ist als Byte-Array der Länge \lstinline{KB_PI_LEN} in Listing~\ref{lst:4-spiControlDev} definiert. Konfigurationsparameter wie \lstinline{KB_PI_LEN} oder die Zykluszeit für den Datenaustausch zwischen SPS und IO-Modulen sind im folgenden Listing~\ref{lst:4-process} definiert.

\begin{lstlisting}[language={c},firstnumber=119,caption={Konfigurationsparameter des Prozessabbildes in project.h\label{lst:4-process}}]
#define INTERVAL_PI_GATE (5*1000*1000)  // 5 ms piGateCommunication |>\setcounter{lstnumber}{128}<|

#define INTERVAL_IO_COM (5*1000*1000)  // 5 ms piIoComm |>\setcounter{lstnumber}{132}<|

#define KB_PD_LEN       512
|>\tikzmarkin[set border color=martiniblue]{KB_PI_LEN_1}<|#define KB_PI_LEN       4096|>\tikzmarkend{KB_PI_LEN_1}<|
\end{lstlisting}

Das zu setzende Bit wurde zu diesem Zeitpunkt erfolgreich in das Prozessabbild der SPS geschrieben.
Es stellt sich die Frage, wie dieses nun an das IO-Modul kommuniziert wird.
Die Kommunikation mit allen angebundenen Modulen ist ebenfalls Aufgabe des piControl-Treibers.

\begin{lstlisting}[language={c},firstnumber=256,caption={Auszug der Methode \lstinline{piIoThread} in \lstinline{revpi_core.c}\label{lst:4-piIoThread}}]
static int piIoThread(void *data)
{
	//TODO int value = 0;
	ktime_t time;
	ktime_t now;
	s64 tDiff;

	hrtimer_init(&piCore_g.ioTimer, CLOCK_MONOTONIC, HRTIMER_MODE_ABS);
	piCore_g.ioTimer.function = piIoTimer;

	pr_info("piIO thread started\n");

	now = hrtimer_cb_get_time(&piCore_g.ioTimer);

	PiBridgeMaster_Reset();

	while (!kthread_should_stop()) {
		if (|>\tikzmarkin[set border color=martinired]{PiBridgeMaster}<|PiBridgeMaster_Run()|>\tikzmarkend{PiBridgeMaster}<| < 0)
			break;
	}

	RevPiDevice_finish();

	pr_info("piIO exit\n");
	return 0;
}
\end{lstlisting}

Der Kernel-Thread \lstinline{piIoThread} ist verantwortlich für den zyklischen Datenaustausch mit den IO-Modulen. In diesem wird fortlaufend die Methode \lstinline{PiBridgeMaster_Run()} aufgerufen, siehe Listing~\ref{lst:4-piIoThread}.

\begin{lstlisting}[language={c},firstnumber=262,caption={Auszug der Methode \lstinline{PiBridgeMaster_Run(void)} in \lstinline{RevPiDevice.c}\label{lst:4-PiBridgeMaster_Run}}]
int PiBridgeMaster_Run(void)
{
	static kbUT_Timer tTimeoutTimer_s;
	static kbUT_Timer tConfigTimeoutTimer_s;
	static int error_cnt;
	static INT8U last_led;
	static unsigned long last_update;
	int ret = 0;
	int i;

	my_rt_mutex_lock(&piCore_g.lockBridgeState);
	if (piCore_g.eBridgeState != piBridgeStop) {
		switch (eRunStatus_s) { |>\setcounter{lstnumber}{514}<|
		    case enPiBridgeMasterStatus_EndOfConfig:|>\setcounter{lstnumber}{621}<|
		    if (|>\tikzmarkin[set border color=martinired]{RevPiDevice}<|RevPiDevice_run()|>\tikzmarkend{RevPiDevice}<|) {
				// an error occured, check error limits |>\setcounter{lstnumber}{641}<|
			} else {
				ret = 1;
			}
			piCore_g.image.drv.i16uRS485ErrorCnt = RevPiDevice_getErrCnt();
			break;
\end{lstlisting}

Die in Listing~\ref{lst:4-PiBridgeMaster_Run} dargestellte Methode ist eine sog. State-Machine. Ist die Konfiguration der IO-Module erfolgreich abgeschlossen, so führt sie bei Aufruf lediglich die Methode \lstinline{RevPiDevice_run()} aus.

\begin{lstlisting}[language={c},firstnumber=140,caption={Auszug der Methode \lstinline{RevPiDevice_run(void)} in \lstinline{RevPiDevice.c}\label{lst:4-RevPiDevice_run}}]
int RevPiDevice_run(void)
{
	INT8U i8uDevice = 0;
	INT32U r;
	int retval = 0;

	RevPiDevices_s.i16uErrorCnt = 0;

	for (i8uDevice = 0; i8uDevice < RevPiDevice_getDevCnt(); i8uDevice++) {
		if (RevPiDevice_getDev(i8uDevice)->i8uActive) {
			switch (RevPiDevice_getDev(i8uDevice)->sId.i16uModulType) {
			case KUNBUS_FW_DESCR_TYP_PI_DIO_14:
			case KUNBUS_FW_DESCR_TYP_PI_DI_16:
			case KUNBUS_FW_DESCR_TYP_PI_DO_16:
				r = |>\tikzmarkin[set border color=martinired]{sendCyclicTelegram}<|piDIOComm_sendCyclicTelegram(i8uDevice)|>\tikzmarkend{sendCyclicTelegram}\setcounter{lstnumber}{166} <|;

				break; |>\setcounter{lstnumber}{216}<|
			}
		}
	} |>\setcounter{lstnumber}{227}<|
	return retval;
}
\end{lstlisting}

Diese iteriert wie in Listing~\ref{lst:4-RevPiDevice_run} abgebildete durch alle gegenwärtig in der SPS konfigurierten Module. Ist das aktuelle Modul als aktiv markiert, so wird anhand eines sog. Firmware-Descriptors entschieden, welche Methode für die Ansteuerung des Moduls aufzurufen ist.

\begin{lstlisting}[language={c},firstnumber=161,caption={Auszug der Methode \lstinline{piDIOComm_sendCyclicTelegram} in \lstinline{piDIOComm.c}\label{lst:4-sendCyclicTelegram}}]
INT32U piDIOComm_sendCyclicTelegram(INT8U i8uDevice_p)
{
	INT32U i32uRv_l = 0;
	SIOGeneric sRequest_l;
	SIOGeneric sResponse_l;
	INT8U len_l, data_out[18], i, p, data_in[70];
	INT8U i8uAddress;
	int ret; |>\setcounter{lstnumber}{239}<|
	
    |>\tikzmarkin[set border color=martinired]{piIoComm}<|ret = piIoComm_send((INT8U *) & sRequest_l, IOPROTOCOL_HEADER_LENGTH + len_l + 1);  |>\tikzmarkend{piIoComm}\setcounter{lstnumber}{298}<|
}
\end{lstlisting}

Im Falle des hier verwendeten DO-Moduls wird die in Listing~\ref{lst:4-sendCyclicTelegram} abgebildete Methode \lstinline{piDIOComm_sendCyclicTelegram()} aufgerufen. Dieser wird ein Zeiger auf das zu schreibende Gerät übergeben. 
Zunächst wird das Prozessabbild mittels eines proprietären, jedoch im Quellcode offen nachvollziehbaren Protokolls in ein \lstinline{sRequest_l} genanntes Byte-Array umgewandelt. Dieser Schritt ist in Listing~\ref{lst:4-sendCyclicTelegram} nicht abgebildet. Anschließend wird \lstinline{piIoComm_send()} ein Zeiger auf die so generierte Schreib-Anfrage übergeben.

\begin{lstlisting}[language={c},firstnumber=220,caption={Auszug der Methode \lstinline{piIOComm_send} in \lstinline{piIOComm.c}\label{lst:4-piIOComm_send}}]
int piIoComm_send(INT8U * buf_p, INT16U i16uLen_p)
{
	ssize_t write_l = 0;
	INT16U i16uSent_l = 0;|>\setcounter{lstnumber}{249}<|

	while (i16uSent_l < i16uLen_p) {
		write_l = vfs_write(piIoComm_fd_m, buf_p + i16uSent_l, i16uLen_p - i16uSent_l, &piIoComm_fd_m->f_pos);
		if (write_l < 0) {
			pr_info_serial("write error %d\n", (int)write_l);
			return -1;
		} 
		i16uSent_l += write_l;|>\setcounter{lstnumber}{263}<|
	}
	clear();
	vfs_fsync(piIoComm_fd_m, 1);
	return 0;
}
\end{lstlisting}

Listing~\ref{lst:4-piIOComm_send} zeigt die Implementierung von \lstinline{piIoComm_send()}. Diese Methode ist für das Schreiben der oben generierten Anfrage auf die seriellen Schnittstelle verantwortlich. Realisiert wird dies mittels der Methode \lstinline{vfs_write()}. Diese ist in \lstinline{<linux/fs.h>} definiert. Sie ermöglicht das Schreiben einer Datei im Userspace aus dem Kernel heraus. Geschrieben wird hier die Datei mit dem Deskriptor \lstinline{piIoComm_fd_m}.
Da die Funktion \lstinline{vfs_write()} durch andere Kernel-Tasks unterbrochen werden kann, ist nicht gewährleistet, dass die gesamte Anfrage mit nur einem Aufruf geschrieben wird. Die oben abgebildete while-Schleife stellt das vollständige Senden der Anfrage sicher.

\begin{lstlisting}[language={c},firstnumber=157,caption={Auszug der Methode \lstinline{piIOComm_open_serial} in \lstinline{piIOComm.c}\label{lst:4-piIOComm_open_serial}}]
int piIoComm_open_serial(void)
{   |>\setcounter{lstnumber}{167}<|
	struct file *fd;	/* Filedeskriptor */
	struct termios newtio;	/* Schnittstellenoptionen */

	|>\tikzmarkin[set border color=martiniblue]{fd}<|/* Port oeffnen - read/write, kein "controlling tty", 
	    Status von DCD ignorieren */
	fd = filp_open(|>\tikzmarkin[set border color=martinired]{tty}<|REV_PI_TTY_DEVICE|>\tikzmarkend{tty}<|, O_RDWR | O_NOCTTY, 0); |>\setcounter{lstnumber}{208}<|
	
	piIoComm_fd_m = fd;                                                      |>\tikzmarkend{fd}\setcounter{lstnumber}{217}<|

	return 0;
}
\end{lstlisting}

Der zum Schreiben auf die serielle Schnittstelle verwendete Datei-Deskriptor wird von der in Listing~\ref{lst:4-piIOComm_open_serial} abgebildeten Methode \lstinline{piIoComm_open_serial()} generiert. 

\begin{lstlisting}[language={c},firstnumber=45,caption={Definition der seriellen Schnittstelle in \lstinline{piIOComm.h}\label{lst:4-REV_PI_TTY_DEVICE}}]
#define REV_PI_TTY_DEVICE	"/dev/ttyAMA0"
\end{lstlisting}

Das in Listing~\ref{lst:4-REV_PI_TTY_DEVICE} definierte Macro verweist auf eine der seriellen Schnittstellen des RaspberryPi.
Die Implementierung des zugehörigen Schnittstellentreibers soll hier nicht weiter untersucht werden. Somit ist an dieser Stelle die Kette vom Setzen einer Variablen auf dem OPC-Server bis hin zur Aktualisierung des Prozessabbilds der IO-Module geschlossen.

% \begin{lstlisting}[language={c},firstnumber={226},caption={Setzen der Scheduler-Priorität auf SCHED\_FIFO in 
% revpi\_common.c\label{lst:2-sched_priority}}]
% param.sched_priority = ktprio->prio;
% ret = sched_setscheduler(child, SCHED_FIFO, &param);
% \end{lstlisting}
% % % Imports nur für Referenzenauflösung während des Schreibens! Vorm Kompilieren auskommentieren!
% \bibliography{0_hauptdatei}
% \input{1_einleitung}
% \input{2_grundlagen}
% \input{3_konzeption}
% \input{4_implementierung}
% \input{5_tests}
% \input{6_zusammenfassung}
% % Ende Imports

\section{Test des OPC-Servers im Gesamtsystem%
  \label{sec:5-tests}}

% % % Imports nur für Referenzenauflösung während des schreibens! Vorm Kompilieren auskommentieren!
% \bibliography{0_hauptdatei}
% \input{1_einleitung}
% \input{2_grundlagen}
% \input{3_konzeption}
% \input{4_implementierung}
% \input{5_tests}
% \input{6_zusammenfassung}
% % Ende Imports

\section{Zusammenfassung und Ausblick%
  \label{sec:6-fazit}}
Der folgende Abschnitt~\ref{sec:6-zusammenfassung} fasst die gewonnenen Erkenntnisse und den Stand der Implementierung zusammen.
Den Abschluss dieser Arbeit bildet der Ausblick in Abschnitt~\ref{sec:6-ausblick}.

\subsection{Zusammenfassung%
     \label{sec:6-zusammenfassung}}

\subsection{Ausblick%
     \label{sec:6-ausblick}}

% \input{anhang}
% % Ende Imports

\section{Implementierung%
  \label{sec:4-implementierung}}
Das folgende Kapitel stellt in Auszügen die Implementierung des OPC-Servers sowie die Anbindung an die IO-Module
der SPS dar. Der Schwerpunkt liegt hierbei auf der Funktionsweise des piControl-Treibers und dessen Integration in das Projekt. Abschnitt~\ref{sec:4-picontrol} erklärt die zum Schreibens eines Bits verwendeten Funktionsaufrufe.
Zuvor soll jedoch in Abschnitt~\ref{sec:4-open62541} der Teil des OPC-Servers vorgestellt werden, welcher auf besagten Treiber zugreift. 

\subsection{Implementierung des OPC-Servers%
     \label{sec:4-open62541}}
Wie im vorangegangenen Abschnitt~\ref{sec:3-integration} begründet, soll die Verknüpfung zwischen dem Prozessabbild der SPS und den auf dem OPC-Server bereitgestellten Werten über sog.\,Datenquellen erfolgen. Hierzu ist zunächst eine Callback-Methode zu implementieren, welche bei einem Lese- oder Schreibzugriff auf eine Variable aufgerufen wird. Die Verknüpfung zwischen Callback-Methode und Variable muss manuell erfolgen.

\begin{lstlisting}[language={c},firstnumber=237,caption={Auszug der Methode \lstinline{linkDataSourceVariable} in \lstinline{variables.c}\label{lst:4-linkDataSourceVariable}}]
extern UA_StatusCode
 linkDataSourceVariable(UA_Server *server, UA_NodeId nodeId) {
     bool readonly = false;
     UA_DataSource dataSourceVariable;
     UA_StatusCode rc; |>\setcounter{lstnumber}{254}<|

     dataSourceVariable.read = readDataSourceVariable;
     if (!readonly)
        dataSourceVariable.write = writeDataSourceVariable;
     else
        dataSourceVariable.write = writeReadonlyDataSourceVariable;

     return UA_Server_setVariableNode_dataSource(server, nodeId, dataSourceVariable);
 }
\end{lstlisting}

\begin{figure}[h]
    \centering
    \includegraphics[width=0.42\textwidth]{doc/img/OPC_RevPiDO.pdf}
    \caption{Auszug des verwendeten Nodesets, hier Digitalausgang 1 des Versuchsaufbaus
      \label{fig:opc-do}}
\end{figure}

Die in Listing~\ref{lst:4-linkDataSourceVariable} abgebildete Methode \lstinline{linkDataSourceVariable()} erzeugt ein Struct vom Typ \lstinline{UA_DataSource}. In diesem werden dem Lesen und Schreiben einer OPC-Variablen entsprechende Callback-Methoden zugewiesen. Die Verknüpfung einer OPC-Variable, genauer ihrer NodeId, mit der zuvor definierten Datenquelle erfolgt über die von open62541 bereitgestellte Methode \lstinline{UA_Server_setVariableNode_dataSource()}. Vor dem Lesen und nach dem Schreiben dieser Variable werden von nun an die entsprechenden Callbacks aufgerufen.
     
\begin{lstlisting}[language={c},firstnumber=168,caption={Auszug des Callbacks \lstinline{writeDataSourceVariable} in \lstinline{variables.c}\label{lst:4-writeDataSourceVariable}}]  
extern UA_StatusCode
 writeDataSourceVariable(UA_Server *server,
            const UA_NodeId *sessionId, void *sessionContext,
            const UA_NodeId *nodeId, void *nodeContext,
            const UA_NumericRange *range, const UA_DataValue *dataValue) {

    UA_StatusCode retval  = UA_STATUSCODE_GOOD;
    UA_NodeId *nameNodeId = UA_malloc(sizeof(UA_NodeId));
    UA_QualifiedName nameQN = UA_QUALIFIEDNAME(1, "Name");
    UA_Variant nameVar;
    UA_Boolean bit;

    retval |= findSiblingByBrowsename(server, nodeId, &nameQN, nameNodeId);
    retval |= UA_Server_readValue(server, *nameNodeId, &nameVar);
    retval |= UA_Boolean_copy(dataValue->value.data, &bit);

    |>\tikzmarkin[set border color=martinired]{writeIO}<|PI_writeSingleIO(String_fromUA_String(nameVar.data), &bit, false);                                                 |>\tikzmarkend{writeIO}<|

    free(nameNodeId);
    return retval;
 }
\end{lstlisting}

Listing~\ref{lst:4-writeDataSourceVariable} zeigt die Callback-Methode, welche nach dem Schreiben einer Variablen auf dem OPC-Server aufgerufen wird.
Dieser Methode wird neben der NodeId der mit ihr verknüpften Variablen auch der Wert dieser in Form eines Zeigers auf ein Struct vom Typ \lstinline{UA_DataValue} übergeben.

Die Gestaltung des hier verwendeten Nodesets sieht vor, dass in einer OPC-Variablen \lstinline{"Name"} der Bezeichner des zu schreibenden Digitalausgangs hinterlegt ist, siehe Abbildung~\ref{fig:opc-do}. Dies erlaubt eine Rekonfiguration der Ein- und Ausgänge der SPS ohne Änderungen im Programmcode des OPC-Servers vornehmen zu müssen.
Es ist daher erforderlich, nach jedem Schreiben einer mit einem Digitalausgang verknüpften Variablen, hier \lstinline{"Value"}, dessen Bezeichner \lstinline{"Name"} abzufragen. 
Dies geschieht in den Zeilen 180 und 181.
Anschließend wird dieser Bezeichner sowie der zu schreibende Wert der Methode \lstinline{PI_writeSingleIO()} übergeben, welche wiederum die Interaktion mit piControl übernimmt (vgl. Abschnitt \ref{sec:4-picontrol}).
 
\subsection{Integration von piControl%
     \label{sec:4-picontrol}}
In Abschnitt~\ref{sec:2-io} wurde die Anbindung der IO-Module des Revolution Pi sowie die Funktionsweise von piControl aus Anwendersicht beschrieben. Die verfügbare Literatur beschränkt sich auch auf lediglich diese Sicht; eine weiterführende Dokumentation für Entwickler gibt es, neben der in Abschnitt~\ref{sec:3-anbindung} vorgestellten Manpage, nicht. 
In diesem Abschnitt soll daher der Quellcode von piControl sowie dessen Verwendung im Projekt genauer betrachtet werden.
Hierzu wird exemplarisch die in Abschnitt~\ref{sec:4-open62541} eingeführte Methode \lstinline{PI_writeSingleIO()} untersucht.
Diese Methode ermöglicht das Setzen eines einzelnen Bits im Prozessabbild der SPS, und damit das Schalten eines digitalen Ausgangs auf einem IO-Modul.
Die äquivalente Methode \lstinline{int piControlGetBitValue(SPIValue *pSpiValue)} zum Lesen eines Bits bzw. Eingangs funktioniert analog und soll daher an dieser Stelle nicht dediziert erörtert werden.

\begin{lstlisting}[language={c},firstnumber=97,
                   caption={Setzen eines phsikalischen, digitalen Ausgangs in \lstinline{revpi.c}
                   \label{lst:4-PI_writeSingleIO}}]
extern void PI_writeSingleIO(char *pszVariableName, bool *bit, bool verbose)
{
	int rc;
	SPIVariable sPiVariable;
	SPIValue sPIValue;

	strncpy(sPiVariable.strVarName, pszVariableName, sizeof(sPiVariable.strVarName));
	rc = piControlGetVariableInfo(&sPiVariable);
	if (rc < 0) {
		printf("Cannot find variable '%s'\n", pszVariableName);
		return;
	}

		sPIValue.i16uAddress = sPiVariable.i16uAddress;
		sPIValue.i8uBit = sPiVariable.i8uBit;
		sPIValue.i8uValue = *bit;
		rc = |>\tikzmarkin[set border color=martinired]{setBitValue}<|piControlSetBitValue(&sPIValue)|>\tikzmarkend{setBitValue}<|;
		if (rc < 0)
			printf("Set bit error %s\n", getWriteError(rc));
		else if (verbose)
			printf("Set bit %d on byte at offset %d. Value %d\n", sPIValue.i8uBit, sPIValue.i16uAddress,
			       sPIValue.i8uValue);
}
\end{lstlisting}

Der Programmcode in Listing~\ref{lst:4-PI_writeSingleIO} ist Teil des implementierten OPC-Servers. In diesem wird auf zwei Funktionen des piControl-Treibers zugegriffen. 
Beiden Methoden wird als Argument ein Zeiger auf ein Struct vom Typ \lstinline{SPIValue} übergeben. Der im Struct abgelegte Name wird mittels \lstinline{piControlGetVariableInfo(&sPIValue)} zu einer Adresse im Prozessabbild aufgelöst. Diese wird in \lstinline{sPIValue.i16uAdress} gespeichert. Der Wert der Variablen wird anschließend mittels \lstinline{piControlSetBitValue(&sPIValue)} an dieser Adresse in das Prozessabbild geschrieben.

\begin{lstlisting}[language={c},firstnumber=309,caption={Methode \lstinline{piControlSetBitValue} in \lstinline{piControlIf.c}\label{lst:4-piControlSetBitValue}}]
int |>\tikzmarkin[set border color=martiniblue]{setBitValueFcn}<|piControlSetBitValue(SPIValue *pSpiValue)|>\tikzmarkend{setBitValueFcn}<|
{
    piControlOpen();

    if (PiControlHandle_g < 0)
	    return -ENODEV;

    pSpiValue->i16uAddress += pSpiValue->i8uBit / 8;
    pSpiValue->i8uBit %= 8;

    if (|>\tikzmarkin[set border color=martinired]{ioctl}<|ioctl(PiControlHandle_g, KB_SET_VALUE, pSpiValue)|>\tikzmarkend{ioctl}<| < 0)
	    return errno;

    return 0;
}
\end{lstlisting}

Die in Listing~\ref{lst:4-piControlSetBitValue} dargestellte Methode \lstinline{piControlSetBitValue} ist lediglich eine Hüllfunktion (häufig auch als Wrapper-Funktion bezeichnet) für einen Aufruf des \lstinline{ioctl} Kernel-Moduls.
Folgende Parameter werden übergeben:
\lstinline{PiControlHandle_g} ist die Referenz auf die Geräte-Datei des piControl-Treibers. \lstinline{KB_SET_VALUE} ist das ioctl-Kommando zum Schreiben eines Bits in das Prozessabbild. Der Zeiger \lstinline{pSpiValue} verweist auf ein Struct des bereits vorgestellten Typs \lstinline{SPIValue}.

\begin{lstlisting}[language={c},firstnumber=80,caption={Methode \lstinline{piControlOpen} in \lstinline{piControlIf.c}\label{lst:4-piControlOpen}}]
void piControlOpen(void)
{
    /* open handle if needed */
    if (PiControlHandle_g < 0)
    {
	    |>\tikzmarkin[set border color=martiniblue]{PiControlHandle}<|PiControlHandle_g = open(PICONTROL_DEVICE, O_RDWR)|>\tikzmarkend{PiControlHandle}<|;
    }
}
\end{lstlisting}

Die in Listing~\ref{lst:4-piControlOpen} dargestellte Methode öffnet, sofern nicht bereits geschehen, die Geräte-Datei. Das Macro \lstinline{PICONTROL_DEVICE} verweist hierbei auf \lstinline{/dev/piControl0}.

\begin{lstlisting}[language={c},firstnumber=721,caption={Methode \lstinline{piControlIoctl} in \lstinline{piControlMain.c}\label{lst:4-piControlIoctl}}]
static long |>\tikzmarkin[set border color=martiniblue, below offset=0.9em]{piControlIoctl}<|piControlIoctl(struct file *file, unsigned int prg_nr, 
                           unsigned long usr_addr)                                      |>\tikzmarkend{piControlIoctl}<|
{
  int status = -EFAULT;
  tpiControlInst *priv;
  int timeout = 10000;	// ms

  if (prg_nr != KB_CONFIG_SEND && prg_nr != KB_CONFIG_START && !isRunning()) {
  	return -EAGAIN;
  }

  priv = (tpiControlInst *) file->private_data;

  if (prg_nr != KB_GET_LAST_MESSAGE) {
  	// clear old message
  	priv->pcErrorMessage[0] = 0;
  }

  switch (prg_nr) {|>\setcounter{lstnumber}{864}<|

    case |>\tikzmarkin[set border color=martiniblue]{KB_SET_VALUE}<|KB_SET_VALUE:|>\tikzmarkend{KB_SET_VALUE}<|
  		{
  			SPIValue *pValue = (SPIValue *) usr_addr;

  			if (!isRunning())
  				return -EFAULT;

  			if (pValue->i16uAddress >= KB_PI_LEN) {
  				status = -EFAULT;
  			} else {
  				INT8U i8uValue_l;
  				my_rt_mutex_lock(&piDev_g.lockPI);
  				i8uValue_l = piDev_g.ai8uPI[pValue->i16uAddress];

  				if (pValue->i8uBit >= 8) {
  					i8uValue_l = pValue->i8uValue;
  				} else {
  					if (pValue->i8uValue)
  						i8uValue_l |= (1 << pValue->i8uBit);
  					else
  						i8uValue_l &= ~(1 << pValue->i8uBit);
  				}

  				|>\tikzmarkin[set border color=martinired]{i8uValue}<|piDev_g.ai8uPI[pValue->i16uAddress] = i8uValue_l;|>\tikzmarkend{i8uValue}<|
  				rt_mutex_unlock(&piDev_g.lockPI);

  #ifdef VERBOSE
  				pr_info("piControlIoctl Addr=%u, bit=%u: %02x %02x\n", pValue->i16uAddress, pValue->i8uBit, pValue->i8uValue, i8uValue_l);
  #endif

  				status = 0;
  			}
  		}
  		break; |>\setcounter{lstnumber}{1314}<|

    default:
      pr_err("Invalid Ioctl");
      return (-EINVAL);
      break;

    }

    return status;
  }
\end{lstlisting}

Listing~\ref{lst:4-piControlIoctl} zeigt in Auszügen die ioctl-Methode des piControl Kernel-Treibers. Diese bekommt folgende Argumente übergeben: \lstinline{struct file *file} enthält den Verweis auf die Geräte-Datei, hier \lstinline{/dev/piControl0}. Der Wert von \lstinline{unsigned int prg_nr} beschreibt die Anfrage an den Treiber, in diesem Fall \lstinline{KB_SET_VALUE}. Das Argument \lstinline{unsigned long usr_addr} enthält einen typ-agnostischen Pointer. Dieser verweist auf einen Speicherbereich, in welchem die zur Bearbeitung der Anfrage notwendigen Daten abgelegt sind. Hier können auch vom Treiber empfangene Daten dem Anwendungsprogramm bereitgestellt werden. 

Die switch-case-Anweisung führt die über das Argument \lstinline{prg_nr} spezifizierte Aktion aus. Hier betrachten wir \lstinline{KB_SET_VALUE}:
Zunächst wird in Zeile 868 der übergebene Zeiger \lstinline{usr_addr} mittels explizitem Typecast zu einem Zeiger des Typs \lstinline{SPIValue *} konvertiert. Da dieser auf Daten im Userspace verweist, ist beim Zugriff durch den Kernel-Treiber besondere Vorsicht geboten.
In Zeile 877 wird mittels Mutex das Prozessabbild \lstinline{piDev_g} für den Zugriff durch andere Threads oder Prozesse gesperrt.
\lstinline{my_rt_mutex_lock} verweist hierbei auf die Funktion \lstinline{rt_mutex_lock} aus \lstinline{linux/sched.h}\footnote{Offenbar wurde hier auch eine alternative Implementierung vorgesehen, siehe revpi\_common.h}

In Zeile 889 wird das Byte \lstinline{i8uValue_l}, welches den zu schreibenden Wert enthält in das Prozessabbild übertragen. Anschließend wird die Mutex auf \lstinline{piDev_g} wieder entsperrt.
\newpage

\begin{lstlisting}[language={c},firstnumber=62,caption={Auszug des Struct \lstinline{spiControlDev} in \lstinline{piControlMain.h}\label{lst:4-spiControlDev}}]
|>\tikzmarkin[set border color=martiniblue]{spiControlDev}<|typedef struct spiControlDev|>\tikzmarkend{spiControlDev}<| {
	// device driver stuff
	int init_step;
	enum revpi_machine machine_type;
	void *machine;
	struct cdev cdev;	// Char device structure
	struct device *dev;
	struct thermal_zone_device *thermal_zone;

	|>\tikzmarkin[set border color=martiniblue]{processImage}<|// process image stuff
	INT8U ai8uPI[KB_PI_LEN];
	INT8U ai8uPIDefault|>\tikzmarkin[set border color=martinired]{KB_PI_LEN_0}<|[KB_PI_LEN]|>\tikzmarkend{KB_PI_LEN_0}<|;
	struct rt_mutex lockPI;        |>\tikzmarkend{processImage}<|
	bool stopIO;
	piDevices *devs; |>\setcounter{lstnumber}{94}<|
} tpiControlDev;
\end{lstlisting}

Das Prozessabbild ist als Byte-Array der Länge \lstinline{KB_PI_LEN} in Listing~\ref{lst:4-spiControlDev} definiert. Konfigurationsparameter wie \lstinline{KB_PI_LEN} oder die Zykluszeit für den Datenaustausch zwischen SPS und IO-Modulen sind im folgenden Listing~\ref{lst:4-process} definiert.

\begin{lstlisting}[language={c},firstnumber=119,caption={Konfigurationsparameter des Prozessabbildes in project.h\label{lst:4-process}}]
#define INTERVAL_PI_GATE (5*1000*1000)  // 5 ms piGateCommunication |>\setcounter{lstnumber}{128}<|

#define INTERVAL_IO_COM (5*1000*1000)  // 5 ms piIoComm |>\setcounter{lstnumber}{132}<|

#define KB_PD_LEN       512
|>\tikzmarkin[set border color=martiniblue]{KB_PI_LEN_1}<|#define KB_PI_LEN       4096|>\tikzmarkend{KB_PI_LEN_1}<|
\end{lstlisting}

Das zu setzende Bit wurde zu diesem Zeitpunkt erfolgreich in das Prozessabbild der SPS geschrieben.
Es stellt sich die Frage, wie dieses nun an das IO-Modul kommuniziert wird.
Die Kommunikation mit allen angebundenen Modulen ist ebenfalls Aufgabe des piControl-Treibers.

\begin{lstlisting}[language={c},firstnumber=256,caption={Auszug der Methode \lstinline{piIoThread} in \lstinline{revpi_core.c}\label{lst:4-piIoThread}}]
static int piIoThread(void *data)
{
	//TODO int value = 0;
	ktime_t time;
	ktime_t now;
	s64 tDiff;

	hrtimer_init(&piCore_g.ioTimer, CLOCK_MONOTONIC, HRTIMER_MODE_ABS);
	piCore_g.ioTimer.function = piIoTimer;

	pr_info("piIO thread started\n");

	now = hrtimer_cb_get_time(&piCore_g.ioTimer);

	PiBridgeMaster_Reset();

	while (!kthread_should_stop()) {
		if (|>\tikzmarkin[set border color=martinired]{PiBridgeMaster}<|PiBridgeMaster_Run()|>\tikzmarkend{PiBridgeMaster}<| < 0)
			break;
	}

	RevPiDevice_finish();

	pr_info("piIO exit\n");
	return 0;
}
\end{lstlisting}

Der Kernel-Thread \lstinline{piIoThread} ist verantwortlich für den zyklischen Datenaustausch mit den IO-Modulen. In diesem wird fortlaufend die Methode \lstinline{PiBridgeMaster_Run()} aufgerufen, siehe Listing~\ref{lst:4-piIoThread}.

\begin{lstlisting}[language={c},firstnumber=262,caption={Auszug der Methode \lstinline{PiBridgeMaster_Run(void)} in \lstinline{RevPiDevice.c}\label{lst:4-PiBridgeMaster_Run}}]
int PiBridgeMaster_Run(void)
{
	static kbUT_Timer tTimeoutTimer_s;
	static kbUT_Timer tConfigTimeoutTimer_s;
	static int error_cnt;
	static INT8U last_led;
	static unsigned long last_update;
	int ret = 0;
	int i;

	my_rt_mutex_lock(&piCore_g.lockBridgeState);
	if (piCore_g.eBridgeState != piBridgeStop) {
		switch (eRunStatus_s) { |>\setcounter{lstnumber}{514}<|
		    case enPiBridgeMasterStatus_EndOfConfig:|>\setcounter{lstnumber}{621}<|
		    if (|>\tikzmarkin[set border color=martinired]{RevPiDevice}<|RevPiDevice_run()|>\tikzmarkend{RevPiDevice}<|) {
				// an error occured, check error limits |>\setcounter{lstnumber}{641}<|
			} else {
				ret = 1;
			}
			piCore_g.image.drv.i16uRS485ErrorCnt = RevPiDevice_getErrCnt();
			break;
\end{lstlisting}

Die in Listing~\ref{lst:4-PiBridgeMaster_Run} dargestellte Methode ist eine sog. State-Machine. Ist die Konfiguration der IO-Module erfolgreich abgeschlossen, so führt sie bei Aufruf lediglich die Methode \lstinline{RevPiDevice_run()} aus.

\begin{lstlisting}[language={c},firstnumber=140,caption={Auszug der Methode \lstinline{RevPiDevice_run(void)} in \lstinline{RevPiDevice.c}\label{lst:4-RevPiDevice_run}}]
int RevPiDevice_run(void)
{
	INT8U i8uDevice = 0;
	INT32U r;
	int retval = 0;

	RevPiDevices_s.i16uErrorCnt = 0;

	for (i8uDevice = 0; i8uDevice < RevPiDevice_getDevCnt(); i8uDevice++) {
		if (RevPiDevice_getDev(i8uDevice)->i8uActive) {
			switch (RevPiDevice_getDev(i8uDevice)->sId.i16uModulType) {
			case KUNBUS_FW_DESCR_TYP_PI_DIO_14:
			case KUNBUS_FW_DESCR_TYP_PI_DI_16:
			case KUNBUS_FW_DESCR_TYP_PI_DO_16:
				r = |>\tikzmarkin[set border color=martinired]{sendCyclicTelegram}<|piDIOComm_sendCyclicTelegram(i8uDevice)|>\tikzmarkend{sendCyclicTelegram}\setcounter{lstnumber}{166} <|;

				break; |>\setcounter{lstnumber}{216}<|
			}
		}
	} |>\setcounter{lstnumber}{227}<|
	return retval;
}
\end{lstlisting}

Diese iteriert wie in Listing~\ref{lst:4-RevPiDevice_run} abgebildete durch alle gegenwärtig in der SPS konfigurierten Module. Ist das aktuelle Modul als aktiv markiert, so wird anhand eines sog. Firmware-Descriptors entschieden, welche Methode für die Ansteuerung des Moduls aufzurufen ist.

\begin{lstlisting}[language={c},firstnumber=161,caption={Auszug der Methode \lstinline{piDIOComm_sendCyclicTelegram} in \lstinline{piDIOComm.c}\label{lst:4-sendCyclicTelegram}}]
INT32U piDIOComm_sendCyclicTelegram(INT8U i8uDevice_p)
{
	INT32U i32uRv_l = 0;
	SIOGeneric sRequest_l;
	SIOGeneric sResponse_l;
	INT8U len_l, data_out[18], i, p, data_in[70];
	INT8U i8uAddress;
	int ret; |>\setcounter{lstnumber}{239}<|
	
    |>\tikzmarkin[set border color=martinired]{piIoComm}<|ret = piIoComm_send((INT8U *) & sRequest_l, IOPROTOCOL_HEADER_LENGTH + len_l + 1);  |>\tikzmarkend{piIoComm}\setcounter{lstnumber}{298}<|
}
\end{lstlisting}

Im Falle des hier verwendeten DO-Moduls wird die in Listing~\ref{lst:4-sendCyclicTelegram} abgebildete Methode \lstinline{piDIOComm_sendCyclicTelegram()} aufgerufen. Dieser wird ein Zeiger auf das zu schreibende Gerät übergeben. 
Zunächst wird das Prozessabbild mittels eines proprietären, jedoch im Quellcode offen nachvollziehbaren Protokolls in ein \lstinline{sRequest_l} genanntes Byte-Array umgewandelt. Dieser Schritt ist in Listing~\ref{lst:4-sendCyclicTelegram} nicht abgebildet. Anschließend wird \lstinline{piIoComm_send()} ein Zeiger auf die so generierte Schreib-Anfrage übergeben.

\begin{lstlisting}[language={c},firstnumber=220,caption={Auszug der Methode \lstinline{piIOComm_send} in \lstinline{piIOComm.c}\label{lst:4-piIOComm_send}}]
int piIoComm_send(INT8U * buf_p, INT16U i16uLen_p)
{
	ssize_t write_l = 0;
	INT16U i16uSent_l = 0;|>\setcounter{lstnumber}{249}<|

	while (i16uSent_l < i16uLen_p) {
		write_l = vfs_write(piIoComm_fd_m, buf_p + i16uSent_l, i16uLen_p - i16uSent_l, &piIoComm_fd_m->f_pos);
		if (write_l < 0) {
			pr_info_serial("write error %d\n", (int)write_l);
			return -1;
		} 
		i16uSent_l += write_l;|>\setcounter{lstnumber}{263}<|
	}
	clear();
	vfs_fsync(piIoComm_fd_m, 1);
	return 0;
}
\end{lstlisting}

Listing~\ref{lst:4-piIOComm_send} zeigt die Implementierung von \lstinline{piIoComm_send()}. Diese Methode ist für das Schreiben der oben generierten Anfrage auf die seriellen Schnittstelle verantwortlich. Realisiert wird dies mittels der Methode \lstinline{vfs_write()}. Diese ist in \lstinline{<linux/fs.h>} definiert. Sie ermöglicht das Schreiben einer Datei im Userspace aus dem Kernel heraus. Geschrieben wird hier die Datei mit dem Deskriptor \lstinline{piIoComm_fd_m}.
Da die Funktion \lstinline{vfs_write()} durch andere Kernel-Tasks unterbrochen werden kann, ist nicht gewährleistet, dass die gesamte Anfrage mit nur einem Aufruf geschrieben wird. Die oben abgebildete while-Schleife stellt das vollständige Senden der Anfrage sicher.

\begin{lstlisting}[language={c},firstnumber=157,caption={Auszug der Methode \lstinline{piIOComm_open_serial} in \lstinline{piIOComm.c}\label{lst:4-piIOComm_open_serial}}]
int piIoComm_open_serial(void)
{   |>\setcounter{lstnumber}{167}<|
	struct file *fd;	/* Filedeskriptor */
	struct termios newtio;	/* Schnittstellenoptionen */

	|>\tikzmarkin[set border color=martiniblue]{fd}<|/* Port oeffnen - read/write, kein "controlling tty", 
	    Status von DCD ignorieren */
	fd = filp_open(|>\tikzmarkin[set border color=martinired]{tty}<|REV_PI_TTY_DEVICE|>\tikzmarkend{tty}<|, O_RDWR | O_NOCTTY, 0); |>\setcounter{lstnumber}{208}<|
	
	piIoComm_fd_m = fd;                                                      |>\tikzmarkend{fd}\setcounter{lstnumber}{217}<|

	return 0;
}
\end{lstlisting}

Der zum Schreiben auf die serielle Schnittstelle verwendete Datei-Deskriptor wird von der in Listing~\ref{lst:4-piIOComm_open_serial} abgebildeten Methode \lstinline{piIoComm_open_serial()} generiert. 

\begin{lstlisting}[language={c},firstnumber=45,caption={Definition der seriellen Schnittstelle in \lstinline{piIOComm.h}\label{lst:4-REV_PI_TTY_DEVICE}}]
#define REV_PI_TTY_DEVICE	"/dev/ttyAMA0"
\end{lstlisting}

Das in Listing~\ref{lst:4-REV_PI_TTY_DEVICE} definierte Macro verweist auf eine der seriellen Schnittstellen des RaspberryPi.
Die Implementierung des zugehörigen Schnittstellentreibers soll hier nicht weiter untersucht werden. Somit ist an dieser Stelle die Kette vom Setzen einer Variablen auf dem OPC-Server bis hin zur Aktualisierung des Prozessabbilds der IO-Module geschlossen.

% \begin{lstlisting}[language={c},firstnumber={226},caption={Setzen der Scheduler-Priorität auf SCHED\_FIFO in 
% revpi\_common.c\label{lst:2-sched_priority}}]
% param.sched_priority = ktprio->prio;
% ret = sched_setscheduler(child, SCHED_FIFO, &param);
% \end{lstlisting}
%% % Imports nur für Referenzenauflösung während des Schreibens! Vorm Kompilieren auskommentieren!
% \bibliography{0_hauptdatei}
% % Mit \section{...} eröffnen wir einen neuen Abschnitt.
% Der Befehl setzt nicht nur den Text in einer größeren,
% fetten Schrift, sondern sorgt außerdem dafür, daß er im
% Inhaltsverzeichnis erscheint.
%
% Mit \label{...} erzeugen wir einen Bezeichner, mit dessen Hilfe
% wir später auf die Nummer des Abschnitts verweisen können (nämlich
% mit~\ref{...}).
%
% Das Kommentarzeichen hinter „Übersicht“ dient dazu, ein
% Leerzeichen zwischen „Übersicht“ und dem \label-Befehl
% zu vermeiden, das andernfalls sichtbar würde – z.B. im
% Inhaltsverzeichnis.
%

% % Imports nur für Referenzenauflösung während des Schreibens! Vorm Kompilieren auskommentieren!
% \bibliography{0_hauptdatei}
% \input{1_einleitung}
%\input{2_grundlagen}
%\input{3_konzeption}
%\input{4_implementierung}
%\input{5_tests}
%\input{6_zusammenfassung}
% % Ende Imports

\section{Einleitung und Motivation%
  \label{sec:1-einleitung}}
Ziel dieses Projektes ist die Integration eines OPC-Servers mit einer auf Linux
basierenden speicherprogrammierbaren Steuerung (SPS). Angeschlossen an diese SPS
ist jeweils ein digitales Ein-/\,bzw.~Ausgabemodul. Die von diesen bereitgestellten
Ein-/\, bzw.~Ausgänge (IO) sollen in der Datenstruktur des OPC-Servers abgebildet
und über diesen für OPC-Clients les-/\,und schreibar sein. Weiterhin sollen einige
Funktionen zur Überwachung und Steuerung der an die SPS angeschlossenen Aktoren
und Sensoren direkt im OPC-Server implementiert werden.
Hiermit stellt dieses Projekt eine der Grundlagen für ein übergeordnetes Projekt,
die cloudbasierte Steuerung eines miniaturisierten Produktions-Systems, dar.

Der hier verwendete OPC-Server ist Teil des sog. open62541 Projekts. Er ist in C
geschrieben und implementiert bereits einen großen Teil der im OPC-UA-Standard
spezifizierten Funktionen.
Als SPS findet ein Revolution Pi 3 der Firma Kunbus Verwendung. Dieser integriert
ein sog. Compute Module der Raspberry Pi Foundation in ein industrietaugliches
Gehäuse und erlaubt die Erweiterung mittels IO- oder Gateway-Modulen. Über diese
erfolgt die Kommunikation mit weiteren Komponenten der Automatisierungstechnik.

Motiviert ist dieses Projekt durch die Beobachtung, dass die Verbreitung offener
Standards sowie freier Software auch in der Automatisierungstechnik zunimmt.
Linux ist ein freies Betriebssystem, OPC-UA ein offen zugänglicher, aktiv gepflegter
und weit verbreiteter Standard. Der Raspberry Pi findet sowohl bei Hobby-Anwendern als
auch in den Bereichen Forschung und Entwicklung sowie bei industriellen Anwendern
Verwendung. Dieses Projekt stellt somit eine für unterschiedliche Anwender interessante
Entwicklung dar.

Im Anschluss an diese einleitende Übersicht im Abschnitt~\ref{sec:1-einleitung} folgt
die Darstellung der wichtigsten Grundlagen in Abschnitt~\ref{sec:2-grundlagen}.
Aufbauend auf diesen Grundlagen folgt die konzeptuelle Ausarbeitung im Abschnitt~\ref{sec:3-konzeption}.
Die Umsetzung wird im Abschnitt~\ref{sec:4-implementierung} erläutert.
Die Leistungsfähigkeit der Implementierung wird in Abschnitt~\ref{sec:5-tests} untersucht.
Eine Zusammenfassung und ein Ausblick schließen die Arbeit in
Abschnitt~\ref{sec:6-fazit} ab. Eventuell noch benötigte Anhänge
finden sich in den Anhängen [...] bis [...].

% % % Imports nur für Referenzenauflösung während des Schreibens! Vorm Kompilieren auskommentieren!
% \bibliography{0_hauptdatei}
% \input{1_einleitung}
% \input{2_grundlagen}
% \input{3_konzeption}
% \input{4_implementierung}
% \input{5_tests}
% \input{6_zusammenfassung}
% % Ende Imports

\section{Grundlagen%
  \label{sec:2-grundlagen}}

\subsection{Speicherprogrammierbare-Steuerung und Linux -- Revolution Pi%
     \label{sec:2-sps}}

\subsubsection{Kunbus RevolutionPi%
        \label{sec:2-revpi}}
Der RevolutionPi 3 ist eine speicherprogrammierbare Steuerung (SPS) des Herstellers
Kunbus GmbH. Kern dieser SPS ist das von der Raspberry Pi Foundation entwickelte
und vertriebene Raspberry Pi Compute Module 3. Dieses integriert ein Broadcom BCM2837
System-on-Chip (SoC) mit vier 1,2GHz Prozessorkernen, 1GB RAM, 4GB eMMC Anwendungsspeicher
und sonstige Peripherie in ein Modul im DDR2-SODIMM Formfaktor. Diese Spezifikationen
sind weitgehend identisch zu denen des ausgesprochen populären Raspberry Pi 3.
Der Revolution Pi profitiert daher von dem gleichen großen Angebot an Software
und Unterstützung wie der Raspberry Pi, ergänzt dessen Hardware jedoch um eine 24V
Spannungsversorgung, die Möglichkeit der Erweiterung durch mehrere industrietaugliche
Ein-/ Ausgabemodule und Gateways sowie ein Gehäuse zur Montage auf einer DIN-Schiene.
\begin{itemize}
  \item{Prozessor: BCM2837}
  \item{Taktfrequenz 1,2 GHz}
  \item{Anzahl Prozessorkerne: 4}
  \item{Arbeitsspeicher: 1 GByte}
  \item{eMMC Flash Speicher: 4 GByte}
  \item{Betriebssystem: Angepasstes Raspbian mit RT-Patch}
  \item{RTC mit 24h Pufferung über wartungsfreien Kondensator}
  \item{Treiber / API: Treiber schreibt zyklisch Prozessdaten in ein Prozessabbild, Zugriff auf Prozessabbild über Linux-Filesystem als API zu Fremdsoftware.}
  \item{Kommunikationsanschlüsse: 2 x USB 2.0 A (je 500 mA belastbar), 1 x Micro-USB, HDMI, Ethernet (RJ45) 10/100 Mbit/s}
  \item{Stromversorgung: min. 10,7 V, max. 28,8 V, maximal 10 Watt}
  \item{Zulässige Umgebungstemperatur: -40 bis +55 C}
  \item{Gehäuseabmessungen: (HxBxL) 96 mm x 22,5 mm x 110,5 mm (ohne gesteckte Stecker)}
  \item{ESD Schutz: 4 kV / 8 kV gemäß EN61131-2 und IEC 61000-6-2}
  \item{Surge / Burst Prüfungen: gemäß EN61131-2 und IEC 61000-6-2 eingekoppelt auf Versorgungsspannung, Ethernet und IO-Leitungen}
  \item{EMI Prüfungen: gemäß EN61131-2 und IEC 61000-6-2}
\end{itemize}

Kunbus bietet eine Auswahl an IO- und Gateway-Modulen zur Erweiterung des Revolution Pi an.
Gateways dienen der Kommunikation mit Systemen oder Komponenten der Automatisierungstechnik
über Protokolle wie PROFIBUS oder EtherCAT. IO-Module erlauben die Überwachung
und Steuerung von digitalen oder analogen Ein- und Ausgängen.

\subsubsection{Zugriff auf IO-Module%
        \label{sec:2-io}}
Der Zugriff auf die Ein- und Ausgänge der IO-Module erfolgt über ein Prozessabbild
und einen hierfür von Kunbus bereitgestellten Treiber, genannt piControl. Dieser
aktualisiert das Prozessabbild zyklisch. Die angestrebte Zykluszeit beträgt 5ms,
kann jedoch je nach Anzahl der angeschlossenen Module auch größer sein. Kunbus
garantiert bei drei IO-Modulen und zwei Gateway-Modulen eine Zykluszeit von 10 ms.
Jedes der IO-Module stellt ein eigenständiges eingebettetes System dar. Es verfügt
über einen Microcontroller, welcher die IOs bereitstellt und über einen RS485-Bus
mit dem Revolution Pi kommuniziert.
% https://revolution.kunbus.de/io-modul/

Lizenz: GPL
% https://github.com/RevolutionPi/piControl

\begin{lstlisting}[language={c},firstnumber={226},caption={Setzen der Scheduler-Priorität auf SCHED\_FIFO in revpi\_common.c\label{lst:2-sched_priority}}]
param.sched_priority = ktprio->prio;
ret = sched_setscheduler(child, SCHED_FIFO,
       &param);
\end{lstlisting}


\subsection{Echtzeit und Multithreading unter Linux -- preemptRT und posix%
     \label{sec:2-echtzeit}}


 Der Linux-Kernel verfügt über mehrere unterschiedliche Preemtion-Modelle:

\begin{itemize}
  \item No Forced Preemption (server):
  Ausgelegt auf maximal möglichen Durchsatz, lediglich Interrupts und
  System-Call-Returns bewirken Präemption.

  \item Voluntary Kernel Preemption (Desktop):
  Neben den implizit bevorrechtigten Interrupts und System-Call-Returns gibt es
  in diesem Modell weitere Abschnitte des Kernels in welchen Preämption explizit
  gestattet ist.

  \item Preemptible Kernel (Low-Latency Desktop):
  In diesem Modell ist der gesamte Kernel, mit Ausnahme sog.~kritischer Abschnitte
  präemptible. Nach jedem kritischen Abschnitt gibt es einen impliziten Präemptions-Punkt.

  \item Preemptible Kernel (Basic RT):
  Dieses Modell ist dem zuvor genannten sehr ähnlich, hier sind jedoch alle Interrupt-Handler
  als eigenständige Threads ausgeführt.

  \item Fully Preemptible Kernel (RT):
  Wie auch bei den beiden zuvor genannten Modellen ist hier der gesamte Kernel
  präemtible, die Anzahl und Dauer der nicht-präemtiblen kritischen Abschnitte
  ist auf ein notwendiges Minimum beschränkt. Alle Interrupt-Handler sind als
  eigenständige Threads ausgeführt, Spinlocks durch Sleeping-Spinlocks und Mutexe
  durch sog.~RT-Mutexe ersetzt.

\end{itemize}
\todo{Spinlocks und Mutexe sowie die RT-Varianten dieser erklären!}

Lediglich mit dem vollständig präemtiblen Kernel kann Echtzeit-Verhalten realisiert werden.

% https://wiki.linuxfoundation.org/realtime/documentation/technical_basics/preemption_models bzw kernel/Kconfig.preempt

\subsubsection{preemptRT%
        \label{sec:2-preemptRT}}
% https://wiki.linuxfoundation.org/realtime/documentation/technical_details/start
% https://wiki.linuxfoundation.org/realtime/documentation/technical_basics/start

Das dem PREEMPT RT Kernel zugrunde liegende Prinzip lässt sich in einer einfachen
Regel ausdrücken: Nur Code, welcher absolut nicht-präemtible sein darf, ist es
gestattet nicht-präemtible zu sein.
Das erklärte Ziel des PREEMPT\_RT Patches ist es folglich, die Menge des nicht-präemtiblen
Codes im Linux-Kernel auf das absolut notwendige Minimum zu reduzieren.

Dies wird durch Verwendung folgender Mechanismen erreicht:

\begin{itemize}
  \item Hochauflösende Timer
  \item Sleeping Spinlocks
  \item Threaded Interrupt Handlers
  \item rt\_mutex
  \item RCU
\end{itemize}


\subsubsection{posix%
        \label{sec:2-posix}}
Ist posix hier wirklich relevant? Debian bzw.~Raspbian sind weitgehend posix
kompatibel, aber wird es hier genutzt? -> JA, open62541 nutzt pthread.h
piControl nutzt kthread.h, und semaphore.h

\subsection{OPC-UA und open62541%
     \label{sec:2-opc}}

\subsubsection{OPC UA%
        \label{sec:2-opcua}}
Open Platform Communications (OPC) ist eine Familie von Standards zur herstellerunabhängigen
Kommunikation von Maschinen (M2M) in der Automatisierungstechnik. Die sog.~OPC Task Force, zu deren
Mitgliedern verschiedene große Firmen der Automatisierungsindustrie gehören, veröffentlichte
die OPC Specification Version 1.0 im August 1996.
Motiviert ist dieser offene Standard durch die Erkenntniss, dass die Anpassung der
zahlreichen Herstellerstandards an individuelle Infrastrukturen und Anlagen einen
großen Mehraufwand verursachen.
Die Wikipedia beschreibt das Anwendungsgebiet für OPC wie folgt:

\glqq{}OPC wird dort eingesetzt, wo Sensoren, Regler und Steuerungen verschiedener Hersteller
ein gemeinsames Netzwerk bilden. Ohne OPC benötigten zwei Geräte zum Datenaustausch
genaue Kenntnis über die Kommunikationsmöglichkeiten des Gegenübers. Erweiterungen
und Austausch gestalten sich entsprechend schwierig. Mit OPC genügt es, für jedes
Gerät genau einmal einen OPC-konformen Treiber zu schreiben. Idealerweise wird
dieser bereits vom Hersteller zur Verfügung gestellt. Ein OPC-Treiber lässt sich
ohne großen Anpassungsaufwand in beliebig große Steuer- und Überwachungssysteme
integrieren.

OPC unterteilt sich in verschiedene Unterstandards, die für den jeweiligen Anwendungsfall
unabhängig voneinander implementiert werden können. OPC lässt sich damit verwenden
für Echtzeitdaten (Überwachung), Datenarchivierung, Alarm-Meldungen und neuerdings
auch direkt zur Steuerung (Befehlsübermittlung).\grqq{}

OPC basiert in der ursprünglichen Spezifikation auf Microsofts DCOM-Spezifikation.
DCOM macht Funktionen und Objekte einer Anwendung anderen Anwendungen im Netzwerk
zugänglich. Der OPC-Standard definiert entsprechende DCOM-Objekte um mit anderen
OPC-Anwendungen Daten austauschen zu können. Die Verwendung von DCOM bindet Anwender
an Betriebssysteme von Microsoft. Die ursprüngliche OPC Spezifikation wird durch die
Entwicklung von OPC Unified Architecture (OPC UA) abgelöst.
OPC UA setzt auf einem eigenen Kommunikationionsstack auf, die Verwendung von DCOM
und damit die Bindung an Microsoft wurden aufgelöst.

Die OPC-UA-Architektur ist eine Service-orientierte Architektur (SOA), deren Struktur
aus mehreren Schichten besteht.

% Wikipedia
Das OPC-Informationsmodell ist nicht mehr nur eine Hierarchie aus Ordnern, Items
und Properties. Es ist ein sogenanntes Full-Mesh-Network aus Nodes, mit dem neben
den Nutzdaten eines Nodes auch Meta- und Diagnoseinformationen repräsentiert werden.
Ein Node ähnelt einem Objekt aus der objektorientierten Programmierung. Ein Node
kann Attribute besitzen, die gelesen werden können (Data Access (DA), Historical
Data Access (HDA)). Es ist möglich Methoden zu definieren und aufzurufen.
Eine Methode besitzt Aufrufargumente und Rückgabewerte. Sie wird durch ein Command
aufgerufen. Weiterhin werden Events unterstützt, die versendet werden können
(AE (Alarms \& Events), DA DataChange), um bestimmte Informationen zwischen Geräten
auszutauschen. Ein Event besitzt unter anderem einen Empfangszeitpunkt, eine Nachricht
und einen Schweregrad. Die o. g. Nodes werden sowohl für die Nutzdaten als auch
alle anderen Arten von Metadaten verwendet. Der damit modellierte OPC-Adressraum
beinhaltet nun auch ein Typmodell, mit dem sämtliche Datentypen spezifiziert werden.

% https://de.wikipedia.org/wiki/Open_Platform_Communications
% https://de.wikipedia.org/wiki/OPC_Unified_Architecture
% https://opcfoundation.org/developer-tools/specifications-unified-architecture
% Von Gerhard Gappmeier - ascolab GmbH, CC BY-SA 3.0, https://de.wikipedia.org/w/index.php?curid=1892069
\subsubsection{open62541%
        \label{sec:2-open62541}}
open62541 ist eine offene und freie Implementierung von OPC UA. Die in C geschriebene
Bibliothek stellt eine beständig zunehmende Anzahl der im OPC UA Standard definierten
Funktionen bereit. Sie kann sowohl zur Erstellung von OPC-Servern als auch -Clients
genutzt werden. Ergänzend zu der unter der Mozilla Public License v2.0 lizensierten
Bibliothek stellt das open62541 Projekt auch Beispielprogramme unter einer CC0 Lizenz
zur Verfügung.

Die Bibliothek eignet sich auch für die Entwicklung auf eingebetteten Systemen und
Microcontrollern. Je nach Umfang der gewünschten Funktionen und des OPC Informationsmodells
beträgt die Größe einer Server-Binary weniger als 100kb. %evtl. kürzen?

\todo{Nodes erklären! Evtl.~oben!}

Folgende Auswahl an Eigenschaften und Funktionen zeichnet die in dieser Arbeit verwendete
Version 0.3 von open62541 aus:
\begin{itemize}
  \item Kommunikationionsstack
  \begin{itemize}
      \item OPC UA Binär-Protokoll (HTTP oder SOAP werden gegenwärtig nicht unterstützt)
      \item Austauschbare Netzwerk-Schicht, welche die Verwendung eigener Netzwerk-APIs
      erlaubt.
      \item Verschlüsselte Kommunikationion
      \item Asynchrone Dienst-Anfragen im Client
  \end{itemize}
  \item Informationsmodell
  \begin{itemize}
    \item Unterstützung aller OPC UA Node-Typen, inkl.~Methoden
    \item Hinzufügen und Entfernen von Nodes und Referenzen zur Laufzeit.
    \item Vererbung und Instanziierung von Objekt- und Variablentypen
    \item Zugriffskontrolle auch für einzelne Nodes
  \end{itemize}
  \item Subscriptions
  \begin{itemize}
    \item Erlaubt die Überwachung (subscriptions / monitoreditems)
    \item Sehr geringer Ressourcenbedarf pro überwachtem Wert
  \end{itemize}
  \item Code-Generierung auf XML-Basis
  \begin{itemize}
    \item Erlaubt die Erstellung von Datentypen
    \item Erlaubt die Generierung des serverseitigen Informationsmodells
  \end{itemize}
\end{itemize}

% https://open62541.org/doc/0.3/


Mozilla Public License
CC0 Lizenz für Beispiele und Plugins

% https://open62541.org/doc/open62541-current.pdf
% https://open62541.org/

% % % Imports nur für Referenzenauflösung während des Schreibens! Vorm Kompilieren auskommentieren!
% \bibliography{0_hauptdatei}
% \input{1_einleitung}
% \input{2_grundlagen}
% \input{3_konzeption}
% \input{4_implementierung}
% \input{5_tests}
% \input{6_zusammenfassung}
% \input{anhang}
% % Ende Imports

\section{Systemkonzept%
  \label{sec:3-konzeption}}
Auf Basis der in Abschnitt \ref{sec:2-grundlagen} vorgestellten Möglichkeiten folgt nun die Ausarbeitung eines Konzepts.
In den folgenden Abschnitten soll näher auf zwei zentrale Aspekte eingegangen werden: Abschnitt~\ref{sec:3-anbindung} stellt Möglichkeiten zum Zugriff auf Variablen bzw.\,Werte im Prozessabbild des Revolution Pi vor; in Abschnitt~\ref{sec:3-integration} wird ein Konzept zur Bereitstellung dieser Variablen auf einem OPC-Server vorgestellt.

\subsection{Anbindung der IO an den OPC-Server%
     \label{sec:3-anbindung}}

Eine Webanwendung mit Bezeichnung PiCtory dient zur Konfiguration der I/O- und virtuellen Module des RevolutionPi. Die Konfiguration liegt im JSON-Format in der Datei \lstinline{/etc/revpi/config.rsc}. Der piControl-Treiber liest diese Datei beim Start. 
Der folgende Auszug aus der Manpage des piControl-Kernelmoduls beschreibt die von diesem zum Lesen und Schreiben einzelner Bits des Prozessabbildes bereitgestellten Funktionen~\citep[vgl.]{web-revpi-manpage}. Sie ist an dieser Stelle weitgehend ungekürzt zitiert, da sie die nutzbare Schnittstelle sehr kompakt beschreibt.

\begin{lstlisting}[breakindent=0pt, numbers=none, caption={Auszug aus der Revolution Pi Programmers Manual\label{lst:4-manpage}}]
KB_FIND_VARIABLE SPIVariable *argp
Find a variable in the process image by its name. A pointer to a structure of type SPIVariable must be passed as argument. [...]
The struct SPIVariable [...] is defined as 
typedef struct SPIVariableStr
{
    char strVarName[32]; // Variable name
    uint16_t i16uAddress; // Address of the byte in the process image
    uint8_t i8uBit; // 0-7 bit position, >= 8 whole byte
    uint16_t i16uLength; // length of the variable in bits.
    // Possible values are 1, 8, 16 and 32
} SPIVariable;

Set and get values of the process image
KB_GET_VALUE SPIValue *argp
[...]
KB_SET_VALUE SPIValue *argp
Write one bit or one byte to the process image [...].  This call is more efficient than the usual calls of seek and write because only one function call is necessary. If more than on application are writing bits in one output byte, this call is the only safe way to set a bit without overwriting the other bits because this call is doing a read-modify-write-cycle. 

The struct SPIValue used by this ioctl is defined as
typedef struct SPIValueStr
{
    uint16_t i16uAddress; // Address of the byte in the process image
    uint8_t i8uBit; // 0-7 bit position, >= 8 whole byte
    uint8_t i8uValue; // Value: 0/1 for bit access, whole byte otherwise
} SPIValue;
\end{lstlisting} 

Die oben beschriebenden Funtkionen \lstinline{KB_FIND_VARIABLE}, \lstinline{KB_GET_VALUE} und \lstinline{KB_SET_VALUE} ermöglichen einen einfachen und (lt.\,Manpage) effizienten Zugriff auf einzelne Bits des Prozessabbildes und damit auch auf die IO des RevolutionPi.
Der Zugriff des OPC-Servers auf das Prozessabbild soll daher mittels dieser Funktionen realisiert werden.
\lstinline{KB_FIND_VARIABLE} kann genutzt werden, um Adressen von Variablen im Prozessabbild mittels ihres Namens aufzulösen.
\lstinline{KB_GET_VALUE} und \lstinline{KB_SET_VALUE} ermöglichen den Zugriff auf die Werte dieser Variablen.


\subsection{Integration des OPC-Servers in das System%
     \label{sec:3-integration}}

open62541 bietet drei Möglichkeiten zum Abgleich von Variablen mit dem Prozessabbild~\citep[vgl.][Tutorials - Connecting a Variable with a Physical Process]{web-open62541}:
\begin{itemize}
    \item Manuelles oder zyklisches Aktualisieren
    \item Variable Value Callback
    \item Variable Datasource
\end{itemize}

Die zyklische Aktualisierung eines oder mehrerer Werte nimmt, abhängig von der Zykluszeit, viele Systemressourcen in Anspruch. Value Callbacks ermöglichen es, einen Variablenwert effizienter mit einer Ressource wie etwa einem Prozessabbild zu synchronisieren. An die Variable wird ein Callback angehängt, welches vor jedem Lesen und nach jedem Schreibvorgang ausgeführt wird.
Der Wert der Variablen wird weiterhin im Variablenknoten auf dem OPC-Server gespeichert, der Abgleich mit der verknüpften Ressource erfolgt durch die Callback-Methoden.

Sogenannte Datenquellen gehen noch einen Schritt weiter. Der Server leitet jede Lese- und Schreibanforderung direkt an eine Callback-Funktion weiter. Beim Lesen liefert der Rückruf eine Kopie des aktuellen Wertes. Die Datenquelle muss intern ein eigenes Speichermanagement implementieren.

Der Zugriff auf die Werte des Prozessabbildes erfolgt, wie in Abschnitt~\ref{sec:3-anbindung} beschrieben, über von piControl bereitgestellte Methoden. Um die durch open62541 gepflegte OPC-Datenstruktur und das durch piControl verwaltete Prozessabbild möglichst effektiv verknüpfen zu können, soll diese Interaktion mittels Datenquellen und den zugehörigen Callbacks implementiert werden.
% % % Imports nur für Referenzenauflösung während des Schreibens! Vorm Kompilieren auskommentieren!
% \bibliography{0_hauptdatei}
% \input{1_einleitung}
% \input{2_grundlagen}
% \input{3_konzeption}
% \input{4_implementierung}
% \input{5_tests}
% \input{6_zusammenfassung}
% \input{anhang}
% % Ende Imports

\section{Implementierung%
  \label{sec:4-implementierung}}
Das folgende Kapitel stellt in Auszügen die Implementierung des OPC-Servers sowie die Anbindung an die IO-Module
der SPS dar. Der Schwerpunkt liegt hierbei auf der Funktionsweise des piControl-Treibers und dessen Integration in das Projekt. Abschnitt~\ref{sec:4-picontrol} erklärt die zum Schreibens eines Bits verwendeten Funktionsaufrufe.
Zuvor soll jedoch in Abschnitt~\ref{sec:4-open62541} der Teil des OPC-Servers vorgestellt werden, welcher auf besagten Treiber zugreift. 

\subsection{Implementierung des OPC-Servers%
     \label{sec:4-open62541}}
Wie im vorangegangenen Abschnitt~\ref{sec:3-integration} begründet, soll die Verknüpfung zwischen dem Prozessabbild der SPS und den auf dem OPC-Server bereitgestellten Werten über sog.\,Datenquellen erfolgen. Hierzu ist zunächst eine Callback-Methode zu implementieren, welche bei einem Lese- oder Schreibzugriff auf eine Variable aufgerufen wird. Die Verknüpfung zwischen Callback-Methode und Variable muss manuell erfolgen.

\begin{lstlisting}[language={c},firstnumber=237,caption={Auszug der Methode \lstinline{linkDataSourceVariable} in \lstinline{variables.c}\label{lst:4-linkDataSourceVariable}}]
extern UA_StatusCode
 linkDataSourceVariable(UA_Server *server, UA_NodeId nodeId) {
     bool readonly = false;
     UA_DataSource dataSourceVariable;
     UA_StatusCode rc; |>\setcounter{lstnumber}{254}<|

     dataSourceVariable.read = readDataSourceVariable;
     if (!readonly)
        dataSourceVariable.write = writeDataSourceVariable;
     else
        dataSourceVariable.write = writeReadonlyDataSourceVariable;

     return UA_Server_setVariableNode_dataSource(server, nodeId, dataSourceVariable);
 }
\end{lstlisting}

\begin{figure}[h]
    \centering
    \includegraphics[width=0.42\textwidth]{doc/img/OPC_RevPiDO.pdf}
    \caption{Auszug des verwendeten Nodesets, hier Digitalausgang 1 des Versuchsaufbaus
      \label{fig:opc-do}}
\end{figure}

Die in Listing~\ref{lst:4-linkDataSourceVariable} abgebildete Methode \lstinline{linkDataSourceVariable()} erzeugt ein Struct vom Typ \lstinline{UA_DataSource}. In diesem werden dem Lesen und Schreiben einer OPC-Variablen entsprechende Callback-Methoden zugewiesen. Die Verknüpfung einer OPC-Variable, genauer ihrer NodeId, mit der zuvor definierten Datenquelle erfolgt über die von open62541 bereitgestellte Methode \lstinline{UA_Server_setVariableNode_dataSource()}. Vor dem Lesen und nach dem Schreiben dieser Variable werden von nun an die entsprechenden Callbacks aufgerufen.
     
\begin{lstlisting}[language={c},firstnumber=168,caption={Auszug des Callbacks \lstinline{writeDataSourceVariable} in \lstinline{variables.c}\label{lst:4-writeDataSourceVariable}}]  
extern UA_StatusCode
 writeDataSourceVariable(UA_Server *server,
            const UA_NodeId *sessionId, void *sessionContext,
            const UA_NodeId *nodeId, void *nodeContext,
            const UA_NumericRange *range, const UA_DataValue *dataValue) {

    UA_StatusCode retval  = UA_STATUSCODE_GOOD;
    UA_NodeId *nameNodeId = UA_malloc(sizeof(UA_NodeId));
    UA_QualifiedName nameQN = UA_QUALIFIEDNAME(1, "Name");
    UA_Variant nameVar;
    UA_Boolean bit;

    retval |= findSiblingByBrowsename(server, nodeId, &nameQN, nameNodeId);
    retval |= UA_Server_readValue(server, *nameNodeId, &nameVar);
    retval |= UA_Boolean_copy(dataValue->value.data, &bit);

    |>\tikzmarkin[set border color=martinired]{writeIO}<|PI_writeSingleIO(String_fromUA_String(nameVar.data), &bit, false);                                                 |>\tikzmarkend{writeIO}<|

    free(nameNodeId);
    return retval;
 }
\end{lstlisting}

Listing~\ref{lst:4-writeDataSourceVariable} zeigt die Callback-Methode, welche nach dem Schreiben einer Variablen auf dem OPC-Server aufgerufen wird.
Dieser Methode wird neben der NodeId der mit ihr verknüpften Variablen auch der Wert dieser in Form eines Zeigers auf ein Struct vom Typ \lstinline{UA_DataValue} übergeben.

Die Gestaltung des hier verwendeten Nodesets sieht vor, dass in einer OPC-Variablen \lstinline{"Name"} der Bezeichner des zu schreibenden Digitalausgangs hinterlegt ist, siehe Abbildung~\ref{fig:opc-do}. Dies erlaubt eine Rekonfiguration der Ein- und Ausgänge der SPS ohne Änderungen im Programmcode des OPC-Servers vornehmen zu müssen.
Es ist daher erforderlich, nach jedem Schreiben einer mit einem Digitalausgang verknüpften Variablen, hier \lstinline{"Value"}, dessen Bezeichner \lstinline{"Name"} abzufragen. 
Dies geschieht in den Zeilen 180 und 181.
Anschließend wird dieser Bezeichner sowie der zu schreibende Wert der Methode \lstinline{PI_writeSingleIO()} übergeben, welche wiederum die Interaktion mit piControl übernimmt (vgl. Abschnitt \ref{sec:4-picontrol}).
 
\subsection{Integration von piControl%
     \label{sec:4-picontrol}}
In Abschnitt~\ref{sec:2-io} wurde die Anbindung der IO-Module des Revolution Pi sowie die Funktionsweise von piControl aus Anwendersicht beschrieben. Die verfügbare Literatur beschränkt sich auch auf lediglich diese Sicht; eine weiterführende Dokumentation für Entwickler gibt es, neben der in Abschnitt~\ref{sec:3-anbindung} vorgestellten Manpage, nicht. 
In diesem Abschnitt soll daher der Quellcode von piControl sowie dessen Verwendung im Projekt genauer betrachtet werden.
Hierzu wird exemplarisch die in Abschnitt~\ref{sec:4-open62541} eingeführte Methode \lstinline{PI_writeSingleIO()} untersucht.
Diese Methode ermöglicht das Setzen eines einzelnen Bits im Prozessabbild der SPS, und damit das Schalten eines digitalen Ausgangs auf einem IO-Modul.
Die äquivalente Methode \lstinline{int piControlGetBitValue(SPIValue *pSpiValue)} zum Lesen eines Bits bzw. Eingangs funktioniert analog und soll daher an dieser Stelle nicht dediziert erörtert werden.

\begin{lstlisting}[language={c},firstnumber=97,
                   caption={Setzen eines phsikalischen, digitalen Ausgangs in \lstinline{revpi.c}
                   \label{lst:4-PI_writeSingleIO}}]
extern void PI_writeSingleIO(char *pszVariableName, bool *bit, bool verbose)
{
	int rc;
	SPIVariable sPiVariable;
	SPIValue sPIValue;

	strncpy(sPiVariable.strVarName, pszVariableName, sizeof(sPiVariable.strVarName));
	rc = piControlGetVariableInfo(&sPiVariable);
	if (rc < 0) {
		printf("Cannot find variable '%s'\n", pszVariableName);
		return;
	}

		sPIValue.i16uAddress = sPiVariable.i16uAddress;
		sPIValue.i8uBit = sPiVariable.i8uBit;
		sPIValue.i8uValue = *bit;
		rc = |>\tikzmarkin[set border color=martinired]{setBitValue}<|piControlSetBitValue(&sPIValue)|>\tikzmarkend{setBitValue}<|;
		if (rc < 0)
			printf("Set bit error %s\n", getWriteError(rc));
		else if (verbose)
			printf("Set bit %d on byte at offset %d. Value %d\n", sPIValue.i8uBit, sPIValue.i16uAddress,
			       sPIValue.i8uValue);
}
\end{lstlisting}

Der Programmcode in Listing~\ref{lst:4-PI_writeSingleIO} ist Teil des implementierten OPC-Servers. In diesem wird auf zwei Funktionen des piControl-Treibers zugegriffen. 
Beiden Methoden wird als Argument ein Zeiger auf ein Struct vom Typ \lstinline{SPIValue} übergeben. Der im Struct abgelegte Name wird mittels \lstinline{piControlGetVariableInfo(&sPIValue)} zu einer Adresse im Prozessabbild aufgelöst. Diese wird in \lstinline{sPIValue.i16uAdress} gespeichert. Der Wert der Variablen wird anschließend mittels \lstinline{piControlSetBitValue(&sPIValue)} an dieser Adresse in das Prozessabbild geschrieben.

\begin{lstlisting}[language={c},firstnumber=309,caption={Methode \lstinline{piControlSetBitValue} in \lstinline{piControlIf.c}\label{lst:4-piControlSetBitValue}}]
int |>\tikzmarkin[set border color=martiniblue]{setBitValueFcn}<|piControlSetBitValue(SPIValue *pSpiValue)|>\tikzmarkend{setBitValueFcn}<|
{
    piControlOpen();

    if (PiControlHandle_g < 0)
	    return -ENODEV;

    pSpiValue->i16uAddress += pSpiValue->i8uBit / 8;
    pSpiValue->i8uBit %= 8;

    if (|>\tikzmarkin[set border color=martinired]{ioctl}<|ioctl(PiControlHandle_g, KB_SET_VALUE, pSpiValue)|>\tikzmarkend{ioctl}<| < 0)
	    return errno;

    return 0;
}
\end{lstlisting}

Die in Listing~\ref{lst:4-piControlSetBitValue} dargestellte Methode \lstinline{piControlSetBitValue} ist lediglich eine Hüllfunktion (häufig auch als Wrapper-Funktion bezeichnet) für einen Aufruf des \lstinline{ioctl} Kernel-Moduls.
Folgende Parameter werden übergeben:
\lstinline{PiControlHandle_g} ist die Referenz auf die Geräte-Datei des piControl-Treibers. \lstinline{KB_SET_VALUE} ist das ioctl-Kommando zum Schreiben eines Bits in das Prozessabbild. Der Zeiger \lstinline{pSpiValue} verweist auf ein Struct des bereits vorgestellten Typs \lstinline{SPIValue}.

\begin{lstlisting}[language={c},firstnumber=80,caption={Methode \lstinline{piControlOpen} in \lstinline{piControlIf.c}\label{lst:4-piControlOpen}}]
void piControlOpen(void)
{
    /* open handle if needed */
    if (PiControlHandle_g < 0)
    {
	    |>\tikzmarkin[set border color=martiniblue]{PiControlHandle}<|PiControlHandle_g = open(PICONTROL_DEVICE, O_RDWR)|>\tikzmarkend{PiControlHandle}<|;
    }
}
\end{lstlisting}

Die in Listing~\ref{lst:4-piControlOpen} dargestellte Methode öffnet, sofern nicht bereits geschehen, die Geräte-Datei. Das Macro \lstinline{PICONTROL_DEVICE} verweist hierbei auf \lstinline{/dev/piControl0}.

\begin{lstlisting}[language={c},firstnumber=721,caption={Methode \lstinline{piControlIoctl} in \lstinline{piControlMain.c}\label{lst:4-piControlIoctl}}]
static long |>\tikzmarkin[set border color=martiniblue, below offset=0.9em]{piControlIoctl}<|piControlIoctl(struct file *file, unsigned int prg_nr, 
                           unsigned long usr_addr)                                      |>\tikzmarkend{piControlIoctl}<|
{
  int status = -EFAULT;
  tpiControlInst *priv;
  int timeout = 10000;	// ms

  if (prg_nr != KB_CONFIG_SEND && prg_nr != KB_CONFIG_START && !isRunning()) {
  	return -EAGAIN;
  }

  priv = (tpiControlInst *) file->private_data;

  if (prg_nr != KB_GET_LAST_MESSAGE) {
  	// clear old message
  	priv->pcErrorMessage[0] = 0;
  }

  switch (prg_nr) {|>\setcounter{lstnumber}{864}<|

    case |>\tikzmarkin[set border color=martiniblue]{KB_SET_VALUE}<|KB_SET_VALUE:|>\tikzmarkend{KB_SET_VALUE}<|
  		{
  			SPIValue *pValue = (SPIValue *) usr_addr;

  			if (!isRunning())
  				return -EFAULT;

  			if (pValue->i16uAddress >= KB_PI_LEN) {
  				status = -EFAULT;
  			} else {
  				INT8U i8uValue_l;
  				my_rt_mutex_lock(&piDev_g.lockPI);
  				i8uValue_l = piDev_g.ai8uPI[pValue->i16uAddress];

  				if (pValue->i8uBit >= 8) {
  					i8uValue_l = pValue->i8uValue;
  				} else {
  					if (pValue->i8uValue)
  						i8uValue_l |= (1 << pValue->i8uBit);
  					else
  						i8uValue_l &= ~(1 << pValue->i8uBit);
  				}

  				|>\tikzmarkin[set border color=martinired]{i8uValue}<|piDev_g.ai8uPI[pValue->i16uAddress] = i8uValue_l;|>\tikzmarkend{i8uValue}<|
  				rt_mutex_unlock(&piDev_g.lockPI);

  #ifdef VERBOSE
  				pr_info("piControlIoctl Addr=%u, bit=%u: %02x %02x\n", pValue->i16uAddress, pValue->i8uBit, pValue->i8uValue, i8uValue_l);
  #endif

  				status = 0;
  			}
  		}
  		break; |>\setcounter{lstnumber}{1314}<|

    default:
      pr_err("Invalid Ioctl");
      return (-EINVAL);
      break;

    }

    return status;
  }
\end{lstlisting}

Listing~\ref{lst:4-piControlIoctl} zeigt in Auszügen die ioctl-Methode des piControl Kernel-Treibers. Diese bekommt folgende Argumente übergeben: \lstinline{struct file *file} enthält den Verweis auf die Geräte-Datei, hier \lstinline{/dev/piControl0}. Der Wert von \lstinline{unsigned int prg_nr} beschreibt die Anfrage an den Treiber, in diesem Fall \lstinline{KB_SET_VALUE}. Das Argument \lstinline{unsigned long usr_addr} enthält einen typ-agnostischen Pointer. Dieser verweist auf einen Speicherbereich, in welchem die zur Bearbeitung der Anfrage notwendigen Daten abgelegt sind. Hier können auch vom Treiber empfangene Daten dem Anwendungsprogramm bereitgestellt werden. 

Die switch-case-Anweisung führt die über das Argument \lstinline{prg_nr} spezifizierte Aktion aus. Hier betrachten wir \lstinline{KB_SET_VALUE}:
Zunächst wird in Zeile 868 der übergebene Zeiger \lstinline{usr_addr} mittels explizitem Typecast zu einem Zeiger des Typs \lstinline{SPIValue *} konvertiert. Da dieser auf Daten im Userspace verweist, ist beim Zugriff durch den Kernel-Treiber besondere Vorsicht geboten.
In Zeile 877 wird mittels Mutex das Prozessabbild \lstinline{piDev_g} für den Zugriff durch andere Threads oder Prozesse gesperrt.
\lstinline{my_rt_mutex_lock} verweist hierbei auf die Funktion \lstinline{rt_mutex_lock} aus \lstinline{linux/sched.h}\footnote{Offenbar wurde hier auch eine alternative Implementierung vorgesehen, siehe revpi\_common.h}

In Zeile 889 wird das Byte \lstinline{i8uValue_l}, welches den zu schreibenden Wert enthält in das Prozessabbild übertragen. Anschließend wird die Mutex auf \lstinline{piDev_g} wieder entsperrt.
\newpage

\begin{lstlisting}[language={c},firstnumber=62,caption={Auszug des Struct \lstinline{spiControlDev} in \lstinline{piControlMain.h}\label{lst:4-spiControlDev}}]
|>\tikzmarkin[set border color=martiniblue]{spiControlDev}<|typedef struct spiControlDev|>\tikzmarkend{spiControlDev}<| {
	// device driver stuff
	int init_step;
	enum revpi_machine machine_type;
	void *machine;
	struct cdev cdev;	// Char device structure
	struct device *dev;
	struct thermal_zone_device *thermal_zone;

	|>\tikzmarkin[set border color=martiniblue]{processImage}<|// process image stuff
	INT8U ai8uPI[KB_PI_LEN];
	INT8U ai8uPIDefault|>\tikzmarkin[set border color=martinired]{KB_PI_LEN_0}<|[KB_PI_LEN]|>\tikzmarkend{KB_PI_LEN_0}<|;
	struct rt_mutex lockPI;        |>\tikzmarkend{processImage}<|
	bool stopIO;
	piDevices *devs; |>\setcounter{lstnumber}{94}<|
} tpiControlDev;
\end{lstlisting}

Das Prozessabbild ist als Byte-Array der Länge \lstinline{KB_PI_LEN} in Listing~\ref{lst:4-spiControlDev} definiert. Konfigurationsparameter wie \lstinline{KB_PI_LEN} oder die Zykluszeit für den Datenaustausch zwischen SPS und IO-Modulen sind im folgenden Listing~\ref{lst:4-process} definiert.

\begin{lstlisting}[language={c},firstnumber=119,caption={Konfigurationsparameter des Prozessabbildes in project.h\label{lst:4-process}}]
#define INTERVAL_PI_GATE (5*1000*1000)  // 5 ms piGateCommunication |>\setcounter{lstnumber}{128}<|

#define INTERVAL_IO_COM (5*1000*1000)  // 5 ms piIoComm |>\setcounter{lstnumber}{132}<|

#define KB_PD_LEN       512
|>\tikzmarkin[set border color=martiniblue]{KB_PI_LEN_1}<|#define KB_PI_LEN       4096|>\tikzmarkend{KB_PI_LEN_1}<|
\end{lstlisting}

Das zu setzende Bit wurde zu diesem Zeitpunkt erfolgreich in das Prozessabbild der SPS geschrieben.
Es stellt sich die Frage, wie dieses nun an das IO-Modul kommuniziert wird.
Die Kommunikation mit allen angebundenen Modulen ist ebenfalls Aufgabe des piControl-Treibers.

\begin{lstlisting}[language={c},firstnumber=256,caption={Auszug der Methode \lstinline{piIoThread} in \lstinline{revpi_core.c}\label{lst:4-piIoThread}}]
static int piIoThread(void *data)
{
	//TODO int value = 0;
	ktime_t time;
	ktime_t now;
	s64 tDiff;

	hrtimer_init(&piCore_g.ioTimer, CLOCK_MONOTONIC, HRTIMER_MODE_ABS);
	piCore_g.ioTimer.function = piIoTimer;

	pr_info("piIO thread started\n");

	now = hrtimer_cb_get_time(&piCore_g.ioTimer);

	PiBridgeMaster_Reset();

	while (!kthread_should_stop()) {
		if (|>\tikzmarkin[set border color=martinired]{PiBridgeMaster}<|PiBridgeMaster_Run()|>\tikzmarkend{PiBridgeMaster}<| < 0)
			break;
	}

	RevPiDevice_finish();

	pr_info("piIO exit\n");
	return 0;
}
\end{lstlisting}

Der Kernel-Thread \lstinline{piIoThread} ist verantwortlich für den zyklischen Datenaustausch mit den IO-Modulen. In diesem wird fortlaufend die Methode \lstinline{PiBridgeMaster_Run()} aufgerufen, siehe Listing~\ref{lst:4-piIoThread}.

\begin{lstlisting}[language={c},firstnumber=262,caption={Auszug der Methode \lstinline{PiBridgeMaster_Run(void)} in \lstinline{RevPiDevice.c}\label{lst:4-PiBridgeMaster_Run}}]
int PiBridgeMaster_Run(void)
{
	static kbUT_Timer tTimeoutTimer_s;
	static kbUT_Timer tConfigTimeoutTimer_s;
	static int error_cnt;
	static INT8U last_led;
	static unsigned long last_update;
	int ret = 0;
	int i;

	my_rt_mutex_lock(&piCore_g.lockBridgeState);
	if (piCore_g.eBridgeState != piBridgeStop) {
		switch (eRunStatus_s) { |>\setcounter{lstnumber}{514}<|
		    case enPiBridgeMasterStatus_EndOfConfig:|>\setcounter{lstnumber}{621}<|
		    if (|>\tikzmarkin[set border color=martinired]{RevPiDevice}<|RevPiDevice_run()|>\tikzmarkend{RevPiDevice}<|) {
				// an error occured, check error limits |>\setcounter{lstnumber}{641}<|
			} else {
				ret = 1;
			}
			piCore_g.image.drv.i16uRS485ErrorCnt = RevPiDevice_getErrCnt();
			break;
\end{lstlisting}

Die in Listing~\ref{lst:4-PiBridgeMaster_Run} dargestellte Methode ist eine sog. State-Machine. Ist die Konfiguration der IO-Module erfolgreich abgeschlossen, so führt sie bei Aufruf lediglich die Methode \lstinline{RevPiDevice_run()} aus.

\begin{lstlisting}[language={c},firstnumber=140,caption={Auszug der Methode \lstinline{RevPiDevice_run(void)} in \lstinline{RevPiDevice.c}\label{lst:4-RevPiDevice_run}}]
int RevPiDevice_run(void)
{
	INT8U i8uDevice = 0;
	INT32U r;
	int retval = 0;

	RevPiDevices_s.i16uErrorCnt = 0;

	for (i8uDevice = 0; i8uDevice < RevPiDevice_getDevCnt(); i8uDevice++) {
		if (RevPiDevice_getDev(i8uDevice)->i8uActive) {
			switch (RevPiDevice_getDev(i8uDevice)->sId.i16uModulType) {
			case KUNBUS_FW_DESCR_TYP_PI_DIO_14:
			case KUNBUS_FW_DESCR_TYP_PI_DI_16:
			case KUNBUS_FW_DESCR_TYP_PI_DO_16:
				r = |>\tikzmarkin[set border color=martinired]{sendCyclicTelegram}<|piDIOComm_sendCyclicTelegram(i8uDevice)|>\tikzmarkend{sendCyclicTelegram}\setcounter{lstnumber}{166} <|;

				break; |>\setcounter{lstnumber}{216}<|
			}
		}
	} |>\setcounter{lstnumber}{227}<|
	return retval;
}
\end{lstlisting}

Diese iteriert wie in Listing~\ref{lst:4-RevPiDevice_run} abgebildete durch alle gegenwärtig in der SPS konfigurierten Module. Ist das aktuelle Modul als aktiv markiert, so wird anhand eines sog. Firmware-Descriptors entschieden, welche Methode für die Ansteuerung des Moduls aufzurufen ist.

\begin{lstlisting}[language={c},firstnumber=161,caption={Auszug der Methode \lstinline{piDIOComm_sendCyclicTelegram} in \lstinline{piDIOComm.c}\label{lst:4-sendCyclicTelegram}}]
INT32U piDIOComm_sendCyclicTelegram(INT8U i8uDevice_p)
{
	INT32U i32uRv_l = 0;
	SIOGeneric sRequest_l;
	SIOGeneric sResponse_l;
	INT8U len_l, data_out[18], i, p, data_in[70];
	INT8U i8uAddress;
	int ret; |>\setcounter{lstnumber}{239}<|
	
    |>\tikzmarkin[set border color=martinired]{piIoComm}<|ret = piIoComm_send((INT8U *) & sRequest_l, IOPROTOCOL_HEADER_LENGTH + len_l + 1);  |>\tikzmarkend{piIoComm}\setcounter{lstnumber}{298}<|
}
\end{lstlisting}

Im Falle des hier verwendeten DO-Moduls wird die in Listing~\ref{lst:4-sendCyclicTelegram} abgebildete Methode \lstinline{piDIOComm_sendCyclicTelegram()} aufgerufen. Dieser wird ein Zeiger auf das zu schreibende Gerät übergeben. 
Zunächst wird das Prozessabbild mittels eines proprietären, jedoch im Quellcode offen nachvollziehbaren Protokolls in ein \lstinline{sRequest_l} genanntes Byte-Array umgewandelt. Dieser Schritt ist in Listing~\ref{lst:4-sendCyclicTelegram} nicht abgebildet. Anschließend wird \lstinline{piIoComm_send()} ein Zeiger auf die so generierte Schreib-Anfrage übergeben.

\begin{lstlisting}[language={c},firstnumber=220,caption={Auszug der Methode \lstinline{piIOComm_send} in \lstinline{piIOComm.c}\label{lst:4-piIOComm_send}}]
int piIoComm_send(INT8U * buf_p, INT16U i16uLen_p)
{
	ssize_t write_l = 0;
	INT16U i16uSent_l = 0;|>\setcounter{lstnumber}{249}<|

	while (i16uSent_l < i16uLen_p) {
		write_l = vfs_write(piIoComm_fd_m, buf_p + i16uSent_l, i16uLen_p - i16uSent_l, &piIoComm_fd_m->f_pos);
		if (write_l < 0) {
			pr_info_serial("write error %d\n", (int)write_l);
			return -1;
		} 
		i16uSent_l += write_l;|>\setcounter{lstnumber}{263}<|
	}
	clear();
	vfs_fsync(piIoComm_fd_m, 1);
	return 0;
}
\end{lstlisting}

Listing~\ref{lst:4-piIOComm_send} zeigt die Implementierung von \lstinline{piIoComm_send()}. Diese Methode ist für das Schreiben der oben generierten Anfrage auf die seriellen Schnittstelle verantwortlich. Realisiert wird dies mittels der Methode \lstinline{vfs_write()}. Diese ist in \lstinline{<linux/fs.h>} definiert. Sie ermöglicht das Schreiben einer Datei im Userspace aus dem Kernel heraus. Geschrieben wird hier die Datei mit dem Deskriptor \lstinline{piIoComm_fd_m}.
Da die Funktion \lstinline{vfs_write()} durch andere Kernel-Tasks unterbrochen werden kann, ist nicht gewährleistet, dass die gesamte Anfrage mit nur einem Aufruf geschrieben wird. Die oben abgebildete while-Schleife stellt das vollständige Senden der Anfrage sicher.

\begin{lstlisting}[language={c},firstnumber=157,caption={Auszug der Methode \lstinline{piIOComm_open_serial} in \lstinline{piIOComm.c}\label{lst:4-piIOComm_open_serial}}]
int piIoComm_open_serial(void)
{   |>\setcounter{lstnumber}{167}<|
	struct file *fd;	/* Filedeskriptor */
	struct termios newtio;	/* Schnittstellenoptionen */

	|>\tikzmarkin[set border color=martiniblue]{fd}<|/* Port oeffnen - read/write, kein "controlling tty", 
	    Status von DCD ignorieren */
	fd = filp_open(|>\tikzmarkin[set border color=martinired]{tty}<|REV_PI_TTY_DEVICE|>\tikzmarkend{tty}<|, O_RDWR | O_NOCTTY, 0); |>\setcounter{lstnumber}{208}<|
	
	piIoComm_fd_m = fd;                                                      |>\tikzmarkend{fd}\setcounter{lstnumber}{217}<|

	return 0;
}
\end{lstlisting}

Der zum Schreiben auf die serielle Schnittstelle verwendete Datei-Deskriptor wird von der in Listing~\ref{lst:4-piIOComm_open_serial} abgebildeten Methode \lstinline{piIoComm_open_serial()} generiert. 

\begin{lstlisting}[language={c},firstnumber=45,caption={Definition der seriellen Schnittstelle in \lstinline{piIOComm.h}\label{lst:4-REV_PI_TTY_DEVICE}}]
#define REV_PI_TTY_DEVICE	"/dev/ttyAMA0"
\end{lstlisting}

Das in Listing~\ref{lst:4-REV_PI_TTY_DEVICE} definierte Macro verweist auf eine der seriellen Schnittstellen des RaspberryPi.
Die Implementierung des zugehörigen Schnittstellentreibers soll hier nicht weiter untersucht werden. Somit ist an dieser Stelle die Kette vom Setzen einer Variablen auf dem OPC-Server bis hin zur Aktualisierung des Prozessabbilds der IO-Module geschlossen.

% \begin{lstlisting}[language={c},firstnumber={226},caption={Setzen der Scheduler-Priorität auf SCHED\_FIFO in 
% revpi\_common.c\label{lst:2-sched_priority}}]
% param.sched_priority = ktprio->prio;
% ret = sched_setscheduler(child, SCHED_FIFO, &param);
% \end{lstlisting}
% % % Imports nur für Referenzenauflösung während des Schreibens! Vorm Kompilieren auskommentieren!
% \bibliography{0_hauptdatei}
% \input{1_einleitung}
% \input{2_grundlagen}
% \input{3_konzeption}
% \input{4_implementierung}
% \input{5_tests}
% \input{6_zusammenfassung}
% % Ende Imports

\section{Test des OPC-Servers im Gesamtsystem%
  \label{sec:5-tests}}

% % % Imports nur für Referenzenauflösung während des schreibens! Vorm Kompilieren auskommentieren!
% \bibliography{0_hauptdatei}
% \input{1_einleitung}
% \input{2_grundlagen}
% \input{3_konzeption}
% \input{4_implementierung}
% \input{5_tests}
% \input{6_zusammenfassung}
% % Ende Imports

\section{Zusammenfassung und Ausblick%
  \label{sec:6-fazit}}
Der folgende Abschnitt~\ref{sec:6-zusammenfassung} fasst die gewonnenen Erkenntnisse und den Stand der Implementierung zusammen.
Den Abschluss dieser Arbeit bildet der Ausblick in Abschnitt~\ref{sec:6-ausblick}.

\subsection{Zusammenfassung%
     \label{sec:6-zusammenfassung}}

\subsection{Ausblick%
     \label{sec:6-ausblick}}

% % Ende Imports

\section{Test des OPC-Servers im Gesamtsystem%
  \label{sec:5-tests}}

%% % Imports nur für Referenzenauflösung während des schreibens! Vorm Kompilieren auskommentieren!
% \bibliography{0_hauptdatei}
% % Mit \section{...} eröffnen wir einen neuen Abschnitt.
% Der Befehl setzt nicht nur den Text in einer größeren,
% fetten Schrift, sondern sorgt außerdem dafür, daß er im
% Inhaltsverzeichnis erscheint.
%
% Mit \label{...} erzeugen wir einen Bezeichner, mit dessen Hilfe
% wir später auf die Nummer des Abschnitts verweisen können (nämlich
% mit~\ref{...}).
%
% Das Kommentarzeichen hinter „Übersicht“ dient dazu, ein
% Leerzeichen zwischen „Übersicht“ und dem \label-Befehl
% zu vermeiden, das andernfalls sichtbar würde – z.B. im
% Inhaltsverzeichnis.
%

% % Imports nur für Referenzenauflösung während des Schreibens! Vorm Kompilieren auskommentieren!
% \bibliography{0_hauptdatei}
% \input{1_einleitung}
%\input{2_grundlagen}
%\input{3_konzeption}
%\input{4_implementierung}
%\input{5_tests}
%\input{6_zusammenfassung}
% % Ende Imports

\section{Einleitung und Motivation%
  \label{sec:1-einleitung}}
Ziel dieses Projektes ist die Integration eines OPC-Servers mit einer auf Linux
basierenden speicherprogrammierbaren Steuerung (SPS). Angeschlossen an diese SPS
ist jeweils ein digitales Ein-/\,bzw.~Ausgabemodul. Die von diesen bereitgestellten
Ein-/\, bzw.~Ausgänge (IO) sollen in der Datenstruktur des OPC-Servers abgebildet
und über diesen für OPC-Clients les-/\,und schreibar sein. Weiterhin sollen einige
Funktionen zur Überwachung und Steuerung der an die SPS angeschlossenen Aktoren
und Sensoren direkt im OPC-Server implementiert werden.
Hiermit stellt dieses Projekt eine der Grundlagen für ein übergeordnetes Projekt,
die cloudbasierte Steuerung eines miniaturisierten Produktions-Systems, dar.

Der hier verwendete OPC-Server ist Teil des sog. open62541 Projekts. Er ist in C
geschrieben und implementiert bereits einen großen Teil der im OPC-UA-Standard
spezifizierten Funktionen.
Als SPS findet ein Revolution Pi 3 der Firma Kunbus Verwendung. Dieser integriert
ein sog. Compute Module der Raspberry Pi Foundation in ein industrietaugliches
Gehäuse und erlaubt die Erweiterung mittels IO- oder Gateway-Modulen. Über diese
erfolgt die Kommunikation mit weiteren Komponenten der Automatisierungstechnik.

Motiviert ist dieses Projekt durch die Beobachtung, dass die Verbreitung offener
Standards sowie freier Software auch in der Automatisierungstechnik zunimmt.
Linux ist ein freies Betriebssystem, OPC-UA ein offen zugänglicher, aktiv gepflegter
und weit verbreiteter Standard. Der Raspberry Pi findet sowohl bei Hobby-Anwendern als
auch in den Bereichen Forschung und Entwicklung sowie bei industriellen Anwendern
Verwendung. Dieses Projekt stellt somit eine für unterschiedliche Anwender interessante
Entwicklung dar.

Im Anschluss an diese einleitende Übersicht im Abschnitt~\ref{sec:1-einleitung} folgt
die Darstellung der wichtigsten Grundlagen in Abschnitt~\ref{sec:2-grundlagen}.
Aufbauend auf diesen Grundlagen folgt die konzeptuelle Ausarbeitung im Abschnitt~\ref{sec:3-konzeption}.
Die Umsetzung wird im Abschnitt~\ref{sec:4-implementierung} erläutert.
Die Leistungsfähigkeit der Implementierung wird in Abschnitt~\ref{sec:5-tests} untersucht.
Eine Zusammenfassung und ein Ausblick schließen die Arbeit in
Abschnitt~\ref{sec:6-fazit} ab. Eventuell noch benötigte Anhänge
finden sich in den Anhängen [...] bis [...].

% % % Imports nur für Referenzenauflösung während des Schreibens! Vorm Kompilieren auskommentieren!
% \bibliography{0_hauptdatei}
% \input{1_einleitung}
% \input{2_grundlagen}
% \input{3_konzeption}
% \input{4_implementierung}
% \input{5_tests}
% \input{6_zusammenfassung}
% % Ende Imports

\section{Grundlagen%
  \label{sec:2-grundlagen}}

\subsection{Speicherprogrammierbare-Steuerung und Linux -- Revolution Pi%
     \label{sec:2-sps}}

\subsubsection{Kunbus RevolutionPi%
        \label{sec:2-revpi}}
Der RevolutionPi 3 ist eine speicherprogrammierbare Steuerung (SPS) des Herstellers
Kunbus GmbH. Kern dieser SPS ist das von der Raspberry Pi Foundation entwickelte
und vertriebene Raspberry Pi Compute Module 3. Dieses integriert ein Broadcom BCM2837
System-on-Chip (SoC) mit vier 1,2GHz Prozessorkernen, 1GB RAM, 4GB eMMC Anwendungsspeicher
und sonstige Peripherie in ein Modul im DDR2-SODIMM Formfaktor. Diese Spezifikationen
sind weitgehend identisch zu denen des ausgesprochen populären Raspberry Pi 3.
Der Revolution Pi profitiert daher von dem gleichen großen Angebot an Software
und Unterstützung wie der Raspberry Pi, ergänzt dessen Hardware jedoch um eine 24V
Spannungsversorgung, die Möglichkeit der Erweiterung durch mehrere industrietaugliche
Ein-/ Ausgabemodule und Gateways sowie ein Gehäuse zur Montage auf einer DIN-Schiene.
\begin{itemize}
  \item{Prozessor: BCM2837}
  \item{Taktfrequenz 1,2 GHz}
  \item{Anzahl Prozessorkerne: 4}
  \item{Arbeitsspeicher: 1 GByte}
  \item{eMMC Flash Speicher: 4 GByte}
  \item{Betriebssystem: Angepasstes Raspbian mit RT-Patch}
  \item{RTC mit 24h Pufferung über wartungsfreien Kondensator}
  \item{Treiber / API: Treiber schreibt zyklisch Prozessdaten in ein Prozessabbild, Zugriff auf Prozessabbild über Linux-Filesystem als API zu Fremdsoftware.}
  \item{Kommunikationsanschlüsse: 2 x USB 2.0 A (je 500 mA belastbar), 1 x Micro-USB, HDMI, Ethernet (RJ45) 10/100 Mbit/s}
  \item{Stromversorgung: min. 10,7 V, max. 28,8 V, maximal 10 Watt}
  \item{Zulässige Umgebungstemperatur: -40 bis +55 C}
  \item{Gehäuseabmessungen: (HxBxL) 96 mm x 22,5 mm x 110,5 mm (ohne gesteckte Stecker)}
  \item{ESD Schutz: 4 kV / 8 kV gemäß EN61131-2 und IEC 61000-6-2}
  \item{Surge / Burst Prüfungen: gemäß EN61131-2 und IEC 61000-6-2 eingekoppelt auf Versorgungsspannung, Ethernet und IO-Leitungen}
  \item{EMI Prüfungen: gemäß EN61131-2 und IEC 61000-6-2}
\end{itemize}

Kunbus bietet eine Auswahl an IO- und Gateway-Modulen zur Erweiterung des Revolution Pi an.
Gateways dienen der Kommunikation mit Systemen oder Komponenten der Automatisierungstechnik
über Protokolle wie PROFIBUS oder EtherCAT. IO-Module erlauben die Überwachung
und Steuerung von digitalen oder analogen Ein- und Ausgängen.

\subsubsection{Zugriff auf IO-Module%
        \label{sec:2-io}}
Der Zugriff auf die Ein- und Ausgänge der IO-Module erfolgt über ein Prozessabbild
und einen hierfür von Kunbus bereitgestellten Treiber, genannt piControl. Dieser
aktualisiert das Prozessabbild zyklisch. Die angestrebte Zykluszeit beträgt 5ms,
kann jedoch je nach Anzahl der angeschlossenen Module auch größer sein. Kunbus
garantiert bei drei IO-Modulen und zwei Gateway-Modulen eine Zykluszeit von 10 ms.
Jedes der IO-Module stellt ein eigenständiges eingebettetes System dar. Es verfügt
über einen Microcontroller, welcher die IOs bereitstellt und über einen RS485-Bus
mit dem Revolution Pi kommuniziert.
% https://revolution.kunbus.de/io-modul/

Lizenz: GPL
% https://github.com/RevolutionPi/piControl

\begin{lstlisting}[language={c},firstnumber={226},caption={Setzen der Scheduler-Priorität auf SCHED\_FIFO in revpi\_common.c\label{lst:2-sched_priority}}]
param.sched_priority = ktprio->prio;
ret = sched_setscheduler(child, SCHED_FIFO,
       &param);
\end{lstlisting}


\subsection{Echtzeit und Multithreading unter Linux -- preemptRT und posix%
     \label{sec:2-echtzeit}}


 Der Linux-Kernel verfügt über mehrere unterschiedliche Preemtion-Modelle:

\begin{itemize}
  \item No Forced Preemption (server):
  Ausgelegt auf maximal möglichen Durchsatz, lediglich Interrupts und
  System-Call-Returns bewirken Präemption.

  \item Voluntary Kernel Preemption (Desktop):
  Neben den implizit bevorrechtigten Interrupts und System-Call-Returns gibt es
  in diesem Modell weitere Abschnitte des Kernels in welchen Preämption explizit
  gestattet ist.

  \item Preemptible Kernel (Low-Latency Desktop):
  In diesem Modell ist der gesamte Kernel, mit Ausnahme sog.~kritischer Abschnitte
  präemptible. Nach jedem kritischen Abschnitt gibt es einen impliziten Präemptions-Punkt.

  \item Preemptible Kernel (Basic RT):
  Dieses Modell ist dem zuvor genannten sehr ähnlich, hier sind jedoch alle Interrupt-Handler
  als eigenständige Threads ausgeführt.

  \item Fully Preemptible Kernel (RT):
  Wie auch bei den beiden zuvor genannten Modellen ist hier der gesamte Kernel
  präemtible, die Anzahl und Dauer der nicht-präemtiblen kritischen Abschnitte
  ist auf ein notwendiges Minimum beschränkt. Alle Interrupt-Handler sind als
  eigenständige Threads ausgeführt, Spinlocks durch Sleeping-Spinlocks und Mutexe
  durch sog.~RT-Mutexe ersetzt.

\end{itemize}
\todo{Spinlocks und Mutexe sowie die RT-Varianten dieser erklären!}

Lediglich mit dem vollständig präemtiblen Kernel kann Echtzeit-Verhalten realisiert werden.

% https://wiki.linuxfoundation.org/realtime/documentation/technical_basics/preemption_models bzw kernel/Kconfig.preempt

\subsubsection{preemptRT%
        \label{sec:2-preemptRT}}
% https://wiki.linuxfoundation.org/realtime/documentation/technical_details/start
% https://wiki.linuxfoundation.org/realtime/documentation/technical_basics/start

Das dem PREEMPT RT Kernel zugrunde liegende Prinzip lässt sich in einer einfachen
Regel ausdrücken: Nur Code, welcher absolut nicht-präemtible sein darf, ist es
gestattet nicht-präemtible zu sein.
Das erklärte Ziel des PREEMPT\_RT Patches ist es folglich, die Menge des nicht-präemtiblen
Codes im Linux-Kernel auf das absolut notwendige Minimum zu reduzieren.

Dies wird durch Verwendung folgender Mechanismen erreicht:

\begin{itemize}
  \item Hochauflösende Timer
  \item Sleeping Spinlocks
  \item Threaded Interrupt Handlers
  \item rt\_mutex
  \item RCU
\end{itemize}


\subsubsection{posix%
        \label{sec:2-posix}}
Ist posix hier wirklich relevant? Debian bzw.~Raspbian sind weitgehend posix
kompatibel, aber wird es hier genutzt? -> JA, open62541 nutzt pthread.h
piControl nutzt kthread.h, und semaphore.h

\subsection{OPC-UA und open62541%
     \label{sec:2-opc}}

\subsubsection{OPC UA%
        \label{sec:2-opcua}}
Open Platform Communications (OPC) ist eine Familie von Standards zur herstellerunabhängigen
Kommunikation von Maschinen (M2M) in der Automatisierungstechnik. Die sog.~OPC Task Force, zu deren
Mitgliedern verschiedene große Firmen der Automatisierungsindustrie gehören, veröffentlichte
die OPC Specification Version 1.0 im August 1996.
Motiviert ist dieser offene Standard durch die Erkenntniss, dass die Anpassung der
zahlreichen Herstellerstandards an individuelle Infrastrukturen und Anlagen einen
großen Mehraufwand verursachen.
Die Wikipedia beschreibt das Anwendungsgebiet für OPC wie folgt:

\glqq{}OPC wird dort eingesetzt, wo Sensoren, Regler und Steuerungen verschiedener Hersteller
ein gemeinsames Netzwerk bilden. Ohne OPC benötigten zwei Geräte zum Datenaustausch
genaue Kenntnis über die Kommunikationsmöglichkeiten des Gegenübers. Erweiterungen
und Austausch gestalten sich entsprechend schwierig. Mit OPC genügt es, für jedes
Gerät genau einmal einen OPC-konformen Treiber zu schreiben. Idealerweise wird
dieser bereits vom Hersteller zur Verfügung gestellt. Ein OPC-Treiber lässt sich
ohne großen Anpassungsaufwand in beliebig große Steuer- und Überwachungssysteme
integrieren.

OPC unterteilt sich in verschiedene Unterstandards, die für den jeweiligen Anwendungsfall
unabhängig voneinander implementiert werden können. OPC lässt sich damit verwenden
für Echtzeitdaten (Überwachung), Datenarchivierung, Alarm-Meldungen und neuerdings
auch direkt zur Steuerung (Befehlsübermittlung).\grqq{}

OPC basiert in der ursprünglichen Spezifikation auf Microsofts DCOM-Spezifikation.
DCOM macht Funktionen und Objekte einer Anwendung anderen Anwendungen im Netzwerk
zugänglich. Der OPC-Standard definiert entsprechende DCOM-Objekte um mit anderen
OPC-Anwendungen Daten austauschen zu können. Die Verwendung von DCOM bindet Anwender
an Betriebssysteme von Microsoft. Die ursprüngliche OPC Spezifikation wird durch die
Entwicklung von OPC Unified Architecture (OPC UA) abgelöst.
OPC UA setzt auf einem eigenen Kommunikationionsstack auf, die Verwendung von DCOM
und damit die Bindung an Microsoft wurden aufgelöst.

Die OPC-UA-Architektur ist eine Service-orientierte Architektur (SOA), deren Struktur
aus mehreren Schichten besteht.

% Wikipedia
Das OPC-Informationsmodell ist nicht mehr nur eine Hierarchie aus Ordnern, Items
und Properties. Es ist ein sogenanntes Full-Mesh-Network aus Nodes, mit dem neben
den Nutzdaten eines Nodes auch Meta- und Diagnoseinformationen repräsentiert werden.
Ein Node ähnelt einem Objekt aus der objektorientierten Programmierung. Ein Node
kann Attribute besitzen, die gelesen werden können (Data Access (DA), Historical
Data Access (HDA)). Es ist möglich Methoden zu definieren und aufzurufen.
Eine Methode besitzt Aufrufargumente und Rückgabewerte. Sie wird durch ein Command
aufgerufen. Weiterhin werden Events unterstützt, die versendet werden können
(AE (Alarms \& Events), DA DataChange), um bestimmte Informationen zwischen Geräten
auszutauschen. Ein Event besitzt unter anderem einen Empfangszeitpunkt, eine Nachricht
und einen Schweregrad. Die o. g. Nodes werden sowohl für die Nutzdaten als auch
alle anderen Arten von Metadaten verwendet. Der damit modellierte OPC-Adressraum
beinhaltet nun auch ein Typmodell, mit dem sämtliche Datentypen spezifiziert werden.

% https://de.wikipedia.org/wiki/Open_Platform_Communications
% https://de.wikipedia.org/wiki/OPC_Unified_Architecture
% https://opcfoundation.org/developer-tools/specifications-unified-architecture
% Von Gerhard Gappmeier - ascolab GmbH, CC BY-SA 3.0, https://de.wikipedia.org/w/index.php?curid=1892069
\subsubsection{open62541%
        \label{sec:2-open62541}}
open62541 ist eine offene und freie Implementierung von OPC UA. Die in C geschriebene
Bibliothek stellt eine beständig zunehmende Anzahl der im OPC UA Standard definierten
Funktionen bereit. Sie kann sowohl zur Erstellung von OPC-Servern als auch -Clients
genutzt werden. Ergänzend zu der unter der Mozilla Public License v2.0 lizensierten
Bibliothek stellt das open62541 Projekt auch Beispielprogramme unter einer CC0 Lizenz
zur Verfügung.

Die Bibliothek eignet sich auch für die Entwicklung auf eingebetteten Systemen und
Microcontrollern. Je nach Umfang der gewünschten Funktionen und des OPC Informationsmodells
beträgt die Größe einer Server-Binary weniger als 100kb. %evtl. kürzen?

\todo{Nodes erklären! Evtl.~oben!}

Folgende Auswahl an Eigenschaften und Funktionen zeichnet die in dieser Arbeit verwendete
Version 0.3 von open62541 aus:
\begin{itemize}
  \item Kommunikationionsstack
  \begin{itemize}
      \item OPC UA Binär-Protokoll (HTTP oder SOAP werden gegenwärtig nicht unterstützt)
      \item Austauschbare Netzwerk-Schicht, welche die Verwendung eigener Netzwerk-APIs
      erlaubt.
      \item Verschlüsselte Kommunikationion
      \item Asynchrone Dienst-Anfragen im Client
  \end{itemize}
  \item Informationsmodell
  \begin{itemize}
    \item Unterstützung aller OPC UA Node-Typen, inkl.~Methoden
    \item Hinzufügen und Entfernen von Nodes und Referenzen zur Laufzeit.
    \item Vererbung und Instanziierung von Objekt- und Variablentypen
    \item Zugriffskontrolle auch für einzelne Nodes
  \end{itemize}
  \item Subscriptions
  \begin{itemize}
    \item Erlaubt die Überwachung (subscriptions / monitoreditems)
    \item Sehr geringer Ressourcenbedarf pro überwachtem Wert
  \end{itemize}
  \item Code-Generierung auf XML-Basis
  \begin{itemize}
    \item Erlaubt die Erstellung von Datentypen
    \item Erlaubt die Generierung des serverseitigen Informationsmodells
  \end{itemize}
\end{itemize}

% https://open62541.org/doc/0.3/


Mozilla Public License
CC0 Lizenz für Beispiele und Plugins

% https://open62541.org/doc/open62541-current.pdf
% https://open62541.org/

% % % Imports nur für Referenzenauflösung während des Schreibens! Vorm Kompilieren auskommentieren!
% \bibliography{0_hauptdatei}
% \input{1_einleitung}
% \input{2_grundlagen}
% \input{3_konzeption}
% \input{4_implementierung}
% \input{5_tests}
% \input{6_zusammenfassung}
% \input{anhang}
% % Ende Imports

\section{Systemkonzept%
  \label{sec:3-konzeption}}
Auf Basis der in Abschnitt \ref{sec:2-grundlagen} vorgestellten Möglichkeiten folgt nun die Ausarbeitung eines Konzepts.
In den folgenden Abschnitten soll näher auf zwei zentrale Aspekte eingegangen werden: Abschnitt~\ref{sec:3-anbindung} stellt Möglichkeiten zum Zugriff auf Variablen bzw.\,Werte im Prozessabbild des Revolution Pi vor; in Abschnitt~\ref{sec:3-integration} wird ein Konzept zur Bereitstellung dieser Variablen auf einem OPC-Server vorgestellt.

\subsection{Anbindung der IO an den OPC-Server%
     \label{sec:3-anbindung}}

Eine Webanwendung mit Bezeichnung PiCtory dient zur Konfiguration der I/O- und virtuellen Module des RevolutionPi. Die Konfiguration liegt im JSON-Format in der Datei \lstinline{/etc/revpi/config.rsc}. Der piControl-Treiber liest diese Datei beim Start. 
Der folgende Auszug aus der Manpage des piControl-Kernelmoduls beschreibt die von diesem zum Lesen und Schreiben einzelner Bits des Prozessabbildes bereitgestellten Funktionen~\citep[vgl.]{web-revpi-manpage}. Sie ist an dieser Stelle weitgehend ungekürzt zitiert, da sie die nutzbare Schnittstelle sehr kompakt beschreibt.

\begin{lstlisting}[breakindent=0pt, numbers=none, caption={Auszug aus der Revolution Pi Programmers Manual\label{lst:4-manpage}}]
KB_FIND_VARIABLE SPIVariable *argp
Find a variable in the process image by its name. A pointer to a structure of type SPIVariable must be passed as argument. [...]
The struct SPIVariable [...] is defined as 
typedef struct SPIVariableStr
{
    char strVarName[32]; // Variable name
    uint16_t i16uAddress; // Address of the byte in the process image
    uint8_t i8uBit; // 0-7 bit position, >= 8 whole byte
    uint16_t i16uLength; // length of the variable in bits.
    // Possible values are 1, 8, 16 and 32
} SPIVariable;

Set and get values of the process image
KB_GET_VALUE SPIValue *argp
[...]
KB_SET_VALUE SPIValue *argp
Write one bit or one byte to the process image [...].  This call is more efficient than the usual calls of seek and write because only one function call is necessary. If more than on application are writing bits in one output byte, this call is the only safe way to set a bit without overwriting the other bits because this call is doing a read-modify-write-cycle. 

The struct SPIValue used by this ioctl is defined as
typedef struct SPIValueStr
{
    uint16_t i16uAddress; // Address of the byte in the process image
    uint8_t i8uBit; // 0-7 bit position, >= 8 whole byte
    uint8_t i8uValue; // Value: 0/1 for bit access, whole byte otherwise
} SPIValue;
\end{lstlisting} 

Die oben beschriebenden Funtkionen \lstinline{KB_FIND_VARIABLE}, \lstinline{KB_GET_VALUE} und \lstinline{KB_SET_VALUE} ermöglichen einen einfachen und (lt.\,Manpage) effizienten Zugriff auf einzelne Bits des Prozessabbildes und damit auch auf die IO des RevolutionPi.
Der Zugriff des OPC-Servers auf das Prozessabbild soll daher mittels dieser Funktionen realisiert werden.
\lstinline{KB_FIND_VARIABLE} kann genutzt werden, um Adressen von Variablen im Prozessabbild mittels ihres Namens aufzulösen.
\lstinline{KB_GET_VALUE} und \lstinline{KB_SET_VALUE} ermöglichen den Zugriff auf die Werte dieser Variablen.


\subsection{Integration des OPC-Servers in das System%
     \label{sec:3-integration}}

open62541 bietet drei Möglichkeiten zum Abgleich von Variablen mit dem Prozessabbild~\citep[vgl.][Tutorials - Connecting a Variable with a Physical Process]{web-open62541}:
\begin{itemize}
    \item Manuelles oder zyklisches Aktualisieren
    \item Variable Value Callback
    \item Variable Datasource
\end{itemize}

Die zyklische Aktualisierung eines oder mehrerer Werte nimmt, abhängig von der Zykluszeit, viele Systemressourcen in Anspruch. Value Callbacks ermöglichen es, einen Variablenwert effizienter mit einer Ressource wie etwa einem Prozessabbild zu synchronisieren. An die Variable wird ein Callback angehängt, welches vor jedem Lesen und nach jedem Schreibvorgang ausgeführt wird.
Der Wert der Variablen wird weiterhin im Variablenknoten auf dem OPC-Server gespeichert, der Abgleich mit der verknüpften Ressource erfolgt durch die Callback-Methoden.

Sogenannte Datenquellen gehen noch einen Schritt weiter. Der Server leitet jede Lese- und Schreibanforderung direkt an eine Callback-Funktion weiter. Beim Lesen liefert der Rückruf eine Kopie des aktuellen Wertes. Die Datenquelle muss intern ein eigenes Speichermanagement implementieren.

Der Zugriff auf die Werte des Prozessabbildes erfolgt, wie in Abschnitt~\ref{sec:3-anbindung} beschrieben, über von piControl bereitgestellte Methoden. Um die durch open62541 gepflegte OPC-Datenstruktur und das durch piControl verwaltete Prozessabbild möglichst effektiv verknüpfen zu können, soll diese Interaktion mittels Datenquellen und den zugehörigen Callbacks implementiert werden.
% % % Imports nur für Referenzenauflösung während des Schreibens! Vorm Kompilieren auskommentieren!
% \bibliography{0_hauptdatei}
% \input{1_einleitung}
% \input{2_grundlagen}
% \input{3_konzeption}
% \input{4_implementierung}
% \input{5_tests}
% \input{6_zusammenfassung}
% \input{anhang}
% % Ende Imports

\section{Implementierung%
  \label{sec:4-implementierung}}
Das folgende Kapitel stellt in Auszügen die Implementierung des OPC-Servers sowie die Anbindung an die IO-Module
der SPS dar. Der Schwerpunkt liegt hierbei auf der Funktionsweise des piControl-Treibers und dessen Integration in das Projekt. Abschnitt~\ref{sec:4-picontrol} erklärt die zum Schreibens eines Bits verwendeten Funktionsaufrufe.
Zuvor soll jedoch in Abschnitt~\ref{sec:4-open62541} der Teil des OPC-Servers vorgestellt werden, welcher auf besagten Treiber zugreift. 

\subsection{Implementierung des OPC-Servers%
     \label{sec:4-open62541}}
Wie im vorangegangenen Abschnitt~\ref{sec:3-integration} begründet, soll die Verknüpfung zwischen dem Prozessabbild der SPS und den auf dem OPC-Server bereitgestellten Werten über sog.\,Datenquellen erfolgen. Hierzu ist zunächst eine Callback-Methode zu implementieren, welche bei einem Lese- oder Schreibzugriff auf eine Variable aufgerufen wird. Die Verknüpfung zwischen Callback-Methode und Variable muss manuell erfolgen.

\begin{lstlisting}[language={c},firstnumber=237,caption={Auszug der Methode \lstinline{linkDataSourceVariable} in \lstinline{variables.c}\label{lst:4-linkDataSourceVariable}}]
extern UA_StatusCode
 linkDataSourceVariable(UA_Server *server, UA_NodeId nodeId) {
     bool readonly = false;
     UA_DataSource dataSourceVariable;
     UA_StatusCode rc; |>\setcounter{lstnumber}{254}<|

     dataSourceVariable.read = readDataSourceVariable;
     if (!readonly)
        dataSourceVariable.write = writeDataSourceVariable;
     else
        dataSourceVariable.write = writeReadonlyDataSourceVariable;

     return UA_Server_setVariableNode_dataSource(server, nodeId, dataSourceVariable);
 }
\end{lstlisting}

\begin{figure}[h]
    \centering
    \includegraphics[width=0.42\textwidth]{doc/img/OPC_RevPiDO.pdf}
    \caption{Auszug des verwendeten Nodesets, hier Digitalausgang 1 des Versuchsaufbaus
      \label{fig:opc-do}}
\end{figure}

Die in Listing~\ref{lst:4-linkDataSourceVariable} abgebildete Methode \lstinline{linkDataSourceVariable()} erzeugt ein Struct vom Typ \lstinline{UA_DataSource}. In diesem werden dem Lesen und Schreiben einer OPC-Variablen entsprechende Callback-Methoden zugewiesen. Die Verknüpfung einer OPC-Variable, genauer ihrer NodeId, mit der zuvor definierten Datenquelle erfolgt über die von open62541 bereitgestellte Methode \lstinline{UA_Server_setVariableNode_dataSource()}. Vor dem Lesen und nach dem Schreiben dieser Variable werden von nun an die entsprechenden Callbacks aufgerufen.
     
\begin{lstlisting}[language={c},firstnumber=168,caption={Auszug des Callbacks \lstinline{writeDataSourceVariable} in \lstinline{variables.c}\label{lst:4-writeDataSourceVariable}}]  
extern UA_StatusCode
 writeDataSourceVariable(UA_Server *server,
            const UA_NodeId *sessionId, void *sessionContext,
            const UA_NodeId *nodeId, void *nodeContext,
            const UA_NumericRange *range, const UA_DataValue *dataValue) {

    UA_StatusCode retval  = UA_STATUSCODE_GOOD;
    UA_NodeId *nameNodeId = UA_malloc(sizeof(UA_NodeId));
    UA_QualifiedName nameQN = UA_QUALIFIEDNAME(1, "Name");
    UA_Variant nameVar;
    UA_Boolean bit;

    retval |= findSiblingByBrowsename(server, nodeId, &nameQN, nameNodeId);
    retval |= UA_Server_readValue(server, *nameNodeId, &nameVar);
    retval |= UA_Boolean_copy(dataValue->value.data, &bit);

    |>\tikzmarkin[set border color=martinired]{writeIO}<|PI_writeSingleIO(String_fromUA_String(nameVar.data), &bit, false);                                                 |>\tikzmarkend{writeIO}<|

    free(nameNodeId);
    return retval;
 }
\end{lstlisting}

Listing~\ref{lst:4-writeDataSourceVariable} zeigt die Callback-Methode, welche nach dem Schreiben einer Variablen auf dem OPC-Server aufgerufen wird.
Dieser Methode wird neben der NodeId der mit ihr verknüpften Variablen auch der Wert dieser in Form eines Zeigers auf ein Struct vom Typ \lstinline{UA_DataValue} übergeben.

Die Gestaltung des hier verwendeten Nodesets sieht vor, dass in einer OPC-Variablen \lstinline{"Name"} der Bezeichner des zu schreibenden Digitalausgangs hinterlegt ist, siehe Abbildung~\ref{fig:opc-do}. Dies erlaubt eine Rekonfiguration der Ein- und Ausgänge der SPS ohne Änderungen im Programmcode des OPC-Servers vornehmen zu müssen.
Es ist daher erforderlich, nach jedem Schreiben einer mit einem Digitalausgang verknüpften Variablen, hier \lstinline{"Value"}, dessen Bezeichner \lstinline{"Name"} abzufragen. 
Dies geschieht in den Zeilen 180 und 181.
Anschließend wird dieser Bezeichner sowie der zu schreibende Wert der Methode \lstinline{PI_writeSingleIO()} übergeben, welche wiederum die Interaktion mit piControl übernimmt (vgl. Abschnitt \ref{sec:4-picontrol}).
 
\subsection{Integration von piControl%
     \label{sec:4-picontrol}}
In Abschnitt~\ref{sec:2-io} wurde die Anbindung der IO-Module des Revolution Pi sowie die Funktionsweise von piControl aus Anwendersicht beschrieben. Die verfügbare Literatur beschränkt sich auch auf lediglich diese Sicht; eine weiterführende Dokumentation für Entwickler gibt es, neben der in Abschnitt~\ref{sec:3-anbindung} vorgestellten Manpage, nicht. 
In diesem Abschnitt soll daher der Quellcode von piControl sowie dessen Verwendung im Projekt genauer betrachtet werden.
Hierzu wird exemplarisch die in Abschnitt~\ref{sec:4-open62541} eingeführte Methode \lstinline{PI_writeSingleIO()} untersucht.
Diese Methode ermöglicht das Setzen eines einzelnen Bits im Prozessabbild der SPS, und damit das Schalten eines digitalen Ausgangs auf einem IO-Modul.
Die äquivalente Methode \lstinline{int piControlGetBitValue(SPIValue *pSpiValue)} zum Lesen eines Bits bzw. Eingangs funktioniert analog und soll daher an dieser Stelle nicht dediziert erörtert werden.

\begin{lstlisting}[language={c},firstnumber=97,
                   caption={Setzen eines phsikalischen, digitalen Ausgangs in \lstinline{revpi.c}
                   \label{lst:4-PI_writeSingleIO}}]
extern void PI_writeSingleIO(char *pszVariableName, bool *bit, bool verbose)
{
	int rc;
	SPIVariable sPiVariable;
	SPIValue sPIValue;

	strncpy(sPiVariable.strVarName, pszVariableName, sizeof(sPiVariable.strVarName));
	rc = piControlGetVariableInfo(&sPiVariable);
	if (rc < 0) {
		printf("Cannot find variable '%s'\n", pszVariableName);
		return;
	}

		sPIValue.i16uAddress = sPiVariable.i16uAddress;
		sPIValue.i8uBit = sPiVariable.i8uBit;
		sPIValue.i8uValue = *bit;
		rc = |>\tikzmarkin[set border color=martinired]{setBitValue}<|piControlSetBitValue(&sPIValue)|>\tikzmarkend{setBitValue}<|;
		if (rc < 0)
			printf("Set bit error %s\n", getWriteError(rc));
		else if (verbose)
			printf("Set bit %d on byte at offset %d. Value %d\n", sPIValue.i8uBit, sPIValue.i16uAddress,
			       sPIValue.i8uValue);
}
\end{lstlisting}

Der Programmcode in Listing~\ref{lst:4-PI_writeSingleIO} ist Teil des implementierten OPC-Servers. In diesem wird auf zwei Funktionen des piControl-Treibers zugegriffen. 
Beiden Methoden wird als Argument ein Zeiger auf ein Struct vom Typ \lstinline{SPIValue} übergeben. Der im Struct abgelegte Name wird mittels \lstinline{piControlGetVariableInfo(&sPIValue)} zu einer Adresse im Prozessabbild aufgelöst. Diese wird in \lstinline{sPIValue.i16uAdress} gespeichert. Der Wert der Variablen wird anschließend mittels \lstinline{piControlSetBitValue(&sPIValue)} an dieser Adresse in das Prozessabbild geschrieben.

\begin{lstlisting}[language={c},firstnumber=309,caption={Methode \lstinline{piControlSetBitValue} in \lstinline{piControlIf.c}\label{lst:4-piControlSetBitValue}}]
int |>\tikzmarkin[set border color=martiniblue]{setBitValueFcn}<|piControlSetBitValue(SPIValue *pSpiValue)|>\tikzmarkend{setBitValueFcn}<|
{
    piControlOpen();

    if (PiControlHandle_g < 0)
	    return -ENODEV;

    pSpiValue->i16uAddress += pSpiValue->i8uBit / 8;
    pSpiValue->i8uBit %= 8;

    if (|>\tikzmarkin[set border color=martinired]{ioctl}<|ioctl(PiControlHandle_g, KB_SET_VALUE, pSpiValue)|>\tikzmarkend{ioctl}<| < 0)
	    return errno;

    return 0;
}
\end{lstlisting}

Die in Listing~\ref{lst:4-piControlSetBitValue} dargestellte Methode \lstinline{piControlSetBitValue} ist lediglich eine Hüllfunktion (häufig auch als Wrapper-Funktion bezeichnet) für einen Aufruf des \lstinline{ioctl} Kernel-Moduls.
Folgende Parameter werden übergeben:
\lstinline{PiControlHandle_g} ist die Referenz auf die Geräte-Datei des piControl-Treibers. \lstinline{KB_SET_VALUE} ist das ioctl-Kommando zum Schreiben eines Bits in das Prozessabbild. Der Zeiger \lstinline{pSpiValue} verweist auf ein Struct des bereits vorgestellten Typs \lstinline{SPIValue}.

\begin{lstlisting}[language={c},firstnumber=80,caption={Methode \lstinline{piControlOpen} in \lstinline{piControlIf.c}\label{lst:4-piControlOpen}}]
void piControlOpen(void)
{
    /* open handle if needed */
    if (PiControlHandle_g < 0)
    {
	    |>\tikzmarkin[set border color=martiniblue]{PiControlHandle}<|PiControlHandle_g = open(PICONTROL_DEVICE, O_RDWR)|>\tikzmarkend{PiControlHandle}<|;
    }
}
\end{lstlisting}

Die in Listing~\ref{lst:4-piControlOpen} dargestellte Methode öffnet, sofern nicht bereits geschehen, die Geräte-Datei. Das Macro \lstinline{PICONTROL_DEVICE} verweist hierbei auf \lstinline{/dev/piControl0}.

\begin{lstlisting}[language={c},firstnumber=721,caption={Methode \lstinline{piControlIoctl} in \lstinline{piControlMain.c}\label{lst:4-piControlIoctl}}]
static long |>\tikzmarkin[set border color=martiniblue, below offset=0.9em]{piControlIoctl}<|piControlIoctl(struct file *file, unsigned int prg_nr, 
                           unsigned long usr_addr)                                      |>\tikzmarkend{piControlIoctl}<|
{
  int status = -EFAULT;
  tpiControlInst *priv;
  int timeout = 10000;	// ms

  if (prg_nr != KB_CONFIG_SEND && prg_nr != KB_CONFIG_START && !isRunning()) {
  	return -EAGAIN;
  }

  priv = (tpiControlInst *) file->private_data;

  if (prg_nr != KB_GET_LAST_MESSAGE) {
  	// clear old message
  	priv->pcErrorMessage[0] = 0;
  }

  switch (prg_nr) {|>\setcounter{lstnumber}{864}<|

    case |>\tikzmarkin[set border color=martiniblue]{KB_SET_VALUE}<|KB_SET_VALUE:|>\tikzmarkend{KB_SET_VALUE}<|
  		{
  			SPIValue *pValue = (SPIValue *) usr_addr;

  			if (!isRunning())
  				return -EFAULT;

  			if (pValue->i16uAddress >= KB_PI_LEN) {
  				status = -EFAULT;
  			} else {
  				INT8U i8uValue_l;
  				my_rt_mutex_lock(&piDev_g.lockPI);
  				i8uValue_l = piDev_g.ai8uPI[pValue->i16uAddress];

  				if (pValue->i8uBit >= 8) {
  					i8uValue_l = pValue->i8uValue;
  				} else {
  					if (pValue->i8uValue)
  						i8uValue_l |= (1 << pValue->i8uBit);
  					else
  						i8uValue_l &= ~(1 << pValue->i8uBit);
  				}

  				|>\tikzmarkin[set border color=martinired]{i8uValue}<|piDev_g.ai8uPI[pValue->i16uAddress] = i8uValue_l;|>\tikzmarkend{i8uValue}<|
  				rt_mutex_unlock(&piDev_g.lockPI);

  #ifdef VERBOSE
  				pr_info("piControlIoctl Addr=%u, bit=%u: %02x %02x\n", pValue->i16uAddress, pValue->i8uBit, pValue->i8uValue, i8uValue_l);
  #endif

  				status = 0;
  			}
  		}
  		break; |>\setcounter{lstnumber}{1314}<|

    default:
      pr_err("Invalid Ioctl");
      return (-EINVAL);
      break;

    }

    return status;
  }
\end{lstlisting}

Listing~\ref{lst:4-piControlIoctl} zeigt in Auszügen die ioctl-Methode des piControl Kernel-Treibers. Diese bekommt folgende Argumente übergeben: \lstinline{struct file *file} enthält den Verweis auf die Geräte-Datei, hier \lstinline{/dev/piControl0}. Der Wert von \lstinline{unsigned int prg_nr} beschreibt die Anfrage an den Treiber, in diesem Fall \lstinline{KB_SET_VALUE}. Das Argument \lstinline{unsigned long usr_addr} enthält einen typ-agnostischen Pointer. Dieser verweist auf einen Speicherbereich, in welchem die zur Bearbeitung der Anfrage notwendigen Daten abgelegt sind. Hier können auch vom Treiber empfangene Daten dem Anwendungsprogramm bereitgestellt werden. 

Die switch-case-Anweisung führt die über das Argument \lstinline{prg_nr} spezifizierte Aktion aus. Hier betrachten wir \lstinline{KB_SET_VALUE}:
Zunächst wird in Zeile 868 der übergebene Zeiger \lstinline{usr_addr} mittels explizitem Typecast zu einem Zeiger des Typs \lstinline{SPIValue *} konvertiert. Da dieser auf Daten im Userspace verweist, ist beim Zugriff durch den Kernel-Treiber besondere Vorsicht geboten.
In Zeile 877 wird mittels Mutex das Prozessabbild \lstinline{piDev_g} für den Zugriff durch andere Threads oder Prozesse gesperrt.
\lstinline{my_rt_mutex_lock} verweist hierbei auf die Funktion \lstinline{rt_mutex_lock} aus \lstinline{linux/sched.h}\footnote{Offenbar wurde hier auch eine alternative Implementierung vorgesehen, siehe revpi\_common.h}

In Zeile 889 wird das Byte \lstinline{i8uValue_l}, welches den zu schreibenden Wert enthält in das Prozessabbild übertragen. Anschließend wird die Mutex auf \lstinline{piDev_g} wieder entsperrt.
\newpage

\begin{lstlisting}[language={c},firstnumber=62,caption={Auszug des Struct \lstinline{spiControlDev} in \lstinline{piControlMain.h}\label{lst:4-spiControlDev}}]
|>\tikzmarkin[set border color=martiniblue]{spiControlDev}<|typedef struct spiControlDev|>\tikzmarkend{spiControlDev}<| {
	// device driver stuff
	int init_step;
	enum revpi_machine machine_type;
	void *machine;
	struct cdev cdev;	// Char device structure
	struct device *dev;
	struct thermal_zone_device *thermal_zone;

	|>\tikzmarkin[set border color=martiniblue]{processImage}<|// process image stuff
	INT8U ai8uPI[KB_PI_LEN];
	INT8U ai8uPIDefault|>\tikzmarkin[set border color=martinired]{KB_PI_LEN_0}<|[KB_PI_LEN]|>\tikzmarkend{KB_PI_LEN_0}<|;
	struct rt_mutex lockPI;        |>\tikzmarkend{processImage}<|
	bool stopIO;
	piDevices *devs; |>\setcounter{lstnumber}{94}<|
} tpiControlDev;
\end{lstlisting}

Das Prozessabbild ist als Byte-Array der Länge \lstinline{KB_PI_LEN} in Listing~\ref{lst:4-spiControlDev} definiert. Konfigurationsparameter wie \lstinline{KB_PI_LEN} oder die Zykluszeit für den Datenaustausch zwischen SPS und IO-Modulen sind im folgenden Listing~\ref{lst:4-process} definiert.

\begin{lstlisting}[language={c},firstnumber=119,caption={Konfigurationsparameter des Prozessabbildes in project.h\label{lst:4-process}}]
#define INTERVAL_PI_GATE (5*1000*1000)  // 5 ms piGateCommunication |>\setcounter{lstnumber}{128}<|

#define INTERVAL_IO_COM (5*1000*1000)  // 5 ms piIoComm |>\setcounter{lstnumber}{132}<|

#define KB_PD_LEN       512
|>\tikzmarkin[set border color=martiniblue]{KB_PI_LEN_1}<|#define KB_PI_LEN       4096|>\tikzmarkend{KB_PI_LEN_1}<|
\end{lstlisting}

Das zu setzende Bit wurde zu diesem Zeitpunkt erfolgreich in das Prozessabbild der SPS geschrieben.
Es stellt sich die Frage, wie dieses nun an das IO-Modul kommuniziert wird.
Die Kommunikation mit allen angebundenen Modulen ist ebenfalls Aufgabe des piControl-Treibers.

\begin{lstlisting}[language={c},firstnumber=256,caption={Auszug der Methode \lstinline{piIoThread} in \lstinline{revpi_core.c}\label{lst:4-piIoThread}}]
static int piIoThread(void *data)
{
	//TODO int value = 0;
	ktime_t time;
	ktime_t now;
	s64 tDiff;

	hrtimer_init(&piCore_g.ioTimer, CLOCK_MONOTONIC, HRTIMER_MODE_ABS);
	piCore_g.ioTimer.function = piIoTimer;

	pr_info("piIO thread started\n");

	now = hrtimer_cb_get_time(&piCore_g.ioTimer);

	PiBridgeMaster_Reset();

	while (!kthread_should_stop()) {
		if (|>\tikzmarkin[set border color=martinired]{PiBridgeMaster}<|PiBridgeMaster_Run()|>\tikzmarkend{PiBridgeMaster}<| < 0)
			break;
	}

	RevPiDevice_finish();

	pr_info("piIO exit\n");
	return 0;
}
\end{lstlisting}

Der Kernel-Thread \lstinline{piIoThread} ist verantwortlich für den zyklischen Datenaustausch mit den IO-Modulen. In diesem wird fortlaufend die Methode \lstinline{PiBridgeMaster_Run()} aufgerufen, siehe Listing~\ref{lst:4-piIoThread}.

\begin{lstlisting}[language={c},firstnumber=262,caption={Auszug der Methode \lstinline{PiBridgeMaster_Run(void)} in \lstinline{RevPiDevice.c}\label{lst:4-PiBridgeMaster_Run}}]
int PiBridgeMaster_Run(void)
{
	static kbUT_Timer tTimeoutTimer_s;
	static kbUT_Timer tConfigTimeoutTimer_s;
	static int error_cnt;
	static INT8U last_led;
	static unsigned long last_update;
	int ret = 0;
	int i;

	my_rt_mutex_lock(&piCore_g.lockBridgeState);
	if (piCore_g.eBridgeState != piBridgeStop) {
		switch (eRunStatus_s) { |>\setcounter{lstnumber}{514}<|
		    case enPiBridgeMasterStatus_EndOfConfig:|>\setcounter{lstnumber}{621}<|
		    if (|>\tikzmarkin[set border color=martinired]{RevPiDevice}<|RevPiDevice_run()|>\tikzmarkend{RevPiDevice}<|) {
				// an error occured, check error limits |>\setcounter{lstnumber}{641}<|
			} else {
				ret = 1;
			}
			piCore_g.image.drv.i16uRS485ErrorCnt = RevPiDevice_getErrCnt();
			break;
\end{lstlisting}

Die in Listing~\ref{lst:4-PiBridgeMaster_Run} dargestellte Methode ist eine sog. State-Machine. Ist die Konfiguration der IO-Module erfolgreich abgeschlossen, so führt sie bei Aufruf lediglich die Methode \lstinline{RevPiDevice_run()} aus.

\begin{lstlisting}[language={c},firstnumber=140,caption={Auszug der Methode \lstinline{RevPiDevice_run(void)} in \lstinline{RevPiDevice.c}\label{lst:4-RevPiDevice_run}}]
int RevPiDevice_run(void)
{
	INT8U i8uDevice = 0;
	INT32U r;
	int retval = 0;

	RevPiDevices_s.i16uErrorCnt = 0;

	for (i8uDevice = 0; i8uDevice < RevPiDevice_getDevCnt(); i8uDevice++) {
		if (RevPiDevice_getDev(i8uDevice)->i8uActive) {
			switch (RevPiDevice_getDev(i8uDevice)->sId.i16uModulType) {
			case KUNBUS_FW_DESCR_TYP_PI_DIO_14:
			case KUNBUS_FW_DESCR_TYP_PI_DI_16:
			case KUNBUS_FW_DESCR_TYP_PI_DO_16:
				r = |>\tikzmarkin[set border color=martinired]{sendCyclicTelegram}<|piDIOComm_sendCyclicTelegram(i8uDevice)|>\tikzmarkend{sendCyclicTelegram}\setcounter{lstnumber}{166} <|;

				break; |>\setcounter{lstnumber}{216}<|
			}
		}
	} |>\setcounter{lstnumber}{227}<|
	return retval;
}
\end{lstlisting}

Diese iteriert wie in Listing~\ref{lst:4-RevPiDevice_run} abgebildete durch alle gegenwärtig in der SPS konfigurierten Module. Ist das aktuelle Modul als aktiv markiert, so wird anhand eines sog. Firmware-Descriptors entschieden, welche Methode für die Ansteuerung des Moduls aufzurufen ist.

\begin{lstlisting}[language={c},firstnumber=161,caption={Auszug der Methode \lstinline{piDIOComm_sendCyclicTelegram} in \lstinline{piDIOComm.c}\label{lst:4-sendCyclicTelegram}}]
INT32U piDIOComm_sendCyclicTelegram(INT8U i8uDevice_p)
{
	INT32U i32uRv_l = 0;
	SIOGeneric sRequest_l;
	SIOGeneric sResponse_l;
	INT8U len_l, data_out[18], i, p, data_in[70];
	INT8U i8uAddress;
	int ret; |>\setcounter{lstnumber}{239}<|
	
    |>\tikzmarkin[set border color=martinired]{piIoComm}<|ret = piIoComm_send((INT8U *) & sRequest_l, IOPROTOCOL_HEADER_LENGTH + len_l + 1);  |>\tikzmarkend{piIoComm}\setcounter{lstnumber}{298}<|
}
\end{lstlisting}

Im Falle des hier verwendeten DO-Moduls wird die in Listing~\ref{lst:4-sendCyclicTelegram} abgebildete Methode \lstinline{piDIOComm_sendCyclicTelegram()} aufgerufen. Dieser wird ein Zeiger auf das zu schreibende Gerät übergeben. 
Zunächst wird das Prozessabbild mittels eines proprietären, jedoch im Quellcode offen nachvollziehbaren Protokolls in ein \lstinline{sRequest_l} genanntes Byte-Array umgewandelt. Dieser Schritt ist in Listing~\ref{lst:4-sendCyclicTelegram} nicht abgebildet. Anschließend wird \lstinline{piIoComm_send()} ein Zeiger auf die so generierte Schreib-Anfrage übergeben.

\begin{lstlisting}[language={c},firstnumber=220,caption={Auszug der Methode \lstinline{piIOComm_send} in \lstinline{piIOComm.c}\label{lst:4-piIOComm_send}}]
int piIoComm_send(INT8U * buf_p, INT16U i16uLen_p)
{
	ssize_t write_l = 0;
	INT16U i16uSent_l = 0;|>\setcounter{lstnumber}{249}<|

	while (i16uSent_l < i16uLen_p) {
		write_l = vfs_write(piIoComm_fd_m, buf_p + i16uSent_l, i16uLen_p - i16uSent_l, &piIoComm_fd_m->f_pos);
		if (write_l < 0) {
			pr_info_serial("write error %d\n", (int)write_l);
			return -1;
		} 
		i16uSent_l += write_l;|>\setcounter{lstnumber}{263}<|
	}
	clear();
	vfs_fsync(piIoComm_fd_m, 1);
	return 0;
}
\end{lstlisting}

Listing~\ref{lst:4-piIOComm_send} zeigt die Implementierung von \lstinline{piIoComm_send()}. Diese Methode ist für das Schreiben der oben generierten Anfrage auf die seriellen Schnittstelle verantwortlich. Realisiert wird dies mittels der Methode \lstinline{vfs_write()}. Diese ist in \lstinline{<linux/fs.h>} definiert. Sie ermöglicht das Schreiben einer Datei im Userspace aus dem Kernel heraus. Geschrieben wird hier die Datei mit dem Deskriptor \lstinline{piIoComm_fd_m}.
Da die Funktion \lstinline{vfs_write()} durch andere Kernel-Tasks unterbrochen werden kann, ist nicht gewährleistet, dass die gesamte Anfrage mit nur einem Aufruf geschrieben wird. Die oben abgebildete while-Schleife stellt das vollständige Senden der Anfrage sicher.

\begin{lstlisting}[language={c},firstnumber=157,caption={Auszug der Methode \lstinline{piIOComm_open_serial} in \lstinline{piIOComm.c}\label{lst:4-piIOComm_open_serial}}]
int piIoComm_open_serial(void)
{   |>\setcounter{lstnumber}{167}<|
	struct file *fd;	/* Filedeskriptor */
	struct termios newtio;	/* Schnittstellenoptionen */

	|>\tikzmarkin[set border color=martiniblue]{fd}<|/* Port oeffnen - read/write, kein "controlling tty", 
	    Status von DCD ignorieren */
	fd = filp_open(|>\tikzmarkin[set border color=martinired]{tty}<|REV_PI_TTY_DEVICE|>\tikzmarkend{tty}<|, O_RDWR | O_NOCTTY, 0); |>\setcounter{lstnumber}{208}<|
	
	piIoComm_fd_m = fd;                                                      |>\tikzmarkend{fd}\setcounter{lstnumber}{217}<|

	return 0;
}
\end{lstlisting}

Der zum Schreiben auf die serielle Schnittstelle verwendete Datei-Deskriptor wird von der in Listing~\ref{lst:4-piIOComm_open_serial} abgebildeten Methode \lstinline{piIoComm_open_serial()} generiert. 

\begin{lstlisting}[language={c},firstnumber=45,caption={Definition der seriellen Schnittstelle in \lstinline{piIOComm.h}\label{lst:4-REV_PI_TTY_DEVICE}}]
#define REV_PI_TTY_DEVICE	"/dev/ttyAMA0"
\end{lstlisting}

Das in Listing~\ref{lst:4-REV_PI_TTY_DEVICE} definierte Macro verweist auf eine der seriellen Schnittstellen des RaspberryPi.
Die Implementierung des zugehörigen Schnittstellentreibers soll hier nicht weiter untersucht werden. Somit ist an dieser Stelle die Kette vom Setzen einer Variablen auf dem OPC-Server bis hin zur Aktualisierung des Prozessabbilds der IO-Module geschlossen.

% \begin{lstlisting}[language={c},firstnumber={226},caption={Setzen der Scheduler-Priorität auf SCHED\_FIFO in 
% revpi\_common.c\label{lst:2-sched_priority}}]
% param.sched_priority = ktprio->prio;
% ret = sched_setscheduler(child, SCHED_FIFO, &param);
% \end{lstlisting}
% % % Imports nur für Referenzenauflösung während des Schreibens! Vorm Kompilieren auskommentieren!
% \bibliography{0_hauptdatei}
% \input{1_einleitung}
% \input{2_grundlagen}
% \input{3_konzeption}
% \input{4_implementierung}
% \input{5_tests}
% \input{6_zusammenfassung}
% % Ende Imports

\section{Test des OPC-Servers im Gesamtsystem%
  \label{sec:5-tests}}

% % % Imports nur für Referenzenauflösung während des schreibens! Vorm Kompilieren auskommentieren!
% \bibliography{0_hauptdatei}
% \input{1_einleitung}
% \input{2_grundlagen}
% \input{3_konzeption}
% \input{4_implementierung}
% \input{5_tests}
% \input{6_zusammenfassung}
% % Ende Imports

\section{Zusammenfassung und Ausblick%
  \label{sec:6-fazit}}
Der folgende Abschnitt~\ref{sec:6-zusammenfassung} fasst die gewonnenen Erkenntnisse und den Stand der Implementierung zusammen.
Den Abschluss dieser Arbeit bildet der Ausblick in Abschnitt~\ref{sec:6-ausblick}.

\subsection{Zusammenfassung%
     \label{sec:6-zusammenfassung}}

\subsection{Ausblick%
     \label{sec:6-ausblick}}

% % Ende Imports

\section{Zusammenfassung und Ausblick%
  \label{sec:6-fazit}}
Der folgende Abschnitt~\ref{sec:6-zusammenfassung} fasst die gewonnenen Erkenntnisse und den Stand der Implementierung zusammen.
Den Abschluss dieser Arbeit bildet der Ausblick in Abschnitt~\ref{sec:6-ausblick}.

\subsection{Zusammenfassung%
     \label{sec:6-zusammenfassung}}

\subsection{Ausblick%
     \label{sec:6-ausblick}}

% % Ende Imports

\section{Einleitung und Motivation%
  \label{sec:1-einleitung}}
Ziel dieses Projektes ist die Integration eines OPC-Servers mit einer auf Linux
basierenden speicherprogrammierbaren Steuerung (SPS). Angeschlossen an diese SPS
ist jeweils ein digitales Ein-/\,bzw.~Ausgabemodul. Die von diesen bereitgestellten
Ein-/\, bzw.~Ausgänge (IO) sollen in der Datenstruktur des OPC-Servers abgebildet
und über diesen für OPC-Clients les-/\,und schreibar sein. Weiterhin sollen einige
Funktionen zur Überwachung und Steuerung der an die SPS angeschlossenen Aktoren
und Sensoren direkt im OPC-Server implementiert werden.
Hiermit stellt dieses Projekt eine der Grundlagen für ein übergeordnetes Projekt,
die cloudbasierte Steuerung eines miniaturisierten Produktions-Systems, dar.

Der hier verwendete OPC-Server ist Teil des sog. open62541 Projekts. Er ist in C
geschrieben und implementiert bereits einen großen Teil der im OPC-UA-Standard
spezifizierten Funktionen.
Als SPS findet ein Revolution Pi 3 der Firma Kunbus Verwendung. Dieser integriert
ein sog. Compute Module der Raspberry Pi Foundation in ein industrietaugliches
Gehäuse und erlaubt die Erweiterung mittels IO- oder Gateway-Modulen. Über diese
erfolgt die Kommunikation mit weiteren Komponenten der Automatisierungstechnik.

Motiviert ist dieses Projekt durch die Beobachtung, dass die Verbreitung offener
Standards sowie freier Software auch in der Automatisierungstechnik zunimmt.
Linux ist ein freies Betriebssystem, OPC-UA ein offen zugänglicher, aktiv gepflegter
und weit verbreiteter Standard. Der Raspberry Pi findet sowohl bei Hobby-Anwendern als
auch in den Bereichen Forschung und Entwicklung sowie bei industriellen Anwendern
Verwendung. Dieses Projekt stellt somit eine für unterschiedliche Anwender interessante
Entwicklung dar.

Im Anschluss an diese einleitende Übersicht im Abschnitt~\ref{sec:1-einleitung} folgt
die Darstellung der wichtigsten Grundlagen in Abschnitt~\ref{sec:2-grundlagen}.
Aufbauend auf diesen Grundlagen folgt die konzeptuelle Ausarbeitung im Abschnitt~\ref{sec:3-konzeption}.
Die Umsetzung wird im Abschnitt~\ref{sec:4-implementierung} erläutert.
Die Leistungsfähigkeit der Implementierung wird in Abschnitt~\ref{sec:5-tests} untersucht.
Eine Zusammenfassung und ein Ausblick schließen die Arbeit in
Abschnitt~\ref{sec:6-fazit} ab. Eventuell noch benötigte Anhänge
finden sich in den Anhängen [...] bis [...].

% % % Imports nur für Referenzenauflösung während des Schreibens! Vorm Kompilieren auskommentieren!
% \bibliography{0_hauptdatei}
% % Mit \section{...} eröffnen wir einen neuen Abschnitt.
% Der Befehl setzt nicht nur den Text in einer größeren,
% fetten Schrift, sondern sorgt außerdem dafür, daß er im
% Inhaltsverzeichnis erscheint.
%
% Mit \label{...} erzeugen wir einen Bezeichner, mit dessen Hilfe
% wir später auf die Nummer des Abschnitts verweisen können (nämlich
% mit~\ref{...}).
%
% Das Kommentarzeichen hinter „Übersicht“ dient dazu, ein
% Leerzeichen zwischen „Übersicht“ und dem \label-Befehl
% zu vermeiden, das andernfalls sichtbar würde – z.B. im
% Inhaltsverzeichnis.
%

% % Imports nur für Referenzenauflösung während des Schreibens! Vorm Kompilieren auskommentieren!
% \bibliography{0_hauptdatei}
% % Mit \section{...} eröffnen wir einen neuen Abschnitt.
% Der Befehl setzt nicht nur den Text in einer größeren,
% fetten Schrift, sondern sorgt außerdem dafür, daß er im
% Inhaltsverzeichnis erscheint.
%
% Mit \label{...} erzeugen wir einen Bezeichner, mit dessen Hilfe
% wir später auf die Nummer des Abschnitts verweisen können (nämlich
% mit~\ref{...}).
%
% Das Kommentarzeichen hinter „Übersicht“ dient dazu, ein
% Leerzeichen zwischen „Übersicht“ und dem \label-Befehl
% zu vermeiden, das andernfalls sichtbar würde – z.B. im
% Inhaltsverzeichnis.
%

% % Imports nur für Referenzenauflösung während des Schreibens! Vorm Kompilieren auskommentieren!
% \bibliography{0_hauptdatei}
% \input{1_einleitung}
%\input{2_grundlagen}
%\input{3_konzeption}
%\input{4_implementierung}
%\input{5_tests}
%\input{6_zusammenfassung}
% % Ende Imports

\section{Einleitung und Motivation%
  \label{sec:1-einleitung}}
Ziel dieses Projektes ist die Integration eines OPC-Servers mit einer auf Linux
basierenden speicherprogrammierbaren Steuerung (SPS). Angeschlossen an diese SPS
ist jeweils ein digitales Ein-/\,bzw.~Ausgabemodul. Die von diesen bereitgestellten
Ein-/\, bzw.~Ausgänge (IO) sollen in der Datenstruktur des OPC-Servers abgebildet
und über diesen für OPC-Clients les-/\,und schreibar sein. Weiterhin sollen einige
Funktionen zur Überwachung und Steuerung der an die SPS angeschlossenen Aktoren
und Sensoren direkt im OPC-Server implementiert werden.
Hiermit stellt dieses Projekt eine der Grundlagen für ein übergeordnetes Projekt,
die cloudbasierte Steuerung eines miniaturisierten Produktions-Systems, dar.

Der hier verwendete OPC-Server ist Teil des sog. open62541 Projekts. Er ist in C
geschrieben und implementiert bereits einen großen Teil der im OPC-UA-Standard
spezifizierten Funktionen.
Als SPS findet ein Revolution Pi 3 der Firma Kunbus Verwendung. Dieser integriert
ein sog. Compute Module der Raspberry Pi Foundation in ein industrietaugliches
Gehäuse und erlaubt die Erweiterung mittels IO- oder Gateway-Modulen. Über diese
erfolgt die Kommunikation mit weiteren Komponenten der Automatisierungstechnik.

Motiviert ist dieses Projekt durch die Beobachtung, dass die Verbreitung offener
Standards sowie freier Software auch in der Automatisierungstechnik zunimmt.
Linux ist ein freies Betriebssystem, OPC-UA ein offen zugänglicher, aktiv gepflegter
und weit verbreiteter Standard. Der Raspberry Pi findet sowohl bei Hobby-Anwendern als
auch in den Bereichen Forschung und Entwicklung sowie bei industriellen Anwendern
Verwendung. Dieses Projekt stellt somit eine für unterschiedliche Anwender interessante
Entwicklung dar.

Im Anschluss an diese einleitende Übersicht im Abschnitt~\ref{sec:1-einleitung} folgt
die Darstellung der wichtigsten Grundlagen in Abschnitt~\ref{sec:2-grundlagen}.
Aufbauend auf diesen Grundlagen folgt die konzeptuelle Ausarbeitung im Abschnitt~\ref{sec:3-konzeption}.
Die Umsetzung wird im Abschnitt~\ref{sec:4-implementierung} erläutert.
Die Leistungsfähigkeit der Implementierung wird in Abschnitt~\ref{sec:5-tests} untersucht.
Eine Zusammenfassung und ein Ausblick schließen die Arbeit in
Abschnitt~\ref{sec:6-fazit} ab. Eventuell noch benötigte Anhänge
finden sich in den Anhängen [...] bis [...].

%% % Imports nur für Referenzenauflösung während des Schreibens! Vorm Kompilieren auskommentieren!
% \bibliography{0_hauptdatei}
% \input{1_einleitung}
% \input{2_grundlagen}
% \input{3_konzeption}
% \input{4_implementierung}
% \input{5_tests}
% \input{6_zusammenfassung}
% % Ende Imports

\section{Grundlagen%
  \label{sec:2-grundlagen}}

\subsection{Speicherprogrammierbare-Steuerung und Linux -- Revolution Pi%
     \label{sec:2-sps}}

\subsubsection{Kunbus RevolutionPi%
        \label{sec:2-revpi}}
Der RevolutionPi 3 ist eine speicherprogrammierbare Steuerung (SPS) des Herstellers
Kunbus GmbH. Kern dieser SPS ist das von der Raspberry Pi Foundation entwickelte
und vertriebene Raspberry Pi Compute Module 3. Dieses integriert ein Broadcom BCM2837
System-on-Chip (SoC) mit vier 1,2GHz Prozessorkernen, 1GB RAM, 4GB eMMC Anwendungsspeicher
und sonstige Peripherie in ein Modul im DDR2-SODIMM Formfaktor. Diese Spezifikationen
sind weitgehend identisch zu denen des ausgesprochen populären Raspberry Pi 3.
Der Revolution Pi profitiert daher von dem gleichen großen Angebot an Software
und Unterstützung wie der Raspberry Pi, ergänzt dessen Hardware jedoch um eine 24V
Spannungsversorgung, die Möglichkeit der Erweiterung durch mehrere industrietaugliche
Ein-/ Ausgabemodule und Gateways sowie ein Gehäuse zur Montage auf einer DIN-Schiene.
\begin{itemize}
  \item{Prozessor: BCM2837}
  \item{Taktfrequenz 1,2 GHz}
  \item{Anzahl Prozessorkerne: 4}
  \item{Arbeitsspeicher: 1 GByte}
  \item{eMMC Flash Speicher: 4 GByte}
  \item{Betriebssystem: Angepasstes Raspbian mit RT-Patch}
  \item{RTC mit 24h Pufferung über wartungsfreien Kondensator}
  \item{Treiber / API: Treiber schreibt zyklisch Prozessdaten in ein Prozessabbild, Zugriff auf Prozessabbild über Linux-Filesystem als API zu Fremdsoftware.}
  \item{Kommunikationsanschlüsse: 2 x USB 2.0 A (je 500 mA belastbar), 1 x Micro-USB, HDMI, Ethernet (RJ45) 10/100 Mbit/s}
  \item{Stromversorgung: min. 10,7 V, max. 28,8 V, maximal 10 Watt}
  \item{Zulässige Umgebungstemperatur: -40 bis +55 C}
  \item{Gehäuseabmessungen: (HxBxL) 96 mm x 22,5 mm x 110,5 mm (ohne gesteckte Stecker)}
  \item{ESD Schutz: 4 kV / 8 kV gemäß EN61131-2 und IEC 61000-6-2}
  \item{Surge / Burst Prüfungen: gemäß EN61131-2 und IEC 61000-6-2 eingekoppelt auf Versorgungsspannung, Ethernet und IO-Leitungen}
  \item{EMI Prüfungen: gemäß EN61131-2 und IEC 61000-6-2}
\end{itemize}

Kunbus bietet eine Auswahl an IO- und Gateway-Modulen zur Erweiterung des Revolution Pi an.
Gateways dienen der Kommunikation mit Systemen oder Komponenten der Automatisierungstechnik
über Protokolle wie PROFIBUS oder EtherCAT. IO-Module erlauben die Überwachung
und Steuerung von digitalen oder analogen Ein- und Ausgängen.

\subsubsection{Zugriff auf IO-Module%
        \label{sec:2-io}}
Der Zugriff auf die Ein- und Ausgänge der IO-Module erfolgt über ein Prozessabbild
und einen hierfür von Kunbus bereitgestellten Treiber, genannt piControl. Dieser
aktualisiert das Prozessabbild zyklisch. Die angestrebte Zykluszeit beträgt 5ms,
kann jedoch je nach Anzahl der angeschlossenen Module auch größer sein. Kunbus
garantiert bei drei IO-Modulen und zwei Gateway-Modulen eine Zykluszeit von 10 ms.
Jedes der IO-Module stellt ein eigenständiges eingebettetes System dar. Es verfügt
über einen Microcontroller, welcher die IOs bereitstellt und über einen RS485-Bus
mit dem Revolution Pi kommuniziert.
% https://revolution.kunbus.de/io-modul/

Lizenz: GPL
% https://github.com/RevolutionPi/piControl

\begin{lstlisting}[language={c},firstnumber={226},caption={Setzen der Scheduler-Priorität auf SCHED\_FIFO in revpi\_common.c\label{lst:2-sched_priority}}]
param.sched_priority = ktprio->prio;
ret = sched_setscheduler(child, SCHED_FIFO,
       &param);
\end{lstlisting}


\subsection{Echtzeit und Multithreading unter Linux -- preemptRT und posix%
     \label{sec:2-echtzeit}}


 Der Linux-Kernel verfügt über mehrere unterschiedliche Preemtion-Modelle:

\begin{itemize}
  \item No Forced Preemption (server):
  Ausgelegt auf maximal möglichen Durchsatz, lediglich Interrupts und
  System-Call-Returns bewirken Präemption.

  \item Voluntary Kernel Preemption (Desktop):
  Neben den implizit bevorrechtigten Interrupts und System-Call-Returns gibt es
  in diesem Modell weitere Abschnitte des Kernels in welchen Preämption explizit
  gestattet ist.

  \item Preemptible Kernel (Low-Latency Desktop):
  In diesem Modell ist der gesamte Kernel, mit Ausnahme sog.~kritischer Abschnitte
  präemptible. Nach jedem kritischen Abschnitt gibt es einen impliziten Präemptions-Punkt.

  \item Preemptible Kernel (Basic RT):
  Dieses Modell ist dem zuvor genannten sehr ähnlich, hier sind jedoch alle Interrupt-Handler
  als eigenständige Threads ausgeführt.

  \item Fully Preemptible Kernel (RT):
  Wie auch bei den beiden zuvor genannten Modellen ist hier der gesamte Kernel
  präemtible, die Anzahl und Dauer der nicht-präemtiblen kritischen Abschnitte
  ist auf ein notwendiges Minimum beschränkt. Alle Interrupt-Handler sind als
  eigenständige Threads ausgeführt, Spinlocks durch Sleeping-Spinlocks und Mutexe
  durch sog.~RT-Mutexe ersetzt.

\end{itemize}
\todo{Spinlocks und Mutexe sowie die RT-Varianten dieser erklären!}

Lediglich mit dem vollständig präemtiblen Kernel kann Echtzeit-Verhalten realisiert werden.

% https://wiki.linuxfoundation.org/realtime/documentation/technical_basics/preemption_models bzw kernel/Kconfig.preempt

\subsubsection{preemptRT%
        \label{sec:2-preemptRT}}
% https://wiki.linuxfoundation.org/realtime/documentation/technical_details/start
% https://wiki.linuxfoundation.org/realtime/documentation/technical_basics/start

Das dem PREEMPT RT Kernel zugrunde liegende Prinzip lässt sich in einer einfachen
Regel ausdrücken: Nur Code, welcher absolut nicht-präemtible sein darf, ist es
gestattet nicht-präemtible zu sein.
Das erklärte Ziel des PREEMPT\_RT Patches ist es folglich, die Menge des nicht-präemtiblen
Codes im Linux-Kernel auf das absolut notwendige Minimum zu reduzieren.

Dies wird durch Verwendung folgender Mechanismen erreicht:

\begin{itemize}
  \item Hochauflösende Timer
  \item Sleeping Spinlocks
  \item Threaded Interrupt Handlers
  \item rt\_mutex
  \item RCU
\end{itemize}


\subsubsection{posix%
        \label{sec:2-posix}}
Ist posix hier wirklich relevant? Debian bzw.~Raspbian sind weitgehend posix
kompatibel, aber wird es hier genutzt? -> JA, open62541 nutzt pthread.h
piControl nutzt kthread.h, und semaphore.h

\subsection{OPC-UA und open62541%
     \label{sec:2-opc}}

\subsubsection{OPC UA%
        \label{sec:2-opcua}}
Open Platform Communications (OPC) ist eine Familie von Standards zur herstellerunabhängigen
Kommunikation von Maschinen (M2M) in der Automatisierungstechnik. Die sog.~OPC Task Force, zu deren
Mitgliedern verschiedene große Firmen der Automatisierungsindustrie gehören, veröffentlichte
die OPC Specification Version 1.0 im August 1996.
Motiviert ist dieser offene Standard durch die Erkenntniss, dass die Anpassung der
zahlreichen Herstellerstandards an individuelle Infrastrukturen und Anlagen einen
großen Mehraufwand verursachen.
Die Wikipedia beschreibt das Anwendungsgebiet für OPC wie folgt:

\glqq{}OPC wird dort eingesetzt, wo Sensoren, Regler und Steuerungen verschiedener Hersteller
ein gemeinsames Netzwerk bilden. Ohne OPC benötigten zwei Geräte zum Datenaustausch
genaue Kenntnis über die Kommunikationsmöglichkeiten des Gegenübers. Erweiterungen
und Austausch gestalten sich entsprechend schwierig. Mit OPC genügt es, für jedes
Gerät genau einmal einen OPC-konformen Treiber zu schreiben. Idealerweise wird
dieser bereits vom Hersteller zur Verfügung gestellt. Ein OPC-Treiber lässt sich
ohne großen Anpassungsaufwand in beliebig große Steuer- und Überwachungssysteme
integrieren.

OPC unterteilt sich in verschiedene Unterstandards, die für den jeweiligen Anwendungsfall
unabhängig voneinander implementiert werden können. OPC lässt sich damit verwenden
für Echtzeitdaten (Überwachung), Datenarchivierung, Alarm-Meldungen und neuerdings
auch direkt zur Steuerung (Befehlsübermittlung).\grqq{}

OPC basiert in der ursprünglichen Spezifikation auf Microsofts DCOM-Spezifikation.
DCOM macht Funktionen und Objekte einer Anwendung anderen Anwendungen im Netzwerk
zugänglich. Der OPC-Standard definiert entsprechende DCOM-Objekte um mit anderen
OPC-Anwendungen Daten austauschen zu können. Die Verwendung von DCOM bindet Anwender
an Betriebssysteme von Microsoft. Die ursprüngliche OPC Spezifikation wird durch die
Entwicklung von OPC Unified Architecture (OPC UA) abgelöst.
OPC UA setzt auf einem eigenen Kommunikationionsstack auf, die Verwendung von DCOM
und damit die Bindung an Microsoft wurden aufgelöst.

Die OPC-UA-Architektur ist eine Service-orientierte Architektur (SOA), deren Struktur
aus mehreren Schichten besteht.

% Wikipedia
Das OPC-Informationsmodell ist nicht mehr nur eine Hierarchie aus Ordnern, Items
und Properties. Es ist ein sogenanntes Full-Mesh-Network aus Nodes, mit dem neben
den Nutzdaten eines Nodes auch Meta- und Diagnoseinformationen repräsentiert werden.
Ein Node ähnelt einem Objekt aus der objektorientierten Programmierung. Ein Node
kann Attribute besitzen, die gelesen werden können (Data Access (DA), Historical
Data Access (HDA)). Es ist möglich Methoden zu definieren und aufzurufen.
Eine Methode besitzt Aufrufargumente und Rückgabewerte. Sie wird durch ein Command
aufgerufen. Weiterhin werden Events unterstützt, die versendet werden können
(AE (Alarms \& Events), DA DataChange), um bestimmte Informationen zwischen Geräten
auszutauschen. Ein Event besitzt unter anderem einen Empfangszeitpunkt, eine Nachricht
und einen Schweregrad. Die o. g. Nodes werden sowohl für die Nutzdaten als auch
alle anderen Arten von Metadaten verwendet. Der damit modellierte OPC-Adressraum
beinhaltet nun auch ein Typmodell, mit dem sämtliche Datentypen spezifiziert werden.

% https://de.wikipedia.org/wiki/Open_Platform_Communications
% https://de.wikipedia.org/wiki/OPC_Unified_Architecture
% https://opcfoundation.org/developer-tools/specifications-unified-architecture
% Von Gerhard Gappmeier - ascolab GmbH, CC BY-SA 3.0, https://de.wikipedia.org/w/index.php?curid=1892069
\subsubsection{open62541%
        \label{sec:2-open62541}}
open62541 ist eine offene und freie Implementierung von OPC UA. Die in C geschriebene
Bibliothek stellt eine beständig zunehmende Anzahl der im OPC UA Standard definierten
Funktionen bereit. Sie kann sowohl zur Erstellung von OPC-Servern als auch -Clients
genutzt werden. Ergänzend zu der unter der Mozilla Public License v2.0 lizensierten
Bibliothek stellt das open62541 Projekt auch Beispielprogramme unter einer CC0 Lizenz
zur Verfügung.

Die Bibliothek eignet sich auch für die Entwicklung auf eingebetteten Systemen und
Microcontrollern. Je nach Umfang der gewünschten Funktionen und des OPC Informationsmodells
beträgt die Größe einer Server-Binary weniger als 100kb. %evtl. kürzen?

\todo{Nodes erklären! Evtl.~oben!}

Folgende Auswahl an Eigenschaften und Funktionen zeichnet die in dieser Arbeit verwendete
Version 0.3 von open62541 aus:
\begin{itemize}
  \item Kommunikationionsstack
  \begin{itemize}
      \item OPC UA Binär-Protokoll (HTTP oder SOAP werden gegenwärtig nicht unterstützt)
      \item Austauschbare Netzwerk-Schicht, welche die Verwendung eigener Netzwerk-APIs
      erlaubt.
      \item Verschlüsselte Kommunikationion
      \item Asynchrone Dienst-Anfragen im Client
  \end{itemize}
  \item Informationsmodell
  \begin{itemize}
    \item Unterstützung aller OPC UA Node-Typen, inkl.~Methoden
    \item Hinzufügen und Entfernen von Nodes und Referenzen zur Laufzeit.
    \item Vererbung und Instanziierung von Objekt- und Variablentypen
    \item Zugriffskontrolle auch für einzelne Nodes
  \end{itemize}
  \item Subscriptions
  \begin{itemize}
    \item Erlaubt die Überwachung (subscriptions / monitoreditems)
    \item Sehr geringer Ressourcenbedarf pro überwachtem Wert
  \end{itemize}
  \item Code-Generierung auf XML-Basis
  \begin{itemize}
    \item Erlaubt die Erstellung von Datentypen
    \item Erlaubt die Generierung des serverseitigen Informationsmodells
  \end{itemize}
\end{itemize}

% https://open62541.org/doc/0.3/


Mozilla Public License
CC0 Lizenz für Beispiele und Plugins

% https://open62541.org/doc/open62541-current.pdf
% https://open62541.org/

%% % Imports nur für Referenzenauflösung während des Schreibens! Vorm Kompilieren auskommentieren!
% \bibliography{0_hauptdatei}
% \input{1_einleitung}
% \input{2_grundlagen}
% \input{3_konzeption}
% \input{4_implementierung}
% \input{5_tests}
% \input{6_zusammenfassung}
% \input{anhang}
% % Ende Imports

\section{Systemkonzept%
  \label{sec:3-konzeption}}
Auf Basis der in Abschnitt \ref{sec:2-grundlagen} vorgestellten Möglichkeiten folgt nun die Ausarbeitung eines Konzepts.
In den folgenden Abschnitten soll näher auf zwei zentrale Aspekte eingegangen werden: Abschnitt~\ref{sec:3-anbindung} stellt Möglichkeiten zum Zugriff auf Variablen bzw.\,Werte im Prozessabbild des Revolution Pi vor; in Abschnitt~\ref{sec:3-integration} wird ein Konzept zur Bereitstellung dieser Variablen auf einem OPC-Server vorgestellt.

\subsection{Anbindung der IO an den OPC-Server%
     \label{sec:3-anbindung}}

Eine Webanwendung mit Bezeichnung PiCtory dient zur Konfiguration der I/O- und virtuellen Module des RevolutionPi. Die Konfiguration liegt im JSON-Format in der Datei \lstinline{/etc/revpi/config.rsc}. Der piControl-Treiber liest diese Datei beim Start. 
Der folgende Auszug aus der Manpage des piControl-Kernelmoduls beschreibt die von diesem zum Lesen und Schreiben einzelner Bits des Prozessabbildes bereitgestellten Funktionen~\citep[vgl.]{web-revpi-manpage}. Sie ist an dieser Stelle weitgehend ungekürzt zitiert, da sie die nutzbare Schnittstelle sehr kompakt beschreibt.

\begin{lstlisting}[breakindent=0pt, numbers=none, caption={Auszug aus der Revolution Pi Programmers Manual\label{lst:4-manpage}}]
KB_FIND_VARIABLE SPIVariable *argp
Find a variable in the process image by its name. A pointer to a structure of type SPIVariable must be passed as argument. [...]
The struct SPIVariable [...] is defined as 
typedef struct SPIVariableStr
{
    char strVarName[32]; // Variable name
    uint16_t i16uAddress; // Address of the byte in the process image
    uint8_t i8uBit; // 0-7 bit position, >= 8 whole byte
    uint16_t i16uLength; // length of the variable in bits.
    // Possible values are 1, 8, 16 and 32
} SPIVariable;

Set and get values of the process image
KB_GET_VALUE SPIValue *argp
[...]
KB_SET_VALUE SPIValue *argp
Write one bit or one byte to the process image [...].  This call is more efficient than the usual calls of seek and write because only one function call is necessary. If more than on application are writing bits in one output byte, this call is the only safe way to set a bit without overwriting the other bits because this call is doing a read-modify-write-cycle. 

The struct SPIValue used by this ioctl is defined as
typedef struct SPIValueStr
{
    uint16_t i16uAddress; // Address of the byte in the process image
    uint8_t i8uBit; // 0-7 bit position, >= 8 whole byte
    uint8_t i8uValue; // Value: 0/1 for bit access, whole byte otherwise
} SPIValue;
\end{lstlisting} 

Die oben beschriebenden Funtkionen \lstinline{KB_FIND_VARIABLE}, \lstinline{KB_GET_VALUE} und \lstinline{KB_SET_VALUE} ermöglichen einen einfachen und (lt.\,Manpage) effizienten Zugriff auf einzelne Bits des Prozessabbildes und damit auch auf die IO des RevolutionPi.
Der Zugriff des OPC-Servers auf das Prozessabbild soll daher mittels dieser Funktionen realisiert werden.
\lstinline{KB_FIND_VARIABLE} kann genutzt werden, um Adressen von Variablen im Prozessabbild mittels ihres Namens aufzulösen.
\lstinline{KB_GET_VALUE} und \lstinline{KB_SET_VALUE} ermöglichen den Zugriff auf die Werte dieser Variablen.


\subsection{Integration des OPC-Servers in das System%
     \label{sec:3-integration}}

open62541 bietet drei Möglichkeiten zum Abgleich von Variablen mit dem Prozessabbild~\citep[vgl.][Tutorials - Connecting a Variable with a Physical Process]{web-open62541}:
\begin{itemize}
    \item Manuelles oder zyklisches Aktualisieren
    \item Variable Value Callback
    \item Variable Datasource
\end{itemize}

Die zyklische Aktualisierung eines oder mehrerer Werte nimmt, abhängig von der Zykluszeit, viele Systemressourcen in Anspruch. Value Callbacks ermöglichen es, einen Variablenwert effizienter mit einer Ressource wie etwa einem Prozessabbild zu synchronisieren. An die Variable wird ein Callback angehängt, welches vor jedem Lesen und nach jedem Schreibvorgang ausgeführt wird.
Der Wert der Variablen wird weiterhin im Variablenknoten auf dem OPC-Server gespeichert, der Abgleich mit der verknüpften Ressource erfolgt durch die Callback-Methoden.

Sogenannte Datenquellen gehen noch einen Schritt weiter. Der Server leitet jede Lese- und Schreibanforderung direkt an eine Callback-Funktion weiter. Beim Lesen liefert der Rückruf eine Kopie des aktuellen Wertes. Die Datenquelle muss intern ein eigenes Speichermanagement implementieren.

Der Zugriff auf die Werte des Prozessabbildes erfolgt, wie in Abschnitt~\ref{sec:3-anbindung} beschrieben, über von piControl bereitgestellte Methoden. Um die durch open62541 gepflegte OPC-Datenstruktur und das durch piControl verwaltete Prozessabbild möglichst effektiv verknüpfen zu können, soll diese Interaktion mittels Datenquellen und den zugehörigen Callbacks implementiert werden.
%% % Imports nur für Referenzenauflösung während des Schreibens! Vorm Kompilieren auskommentieren!
% \bibliography{0_hauptdatei}
% \input{1_einleitung}
% \input{2_grundlagen}
% \input{3_konzeption}
% \input{4_implementierung}
% \input{5_tests}
% \input{6_zusammenfassung}
% \input{anhang}
% % Ende Imports

\section{Implementierung%
  \label{sec:4-implementierung}}
Das folgende Kapitel stellt in Auszügen die Implementierung des OPC-Servers sowie die Anbindung an die IO-Module
der SPS dar. Der Schwerpunkt liegt hierbei auf der Funktionsweise des piControl-Treibers und dessen Integration in das Projekt. Abschnitt~\ref{sec:4-picontrol} erklärt die zum Schreibens eines Bits verwendeten Funktionsaufrufe.
Zuvor soll jedoch in Abschnitt~\ref{sec:4-open62541} der Teil des OPC-Servers vorgestellt werden, welcher auf besagten Treiber zugreift. 

\subsection{Implementierung des OPC-Servers%
     \label{sec:4-open62541}}
Wie im vorangegangenen Abschnitt~\ref{sec:3-integration} begründet, soll die Verknüpfung zwischen dem Prozessabbild der SPS und den auf dem OPC-Server bereitgestellten Werten über sog.\,Datenquellen erfolgen. Hierzu ist zunächst eine Callback-Methode zu implementieren, welche bei einem Lese- oder Schreibzugriff auf eine Variable aufgerufen wird. Die Verknüpfung zwischen Callback-Methode und Variable muss manuell erfolgen.

\begin{lstlisting}[language={c},firstnumber=237,caption={Auszug der Methode \lstinline{linkDataSourceVariable} in \lstinline{variables.c}\label{lst:4-linkDataSourceVariable}}]
extern UA_StatusCode
 linkDataSourceVariable(UA_Server *server, UA_NodeId nodeId) {
     bool readonly = false;
     UA_DataSource dataSourceVariable;
     UA_StatusCode rc; |>\setcounter{lstnumber}{254}<|

     dataSourceVariable.read = readDataSourceVariable;
     if (!readonly)
        dataSourceVariable.write = writeDataSourceVariable;
     else
        dataSourceVariable.write = writeReadonlyDataSourceVariable;

     return UA_Server_setVariableNode_dataSource(server, nodeId, dataSourceVariable);
 }
\end{lstlisting}

\begin{figure}[h]
    \centering
    \includegraphics[width=0.42\textwidth]{doc/img/OPC_RevPiDO.pdf}
    \caption{Auszug des verwendeten Nodesets, hier Digitalausgang 1 des Versuchsaufbaus
      \label{fig:opc-do}}
\end{figure}

Die in Listing~\ref{lst:4-linkDataSourceVariable} abgebildete Methode \lstinline{linkDataSourceVariable()} erzeugt ein Struct vom Typ \lstinline{UA_DataSource}. In diesem werden dem Lesen und Schreiben einer OPC-Variablen entsprechende Callback-Methoden zugewiesen. Die Verknüpfung einer OPC-Variable, genauer ihrer NodeId, mit der zuvor definierten Datenquelle erfolgt über die von open62541 bereitgestellte Methode \lstinline{UA_Server_setVariableNode_dataSource()}. Vor dem Lesen und nach dem Schreiben dieser Variable werden von nun an die entsprechenden Callbacks aufgerufen.
     
\begin{lstlisting}[language={c},firstnumber=168,caption={Auszug des Callbacks \lstinline{writeDataSourceVariable} in \lstinline{variables.c}\label{lst:4-writeDataSourceVariable}}]  
extern UA_StatusCode
 writeDataSourceVariable(UA_Server *server,
            const UA_NodeId *sessionId, void *sessionContext,
            const UA_NodeId *nodeId, void *nodeContext,
            const UA_NumericRange *range, const UA_DataValue *dataValue) {

    UA_StatusCode retval  = UA_STATUSCODE_GOOD;
    UA_NodeId *nameNodeId = UA_malloc(sizeof(UA_NodeId));
    UA_QualifiedName nameQN = UA_QUALIFIEDNAME(1, "Name");
    UA_Variant nameVar;
    UA_Boolean bit;

    retval |= findSiblingByBrowsename(server, nodeId, &nameQN, nameNodeId);
    retval |= UA_Server_readValue(server, *nameNodeId, &nameVar);
    retval |= UA_Boolean_copy(dataValue->value.data, &bit);

    |>\tikzmarkin[set border color=martinired]{writeIO}<|PI_writeSingleIO(String_fromUA_String(nameVar.data), &bit, false);                                                 |>\tikzmarkend{writeIO}<|

    free(nameNodeId);
    return retval;
 }
\end{lstlisting}

Listing~\ref{lst:4-writeDataSourceVariable} zeigt die Callback-Methode, welche nach dem Schreiben einer Variablen auf dem OPC-Server aufgerufen wird.
Dieser Methode wird neben der NodeId der mit ihr verknüpften Variablen auch der Wert dieser in Form eines Zeigers auf ein Struct vom Typ \lstinline{UA_DataValue} übergeben.

Die Gestaltung des hier verwendeten Nodesets sieht vor, dass in einer OPC-Variablen \lstinline{"Name"} der Bezeichner des zu schreibenden Digitalausgangs hinterlegt ist, siehe Abbildung~\ref{fig:opc-do}. Dies erlaubt eine Rekonfiguration der Ein- und Ausgänge der SPS ohne Änderungen im Programmcode des OPC-Servers vornehmen zu müssen.
Es ist daher erforderlich, nach jedem Schreiben einer mit einem Digitalausgang verknüpften Variablen, hier \lstinline{"Value"}, dessen Bezeichner \lstinline{"Name"} abzufragen. 
Dies geschieht in den Zeilen 180 und 181.
Anschließend wird dieser Bezeichner sowie der zu schreibende Wert der Methode \lstinline{PI_writeSingleIO()} übergeben, welche wiederum die Interaktion mit piControl übernimmt (vgl. Abschnitt \ref{sec:4-picontrol}).
 
\subsection{Integration von piControl%
     \label{sec:4-picontrol}}
In Abschnitt~\ref{sec:2-io} wurde die Anbindung der IO-Module des Revolution Pi sowie die Funktionsweise von piControl aus Anwendersicht beschrieben. Die verfügbare Literatur beschränkt sich auch auf lediglich diese Sicht; eine weiterführende Dokumentation für Entwickler gibt es, neben der in Abschnitt~\ref{sec:3-anbindung} vorgestellten Manpage, nicht. 
In diesem Abschnitt soll daher der Quellcode von piControl sowie dessen Verwendung im Projekt genauer betrachtet werden.
Hierzu wird exemplarisch die in Abschnitt~\ref{sec:4-open62541} eingeführte Methode \lstinline{PI_writeSingleIO()} untersucht.
Diese Methode ermöglicht das Setzen eines einzelnen Bits im Prozessabbild der SPS, und damit das Schalten eines digitalen Ausgangs auf einem IO-Modul.
Die äquivalente Methode \lstinline{int piControlGetBitValue(SPIValue *pSpiValue)} zum Lesen eines Bits bzw. Eingangs funktioniert analog und soll daher an dieser Stelle nicht dediziert erörtert werden.

\begin{lstlisting}[language={c},firstnumber=97,
                   caption={Setzen eines phsikalischen, digitalen Ausgangs in \lstinline{revpi.c}
                   \label{lst:4-PI_writeSingleIO}}]
extern void PI_writeSingleIO(char *pszVariableName, bool *bit, bool verbose)
{
	int rc;
	SPIVariable sPiVariable;
	SPIValue sPIValue;

	strncpy(sPiVariable.strVarName, pszVariableName, sizeof(sPiVariable.strVarName));
	rc = piControlGetVariableInfo(&sPiVariable);
	if (rc < 0) {
		printf("Cannot find variable '%s'\n", pszVariableName);
		return;
	}

		sPIValue.i16uAddress = sPiVariable.i16uAddress;
		sPIValue.i8uBit = sPiVariable.i8uBit;
		sPIValue.i8uValue = *bit;
		rc = |>\tikzmarkin[set border color=martinired]{setBitValue}<|piControlSetBitValue(&sPIValue)|>\tikzmarkend{setBitValue}<|;
		if (rc < 0)
			printf("Set bit error %s\n", getWriteError(rc));
		else if (verbose)
			printf("Set bit %d on byte at offset %d. Value %d\n", sPIValue.i8uBit, sPIValue.i16uAddress,
			       sPIValue.i8uValue);
}
\end{lstlisting}

Der Programmcode in Listing~\ref{lst:4-PI_writeSingleIO} ist Teil des implementierten OPC-Servers. In diesem wird auf zwei Funktionen des piControl-Treibers zugegriffen. 
Beiden Methoden wird als Argument ein Zeiger auf ein Struct vom Typ \lstinline{SPIValue} übergeben. Der im Struct abgelegte Name wird mittels \lstinline{piControlGetVariableInfo(&sPIValue)} zu einer Adresse im Prozessabbild aufgelöst. Diese wird in \lstinline{sPIValue.i16uAdress} gespeichert. Der Wert der Variablen wird anschließend mittels \lstinline{piControlSetBitValue(&sPIValue)} an dieser Adresse in das Prozessabbild geschrieben.

\begin{lstlisting}[language={c},firstnumber=309,caption={Methode \lstinline{piControlSetBitValue} in \lstinline{piControlIf.c}\label{lst:4-piControlSetBitValue}}]
int |>\tikzmarkin[set border color=martiniblue]{setBitValueFcn}<|piControlSetBitValue(SPIValue *pSpiValue)|>\tikzmarkend{setBitValueFcn}<|
{
    piControlOpen();

    if (PiControlHandle_g < 0)
	    return -ENODEV;

    pSpiValue->i16uAddress += pSpiValue->i8uBit / 8;
    pSpiValue->i8uBit %= 8;

    if (|>\tikzmarkin[set border color=martinired]{ioctl}<|ioctl(PiControlHandle_g, KB_SET_VALUE, pSpiValue)|>\tikzmarkend{ioctl}<| < 0)
	    return errno;

    return 0;
}
\end{lstlisting}

Die in Listing~\ref{lst:4-piControlSetBitValue} dargestellte Methode \lstinline{piControlSetBitValue} ist lediglich eine Hüllfunktion (häufig auch als Wrapper-Funktion bezeichnet) für einen Aufruf des \lstinline{ioctl} Kernel-Moduls.
Folgende Parameter werden übergeben:
\lstinline{PiControlHandle_g} ist die Referenz auf die Geräte-Datei des piControl-Treibers. \lstinline{KB_SET_VALUE} ist das ioctl-Kommando zum Schreiben eines Bits in das Prozessabbild. Der Zeiger \lstinline{pSpiValue} verweist auf ein Struct des bereits vorgestellten Typs \lstinline{SPIValue}.

\begin{lstlisting}[language={c},firstnumber=80,caption={Methode \lstinline{piControlOpen} in \lstinline{piControlIf.c}\label{lst:4-piControlOpen}}]
void piControlOpen(void)
{
    /* open handle if needed */
    if (PiControlHandle_g < 0)
    {
	    |>\tikzmarkin[set border color=martiniblue]{PiControlHandle}<|PiControlHandle_g = open(PICONTROL_DEVICE, O_RDWR)|>\tikzmarkend{PiControlHandle}<|;
    }
}
\end{lstlisting}

Die in Listing~\ref{lst:4-piControlOpen} dargestellte Methode öffnet, sofern nicht bereits geschehen, die Geräte-Datei. Das Macro \lstinline{PICONTROL_DEVICE} verweist hierbei auf \lstinline{/dev/piControl0}.

\begin{lstlisting}[language={c},firstnumber=721,caption={Methode \lstinline{piControlIoctl} in \lstinline{piControlMain.c}\label{lst:4-piControlIoctl}}]
static long |>\tikzmarkin[set border color=martiniblue, below offset=0.9em]{piControlIoctl}<|piControlIoctl(struct file *file, unsigned int prg_nr, 
                           unsigned long usr_addr)                                      |>\tikzmarkend{piControlIoctl}<|
{
  int status = -EFAULT;
  tpiControlInst *priv;
  int timeout = 10000;	// ms

  if (prg_nr != KB_CONFIG_SEND && prg_nr != KB_CONFIG_START && !isRunning()) {
  	return -EAGAIN;
  }

  priv = (tpiControlInst *) file->private_data;

  if (prg_nr != KB_GET_LAST_MESSAGE) {
  	// clear old message
  	priv->pcErrorMessage[0] = 0;
  }

  switch (prg_nr) {|>\setcounter{lstnumber}{864}<|

    case |>\tikzmarkin[set border color=martiniblue]{KB_SET_VALUE}<|KB_SET_VALUE:|>\tikzmarkend{KB_SET_VALUE}<|
  		{
  			SPIValue *pValue = (SPIValue *) usr_addr;

  			if (!isRunning())
  				return -EFAULT;

  			if (pValue->i16uAddress >= KB_PI_LEN) {
  				status = -EFAULT;
  			} else {
  				INT8U i8uValue_l;
  				my_rt_mutex_lock(&piDev_g.lockPI);
  				i8uValue_l = piDev_g.ai8uPI[pValue->i16uAddress];

  				if (pValue->i8uBit >= 8) {
  					i8uValue_l = pValue->i8uValue;
  				} else {
  					if (pValue->i8uValue)
  						i8uValue_l |= (1 << pValue->i8uBit);
  					else
  						i8uValue_l &= ~(1 << pValue->i8uBit);
  				}

  				|>\tikzmarkin[set border color=martinired]{i8uValue}<|piDev_g.ai8uPI[pValue->i16uAddress] = i8uValue_l;|>\tikzmarkend{i8uValue}<|
  				rt_mutex_unlock(&piDev_g.lockPI);

  #ifdef VERBOSE
  				pr_info("piControlIoctl Addr=%u, bit=%u: %02x %02x\n", pValue->i16uAddress, pValue->i8uBit, pValue->i8uValue, i8uValue_l);
  #endif

  				status = 0;
  			}
  		}
  		break; |>\setcounter{lstnumber}{1314}<|

    default:
      pr_err("Invalid Ioctl");
      return (-EINVAL);
      break;

    }

    return status;
  }
\end{lstlisting}

Listing~\ref{lst:4-piControlIoctl} zeigt in Auszügen die ioctl-Methode des piControl Kernel-Treibers. Diese bekommt folgende Argumente übergeben: \lstinline{struct file *file} enthält den Verweis auf die Geräte-Datei, hier \lstinline{/dev/piControl0}. Der Wert von \lstinline{unsigned int prg_nr} beschreibt die Anfrage an den Treiber, in diesem Fall \lstinline{KB_SET_VALUE}. Das Argument \lstinline{unsigned long usr_addr} enthält einen typ-agnostischen Pointer. Dieser verweist auf einen Speicherbereich, in welchem die zur Bearbeitung der Anfrage notwendigen Daten abgelegt sind. Hier können auch vom Treiber empfangene Daten dem Anwendungsprogramm bereitgestellt werden. 

Die switch-case-Anweisung führt die über das Argument \lstinline{prg_nr} spezifizierte Aktion aus. Hier betrachten wir \lstinline{KB_SET_VALUE}:
Zunächst wird in Zeile 868 der übergebene Zeiger \lstinline{usr_addr} mittels explizitem Typecast zu einem Zeiger des Typs \lstinline{SPIValue *} konvertiert. Da dieser auf Daten im Userspace verweist, ist beim Zugriff durch den Kernel-Treiber besondere Vorsicht geboten.
In Zeile 877 wird mittels Mutex das Prozessabbild \lstinline{piDev_g} für den Zugriff durch andere Threads oder Prozesse gesperrt.
\lstinline{my_rt_mutex_lock} verweist hierbei auf die Funktion \lstinline{rt_mutex_lock} aus \lstinline{linux/sched.h}\footnote{Offenbar wurde hier auch eine alternative Implementierung vorgesehen, siehe revpi\_common.h}

In Zeile 889 wird das Byte \lstinline{i8uValue_l}, welches den zu schreibenden Wert enthält in das Prozessabbild übertragen. Anschließend wird die Mutex auf \lstinline{piDev_g} wieder entsperrt.
\newpage

\begin{lstlisting}[language={c},firstnumber=62,caption={Auszug des Struct \lstinline{spiControlDev} in \lstinline{piControlMain.h}\label{lst:4-spiControlDev}}]
|>\tikzmarkin[set border color=martiniblue]{spiControlDev}<|typedef struct spiControlDev|>\tikzmarkend{spiControlDev}<| {
	// device driver stuff
	int init_step;
	enum revpi_machine machine_type;
	void *machine;
	struct cdev cdev;	// Char device structure
	struct device *dev;
	struct thermal_zone_device *thermal_zone;

	|>\tikzmarkin[set border color=martiniblue]{processImage}<|// process image stuff
	INT8U ai8uPI[KB_PI_LEN];
	INT8U ai8uPIDefault|>\tikzmarkin[set border color=martinired]{KB_PI_LEN_0}<|[KB_PI_LEN]|>\tikzmarkend{KB_PI_LEN_0}<|;
	struct rt_mutex lockPI;        |>\tikzmarkend{processImage}<|
	bool stopIO;
	piDevices *devs; |>\setcounter{lstnumber}{94}<|
} tpiControlDev;
\end{lstlisting}

Das Prozessabbild ist als Byte-Array der Länge \lstinline{KB_PI_LEN} in Listing~\ref{lst:4-spiControlDev} definiert. Konfigurationsparameter wie \lstinline{KB_PI_LEN} oder die Zykluszeit für den Datenaustausch zwischen SPS und IO-Modulen sind im folgenden Listing~\ref{lst:4-process} definiert.

\begin{lstlisting}[language={c},firstnumber=119,caption={Konfigurationsparameter des Prozessabbildes in project.h\label{lst:4-process}}]
#define INTERVAL_PI_GATE (5*1000*1000)  // 5 ms piGateCommunication |>\setcounter{lstnumber}{128}<|

#define INTERVAL_IO_COM (5*1000*1000)  // 5 ms piIoComm |>\setcounter{lstnumber}{132}<|

#define KB_PD_LEN       512
|>\tikzmarkin[set border color=martiniblue]{KB_PI_LEN_1}<|#define KB_PI_LEN       4096|>\tikzmarkend{KB_PI_LEN_1}<|
\end{lstlisting}

Das zu setzende Bit wurde zu diesem Zeitpunkt erfolgreich in das Prozessabbild der SPS geschrieben.
Es stellt sich die Frage, wie dieses nun an das IO-Modul kommuniziert wird.
Die Kommunikation mit allen angebundenen Modulen ist ebenfalls Aufgabe des piControl-Treibers.

\begin{lstlisting}[language={c},firstnumber=256,caption={Auszug der Methode \lstinline{piIoThread} in \lstinline{revpi_core.c}\label{lst:4-piIoThread}}]
static int piIoThread(void *data)
{
	//TODO int value = 0;
	ktime_t time;
	ktime_t now;
	s64 tDiff;

	hrtimer_init(&piCore_g.ioTimer, CLOCK_MONOTONIC, HRTIMER_MODE_ABS);
	piCore_g.ioTimer.function = piIoTimer;

	pr_info("piIO thread started\n");

	now = hrtimer_cb_get_time(&piCore_g.ioTimer);

	PiBridgeMaster_Reset();

	while (!kthread_should_stop()) {
		if (|>\tikzmarkin[set border color=martinired]{PiBridgeMaster}<|PiBridgeMaster_Run()|>\tikzmarkend{PiBridgeMaster}<| < 0)
			break;
	}

	RevPiDevice_finish();

	pr_info("piIO exit\n");
	return 0;
}
\end{lstlisting}

Der Kernel-Thread \lstinline{piIoThread} ist verantwortlich für den zyklischen Datenaustausch mit den IO-Modulen. In diesem wird fortlaufend die Methode \lstinline{PiBridgeMaster_Run()} aufgerufen, siehe Listing~\ref{lst:4-piIoThread}.

\begin{lstlisting}[language={c},firstnumber=262,caption={Auszug der Methode \lstinline{PiBridgeMaster_Run(void)} in \lstinline{RevPiDevice.c}\label{lst:4-PiBridgeMaster_Run}}]
int PiBridgeMaster_Run(void)
{
	static kbUT_Timer tTimeoutTimer_s;
	static kbUT_Timer tConfigTimeoutTimer_s;
	static int error_cnt;
	static INT8U last_led;
	static unsigned long last_update;
	int ret = 0;
	int i;

	my_rt_mutex_lock(&piCore_g.lockBridgeState);
	if (piCore_g.eBridgeState != piBridgeStop) {
		switch (eRunStatus_s) { |>\setcounter{lstnumber}{514}<|
		    case enPiBridgeMasterStatus_EndOfConfig:|>\setcounter{lstnumber}{621}<|
		    if (|>\tikzmarkin[set border color=martinired]{RevPiDevice}<|RevPiDevice_run()|>\tikzmarkend{RevPiDevice}<|) {
				// an error occured, check error limits |>\setcounter{lstnumber}{641}<|
			} else {
				ret = 1;
			}
			piCore_g.image.drv.i16uRS485ErrorCnt = RevPiDevice_getErrCnt();
			break;
\end{lstlisting}

Die in Listing~\ref{lst:4-PiBridgeMaster_Run} dargestellte Methode ist eine sog. State-Machine. Ist die Konfiguration der IO-Module erfolgreich abgeschlossen, so führt sie bei Aufruf lediglich die Methode \lstinline{RevPiDevice_run()} aus.

\begin{lstlisting}[language={c},firstnumber=140,caption={Auszug der Methode \lstinline{RevPiDevice_run(void)} in \lstinline{RevPiDevice.c}\label{lst:4-RevPiDevice_run}}]
int RevPiDevice_run(void)
{
	INT8U i8uDevice = 0;
	INT32U r;
	int retval = 0;

	RevPiDevices_s.i16uErrorCnt = 0;

	for (i8uDevice = 0; i8uDevice < RevPiDevice_getDevCnt(); i8uDevice++) {
		if (RevPiDevice_getDev(i8uDevice)->i8uActive) {
			switch (RevPiDevice_getDev(i8uDevice)->sId.i16uModulType) {
			case KUNBUS_FW_DESCR_TYP_PI_DIO_14:
			case KUNBUS_FW_DESCR_TYP_PI_DI_16:
			case KUNBUS_FW_DESCR_TYP_PI_DO_16:
				r = |>\tikzmarkin[set border color=martinired]{sendCyclicTelegram}<|piDIOComm_sendCyclicTelegram(i8uDevice)|>\tikzmarkend{sendCyclicTelegram}\setcounter{lstnumber}{166} <|;

				break; |>\setcounter{lstnumber}{216}<|
			}
		}
	} |>\setcounter{lstnumber}{227}<|
	return retval;
}
\end{lstlisting}

Diese iteriert wie in Listing~\ref{lst:4-RevPiDevice_run} abgebildete durch alle gegenwärtig in der SPS konfigurierten Module. Ist das aktuelle Modul als aktiv markiert, so wird anhand eines sog. Firmware-Descriptors entschieden, welche Methode für die Ansteuerung des Moduls aufzurufen ist.

\begin{lstlisting}[language={c},firstnumber=161,caption={Auszug der Methode \lstinline{piDIOComm_sendCyclicTelegram} in \lstinline{piDIOComm.c}\label{lst:4-sendCyclicTelegram}}]
INT32U piDIOComm_sendCyclicTelegram(INT8U i8uDevice_p)
{
	INT32U i32uRv_l = 0;
	SIOGeneric sRequest_l;
	SIOGeneric sResponse_l;
	INT8U len_l, data_out[18], i, p, data_in[70];
	INT8U i8uAddress;
	int ret; |>\setcounter{lstnumber}{239}<|
	
    |>\tikzmarkin[set border color=martinired]{piIoComm}<|ret = piIoComm_send((INT8U *) & sRequest_l, IOPROTOCOL_HEADER_LENGTH + len_l + 1);  |>\tikzmarkend{piIoComm}\setcounter{lstnumber}{298}<|
}
\end{lstlisting}

Im Falle des hier verwendeten DO-Moduls wird die in Listing~\ref{lst:4-sendCyclicTelegram} abgebildete Methode \lstinline{piDIOComm_sendCyclicTelegram()} aufgerufen. Dieser wird ein Zeiger auf das zu schreibende Gerät übergeben. 
Zunächst wird das Prozessabbild mittels eines proprietären, jedoch im Quellcode offen nachvollziehbaren Protokolls in ein \lstinline{sRequest_l} genanntes Byte-Array umgewandelt. Dieser Schritt ist in Listing~\ref{lst:4-sendCyclicTelegram} nicht abgebildet. Anschließend wird \lstinline{piIoComm_send()} ein Zeiger auf die so generierte Schreib-Anfrage übergeben.

\begin{lstlisting}[language={c},firstnumber=220,caption={Auszug der Methode \lstinline{piIOComm_send} in \lstinline{piIOComm.c}\label{lst:4-piIOComm_send}}]
int piIoComm_send(INT8U * buf_p, INT16U i16uLen_p)
{
	ssize_t write_l = 0;
	INT16U i16uSent_l = 0;|>\setcounter{lstnumber}{249}<|

	while (i16uSent_l < i16uLen_p) {
		write_l = vfs_write(piIoComm_fd_m, buf_p + i16uSent_l, i16uLen_p - i16uSent_l, &piIoComm_fd_m->f_pos);
		if (write_l < 0) {
			pr_info_serial("write error %d\n", (int)write_l);
			return -1;
		} 
		i16uSent_l += write_l;|>\setcounter{lstnumber}{263}<|
	}
	clear();
	vfs_fsync(piIoComm_fd_m, 1);
	return 0;
}
\end{lstlisting}

Listing~\ref{lst:4-piIOComm_send} zeigt die Implementierung von \lstinline{piIoComm_send()}. Diese Methode ist für das Schreiben der oben generierten Anfrage auf die seriellen Schnittstelle verantwortlich. Realisiert wird dies mittels der Methode \lstinline{vfs_write()}. Diese ist in \lstinline{<linux/fs.h>} definiert. Sie ermöglicht das Schreiben einer Datei im Userspace aus dem Kernel heraus. Geschrieben wird hier die Datei mit dem Deskriptor \lstinline{piIoComm_fd_m}.
Da die Funktion \lstinline{vfs_write()} durch andere Kernel-Tasks unterbrochen werden kann, ist nicht gewährleistet, dass die gesamte Anfrage mit nur einem Aufruf geschrieben wird. Die oben abgebildete while-Schleife stellt das vollständige Senden der Anfrage sicher.

\begin{lstlisting}[language={c},firstnumber=157,caption={Auszug der Methode \lstinline{piIOComm_open_serial} in \lstinline{piIOComm.c}\label{lst:4-piIOComm_open_serial}}]
int piIoComm_open_serial(void)
{   |>\setcounter{lstnumber}{167}<|
	struct file *fd;	/* Filedeskriptor */
	struct termios newtio;	/* Schnittstellenoptionen */

	|>\tikzmarkin[set border color=martiniblue]{fd}<|/* Port oeffnen - read/write, kein "controlling tty", 
	    Status von DCD ignorieren */
	fd = filp_open(|>\tikzmarkin[set border color=martinired]{tty}<|REV_PI_TTY_DEVICE|>\tikzmarkend{tty}<|, O_RDWR | O_NOCTTY, 0); |>\setcounter{lstnumber}{208}<|
	
	piIoComm_fd_m = fd;                                                      |>\tikzmarkend{fd}\setcounter{lstnumber}{217}<|

	return 0;
}
\end{lstlisting}

Der zum Schreiben auf die serielle Schnittstelle verwendete Datei-Deskriptor wird von der in Listing~\ref{lst:4-piIOComm_open_serial} abgebildeten Methode \lstinline{piIoComm_open_serial()} generiert. 

\begin{lstlisting}[language={c},firstnumber=45,caption={Definition der seriellen Schnittstelle in \lstinline{piIOComm.h}\label{lst:4-REV_PI_TTY_DEVICE}}]
#define REV_PI_TTY_DEVICE	"/dev/ttyAMA0"
\end{lstlisting}

Das in Listing~\ref{lst:4-REV_PI_TTY_DEVICE} definierte Macro verweist auf eine der seriellen Schnittstellen des RaspberryPi.
Die Implementierung des zugehörigen Schnittstellentreibers soll hier nicht weiter untersucht werden. Somit ist an dieser Stelle die Kette vom Setzen einer Variablen auf dem OPC-Server bis hin zur Aktualisierung des Prozessabbilds der IO-Module geschlossen.

% \begin{lstlisting}[language={c},firstnumber={226},caption={Setzen der Scheduler-Priorität auf SCHED\_FIFO in 
% revpi\_common.c\label{lst:2-sched_priority}}]
% param.sched_priority = ktprio->prio;
% ret = sched_setscheduler(child, SCHED_FIFO, &param);
% \end{lstlisting}
%% % Imports nur für Referenzenauflösung während des Schreibens! Vorm Kompilieren auskommentieren!
% \bibliography{0_hauptdatei}
% \input{1_einleitung}
% \input{2_grundlagen}
% \input{3_konzeption}
% \input{4_implementierung}
% \input{5_tests}
% \input{6_zusammenfassung}
% % Ende Imports

\section{Test des OPC-Servers im Gesamtsystem%
  \label{sec:5-tests}}

%% % Imports nur für Referenzenauflösung während des schreibens! Vorm Kompilieren auskommentieren!
% \bibliography{0_hauptdatei}
% \input{1_einleitung}
% \input{2_grundlagen}
% \input{3_konzeption}
% \input{4_implementierung}
% \input{5_tests}
% \input{6_zusammenfassung}
% % Ende Imports

\section{Zusammenfassung und Ausblick%
  \label{sec:6-fazit}}
Der folgende Abschnitt~\ref{sec:6-zusammenfassung} fasst die gewonnenen Erkenntnisse und den Stand der Implementierung zusammen.
Den Abschluss dieser Arbeit bildet der Ausblick in Abschnitt~\ref{sec:6-ausblick}.

\subsection{Zusammenfassung%
     \label{sec:6-zusammenfassung}}

\subsection{Ausblick%
     \label{sec:6-ausblick}}

% % Ende Imports

\section{Einleitung und Motivation%
  \label{sec:1-einleitung}}
Ziel dieses Projektes ist die Integration eines OPC-Servers mit einer auf Linux
basierenden speicherprogrammierbaren Steuerung (SPS). Angeschlossen an diese SPS
ist jeweils ein digitales Ein-/\,bzw.~Ausgabemodul. Die von diesen bereitgestellten
Ein-/\, bzw.~Ausgänge (IO) sollen in der Datenstruktur des OPC-Servers abgebildet
und über diesen für OPC-Clients les-/\,und schreibar sein. Weiterhin sollen einige
Funktionen zur Überwachung und Steuerung der an die SPS angeschlossenen Aktoren
und Sensoren direkt im OPC-Server implementiert werden.
Hiermit stellt dieses Projekt eine der Grundlagen für ein übergeordnetes Projekt,
die cloudbasierte Steuerung eines miniaturisierten Produktions-Systems, dar.

Der hier verwendete OPC-Server ist Teil des sog. open62541 Projekts. Er ist in C
geschrieben und implementiert bereits einen großen Teil der im OPC-UA-Standard
spezifizierten Funktionen.
Als SPS findet ein Revolution Pi 3 der Firma Kunbus Verwendung. Dieser integriert
ein sog. Compute Module der Raspberry Pi Foundation in ein industrietaugliches
Gehäuse und erlaubt die Erweiterung mittels IO- oder Gateway-Modulen. Über diese
erfolgt die Kommunikation mit weiteren Komponenten der Automatisierungstechnik.

Motiviert ist dieses Projekt durch die Beobachtung, dass die Verbreitung offener
Standards sowie freier Software auch in der Automatisierungstechnik zunimmt.
Linux ist ein freies Betriebssystem, OPC-UA ein offen zugänglicher, aktiv gepflegter
und weit verbreiteter Standard. Der Raspberry Pi findet sowohl bei Hobby-Anwendern als
auch in den Bereichen Forschung und Entwicklung sowie bei industriellen Anwendern
Verwendung. Dieses Projekt stellt somit eine für unterschiedliche Anwender interessante
Entwicklung dar.

Im Anschluss an diese einleitende Übersicht im Abschnitt~\ref{sec:1-einleitung} folgt
die Darstellung der wichtigsten Grundlagen in Abschnitt~\ref{sec:2-grundlagen}.
Aufbauend auf diesen Grundlagen folgt die konzeptuelle Ausarbeitung im Abschnitt~\ref{sec:3-konzeption}.
Die Umsetzung wird im Abschnitt~\ref{sec:4-implementierung} erläutert.
Die Leistungsfähigkeit der Implementierung wird in Abschnitt~\ref{sec:5-tests} untersucht.
Eine Zusammenfassung und ein Ausblick schließen die Arbeit in
Abschnitt~\ref{sec:6-fazit} ab. Eventuell noch benötigte Anhänge
finden sich in den Anhängen [...] bis [...].

% % % Imports nur für Referenzenauflösung während des Schreibens! Vorm Kompilieren auskommentieren!
% \bibliography{0_hauptdatei}
% % Mit \section{...} eröffnen wir einen neuen Abschnitt.
% Der Befehl setzt nicht nur den Text in einer größeren,
% fetten Schrift, sondern sorgt außerdem dafür, daß er im
% Inhaltsverzeichnis erscheint.
%
% Mit \label{...} erzeugen wir einen Bezeichner, mit dessen Hilfe
% wir später auf die Nummer des Abschnitts verweisen können (nämlich
% mit~\ref{...}).
%
% Das Kommentarzeichen hinter „Übersicht“ dient dazu, ein
% Leerzeichen zwischen „Übersicht“ und dem \label-Befehl
% zu vermeiden, das andernfalls sichtbar würde – z.B. im
% Inhaltsverzeichnis.
%

% % Imports nur für Referenzenauflösung während des Schreibens! Vorm Kompilieren auskommentieren!
% \bibliography{0_hauptdatei}
% \input{1_einleitung}
%\input{2_grundlagen}
%\input{3_konzeption}
%\input{4_implementierung}
%\input{5_tests}
%\input{6_zusammenfassung}
% % Ende Imports

\section{Einleitung und Motivation%
  \label{sec:1-einleitung}}
Ziel dieses Projektes ist die Integration eines OPC-Servers mit einer auf Linux
basierenden speicherprogrammierbaren Steuerung (SPS). Angeschlossen an diese SPS
ist jeweils ein digitales Ein-/\,bzw.~Ausgabemodul. Die von diesen bereitgestellten
Ein-/\, bzw.~Ausgänge (IO) sollen in der Datenstruktur des OPC-Servers abgebildet
und über diesen für OPC-Clients les-/\,und schreibar sein. Weiterhin sollen einige
Funktionen zur Überwachung und Steuerung der an die SPS angeschlossenen Aktoren
und Sensoren direkt im OPC-Server implementiert werden.
Hiermit stellt dieses Projekt eine der Grundlagen für ein übergeordnetes Projekt,
die cloudbasierte Steuerung eines miniaturisierten Produktions-Systems, dar.

Der hier verwendete OPC-Server ist Teil des sog. open62541 Projekts. Er ist in C
geschrieben und implementiert bereits einen großen Teil der im OPC-UA-Standard
spezifizierten Funktionen.
Als SPS findet ein Revolution Pi 3 der Firma Kunbus Verwendung. Dieser integriert
ein sog. Compute Module der Raspberry Pi Foundation in ein industrietaugliches
Gehäuse und erlaubt die Erweiterung mittels IO- oder Gateway-Modulen. Über diese
erfolgt die Kommunikation mit weiteren Komponenten der Automatisierungstechnik.

Motiviert ist dieses Projekt durch die Beobachtung, dass die Verbreitung offener
Standards sowie freier Software auch in der Automatisierungstechnik zunimmt.
Linux ist ein freies Betriebssystem, OPC-UA ein offen zugänglicher, aktiv gepflegter
und weit verbreiteter Standard. Der Raspberry Pi findet sowohl bei Hobby-Anwendern als
auch in den Bereichen Forschung und Entwicklung sowie bei industriellen Anwendern
Verwendung. Dieses Projekt stellt somit eine für unterschiedliche Anwender interessante
Entwicklung dar.

Im Anschluss an diese einleitende Übersicht im Abschnitt~\ref{sec:1-einleitung} folgt
die Darstellung der wichtigsten Grundlagen in Abschnitt~\ref{sec:2-grundlagen}.
Aufbauend auf diesen Grundlagen folgt die konzeptuelle Ausarbeitung im Abschnitt~\ref{sec:3-konzeption}.
Die Umsetzung wird im Abschnitt~\ref{sec:4-implementierung} erläutert.
Die Leistungsfähigkeit der Implementierung wird in Abschnitt~\ref{sec:5-tests} untersucht.
Eine Zusammenfassung und ein Ausblick schließen die Arbeit in
Abschnitt~\ref{sec:6-fazit} ab. Eventuell noch benötigte Anhänge
finden sich in den Anhängen [...] bis [...].

% % % Imports nur für Referenzenauflösung während des Schreibens! Vorm Kompilieren auskommentieren!
% \bibliography{0_hauptdatei}
% \input{1_einleitung}
% \input{2_grundlagen}
% \input{3_konzeption}
% \input{4_implementierung}
% \input{5_tests}
% \input{6_zusammenfassung}
% % Ende Imports

\section{Grundlagen%
  \label{sec:2-grundlagen}}

\subsection{Speicherprogrammierbare-Steuerung und Linux -- Revolution Pi%
     \label{sec:2-sps}}

\subsubsection{Kunbus RevolutionPi%
        \label{sec:2-revpi}}
Der RevolutionPi 3 ist eine speicherprogrammierbare Steuerung (SPS) des Herstellers
Kunbus GmbH. Kern dieser SPS ist das von der Raspberry Pi Foundation entwickelte
und vertriebene Raspberry Pi Compute Module 3. Dieses integriert ein Broadcom BCM2837
System-on-Chip (SoC) mit vier 1,2GHz Prozessorkernen, 1GB RAM, 4GB eMMC Anwendungsspeicher
und sonstige Peripherie in ein Modul im DDR2-SODIMM Formfaktor. Diese Spezifikationen
sind weitgehend identisch zu denen des ausgesprochen populären Raspberry Pi 3.
Der Revolution Pi profitiert daher von dem gleichen großen Angebot an Software
und Unterstützung wie der Raspberry Pi, ergänzt dessen Hardware jedoch um eine 24V
Spannungsversorgung, die Möglichkeit der Erweiterung durch mehrere industrietaugliche
Ein-/ Ausgabemodule und Gateways sowie ein Gehäuse zur Montage auf einer DIN-Schiene.
\begin{itemize}
  \item{Prozessor: BCM2837}
  \item{Taktfrequenz 1,2 GHz}
  \item{Anzahl Prozessorkerne: 4}
  \item{Arbeitsspeicher: 1 GByte}
  \item{eMMC Flash Speicher: 4 GByte}
  \item{Betriebssystem: Angepasstes Raspbian mit RT-Patch}
  \item{RTC mit 24h Pufferung über wartungsfreien Kondensator}
  \item{Treiber / API: Treiber schreibt zyklisch Prozessdaten in ein Prozessabbild, Zugriff auf Prozessabbild über Linux-Filesystem als API zu Fremdsoftware.}
  \item{Kommunikationsanschlüsse: 2 x USB 2.0 A (je 500 mA belastbar), 1 x Micro-USB, HDMI, Ethernet (RJ45) 10/100 Mbit/s}
  \item{Stromversorgung: min. 10,7 V, max. 28,8 V, maximal 10 Watt}
  \item{Zulässige Umgebungstemperatur: -40 bis +55 C}
  \item{Gehäuseabmessungen: (HxBxL) 96 mm x 22,5 mm x 110,5 mm (ohne gesteckte Stecker)}
  \item{ESD Schutz: 4 kV / 8 kV gemäß EN61131-2 und IEC 61000-6-2}
  \item{Surge / Burst Prüfungen: gemäß EN61131-2 und IEC 61000-6-2 eingekoppelt auf Versorgungsspannung, Ethernet und IO-Leitungen}
  \item{EMI Prüfungen: gemäß EN61131-2 und IEC 61000-6-2}
\end{itemize}

Kunbus bietet eine Auswahl an IO- und Gateway-Modulen zur Erweiterung des Revolution Pi an.
Gateways dienen der Kommunikation mit Systemen oder Komponenten der Automatisierungstechnik
über Protokolle wie PROFIBUS oder EtherCAT. IO-Module erlauben die Überwachung
und Steuerung von digitalen oder analogen Ein- und Ausgängen.

\subsubsection{Zugriff auf IO-Module%
        \label{sec:2-io}}
Der Zugriff auf die Ein- und Ausgänge der IO-Module erfolgt über ein Prozessabbild
und einen hierfür von Kunbus bereitgestellten Treiber, genannt piControl. Dieser
aktualisiert das Prozessabbild zyklisch. Die angestrebte Zykluszeit beträgt 5ms,
kann jedoch je nach Anzahl der angeschlossenen Module auch größer sein. Kunbus
garantiert bei drei IO-Modulen und zwei Gateway-Modulen eine Zykluszeit von 10 ms.
Jedes der IO-Module stellt ein eigenständiges eingebettetes System dar. Es verfügt
über einen Microcontroller, welcher die IOs bereitstellt und über einen RS485-Bus
mit dem Revolution Pi kommuniziert.
% https://revolution.kunbus.de/io-modul/

Lizenz: GPL
% https://github.com/RevolutionPi/piControl

\begin{lstlisting}[language={c},firstnumber={226},caption={Setzen der Scheduler-Priorität auf SCHED\_FIFO in revpi\_common.c\label{lst:2-sched_priority}}]
param.sched_priority = ktprio->prio;
ret = sched_setscheduler(child, SCHED_FIFO,
       &param);
\end{lstlisting}


\subsection{Echtzeit und Multithreading unter Linux -- preemptRT und posix%
     \label{sec:2-echtzeit}}


 Der Linux-Kernel verfügt über mehrere unterschiedliche Preemtion-Modelle:

\begin{itemize}
  \item No Forced Preemption (server):
  Ausgelegt auf maximal möglichen Durchsatz, lediglich Interrupts und
  System-Call-Returns bewirken Präemption.

  \item Voluntary Kernel Preemption (Desktop):
  Neben den implizit bevorrechtigten Interrupts und System-Call-Returns gibt es
  in diesem Modell weitere Abschnitte des Kernels in welchen Preämption explizit
  gestattet ist.

  \item Preemptible Kernel (Low-Latency Desktop):
  In diesem Modell ist der gesamte Kernel, mit Ausnahme sog.~kritischer Abschnitte
  präemptible. Nach jedem kritischen Abschnitt gibt es einen impliziten Präemptions-Punkt.

  \item Preemptible Kernel (Basic RT):
  Dieses Modell ist dem zuvor genannten sehr ähnlich, hier sind jedoch alle Interrupt-Handler
  als eigenständige Threads ausgeführt.

  \item Fully Preemptible Kernel (RT):
  Wie auch bei den beiden zuvor genannten Modellen ist hier der gesamte Kernel
  präemtible, die Anzahl und Dauer der nicht-präemtiblen kritischen Abschnitte
  ist auf ein notwendiges Minimum beschränkt. Alle Interrupt-Handler sind als
  eigenständige Threads ausgeführt, Spinlocks durch Sleeping-Spinlocks und Mutexe
  durch sog.~RT-Mutexe ersetzt.

\end{itemize}
\todo{Spinlocks und Mutexe sowie die RT-Varianten dieser erklären!}

Lediglich mit dem vollständig präemtiblen Kernel kann Echtzeit-Verhalten realisiert werden.

% https://wiki.linuxfoundation.org/realtime/documentation/technical_basics/preemption_models bzw kernel/Kconfig.preempt

\subsubsection{preemptRT%
        \label{sec:2-preemptRT}}
% https://wiki.linuxfoundation.org/realtime/documentation/technical_details/start
% https://wiki.linuxfoundation.org/realtime/documentation/technical_basics/start

Das dem PREEMPT RT Kernel zugrunde liegende Prinzip lässt sich in einer einfachen
Regel ausdrücken: Nur Code, welcher absolut nicht-präemtible sein darf, ist es
gestattet nicht-präemtible zu sein.
Das erklärte Ziel des PREEMPT\_RT Patches ist es folglich, die Menge des nicht-präemtiblen
Codes im Linux-Kernel auf das absolut notwendige Minimum zu reduzieren.

Dies wird durch Verwendung folgender Mechanismen erreicht:

\begin{itemize}
  \item Hochauflösende Timer
  \item Sleeping Spinlocks
  \item Threaded Interrupt Handlers
  \item rt\_mutex
  \item RCU
\end{itemize}


\subsubsection{posix%
        \label{sec:2-posix}}
Ist posix hier wirklich relevant? Debian bzw.~Raspbian sind weitgehend posix
kompatibel, aber wird es hier genutzt? -> JA, open62541 nutzt pthread.h
piControl nutzt kthread.h, und semaphore.h

\subsection{OPC-UA und open62541%
     \label{sec:2-opc}}

\subsubsection{OPC UA%
        \label{sec:2-opcua}}
Open Platform Communications (OPC) ist eine Familie von Standards zur herstellerunabhängigen
Kommunikation von Maschinen (M2M) in der Automatisierungstechnik. Die sog.~OPC Task Force, zu deren
Mitgliedern verschiedene große Firmen der Automatisierungsindustrie gehören, veröffentlichte
die OPC Specification Version 1.0 im August 1996.
Motiviert ist dieser offene Standard durch die Erkenntniss, dass die Anpassung der
zahlreichen Herstellerstandards an individuelle Infrastrukturen und Anlagen einen
großen Mehraufwand verursachen.
Die Wikipedia beschreibt das Anwendungsgebiet für OPC wie folgt:

\glqq{}OPC wird dort eingesetzt, wo Sensoren, Regler und Steuerungen verschiedener Hersteller
ein gemeinsames Netzwerk bilden. Ohne OPC benötigten zwei Geräte zum Datenaustausch
genaue Kenntnis über die Kommunikationsmöglichkeiten des Gegenübers. Erweiterungen
und Austausch gestalten sich entsprechend schwierig. Mit OPC genügt es, für jedes
Gerät genau einmal einen OPC-konformen Treiber zu schreiben. Idealerweise wird
dieser bereits vom Hersteller zur Verfügung gestellt. Ein OPC-Treiber lässt sich
ohne großen Anpassungsaufwand in beliebig große Steuer- und Überwachungssysteme
integrieren.

OPC unterteilt sich in verschiedene Unterstandards, die für den jeweiligen Anwendungsfall
unabhängig voneinander implementiert werden können. OPC lässt sich damit verwenden
für Echtzeitdaten (Überwachung), Datenarchivierung, Alarm-Meldungen und neuerdings
auch direkt zur Steuerung (Befehlsübermittlung).\grqq{}

OPC basiert in der ursprünglichen Spezifikation auf Microsofts DCOM-Spezifikation.
DCOM macht Funktionen und Objekte einer Anwendung anderen Anwendungen im Netzwerk
zugänglich. Der OPC-Standard definiert entsprechende DCOM-Objekte um mit anderen
OPC-Anwendungen Daten austauschen zu können. Die Verwendung von DCOM bindet Anwender
an Betriebssysteme von Microsoft. Die ursprüngliche OPC Spezifikation wird durch die
Entwicklung von OPC Unified Architecture (OPC UA) abgelöst.
OPC UA setzt auf einem eigenen Kommunikationionsstack auf, die Verwendung von DCOM
und damit die Bindung an Microsoft wurden aufgelöst.

Die OPC-UA-Architektur ist eine Service-orientierte Architektur (SOA), deren Struktur
aus mehreren Schichten besteht.

% Wikipedia
Das OPC-Informationsmodell ist nicht mehr nur eine Hierarchie aus Ordnern, Items
und Properties. Es ist ein sogenanntes Full-Mesh-Network aus Nodes, mit dem neben
den Nutzdaten eines Nodes auch Meta- und Diagnoseinformationen repräsentiert werden.
Ein Node ähnelt einem Objekt aus der objektorientierten Programmierung. Ein Node
kann Attribute besitzen, die gelesen werden können (Data Access (DA), Historical
Data Access (HDA)). Es ist möglich Methoden zu definieren und aufzurufen.
Eine Methode besitzt Aufrufargumente und Rückgabewerte. Sie wird durch ein Command
aufgerufen. Weiterhin werden Events unterstützt, die versendet werden können
(AE (Alarms \& Events), DA DataChange), um bestimmte Informationen zwischen Geräten
auszutauschen. Ein Event besitzt unter anderem einen Empfangszeitpunkt, eine Nachricht
und einen Schweregrad. Die o. g. Nodes werden sowohl für die Nutzdaten als auch
alle anderen Arten von Metadaten verwendet. Der damit modellierte OPC-Adressraum
beinhaltet nun auch ein Typmodell, mit dem sämtliche Datentypen spezifiziert werden.

% https://de.wikipedia.org/wiki/Open_Platform_Communications
% https://de.wikipedia.org/wiki/OPC_Unified_Architecture
% https://opcfoundation.org/developer-tools/specifications-unified-architecture
% Von Gerhard Gappmeier - ascolab GmbH, CC BY-SA 3.0, https://de.wikipedia.org/w/index.php?curid=1892069
\subsubsection{open62541%
        \label{sec:2-open62541}}
open62541 ist eine offene und freie Implementierung von OPC UA. Die in C geschriebene
Bibliothek stellt eine beständig zunehmende Anzahl der im OPC UA Standard definierten
Funktionen bereit. Sie kann sowohl zur Erstellung von OPC-Servern als auch -Clients
genutzt werden. Ergänzend zu der unter der Mozilla Public License v2.0 lizensierten
Bibliothek stellt das open62541 Projekt auch Beispielprogramme unter einer CC0 Lizenz
zur Verfügung.

Die Bibliothek eignet sich auch für die Entwicklung auf eingebetteten Systemen und
Microcontrollern. Je nach Umfang der gewünschten Funktionen und des OPC Informationsmodells
beträgt die Größe einer Server-Binary weniger als 100kb. %evtl. kürzen?

\todo{Nodes erklären! Evtl.~oben!}

Folgende Auswahl an Eigenschaften und Funktionen zeichnet die in dieser Arbeit verwendete
Version 0.3 von open62541 aus:
\begin{itemize}
  \item Kommunikationionsstack
  \begin{itemize}
      \item OPC UA Binär-Protokoll (HTTP oder SOAP werden gegenwärtig nicht unterstützt)
      \item Austauschbare Netzwerk-Schicht, welche die Verwendung eigener Netzwerk-APIs
      erlaubt.
      \item Verschlüsselte Kommunikationion
      \item Asynchrone Dienst-Anfragen im Client
  \end{itemize}
  \item Informationsmodell
  \begin{itemize}
    \item Unterstützung aller OPC UA Node-Typen, inkl.~Methoden
    \item Hinzufügen und Entfernen von Nodes und Referenzen zur Laufzeit.
    \item Vererbung und Instanziierung von Objekt- und Variablentypen
    \item Zugriffskontrolle auch für einzelne Nodes
  \end{itemize}
  \item Subscriptions
  \begin{itemize}
    \item Erlaubt die Überwachung (subscriptions / monitoreditems)
    \item Sehr geringer Ressourcenbedarf pro überwachtem Wert
  \end{itemize}
  \item Code-Generierung auf XML-Basis
  \begin{itemize}
    \item Erlaubt die Erstellung von Datentypen
    \item Erlaubt die Generierung des serverseitigen Informationsmodells
  \end{itemize}
\end{itemize}

% https://open62541.org/doc/0.3/


Mozilla Public License
CC0 Lizenz für Beispiele und Plugins

% https://open62541.org/doc/open62541-current.pdf
% https://open62541.org/

% % % Imports nur für Referenzenauflösung während des Schreibens! Vorm Kompilieren auskommentieren!
% \bibliography{0_hauptdatei}
% \input{1_einleitung}
% \input{2_grundlagen}
% \input{3_konzeption}
% \input{4_implementierung}
% \input{5_tests}
% \input{6_zusammenfassung}
% \input{anhang}
% % Ende Imports

\section{Systemkonzept%
  \label{sec:3-konzeption}}
Auf Basis der in Abschnitt \ref{sec:2-grundlagen} vorgestellten Möglichkeiten folgt nun die Ausarbeitung eines Konzepts.
In den folgenden Abschnitten soll näher auf zwei zentrale Aspekte eingegangen werden: Abschnitt~\ref{sec:3-anbindung} stellt Möglichkeiten zum Zugriff auf Variablen bzw.\,Werte im Prozessabbild des Revolution Pi vor; in Abschnitt~\ref{sec:3-integration} wird ein Konzept zur Bereitstellung dieser Variablen auf einem OPC-Server vorgestellt.

\subsection{Anbindung der IO an den OPC-Server%
     \label{sec:3-anbindung}}

Eine Webanwendung mit Bezeichnung PiCtory dient zur Konfiguration der I/O- und virtuellen Module des RevolutionPi. Die Konfiguration liegt im JSON-Format in der Datei \lstinline{/etc/revpi/config.rsc}. Der piControl-Treiber liest diese Datei beim Start. 
Der folgende Auszug aus der Manpage des piControl-Kernelmoduls beschreibt die von diesem zum Lesen und Schreiben einzelner Bits des Prozessabbildes bereitgestellten Funktionen~\citep[vgl.]{web-revpi-manpage}. Sie ist an dieser Stelle weitgehend ungekürzt zitiert, da sie die nutzbare Schnittstelle sehr kompakt beschreibt.

\begin{lstlisting}[breakindent=0pt, numbers=none, caption={Auszug aus der Revolution Pi Programmers Manual\label{lst:4-manpage}}]
KB_FIND_VARIABLE SPIVariable *argp
Find a variable in the process image by its name. A pointer to a structure of type SPIVariable must be passed as argument. [...]
The struct SPIVariable [...] is defined as 
typedef struct SPIVariableStr
{
    char strVarName[32]; // Variable name
    uint16_t i16uAddress; // Address of the byte in the process image
    uint8_t i8uBit; // 0-7 bit position, >= 8 whole byte
    uint16_t i16uLength; // length of the variable in bits.
    // Possible values are 1, 8, 16 and 32
} SPIVariable;

Set and get values of the process image
KB_GET_VALUE SPIValue *argp
[...]
KB_SET_VALUE SPIValue *argp
Write one bit or one byte to the process image [...].  This call is more efficient than the usual calls of seek and write because only one function call is necessary. If more than on application are writing bits in one output byte, this call is the only safe way to set a bit without overwriting the other bits because this call is doing a read-modify-write-cycle. 

The struct SPIValue used by this ioctl is defined as
typedef struct SPIValueStr
{
    uint16_t i16uAddress; // Address of the byte in the process image
    uint8_t i8uBit; // 0-7 bit position, >= 8 whole byte
    uint8_t i8uValue; // Value: 0/1 for bit access, whole byte otherwise
} SPIValue;
\end{lstlisting} 

Die oben beschriebenden Funtkionen \lstinline{KB_FIND_VARIABLE}, \lstinline{KB_GET_VALUE} und \lstinline{KB_SET_VALUE} ermöglichen einen einfachen und (lt.\,Manpage) effizienten Zugriff auf einzelne Bits des Prozessabbildes und damit auch auf die IO des RevolutionPi.
Der Zugriff des OPC-Servers auf das Prozessabbild soll daher mittels dieser Funktionen realisiert werden.
\lstinline{KB_FIND_VARIABLE} kann genutzt werden, um Adressen von Variablen im Prozessabbild mittels ihres Namens aufzulösen.
\lstinline{KB_GET_VALUE} und \lstinline{KB_SET_VALUE} ermöglichen den Zugriff auf die Werte dieser Variablen.


\subsection{Integration des OPC-Servers in das System%
     \label{sec:3-integration}}

open62541 bietet drei Möglichkeiten zum Abgleich von Variablen mit dem Prozessabbild~\citep[vgl.][Tutorials - Connecting a Variable with a Physical Process]{web-open62541}:
\begin{itemize}
    \item Manuelles oder zyklisches Aktualisieren
    \item Variable Value Callback
    \item Variable Datasource
\end{itemize}

Die zyklische Aktualisierung eines oder mehrerer Werte nimmt, abhängig von der Zykluszeit, viele Systemressourcen in Anspruch. Value Callbacks ermöglichen es, einen Variablenwert effizienter mit einer Ressource wie etwa einem Prozessabbild zu synchronisieren. An die Variable wird ein Callback angehängt, welches vor jedem Lesen und nach jedem Schreibvorgang ausgeführt wird.
Der Wert der Variablen wird weiterhin im Variablenknoten auf dem OPC-Server gespeichert, der Abgleich mit der verknüpften Ressource erfolgt durch die Callback-Methoden.

Sogenannte Datenquellen gehen noch einen Schritt weiter. Der Server leitet jede Lese- und Schreibanforderung direkt an eine Callback-Funktion weiter. Beim Lesen liefert der Rückruf eine Kopie des aktuellen Wertes. Die Datenquelle muss intern ein eigenes Speichermanagement implementieren.

Der Zugriff auf die Werte des Prozessabbildes erfolgt, wie in Abschnitt~\ref{sec:3-anbindung} beschrieben, über von piControl bereitgestellte Methoden. Um die durch open62541 gepflegte OPC-Datenstruktur und das durch piControl verwaltete Prozessabbild möglichst effektiv verknüpfen zu können, soll diese Interaktion mittels Datenquellen und den zugehörigen Callbacks implementiert werden.
% % % Imports nur für Referenzenauflösung während des Schreibens! Vorm Kompilieren auskommentieren!
% \bibliography{0_hauptdatei}
% \input{1_einleitung}
% \input{2_grundlagen}
% \input{3_konzeption}
% \input{4_implementierung}
% \input{5_tests}
% \input{6_zusammenfassung}
% \input{anhang}
% % Ende Imports

\section{Implementierung%
  \label{sec:4-implementierung}}
Das folgende Kapitel stellt in Auszügen die Implementierung des OPC-Servers sowie die Anbindung an die IO-Module
der SPS dar. Der Schwerpunkt liegt hierbei auf der Funktionsweise des piControl-Treibers und dessen Integration in das Projekt. Abschnitt~\ref{sec:4-picontrol} erklärt die zum Schreibens eines Bits verwendeten Funktionsaufrufe.
Zuvor soll jedoch in Abschnitt~\ref{sec:4-open62541} der Teil des OPC-Servers vorgestellt werden, welcher auf besagten Treiber zugreift. 

\subsection{Implementierung des OPC-Servers%
     \label{sec:4-open62541}}
Wie im vorangegangenen Abschnitt~\ref{sec:3-integration} begründet, soll die Verknüpfung zwischen dem Prozessabbild der SPS und den auf dem OPC-Server bereitgestellten Werten über sog.\,Datenquellen erfolgen. Hierzu ist zunächst eine Callback-Methode zu implementieren, welche bei einem Lese- oder Schreibzugriff auf eine Variable aufgerufen wird. Die Verknüpfung zwischen Callback-Methode und Variable muss manuell erfolgen.

\begin{lstlisting}[language={c},firstnumber=237,caption={Auszug der Methode \lstinline{linkDataSourceVariable} in \lstinline{variables.c}\label{lst:4-linkDataSourceVariable}}]
extern UA_StatusCode
 linkDataSourceVariable(UA_Server *server, UA_NodeId nodeId) {
     bool readonly = false;
     UA_DataSource dataSourceVariable;
     UA_StatusCode rc; |>\setcounter{lstnumber}{254}<|

     dataSourceVariable.read = readDataSourceVariable;
     if (!readonly)
        dataSourceVariable.write = writeDataSourceVariable;
     else
        dataSourceVariable.write = writeReadonlyDataSourceVariable;

     return UA_Server_setVariableNode_dataSource(server, nodeId, dataSourceVariable);
 }
\end{lstlisting}

\begin{figure}[h]
    \centering
    \includegraphics[width=0.42\textwidth]{doc/img/OPC_RevPiDO.pdf}
    \caption{Auszug des verwendeten Nodesets, hier Digitalausgang 1 des Versuchsaufbaus
      \label{fig:opc-do}}
\end{figure}

Die in Listing~\ref{lst:4-linkDataSourceVariable} abgebildete Methode \lstinline{linkDataSourceVariable()} erzeugt ein Struct vom Typ \lstinline{UA_DataSource}. In diesem werden dem Lesen und Schreiben einer OPC-Variablen entsprechende Callback-Methoden zugewiesen. Die Verknüpfung einer OPC-Variable, genauer ihrer NodeId, mit der zuvor definierten Datenquelle erfolgt über die von open62541 bereitgestellte Methode \lstinline{UA_Server_setVariableNode_dataSource()}. Vor dem Lesen und nach dem Schreiben dieser Variable werden von nun an die entsprechenden Callbacks aufgerufen.
     
\begin{lstlisting}[language={c},firstnumber=168,caption={Auszug des Callbacks \lstinline{writeDataSourceVariable} in \lstinline{variables.c}\label{lst:4-writeDataSourceVariable}}]  
extern UA_StatusCode
 writeDataSourceVariable(UA_Server *server,
            const UA_NodeId *sessionId, void *sessionContext,
            const UA_NodeId *nodeId, void *nodeContext,
            const UA_NumericRange *range, const UA_DataValue *dataValue) {

    UA_StatusCode retval  = UA_STATUSCODE_GOOD;
    UA_NodeId *nameNodeId = UA_malloc(sizeof(UA_NodeId));
    UA_QualifiedName nameQN = UA_QUALIFIEDNAME(1, "Name");
    UA_Variant nameVar;
    UA_Boolean bit;

    retval |= findSiblingByBrowsename(server, nodeId, &nameQN, nameNodeId);
    retval |= UA_Server_readValue(server, *nameNodeId, &nameVar);
    retval |= UA_Boolean_copy(dataValue->value.data, &bit);

    |>\tikzmarkin[set border color=martinired]{writeIO}<|PI_writeSingleIO(String_fromUA_String(nameVar.data), &bit, false);                                                 |>\tikzmarkend{writeIO}<|

    free(nameNodeId);
    return retval;
 }
\end{lstlisting}

Listing~\ref{lst:4-writeDataSourceVariable} zeigt die Callback-Methode, welche nach dem Schreiben einer Variablen auf dem OPC-Server aufgerufen wird.
Dieser Methode wird neben der NodeId der mit ihr verknüpften Variablen auch der Wert dieser in Form eines Zeigers auf ein Struct vom Typ \lstinline{UA_DataValue} übergeben.

Die Gestaltung des hier verwendeten Nodesets sieht vor, dass in einer OPC-Variablen \lstinline{"Name"} der Bezeichner des zu schreibenden Digitalausgangs hinterlegt ist, siehe Abbildung~\ref{fig:opc-do}. Dies erlaubt eine Rekonfiguration der Ein- und Ausgänge der SPS ohne Änderungen im Programmcode des OPC-Servers vornehmen zu müssen.
Es ist daher erforderlich, nach jedem Schreiben einer mit einem Digitalausgang verknüpften Variablen, hier \lstinline{"Value"}, dessen Bezeichner \lstinline{"Name"} abzufragen. 
Dies geschieht in den Zeilen 180 und 181.
Anschließend wird dieser Bezeichner sowie der zu schreibende Wert der Methode \lstinline{PI_writeSingleIO()} übergeben, welche wiederum die Interaktion mit piControl übernimmt (vgl. Abschnitt \ref{sec:4-picontrol}).
 
\subsection{Integration von piControl%
     \label{sec:4-picontrol}}
In Abschnitt~\ref{sec:2-io} wurde die Anbindung der IO-Module des Revolution Pi sowie die Funktionsweise von piControl aus Anwendersicht beschrieben. Die verfügbare Literatur beschränkt sich auch auf lediglich diese Sicht; eine weiterführende Dokumentation für Entwickler gibt es, neben der in Abschnitt~\ref{sec:3-anbindung} vorgestellten Manpage, nicht. 
In diesem Abschnitt soll daher der Quellcode von piControl sowie dessen Verwendung im Projekt genauer betrachtet werden.
Hierzu wird exemplarisch die in Abschnitt~\ref{sec:4-open62541} eingeführte Methode \lstinline{PI_writeSingleIO()} untersucht.
Diese Methode ermöglicht das Setzen eines einzelnen Bits im Prozessabbild der SPS, und damit das Schalten eines digitalen Ausgangs auf einem IO-Modul.
Die äquivalente Methode \lstinline{int piControlGetBitValue(SPIValue *pSpiValue)} zum Lesen eines Bits bzw. Eingangs funktioniert analog und soll daher an dieser Stelle nicht dediziert erörtert werden.

\begin{lstlisting}[language={c},firstnumber=97,
                   caption={Setzen eines phsikalischen, digitalen Ausgangs in \lstinline{revpi.c}
                   \label{lst:4-PI_writeSingleIO}}]
extern void PI_writeSingleIO(char *pszVariableName, bool *bit, bool verbose)
{
	int rc;
	SPIVariable sPiVariable;
	SPIValue sPIValue;

	strncpy(sPiVariable.strVarName, pszVariableName, sizeof(sPiVariable.strVarName));
	rc = piControlGetVariableInfo(&sPiVariable);
	if (rc < 0) {
		printf("Cannot find variable '%s'\n", pszVariableName);
		return;
	}

		sPIValue.i16uAddress = sPiVariable.i16uAddress;
		sPIValue.i8uBit = sPiVariable.i8uBit;
		sPIValue.i8uValue = *bit;
		rc = |>\tikzmarkin[set border color=martinired]{setBitValue}<|piControlSetBitValue(&sPIValue)|>\tikzmarkend{setBitValue}<|;
		if (rc < 0)
			printf("Set bit error %s\n", getWriteError(rc));
		else if (verbose)
			printf("Set bit %d on byte at offset %d. Value %d\n", sPIValue.i8uBit, sPIValue.i16uAddress,
			       sPIValue.i8uValue);
}
\end{lstlisting}

Der Programmcode in Listing~\ref{lst:4-PI_writeSingleIO} ist Teil des implementierten OPC-Servers. In diesem wird auf zwei Funktionen des piControl-Treibers zugegriffen. 
Beiden Methoden wird als Argument ein Zeiger auf ein Struct vom Typ \lstinline{SPIValue} übergeben. Der im Struct abgelegte Name wird mittels \lstinline{piControlGetVariableInfo(&sPIValue)} zu einer Adresse im Prozessabbild aufgelöst. Diese wird in \lstinline{sPIValue.i16uAdress} gespeichert. Der Wert der Variablen wird anschließend mittels \lstinline{piControlSetBitValue(&sPIValue)} an dieser Adresse in das Prozessabbild geschrieben.

\begin{lstlisting}[language={c},firstnumber=309,caption={Methode \lstinline{piControlSetBitValue} in \lstinline{piControlIf.c}\label{lst:4-piControlSetBitValue}}]
int |>\tikzmarkin[set border color=martiniblue]{setBitValueFcn}<|piControlSetBitValue(SPIValue *pSpiValue)|>\tikzmarkend{setBitValueFcn}<|
{
    piControlOpen();

    if (PiControlHandle_g < 0)
	    return -ENODEV;

    pSpiValue->i16uAddress += pSpiValue->i8uBit / 8;
    pSpiValue->i8uBit %= 8;

    if (|>\tikzmarkin[set border color=martinired]{ioctl}<|ioctl(PiControlHandle_g, KB_SET_VALUE, pSpiValue)|>\tikzmarkend{ioctl}<| < 0)
	    return errno;

    return 0;
}
\end{lstlisting}

Die in Listing~\ref{lst:4-piControlSetBitValue} dargestellte Methode \lstinline{piControlSetBitValue} ist lediglich eine Hüllfunktion (häufig auch als Wrapper-Funktion bezeichnet) für einen Aufruf des \lstinline{ioctl} Kernel-Moduls.
Folgende Parameter werden übergeben:
\lstinline{PiControlHandle_g} ist die Referenz auf die Geräte-Datei des piControl-Treibers. \lstinline{KB_SET_VALUE} ist das ioctl-Kommando zum Schreiben eines Bits in das Prozessabbild. Der Zeiger \lstinline{pSpiValue} verweist auf ein Struct des bereits vorgestellten Typs \lstinline{SPIValue}.

\begin{lstlisting}[language={c},firstnumber=80,caption={Methode \lstinline{piControlOpen} in \lstinline{piControlIf.c}\label{lst:4-piControlOpen}}]
void piControlOpen(void)
{
    /* open handle if needed */
    if (PiControlHandle_g < 0)
    {
	    |>\tikzmarkin[set border color=martiniblue]{PiControlHandle}<|PiControlHandle_g = open(PICONTROL_DEVICE, O_RDWR)|>\tikzmarkend{PiControlHandle}<|;
    }
}
\end{lstlisting}

Die in Listing~\ref{lst:4-piControlOpen} dargestellte Methode öffnet, sofern nicht bereits geschehen, die Geräte-Datei. Das Macro \lstinline{PICONTROL_DEVICE} verweist hierbei auf \lstinline{/dev/piControl0}.

\begin{lstlisting}[language={c},firstnumber=721,caption={Methode \lstinline{piControlIoctl} in \lstinline{piControlMain.c}\label{lst:4-piControlIoctl}}]
static long |>\tikzmarkin[set border color=martiniblue, below offset=0.9em]{piControlIoctl}<|piControlIoctl(struct file *file, unsigned int prg_nr, 
                           unsigned long usr_addr)                                      |>\tikzmarkend{piControlIoctl}<|
{
  int status = -EFAULT;
  tpiControlInst *priv;
  int timeout = 10000;	// ms

  if (prg_nr != KB_CONFIG_SEND && prg_nr != KB_CONFIG_START && !isRunning()) {
  	return -EAGAIN;
  }

  priv = (tpiControlInst *) file->private_data;

  if (prg_nr != KB_GET_LAST_MESSAGE) {
  	// clear old message
  	priv->pcErrorMessage[0] = 0;
  }

  switch (prg_nr) {|>\setcounter{lstnumber}{864}<|

    case |>\tikzmarkin[set border color=martiniblue]{KB_SET_VALUE}<|KB_SET_VALUE:|>\tikzmarkend{KB_SET_VALUE}<|
  		{
  			SPIValue *pValue = (SPIValue *) usr_addr;

  			if (!isRunning())
  				return -EFAULT;

  			if (pValue->i16uAddress >= KB_PI_LEN) {
  				status = -EFAULT;
  			} else {
  				INT8U i8uValue_l;
  				my_rt_mutex_lock(&piDev_g.lockPI);
  				i8uValue_l = piDev_g.ai8uPI[pValue->i16uAddress];

  				if (pValue->i8uBit >= 8) {
  					i8uValue_l = pValue->i8uValue;
  				} else {
  					if (pValue->i8uValue)
  						i8uValue_l |= (1 << pValue->i8uBit);
  					else
  						i8uValue_l &= ~(1 << pValue->i8uBit);
  				}

  				|>\tikzmarkin[set border color=martinired]{i8uValue}<|piDev_g.ai8uPI[pValue->i16uAddress] = i8uValue_l;|>\tikzmarkend{i8uValue}<|
  				rt_mutex_unlock(&piDev_g.lockPI);

  #ifdef VERBOSE
  				pr_info("piControlIoctl Addr=%u, bit=%u: %02x %02x\n", pValue->i16uAddress, pValue->i8uBit, pValue->i8uValue, i8uValue_l);
  #endif

  				status = 0;
  			}
  		}
  		break; |>\setcounter{lstnumber}{1314}<|

    default:
      pr_err("Invalid Ioctl");
      return (-EINVAL);
      break;

    }

    return status;
  }
\end{lstlisting}

Listing~\ref{lst:4-piControlIoctl} zeigt in Auszügen die ioctl-Methode des piControl Kernel-Treibers. Diese bekommt folgende Argumente übergeben: \lstinline{struct file *file} enthält den Verweis auf die Geräte-Datei, hier \lstinline{/dev/piControl0}. Der Wert von \lstinline{unsigned int prg_nr} beschreibt die Anfrage an den Treiber, in diesem Fall \lstinline{KB_SET_VALUE}. Das Argument \lstinline{unsigned long usr_addr} enthält einen typ-agnostischen Pointer. Dieser verweist auf einen Speicherbereich, in welchem die zur Bearbeitung der Anfrage notwendigen Daten abgelegt sind. Hier können auch vom Treiber empfangene Daten dem Anwendungsprogramm bereitgestellt werden. 

Die switch-case-Anweisung führt die über das Argument \lstinline{prg_nr} spezifizierte Aktion aus. Hier betrachten wir \lstinline{KB_SET_VALUE}:
Zunächst wird in Zeile 868 der übergebene Zeiger \lstinline{usr_addr} mittels explizitem Typecast zu einem Zeiger des Typs \lstinline{SPIValue *} konvertiert. Da dieser auf Daten im Userspace verweist, ist beim Zugriff durch den Kernel-Treiber besondere Vorsicht geboten.
In Zeile 877 wird mittels Mutex das Prozessabbild \lstinline{piDev_g} für den Zugriff durch andere Threads oder Prozesse gesperrt.
\lstinline{my_rt_mutex_lock} verweist hierbei auf die Funktion \lstinline{rt_mutex_lock} aus \lstinline{linux/sched.h}\footnote{Offenbar wurde hier auch eine alternative Implementierung vorgesehen, siehe revpi\_common.h}

In Zeile 889 wird das Byte \lstinline{i8uValue_l}, welches den zu schreibenden Wert enthält in das Prozessabbild übertragen. Anschließend wird die Mutex auf \lstinline{piDev_g} wieder entsperrt.
\newpage

\begin{lstlisting}[language={c},firstnumber=62,caption={Auszug des Struct \lstinline{spiControlDev} in \lstinline{piControlMain.h}\label{lst:4-spiControlDev}}]
|>\tikzmarkin[set border color=martiniblue]{spiControlDev}<|typedef struct spiControlDev|>\tikzmarkend{spiControlDev}<| {
	// device driver stuff
	int init_step;
	enum revpi_machine machine_type;
	void *machine;
	struct cdev cdev;	// Char device structure
	struct device *dev;
	struct thermal_zone_device *thermal_zone;

	|>\tikzmarkin[set border color=martiniblue]{processImage}<|// process image stuff
	INT8U ai8uPI[KB_PI_LEN];
	INT8U ai8uPIDefault|>\tikzmarkin[set border color=martinired]{KB_PI_LEN_0}<|[KB_PI_LEN]|>\tikzmarkend{KB_PI_LEN_0}<|;
	struct rt_mutex lockPI;        |>\tikzmarkend{processImage}<|
	bool stopIO;
	piDevices *devs; |>\setcounter{lstnumber}{94}<|
} tpiControlDev;
\end{lstlisting}

Das Prozessabbild ist als Byte-Array der Länge \lstinline{KB_PI_LEN} in Listing~\ref{lst:4-spiControlDev} definiert. Konfigurationsparameter wie \lstinline{KB_PI_LEN} oder die Zykluszeit für den Datenaustausch zwischen SPS und IO-Modulen sind im folgenden Listing~\ref{lst:4-process} definiert.

\begin{lstlisting}[language={c},firstnumber=119,caption={Konfigurationsparameter des Prozessabbildes in project.h\label{lst:4-process}}]
#define INTERVAL_PI_GATE (5*1000*1000)  // 5 ms piGateCommunication |>\setcounter{lstnumber}{128}<|

#define INTERVAL_IO_COM (5*1000*1000)  // 5 ms piIoComm |>\setcounter{lstnumber}{132}<|

#define KB_PD_LEN       512
|>\tikzmarkin[set border color=martiniblue]{KB_PI_LEN_1}<|#define KB_PI_LEN       4096|>\tikzmarkend{KB_PI_LEN_1}<|
\end{lstlisting}

Das zu setzende Bit wurde zu diesem Zeitpunkt erfolgreich in das Prozessabbild der SPS geschrieben.
Es stellt sich die Frage, wie dieses nun an das IO-Modul kommuniziert wird.
Die Kommunikation mit allen angebundenen Modulen ist ebenfalls Aufgabe des piControl-Treibers.

\begin{lstlisting}[language={c},firstnumber=256,caption={Auszug der Methode \lstinline{piIoThread} in \lstinline{revpi_core.c}\label{lst:4-piIoThread}}]
static int piIoThread(void *data)
{
	//TODO int value = 0;
	ktime_t time;
	ktime_t now;
	s64 tDiff;

	hrtimer_init(&piCore_g.ioTimer, CLOCK_MONOTONIC, HRTIMER_MODE_ABS);
	piCore_g.ioTimer.function = piIoTimer;

	pr_info("piIO thread started\n");

	now = hrtimer_cb_get_time(&piCore_g.ioTimer);

	PiBridgeMaster_Reset();

	while (!kthread_should_stop()) {
		if (|>\tikzmarkin[set border color=martinired]{PiBridgeMaster}<|PiBridgeMaster_Run()|>\tikzmarkend{PiBridgeMaster}<| < 0)
			break;
	}

	RevPiDevice_finish();

	pr_info("piIO exit\n");
	return 0;
}
\end{lstlisting}

Der Kernel-Thread \lstinline{piIoThread} ist verantwortlich für den zyklischen Datenaustausch mit den IO-Modulen. In diesem wird fortlaufend die Methode \lstinline{PiBridgeMaster_Run()} aufgerufen, siehe Listing~\ref{lst:4-piIoThread}.

\begin{lstlisting}[language={c},firstnumber=262,caption={Auszug der Methode \lstinline{PiBridgeMaster_Run(void)} in \lstinline{RevPiDevice.c}\label{lst:4-PiBridgeMaster_Run}}]
int PiBridgeMaster_Run(void)
{
	static kbUT_Timer tTimeoutTimer_s;
	static kbUT_Timer tConfigTimeoutTimer_s;
	static int error_cnt;
	static INT8U last_led;
	static unsigned long last_update;
	int ret = 0;
	int i;

	my_rt_mutex_lock(&piCore_g.lockBridgeState);
	if (piCore_g.eBridgeState != piBridgeStop) {
		switch (eRunStatus_s) { |>\setcounter{lstnumber}{514}<|
		    case enPiBridgeMasterStatus_EndOfConfig:|>\setcounter{lstnumber}{621}<|
		    if (|>\tikzmarkin[set border color=martinired]{RevPiDevice}<|RevPiDevice_run()|>\tikzmarkend{RevPiDevice}<|) {
				// an error occured, check error limits |>\setcounter{lstnumber}{641}<|
			} else {
				ret = 1;
			}
			piCore_g.image.drv.i16uRS485ErrorCnt = RevPiDevice_getErrCnt();
			break;
\end{lstlisting}

Die in Listing~\ref{lst:4-PiBridgeMaster_Run} dargestellte Methode ist eine sog. State-Machine. Ist die Konfiguration der IO-Module erfolgreich abgeschlossen, so führt sie bei Aufruf lediglich die Methode \lstinline{RevPiDevice_run()} aus.

\begin{lstlisting}[language={c},firstnumber=140,caption={Auszug der Methode \lstinline{RevPiDevice_run(void)} in \lstinline{RevPiDevice.c}\label{lst:4-RevPiDevice_run}}]
int RevPiDevice_run(void)
{
	INT8U i8uDevice = 0;
	INT32U r;
	int retval = 0;

	RevPiDevices_s.i16uErrorCnt = 0;

	for (i8uDevice = 0; i8uDevice < RevPiDevice_getDevCnt(); i8uDevice++) {
		if (RevPiDevice_getDev(i8uDevice)->i8uActive) {
			switch (RevPiDevice_getDev(i8uDevice)->sId.i16uModulType) {
			case KUNBUS_FW_DESCR_TYP_PI_DIO_14:
			case KUNBUS_FW_DESCR_TYP_PI_DI_16:
			case KUNBUS_FW_DESCR_TYP_PI_DO_16:
				r = |>\tikzmarkin[set border color=martinired]{sendCyclicTelegram}<|piDIOComm_sendCyclicTelegram(i8uDevice)|>\tikzmarkend{sendCyclicTelegram}\setcounter{lstnumber}{166} <|;

				break; |>\setcounter{lstnumber}{216}<|
			}
		}
	} |>\setcounter{lstnumber}{227}<|
	return retval;
}
\end{lstlisting}

Diese iteriert wie in Listing~\ref{lst:4-RevPiDevice_run} abgebildete durch alle gegenwärtig in der SPS konfigurierten Module. Ist das aktuelle Modul als aktiv markiert, so wird anhand eines sog. Firmware-Descriptors entschieden, welche Methode für die Ansteuerung des Moduls aufzurufen ist.

\begin{lstlisting}[language={c},firstnumber=161,caption={Auszug der Methode \lstinline{piDIOComm_sendCyclicTelegram} in \lstinline{piDIOComm.c}\label{lst:4-sendCyclicTelegram}}]
INT32U piDIOComm_sendCyclicTelegram(INT8U i8uDevice_p)
{
	INT32U i32uRv_l = 0;
	SIOGeneric sRequest_l;
	SIOGeneric sResponse_l;
	INT8U len_l, data_out[18], i, p, data_in[70];
	INT8U i8uAddress;
	int ret; |>\setcounter{lstnumber}{239}<|
	
    |>\tikzmarkin[set border color=martinired]{piIoComm}<|ret = piIoComm_send((INT8U *) & sRequest_l, IOPROTOCOL_HEADER_LENGTH + len_l + 1);  |>\tikzmarkend{piIoComm}\setcounter{lstnumber}{298}<|
}
\end{lstlisting}

Im Falle des hier verwendeten DO-Moduls wird die in Listing~\ref{lst:4-sendCyclicTelegram} abgebildete Methode \lstinline{piDIOComm_sendCyclicTelegram()} aufgerufen. Dieser wird ein Zeiger auf das zu schreibende Gerät übergeben. 
Zunächst wird das Prozessabbild mittels eines proprietären, jedoch im Quellcode offen nachvollziehbaren Protokolls in ein \lstinline{sRequest_l} genanntes Byte-Array umgewandelt. Dieser Schritt ist in Listing~\ref{lst:4-sendCyclicTelegram} nicht abgebildet. Anschließend wird \lstinline{piIoComm_send()} ein Zeiger auf die so generierte Schreib-Anfrage übergeben.

\begin{lstlisting}[language={c},firstnumber=220,caption={Auszug der Methode \lstinline{piIOComm_send} in \lstinline{piIOComm.c}\label{lst:4-piIOComm_send}}]
int piIoComm_send(INT8U * buf_p, INT16U i16uLen_p)
{
	ssize_t write_l = 0;
	INT16U i16uSent_l = 0;|>\setcounter{lstnumber}{249}<|

	while (i16uSent_l < i16uLen_p) {
		write_l = vfs_write(piIoComm_fd_m, buf_p + i16uSent_l, i16uLen_p - i16uSent_l, &piIoComm_fd_m->f_pos);
		if (write_l < 0) {
			pr_info_serial("write error %d\n", (int)write_l);
			return -1;
		} 
		i16uSent_l += write_l;|>\setcounter{lstnumber}{263}<|
	}
	clear();
	vfs_fsync(piIoComm_fd_m, 1);
	return 0;
}
\end{lstlisting}

Listing~\ref{lst:4-piIOComm_send} zeigt die Implementierung von \lstinline{piIoComm_send()}. Diese Methode ist für das Schreiben der oben generierten Anfrage auf die seriellen Schnittstelle verantwortlich. Realisiert wird dies mittels der Methode \lstinline{vfs_write()}. Diese ist in \lstinline{<linux/fs.h>} definiert. Sie ermöglicht das Schreiben einer Datei im Userspace aus dem Kernel heraus. Geschrieben wird hier die Datei mit dem Deskriptor \lstinline{piIoComm_fd_m}.
Da die Funktion \lstinline{vfs_write()} durch andere Kernel-Tasks unterbrochen werden kann, ist nicht gewährleistet, dass die gesamte Anfrage mit nur einem Aufruf geschrieben wird. Die oben abgebildete while-Schleife stellt das vollständige Senden der Anfrage sicher.

\begin{lstlisting}[language={c},firstnumber=157,caption={Auszug der Methode \lstinline{piIOComm_open_serial} in \lstinline{piIOComm.c}\label{lst:4-piIOComm_open_serial}}]
int piIoComm_open_serial(void)
{   |>\setcounter{lstnumber}{167}<|
	struct file *fd;	/* Filedeskriptor */
	struct termios newtio;	/* Schnittstellenoptionen */

	|>\tikzmarkin[set border color=martiniblue]{fd}<|/* Port oeffnen - read/write, kein "controlling tty", 
	    Status von DCD ignorieren */
	fd = filp_open(|>\tikzmarkin[set border color=martinired]{tty}<|REV_PI_TTY_DEVICE|>\tikzmarkend{tty}<|, O_RDWR | O_NOCTTY, 0); |>\setcounter{lstnumber}{208}<|
	
	piIoComm_fd_m = fd;                                                      |>\tikzmarkend{fd}\setcounter{lstnumber}{217}<|

	return 0;
}
\end{lstlisting}

Der zum Schreiben auf die serielle Schnittstelle verwendete Datei-Deskriptor wird von der in Listing~\ref{lst:4-piIOComm_open_serial} abgebildeten Methode \lstinline{piIoComm_open_serial()} generiert. 

\begin{lstlisting}[language={c},firstnumber=45,caption={Definition der seriellen Schnittstelle in \lstinline{piIOComm.h}\label{lst:4-REV_PI_TTY_DEVICE}}]
#define REV_PI_TTY_DEVICE	"/dev/ttyAMA0"
\end{lstlisting}

Das in Listing~\ref{lst:4-REV_PI_TTY_DEVICE} definierte Macro verweist auf eine der seriellen Schnittstellen des RaspberryPi.
Die Implementierung des zugehörigen Schnittstellentreibers soll hier nicht weiter untersucht werden. Somit ist an dieser Stelle die Kette vom Setzen einer Variablen auf dem OPC-Server bis hin zur Aktualisierung des Prozessabbilds der IO-Module geschlossen.

% \begin{lstlisting}[language={c},firstnumber={226},caption={Setzen der Scheduler-Priorität auf SCHED\_FIFO in 
% revpi\_common.c\label{lst:2-sched_priority}}]
% param.sched_priority = ktprio->prio;
% ret = sched_setscheduler(child, SCHED_FIFO, &param);
% \end{lstlisting}
% % % Imports nur für Referenzenauflösung während des Schreibens! Vorm Kompilieren auskommentieren!
% \bibliography{0_hauptdatei}
% \input{1_einleitung}
% \input{2_grundlagen}
% \input{3_konzeption}
% \input{4_implementierung}
% \input{5_tests}
% \input{6_zusammenfassung}
% % Ende Imports

\section{Test des OPC-Servers im Gesamtsystem%
  \label{sec:5-tests}}

% % % Imports nur für Referenzenauflösung während des schreibens! Vorm Kompilieren auskommentieren!
% \bibliography{0_hauptdatei}
% \input{1_einleitung}
% \input{2_grundlagen}
% \input{3_konzeption}
% \input{4_implementierung}
% \input{5_tests}
% \input{6_zusammenfassung}
% % Ende Imports

\section{Zusammenfassung und Ausblick%
  \label{sec:6-fazit}}
Der folgende Abschnitt~\ref{sec:6-zusammenfassung} fasst die gewonnenen Erkenntnisse und den Stand der Implementierung zusammen.
Den Abschluss dieser Arbeit bildet der Ausblick in Abschnitt~\ref{sec:6-ausblick}.

\subsection{Zusammenfassung%
     \label{sec:6-zusammenfassung}}

\subsection{Ausblick%
     \label{sec:6-ausblick}}

% % Ende Imports

\section{Grundlagen%
  \label{sec:2-grundlagen}}

\subsection{Speicherprogrammierbare-Steuerung und Linux -- Revolution Pi%
     \label{sec:2-sps}}

\subsubsection{Kunbus RevolutionPi%
        \label{sec:2-revpi}}
Der RevolutionPi 3 ist eine speicherprogrammierbare Steuerung (SPS) des Herstellers
Kunbus GmbH. Kern dieser SPS ist das von der Raspberry Pi Foundation entwickelte
und vertriebene Raspberry Pi Compute Module 3. Dieses integriert ein Broadcom BCM2837
System-on-Chip (SoC) mit vier 1,2GHz Prozessorkernen, 1GB RAM, 4GB eMMC Anwendungsspeicher
und sonstige Peripherie in ein Modul im DDR2-SODIMM Formfaktor. Diese Spezifikationen
sind weitgehend identisch zu denen des ausgesprochen populären Raspberry Pi 3.
Der Revolution Pi profitiert daher von dem gleichen großen Angebot an Software
und Unterstützung wie der Raspberry Pi, ergänzt dessen Hardware jedoch um eine 24V
Spannungsversorgung, die Möglichkeit der Erweiterung durch mehrere industrietaugliche
Ein-/ Ausgabemodule und Gateways sowie ein Gehäuse zur Montage auf einer DIN-Schiene.
\begin{itemize}
  \item{Prozessor: BCM2837}
  \item{Taktfrequenz 1,2 GHz}
  \item{Anzahl Prozessorkerne: 4}
  \item{Arbeitsspeicher: 1 GByte}
  \item{eMMC Flash Speicher: 4 GByte}
  \item{Betriebssystem: Angepasstes Raspbian mit RT-Patch}
  \item{RTC mit 24h Pufferung über wartungsfreien Kondensator}
  \item{Treiber / API: Treiber schreibt zyklisch Prozessdaten in ein Prozessabbild, Zugriff auf Prozessabbild über Linux-Filesystem als API zu Fremdsoftware.}
  \item{Kommunikationsanschlüsse: 2 x USB 2.0 A (je 500 mA belastbar), 1 x Micro-USB, HDMI, Ethernet (RJ45) 10/100 Mbit/s}
  \item{Stromversorgung: min. 10,7 V, max. 28,8 V, maximal 10 Watt}
  \item{Zulässige Umgebungstemperatur: -40 bis +55 C}
  \item{Gehäuseabmessungen: (HxBxL) 96 mm x 22,5 mm x 110,5 mm (ohne gesteckte Stecker)}
  \item{ESD Schutz: 4 kV / 8 kV gemäß EN61131-2 und IEC 61000-6-2}
  \item{Surge / Burst Prüfungen: gemäß EN61131-2 und IEC 61000-6-2 eingekoppelt auf Versorgungsspannung, Ethernet und IO-Leitungen}
  \item{EMI Prüfungen: gemäß EN61131-2 und IEC 61000-6-2}
\end{itemize}

Kunbus bietet eine Auswahl an IO- und Gateway-Modulen zur Erweiterung des Revolution Pi an.
Gateways dienen der Kommunikation mit Systemen oder Komponenten der Automatisierungstechnik
über Protokolle wie PROFIBUS oder EtherCAT. IO-Module erlauben die Überwachung
und Steuerung von digitalen oder analogen Ein- und Ausgängen.

\subsubsection{Zugriff auf IO-Module%
        \label{sec:2-io}}
Der Zugriff auf die Ein- und Ausgänge der IO-Module erfolgt über ein Prozessabbild
und einen hierfür von Kunbus bereitgestellten Treiber, genannt piControl. Dieser
aktualisiert das Prozessabbild zyklisch. Die angestrebte Zykluszeit beträgt 5ms,
kann jedoch je nach Anzahl der angeschlossenen Module auch größer sein. Kunbus
garantiert bei drei IO-Modulen und zwei Gateway-Modulen eine Zykluszeit von 10 ms.
Jedes der IO-Module stellt ein eigenständiges eingebettetes System dar. Es verfügt
über einen Microcontroller, welcher die IOs bereitstellt und über einen RS485-Bus
mit dem Revolution Pi kommuniziert.
% https://revolution.kunbus.de/io-modul/

Lizenz: GPL
% https://github.com/RevolutionPi/piControl

\begin{lstlisting}[language={c},firstnumber={226},caption={Setzen der Scheduler-Priorität auf SCHED\_FIFO in revpi\_common.c\label{lst:2-sched_priority}}]
param.sched_priority = ktprio->prio;
ret = sched_setscheduler(child, SCHED_FIFO,
       &param);
\end{lstlisting}


\subsection{Echtzeit und Multithreading unter Linux -- preemptRT und posix%
     \label{sec:2-echtzeit}}


 Der Linux-Kernel verfügt über mehrere unterschiedliche Preemtion-Modelle:

\begin{itemize}
  \item No Forced Preemption (server):
  Ausgelegt auf maximal möglichen Durchsatz, lediglich Interrupts und
  System-Call-Returns bewirken Präemption.

  \item Voluntary Kernel Preemption (Desktop):
  Neben den implizit bevorrechtigten Interrupts und System-Call-Returns gibt es
  in diesem Modell weitere Abschnitte des Kernels in welchen Preämption explizit
  gestattet ist.

  \item Preemptible Kernel (Low-Latency Desktop):
  In diesem Modell ist der gesamte Kernel, mit Ausnahme sog.~kritischer Abschnitte
  präemptible. Nach jedem kritischen Abschnitt gibt es einen impliziten Präemptions-Punkt.

  \item Preemptible Kernel (Basic RT):
  Dieses Modell ist dem zuvor genannten sehr ähnlich, hier sind jedoch alle Interrupt-Handler
  als eigenständige Threads ausgeführt.

  \item Fully Preemptible Kernel (RT):
  Wie auch bei den beiden zuvor genannten Modellen ist hier der gesamte Kernel
  präemtible, die Anzahl und Dauer der nicht-präemtiblen kritischen Abschnitte
  ist auf ein notwendiges Minimum beschränkt. Alle Interrupt-Handler sind als
  eigenständige Threads ausgeführt, Spinlocks durch Sleeping-Spinlocks und Mutexe
  durch sog.~RT-Mutexe ersetzt.

\end{itemize}
\todo{Spinlocks und Mutexe sowie die RT-Varianten dieser erklären!}

Lediglich mit dem vollständig präemtiblen Kernel kann Echtzeit-Verhalten realisiert werden.

% https://wiki.linuxfoundation.org/realtime/documentation/technical_basics/preemption_models bzw kernel/Kconfig.preempt

\subsubsection{preemptRT%
        \label{sec:2-preemptRT}}
% https://wiki.linuxfoundation.org/realtime/documentation/technical_details/start
% https://wiki.linuxfoundation.org/realtime/documentation/technical_basics/start

Das dem PREEMPT RT Kernel zugrunde liegende Prinzip lässt sich in einer einfachen
Regel ausdrücken: Nur Code, welcher absolut nicht-präemtible sein darf, ist es
gestattet nicht-präemtible zu sein.
Das erklärte Ziel des PREEMPT\_RT Patches ist es folglich, die Menge des nicht-präemtiblen
Codes im Linux-Kernel auf das absolut notwendige Minimum zu reduzieren.

Dies wird durch Verwendung folgender Mechanismen erreicht:

\begin{itemize}
  \item Hochauflösende Timer
  \item Sleeping Spinlocks
  \item Threaded Interrupt Handlers
  \item rt\_mutex
  \item RCU
\end{itemize}


\subsubsection{posix%
        \label{sec:2-posix}}
Ist posix hier wirklich relevant? Debian bzw.~Raspbian sind weitgehend posix
kompatibel, aber wird es hier genutzt? -> JA, open62541 nutzt pthread.h
piControl nutzt kthread.h, und semaphore.h

\subsection{OPC-UA und open62541%
     \label{sec:2-opc}}

\subsubsection{OPC UA%
        \label{sec:2-opcua}}
Open Platform Communications (OPC) ist eine Familie von Standards zur herstellerunabhängigen
Kommunikation von Maschinen (M2M) in der Automatisierungstechnik. Die sog.~OPC Task Force, zu deren
Mitgliedern verschiedene große Firmen der Automatisierungsindustrie gehören, veröffentlichte
die OPC Specification Version 1.0 im August 1996.
Motiviert ist dieser offene Standard durch die Erkenntniss, dass die Anpassung der
zahlreichen Herstellerstandards an individuelle Infrastrukturen und Anlagen einen
großen Mehraufwand verursachen.
Die Wikipedia beschreibt das Anwendungsgebiet für OPC wie folgt:

\glqq{}OPC wird dort eingesetzt, wo Sensoren, Regler und Steuerungen verschiedener Hersteller
ein gemeinsames Netzwerk bilden. Ohne OPC benötigten zwei Geräte zum Datenaustausch
genaue Kenntnis über die Kommunikationsmöglichkeiten des Gegenübers. Erweiterungen
und Austausch gestalten sich entsprechend schwierig. Mit OPC genügt es, für jedes
Gerät genau einmal einen OPC-konformen Treiber zu schreiben. Idealerweise wird
dieser bereits vom Hersteller zur Verfügung gestellt. Ein OPC-Treiber lässt sich
ohne großen Anpassungsaufwand in beliebig große Steuer- und Überwachungssysteme
integrieren.

OPC unterteilt sich in verschiedene Unterstandards, die für den jeweiligen Anwendungsfall
unabhängig voneinander implementiert werden können. OPC lässt sich damit verwenden
für Echtzeitdaten (Überwachung), Datenarchivierung, Alarm-Meldungen und neuerdings
auch direkt zur Steuerung (Befehlsübermittlung).\grqq{}

OPC basiert in der ursprünglichen Spezifikation auf Microsofts DCOM-Spezifikation.
DCOM macht Funktionen und Objekte einer Anwendung anderen Anwendungen im Netzwerk
zugänglich. Der OPC-Standard definiert entsprechende DCOM-Objekte um mit anderen
OPC-Anwendungen Daten austauschen zu können. Die Verwendung von DCOM bindet Anwender
an Betriebssysteme von Microsoft. Die ursprüngliche OPC Spezifikation wird durch die
Entwicklung von OPC Unified Architecture (OPC UA) abgelöst.
OPC UA setzt auf einem eigenen Kommunikationionsstack auf, die Verwendung von DCOM
und damit die Bindung an Microsoft wurden aufgelöst.

Die OPC-UA-Architektur ist eine Service-orientierte Architektur (SOA), deren Struktur
aus mehreren Schichten besteht.

% Wikipedia
Das OPC-Informationsmodell ist nicht mehr nur eine Hierarchie aus Ordnern, Items
und Properties. Es ist ein sogenanntes Full-Mesh-Network aus Nodes, mit dem neben
den Nutzdaten eines Nodes auch Meta- und Diagnoseinformationen repräsentiert werden.
Ein Node ähnelt einem Objekt aus der objektorientierten Programmierung. Ein Node
kann Attribute besitzen, die gelesen werden können (Data Access (DA), Historical
Data Access (HDA)). Es ist möglich Methoden zu definieren und aufzurufen.
Eine Methode besitzt Aufrufargumente und Rückgabewerte. Sie wird durch ein Command
aufgerufen. Weiterhin werden Events unterstützt, die versendet werden können
(AE (Alarms \& Events), DA DataChange), um bestimmte Informationen zwischen Geräten
auszutauschen. Ein Event besitzt unter anderem einen Empfangszeitpunkt, eine Nachricht
und einen Schweregrad. Die o. g. Nodes werden sowohl für die Nutzdaten als auch
alle anderen Arten von Metadaten verwendet. Der damit modellierte OPC-Adressraum
beinhaltet nun auch ein Typmodell, mit dem sämtliche Datentypen spezifiziert werden.

% https://de.wikipedia.org/wiki/Open_Platform_Communications
% https://de.wikipedia.org/wiki/OPC_Unified_Architecture
% https://opcfoundation.org/developer-tools/specifications-unified-architecture
% Von Gerhard Gappmeier - ascolab GmbH, CC BY-SA 3.0, https://de.wikipedia.org/w/index.php?curid=1892069
\subsubsection{open62541%
        \label{sec:2-open62541}}
open62541 ist eine offene und freie Implementierung von OPC UA. Die in C geschriebene
Bibliothek stellt eine beständig zunehmende Anzahl der im OPC UA Standard definierten
Funktionen bereit. Sie kann sowohl zur Erstellung von OPC-Servern als auch -Clients
genutzt werden. Ergänzend zu der unter der Mozilla Public License v2.0 lizensierten
Bibliothek stellt das open62541 Projekt auch Beispielprogramme unter einer CC0 Lizenz
zur Verfügung.

Die Bibliothek eignet sich auch für die Entwicklung auf eingebetteten Systemen und
Microcontrollern. Je nach Umfang der gewünschten Funktionen und des OPC Informationsmodells
beträgt die Größe einer Server-Binary weniger als 100kb. %evtl. kürzen?

\todo{Nodes erklären! Evtl.~oben!}

Folgende Auswahl an Eigenschaften und Funktionen zeichnet die in dieser Arbeit verwendete
Version 0.3 von open62541 aus:
\begin{itemize}
  \item Kommunikationionsstack
  \begin{itemize}
      \item OPC UA Binär-Protokoll (HTTP oder SOAP werden gegenwärtig nicht unterstützt)
      \item Austauschbare Netzwerk-Schicht, welche die Verwendung eigener Netzwerk-APIs
      erlaubt.
      \item Verschlüsselte Kommunikationion
      \item Asynchrone Dienst-Anfragen im Client
  \end{itemize}
  \item Informationsmodell
  \begin{itemize}
    \item Unterstützung aller OPC UA Node-Typen, inkl.~Methoden
    \item Hinzufügen und Entfernen von Nodes und Referenzen zur Laufzeit.
    \item Vererbung und Instanziierung von Objekt- und Variablentypen
    \item Zugriffskontrolle auch für einzelne Nodes
  \end{itemize}
  \item Subscriptions
  \begin{itemize}
    \item Erlaubt die Überwachung (subscriptions / monitoreditems)
    \item Sehr geringer Ressourcenbedarf pro überwachtem Wert
  \end{itemize}
  \item Code-Generierung auf XML-Basis
  \begin{itemize}
    \item Erlaubt die Erstellung von Datentypen
    \item Erlaubt die Generierung des serverseitigen Informationsmodells
  \end{itemize}
\end{itemize}

% https://open62541.org/doc/0.3/


Mozilla Public License
CC0 Lizenz für Beispiele und Plugins

% https://open62541.org/doc/open62541-current.pdf
% https://open62541.org/

% % % Imports nur für Referenzenauflösung während des Schreibens! Vorm Kompilieren auskommentieren!
% \bibliography{0_hauptdatei}
% % Mit \section{...} eröffnen wir einen neuen Abschnitt.
% Der Befehl setzt nicht nur den Text in einer größeren,
% fetten Schrift, sondern sorgt außerdem dafür, daß er im
% Inhaltsverzeichnis erscheint.
%
% Mit \label{...} erzeugen wir einen Bezeichner, mit dessen Hilfe
% wir später auf die Nummer des Abschnitts verweisen können (nämlich
% mit~\ref{...}).
%
% Das Kommentarzeichen hinter „Übersicht“ dient dazu, ein
% Leerzeichen zwischen „Übersicht“ und dem \label-Befehl
% zu vermeiden, das andernfalls sichtbar würde – z.B. im
% Inhaltsverzeichnis.
%

% % Imports nur für Referenzenauflösung während des Schreibens! Vorm Kompilieren auskommentieren!
% \bibliography{0_hauptdatei}
% \input{1_einleitung}
%\input{2_grundlagen}
%\input{3_konzeption}
%\input{4_implementierung}
%\input{5_tests}
%\input{6_zusammenfassung}
% % Ende Imports

\section{Einleitung und Motivation%
  \label{sec:1-einleitung}}
Ziel dieses Projektes ist die Integration eines OPC-Servers mit einer auf Linux
basierenden speicherprogrammierbaren Steuerung (SPS). Angeschlossen an diese SPS
ist jeweils ein digitales Ein-/\,bzw.~Ausgabemodul. Die von diesen bereitgestellten
Ein-/\, bzw.~Ausgänge (IO) sollen in der Datenstruktur des OPC-Servers abgebildet
und über diesen für OPC-Clients les-/\,und schreibar sein. Weiterhin sollen einige
Funktionen zur Überwachung und Steuerung der an die SPS angeschlossenen Aktoren
und Sensoren direkt im OPC-Server implementiert werden.
Hiermit stellt dieses Projekt eine der Grundlagen für ein übergeordnetes Projekt,
die cloudbasierte Steuerung eines miniaturisierten Produktions-Systems, dar.

Der hier verwendete OPC-Server ist Teil des sog. open62541 Projekts. Er ist in C
geschrieben und implementiert bereits einen großen Teil der im OPC-UA-Standard
spezifizierten Funktionen.
Als SPS findet ein Revolution Pi 3 der Firma Kunbus Verwendung. Dieser integriert
ein sog. Compute Module der Raspberry Pi Foundation in ein industrietaugliches
Gehäuse und erlaubt die Erweiterung mittels IO- oder Gateway-Modulen. Über diese
erfolgt die Kommunikation mit weiteren Komponenten der Automatisierungstechnik.

Motiviert ist dieses Projekt durch die Beobachtung, dass die Verbreitung offener
Standards sowie freier Software auch in der Automatisierungstechnik zunimmt.
Linux ist ein freies Betriebssystem, OPC-UA ein offen zugänglicher, aktiv gepflegter
und weit verbreiteter Standard. Der Raspberry Pi findet sowohl bei Hobby-Anwendern als
auch in den Bereichen Forschung und Entwicklung sowie bei industriellen Anwendern
Verwendung. Dieses Projekt stellt somit eine für unterschiedliche Anwender interessante
Entwicklung dar.

Im Anschluss an diese einleitende Übersicht im Abschnitt~\ref{sec:1-einleitung} folgt
die Darstellung der wichtigsten Grundlagen in Abschnitt~\ref{sec:2-grundlagen}.
Aufbauend auf diesen Grundlagen folgt die konzeptuelle Ausarbeitung im Abschnitt~\ref{sec:3-konzeption}.
Die Umsetzung wird im Abschnitt~\ref{sec:4-implementierung} erläutert.
Die Leistungsfähigkeit der Implementierung wird in Abschnitt~\ref{sec:5-tests} untersucht.
Eine Zusammenfassung und ein Ausblick schließen die Arbeit in
Abschnitt~\ref{sec:6-fazit} ab. Eventuell noch benötigte Anhänge
finden sich in den Anhängen [...] bis [...].

% % % Imports nur für Referenzenauflösung während des Schreibens! Vorm Kompilieren auskommentieren!
% \bibliography{0_hauptdatei}
% \input{1_einleitung}
% \input{2_grundlagen}
% \input{3_konzeption}
% \input{4_implementierung}
% \input{5_tests}
% \input{6_zusammenfassung}
% % Ende Imports

\section{Grundlagen%
  \label{sec:2-grundlagen}}

\subsection{Speicherprogrammierbare-Steuerung und Linux -- Revolution Pi%
     \label{sec:2-sps}}

\subsubsection{Kunbus RevolutionPi%
        \label{sec:2-revpi}}
Der RevolutionPi 3 ist eine speicherprogrammierbare Steuerung (SPS) des Herstellers
Kunbus GmbH. Kern dieser SPS ist das von der Raspberry Pi Foundation entwickelte
und vertriebene Raspberry Pi Compute Module 3. Dieses integriert ein Broadcom BCM2837
System-on-Chip (SoC) mit vier 1,2GHz Prozessorkernen, 1GB RAM, 4GB eMMC Anwendungsspeicher
und sonstige Peripherie in ein Modul im DDR2-SODIMM Formfaktor. Diese Spezifikationen
sind weitgehend identisch zu denen des ausgesprochen populären Raspberry Pi 3.
Der Revolution Pi profitiert daher von dem gleichen großen Angebot an Software
und Unterstützung wie der Raspberry Pi, ergänzt dessen Hardware jedoch um eine 24V
Spannungsversorgung, die Möglichkeit der Erweiterung durch mehrere industrietaugliche
Ein-/ Ausgabemodule und Gateways sowie ein Gehäuse zur Montage auf einer DIN-Schiene.
\begin{itemize}
  \item{Prozessor: BCM2837}
  \item{Taktfrequenz 1,2 GHz}
  \item{Anzahl Prozessorkerne: 4}
  \item{Arbeitsspeicher: 1 GByte}
  \item{eMMC Flash Speicher: 4 GByte}
  \item{Betriebssystem: Angepasstes Raspbian mit RT-Patch}
  \item{RTC mit 24h Pufferung über wartungsfreien Kondensator}
  \item{Treiber / API: Treiber schreibt zyklisch Prozessdaten in ein Prozessabbild, Zugriff auf Prozessabbild über Linux-Filesystem als API zu Fremdsoftware.}
  \item{Kommunikationsanschlüsse: 2 x USB 2.0 A (je 500 mA belastbar), 1 x Micro-USB, HDMI, Ethernet (RJ45) 10/100 Mbit/s}
  \item{Stromversorgung: min. 10,7 V, max. 28,8 V, maximal 10 Watt}
  \item{Zulässige Umgebungstemperatur: -40 bis +55 C}
  \item{Gehäuseabmessungen: (HxBxL) 96 mm x 22,5 mm x 110,5 mm (ohne gesteckte Stecker)}
  \item{ESD Schutz: 4 kV / 8 kV gemäß EN61131-2 und IEC 61000-6-2}
  \item{Surge / Burst Prüfungen: gemäß EN61131-2 und IEC 61000-6-2 eingekoppelt auf Versorgungsspannung, Ethernet und IO-Leitungen}
  \item{EMI Prüfungen: gemäß EN61131-2 und IEC 61000-6-2}
\end{itemize}

Kunbus bietet eine Auswahl an IO- und Gateway-Modulen zur Erweiterung des Revolution Pi an.
Gateways dienen der Kommunikation mit Systemen oder Komponenten der Automatisierungstechnik
über Protokolle wie PROFIBUS oder EtherCAT. IO-Module erlauben die Überwachung
und Steuerung von digitalen oder analogen Ein- und Ausgängen.

\subsubsection{Zugriff auf IO-Module%
        \label{sec:2-io}}
Der Zugriff auf die Ein- und Ausgänge der IO-Module erfolgt über ein Prozessabbild
und einen hierfür von Kunbus bereitgestellten Treiber, genannt piControl. Dieser
aktualisiert das Prozessabbild zyklisch. Die angestrebte Zykluszeit beträgt 5ms,
kann jedoch je nach Anzahl der angeschlossenen Module auch größer sein. Kunbus
garantiert bei drei IO-Modulen und zwei Gateway-Modulen eine Zykluszeit von 10 ms.
Jedes der IO-Module stellt ein eigenständiges eingebettetes System dar. Es verfügt
über einen Microcontroller, welcher die IOs bereitstellt und über einen RS485-Bus
mit dem Revolution Pi kommuniziert.
% https://revolution.kunbus.de/io-modul/

Lizenz: GPL
% https://github.com/RevolutionPi/piControl

\begin{lstlisting}[language={c},firstnumber={226},caption={Setzen der Scheduler-Priorität auf SCHED\_FIFO in revpi\_common.c\label{lst:2-sched_priority}}]
param.sched_priority = ktprio->prio;
ret = sched_setscheduler(child, SCHED_FIFO,
       &param);
\end{lstlisting}


\subsection{Echtzeit und Multithreading unter Linux -- preemptRT und posix%
     \label{sec:2-echtzeit}}


 Der Linux-Kernel verfügt über mehrere unterschiedliche Preemtion-Modelle:

\begin{itemize}
  \item No Forced Preemption (server):
  Ausgelegt auf maximal möglichen Durchsatz, lediglich Interrupts und
  System-Call-Returns bewirken Präemption.

  \item Voluntary Kernel Preemption (Desktop):
  Neben den implizit bevorrechtigten Interrupts und System-Call-Returns gibt es
  in diesem Modell weitere Abschnitte des Kernels in welchen Preämption explizit
  gestattet ist.

  \item Preemptible Kernel (Low-Latency Desktop):
  In diesem Modell ist der gesamte Kernel, mit Ausnahme sog.~kritischer Abschnitte
  präemptible. Nach jedem kritischen Abschnitt gibt es einen impliziten Präemptions-Punkt.

  \item Preemptible Kernel (Basic RT):
  Dieses Modell ist dem zuvor genannten sehr ähnlich, hier sind jedoch alle Interrupt-Handler
  als eigenständige Threads ausgeführt.

  \item Fully Preemptible Kernel (RT):
  Wie auch bei den beiden zuvor genannten Modellen ist hier der gesamte Kernel
  präemtible, die Anzahl und Dauer der nicht-präemtiblen kritischen Abschnitte
  ist auf ein notwendiges Minimum beschränkt. Alle Interrupt-Handler sind als
  eigenständige Threads ausgeführt, Spinlocks durch Sleeping-Spinlocks und Mutexe
  durch sog.~RT-Mutexe ersetzt.

\end{itemize}
\todo{Spinlocks und Mutexe sowie die RT-Varianten dieser erklären!}

Lediglich mit dem vollständig präemtiblen Kernel kann Echtzeit-Verhalten realisiert werden.

% https://wiki.linuxfoundation.org/realtime/documentation/technical_basics/preemption_models bzw kernel/Kconfig.preempt

\subsubsection{preemptRT%
        \label{sec:2-preemptRT}}
% https://wiki.linuxfoundation.org/realtime/documentation/technical_details/start
% https://wiki.linuxfoundation.org/realtime/documentation/technical_basics/start

Das dem PREEMPT RT Kernel zugrunde liegende Prinzip lässt sich in einer einfachen
Regel ausdrücken: Nur Code, welcher absolut nicht-präemtible sein darf, ist es
gestattet nicht-präemtible zu sein.
Das erklärte Ziel des PREEMPT\_RT Patches ist es folglich, die Menge des nicht-präemtiblen
Codes im Linux-Kernel auf das absolut notwendige Minimum zu reduzieren.

Dies wird durch Verwendung folgender Mechanismen erreicht:

\begin{itemize}
  \item Hochauflösende Timer
  \item Sleeping Spinlocks
  \item Threaded Interrupt Handlers
  \item rt\_mutex
  \item RCU
\end{itemize}


\subsubsection{posix%
        \label{sec:2-posix}}
Ist posix hier wirklich relevant? Debian bzw.~Raspbian sind weitgehend posix
kompatibel, aber wird es hier genutzt? -> JA, open62541 nutzt pthread.h
piControl nutzt kthread.h, und semaphore.h

\subsection{OPC-UA und open62541%
     \label{sec:2-opc}}

\subsubsection{OPC UA%
        \label{sec:2-opcua}}
Open Platform Communications (OPC) ist eine Familie von Standards zur herstellerunabhängigen
Kommunikation von Maschinen (M2M) in der Automatisierungstechnik. Die sog.~OPC Task Force, zu deren
Mitgliedern verschiedene große Firmen der Automatisierungsindustrie gehören, veröffentlichte
die OPC Specification Version 1.0 im August 1996.
Motiviert ist dieser offene Standard durch die Erkenntniss, dass die Anpassung der
zahlreichen Herstellerstandards an individuelle Infrastrukturen und Anlagen einen
großen Mehraufwand verursachen.
Die Wikipedia beschreibt das Anwendungsgebiet für OPC wie folgt:

\glqq{}OPC wird dort eingesetzt, wo Sensoren, Regler und Steuerungen verschiedener Hersteller
ein gemeinsames Netzwerk bilden. Ohne OPC benötigten zwei Geräte zum Datenaustausch
genaue Kenntnis über die Kommunikationsmöglichkeiten des Gegenübers. Erweiterungen
und Austausch gestalten sich entsprechend schwierig. Mit OPC genügt es, für jedes
Gerät genau einmal einen OPC-konformen Treiber zu schreiben. Idealerweise wird
dieser bereits vom Hersteller zur Verfügung gestellt. Ein OPC-Treiber lässt sich
ohne großen Anpassungsaufwand in beliebig große Steuer- und Überwachungssysteme
integrieren.

OPC unterteilt sich in verschiedene Unterstandards, die für den jeweiligen Anwendungsfall
unabhängig voneinander implementiert werden können. OPC lässt sich damit verwenden
für Echtzeitdaten (Überwachung), Datenarchivierung, Alarm-Meldungen und neuerdings
auch direkt zur Steuerung (Befehlsübermittlung).\grqq{}

OPC basiert in der ursprünglichen Spezifikation auf Microsofts DCOM-Spezifikation.
DCOM macht Funktionen und Objekte einer Anwendung anderen Anwendungen im Netzwerk
zugänglich. Der OPC-Standard definiert entsprechende DCOM-Objekte um mit anderen
OPC-Anwendungen Daten austauschen zu können. Die Verwendung von DCOM bindet Anwender
an Betriebssysteme von Microsoft. Die ursprüngliche OPC Spezifikation wird durch die
Entwicklung von OPC Unified Architecture (OPC UA) abgelöst.
OPC UA setzt auf einem eigenen Kommunikationionsstack auf, die Verwendung von DCOM
und damit die Bindung an Microsoft wurden aufgelöst.

Die OPC-UA-Architektur ist eine Service-orientierte Architektur (SOA), deren Struktur
aus mehreren Schichten besteht.

% Wikipedia
Das OPC-Informationsmodell ist nicht mehr nur eine Hierarchie aus Ordnern, Items
und Properties. Es ist ein sogenanntes Full-Mesh-Network aus Nodes, mit dem neben
den Nutzdaten eines Nodes auch Meta- und Diagnoseinformationen repräsentiert werden.
Ein Node ähnelt einem Objekt aus der objektorientierten Programmierung. Ein Node
kann Attribute besitzen, die gelesen werden können (Data Access (DA), Historical
Data Access (HDA)). Es ist möglich Methoden zu definieren und aufzurufen.
Eine Methode besitzt Aufrufargumente und Rückgabewerte. Sie wird durch ein Command
aufgerufen. Weiterhin werden Events unterstützt, die versendet werden können
(AE (Alarms \& Events), DA DataChange), um bestimmte Informationen zwischen Geräten
auszutauschen. Ein Event besitzt unter anderem einen Empfangszeitpunkt, eine Nachricht
und einen Schweregrad. Die o. g. Nodes werden sowohl für die Nutzdaten als auch
alle anderen Arten von Metadaten verwendet. Der damit modellierte OPC-Adressraum
beinhaltet nun auch ein Typmodell, mit dem sämtliche Datentypen spezifiziert werden.

% https://de.wikipedia.org/wiki/Open_Platform_Communications
% https://de.wikipedia.org/wiki/OPC_Unified_Architecture
% https://opcfoundation.org/developer-tools/specifications-unified-architecture
% Von Gerhard Gappmeier - ascolab GmbH, CC BY-SA 3.0, https://de.wikipedia.org/w/index.php?curid=1892069
\subsubsection{open62541%
        \label{sec:2-open62541}}
open62541 ist eine offene und freie Implementierung von OPC UA. Die in C geschriebene
Bibliothek stellt eine beständig zunehmende Anzahl der im OPC UA Standard definierten
Funktionen bereit. Sie kann sowohl zur Erstellung von OPC-Servern als auch -Clients
genutzt werden. Ergänzend zu der unter der Mozilla Public License v2.0 lizensierten
Bibliothek stellt das open62541 Projekt auch Beispielprogramme unter einer CC0 Lizenz
zur Verfügung.

Die Bibliothek eignet sich auch für die Entwicklung auf eingebetteten Systemen und
Microcontrollern. Je nach Umfang der gewünschten Funktionen und des OPC Informationsmodells
beträgt die Größe einer Server-Binary weniger als 100kb. %evtl. kürzen?

\todo{Nodes erklären! Evtl.~oben!}

Folgende Auswahl an Eigenschaften und Funktionen zeichnet die in dieser Arbeit verwendete
Version 0.3 von open62541 aus:
\begin{itemize}
  \item Kommunikationionsstack
  \begin{itemize}
      \item OPC UA Binär-Protokoll (HTTP oder SOAP werden gegenwärtig nicht unterstützt)
      \item Austauschbare Netzwerk-Schicht, welche die Verwendung eigener Netzwerk-APIs
      erlaubt.
      \item Verschlüsselte Kommunikationion
      \item Asynchrone Dienst-Anfragen im Client
  \end{itemize}
  \item Informationsmodell
  \begin{itemize}
    \item Unterstützung aller OPC UA Node-Typen, inkl.~Methoden
    \item Hinzufügen und Entfernen von Nodes und Referenzen zur Laufzeit.
    \item Vererbung und Instanziierung von Objekt- und Variablentypen
    \item Zugriffskontrolle auch für einzelne Nodes
  \end{itemize}
  \item Subscriptions
  \begin{itemize}
    \item Erlaubt die Überwachung (subscriptions / monitoreditems)
    \item Sehr geringer Ressourcenbedarf pro überwachtem Wert
  \end{itemize}
  \item Code-Generierung auf XML-Basis
  \begin{itemize}
    \item Erlaubt die Erstellung von Datentypen
    \item Erlaubt die Generierung des serverseitigen Informationsmodells
  \end{itemize}
\end{itemize}

% https://open62541.org/doc/0.3/


Mozilla Public License
CC0 Lizenz für Beispiele und Plugins

% https://open62541.org/doc/open62541-current.pdf
% https://open62541.org/

% % % Imports nur für Referenzenauflösung während des Schreibens! Vorm Kompilieren auskommentieren!
% \bibliography{0_hauptdatei}
% \input{1_einleitung}
% \input{2_grundlagen}
% \input{3_konzeption}
% \input{4_implementierung}
% \input{5_tests}
% \input{6_zusammenfassung}
% \input{anhang}
% % Ende Imports

\section{Systemkonzept%
  \label{sec:3-konzeption}}
Auf Basis der in Abschnitt \ref{sec:2-grundlagen} vorgestellten Möglichkeiten folgt nun die Ausarbeitung eines Konzepts.
In den folgenden Abschnitten soll näher auf zwei zentrale Aspekte eingegangen werden: Abschnitt~\ref{sec:3-anbindung} stellt Möglichkeiten zum Zugriff auf Variablen bzw.\,Werte im Prozessabbild des Revolution Pi vor; in Abschnitt~\ref{sec:3-integration} wird ein Konzept zur Bereitstellung dieser Variablen auf einem OPC-Server vorgestellt.

\subsection{Anbindung der IO an den OPC-Server%
     \label{sec:3-anbindung}}

Eine Webanwendung mit Bezeichnung PiCtory dient zur Konfiguration der I/O- und virtuellen Module des RevolutionPi. Die Konfiguration liegt im JSON-Format in der Datei \lstinline{/etc/revpi/config.rsc}. Der piControl-Treiber liest diese Datei beim Start. 
Der folgende Auszug aus der Manpage des piControl-Kernelmoduls beschreibt die von diesem zum Lesen und Schreiben einzelner Bits des Prozessabbildes bereitgestellten Funktionen~\citep[vgl.]{web-revpi-manpage}. Sie ist an dieser Stelle weitgehend ungekürzt zitiert, da sie die nutzbare Schnittstelle sehr kompakt beschreibt.

\begin{lstlisting}[breakindent=0pt, numbers=none, caption={Auszug aus der Revolution Pi Programmers Manual\label{lst:4-manpage}}]
KB_FIND_VARIABLE SPIVariable *argp
Find a variable in the process image by its name. A pointer to a structure of type SPIVariable must be passed as argument. [...]
The struct SPIVariable [...] is defined as 
typedef struct SPIVariableStr
{
    char strVarName[32]; // Variable name
    uint16_t i16uAddress; // Address of the byte in the process image
    uint8_t i8uBit; // 0-7 bit position, >= 8 whole byte
    uint16_t i16uLength; // length of the variable in bits.
    // Possible values are 1, 8, 16 and 32
} SPIVariable;

Set and get values of the process image
KB_GET_VALUE SPIValue *argp
[...]
KB_SET_VALUE SPIValue *argp
Write one bit or one byte to the process image [...].  This call is more efficient than the usual calls of seek and write because only one function call is necessary. If more than on application are writing bits in one output byte, this call is the only safe way to set a bit without overwriting the other bits because this call is doing a read-modify-write-cycle. 

The struct SPIValue used by this ioctl is defined as
typedef struct SPIValueStr
{
    uint16_t i16uAddress; // Address of the byte in the process image
    uint8_t i8uBit; // 0-7 bit position, >= 8 whole byte
    uint8_t i8uValue; // Value: 0/1 for bit access, whole byte otherwise
} SPIValue;
\end{lstlisting} 

Die oben beschriebenden Funtkionen \lstinline{KB_FIND_VARIABLE}, \lstinline{KB_GET_VALUE} und \lstinline{KB_SET_VALUE} ermöglichen einen einfachen und (lt.\,Manpage) effizienten Zugriff auf einzelne Bits des Prozessabbildes und damit auch auf die IO des RevolutionPi.
Der Zugriff des OPC-Servers auf das Prozessabbild soll daher mittels dieser Funktionen realisiert werden.
\lstinline{KB_FIND_VARIABLE} kann genutzt werden, um Adressen von Variablen im Prozessabbild mittels ihres Namens aufzulösen.
\lstinline{KB_GET_VALUE} und \lstinline{KB_SET_VALUE} ermöglichen den Zugriff auf die Werte dieser Variablen.


\subsection{Integration des OPC-Servers in das System%
     \label{sec:3-integration}}

open62541 bietet drei Möglichkeiten zum Abgleich von Variablen mit dem Prozessabbild~\citep[vgl.][Tutorials - Connecting a Variable with a Physical Process]{web-open62541}:
\begin{itemize}
    \item Manuelles oder zyklisches Aktualisieren
    \item Variable Value Callback
    \item Variable Datasource
\end{itemize}

Die zyklische Aktualisierung eines oder mehrerer Werte nimmt, abhängig von der Zykluszeit, viele Systemressourcen in Anspruch. Value Callbacks ermöglichen es, einen Variablenwert effizienter mit einer Ressource wie etwa einem Prozessabbild zu synchronisieren. An die Variable wird ein Callback angehängt, welches vor jedem Lesen und nach jedem Schreibvorgang ausgeführt wird.
Der Wert der Variablen wird weiterhin im Variablenknoten auf dem OPC-Server gespeichert, der Abgleich mit der verknüpften Ressource erfolgt durch die Callback-Methoden.

Sogenannte Datenquellen gehen noch einen Schritt weiter. Der Server leitet jede Lese- und Schreibanforderung direkt an eine Callback-Funktion weiter. Beim Lesen liefert der Rückruf eine Kopie des aktuellen Wertes. Die Datenquelle muss intern ein eigenes Speichermanagement implementieren.

Der Zugriff auf die Werte des Prozessabbildes erfolgt, wie in Abschnitt~\ref{sec:3-anbindung} beschrieben, über von piControl bereitgestellte Methoden. Um die durch open62541 gepflegte OPC-Datenstruktur und das durch piControl verwaltete Prozessabbild möglichst effektiv verknüpfen zu können, soll diese Interaktion mittels Datenquellen und den zugehörigen Callbacks implementiert werden.
% % % Imports nur für Referenzenauflösung während des Schreibens! Vorm Kompilieren auskommentieren!
% \bibliography{0_hauptdatei}
% \input{1_einleitung}
% \input{2_grundlagen}
% \input{3_konzeption}
% \input{4_implementierung}
% \input{5_tests}
% \input{6_zusammenfassung}
% \input{anhang}
% % Ende Imports

\section{Implementierung%
  \label{sec:4-implementierung}}
Das folgende Kapitel stellt in Auszügen die Implementierung des OPC-Servers sowie die Anbindung an die IO-Module
der SPS dar. Der Schwerpunkt liegt hierbei auf der Funktionsweise des piControl-Treibers und dessen Integration in das Projekt. Abschnitt~\ref{sec:4-picontrol} erklärt die zum Schreibens eines Bits verwendeten Funktionsaufrufe.
Zuvor soll jedoch in Abschnitt~\ref{sec:4-open62541} der Teil des OPC-Servers vorgestellt werden, welcher auf besagten Treiber zugreift. 

\subsection{Implementierung des OPC-Servers%
     \label{sec:4-open62541}}
Wie im vorangegangenen Abschnitt~\ref{sec:3-integration} begründet, soll die Verknüpfung zwischen dem Prozessabbild der SPS und den auf dem OPC-Server bereitgestellten Werten über sog.\,Datenquellen erfolgen. Hierzu ist zunächst eine Callback-Methode zu implementieren, welche bei einem Lese- oder Schreibzugriff auf eine Variable aufgerufen wird. Die Verknüpfung zwischen Callback-Methode und Variable muss manuell erfolgen.

\begin{lstlisting}[language={c},firstnumber=237,caption={Auszug der Methode \lstinline{linkDataSourceVariable} in \lstinline{variables.c}\label{lst:4-linkDataSourceVariable}}]
extern UA_StatusCode
 linkDataSourceVariable(UA_Server *server, UA_NodeId nodeId) {
     bool readonly = false;
     UA_DataSource dataSourceVariable;
     UA_StatusCode rc; |>\setcounter{lstnumber}{254}<|

     dataSourceVariable.read = readDataSourceVariable;
     if (!readonly)
        dataSourceVariable.write = writeDataSourceVariable;
     else
        dataSourceVariable.write = writeReadonlyDataSourceVariable;

     return UA_Server_setVariableNode_dataSource(server, nodeId, dataSourceVariable);
 }
\end{lstlisting}

\begin{figure}[h]
    \centering
    \includegraphics[width=0.42\textwidth]{doc/img/OPC_RevPiDO.pdf}
    \caption{Auszug des verwendeten Nodesets, hier Digitalausgang 1 des Versuchsaufbaus
      \label{fig:opc-do}}
\end{figure}

Die in Listing~\ref{lst:4-linkDataSourceVariable} abgebildete Methode \lstinline{linkDataSourceVariable()} erzeugt ein Struct vom Typ \lstinline{UA_DataSource}. In diesem werden dem Lesen und Schreiben einer OPC-Variablen entsprechende Callback-Methoden zugewiesen. Die Verknüpfung einer OPC-Variable, genauer ihrer NodeId, mit der zuvor definierten Datenquelle erfolgt über die von open62541 bereitgestellte Methode \lstinline{UA_Server_setVariableNode_dataSource()}. Vor dem Lesen und nach dem Schreiben dieser Variable werden von nun an die entsprechenden Callbacks aufgerufen.
     
\begin{lstlisting}[language={c},firstnumber=168,caption={Auszug des Callbacks \lstinline{writeDataSourceVariable} in \lstinline{variables.c}\label{lst:4-writeDataSourceVariable}}]  
extern UA_StatusCode
 writeDataSourceVariable(UA_Server *server,
            const UA_NodeId *sessionId, void *sessionContext,
            const UA_NodeId *nodeId, void *nodeContext,
            const UA_NumericRange *range, const UA_DataValue *dataValue) {

    UA_StatusCode retval  = UA_STATUSCODE_GOOD;
    UA_NodeId *nameNodeId = UA_malloc(sizeof(UA_NodeId));
    UA_QualifiedName nameQN = UA_QUALIFIEDNAME(1, "Name");
    UA_Variant nameVar;
    UA_Boolean bit;

    retval |= findSiblingByBrowsename(server, nodeId, &nameQN, nameNodeId);
    retval |= UA_Server_readValue(server, *nameNodeId, &nameVar);
    retval |= UA_Boolean_copy(dataValue->value.data, &bit);

    |>\tikzmarkin[set border color=martinired]{writeIO}<|PI_writeSingleIO(String_fromUA_String(nameVar.data), &bit, false);                                                 |>\tikzmarkend{writeIO}<|

    free(nameNodeId);
    return retval;
 }
\end{lstlisting}

Listing~\ref{lst:4-writeDataSourceVariable} zeigt die Callback-Methode, welche nach dem Schreiben einer Variablen auf dem OPC-Server aufgerufen wird.
Dieser Methode wird neben der NodeId der mit ihr verknüpften Variablen auch der Wert dieser in Form eines Zeigers auf ein Struct vom Typ \lstinline{UA_DataValue} übergeben.

Die Gestaltung des hier verwendeten Nodesets sieht vor, dass in einer OPC-Variablen \lstinline{"Name"} der Bezeichner des zu schreibenden Digitalausgangs hinterlegt ist, siehe Abbildung~\ref{fig:opc-do}. Dies erlaubt eine Rekonfiguration der Ein- und Ausgänge der SPS ohne Änderungen im Programmcode des OPC-Servers vornehmen zu müssen.
Es ist daher erforderlich, nach jedem Schreiben einer mit einem Digitalausgang verknüpften Variablen, hier \lstinline{"Value"}, dessen Bezeichner \lstinline{"Name"} abzufragen. 
Dies geschieht in den Zeilen 180 und 181.
Anschließend wird dieser Bezeichner sowie der zu schreibende Wert der Methode \lstinline{PI_writeSingleIO()} übergeben, welche wiederum die Interaktion mit piControl übernimmt (vgl. Abschnitt \ref{sec:4-picontrol}).
 
\subsection{Integration von piControl%
     \label{sec:4-picontrol}}
In Abschnitt~\ref{sec:2-io} wurde die Anbindung der IO-Module des Revolution Pi sowie die Funktionsweise von piControl aus Anwendersicht beschrieben. Die verfügbare Literatur beschränkt sich auch auf lediglich diese Sicht; eine weiterführende Dokumentation für Entwickler gibt es, neben der in Abschnitt~\ref{sec:3-anbindung} vorgestellten Manpage, nicht. 
In diesem Abschnitt soll daher der Quellcode von piControl sowie dessen Verwendung im Projekt genauer betrachtet werden.
Hierzu wird exemplarisch die in Abschnitt~\ref{sec:4-open62541} eingeführte Methode \lstinline{PI_writeSingleIO()} untersucht.
Diese Methode ermöglicht das Setzen eines einzelnen Bits im Prozessabbild der SPS, und damit das Schalten eines digitalen Ausgangs auf einem IO-Modul.
Die äquivalente Methode \lstinline{int piControlGetBitValue(SPIValue *pSpiValue)} zum Lesen eines Bits bzw. Eingangs funktioniert analog und soll daher an dieser Stelle nicht dediziert erörtert werden.

\begin{lstlisting}[language={c},firstnumber=97,
                   caption={Setzen eines phsikalischen, digitalen Ausgangs in \lstinline{revpi.c}
                   \label{lst:4-PI_writeSingleIO}}]
extern void PI_writeSingleIO(char *pszVariableName, bool *bit, bool verbose)
{
	int rc;
	SPIVariable sPiVariable;
	SPIValue sPIValue;

	strncpy(sPiVariable.strVarName, pszVariableName, sizeof(sPiVariable.strVarName));
	rc = piControlGetVariableInfo(&sPiVariable);
	if (rc < 0) {
		printf("Cannot find variable '%s'\n", pszVariableName);
		return;
	}

		sPIValue.i16uAddress = sPiVariable.i16uAddress;
		sPIValue.i8uBit = sPiVariable.i8uBit;
		sPIValue.i8uValue = *bit;
		rc = |>\tikzmarkin[set border color=martinired]{setBitValue}<|piControlSetBitValue(&sPIValue)|>\tikzmarkend{setBitValue}<|;
		if (rc < 0)
			printf("Set bit error %s\n", getWriteError(rc));
		else if (verbose)
			printf("Set bit %d on byte at offset %d. Value %d\n", sPIValue.i8uBit, sPIValue.i16uAddress,
			       sPIValue.i8uValue);
}
\end{lstlisting}

Der Programmcode in Listing~\ref{lst:4-PI_writeSingleIO} ist Teil des implementierten OPC-Servers. In diesem wird auf zwei Funktionen des piControl-Treibers zugegriffen. 
Beiden Methoden wird als Argument ein Zeiger auf ein Struct vom Typ \lstinline{SPIValue} übergeben. Der im Struct abgelegte Name wird mittels \lstinline{piControlGetVariableInfo(&sPIValue)} zu einer Adresse im Prozessabbild aufgelöst. Diese wird in \lstinline{sPIValue.i16uAdress} gespeichert. Der Wert der Variablen wird anschließend mittels \lstinline{piControlSetBitValue(&sPIValue)} an dieser Adresse in das Prozessabbild geschrieben.

\begin{lstlisting}[language={c},firstnumber=309,caption={Methode \lstinline{piControlSetBitValue} in \lstinline{piControlIf.c}\label{lst:4-piControlSetBitValue}}]
int |>\tikzmarkin[set border color=martiniblue]{setBitValueFcn}<|piControlSetBitValue(SPIValue *pSpiValue)|>\tikzmarkend{setBitValueFcn}<|
{
    piControlOpen();

    if (PiControlHandle_g < 0)
	    return -ENODEV;

    pSpiValue->i16uAddress += pSpiValue->i8uBit / 8;
    pSpiValue->i8uBit %= 8;

    if (|>\tikzmarkin[set border color=martinired]{ioctl}<|ioctl(PiControlHandle_g, KB_SET_VALUE, pSpiValue)|>\tikzmarkend{ioctl}<| < 0)
	    return errno;

    return 0;
}
\end{lstlisting}

Die in Listing~\ref{lst:4-piControlSetBitValue} dargestellte Methode \lstinline{piControlSetBitValue} ist lediglich eine Hüllfunktion (häufig auch als Wrapper-Funktion bezeichnet) für einen Aufruf des \lstinline{ioctl} Kernel-Moduls.
Folgende Parameter werden übergeben:
\lstinline{PiControlHandle_g} ist die Referenz auf die Geräte-Datei des piControl-Treibers. \lstinline{KB_SET_VALUE} ist das ioctl-Kommando zum Schreiben eines Bits in das Prozessabbild. Der Zeiger \lstinline{pSpiValue} verweist auf ein Struct des bereits vorgestellten Typs \lstinline{SPIValue}.

\begin{lstlisting}[language={c},firstnumber=80,caption={Methode \lstinline{piControlOpen} in \lstinline{piControlIf.c}\label{lst:4-piControlOpen}}]
void piControlOpen(void)
{
    /* open handle if needed */
    if (PiControlHandle_g < 0)
    {
	    |>\tikzmarkin[set border color=martiniblue]{PiControlHandle}<|PiControlHandle_g = open(PICONTROL_DEVICE, O_RDWR)|>\tikzmarkend{PiControlHandle}<|;
    }
}
\end{lstlisting}

Die in Listing~\ref{lst:4-piControlOpen} dargestellte Methode öffnet, sofern nicht bereits geschehen, die Geräte-Datei. Das Macro \lstinline{PICONTROL_DEVICE} verweist hierbei auf \lstinline{/dev/piControl0}.

\begin{lstlisting}[language={c},firstnumber=721,caption={Methode \lstinline{piControlIoctl} in \lstinline{piControlMain.c}\label{lst:4-piControlIoctl}}]
static long |>\tikzmarkin[set border color=martiniblue, below offset=0.9em]{piControlIoctl}<|piControlIoctl(struct file *file, unsigned int prg_nr, 
                           unsigned long usr_addr)                                      |>\tikzmarkend{piControlIoctl}<|
{
  int status = -EFAULT;
  tpiControlInst *priv;
  int timeout = 10000;	// ms

  if (prg_nr != KB_CONFIG_SEND && prg_nr != KB_CONFIG_START && !isRunning()) {
  	return -EAGAIN;
  }

  priv = (tpiControlInst *) file->private_data;

  if (prg_nr != KB_GET_LAST_MESSAGE) {
  	// clear old message
  	priv->pcErrorMessage[0] = 0;
  }

  switch (prg_nr) {|>\setcounter{lstnumber}{864}<|

    case |>\tikzmarkin[set border color=martiniblue]{KB_SET_VALUE}<|KB_SET_VALUE:|>\tikzmarkend{KB_SET_VALUE}<|
  		{
  			SPIValue *pValue = (SPIValue *) usr_addr;

  			if (!isRunning())
  				return -EFAULT;

  			if (pValue->i16uAddress >= KB_PI_LEN) {
  				status = -EFAULT;
  			} else {
  				INT8U i8uValue_l;
  				my_rt_mutex_lock(&piDev_g.lockPI);
  				i8uValue_l = piDev_g.ai8uPI[pValue->i16uAddress];

  				if (pValue->i8uBit >= 8) {
  					i8uValue_l = pValue->i8uValue;
  				} else {
  					if (pValue->i8uValue)
  						i8uValue_l |= (1 << pValue->i8uBit);
  					else
  						i8uValue_l &= ~(1 << pValue->i8uBit);
  				}

  				|>\tikzmarkin[set border color=martinired]{i8uValue}<|piDev_g.ai8uPI[pValue->i16uAddress] = i8uValue_l;|>\tikzmarkend{i8uValue}<|
  				rt_mutex_unlock(&piDev_g.lockPI);

  #ifdef VERBOSE
  				pr_info("piControlIoctl Addr=%u, bit=%u: %02x %02x\n", pValue->i16uAddress, pValue->i8uBit, pValue->i8uValue, i8uValue_l);
  #endif

  				status = 0;
  			}
  		}
  		break; |>\setcounter{lstnumber}{1314}<|

    default:
      pr_err("Invalid Ioctl");
      return (-EINVAL);
      break;

    }

    return status;
  }
\end{lstlisting}

Listing~\ref{lst:4-piControlIoctl} zeigt in Auszügen die ioctl-Methode des piControl Kernel-Treibers. Diese bekommt folgende Argumente übergeben: \lstinline{struct file *file} enthält den Verweis auf die Geräte-Datei, hier \lstinline{/dev/piControl0}. Der Wert von \lstinline{unsigned int prg_nr} beschreibt die Anfrage an den Treiber, in diesem Fall \lstinline{KB_SET_VALUE}. Das Argument \lstinline{unsigned long usr_addr} enthält einen typ-agnostischen Pointer. Dieser verweist auf einen Speicherbereich, in welchem die zur Bearbeitung der Anfrage notwendigen Daten abgelegt sind. Hier können auch vom Treiber empfangene Daten dem Anwendungsprogramm bereitgestellt werden. 

Die switch-case-Anweisung führt die über das Argument \lstinline{prg_nr} spezifizierte Aktion aus. Hier betrachten wir \lstinline{KB_SET_VALUE}:
Zunächst wird in Zeile 868 der übergebene Zeiger \lstinline{usr_addr} mittels explizitem Typecast zu einem Zeiger des Typs \lstinline{SPIValue *} konvertiert. Da dieser auf Daten im Userspace verweist, ist beim Zugriff durch den Kernel-Treiber besondere Vorsicht geboten.
In Zeile 877 wird mittels Mutex das Prozessabbild \lstinline{piDev_g} für den Zugriff durch andere Threads oder Prozesse gesperrt.
\lstinline{my_rt_mutex_lock} verweist hierbei auf die Funktion \lstinline{rt_mutex_lock} aus \lstinline{linux/sched.h}\footnote{Offenbar wurde hier auch eine alternative Implementierung vorgesehen, siehe revpi\_common.h}

In Zeile 889 wird das Byte \lstinline{i8uValue_l}, welches den zu schreibenden Wert enthält in das Prozessabbild übertragen. Anschließend wird die Mutex auf \lstinline{piDev_g} wieder entsperrt.
\newpage

\begin{lstlisting}[language={c},firstnumber=62,caption={Auszug des Struct \lstinline{spiControlDev} in \lstinline{piControlMain.h}\label{lst:4-spiControlDev}}]
|>\tikzmarkin[set border color=martiniblue]{spiControlDev}<|typedef struct spiControlDev|>\tikzmarkend{spiControlDev}<| {
	// device driver stuff
	int init_step;
	enum revpi_machine machine_type;
	void *machine;
	struct cdev cdev;	// Char device structure
	struct device *dev;
	struct thermal_zone_device *thermal_zone;

	|>\tikzmarkin[set border color=martiniblue]{processImage}<|// process image stuff
	INT8U ai8uPI[KB_PI_LEN];
	INT8U ai8uPIDefault|>\tikzmarkin[set border color=martinired]{KB_PI_LEN_0}<|[KB_PI_LEN]|>\tikzmarkend{KB_PI_LEN_0}<|;
	struct rt_mutex lockPI;        |>\tikzmarkend{processImage}<|
	bool stopIO;
	piDevices *devs; |>\setcounter{lstnumber}{94}<|
} tpiControlDev;
\end{lstlisting}

Das Prozessabbild ist als Byte-Array der Länge \lstinline{KB_PI_LEN} in Listing~\ref{lst:4-spiControlDev} definiert. Konfigurationsparameter wie \lstinline{KB_PI_LEN} oder die Zykluszeit für den Datenaustausch zwischen SPS und IO-Modulen sind im folgenden Listing~\ref{lst:4-process} definiert.

\begin{lstlisting}[language={c},firstnumber=119,caption={Konfigurationsparameter des Prozessabbildes in project.h\label{lst:4-process}}]
#define INTERVAL_PI_GATE (5*1000*1000)  // 5 ms piGateCommunication |>\setcounter{lstnumber}{128}<|

#define INTERVAL_IO_COM (5*1000*1000)  // 5 ms piIoComm |>\setcounter{lstnumber}{132}<|

#define KB_PD_LEN       512
|>\tikzmarkin[set border color=martiniblue]{KB_PI_LEN_1}<|#define KB_PI_LEN       4096|>\tikzmarkend{KB_PI_LEN_1}<|
\end{lstlisting}

Das zu setzende Bit wurde zu diesem Zeitpunkt erfolgreich in das Prozessabbild der SPS geschrieben.
Es stellt sich die Frage, wie dieses nun an das IO-Modul kommuniziert wird.
Die Kommunikation mit allen angebundenen Modulen ist ebenfalls Aufgabe des piControl-Treibers.

\begin{lstlisting}[language={c},firstnumber=256,caption={Auszug der Methode \lstinline{piIoThread} in \lstinline{revpi_core.c}\label{lst:4-piIoThread}}]
static int piIoThread(void *data)
{
	//TODO int value = 0;
	ktime_t time;
	ktime_t now;
	s64 tDiff;

	hrtimer_init(&piCore_g.ioTimer, CLOCK_MONOTONIC, HRTIMER_MODE_ABS);
	piCore_g.ioTimer.function = piIoTimer;

	pr_info("piIO thread started\n");

	now = hrtimer_cb_get_time(&piCore_g.ioTimer);

	PiBridgeMaster_Reset();

	while (!kthread_should_stop()) {
		if (|>\tikzmarkin[set border color=martinired]{PiBridgeMaster}<|PiBridgeMaster_Run()|>\tikzmarkend{PiBridgeMaster}<| < 0)
			break;
	}

	RevPiDevice_finish();

	pr_info("piIO exit\n");
	return 0;
}
\end{lstlisting}

Der Kernel-Thread \lstinline{piIoThread} ist verantwortlich für den zyklischen Datenaustausch mit den IO-Modulen. In diesem wird fortlaufend die Methode \lstinline{PiBridgeMaster_Run()} aufgerufen, siehe Listing~\ref{lst:4-piIoThread}.

\begin{lstlisting}[language={c},firstnumber=262,caption={Auszug der Methode \lstinline{PiBridgeMaster_Run(void)} in \lstinline{RevPiDevice.c}\label{lst:4-PiBridgeMaster_Run}}]
int PiBridgeMaster_Run(void)
{
	static kbUT_Timer tTimeoutTimer_s;
	static kbUT_Timer tConfigTimeoutTimer_s;
	static int error_cnt;
	static INT8U last_led;
	static unsigned long last_update;
	int ret = 0;
	int i;

	my_rt_mutex_lock(&piCore_g.lockBridgeState);
	if (piCore_g.eBridgeState != piBridgeStop) {
		switch (eRunStatus_s) { |>\setcounter{lstnumber}{514}<|
		    case enPiBridgeMasterStatus_EndOfConfig:|>\setcounter{lstnumber}{621}<|
		    if (|>\tikzmarkin[set border color=martinired]{RevPiDevice}<|RevPiDevice_run()|>\tikzmarkend{RevPiDevice}<|) {
				// an error occured, check error limits |>\setcounter{lstnumber}{641}<|
			} else {
				ret = 1;
			}
			piCore_g.image.drv.i16uRS485ErrorCnt = RevPiDevice_getErrCnt();
			break;
\end{lstlisting}

Die in Listing~\ref{lst:4-PiBridgeMaster_Run} dargestellte Methode ist eine sog. State-Machine. Ist die Konfiguration der IO-Module erfolgreich abgeschlossen, so führt sie bei Aufruf lediglich die Methode \lstinline{RevPiDevice_run()} aus.

\begin{lstlisting}[language={c},firstnumber=140,caption={Auszug der Methode \lstinline{RevPiDevice_run(void)} in \lstinline{RevPiDevice.c}\label{lst:4-RevPiDevice_run}}]
int RevPiDevice_run(void)
{
	INT8U i8uDevice = 0;
	INT32U r;
	int retval = 0;

	RevPiDevices_s.i16uErrorCnt = 0;

	for (i8uDevice = 0; i8uDevice < RevPiDevice_getDevCnt(); i8uDevice++) {
		if (RevPiDevice_getDev(i8uDevice)->i8uActive) {
			switch (RevPiDevice_getDev(i8uDevice)->sId.i16uModulType) {
			case KUNBUS_FW_DESCR_TYP_PI_DIO_14:
			case KUNBUS_FW_DESCR_TYP_PI_DI_16:
			case KUNBUS_FW_DESCR_TYP_PI_DO_16:
				r = |>\tikzmarkin[set border color=martinired]{sendCyclicTelegram}<|piDIOComm_sendCyclicTelegram(i8uDevice)|>\tikzmarkend{sendCyclicTelegram}\setcounter{lstnumber}{166} <|;

				break; |>\setcounter{lstnumber}{216}<|
			}
		}
	} |>\setcounter{lstnumber}{227}<|
	return retval;
}
\end{lstlisting}

Diese iteriert wie in Listing~\ref{lst:4-RevPiDevice_run} abgebildete durch alle gegenwärtig in der SPS konfigurierten Module. Ist das aktuelle Modul als aktiv markiert, so wird anhand eines sog. Firmware-Descriptors entschieden, welche Methode für die Ansteuerung des Moduls aufzurufen ist.

\begin{lstlisting}[language={c},firstnumber=161,caption={Auszug der Methode \lstinline{piDIOComm_sendCyclicTelegram} in \lstinline{piDIOComm.c}\label{lst:4-sendCyclicTelegram}}]
INT32U piDIOComm_sendCyclicTelegram(INT8U i8uDevice_p)
{
	INT32U i32uRv_l = 0;
	SIOGeneric sRequest_l;
	SIOGeneric sResponse_l;
	INT8U len_l, data_out[18], i, p, data_in[70];
	INT8U i8uAddress;
	int ret; |>\setcounter{lstnumber}{239}<|
	
    |>\tikzmarkin[set border color=martinired]{piIoComm}<|ret = piIoComm_send((INT8U *) & sRequest_l, IOPROTOCOL_HEADER_LENGTH + len_l + 1);  |>\tikzmarkend{piIoComm}\setcounter{lstnumber}{298}<|
}
\end{lstlisting}

Im Falle des hier verwendeten DO-Moduls wird die in Listing~\ref{lst:4-sendCyclicTelegram} abgebildete Methode \lstinline{piDIOComm_sendCyclicTelegram()} aufgerufen. Dieser wird ein Zeiger auf das zu schreibende Gerät übergeben. 
Zunächst wird das Prozessabbild mittels eines proprietären, jedoch im Quellcode offen nachvollziehbaren Protokolls in ein \lstinline{sRequest_l} genanntes Byte-Array umgewandelt. Dieser Schritt ist in Listing~\ref{lst:4-sendCyclicTelegram} nicht abgebildet. Anschließend wird \lstinline{piIoComm_send()} ein Zeiger auf die so generierte Schreib-Anfrage übergeben.

\begin{lstlisting}[language={c},firstnumber=220,caption={Auszug der Methode \lstinline{piIOComm_send} in \lstinline{piIOComm.c}\label{lst:4-piIOComm_send}}]
int piIoComm_send(INT8U * buf_p, INT16U i16uLen_p)
{
	ssize_t write_l = 0;
	INT16U i16uSent_l = 0;|>\setcounter{lstnumber}{249}<|

	while (i16uSent_l < i16uLen_p) {
		write_l = vfs_write(piIoComm_fd_m, buf_p + i16uSent_l, i16uLen_p - i16uSent_l, &piIoComm_fd_m->f_pos);
		if (write_l < 0) {
			pr_info_serial("write error %d\n", (int)write_l);
			return -1;
		} 
		i16uSent_l += write_l;|>\setcounter{lstnumber}{263}<|
	}
	clear();
	vfs_fsync(piIoComm_fd_m, 1);
	return 0;
}
\end{lstlisting}

Listing~\ref{lst:4-piIOComm_send} zeigt die Implementierung von \lstinline{piIoComm_send()}. Diese Methode ist für das Schreiben der oben generierten Anfrage auf die seriellen Schnittstelle verantwortlich. Realisiert wird dies mittels der Methode \lstinline{vfs_write()}. Diese ist in \lstinline{<linux/fs.h>} definiert. Sie ermöglicht das Schreiben einer Datei im Userspace aus dem Kernel heraus. Geschrieben wird hier die Datei mit dem Deskriptor \lstinline{piIoComm_fd_m}.
Da die Funktion \lstinline{vfs_write()} durch andere Kernel-Tasks unterbrochen werden kann, ist nicht gewährleistet, dass die gesamte Anfrage mit nur einem Aufruf geschrieben wird. Die oben abgebildete while-Schleife stellt das vollständige Senden der Anfrage sicher.

\begin{lstlisting}[language={c},firstnumber=157,caption={Auszug der Methode \lstinline{piIOComm_open_serial} in \lstinline{piIOComm.c}\label{lst:4-piIOComm_open_serial}}]
int piIoComm_open_serial(void)
{   |>\setcounter{lstnumber}{167}<|
	struct file *fd;	/* Filedeskriptor */
	struct termios newtio;	/* Schnittstellenoptionen */

	|>\tikzmarkin[set border color=martiniblue]{fd}<|/* Port oeffnen - read/write, kein "controlling tty", 
	    Status von DCD ignorieren */
	fd = filp_open(|>\tikzmarkin[set border color=martinired]{tty}<|REV_PI_TTY_DEVICE|>\tikzmarkend{tty}<|, O_RDWR | O_NOCTTY, 0); |>\setcounter{lstnumber}{208}<|
	
	piIoComm_fd_m = fd;                                                      |>\tikzmarkend{fd}\setcounter{lstnumber}{217}<|

	return 0;
}
\end{lstlisting}

Der zum Schreiben auf die serielle Schnittstelle verwendete Datei-Deskriptor wird von der in Listing~\ref{lst:4-piIOComm_open_serial} abgebildeten Methode \lstinline{piIoComm_open_serial()} generiert. 

\begin{lstlisting}[language={c},firstnumber=45,caption={Definition der seriellen Schnittstelle in \lstinline{piIOComm.h}\label{lst:4-REV_PI_TTY_DEVICE}}]
#define REV_PI_TTY_DEVICE	"/dev/ttyAMA0"
\end{lstlisting}

Das in Listing~\ref{lst:4-REV_PI_TTY_DEVICE} definierte Macro verweist auf eine der seriellen Schnittstellen des RaspberryPi.
Die Implementierung des zugehörigen Schnittstellentreibers soll hier nicht weiter untersucht werden. Somit ist an dieser Stelle die Kette vom Setzen einer Variablen auf dem OPC-Server bis hin zur Aktualisierung des Prozessabbilds der IO-Module geschlossen.

% \begin{lstlisting}[language={c},firstnumber={226},caption={Setzen der Scheduler-Priorität auf SCHED\_FIFO in 
% revpi\_common.c\label{lst:2-sched_priority}}]
% param.sched_priority = ktprio->prio;
% ret = sched_setscheduler(child, SCHED_FIFO, &param);
% \end{lstlisting}
% % % Imports nur für Referenzenauflösung während des Schreibens! Vorm Kompilieren auskommentieren!
% \bibliography{0_hauptdatei}
% \input{1_einleitung}
% \input{2_grundlagen}
% \input{3_konzeption}
% \input{4_implementierung}
% \input{5_tests}
% \input{6_zusammenfassung}
% % Ende Imports

\section{Test des OPC-Servers im Gesamtsystem%
  \label{sec:5-tests}}

% % % Imports nur für Referenzenauflösung während des schreibens! Vorm Kompilieren auskommentieren!
% \bibliography{0_hauptdatei}
% \input{1_einleitung}
% \input{2_grundlagen}
% \input{3_konzeption}
% \input{4_implementierung}
% \input{5_tests}
% \input{6_zusammenfassung}
% % Ende Imports

\section{Zusammenfassung und Ausblick%
  \label{sec:6-fazit}}
Der folgende Abschnitt~\ref{sec:6-zusammenfassung} fasst die gewonnenen Erkenntnisse und den Stand der Implementierung zusammen.
Den Abschluss dieser Arbeit bildet der Ausblick in Abschnitt~\ref{sec:6-ausblick}.

\subsection{Zusammenfassung%
     \label{sec:6-zusammenfassung}}

\subsection{Ausblick%
     \label{sec:6-ausblick}}

% \input{anhang}
% % Ende Imports

\section{Systemkonzept%
  \label{sec:3-konzeption}}
Auf Basis der in Abschnitt \ref{sec:2-grundlagen} vorgestellten Möglichkeiten folgt nun die Ausarbeitung eines Konzepts.
In den folgenden Abschnitten soll näher auf zwei zentrale Aspekte eingegangen werden: Abschnitt~\ref{sec:3-anbindung} stellt Möglichkeiten zum Zugriff auf Variablen bzw.\,Werte im Prozessabbild des Revolution Pi vor; in Abschnitt~\ref{sec:3-integration} wird ein Konzept zur Bereitstellung dieser Variablen auf einem OPC-Server vorgestellt.

\subsection{Anbindung der IO an den OPC-Server%
     \label{sec:3-anbindung}}

Eine Webanwendung mit Bezeichnung PiCtory dient zur Konfiguration der I/O- und virtuellen Module des RevolutionPi. Die Konfiguration liegt im JSON-Format in der Datei \lstinline{/etc/revpi/config.rsc}. Der piControl-Treiber liest diese Datei beim Start. 
Der folgende Auszug aus der Manpage des piControl-Kernelmoduls beschreibt die von diesem zum Lesen und Schreiben einzelner Bits des Prozessabbildes bereitgestellten Funktionen~\citep[vgl.]{web-revpi-manpage}. Sie ist an dieser Stelle weitgehend ungekürzt zitiert, da sie die nutzbare Schnittstelle sehr kompakt beschreibt.

\begin{lstlisting}[breakindent=0pt, numbers=none, caption={Auszug aus der Revolution Pi Programmers Manual\label{lst:4-manpage}}]
KB_FIND_VARIABLE SPIVariable *argp
Find a variable in the process image by its name. A pointer to a structure of type SPIVariable must be passed as argument. [...]
The struct SPIVariable [...] is defined as 
typedef struct SPIVariableStr
{
    char strVarName[32]; // Variable name
    uint16_t i16uAddress; // Address of the byte in the process image
    uint8_t i8uBit; // 0-7 bit position, >= 8 whole byte
    uint16_t i16uLength; // length of the variable in bits.
    // Possible values are 1, 8, 16 and 32
} SPIVariable;

Set and get values of the process image
KB_GET_VALUE SPIValue *argp
[...]
KB_SET_VALUE SPIValue *argp
Write one bit or one byte to the process image [...].  This call is more efficient than the usual calls of seek and write because only one function call is necessary. If more than on application are writing bits in one output byte, this call is the only safe way to set a bit without overwriting the other bits because this call is doing a read-modify-write-cycle. 

The struct SPIValue used by this ioctl is defined as
typedef struct SPIValueStr
{
    uint16_t i16uAddress; // Address of the byte in the process image
    uint8_t i8uBit; // 0-7 bit position, >= 8 whole byte
    uint8_t i8uValue; // Value: 0/1 for bit access, whole byte otherwise
} SPIValue;
\end{lstlisting} 

Die oben beschriebenden Funtkionen \lstinline{KB_FIND_VARIABLE}, \lstinline{KB_GET_VALUE} und \lstinline{KB_SET_VALUE} ermöglichen einen einfachen und (lt.\,Manpage) effizienten Zugriff auf einzelne Bits des Prozessabbildes und damit auch auf die IO des RevolutionPi.
Der Zugriff des OPC-Servers auf das Prozessabbild soll daher mittels dieser Funktionen realisiert werden.
\lstinline{KB_FIND_VARIABLE} kann genutzt werden, um Adressen von Variablen im Prozessabbild mittels ihres Namens aufzulösen.
\lstinline{KB_GET_VALUE} und \lstinline{KB_SET_VALUE} ermöglichen den Zugriff auf die Werte dieser Variablen.


\subsection{Integration des OPC-Servers in das System%
     \label{sec:3-integration}}

open62541 bietet drei Möglichkeiten zum Abgleich von Variablen mit dem Prozessabbild~\citep[vgl.][Tutorials - Connecting a Variable with a Physical Process]{web-open62541}:
\begin{itemize}
    \item Manuelles oder zyklisches Aktualisieren
    \item Variable Value Callback
    \item Variable Datasource
\end{itemize}

Die zyklische Aktualisierung eines oder mehrerer Werte nimmt, abhängig von der Zykluszeit, viele Systemressourcen in Anspruch. Value Callbacks ermöglichen es, einen Variablenwert effizienter mit einer Ressource wie etwa einem Prozessabbild zu synchronisieren. An die Variable wird ein Callback angehängt, welches vor jedem Lesen und nach jedem Schreibvorgang ausgeführt wird.
Der Wert der Variablen wird weiterhin im Variablenknoten auf dem OPC-Server gespeichert, der Abgleich mit der verknüpften Ressource erfolgt durch die Callback-Methoden.

Sogenannte Datenquellen gehen noch einen Schritt weiter. Der Server leitet jede Lese- und Schreibanforderung direkt an eine Callback-Funktion weiter. Beim Lesen liefert der Rückruf eine Kopie des aktuellen Wertes. Die Datenquelle muss intern ein eigenes Speichermanagement implementieren.

Der Zugriff auf die Werte des Prozessabbildes erfolgt, wie in Abschnitt~\ref{sec:3-anbindung} beschrieben, über von piControl bereitgestellte Methoden. Um die durch open62541 gepflegte OPC-Datenstruktur und das durch piControl verwaltete Prozessabbild möglichst effektiv verknüpfen zu können, soll diese Interaktion mittels Datenquellen und den zugehörigen Callbacks implementiert werden.
% % % Imports nur für Referenzenauflösung während des Schreibens! Vorm Kompilieren auskommentieren!
% \bibliography{0_hauptdatei}
% % Mit \section{...} eröffnen wir einen neuen Abschnitt.
% Der Befehl setzt nicht nur den Text in einer größeren,
% fetten Schrift, sondern sorgt außerdem dafür, daß er im
% Inhaltsverzeichnis erscheint.
%
% Mit \label{...} erzeugen wir einen Bezeichner, mit dessen Hilfe
% wir später auf die Nummer des Abschnitts verweisen können (nämlich
% mit~\ref{...}).
%
% Das Kommentarzeichen hinter „Übersicht“ dient dazu, ein
% Leerzeichen zwischen „Übersicht“ und dem \label-Befehl
% zu vermeiden, das andernfalls sichtbar würde – z.B. im
% Inhaltsverzeichnis.
%

% % Imports nur für Referenzenauflösung während des Schreibens! Vorm Kompilieren auskommentieren!
% \bibliography{0_hauptdatei}
% \input{1_einleitung}
%\input{2_grundlagen}
%\input{3_konzeption}
%\input{4_implementierung}
%\input{5_tests}
%\input{6_zusammenfassung}
% % Ende Imports

\section{Einleitung und Motivation%
  \label{sec:1-einleitung}}
Ziel dieses Projektes ist die Integration eines OPC-Servers mit einer auf Linux
basierenden speicherprogrammierbaren Steuerung (SPS). Angeschlossen an diese SPS
ist jeweils ein digitales Ein-/\,bzw.~Ausgabemodul. Die von diesen bereitgestellten
Ein-/\, bzw.~Ausgänge (IO) sollen in der Datenstruktur des OPC-Servers abgebildet
und über diesen für OPC-Clients les-/\,und schreibar sein. Weiterhin sollen einige
Funktionen zur Überwachung und Steuerung der an die SPS angeschlossenen Aktoren
und Sensoren direkt im OPC-Server implementiert werden.
Hiermit stellt dieses Projekt eine der Grundlagen für ein übergeordnetes Projekt,
die cloudbasierte Steuerung eines miniaturisierten Produktions-Systems, dar.

Der hier verwendete OPC-Server ist Teil des sog. open62541 Projekts. Er ist in C
geschrieben und implementiert bereits einen großen Teil der im OPC-UA-Standard
spezifizierten Funktionen.
Als SPS findet ein Revolution Pi 3 der Firma Kunbus Verwendung. Dieser integriert
ein sog. Compute Module der Raspberry Pi Foundation in ein industrietaugliches
Gehäuse und erlaubt die Erweiterung mittels IO- oder Gateway-Modulen. Über diese
erfolgt die Kommunikation mit weiteren Komponenten der Automatisierungstechnik.

Motiviert ist dieses Projekt durch die Beobachtung, dass die Verbreitung offener
Standards sowie freier Software auch in der Automatisierungstechnik zunimmt.
Linux ist ein freies Betriebssystem, OPC-UA ein offen zugänglicher, aktiv gepflegter
und weit verbreiteter Standard. Der Raspberry Pi findet sowohl bei Hobby-Anwendern als
auch in den Bereichen Forschung und Entwicklung sowie bei industriellen Anwendern
Verwendung. Dieses Projekt stellt somit eine für unterschiedliche Anwender interessante
Entwicklung dar.

Im Anschluss an diese einleitende Übersicht im Abschnitt~\ref{sec:1-einleitung} folgt
die Darstellung der wichtigsten Grundlagen in Abschnitt~\ref{sec:2-grundlagen}.
Aufbauend auf diesen Grundlagen folgt die konzeptuelle Ausarbeitung im Abschnitt~\ref{sec:3-konzeption}.
Die Umsetzung wird im Abschnitt~\ref{sec:4-implementierung} erläutert.
Die Leistungsfähigkeit der Implementierung wird in Abschnitt~\ref{sec:5-tests} untersucht.
Eine Zusammenfassung und ein Ausblick schließen die Arbeit in
Abschnitt~\ref{sec:6-fazit} ab. Eventuell noch benötigte Anhänge
finden sich in den Anhängen [...] bis [...].

% % % Imports nur für Referenzenauflösung während des Schreibens! Vorm Kompilieren auskommentieren!
% \bibliography{0_hauptdatei}
% \input{1_einleitung}
% \input{2_grundlagen}
% \input{3_konzeption}
% \input{4_implementierung}
% \input{5_tests}
% \input{6_zusammenfassung}
% % Ende Imports

\section{Grundlagen%
  \label{sec:2-grundlagen}}

\subsection{Speicherprogrammierbare-Steuerung und Linux -- Revolution Pi%
     \label{sec:2-sps}}

\subsubsection{Kunbus RevolutionPi%
        \label{sec:2-revpi}}
Der RevolutionPi 3 ist eine speicherprogrammierbare Steuerung (SPS) des Herstellers
Kunbus GmbH. Kern dieser SPS ist das von der Raspberry Pi Foundation entwickelte
und vertriebene Raspberry Pi Compute Module 3. Dieses integriert ein Broadcom BCM2837
System-on-Chip (SoC) mit vier 1,2GHz Prozessorkernen, 1GB RAM, 4GB eMMC Anwendungsspeicher
und sonstige Peripherie in ein Modul im DDR2-SODIMM Formfaktor. Diese Spezifikationen
sind weitgehend identisch zu denen des ausgesprochen populären Raspberry Pi 3.
Der Revolution Pi profitiert daher von dem gleichen großen Angebot an Software
und Unterstützung wie der Raspberry Pi, ergänzt dessen Hardware jedoch um eine 24V
Spannungsversorgung, die Möglichkeit der Erweiterung durch mehrere industrietaugliche
Ein-/ Ausgabemodule und Gateways sowie ein Gehäuse zur Montage auf einer DIN-Schiene.
\begin{itemize}
  \item{Prozessor: BCM2837}
  \item{Taktfrequenz 1,2 GHz}
  \item{Anzahl Prozessorkerne: 4}
  \item{Arbeitsspeicher: 1 GByte}
  \item{eMMC Flash Speicher: 4 GByte}
  \item{Betriebssystem: Angepasstes Raspbian mit RT-Patch}
  \item{RTC mit 24h Pufferung über wartungsfreien Kondensator}
  \item{Treiber / API: Treiber schreibt zyklisch Prozessdaten in ein Prozessabbild, Zugriff auf Prozessabbild über Linux-Filesystem als API zu Fremdsoftware.}
  \item{Kommunikationsanschlüsse: 2 x USB 2.0 A (je 500 mA belastbar), 1 x Micro-USB, HDMI, Ethernet (RJ45) 10/100 Mbit/s}
  \item{Stromversorgung: min. 10,7 V, max. 28,8 V, maximal 10 Watt}
  \item{Zulässige Umgebungstemperatur: -40 bis +55 C}
  \item{Gehäuseabmessungen: (HxBxL) 96 mm x 22,5 mm x 110,5 mm (ohne gesteckte Stecker)}
  \item{ESD Schutz: 4 kV / 8 kV gemäß EN61131-2 und IEC 61000-6-2}
  \item{Surge / Burst Prüfungen: gemäß EN61131-2 und IEC 61000-6-2 eingekoppelt auf Versorgungsspannung, Ethernet und IO-Leitungen}
  \item{EMI Prüfungen: gemäß EN61131-2 und IEC 61000-6-2}
\end{itemize}

Kunbus bietet eine Auswahl an IO- und Gateway-Modulen zur Erweiterung des Revolution Pi an.
Gateways dienen der Kommunikation mit Systemen oder Komponenten der Automatisierungstechnik
über Protokolle wie PROFIBUS oder EtherCAT. IO-Module erlauben die Überwachung
und Steuerung von digitalen oder analogen Ein- und Ausgängen.

\subsubsection{Zugriff auf IO-Module%
        \label{sec:2-io}}
Der Zugriff auf die Ein- und Ausgänge der IO-Module erfolgt über ein Prozessabbild
und einen hierfür von Kunbus bereitgestellten Treiber, genannt piControl. Dieser
aktualisiert das Prozessabbild zyklisch. Die angestrebte Zykluszeit beträgt 5ms,
kann jedoch je nach Anzahl der angeschlossenen Module auch größer sein. Kunbus
garantiert bei drei IO-Modulen und zwei Gateway-Modulen eine Zykluszeit von 10 ms.
Jedes der IO-Module stellt ein eigenständiges eingebettetes System dar. Es verfügt
über einen Microcontroller, welcher die IOs bereitstellt und über einen RS485-Bus
mit dem Revolution Pi kommuniziert.
% https://revolution.kunbus.de/io-modul/

Lizenz: GPL
% https://github.com/RevolutionPi/piControl

\begin{lstlisting}[language={c},firstnumber={226},caption={Setzen der Scheduler-Priorität auf SCHED\_FIFO in revpi\_common.c\label{lst:2-sched_priority}}]
param.sched_priority = ktprio->prio;
ret = sched_setscheduler(child, SCHED_FIFO,
       &param);
\end{lstlisting}


\subsection{Echtzeit und Multithreading unter Linux -- preemptRT und posix%
     \label{sec:2-echtzeit}}


 Der Linux-Kernel verfügt über mehrere unterschiedliche Preemtion-Modelle:

\begin{itemize}
  \item No Forced Preemption (server):
  Ausgelegt auf maximal möglichen Durchsatz, lediglich Interrupts und
  System-Call-Returns bewirken Präemption.

  \item Voluntary Kernel Preemption (Desktop):
  Neben den implizit bevorrechtigten Interrupts und System-Call-Returns gibt es
  in diesem Modell weitere Abschnitte des Kernels in welchen Preämption explizit
  gestattet ist.

  \item Preemptible Kernel (Low-Latency Desktop):
  In diesem Modell ist der gesamte Kernel, mit Ausnahme sog.~kritischer Abschnitte
  präemptible. Nach jedem kritischen Abschnitt gibt es einen impliziten Präemptions-Punkt.

  \item Preemptible Kernel (Basic RT):
  Dieses Modell ist dem zuvor genannten sehr ähnlich, hier sind jedoch alle Interrupt-Handler
  als eigenständige Threads ausgeführt.

  \item Fully Preemptible Kernel (RT):
  Wie auch bei den beiden zuvor genannten Modellen ist hier der gesamte Kernel
  präemtible, die Anzahl und Dauer der nicht-präemtiblen kritischen Abschnitte
  ist auf ein notwendiges Minimum beschränkt. Alle Interrupt-Handler sind als
  eigenständige Threads ausgeführt, Spinlocks durch Sleeping-Spinlocks und Mutexe
  durch sog.~RT-Mutexe ersetzt.

\end{itemize}
\todo{Spinlocks und Mutexe sowie die RT-Varianten dieser erklären!}

Lediglich mit dem vollständig präemtiblen Kernel kann Echtzeit-Verhalten realisiert werden.

% https://wiki.linuxfoundation.org/realtime/documentation/technical_basics/preemption_models bzw kernel/Kconfig.preempt

\subsubsection{preemptRT%
        \label{sec:2-preemptRT}}
% https://wiki.linuxfoundation.org/realtime/documentation/technical_details/start
% https://wiki.linuxfoundation.org/realtime/documentation/technical_basics/start

Das dem PREEMPT RT Kernel zugrunde liegende Prinzip lässt sich in einer einfachen
Regel ausdrücken: Nur Code, welcher absolut nicht-präemtible sein darf, ist es
gestattet nicht-präemtible zu sein.
Das erklärte Ziel des PREEMPT\_RT Patches ist es folglich, die Menge des nicht-präemtiblen
Codes im Linux-Kernel auf das absolut notwendige Minimum zu reduzieren.

Dies wird durch Verwendung folgender Mechanismen erreicht:

\begin{itemize}
  \item Hochauflösende Timer
  \item Sleeping Spinlocks
  \item Threaded Interrupt Handlers
  \item rt\_mutex
  \item RCU
\end{itemize}


\subsubsection{posix%
        \label{sec:2-posix}}
Ist posix hier wirklich relevant? Debian bzw.~Raspbian sind weitgehend posix
kompatibel, aber wird es hier genutzt? -> JA, open62541 nutzt pthread.h
piControl nutzt kthread.h, und semaphore.h

\subsection{OPC-UA und open62541%
     \label{sec:2-opc}}

\subsubsection{OPC UA%
        \label{sec:2-opcua}}
Open Platform Communications (OPC) ist eine Familie von Standards zur herstellerunabhängigen
Kommunikation von Maschinen (M2M) in der Automatisierungstechnik. Die sog.~OPC Task Force, zu deren
Mitgliedern verschiedene große Firmen der Automatisierungsindustrie gehören, veröffentlichte
die OPC Specification Version 1.0 im August 1996.
Motiviert ist dieser offene Standard durch die Erkenntniss, dass die Anpassung der
zahlreichen Herstellerstandards an individuelle Infrastrukturen und Anlagen einen
großen Mehraufwand verursachen.
Die Wikipedia beschreibt das Anwendungsgebiet für OPC wie folgt:

\glqq{}OPC wird dort eingesetzt, wo Sensoren, Regler und Steuerungen verschiedener Hersteller
ein gemeinsames Netzwerk bilden. Ohne OPC benötigten zwei Geräte zum Datenaustausch
genaue Kenntnis über die Kommunikationsmöglichkeiten des Gegenübers. Erweiterungen
und Austausch gestalten sich entsprechend schwierig. Mit OPC genügt es, für jedes
Gerät genau einmal einen OPC-konformen Treiber zu schreiben. Idealerweise wird
dieser bereits vom Hersteller zur Verfügung gestellt. Ein OPC-Treiber lässt sich
ohne großen Anpassungsaufwand in beliebig große Steuer- und Überwachungssysteme
integrieren.

OPC unterteilt sich in verschiedene Unterstandards, die für den jeweiligen Anwendungsfall
unabhängig voneinander implementiert werden können. OPC lässt sich damit verwenden
für Echtzeitdaten (Überwachung), Datenarchivierung, Alarm-Meldungen und neuerdings
auch direkt zur Steuerung (Befehlsübermittlung).\grqq{}

OPC basiert in der ursprünglichen Spezifikation auf Microsofts DCOM-Spezifikation.
DCOM macht Funktionen und Objekte einer Anwendung anderen Anwendungen im Netzwerk
zugänglich. Der OPC-Standard definiert entsprechende DCOM-Objekte um mit anderen
OPC-Anwendungen Daten austauschen zu können. Die Verwendung von DCOM bindet Anwender
an Betriebssysteme von Microsoft. Die ursprüngliche OPC Spezifikation wird durch die
Entwicklung von OPC Unified Architecture (OPC UA) abgelöst.
OPC UA setzt auf einem eigenen Kommunikationionsstack auf, die Verwendung von DCOM
und damit die Bindung an Microsoft wurden aufgelöst.

Die OPC-UA-Architektur ist eine Service-orientierte Architektur (SOA), deren Struktur
aus mehreren Schichten besteht.

% Wikipedia
Das OPC-Informationsmodell ist nicht mehr nur eine Hierarchie aus Ordnern, Items
und Properties. Es ist ein sogenanntes Full-Mesh-Network aus Nodes, mit dem neben
den Nutzdaten eines Nodes auch Meta- und Diagnoseinformationen repräsentiert werden.
Ein Node ähnelt einem Objekt aus der objektorientierten Programmierung. Ein Node
kann Attribute besitzen, die gelesen werden können (Data Access (DA), Historical
Data Access (HDA)). Es ist möglich Methoden zu definieren und aufzurufen.
Eine Methode besitzt Aufrufargumente und Rückgabewerte. Sie wird durch ein Command
aufgerufen. Weiterhin werden Events unterstützt, die versendet werden können
(AE (Alarms \& Events), DA DataChange), um bestimmte Informationen zwischen Geräten
auszutauschen. Ein Event besitzt unter anderem einen Empfangszeitpunkt, eine Nachricht
und einen Schweregrad. Die o. g. Nodes werden sowohl für die Nutzdaten als auch
alle anderen Arten von Metadaten verwendet. Der damit modellierte OPC-Adressraum
beinhaltet nun auch ein Typmodell, mit dem sämtliche Datentypen spezifiziert werden.

% https://de.wikipedia.org/wiki/Open_Platform_Communications
% https://de.wikipedia.org/wiki/OPC_Unified_Architecture
% https://opcfoundation.org/developer-tools/specifications-unified-architecture
% Von Gerhard Gappmeier - ascolab GmbH, CC BY-SA 3.0, https://de.wikipedia.org/w/index.php?curid=1892069
\subsubsection{open62541%
        \label{sec:2-open62541}}
open62541 ist eine offene und freie Implementierung von OPC UA. Die in C geschriebene
Bibliothek stellt eine beständig zunehmende Anzahl der im OPC UA Standard definierten
Funktionen bereit. Sie kann sowohl zur Erstellung von OPC-Servern als auch -Clients
genutzt werden. Ergänzend zu der unter der Mozilla Public License v2.0 lizensierten
Bibliothek stellt das open62541 Projekt auch Beispielprogramme unter einer CC0 Lizenz
zur Verfügung.

Die Bibliothek eignet sich auch für die Entwicklung auf eingebetteten Systemen und
Microcontrollern. Je nach Umfang der gewünschten Funktionen und des OPC Informationsmodells
beträgt die Größe einer Server-Binary weniger als 100kb. %evtl. kürzen?

\todo{Nodes erklären! Evtl.~oben!}

Folgende Auswahl an Eigenschaften und Funktionen zeichnet die in dieser Arbeit verwendete
Version 0.3 von open62541 aus:
\begin{itemize}
  \item Kommunikationionsstack
  \begin{itemize}
      \item OPC UA Binär-Protokoll (HTTP oder SOAP werden gegenwärtig nicht unterstützt)
      \item Austauschbare Netzwerk-Schicht, welche die Verwendung eigener Netzwerk-APIs
      erlaubt.
      \item Verschlüsselte Kommunikationion
      \item Asynchrone Dienst-Anfragen im Client
  \end{itemize}
  \item Informationsmodell
  \begin{itemize}
    \item Unterstützung aller OPC UA Node-Typen, inkl.~Methoden
    \item Hinzufügen und Entfernen von Nodes und Referenzen zur Laufzeit.
    \item Vererbung und Instanziierung von Objekt- und Variablentypen
    \item Zugriffskontrolle auch für einzelne Nodes
  \end{itemize}
  \item Subscriptions
  \begin{itemize}
    \item Erlaubt die Überwachung (subscriptions / monitoreditems)
    \item Sehr geringer Ressourcenbedarf pro überwachtem Wert
  \end{itemize}
  \item Code-Generierung auf XML-Basis
  \begin{itemize}
    \item Erlaubt die Erstellung von Datentypen
    \item Erlaubt die Generierung des serverseitigen Informationsmodells
  \end{itemize}
\end{itemize}

% https://open62541.org/doc/0.3/


Mozilla Public License
CC0 Lizenz für Beispiele und Plugins

% https://open62541.org/doc/open62541-current.pdf
% https://open62541.org/

% % % Imports nur für Referenzenauflösung während des Schreibens! Vorm Kompilieren auskommentieren!
% \bibliography{0_hauptdatei}
% \input{1_einleitung}
% \input{2_grundlagen}
% \input{3_konzeption}
% \input{4_implementierung}
% \input{5_tests}
% \input{6_zusammenfassung}
% \input{anhang}
% % Ende Imports

\section{Systemkonzept%
  \label{sec:3-konzeption}}
Auf Basis der in Abschnitt \ref{sec:2-grundlagen} vorgestellten Möglichkeiten folgt nun die Ausarbeitung eines Konzepts.
In den folgenden Abschnitten soll näher auf zwei zentrale Aspekte eingegangen werden: Abschnitt~\ref{sec:3-anbindung} stellt Möglichkeiten zum Zugriff auf Variablen bzw.\,Werte im Prozessabbild des Revolution Pi vor; in Abschnitt~\ref{sec:3-integration} wird ein Konzept zur Bereitstellung dieser Variablen auf einem OPC-Server vorgestellt.

\subsection{Anbindung der IO an den OPC-Server%
     \label{sec:3-anbindung}}

Eine Webanwendung mit Bezeichnung PiCtory dient zur Konfiguration der I/O- und virtuellen Module des RevolutionPi. Die Konfiguration liegt im JSON-Format in der Datei \lstinline{/etc/revpi/config.rsc}. Der piControl-Treiber liest diese Datei beim Start. 
Der folgende Auszug aus der Manpage des piControl-Kernelmoduls beschreibt die von diesem zum Lesen und Schreiben einzelner Bits des Prozessabbildes bereitgestellten Funktionen~\citep[vgl.]{web-revpi-manpage}. Sie ist an dieser Stelle weitgehend ungekürzt zitiert, da sie die nutzbare Schnittstelle sehr kompakt beschreibt.

\begin{lstlisting}[breakindent=0pt, numbers=none, caption={Auszug aus der Revolution Pi Programmers Manual\label{lst:4-manpage}}]
KB_FIND_VARIABLE SPIVariable *argp
Find a variable in the process image by its name. A pointer to a structure of type SPIVariable must be passed as argument. [...]
The struct SPIVariable [...] is defined as 
typedef struct SPIVariableStr
{
    char strVarName[32]; // Variable name
    uint16_t i16uAddress; // Address of the byte in the process image
    uint8_t i8uBit; // 0-7 bit position, >= 8 whole byte
    uint16_t i16uLength; // length of the variable in bits.
    // Possible values are 1, 8, 16 and 32
} SPIVariable;

Set and get values of the process image
KB_GET_VALUE SPIValue *argp
[...]
KB_SET_VALUE SPIValue *argp
Write one bit or one byte to the process image [...].  This call is more efficient than the usual calls of seek and write because only one function call is necessary. If more than on application are writing bits in one output byte, this call is the only safe way to set a bit without overwriting the other bits because this call is doing a read-modify-write-cycle. 

The struct SPIValue used by this ioctl is defined as
typedef struct SPIValueStr
{
    uint16_t i16uAddress; // Address of the byte in the process image
    uint8_t i8uBit; // 0-7 bit position, >= 8 whole byte
    uint8_t i8uValue; // Value: 0/1 for bit access, whole byte otherwise
} SPIValue;
\end{lstlisting} 

Die oben beschriebenden Funtkionen \lstinline{KB_FIND_VARIABLE}, \lstinline{KB_GET_VALUE} und \lstinline{KB_SET_VALUE} ermöglichen einen einfachen und (lt.\,Manpage) effizienten Zugriff auf einzelne Bits des Prozessabbildes und damit auch auf die IO des RevolutionPi.
Der Zugriff des OPC-Servers auf das Prozessabbild soll daher mittels dieser Funktionen realisiert werden.
\lstinline{KB_FIND_VARIABLE} kann genutzt werden, um Adressen von Variablen im Prozessabbild mittels ihres Namens aufzulösen.
\lstinline{KB_GET_VALUE} und \lstinline{KB_SET_VALUE} ermöglichen den Zugriff auf die Werte dieser Variablen.


\subsection{Integration des OPC-Servers in das System%
     \label{sec:3-integration}}

open62541 bietet drei Möglichkeiten zum Abgleich von Variablen mit dem Prozessabbild~\citep[vgl.][Tutorials - Connecting a Variable with a Physical Process]{web-open62541}:
\begin{itemize}
    \item Manuelles oder zyklisches Aktualisieren
    \item Variable Value Callback
    \item Variable Datasource
\end{itemize}

Die zyklische Aktualisierung eines oder mehrerer Werte nimmt, abhängig von der Zykluszeit, viele Systemressourcen in Anspruch. Value Callbacks ermöglichen es, einen Variablenwert effizienter mit einer Ressource wie etwa einem Prozessabbild zu synchronisieren. An die Variable wird ein Callback angehängt, welches vor jedem Lesen und nach jedem Schreibvorgang ausgeführt wird.
Der Wert der Variablen wird weiterhin im Variablenknoten auf dem OPC-Server gespeichert, der Abgleich mit der verknüpften Ressource erfolgt durch die Callback-Methoden.

Sogenannte Datenquellen gehen noch einen Schritt weiter. Der Server leitet jede Lese- und Schreibanforderung direkt an eine Callback-Funktion weiter. Beim Lesen liefert der Rückruf eine Kopie des aktuellen Wertes. Die Datenquelle muss intern ein eigenes Speichermanagement implementieren.

Der Zugriff auf die Werte des Prozessabbildes erfolgt, wie in Abschnitt~\ref{sec:3-anbindung} beschrieben, über von piControl bereitgestellte Methoden. Um die durch open62541 gepflegte OPC-Datenstruktur und das durch piControl verwaltete Prozessabbild möglichst effektiv verknüpfen zu können, soll diese Interaktion mittels Datenquellen und den zugehörigen Callbacks implementiert werden.
% % % Imports nur für Referenzenauflösung während des Schreibens! Vorm Kompilieren auskommentieren!
% \bibliography{0_hauptdatei}
% \input{1_einleitung}
% \input{2_grundlagen}
% \input{3_konzeption}
% \input{4_implementierung}
% \input{5_tests}
% \input{6_zusammenfassung}
% \input{anhang}
% % Ende Imports

\section{Implementierung%
  \label{sec:4-implementierung}}
Das folgende Kapitel stellt in Auszügen die Implementierung des OPC-Servers sowie die Anbindung an die IO-Module
der SPS dar. Der Schwerpunkt liegt hierbei auf der Funktionsweise des piControl-Treibers und dessen Integration in das Projekt. Abschnitt~\ref{sec:4-picontrol} erklärt die zum Schreibens eines Bits verwendeten Funktionsaufrufe.
Zuvor soll jedoch in Abschnitt~\ref{sec:4-open62541} der Teil des OPC-Servers vorgestellt werden, welcher auf besagten Treiber zugreift. 

\subsection{Implementierung des OPC-Servers%
     \label{sec:4-open62541}}
Wie im vorangegangenen Abschnitt~\ref{sec:3-integration} begründet, soll die Verknüpfung zwischen dem Prozessabbild der SPS und den auf dem OPC-Server bereitgestellten Werten über sog.\,Datenquellen erfolgen. Hierzu ist zunächst eine Callback-Methode zu implementieren, welche bei einem Lese- oder Schreibzugriff auf eine Variable aufgerufen wird. Die Verknüpfung zwischen Callback-Methode und Variable muss manuell erfolgen.

\begin{lstlisting}[language={c},firstnumber=237,caption={Auszug der Methode \lstinline{linkDataSourceVariable} in \lstinline{variables.c}\label{lst:4-linkDataSourceVariable}}]
extern UA_StatusCode
 linkDataSourceVariable(UA_Server *server, UA_NodeId nodeId) {
     bool readonly = false;
     UA_DataSource dataSourceVariable;
     UA_StatusCode rc; |>\setcounter{lstnumber}{254}<|

     dataSourceVariable.read = readDataSourceVariable;
     if (!readonly)
        dataSourceVariable.write = writeDataSourceVariable;
     else
        dataSourceVariable.write = writeReadonlyDataSourceVariable;

     return UA_Server_setVariableNode_dataSource(server, nodeId, dataSourceVariable);
 }
\end{lstlisting}

\begin{figure}[h]
    \centering
    \includegraphics[width=0.42\textwidth]{doc/img/OPC_RevPiDO.pdf}
    \caption{Auszug des verwendeten Nodesets, hier Digitalausgang 1 des Versuchsaufbaus
      \label{fig:opc-do}}
\end{figure}

Die in Listing~\ref{lst:4-linkDataSourceVariable} abgebildete Methode \lstinline{linkDataSourceVariable()} erzeugt ein Struct vom Typ \lstinline{UA_DataSource}. In diesem werden dem Lesen und Schreiben einer OPC-Variablen entsprechende Callback-Methoden zugewiesen. Die Verknüpfung einer OPC-Variable, genauer ihrer NodeId, mit der zuvor definierten Datenquelle erfolgt über die von open62541 bereitgestellte Methode \lstinline{UA_Server_setVariableNode_dataSource()}. Vor dem Lesen und nach dem Schreiben dieser Variable werden von nun an die entsprechenden Callbacks aufgerufen.
     
\begin{lstlisting}[language={c},firstnumber=168,caption={Auszug des Callbacks \lstinline{writeDataSourceVariable} in \lstinline{variables.c}\label{lst:4-writeDataSourceVariable}}]  
extern UA_StatusCode
 writeDataSourceVariable(UA_Server *server,
            const UA_NodeId *sessionId, void *sessionContext,
            const UA_NodeId *nodeId, void *nodeContext,
            const UA_NumericRange *range, const UA_DataValue *dataValue) {

    UA_StatusCode retval  = UA_STATUSCODE_GOOD;
    UA_NodeId *nameNodeId = UA_malloc(sizeof(UA_NodeId));
    UA_QualifiedName nameQN = UA_QUALIFIEDNAME(1, "Name");
    UA_Variant nameVar;
    UA_Boolean bit;

    retval |= findSiblingByBrowsename(server, nodeId, &nameQN, nameNodeId);
    retval |= UA_Server_readValue(server, *nameNodeId, &nameVar);
    retval |= UA_Boolean_copy(dataValue->value.data, &bit);

    |>\tikzmarkin[set border color=martinired]{writeIO}<|PI_writeSingleIO(String_fromUA_String(nameVar.data), &bit, false);                                                 |>\tikzmarkend{writeIO}<|

    free(nameNodeId);
    return retval;
 }
\end{lstlisting}

Listing~\ref{lst:4-writeDataSourceVariable} zeigt die Callback-Methode, welche nach dem Schreiben einer Variablen auf dem OPC-Server aufgerufen wird.
Dieser Methode wird neben der NodeId der mit ihr verknüpften Variablen auch der Wert dieser in Form eines Zeigers auf ein Struct vom Typ \lstinline{UA_DataValue} übergeben.

Die Gestaltung des hier verwendeten Nodesets sieht vor, dass in einer OPC-Variablen \lstinline{"Name"} der Bezeichner des zu schreibenden Digitalausgangs hinterlegt ist, siehe Abbildung~\ref{fig:opc-do}. Dies erlaubt eine Rekonfiguration der Ein- und Ausgänge der SPS ohne Änderungen im Programmcode des OPC-Servers vornehmen zu müssen.
Es ist daher erforderlich, nach jedem Schreiben einer mit einem Digitalausgang verknüpften Variablen, hier \lstinline{"Value"}, dessen Bezeichner \lstinline{"Name"} abzufragen. 
Dies geschieht in den Zeilen 180 und 181.
Anschließend wird dieser Bezeichner sowie der zu schreibende Wert der Methode \lstinline{PI_writeSingleIO()} übergeben, welche wiederum die Interaktion mit piControl übernimmt (vgl. Abschnitt \ref{sec:4-picontrol}).
 
\subsection{Integration von piControl%
     \label{sec:4-picontrol}}
In Abschnitt~\ref{sec:2-io} wurde die Anbindung der IO-Module des Revolution Pi sowie die Funktionsweise von piControl aus Anwendersicht beschrieben. Die verfügbare Literatur beschränkt sich auch auf lediglich diese Sicht; eine weiterführende Dokumentation für Entwickler gibt es, neben der in Abschnitt~\ref{sec:3-anbindung} vorgestellten Manpage, nicht. 
In diesem Abschnitt soll daher der Quellcode von piControl sowie dessen Verwendung im Projekt genauer betrachtet werden.
Hierzu wird exemplarisch die in Abschnitt~\ref{sec:4-open62541} eingeführte Methode \lstinline{PI_writeSingleIO()} untersucht.
Diese Methode ermöglicht das Setzen eines einzelnen Bits im Prozessabbild der SPS, und damit das Schalten eines digitalen Ausgangs auf einem IO-Modul.
Die äquivalente Methode \lstinline{int piControlGetBitValue(SPIValue *pSpiValue)} zum Lesen eines Bits bzw. Eingangs funktioniert analog und soll daher an dieser Stelle nicht dediziert erörtert werden.

\begin{lstlisting}[language={c},firstnumber=97,
                   caption={Setzen eines phsikalischen, digitalen Ausgangs in \lstinline{revpi.c}
                   \label{lst:4-PI_writeSingleIO}}]
extern void PI_writeSingleIO(char *pszVariableName, bool *bit, bool verbose)
{
	int rc;
	SPIVariable sPiVariable;
	SPIValue sPIValue;

	strncpy(sPiVariable.strVarName, pszVariableName, sizeof(sPiVariable.strVarName));
	rc = piControlGetVariableInfo(&sPiVariable);
	if (rc < 0) {
		printf("Cannot find variable '%s'\n", pszVariableName);
		return;
	}

		sPIValue.i16uAddress = sPiVariable.i16uAddress;
		sPIValue.i8uBit = sPiVariable.i8uBit;
		sPIValue.i8uValue = *bit;
		rc = |>\tikzmarkin[set border color=martinired]{setBitValue}<|piControlSetBitValue(&sPIValue)|>\tikzmarkend{setBitValue}<|;
		if (rc < 0)
			printf("Set bit error %s\n", getWriteError(rc));
		else if (verbose)
			printf("Set bit %d on byte at offset %d. Value %d\n", sPIValue.i8uBit, sPIValue.i16uAddress,
			       sPIValue.i8uValue);
}
\end{lstlisting}

Der Programmcode in Listing~\ref{lst:4-PI_writeSingleIO} ist Teil des implementierten OPC-Servers. In diesem wird auf zwei Funktionen des piControl-Treibers zugegriffen. 
Beiden Methoden wird als Argument ein Zeiger auf ein Struct vom Typ \lstinline{SPIValue} übergeben. Der im Struct abgelegte Name wird mittels \lstinline{piControlGetVariableInfo(&sPIValue)} zu einer Adresse im Prozessabbild aufgelöst. Diese wird in \lstinline{sPIValue.i16uAdress} gespeichert. Der Wert der Variablen wird anschließend mittels \lstinline{piControlSetBitValue(&sPIValue)} an dieser Adresse in das Prozessabbild geschrieben.

\begin{lstlisting}[language={c},firstnumber=309,caption={Methode \lstinline{piControlSetBitValue} in \lstinline{piControlIf.c}\label{lst:4-piControlSetBitValue}}]
int |>\tikzmarkin[set border color=martiniblue]{setBitValueFcn}<|piControlSetBitValue(SPIValue *pSpiValue)|>\tikzmarkend{setBitValueFcn}<|
{
    piControlOpen();

    if (PiControlHandle_g < 0)
	    return -ENODEV;

    pSpiValue->i16uAddress += pSpiValue->i8uBit / 8;
    pSpiValue->i8uBit %= 8;

    if (|>\tikzmarkin[set border color=martinired]{ioctl}<|ioctl(PiControlHandle_g, KB_SET_VALUE, pSpiValue)|>\tikzmarkend{ioctl}<| < 0)
	    return errno;

    return 0;
}
\end{lstlisting}

Die in Listing~\ref{lst:4-piControlSetBitValue} dargestellte Methode \lstinline{piControlSetBitValue} ist lediglich eine Hüllfunktion (häufig auch als Wrapper-Funktion bezeichnet) für einen Aufruf des \lstinline{ioctl} Kernel-Moduls.
Folgende Parameter werden übergeben:
\lstinline{PiControlHandle_g} ist die Referenz auf die Geräte-Datei des piControl-Treibers. \lstinline{KB_SET_VALUE} ist das ioctl-Kommando zum Schreiben eines Bits in das Prozessabbild. Der Zeiger \lstinline{pSpiValue} verweist auf ein Struct des bereits vorgestellten Typs \lstinline{SPIValue}.

\begin{lstlisting}[language={c},firstnumber=80,caption={Methode \lstinline{piControlOpen} in \lstinline{piControlIf.c}\label{lst:4-piControlOpen}}]
void piControlOpen(void)
{
    /* open handle if needed */
    if (PiControlHandle_g < 0)
    {
	    |>\tikzmarkin[set border color=martiniblue]{PiControlHandle}<|PiControlHandle_g = open(PICONTROL_DEVICE, O_RDWR)|>\tikzmarkend{PiControlHandle}<|;
    }
}
\end{lstlisting}

Die in Listing~\ref{lst:4-piControlOpen} dargestellte Methode öffnet, sofern nicht bereits geschehen, die Geräte-Datei. Das Macro \lstinline{PICONTROL_DEVICE} verweist hierbei auf \lstinline{/dev/piControl0}.

\begin{lstlisting}[language={c},firstnumber=721,caption={Methode \lstinline{piControlIoctl} in \lstinline{piControlMain.c}\label{lst:4-piControlIoctl}}]
static long |>\tikzmarkin[set border color=martiniblue, below offset=0.9em]{piControlIoctl}<|piControlIoctl(struct file *file, unsigned int prg_nr, 
                           unsigned long usr_addr)                                      |>\tikzmarkend{piControlIoctl}<|
{
  int status = -EFAULT;
  tpiControlInst *priv;
  int timeout = 10000;	// ms

  if (prg_nr != KB_CONFIG_SEND && prg_nr != KB_CONFIG_START && !isRunning()) {
  	return -EAGAIN;
  }

  priv = (tpiControlInst *) file->private_data;

  if (prg_nr != KB_GET_LAST_MESSAGE) {
  	// clear old message
  	priv->pcErrorMessage[0] = 0;
  }

  switch (prg_nr) {|>\setcounter{lstnumber}{864}<|

    case |>\tikzmarkin[set border color=martiniblue]{KB_SET_VALUE}<|KB_SET_VALUE:|>\tikzmarkend{KB_SET_VALUE}<|
  		{
  			SPIValue *pValue = (SPIValue *) usr_addr;

  			if (!isRunning())
  				return -EFAULT;

  			if (pValue->i16uAddress >= KB_PI_LEN) {
  				status = -EFAULT;
  			} else {
  				INT8U i8uValue_l;
  				my_rt_mutex_lock(&piDev_g.lockPI);
  				i8uValue_l = piDev_g.ai8uPI[pValue->i16uAddress];

  				if (pValue->i8uBit >= 8) {
  					i8uValue_l = pValue->i8uValue;
  				} else {
  					if (pValue->i8uValue)
  						i8uValue_l |= (1 << pValue->i8uBit);
  					else
  						i8uValue_l &= ~(1 << pValue->i8uBit);
  				}

  				|>\tikzmarkin[set border color=martinired]{i8uValue}<|piDev_g.ai8uPI[pValue->i16uAddress] = i8uValue_l;|>\tikzmarkend{i8uValue}<|
  				rt_mutex_unlock(&piDev_g.lockPI);

  #ifdef VERBOSE
  				pr_info("piControlIoctl Addr=%u, bit=%u: %02x %02x\n", pValue->i16uAddress, pValue->i8uBit, pValue->i8uValue, i8uValue_l);
  #endif

  				status = 0;
  			}
  		}
  		break; |>\setcounter{lstnumber}{1314}<|

    default:
      pr_err("Invalid Ioctl");
      return (-EINVAL);
      break;

    }

    return status;
  }
\end{lstlisting}

Listing~\ref{lst:4-piControlIoctl} zeigt in Auszügen die ioctl-Methode des piControl Kernel-Treibers. Diese bekommt folgende Argumente übergeben: \lstinline{struct file *file} enthält den Verweis auf die Geräte-Datei, hier \lstinline{/dev/piControl0}. Der Wert von \lstinline{unsigned int prg_nr} beschreibt die Anfrage an den Treiber, in diesem Fall \lstinline{KB_SET_VALUE}. Das Argument \lstinline{unsigned long usr_addr} enthält einen typ-agnostischen Pointer. Dieser verweist auf einen Speicherbereich, in welchem die zur Bearbeitung der Anfrage notwendigen Daten abgelegt sind. Hier können auch vom Treiber empfangene Daten dem Anwendungsprogramm bereitgestellt werden. 

Die switch-case-Anweisung führt die über das Argument \lstinline{prg_nr} spezifizierte Aktion aus. Hier betrachten wir \lstinline{KB_SET_VALUE}:
Zunächst wird in Zeile 868 der übergebene Zeiger \lstinline{usr_addr} mittels explizitem Typecast zu einem Zeiger des Typs \lstinline{SPIValue *} konvertiert. Da dieser auf Daten im Userspace verweist, ist beim Zugriff durch den Kernel-Treiber besondere Vorsicht geboten.
In Zeile 877 wird mittels Mutex das Prozessabbild \lstinline{piDev_g} für den Zugriff durch andere Threads oder Prozesse gesperrt.
\lstinline{my_rt_mutex_lock} verweist hierbei auf die Funktion \lstinline{rt_mutex_lock} aus \lstinline{linux/sched.h}\footnote{Offenbar wurde hier auch eine alternative Implementierung vorgesehen, siehe revpi\_common.h}

In Zeile 889 wird das Byte \lstinline{i8uValue_l}, welches den zu schreibenden Wert enthält in das Prozessabbild übertragen. Anschließend wird die Mutex auf \lstinline{piDev_g} wieder entsperrt.
\newpage

\begin{lstlisting}[language={c},firstnumber=62,caption={Auszug des Struct \lstinline{spiControlDev} in \lstinline{piControlMain.h}\label{lst:4-spiControlDev}}]
|>\tikzmarkin[set border color=martiniblue]{spiControlDev}<|typedef struct spiControlDev|>\tikzmarkend{spiControlDev}<| {
	// device driver stuff
	int init_step;
	enum revpi_machine machine_type;
	void *machine;
	struct cdev cdev;	// Char device structure
	struct device *dev;
	struct thermal_zone_device *thermal_zone;

	|>\tikzmarkin[set border color=martiniblue]{processImage}<|// process image stuff
	INT8U ai8uPI[KB_PI_LEN];
	INT8U ai8uPIDefault|>\tikzmarkin[set border color=martinired]{KB_PI_LEN_0}<|[KB_PI_LEN]|>\tikzmarkend{KB_PI_LEN_0}<|;
	struct rt_mutex lockPI;        |>\tikzmarkend{processImage}<|
	bool stopIO;
	piDevices *devs; |>\setcounter{lstnumber}{94}<|
} tpiControlDev;
\end{lstlisting}

Das Prozessabbild ist als Byte-Array der Länge \lstinline{KB_PI_LEN} in Listing~\ref{lst:4-spiControlDev} definiert. Konfigurationsparameter wie \lstinline{KB_PI_LEN} oder die Zykluszeit für den Datenaustausch zwischen SPS und IO-Modulen sind im folgenden Listing~\ref{lst:4-process} definiert.

\begin{lstlisting}[language={c},firstnumber=119,caption={Konfigurationsparameter des Prozessabbildes in project.h\label{lst:4-process}}]
#define INTERVAL_PI_GATE (5*1000*1000)  // 5 ms piGateCommunication |>\setcounter{lstnumber}{128}<|

#define INTERVAL_IO_COM (5*1000*1000)  // 5 ms piIoComm |>\setcounter{lstnumber}{132}<|

#define KB_PD_LEN       512
|>\tikzmarkin[set border color=martiniblue]{KB_PI_LEN_1}<|#define KB_PI_LEN       4096|>\tikzmarkend{KB_PI_LEN_1}<|
\end{lstlisting}

Das zu setzende Bit wurde zu diesem Zeitpunkt erfolgreich in das Prozessabbild der SPS geschrieben.
Es stellt sich die Frage, wie dieses nun an das IO-Modul kommuniziert wird.
Die Kommunikation mit allen angebundenen Modulen ist ebenfalls Aufgabe des piControl-Treibers.

\begin{lstlisting}[language={c},firstnumber=256,caption={Auszug der Methode \lstinline{piIoThread} in \lstinline{revpi_core.c}\label{lst:4-piIoThread}}]
static int piIoThread(void *data)
{
	//TODO int value = 0;
	ktime_t time;
	ktime_t now;
	s64 tDiff;

	hrtimer_init(&piCore_g.ioTimer, CLOCK_MONOTONIC, HRTIMER_MODE_ABS);
	piCore_g.ioTimer.function = piIoTimer;

	pr_info("piIO thread started\n");

	now = hrtimer_cb_get_time(&piCore_g.ioTimer);

	PiBridgeMaster_Reset();

	while (!kthread_should_stop()) {
		if (|>\tikzmarkin[set border color=martinired]{PiBridgeMaster}<|PiBridgeMaster_Run()|>\tikzmarkend{PiBridgeMaster}<| < 0)
			break;
	}

	RevPiDevice_finish();

	pr_info("piIO exit\n");
	return 0;
}
\end{lstlisting}

Der Kernel-Thread \lstinline{piIoThread} ist verantwortlich für den zyklischen Datenaustausch mit den IO-Modulen. In diesem wird fortlaufend die Methode \lstinline{PiBridgeMaster_Run()} aufgerufen, siehe Listing~\ref{lst:4-piIoThread}.

\begin{lstlisting}[language={c},firstnumber=262,caption={Auszug der Methode \lstinline{PiBridgeMaster_Run(void)} in \lstinline{RevPiDevice.c}\label{lst:4-PiBridgeMaster_Run}}]
int PiBridgeMaster_Run(void)
{
	static kbUT_Timer tTimeoutTimer_s;
	static kbUT_Timer tConfigTimeoutTimer_s;
	static int error_cnt;
	static INT8U last_led;
	static unsigned long last_update;
	int ret = 0;
	int i;

	my_rt_mutex_lock(&piCore_g.lockBridgeState);
	if (piCore_g.eBridgeState != piBridgeStop) {
		switch (eRunStatus_s) { |>\setcounter{lstnumber}{514}<|
		    case enPiBridgeMasterStatus_EndOfConfig:|>\setcounter{lstnumber}{621}<|
		    if (|>\tikzmarkin[set border color=martinired]{RevPiDevice}<|RevPiDevice_run()|>\tikzmarkend{RevPiDevice}<|) {
				// an error occured, check error limits |>\setcounter{lstnumber}{641}<|
			} else {
				ret = 1;
			}
			piCore_g.image.drv.i16uRS485ErrorCnt = RevPiDevice_getErrCnt();
			break;
\end{lstlisting}

Die in Listing~\ref{lst:4-PiBridgeMaster_Run} dargestellte Methode ist eine sog. State-Machine. Ist die Konfiguration der IO-Module erfolgreich abgeschlossen, so führt sie bei Aufruf lediglich die Methode \lstinline{RevPiDevice_run()} aus.

\begin{lstlisting}[language={c},firstnumber=140,caption={Auszug der Methode \lstinline{RevPiDevice_run(void)} in \lstinline{RevPiDevice.c}\label{lst:4-RevPiDevice_run}}]
int RevPiDevice_run(void)
{
	INT8U i8uDevice = 0;
	INT32U r;
	int retval = 0;

	RevPiDevices_s.i16uErrorCnt = 0;

	for (i8uDevice = 0; i8uDevice < RevPiDevice_getDevCnt(); i8uDevice++) {
		if (RevPiDevice_getDev(i8uDevice)->i8uActive) {
			switch (RevPiDevice_getDev(i8uDevice)->sId.i16uModulType) {
			case KUNBUS_FW_DESCR_TYP_PI_DIO_14:
			case KUNBUS_FW_DESCR_TYP_PI_DI_16:
			case KUNBUS_FW_DESCR_TYP_PI_DO_16:
				r = |>\tikzmarkin[set border color=martinired]{sendCyclicTelegram}<|piDIOComm_sendCyclicTelegram(i8uDevice)|>\tikzmarkend{sendCyclicTelegram}\setcounter{lstnumber}{166} <|;

				break; |>\setcounter{lstnumber}{216}<|
			}
		}
	} |>\setcounter{lstnumber}{227}<|
	return retval;
}
\end{lstlisting}

Diese iteriert wie in Listing~\ref{lst:4-RevPiDevice_run} abgebildete durch alle gegenwärtig in der SPS konfigurierten Module. Ist das aktuelle Modul als aktiv markiert, so wird anhand eines sog. Firmware-Descriptors entschieden, welche Methode für die Ansteuerung des Moduls aufzurufen ist.

\begin{lstlisting}[language={c},firstnumber=161,caption={Auszug der Methode \lstinline{piDIOComm_sendCyclicTelegram} in \lstinline{piDIOComm.c}\label{lst:4-sendCyclicTelegram}}]
INT32U piDIOComm_sendCyclicTelegram(INT8U i8uDevice_p)
{
	INT32U i32uRv_l = 0;
	SIOGeneric sRequest_l;
	SIOGeneric sResponse_l;
	INT8U len_l, data_out[18], i, p, data_in[70];
	INT8U i8uAddress;
	int ret; |>\setcounter{lstnumber}{239}<|
	
    |>\tikzmarkin[set border color=martinired]{piIoComm}<|ret = piIoComm_send((INT8U *) & sRequest_l, IOPROTOCOL_HEADER_LENGTH + len_l + 1);  |>\tikzmarkend{piIoComm}\setcounter{lstnumber}{298}<|
}
\end{lstlisting}

Im Falle des hier verwendeten DO-Moduls wird die in Listing~\ref{lst:4-sendCyclicTelegram} abgebildete Methode \lstinline{piDIOComm_sendCyclicTelegram()} aufgerufen. Dieser wird ein Zeiger auf das zu schreibende Gerät übergeben. 
Zunächst wird das Prozessabbild mittels eines proprietären, jedoch im Quellcode offen nachvollziehbaren Protokolls in ein \lstinline{sRequest_l} genanntes Byte-Array umgewandelt. Dieser Schritt ist in Listing~\ref{lst:4-sendCyclicTelegram} nicht abgebildet. Anschließend wird \lstinline{piIoComm_send()} ein Zeiger auf die so generierte Schreib-Anfrage übergeben.

\begin{lstlisting}[language={c},firstnumber=220,caption={Auszug der Methode \lstinline{piIOComm_send} in \lstinline{piIOComm.c}\label{lst:4-piIOComm_send}}]
int piIoComm_send(INT8U * buf_p, INT16U i16uLen_p)
{
	ssize_t write_l = 0;
	INT16U i16uSent_l = 0;|>\setcounter{lstnumber}{249}<|

	while (i16uSent_l < i16uLen_p) {
		write_l = vfs_write(piIoComm_fd_m, buf_p + i16uSent_l, i16uLen_p - i16uSent_l, &piIoComm_fd_m->f_pos);
		if (write_l < 0) {
			pr_info_serial("write error %d\n", (int)write_l);
			return -1;
		} 
		i16uSent_l += write_l;|>\setcounter{lstnumber}{263}<|
	}
	clear();
	vfs_fsync(piIoComm_fd_m, 1);
	return 0;
}
\end{lstlisting}

Listing~\ref{lst:4-piIOComm_send} zeigt die Implementierung von \lstinline{piIoComm_send()}. Diese Methode ist für das Schreiben der oben generierten Anfrage auf die seriellen Schnittstelle verantwortlich. Realisiert wird dies mittels der Methode \lstinline{vfs_write()}. Diese ist in \lstinline{<linux/fs.h>} definiert. Sie ermöglicht das Schreiben einer Datei im Userspace aus dem Kernel heraus. Geschrieben wird hier die Datei mit dem Deskriptor \lstinline{piIoComm_fd_m}.
Da die Funktion \lstinline{vfs_write()} durch andere Kernel-Tasks unterbrochen werden kann, ist nicht gewährleistet, dass die gesamte Anfrage mit nur einem Aufruf geschrieben wird. Die oben abgebildete while-Schleife stellt das vollständige Senden der Anfrage sicher.

\begin{lstlisting}[language={c},firstnumber=157,caption={Auszug der Methode \lstinline{piIOComm_open_serial} in \lstinline{piIOComm.c}\label{lst:4-piIOComm_open_serial}}]
int piIoComm_open_serial(void)
{   |>\setcounter{lstnumber}{167}<|
	struct file *fd;	/* Filedeskriptor */
	struct termios newtio;	/* Schnittstellenoptionen */

	|>\tikzmarkin[set border color=martiniblue]{fd}<|/* Port oeffnen - read/write, kein "controlling tty", 
	    Status von DCD ignorieren */
	fd = filp_open(|>\tikzmarkin[set border color=martinired]{tty}<|REV_PI_TTY_DEVICE|>\tikzmarkend{tty}<|, O_RDWR | O_NOCTTY, 0); |>\setcounter{lstnumber}{208}<|
	
	piIoComm_fd_m = fd;                                                      |>\tikzmarkend{fd}\setcounter{lstnumber}{217}<|

	return 0;
}
\end{lstlisting}

Der zum Schreiben auf die serielle Schnittstelle verwendete Datei-Deskriptor wird von der in Listing~\ref{lst:4-piIOComm_open_serial} abgebildeten Methode \lstinline{piIoComm_open_serial()} generiert. 

\begin{lstlisting}[language={c},firstnumber=45,caption={Definition der seriellen Schnittstelle in \lstinline{piIOComm.h}\label{lst:4-REV_PI_TTY_DEVICE}}]
#define REV_PI_TTY_DEVICE	"/dev/ttyAMA0"
\end{lstlisting}

Das in Listing~\ref{lst:4-REV_PI_TTY_DEVICE} definierte Macro verweist auf eine der seriellen Schnittstellen des RaspberryPi.
Die Implementierung des zugehörigen Schnittstellentreibers soll hier nicht weiter untersucht werden. Somit ist an dieser Stelle die Kette vom Setzen einer Variablen auf dem OPC-Server bis hin zur Aktualisierung des Prozessabbilds der IO-Module geschlossen.

% \begin{lstlisting}[language={c},firstnumber={226},caption={Setzen der Scheduler-Priorität auf SCHED\_FIFO in 
% revpi\_common.c\label{lst:2-sched_priority}}]
% param.sched_priority = ktprio->prio;
% ret = sched_setscheduler(child, SCHED_FIFO, &param);
% \end{lstlisting}
% % % Imports nur für Referenzenauflösung während des Schreibens! Vorm Kompilieren auskommentieren!
% \bibliography{0_hauptdatei}
% \input{1_einleitung}
% \input{2_grundlagen}
% \input{3_konzeption}
% \input{4_implementierung}
% \input{5_tests}
% \input{6_zusammenfassung}
% % Ende Imports

\section{Test des OPC-Servers im Gesamtsystem%
  \label{sec:5-tests}}

% % % Imports nur für Referenzenauflösung während des schreibens! Vorm Kompilieren auskommentieren!
% \bibliography{0_hauptdatei}
% \input{1_einleitung}
% \input{2_grundlagen}
% \input{3_konzeption}
% \input{4_implementierung}
% \input{5_tests}
% \input{6_zusammenfassung}
% % Ende Imports

\section{Zusammenfassung und Ausblick%
  \label{sec:6-fazit}}
Der folgende Abschnitt~\ref{sec:6-zusammenfassung} fasst die gewonnenen Erkenntnisse und den Stand der Implementierung zusammen.
Den Abschluss dieser Arbeit bildet der Ausblick in Abschnitt~\ref{sec:6-ausblick}.

\subsection{Zusammenfassung%
     \label{sec:6-zusammenfassung}}

\subsection{Ausblick%
     \label{sec:6-ausblick}}

% \input{anhang}
% % Ende Imports

\section{Implementierung%
  \label{sec:4-implementierung}}
Das folgende Kapitel stellt in Auszügen die Implementierung des OPC-Servers sowie die Anbindung an die IO-Module
der SPS dar. Der Schwerpunkt liegt hierbei auf der Funktionsweise des piControl-Treibers und dessen Integration in das Projekt. Abschnitt~\ref{sec:4-picontrol} erklärt die zum Schreibens eines Bits verwendeten Funktionsaufrufe.
Zuvor soll jedoch in Abschnitt~\ref{sec:4-open62541} der Teil des OPC-Servers vorgestellt werden, welcher auf besagten Treiber zugreift. 

\subsection{Implementierung des OPC-Servers%
     \label{sec:4-open62541}}
Wie im vorangegangenen Abschnitt~\ref{sec:3-integration} begründet, soll die Verknüpfung zwischen dem Prozessabbild der SPS und den auf dem OPC-Server bereitgestellten Werten über sog.\,Datenquellen erfolgen. Hierzu ist zunächst eine Callback-Methode zu implementieren, welche bei einem Lese- oder Schreibzugriff auf eine Variable aufgerufen wird. Die Verknüpfung zwischen Callback-Methode und Variable muss manuell erfolgen.

\begin{lstlisting}[language={c},firstnumber=237,caption={Auszug der Methode \lstinline{linkDataSourceVariable} in \lstinline{variables.c}\label{lst:4-linkDataSourceVariable}}]
extern UA_StatusCode
 linkDataSourceVariable(UA_Server *server, UA_NodeId nodeId) {
     bool readonly = false;
     UA_DataSource dataSourceVariable;
     UA_StatusCode rc; |>\setcounter{lstnumber}{254}<|

     dataSourceVariable.read = readDataSourceVariable;
     if (!readonly)
        dataSourceVariable.write = writeDataSourceVariable;
     else
        dataSourceVariable.write = writeReadonlyDataSourceVariable;

     return UA_Server_setVariableNode_dataSource(server, nodeId, dataSourceVariable);
 }
\end{lstlisting}

\begin{figure}[h]
    \centering
    \includegraphics[width=0.42\textwidth]{doc/img/OPC_RevPiDO.pdf}
    \caption{Auszug des verwendeten Nodesets, hier Digitalausgang 1 des Versuchsaufbaus
      \label{fig:opc-do}}
\end{figure}

Die in Listing~\ref{lst:4-linkDataSourceVariable} abgebildete Methode \lstinline{linkDataSourceVariable()} erzeugt ein Struct vom Typ \lstinline{UA_DataSource}. In diesem werden dem Lesen und Schreiben einer OPC-Variablen entsprechende Callback-Methoden zugewiesen. Die Verknüpfung einer OPC-Variable, genauer ihrer NodeId, mit der zuvor definierten Datenquelle erfolgt über die von open62541 bereitgestellte Methode \lstinline{UA_Server_setVariableNode_dataSource()}. Vor dem Lesen und nach dem Schreiben dieser Variable werden von nun an die entsprechenden Callbacks aufgerufen.
     
\begin{lstlisting}[language={c},firstnumber=168,caption={Auszug des Callbacks \lstinline{writeDataSourceVariable} in \lstinline{variables.c}\label{lst:4-writeDataSourceVariable}}]  
extern UA_StatusCode
 writeDataSourceVariable(UA_Server *server,
            const UA_NodeId *sessionId, void *sessionContext,
            const UA_NodeId *nodeId, void *nodeContext,
            const UA_NumericRange *range, const UA_DataValue *dataValue) {

    UA_StatusCode retval  = UA_STATUSCODE_GOOD;
    UA_NodeId *nameNodeId = UA_malloc(sizeof(UA_NodeId));
    UA_QualifiedName nameQN = UA_QUALIFIEDNAME(1, "Name");
    UA_Variant nameVar;
    UA_Boolean bit;

    retval |= findSiblingByBrowsename(server, nodeId, &nameQN, nameNodeId);
    retval |= UA_Server_readValue(server, *nameNodeId, &nameVar);
    retval |= UA_Boolean_copy(dataValue->value.data, &bit);

    |>\tikzmarkin[set border color=martinired]{writeIO}<|PI_writeSingleIO(String_fromUA_String(nameVar.data), &bit, false);                                                 |>\tikzmarkend{writeIO}<|

    free(nameNodeId);
    return retval;
 }
\end{lstlisting}

Listing~\ref{lst:4-writeDataSourceVariable} zeigt die Callback-Methode, welche nach dem Schreiben einer Variablen auf dem OPC-Server aufgerufen wird.
Dieser Methode wird neben der NodeId der mit ihr verknüpften Variablen auch der Wert dieser in Form eines Zeigers auf ein Struct vom Typ \lstinline{UA_DataValue} übergeben.

Die Gestaltung des hier verwendeten Nodesets sieht vor, dass in einer OPC-Variablen \lstinline{"Name"} der Bezeichner des zu schreibenden Digitalausgangs hinterlegt ist, siehe Abbildung~\ref{fig:opc-do}. Dies erlaubt eine Rekonfiguration der Ein- und Ausgänge der SPS ohne Änderungen im Programmcode des OPC-Servers vornehmen zu müssen.
Es ist daher erforderlich, nach jedem Schreiben einer mit einem Digitalausgang verknüpften Variablen, hier \lstinline{"Value"}, dessen Bezeichner \lstinline{"Name"} abzufragen. 
Dies geschieht in den Zeilen 180 und 181.
Anschließend wird dieser Bezeichner sowie der zu schreibende Wert der Methode \lstinline{PI_writeSingleIO()} übergeben, welche wiederum die Interaktion mit piControl übernimmt (vgl. Abschnitt \ref{sec:4-picontrol}).
 
\subsection{Integration von piControl%
     \label{sec:4-picontrol}}
In Abschnitt~\ref{sec:2-io} wurde die Anbindung der IO-Module des Revolution Pi sowie die Funktionsweise von piControl aus Anwendersicht beschrieben. Die verfügbare Literatur beschränkt sich auch auf lediglich diese Sicht; eine weiterführende Dokumentation für Entwickler gibt es, neben der in Abschnitt~\ref{sec:3-anbindung} vorgestellten Manpage, nicht. 
In diesem Abschnitt soll daher der Quellcode von piControl sowie dessen Verwendung im Projekt genauer betrachtet werden.
Hierzu wird exemplarisch die in Abschnitt~\ref{sec:4-open62541} eingeführte Methode \lstinline{PI_writeSingleIO()} untersucht.
Diese Methode ermöglicht das Setzen eines einzelnen Bits im Prozessabbild der SPS, und damit das Schalten eines digitalen Ausgangs auf einem IO-Modul.
Die äquivalente Methode \lstinline{int piControlGetBitValue(SPIValue *pSpiValue)} zum Lesen eines Bits bzw. Eingangs funktioniert analog und soll daher an dieser Stelle nicht dediziert erörtert werden.

\begin{lstlisting}[language={c},firstnumber=97,
                   caption={Setzen eines phsikalischen, digitalen Ausgangs in \lstinline{revpi.c}
                   \label{lst:4-PI_writeSingleIO}}]
extern void PI_writeSingleIO(char *pszVariableName, bool *bit, bool verbose)
{
	int rc;
	SPIVariable sPiVariable;
	SPIValue sPIValue;

	strncpy(sPiVariable.strVarName, pszVariableName, sizeof(sPiVariable.strVarName));
	rc = piControlGetVariableInfo(&sPiVariable);
	if (rc < 0) {
		printf("Cannot find variable '%s'\n", pszVariableName);
		return;
	}

		sPIValue.i16uAddress = sPiVariable.i16uAddress;
		sPIValue.i8uBit = sPiVariable.i8uBit;
		sPIValue.i8uValue = *bit;
		rc = |>\tikzmarkin[set border color=martinired]{setBitValue}<|piControlSetBitValue(&sPIValue)|>\tikzmarkend{setBitValue}<|;
		if (rc < 0)
			printf("Set bit error %s\n", getWriteError(rc));
		else if (verbose)
			printf("Set bit %d on byte at offset %d. Value %d\n", sPIValue.i8uBit, sPIValue.i16uAddress,
			       sPIValue.i8uValue);
}
\end{lstlisting}

Der Programmcode in Listing~\ref{lst:4-PI_writeSingleIO} ist Teil des implementierten OPC-Servers. In diesem wird auf zwei Funktionen des piControl-Treibers zugegriffen. 
Beiden Methoden wird als Argument ein Zeiger auf ein Struct vom Typ \lstinline{SPIValue} übergeben. Der im Struct abgelegte Name wird mittels \lstinline{piControlGetVariableInfo(&sPIValue)} zu einer Adresse im Prozessabbild aufgelöst. Diese wird in \lstinline{sPIValue.i16uAdress} gespeichert. Der Wert der Variablen wird anschließend mittels \lstinline{piControlSetBitValue(&sPIValue)} an dieser Adresse in das Prozessabbild geschrieben.

\begin{lstlisting}[language={c},firstnumber=309,caption={Methode \lstinline{piControlSetBitValue} in \lstinline{piControlIf.c}\label{lst:4-piControlSetBitValue}}]
int |>\tikzmarkin[set border color=martiniblue]{setBitValueFcn}<|piControlSetBitValue(SPIValue *pSpiValue)|>\tikzmarkend{setBitValueFcn}<|
{
    piControlOpen();

    if (PiControlHandle_g < 0)
	    return -ENODEV;

    pSpiValue->i16uAddress += pSpiValue->i8uBit / 8;
    pSpiValue->i8uBit %= 8;

    if (|>\tikzmarkin[set border color=martinired]{ioctl}<|ioctl(PiControlHandle_g, KB_SET_VALUE, pSpiValue)|>\tikzmarkend{ioctl}<| < 0)
	    return errno;

    return 0;
}
\end{lstlisting}

Die in Listing~\ref{lst:4-piControlSetBitValue} dargestellte Methode \lstinline{piControlSetBitValue} ist lediglich eine Hüllfunktion (häufig auch als Wrapper-Funktion bezeichnet) für einen Aufruf des \lstinline{ioctl} Kernel-Moduls.
Folgende Parameter werden übergeben:
\lstinline{PiControlHandle_g} ist die Referenz auf die Geräte-Datei des piControl-Treibers. \lstinline{KB_SET_VALUE} ist das ioctl-Kommando zum Schreiben eines Bits in das Prozessabbild. Der Zeiger \lstinline{pSpiValue} verweist auf ein Struct des bereits vorgestellten Typs \lstinline{SPIValue}.

\begin{lstlisting}[language={c},firstnumber=80,caption={Methode \lstinline{piControlOpen} in \lstinline{piControlIf.c}\label{lst:4-piControlOpen}}]
void piControlOpen(void)
{
    /* open handle if needed */
    if (PiControlHandle_g < 0)
    {
	    |>\tikzmarkin[set border color=martiniblue]{PiControlHandle}<|PiControlHandle_g = open(PICONTROL_DEVICE, O_RDWR)|>\tikzmarkend{PiControlHandle}<|;
    }
}
\end{lstlisting}

Die in Listing~\ref{lst:4-piControlOpen} dargestellte Methode öffnet, sofern nicht bereits geschehen, die Geräte-Datei. Das Macro \lstinline{PICONTROL_DEVICE} verweist hierbei auf \lstinline{/dev/piControl0}.

\begin{lstlisting}[language={c},firstnumber=721,caption={Methode \lstinline{piControlIoctl} in \lstinline{piControlMain.c}\label{lst:4-piControlIoctl}}]
static long |>\tikzmarkin[set border color=martiniblue, below offset=0.9em]{piControlIoctl}<|piControlIoctl(struct file *file, unsigned int prg_nr, 
                           unsigned long usr_addr)                                      |>\tikzmarkend{piControlIoctl}<|
{
  int status = -EFAULT;
  tpiControlInst *priv;
  int timeout = 10000;	// ms

  if (prg_nr != KB_CONFIG_SEND && prg_nr != KB_CONFIG_START && !isRunning()) {
  	return -EAGAIN;
  }

  priv = (tpiControlInst *) file->private_data;

  if (prg_nr != KB_GET_LAST_MESSAGE) {
  	// clear old message
  	priv->pcErrorMessage[0] = 0;
  }

  switch (prg_nr) {|>\setcounter{lstnumber}{864}<|

    case |>\tikzmarkin[set border color=martiniblue]{KB_SET_VALUE}<|KB_SET_VALUE:|>\tikzmarkend{KB_SET_VALUE}<|
  		{
  			SPIValue *pValue = (SPIValue *) usr_addr;

  			if (!isRunning())
  				return -EFAULT;

  			if (pValue->i16uAddress >= KB_PI_LEN) {
  				status = -EFAULT;
  			} else {
  				INT8U i8uValue_l;
  				my_rt_mutex_lock(&piDev_g.lockPI);
  				i8uValue_l = piDev_g.ai8uPI[pValue->i16uAddress];

  				if (pValue->i8uBit >= 8) {
  					i8uValue_l = pValue->i8uValue;
  				} else {
  					if (pValue->i8uValue)
  						i8uValue_l |= (1 << pValue->i8uBit);
  					else
  						i8uValue_l &= ~(1 << pValue->i8uBit);
  				}

  				|>\tikzmarkin[set border color=martinired]{i8uValue}<|piDev_g.ai8uPI[pValue->i16uAddress] = i8uValue_l;|>\tikzmarkend{i8uValue}<|
  				rt_mutex_unlock(&piDev_g.lockPI);

  #ifdef VERBOSE
  				pr_info("piControlIoctl Addr=%u, bit=%u: %02x %02x\n", pValue->i16uAddress, pValue->i8uBit, pValue->i8uValue, i8uValue_l);
  #endif

  				status = 0;
  			}
  		}
  		break; |>\setcounter{lstnumber}{1314}<|

    default:
      pr_err("Invalid Ioctl");
      return (-EINVAL);
      break;

    }

    return status;
  }
\end{lstlisting}

Listing~\ref{lst:4-piControlIoctl} zeigt in Auszügen die ioctl-Methode des piControl Kernel-Treibers. Diese bekommt folgende Argumente übergeben: \lstinline{struct file *file} enthält den Verweis auf die Geräte-Datei, hier \lstinline{/dev/piControl0}. Der Wert von \lstinline{unsigned int prg_nr} beschreibt die Anfrage an den Treiber, in diesem Fall \lstinline{KB_SET_VALUE}. Das Argument \lstinline{unsigned long usr_addr} enthält einen typ-agnostischen Pointer. Dieser verweist auf einen Speicherbereich, in welchem die zur Bearbeitung der Anfrage notwendigen Daten abgelegt sind. Hier können auch vom Treiber empfangene Daten dem Anwendungsprogramm bereitgestellt werden. 

Die switch-case-Anweisung führt die über das Argument \lstinline{prg_nr} spezifizierte Aktion aus. Hier betrachten wir \lstinline{KB_SET_VALUE}:
Zunächst wird in Zeile 868 der übergebene Zeiger \lstinline{usr_addr} mittels explizitem Typecast zu einem Zeiger des Typs \lstinline{SPIValue *} konvertiert. Da dieser auf Daten im Userspace verweist, ist beim Zugriff durch den Kernel-Treiber besondere Vorsicht geboten.
In Zeile 877 wird mittels Mutex das Prozessabbild \lstinline{piDev_g} für den Zugriff durch andere Threads oder Prozesse gesperrt.
\lstinline{my_rt_mutex_lock} verweist hierbei auf die Funktion \lstinline{rt_mutex_lock} aus \lstinline{linux/sched.h}\footnote{Offenbar wurde hier auch eine alternative Implementierung vorgesehen, siehe revpi\_common.h}

In Zeile 889 wird das Byte \lstinline{i8uValue_l}, welches den zu schreibenden Wert enthält in das Prozessabbild übertragen. Anschließend wird die Mutex auf \lstinline{piDev_g} wieder entsperrt.
\newpage

\begin{lstlisting}[language={c},firstnumber=62,caption={Auszug des Struct \lstinline{spiControlDev} in \lstinline{piControlMain.h}\label{lst:4-spiControlDev}}]
|>\tikzmarkin[set border color=martiniblue]{spiControlDev}<|typedef struct spiControlDev|>\tikzmarkend{spiControlDev}<| {
	// device driver stuff
	int init_step;
	enum revpi_machine machine_type;
	void *machine;
	struct cdev cdev;	// Char device structure
	struct device *dev;
	struct thermal_zone_device *thermal_zone;

	|>\tikzmarkin[set border color=martiniblue]{processImage}<|// process image stuff
	INT8U ai8uPI[KB_PI_LEN];
	INT8U ai8uPIDefault|>\tikzmarkin[set border color=martinired]{KB_PI_LEN_0}<|[KB_PI_LEN]|>\tikzmarkend{KB_PI_LEN_0}<|;
	struct rt_mutex lockPI;        |>\tikzmarkend{processImage}<|
	bool stopIO;
	piDevices *devs; |>\setcounter{lstnumber}{94}<|
} tpiControlDev;
\end{lstlisting}

Das Prozessabbild ist als Byte-Array der Länge \lstinline{KB_PI_LEN} in Listing~\ref{lst:4-spiControlDev} definiert. Konfigurationsparameter wie \lstinline{KB_PI_LEN} oder die Zykluszeit für den Datenaustausch zwischen SPS und IO-Modulen sind im folgenden Listing~\ref{lst:4-process} definiert.

\begin{lstlisting}[language={c},firstnumber=119,caption={Konfigurationsparameter des Prozessabbildes in project.h\label{lst:4-process}}]
#define INTERVAL_PI_GATE (5*1000*1000)  // 5 ms piGateCommunication |>\setcounter{lstnumber}{128}<|

#define INTERVAL_IO_COM (5*1000*1000)  // 5 ms piIoComm |>\setcounter{lstnumber}{132}<|

#define KB_PD_LEN       512
|>\tikzmarkin[set border color=martiniblue]{KB_PI_LEN_1}<|#define KB_PI_LEN       4096|>\tikzmarkend{KB_PI_LEN_1}<|
\end{lstlisting}

Das zu setzende Bit wurde zu diesem Zeitpunkt erfolgreich in das Prozessabbild der SPS geschrieben.
Es stellt sich die Frage, wie dieses nun an das IO-Modul kommuniziert wird.
Die Kommunikation mit allen angebundenen Modulen ist ebenfalls Aufgabe des piControl-Treibers.

\begin{lstlisting}[language={c},firstnumber=256,caption={Auszug der Methode \lstinline{piIoThread} in \lstinline{revpi_core.c}\label{lst:4-piIoThread}}]
static int piIoThread(void *data)
{
	//TODO int value = 0;
	ktime_t time;
	ktime_t now;
	s64 tDiff;

	hrtimer_init(&piCore_g.ioTimer, CLOCK_MONOTONIC, HRTIMER_MODE_ABS);
	piCore_g.ioTimer.function = piIoTimer;

	pr_info("piIO thread started\n");

	now = hrtimer_cb_get_time(&piCore_g.ioTimer);

	PiBridgeMaster_Reset();

	while (!kthread_should_stop()) {
		if (|>\tikzmarkin[set border color=martinired]{PiBridgeMaster}<|PiBridgeMaster_Run()|>\tikzmarkend{PiBridgeMaster}<| < 0)
			break;
	}

	RevPiDevice_finish();

	pr_info("piIO exit\n");
	return 0;
}
\end{lstlisting}

Der Kernel-Thread \lstinline{piIoThread} ist verantwortlich für den zyklischen Datenaustausch mit den IO-Modulen. In diesem wird fortlaufend die Methode \lstinline{PiBridgeMaster_Run()} aufgerufen, siehe Listing~\ref{lst:4-piIoThread}.

\begin{lstlisting}[language={c},firstnumber=262,caption={Auszug der Methode \lstinline{PiBridgeMaster_Run(void)} in \lstinline{RevPiDevice.c}\label{lst:4-PiBridgeMaster_Run}}]
int PiBridgeMaster_Run(void)
{
	static kbUT_Timer tTimeoutTimer_s;
	static kbUT_Timer tConfigTimeoutTimer_s;
	static int error_cnt;
	static INT8U last_led;
	static unsigned long last_update;
	int ret = 0;
	int i;

	my_rt_mutex_lock(&piCore_g.lockBridgeState);
	if (piCore_g.eBridgeState != piBridgeStop) {
		switch (eRunStatus_s) { |>\setcounter{lstnumber}{514}<|
		    case enPiBridgeMasterStatus_EndOfConfig:|>\setcounter{lstnumber}{621}<|
		    if (|>\tikzmarkin[set border color=martinired]{RevPiDevice}<|RevPiDevice_run()|>\tikzmarkend{RevPiDevice}<|) {
				// an error occured, check error limits |>\setcounter{lstnumber}{641}<|
			} else {
				ret = 1;
			}
			piCore_g.image.drv.i16uRS485ErrorCnt = RevPiDevice_getErrCnt();
			break;
\end{lstlisting}

Die in Listing~\ref{lst:4-PiBridgeMaster_Run} dargestellte Methode ist eine sog. State-Machine. Ist die Konfiguration der IO-Module erfolgreich abgeschlossen, so führt sie bei Aufruf lediglich die Methode \lstinline{RevPiDevice_run()} aus.

\begin{lstlisting}[language={c},firstnumber=140,caption={Auszug der Methode \lstinline{RevPiDevice_run(void)} in \lstinline{RevPiDevice.c}\label{lst:4-RevPiDevice_run}}]
int RevPiDevice_run(void)
{
	INT8U i8uDevice = 0;
	INT32U r;
	int retval = 0;

	RevPiDevices_s.i16uErrorCnt = 0;

	for (i8uDevice = 0; i8uDevice < RevPiDevice_getDevCnt(); i8uDevice++) {
		if (RevPiDevice_getDev(i8uDevice)->i8uActive) {
			switch (RevPiDevice_getDev(i8uDevice)->sId.i16uModulType) {
			case KUNBUS_FW_DESCR_TYP_PI_DIO_14:
			case KUNBUS_FW_DESCR_TYP_PI_DI_16:
			case KUNBUS_FW_DESCR_TYP_PI_DO_16:
				r = |>\tikzmarkin[set border color=martinired]{sendCyclicTelegram}<|piDIOComm_sendCyclicTelegram(i8uDevice)|>\tikzmarkend{sendCyclicTelegram}\setcounter{lstnumber}{166} <|;

				break; |>\setcounter{lstnumber}{216}<|
			}
		}
	} |>\setcounter{lstnumber}{227}<|
	return retval;
}
\end{lstlisting}

Diese iteriert wie in Listing~\ref{lst:4-RevPiDevice_run} abgebildete durch alle gegenwärtig in der SPS konfigurierten Module. Ist das aktuelle Modul als aktiv markiert, so wird anhand eines sog. Firmware-Descriptors entschieden, welche Methode für die Ansteuerung des Moduls aufzurufen ist.

\begin{lstlisting}[language={c},firstnumber=161,caption={Auszug der Methode \lstinline{piDIOComm_sendCyclicTelegram} in \lstinline{piDIOComm.c}\label{lst:4-sendCyclicTelegram}}]
INT32U piDIOComm_sendCyclicTelegram(INT8U i8uDevice_p)
{
	INT32U i32uRv_l = 0;
	SIOGeneric sRequest_l;
	SIOGeneric sResponse_l;
	INT8U len_l, data_out[18], i, p, data_in[70];
	INT8U i8uAddress;
	int ret; |>\setcounter{lstnumber}{239}<|
	
    |>\tikzmarkin[set border color=martinired]{piIoComm}<|ret = piIoComm_send((INT8U *) & sRequest_l, IOPROTOCOL_HEADER_LENGTH + len_l + 1);  |>\tikzmarkend{piIoComm}\setcounter{lstnumber}{298}<|
}
\end{lstlisting}

Im Falle des hier verwendeten DO-Moduls wird die in Listing~\ref{lst:4-sendCyclicTelegram} abgebildete Methode \lstinline{piDIOComm_sendCyclicTelegram()} aufgerufen. Dieser wird ein Zeiger auf das zu schreibende Gerät übergeben. 
Zunächst wird das Prozessabbild mittels eines proprietären, jedoch im Quellcode offen nachvollziehbaren Protokolls in ein \lstinline{sRequest_l} genanntes Byte-Array umgewandelt. Dieser Schritt ist in Listing~\ref{lst:4-sendCyclicTelegram} nicht abgebildet. Anschließend wird \lstinline{piIoComm_send()} ein Zeiger auf die so generierte Schreib-Anfrage übergeben.

\begin{lstlisting}[language={c},firstnumber=220,caption={Auszug der Methode \lstinline{piIOComm_send} in \lstinline{piIOComm.c}\label{lst:4-piIOComm_send}}]
int piIoComm_send(INT8U * buf_p, INT16U i16uLen_p)
{
	ssize_t write_l = 0;
	INT16U i16uSent_l = 0;|>\setcounter{lstnumber}{249}<|

	while (i16uSent_l < i16uLen_p) {
		write_l = vfs_write(piIoComm_fd_m, buf_p + i16uSent_l, i16uLen_p - i16uSent_l, &piIoComm_fd_m->f_pos);
		if (write_l < 0) {
			pr_info_serial("write error %d\n", (int)write_l);
			return -1;
		} 
		i16uSent_l += write_l;|>\setcounter{lstnumber}{263}<|
	}
	clear();
	vfs_fsync(piIoComm_fd_m, 1);
	return 0;
}
\end{lstlisting}

Listing~\ref{lst:4-piIOComm_send} zeigt die Implementierung von \lstinline{piIoComm_send()}. Diese Methode ist für das Schreiben der oben generierten Anfrage auf die seriellen Schnittstelle verantwortlich. Realisiert wird dies mittels der Methode \lstinline{vfs_write()}. Diese ist in \lstinline{<linux/fs.h>} definiert. Sie ermöglicht das Schreiben einer Datei im Userspace aus dem Kernel heraus. Geschrieben wird hier die Datei mit dem Deskriptor \lstinline{piIoComm_fd_m}.
Da die Funktion \lstinline{vfs_write()} durch andere Kernel-Tasks unterbrochen werden kann, ist nicht gewährleistet, dass die gesamte Anfrage mit nur einem Aufruf geschrieben wird. Die oben abgebildete while-Schleife stellt das vollständige Senden der Anfrage sicher.

\begin{lstlisting}[language={c},firstnumber=157,caption={Auszug der Methode \lstinline{piIOComm_open_serial} in \lstinline{piIOComm.c}\label{lst:4-piIOComm_open_serial}}]
int piIoComm_open_serial(void)
{   |>\setcounter{lstnumber}{167}<|
	struct file *fd;	/* Filedeskriptor */
	struct termios newtio;	/* Schnittstellenoptionen */

	|>\tikzmarkin[set border color=martiniblue]{fd}<|/* Port oeffnen - read/write, kein "controlling tty", 
	    Status von DCD ignorieren */
	fd = filp_open(|>\tikzmarkin[set border color=martinired]{tty}<|REV_PI_TTY_DEVICE|>\tikzmarkend{tty}<|, O_RDWR | O_NOCTTY, 0); |>\setcounter{lstnumber}{208}<|
	
	piIoComm_fd_m = fd;                                                      |>\tikzmarkend{fd}\setcounter{lstnumber}{217}<|

	return 0;
}
\end{lstlisting}

Der zum Schreiben auf die serielle Schnittstelle verwendete Datei-Deskriptor wird von der in Listing~\ref{lst:4-piIOComm_open_serial} abgebildeten Methode \lstinline{piIoComm_open_serial()} generiert. 

\begin{lstlisting}[language={c},firstnumber=45,caption={Definition der seriellen Schnittstelle in \lstinline{piIOComm.h}\label{lst:4-REV_PI_TTY_DEVICE}}]
#define REV_PI_TTY_DEVICE	"/dev/ttyAMA0"
\end{lstlisting}

Das in Listing~\ref{lst:4-REV_PI_TTY_DEVICE} definierte Macro verweist auf eine der seriellen Schnittstellen des RaspberryPi.
Die Implementierung des zugehörigen Schnittstellentreibers soll hier nicht weiter untersucht werden. Somit ist an dieser Stelle die Kette vom Setzen einer Variablen auf dem OPC-Server bis hin zur Aktualisierung des Prozessabbilds der IO-Module geschlossen.

% \begin{lstlisting}[language={c},firstnumber={226},caption={Setzen der Scheduler-Priorität auf SCHED\_FIFO in 
% revpi\_common.c\label{lst:2-sched_priority}}]
% param.sched_priority = ktprio->prio;
% ret = sched_setscheduler(child, SCHED_FIFO, &param);
% \end{lstlisting}
% % % Imports nur für Referenzenauflösung während des Schreibens! Vorm Kompilieren auskommentieren!
% \bibliography{0_hauptdatei}
% % Mit \section{...} eröffnen wir einen neuen Abschnitt.
% Der Befehl setzt nicht nur den Text in einer größeren,
% fetten Schrift, sondern sorgt außerdem dafür, daß er im
% Inhaltsverzeichnis erscheint.
%
% Mit \label{...} erzeugen wir einen Bezeichner, mit dessen Hilfe
% wir später auf die Nummer des Abschnitts verweisen können (nämlich
% mit~\ref{...}).
%
% Das Kommentarzeichen hinter „Übersicht“ dient dazu, ein
% Leerzeichen zwischen „Übersicht“ und dem \label-Befehl
% zu vermeiden, das andernfalls sichtbar würde – z.B. im
% Inhaltsverzeichnis.
%

% % Imports nur für Referenzenauflösung während des Schreibens! Vorm Kompilieren auskommentieren!
% \bibliography{0_hauptdatei}
% \input{1_einleitung}
%\input{2_grundlagen}
%\input{3_konzeption}
%\input{4_implementierung}
%\input{5_tests}
%\input{6_zusammenfassung}
% % Ende Imports

\section{Einleitung und Motivation%
  \label{sec:1-einleitung}}
Ziel dieses Projektes ist die Integration eines OPC-Servers mit einer auf Linux
basierenden speicherprogrammierbaren Steuerung (SPS). Angeschlossen an diese SPS
ist jeweils ein digitales Ein-/\,bzw.~Ausgabemodul. Die von diesen bereitgestellten
Ein-/\, bzw.~Ausgänge (IO) sollen in der Datenstruktur des OPC-Servers abgebildet
und über diesen für OPC-Clients les-/\,und schreibar sein. Weiterhin sollen einige
Funktionen zur Überwachung und Steuerung der an die SPS angeschlossenen Aktoren
und Sensoren direkt im OPC-Server implementiert werden.
Hiermit stellt dieses Projekt eine der Grundlagen für ein übergeordnetes Projekt,
die cloudbasierte Steuerung eines miniaturisierten Produktions-Systems, dar.

Der hier verwendete OPC-Server ist Teil des sog. open62541 Projekts. Er ist in C
geschrieben und implementiert bereits einen großen Teil der im OPC-UA-Standard
spezifizierten Funktionen.
Als SPS findet ein Revolution Pi 3 der Firma Kunbus Verwendung. Dieser integriert
ein sog. Compute Module der Raspberry Pi Foundation in ein industrietaugliches
Gehäuse und erlaubt die Erweiterung mittels IO- oder Gateway-Modulen. Über diese
erfolgt die Kommunikation mit weiteren Komponenten der Automatisierungstechnik.

Motiviert ist dieses Projekt durch die Beobachtung, dass die Verbreitung offener
Standards sowie freier Software auch in der Automatisierungstechnik zunimmt.
Linux ist ein freies Betriebssystem, OPC-UA ein offen zugänglicher, aktiv gepflegter
und weit verbreiteter Standard. Der Raspberry Pi findet sowohl bei Hobby-Anwendern als
auch in den Bereichen Forschung und Entwicklung sowie bei industriellen Anwendern
Verwendung. Dieses Projekt stellt somit eine für unterschiedliche Anwender interessante
Entwicklung dar.

Im Anschluss an diese einleitende Übersicht im Abschnitt~\ref{sec:1-einleitung} folgt
die Darstellung der wichtigsten Grundlagen in Abschnitt~\ref{sec:2-grundlagen}.
Aufbauend auf diesen Grundlagen folgt die konzeptuelle Ausarbeitung im Abschnitt~\ref{sec:3-konzeption}.
Die Umsetzung wird im Abschnitt~\ref{sec:4-implementierung} erläutert.
Die Leistungsfähigkeit der Implementierung wird in Abschnitt~\ref{sec:5-tests} untersucht.
Eine Zusammenfassung und ein Ausblick schließen die Arbeit in
Abschnitt~\ref{sec:6-fazit} ab. Eventuell noch benötigte Anhänge
finden sich in den Anhängen [...] bis [...].

% % % Imports nur für Referenzenauflösung während des Schreibens! Vorm Kompilieren auskommentieren!
% \bibliography{0_hauptdatei}
% \input{1_einleitung}
% \input{2_grundlagen}
% \input{3_konzeption}
% \input{4_implementierung}
% \input{5_tests}
% \input{6_zusammenfassung}
% % Ende Imports

\section{Grundlagen%
  \label{sec:2-grundlagen}}

\subsection{Speicherprogrammierbare-Steuerung und Linux -- Revolution Pi%
     \label{sec:2-sps}}

\subsubsection{Kunbus RevolutionPi%
        \label{sec:2-revpi}}
Der RevolutionPi 3 ist eine speicherprogrammierbare Steuerung (SPS) des Herstellers
Kunbus GmbH. Kern dieser SPS ist das von der Raspberry Pi Foundation entwickelte
und vertriebene Raspberry Pi Compute Module 3. Dieses integriert ein Broadcom BCM2837
System-on-Chip (SoC) mit vier 1,2GHz Prozessorkernen, 1GB RAM, 4GB eMMC Anwendungsspeicher
und sonstige Peripherie in ein Modul im DDR2-SODIMM Formfaktor. Diese Spezifikationen
sind weitgehend identisch zu denen des ausgesprochen populären Raspberry Pi 3.
Der Revolution Pi profitiert daher von dem gleichen großen Angebot an Software
und Unterstützung wie der Raspberry Pi, ergänzt dessen Hardware jedoch um eine 24V
Spannungsversorgung, die Möglichkeit der Erweiterung durch mehrere industrietaugliche
Ein-/ Ausgabemodule und Gateways sowie ein Gehäuse zur Montage auf einer DIN-Schiene.
\begin{itemize}
  \item{Prozessor: BCM2837}
  \item{Taktfrequenz 1,2 GHz}
  \item{Anzahl Prozessorkerne: 4}
  \item{Arbeitsspeicher: 1 GByte}
  \item{eMMC Flash Speicher: 4 GByte}
  \item{Betriebssystem: Angepasstes Raspbian mit RT-Patch}
  \item{RTC mit 24h Pufferung über wartungsfreien Kondensator}
  \item{Treiber / API: Treiber schreibt zyklisch Prozessdaten in ein Prozessabbild, Zugriff auf Prozessabbild über Linux-Filesystem als API zu Fremdsoftware.}
  \item{Kommunikationsanschlüsse: 2 x USB 2.0 A (je 500 mA belastbar), 1 x Micro-USB, HDMI, Ethernet (RJ45) 10/100 Mbit/s}
  \item{Stromversorgung: min. 10,7 V, max. 28,8 V, maximal 10 Watt}
  \item{Zulässige Umgebungstemperatur: -40 bis +55 C}
  \item{Gehäuseabmessungen: (HxBxL) 96 mm x 22,5 mm x 110,5 mm (ohne gesteckte Stecker)}
  \item{ESD Schutz: 4 kV / 8 kV gemäß EN61131-2 und IEC 61000-6-2}
  \item{Surge / Burst Prüfungen: gemäß EN61131-2 und IEC 61000-6-2 eingekoppelt auf Versorgungsspannung, Ethernet und IO-Leitungen}
  \item{EMI Prüfungen: gemäß EN61131-2 und IEC 61000-6-2}
\end{itemize}

Kunbus bietet eine Auswahl an IO- und Gateway-Modulen zur Erweiterung des Revolution Pi an.
Gateways dienen der Kommunikation mit Systemen oder Komponenten der Automatisierungstechnik
über Protokolle wie PROFIBUS oder EtherCAT. IO-Module erlauben die Überwachung
und Steuerung von digitalen oder analogen Ein- und Ausgängen.

\subsubsection{Zugriff auf IO-Module%
        \label{sec:2-io}}
Der Zugriff auf die Ein- und Ausgänge der IO-Module erfolgt über ein Prozessabbild
und einen hierfür von Kunbus bereitgestellten Treiber, genannt piControl. Dieser
aktualisiert das Prozessabbild zyklisch. Die angestrebte Zykluszeit beträgt 5ms,
kann jedoch je nach Anzahl der angeschlossenen Module auch größer sein. Kunbus
garantiert bei drei IO-Modulen und zwei Gateway-Modulen eine Zykluszeit von 10 ms.
Jedes der IO-Module stellt ein eigenständiges eingebettetes System dar. Es verfügt
über einen Microcontroller, welcher die IOs bereitstellt und über einen RS485-Bus
mit dem Revolution Pi kommuniziert.
% https://revolution.kunbus.de/io-modul/

Lizenz: GPL
% https://github.com/RevolutionPi/piControl

\begin{lstlisting}[language={c},firstnumber={226},caption={Setzen der Scheduler-Priorität auf SCHED\_FIFO in revpi\_common.c\label{lst:2-sched_priority}}]
param.sched_priority = ktprio->prio;
ret = sched_setscheduler(child, SCHED_FIFO,
       &param);
\end{lstlisting}


\subsection{Echtzeit und Multithreading unter Linux -- preemptRT und posix%
     \label{sec:2-echtzeit}}


 Der Linux-Kernel verfügt über mehrere unterschiedliche Preemtion-Modelle:

\begin{itemize}
  \item No Forced Preemption (server):
  Ausgelegt auf maximal möglichen Durchsatz, lediglich Interrupts und
  System-Call-Returns bewirken Präemption.

  \item Voluntary Kernel Preemption (Desktop):
  Neben den implizit bevorrechtigten Interrupts und System-Call-Returns gibt es
  in diesem Modell weitere Abschnitte des Kernels in welchen Preämption explizit
  gestattet ist.

  \item Preemptible Kernel (Low-Latency Desktop):
  In diesem Modell ist der gesamte Kernel, mit Ausnahme sog.~kritischer Abschnitte
  präemptible. Nach jedem kritischen Abschnitt gibt es einen impliziten Präemptions-Punkt.

  \item Preemptible Kernel (Basic RT):
  Dieses Modell ist dem zuvor genannten sehr ähnlich, hier sind jedoch alle Interrupt-Handler
  als eigenständige Threads ausgeführt.

  \item Fully Preemptible Kernel (RT):
  Wie auch bei den beiden zuvor genannten Modellen ist hier der gesamte Kernel
  präemtible, die Anzahl und Dauer der nicht-präemtiblen kritischen Abschnitte
  ist auf ein notwendiges Minimum beschränkt. Alle Interrupt-Handler sind als
  eigenständige Threads ausgeführt, Spinlocks durch Sleeping-Spinlocks und Mutexe
  durch sog.~RT-Mutexe ersetzt.

\end{itemize}
\todo{Spinlocks und Mutexe sowie die RT-Varianten dieser erklären!}

Lediglich mit dem vollständig präemtiblen Kernel kann Echtzeit-Verhalten realisiert werden.

% https://wiki.linuxfoundation.org/realtime/documentation/technical_basics/preemption_models bzw kernel/Kconfig.preempt

\subsubsection{preemptRT%
        \label{sec:2-preemptRT}}
% https://wiki.linuxfoundation.org/realtime/documentation/technical_details/start
% https://wiki.linuxfoundation.org/realtime/documentation/technical_basics/start

Das dem PREEMPT RT Kernel zugrunde liegende Prinzip lässt sich in einer einfachen
Regel ausdrücken: Nur Code, welcher absolut nicht-präemtible sein darf, ist es
gestattet nicht-präemtible zu sein.
Das erklärte Ziel des PREEMPT\_RT Patches ist es folglich, die Menge des nicht-präemtiblen
Codes im Linux-Kernel auf das absolut notwendige Minimum zu reduzieren.

Dies wird durch Verwendung folgender Mechanismen erreicht:

\begin{itemize}
  \item Hochauflösende Timer
  \item Sleeping Spinlocks
  \item Threaded Interrupt Handlers
  \item rt\_mutex
  \item RCU
\end{itemize}


\subsubsection{posix%
        \label{sec:2-posix}}
Ist posix hier wirklich relevant? Debian bzw.~Raspbian sind weitgehend posix
kompatibel, aber wird es hier genutzt? -> JA, open62541 nutzt pthread.h
piControl nutzt kthread.h, und semaphore.h

\subsection{OPC-UA und open62541%
     \label{sec:2-opc}}

\subsubsection{OPC UA%
        \label{sec:2-opcua}}
Open Platform Communications (OPC) ist eine Familie von Standards zur herstellerunabhängigen
Kommunikation von Maschinen (M2M) in der Automatisierungstechnik. Die sog.~OPC Task Force, zu deren
Mitgliedern verschiedene große Firmen der Automatisierungsindustrie gehören, veröffentlichte
die OPC Specification Version 1.0 im August 1996.
Motiviert ist dieser offene Standard durch die Erkenntniss, dass die Anpassung der
zahlreichen Herstellerstandards an individuelle Infrastrukturen und Anlagen einen
großen Mehraufwand verursachen.
Die Wikipedia beschreibt das Anwendungsgebiet für OPC wie folgt:

\glqq{}OPC wird dort eingesetzt, wo Sensoren, Regler und Steuerungen verschiedener Hersteller
ein gemeinsames Netzwerk bilden. Ohne OPC benötigten zwei Geräte zum Datenaustausch
genaue Kenntnis über die Kommunikationsmöglichkeiten des Gegenübers. Erweiterungen
und Austausch gestalten sich entsprechend schwierig. Mit OPC genügt es, für jedes
Gerät genau einmal einen OPC-konformen Treiber zu schreiben. Idealerweise wird
dieser bereits vom Hersteller zur Verfügung gestellt. Ein OPC-Treiber lässt sich
ohne großen Anpassungsaufwand in beliebig große Steuer- und Überwachungssysteme
integrieren.

OPC unterteilt sich in verschiedene Unterstandards, die für den jeweiligen Anwendungsfall
unabhängig voneinander implementiert werden können. OPC lässt sich damit verwenden
für Echtzeitdaten (Überwachung), Datenarchivierung, Alarm-Meldungen und neuerdings
auch direkt zur Steuerung (Befehlsübermittlung).\grqq{}

OPC basiert in der ursprünglichen Spezifikation auf Microsofts DCOM-Spezifikation.
DCOM macht Funktionen und Objekte einer Anwendung anderen Anwendungen im Netzwerk
zugänglich. Der OPC-Standard definiert entsprechende DCOM-Objekte um mit anderen
OPC-Anwendungen Daten austauschen zu können. Die Verwendung von DCOM bindet Anwender
an Betriebssysteme von Microsoft. Die ursprüngliche OPC Spezifikation wird durch die
Entwicklung von OPC Unified Architecture (OPC UA) abgelöst.
OPC UA setzt auf einem eigenen Kommunikationionsstack auf, die Verwendung von DCOM
und damit die Bindung an Microsoft wurden aufgelöst.

Die OPC-UA-Architektur ist eine Service-orientierte Architektur (SOA), deren Struktur
aus mehreren Schichten besteht.

% Wikipedia
Das OPC-Informationsmodell ist nicht mehr nur eine Hierarchie aus Ordnern, Items
und Properties. Es ist ein sogenanntes Full-Mesh-Network aus Nodes, mit dem neben
den Nutzdaten eines Nodes auch Meta- und Diagnoseinformationen repräsentiert werden.
Ein Node ähnelt einem Objekt aus der objektorientierten Programmierung. Ein Node
kann Attribute besitzen, die gelesen werden können (Data Access (DA), Historical
Data Access (HDA)). Es ist möglich Methoden zu definieren und aufzurufen.
Eine Methode besitzt Aufrufargumente und Rückgabewerte. Sie wird durch ein Command
aufgerufen. Weiterhin werden Events unterstützt, die versendet werden können
(AE (Alarms \& Events), DA DataChange), um bestimmte Informationen zwischen Geräten
auszutauschen. Ein Event besitzt unter anderem einen Empfangszeitpunkt, eine Nachricht
und einen Schweregrad. Die o. g. Nodes werden sowohl für die Nutzdaten als auch
alle anderen Arten von Metadaten verwendet. Der damit modellierte OPC-Adressraum
beinhaltet nun auch ein Typmodell, mit dem sämtliche Datentypen spezifiziert werden.

% https://de.wikipedia.org/wiki/Open_Platform_Communications
% https://de.wikipedia.org/wiki/OPC_Unified_Architecture
% https://opcfoundation.org/developer-tools/specifications-unified-architecture
% Von Gerhard Gappmeier - ascolab GmbH, CC BY-SA 3.0, https://de.wikipedia.org/w/index.php?curid=1892069
\subsubsection{open62541%
        \label{sec:2-open62541}}
open62541 ist eine offene und freie Implementierung von OPC UA. Die in C geschriebene
Bibliothek stellt eine beständig zunehmende Anzahl der im OPC UA Standard definierten
Funktionen bereit. Sie kann sowohl zur Erstellung von OPC-Servern als auch -Clients
genutzt werden. Ergänzend zu der unter der Mozilla Public License v2.0 lizensierten
Bibliothek stellt das open62541 Projekt auch Beispielprogramme unter einer CC0 Lizenz
zur Verfügung.

Die Bibliothek eignet sich auch für die Entwicklung auf eingebetteten Systemen und
Microcontrollern. Je nach Umfang der gewünschten Funktionen und des OPC Informationsmodells
beträgt die Größe einer Server-Binary weniger als 100kb. %evtl. kürzen?

\todo{Nodes erklären! Evtl.~oben!}

Folgende Auswahl an Eigenschaften und Funktionen zeichnet die in dieser Arbeit verwendete
Version 0.3 von open62541 aus:
\begin{itemize}
  \item Kommunikationionsstack
  \begin{itemize}
      \item OPC UA Binär-Protokoll (HTTP oder SOAP werden gegenwärtig nicht unterstützt)
      \item Austauschbare Netzwerk-Schicht, welche die Verwendung eigener Netzwerk-APIs
      erlaubt.
      \item Verschlüsselte Kommunikationion
      \item Asynchrone Dienst-Anfragen im Client
  \end{itemize}
  \item Informationsmodell
  \begin{itemize}
    \item Unterstützung aller OPC UA Node-Typen, inkl.~Methoden
    \item Hinzufügen und Entfernen von Nodes und Referenzen zur Laufzeit.
    \item Vererbung und Instanziierung von Objekt- und Variablentypen
    \item Zugriffskontrolle auch für einzelne Nodes
  \end{itemize}
  \item Subscriptions
  \begin{itemize}
    \item Erlaubt die Überwachung (subscriptions / monitoreditems)
    \item Sehr geringer Ressourcenbedarf pro überwachtem Wert
  \end{itemize}
  \item Code-Generierung auf XML-Basis
  \begin{itemize}
    \item Erlaubt die Erstellung von Datentypen
    \item Erlaubt die Generierung des serverseitigen Informationsmodells
  \end{itemize}
\end{itemize}

% https://open62541.org/doc/0.3/


Mozilla Public License
CC0 Lizenz für Beispiele und Plugins

% https://open62541.org/doc/open62541-current.pdf
% https://open62541.org/

% % % Imports nur für Referenzenauflösung während des Schreibens! Vorm Kompilieren auskommentieren!
% \bibliography{0_hauptdatei}
% \input{1_einleitung}
% \input{2_grundlagen}
% \input{3_konzeption}
% \input{4_implementierung}
% \input{5_tests}
% \input{6_zusammenfassung}
% \input{anhang}
% % Ende Imports

\section{Systemkonzept%
  \label{sec:3-konzeption}}
Auf Basis der in Abschnitt \ref{sec:2-grundlagen} vorgestellten Möglichkeiten folgt nun die Ausarbeitung eines Konzepts.
In den folgenden Abschnitten soll näher auf zwei zentrale Aspekte eingegangen werden: Abschnitt~\ref{sec:3-anbindung} stellt Möglichkeiten zum Zugriff auf Variablen bzw.\,Werte im Prozessabbild des Revolution Pi vor; in Abschnitt~\ref{sec:3-integration} wird ein Konzept zur Bereitstellung dieser Variablen auf einem OPC-Server vorgestellt.

\subsection{Anbindung der IO an den OPC-Server%
     \label{sec:3-anbindung}}

Eine Webanwendung mit Bezeichnung PiCtory dient zur Konfiguration der I/O- und virtuellen Module des RevolutionPi. Die Konfiguration liegt im JSON-Format in der Datei \lstinline{/etc/revpi/config.rsc}. Der piControl-Treiber liest diese Datei beim Start. 
Der folgende Auszug aus der Manpage des piControl-Kernelmoduls beschreibt die von diesem zum Lesen und Schreiben einzelner Bits des Prozessabbildes bereitgestellten Funktionen~\citep[vgl.]{web-revpi-manpage}. Sie ist an dieser Stelle weitgehend ungekürzt zitiert, da sie die nutzbare Schnittstelle sehr kompakt beschreibt.

\begin{lstlisting}[breakindent=0pt, numbers=none, caption={Auszug aus der Revolution Pi Programmers Manual\label{lst:4-manpage}}]
KB_FIND_VARIABLE SPIVariable *argp
Find a variable in the process image by its name. A pointer to a structure of type SPIVariable must be passed as argument. [...]
The struct SPIVariable [...] is defined as 
typedef struct SPIVariableStr
{
    char strVarName[32]; // Variable name
    uint16_t i16uAddress; // Address of the byte in the process image
    uint8_t i8uBit; // 0-7 bit position, >= 8 whole byte
    uint16_t i16uLength; // length of the variable in bits.
    // Possible values are 1, 8, 16 and 32
} SPIVariable;

Set and get values of the process image
KB_GET_VALUE SPIValue *argp
[...]
KB_SET_VALUE SPIValue *argp
Write one bit or one byte to the process image [...].  This call is more efficient than the usual calls of seek and write because only one function call is necessary. If more than on application are writing bits in one output byte, this call is the only safe way to set a bit without overwriting the other bits because this call is doing a read-modify-write-cycle. 

The struct SPIValue used by this ioctl is defined as
typedef struct SPIValueStr
{
    uint16_t i16uAddress; // Address of the byte in the process image
    uint8_t i8uBit; // 0-7 bit position, >= 8 whole byte
    uint8_t i8uValue; // Value: 0/1 for bit access, whole byte otherwise
} SPIValue;
\end{lstlisting} 

Die oben beschriebenden Funtkionen \lstinline{KB_FIND_VARIABLE}, \lstinline{KB_GET_VALUE} und \lstinline{KB_SET_VALUE} ermöglichen einen einfachen und (lt.\,Manpage) effizienten Zugriff auf einzelne Bits des Prozessabbildes und damit auch auf die IO des RevolutionPi.
Der Zugriff des OPC-Servers auf das Prozessabbild soll daher mittels dieser Funktionen realisiert werden.
\lstinline{KB_FIND_VARIABLE} kann genutzt werden, um Adressen von Variablen im Prozessabbild mittels ihres Namens aufzulösen.
\lstinline{KB_GET_VALUE} und \lstinline{KB_SET_VALUE} ermöglichen den Zugriff auf die Werte dieser Variablen.


\subsection{Integration des OPC-Servers in das System%
     \label{sec:3-integration}}

open62541 bietet drei Möglichkeiten zum Abgleich von Variablen mit dem Prozessabbild~\citep[vgl.][Tutorials - Connecting a Variable with a Physical Process]{web-open62541}:
\begin{itemize}
    \item Manuelles oder zyklisches Aktualisieren
    \item Variable Value Callback
    \item Variable Datasource
\end{itemize}

Die zyklische Aktualisierung eines oder mehrerer Werte nimmt, abhängig von der Zykluszeit, viele Systemressourcen in Anspruch. Value Callbacks ermöglichen es, einen Variablenwert effizienter mit einer Ressource wie etwa einem Prozessabbild zu synchronisieren. An die Variable wird ein Callback angehängt, welches vor jedem Lesen und nach jedem Schreibvorgang ausgeführt wird.
Der Wert der Variablen wird weiterhin im Variablenknoten auf dem OPC-Server gespeichert, der Abgleich mit der verknüpften Ressource erfolgt durch die Callback-Methoden.

Sogenannte Datenquellen gehen noch einen Schritt weiter. Der Server leitet jede Lese- und Schreibanforderung direkt an eine Callback-Funktion weiter. Beim Lesen liefert der Rückruf eine Kopie des aktuellen Wertes. Die Datenquelle muss intern ein eigenes Speichermanagement implementieren.

Der Zugriff auf die Werte des Prozessabbildes erfolgt, wie in Abschnitt~\ref{sec:3-anbindung} beschrieben, über von piControl bereitgestellte Methoden. Um die durch open62541 gepflegte OPC-Datenstruktur und das durch piControl verwaltete Prozessabbild möglichst effektiv verknüpfen zu können, soll diese Interaktion mittels Datenquellen und den zugehörigen Callbacks implementiert werden.
% % % Imports nur für Referenzenauflösung während des Schreibens! Vorm Kompilieren auskommentieren!
% \bibliography{0_hauptdatei}
% \input{1_einleitung}
% \input{2_grundlagen}
% \input{3_konzeption}
% \input{4_implementierung}
% \input{5_tests}
% \input{6_zusammenfassung}
% \input{anhang}
% % Ende Imports

\section{Implementierung%
  \label{sec:4-implementierung}}
Das folgende Kapitel stellt in Auszügen die Implementierung des OPC-Servers sowie die Anbindung an die IO-Module
der SPS dar. Der Schwerpunkt liegt hierbei auf der Funktionsweise des piControl-Treibers und dessen Integration in das Projekt. Abschnitt~\ref{sec:4-picontrol} erklärt die zum Schreibens eines Bits verwendeten Funktionsaufrufe.
Zuvor soll jedoch in Abschnitt~\ref{sec:4-open62541} der Teil des OPC-Servers vorgestellt werden, welcher auf besagten Treiber zugreift. 

\subsection{Implementierung des OPC-Servers%
     \label{sec:4-open62541}}
Wie im vorangegangenen Abschnitt~\ref{sec:3-integration} begründet, soll die Verknüpfung zwischen dem Prozessabbild der SPS und den auf dem OPC-Server bereitgestellten Werten über sog.\,Datenquellen erfolgen. Hierzu ist zunächst eine Callback-Methode zu implementieren, welche bei einem Lese- oder Schreibzugriff auf eine Variable aufgerufen wird. Die Verknüpfung zwischen Callback-Methode und Variable muss manuell erfolgen.

\begin{lstlisting}[language={c},firstnumber=237,caption={Auszug der Methode \lstinline{linkDataSourceVariable} in \lstinline{variables.c}\label{lst:4-linkDataSourceVariable}}]
extern UA_StatusCode
 linkDataSourceVariable(UA_Server *server, UA_NodeId nodeId) {
     bool readonly = false;
     UA_DataSource dataSourceVariable;
     UA_StatusCode rc; |>\setcounter{lstnumber}{254}<|

     dataSourceVariable.read = readDataSourceVariable;
     if (!readonly)
        dataSourceVariable.write = writeDataSourceVariable;
     else
        dataSourceVariable.write = writeReadonlyDataSourceVariable;

     return UA_Server_setVariableNode_dataSource(server, nodeId, dataSourceVariable);
 }
\end{lstlisting}

\begin{figure}[h]
    \centering
    \includegraphics[width=0.42\textwidth]{doc/img/OPC_RevPiDO.pdf}
    \caption{Auszug des verwendeten Nodesets, hier Digitalausgang 1 des Versuchsaufbaus
      \label{fig:opc-do}}
\end{figure}

Die in Listing~\ref{lst:4-linkDataSourceVariable} abgebildete Methode \lstinline{linkDataSourceVariable()} erzeugt ein Struct vom Typ \lstinline{UA_DataSource}. In diesem werden dem Lesen und Schreiben einer OPC-Variablen entsprechende Callback-Methoden zugewiesen. Die Verknüpfung einer OPC-Variable, genauer ihrer NodeId, mit der zuvor definierten Datenquelle erfolgt über die von open62541 bereitgestellte Methode \lstinline{UA_Server_setVariableNode_dataSource()}. Vor dem Lesen und nach dem Schreiben dieser Variable werden von nun an die entsprechenden Callbacks aufgerufen.
     
\begin{lstlisting}[language={c},firstnumber=168,caption={Auszug des Callbacks \lstinline{writeDataSourceVariable} in \lstinline{variables.c}\label{lst:4-writeDataSourceVariable}}]  
extern UA_StatusCode
 writeDataSourceVariable(UA_Server *server,
            const UA_NodeId *sessionId, void *sessionContext,
            const UA_NodeId *nodeId, void *nodeContext,
            const UA_NumericRange *range, const UA_DataValue *dataValue) {

    UA_StatusCode retval  = UA_STATUSCODE_GOOD;
    UA_NodeId *nameNodeId = UA_malloc(sizeof(UA_NodeId));
    UA_QualifiedName nameQN = UA_QUALIFIEDNAME(1, "Name");
    UA_Variant nameVar;
    UA_Boolean bit;

    retval |= findSiblingByBrowsename(server, nodeId, &nameQN, nameNodeId);
    retval |= UA_Server_readValue(server, *nameNodeId, &nameVar);
    retval |= UA_Boolean_copy(dataValue->value.data, &bit);

    |>\tikzmarkin[set border color=martinired]{writeIO}<|PI_writeSingleIO(String_fromUA_String(nameVar.data), &bit, false);                                                 |>\tikzmarkend{writeIO}<|

    free(nameNodeId);
    return retval;
 }
\end{lstlisting}

Listing~\ref{lst:4-writeDataSourceVariable} zeigt die Callback-Methode, welche nach dem Schreiben einer Variablen auf dem OPC-Server aufgerufen wird.
Dieser Methode wird neben der NodeId der mit ihr verknüpften Variablen auch der Wert dieser in Form eines Zeigers auf ein Struct vom Typ \lstinline{UA_DataValue} übergeben.

Die Gestaltung des hier verwendeten Nodesets sieht vor, dass in einer OPC-Variablen \lstinline{"Name"} der Bezeichner des zu schreibenden Digitalausgangs hinterlegt ist, siehe Abbildung~\ref{fig:opc-do}. Dies erlaubt eine Rekonfiguration der Ein- und Ausgänge der SPS ohne Änderungen im Programmcode des OPC-Servers vornehmen zu müssen.
Es ist daher erforderlich, nach jedem Schreiben einer mit einem Digitalausgang verknüpften Variablen, hier \lstinline{"Value"}, dessen Bezeichner \lstinline{"Name"} abzufragen. 
Dies geschieht in den Zeilen 180 und 181.
Anschließend wird dieser Bezeichner sowie der zu schreibende Wert der Methode \lstinline{PI_writeSingleIO()} übergeben, welche wiederum die Interaktion mit piControl übernimmt (vgl. Abschnitt \ref{sec:4-picontrol}).
 
\subsection{Integration von piControl%
     \label{sec:4-picontrol}}
In Abschnitt~\ref{sec:2-io} wurde die Anbindung der IO-Module des Revolution Pi sowie die Funktionsweise von piControl aus Anwendersicht beschrieben. Die verfügbare Literatur beschränkt sich auch auf lediglich diese Sicht; eine weiterführende Dokumentation für Entwickler gibt es, neben der in Abschnitt~\ref{sec:3-anbindung} vorgestellten Manpage, nicht. 
In diesem Abschnitt soll daher der Quellcode von piControl sowie dessen Verwendung im Projekt genauer betrachtet werden.
Hierzu wird exemplarisch die in Abschnitt~\ref{sec:4-open62541} eingeführte Methode \lstinline{PI_writeSingleIO()} untersucht.
Diese Methode ermöglicht das Setzen eines einzelnen Bits im Prozessabbild der SPS, und damit das Schalten eines digitalen Ausgangs auf einem IO-Modul.
Die äquivalente Methode \lstinline{int piControlGetBitValue(SPIValue *pSpiValue)} zum Lesen eines Bits bzw. Eingangs funktioniert analog und soll daher an dieser Stelle nicht dediziert erörtert werden.

\begin{lstlisting}[language={c},firstnumber=97,
                   caption={Setzen eines phsikalischen, digitalen Ausgangs in \lstinline{revpi.c}
                   \label{lst:4-PI_writeSingleIO}}]
extern void PI_writeSingleIO(char *pszVariableName, bool *bit, bool verbose)
{
	int rc;
	SPIVariable sPiVariable;
	SPIValue sPIValue;

	strncpy(sPiVariable.strVarName, pszVariableName, sizeof(sPiVariable.strVarName));
	rc = piControlGetVariableInfo(&sPiVariable);
	if (rc < 0) {
		printf("Cannot find variable '%s'\n", pszVariableName);
		return;
	}

		sPIValue.i16uAddress = sPiVariable.i16uAddress;
		sPIValue.i8uBit = sPiVariable.i8uBit;
		sPIValue.i8uValue = *bit;
		rc = |>\tikzmarkin[set border color=martinired]{setBitValue}<|piControlSetBitValue(&sPIValue)|>\tikzmarkend{setBitValue}<|;
		if (rc < 0)
			printf("Set bit error %s\n", getWriteError(rc));
		else if (verbose)
			printf("Set bit %d on byte at offset %d. Value %d\n", sPIValue.i8uBit, sPIValue.i16uAddress,
			       sPIValue.i8uValue);
}
\end{lstlisting}

Der Programmcode in Listing~\ref{lst:4-PI_writeSingleIO} ist Teil des implementierten OPC-Servers. In diesem wird auf zwei Funktionen des piControl-Treibers zugegriffen. 
Beiden Methoden wird als Argument ein Zeiger auf ein Struct vom Typ \lstinline{SPIValue} übergeben. Der im Struct abgelegte Name wird mittels \lstinline{piControlGetVariableInfo(&sPIValue)} zu einer Adresse im Prozessabbild aufgelöst. Diese wird in \lstinline{sPIValue.i16uAdress} gespeichert. Der Wert der Variablen wird anschließend mittels \lstinline{piControlSetBitValue(&sPIValue)} an dieser Adresse in das Prozessabbild geschrieben.

\begin{lstlisting}[language={c},firstnumber=309,caption={Methode \lstinline{piControlSetBitValue} in \lstinline{piControlIf.c}\label{lst:4-piControlSetBitValue}}]
int |>\tikzmarkin[set border color=martiniblue]{setBitValueFcn}<|piControlSetBitValue(SPIValue *pSpiValue)|>\tikzmarkend{setBitValueFcn}<|
{
    piControlOpen();

    if (PiControlHandle_g < 0)
	    return -ENODEV;

    pSpiValue->i16uAddress += pSpiValue->i8uBit / 8;
    pSpiValue->i8uBit %= 8;

    if (|>\tikzmarkin[set border color=martinired]{ioctl}<|ioctl(PiControlHandle_g, KB_SET_VALUE, pSpiValue)|>\tikzmarkend{ioctl}<| < 0)
	    return errno;

    return 0;
}
\end{lstlisting}

Die in Listing~\ref{lst:4-piControlSetBitValue} dargestellte Methode \lstinline{piControlSetBitValue} ist lediglich eine Hüllfunktion (häufig auch als Wrapper-Funktion bezeichnet) für einen Aufruf des \lstinline{ioctl} Kernel-Moduls.
Folgende Parameter werden übergeben:
\lstinline{PiControlHandle_g} ist die Referenz auf die Geräte-Datei des piControl-Treibers. \lstinline{KB_SET_VALUE} ist das ioctl-Kommando zum Schreiben eines Bits in das Prozessabbild. Der Zeiger \lstinline{pSpiValue} verweist auf ein Struct des bereits vorgestellten Typs \lstinline{SPIValue}.

\begin{lstlisting}[language={c},firstnumber=80,caption={Methode \lstinline{piControlOpen} in \lstinline{piControlIf.c}\label{lst:4-piControlOpen}}]
void piControlOpen(void)
{
    /* open handle if needed */
    if (PiControlHandle_g < 0)
    {
	    |>\tikzmarkin[set border color=martiniblue]{PiControlHandle}<|PiControlHandle_g = open(PICONTROL_DEVICE, O_RDWR)|>\tikzmarkend{PiControlHandle}<|;
    }
}
\end{lstlisting}

Die in Listing~\ref{lst:4-piControlOpen} dargestellte Methode öffnet, sofern nicht bereits geschehen, die Geräte-Datei. Das Macro \lstinline{PICONTROL_DEVICE} verweist hierbei auf \lstinline{/dev/piControl0}.

\begin{lstlisting}[language={c},firstnumber=721,caption={Methode \lstinline{piControlIoctl} in \lstinline{piControlMain.c}\label{lst:4-piControlIoctl}}]
static long |>\tikzmarkin[set border color=martiniblue, below offset=0.9em]{piControlIoctl}<|piControlIoctl(struct file *file, unsigned int prg_nr, 
                           unsigned long usr_addr)                                      |>\tikzmarkend{piControlIoctl}<|
{
  int status = -EFAULT;
  tpiControlInst *priv;
  int timeout = 10000;	// ms

  if (prg_nr != KB_CONFIG_SEND && prg_nr != KB_CONFIG_START && !isRunning()) {
  	return -EAGAIN;
  }

  priv = (tpiControlInst *) file->private_data;

  if (prg_nr != KB_GET_LAST_MESSAGE) {
  	// clear old message
  	priv->pcErrorMessage[0] = 0;
  }

  switch (prg_nr) {|>\setcounter{lstnumber}{864}<|

    case |>\tikzmarkin[set border color=martiniblue]{KB_SET_VALUE}<|KB_SET_VALUE:|>\tikzmarkend{KB_SET_VALUE}<|
  		{
  			SPIValue *pValue = (SPIValue *) usr_addr;

  			if (!isRunning())
  				return -EFAULT;

  			if (pValue->i16uAddress >= KB_PI_LEN) {
  				status = -EFAULT;
  			} else {
  				INT8U i8uValue_l;
  				my_rt_mutex_lock(&piDev_g.lockPI);
  				i8uValue_l = piDev_g.ai8uPI[pValue->i16uAddress];

  				if (pValue->i8uBit >= 8) {
  					i8uValue_l = pValue->i8uValue;
  				} else {
  					if (pValue->i8uValue)
  						i8uValue_l |= (1 << pValue->i8uBit);
  					else
  						i8uValue_l &= ~(1 << pValue->i8uBit);
  				}

  				|>\tikzmarkin[set border color=martinired]{i8uValue}<|piDev_g.ai8uPI[pValue->i16uAddress] = i8uValue_l;|>\tikzmarkend{i8uValue}<|
  				rt_mutex_unlock(&piDev_g.lockPI);

  #ifdef VERBOSE
  				pr_info("piControlIoctl Addr=%u, bit=%u: %02x %02x\n", pValue->i16uAddress, pValue->i8uBit, pValue->i8uValue, i8uValue_l);
  #endif

  				status = 0;
  			}
  		}
  		break; |>\setcounter{lstnumber}{1314}<|

    default:
      pr_err("Invalid Ioctl");
      return (-EINVAL);
      break;

    }

    return status;
  }
\end{lstlisting}

Listing~\ref{lst:4-piControlIoctl} zeigt in Auszügen die ioctl-Methode des piControl Kernel-Treibers. Diese bekommt folgende Argumente übergeben: \lstinline{struct file *file} enthält den Verweis auf die Geräte-Datei, hier \lstinline{/dev/piControl0}. Der Wert von \lstinline{unsigned int prg_nr} beschreibt die Anfrage an den Treiber, in diesem Fall \lstinline{KB_SET_VALUE}. Das Argument \lstinline{unsigned long usr_addr} enthält einen typ-agnostischen Pointer. Dieser verweist auf einen Speicherbereich, in welchem die zur Bearbeitung der Anfrage notwendigen Daten abgelegt sind. Hier können auch vom Treiber empfangene Daten dem Anwendungsprogramm bereitgestellt werden. 

Die switch-case-Anweisung führt die über das Argument \lstinline{prg_nr} spezifizierte Aktion aus. Hier betrachten wir \lstinline{KB_SET_VALUE}:
Zunächst wird in Zeile 868 der übergebene Zeiger \lstinline{usr_addr} mittels explizitem Typecast zu einem Zeiger des Typs \lstinline{SPIValue *} konvertiert. Da dieser auf Daten im Userspace verweist, ist beim Zugriff durch den Kernel-Treiber besondere Vorsicht geboten.
In Zeile 877 wird mittels Mutex das Prozessabbild \lstinline{piDev_g} für den Zugriff durch andere Threads oder Prozesse gesperrt.
\lstinline{my_rt_mutex_lock} verweist hierbei auf die Funktion \lstinline{rt_mutex_lock} aus \lstinline{linux/sched.h}\footnote{Offenbar wurde hier auch eine alternative Implementierung vorgesehen, siehe revpi\_common.h}

In Zeile 889 wird das Byte \lstinline{i8uValue_l}, welches den zu schreibenden Wert enthält in das Prozessabbild übertragen. Anschließend wird die Mutex auf \lstinline{piDev_g} wieder entsperrt.
\newpage

\begin{lstlisting}[language={c},firstnumber=62,caption={Auszug des Struct \lstinline{spiControlDev} in \lstinline{piControlMain.h}\label{lst:4-spiControlDev}}]
|>\tikzmarkin[set border color=martiniblue]{spiControlDev}<|typedef struct spiControlDev|>\tikzmarkend{spiControlDev}<| {
	// device driver stuff
	int init_step;
	enum revpi_machine machine_type;
	void *machine;
	struct cdev cdev;	// Char device structure
	struct device *dev;
	struct thermal_zone_device *thermal_zone;

	|>\tikzmarkin[set border color=martiniblue]{processImage}<|// process image stuff
	INT8U ai8uPI[KB_PI_LEN];
	INT8U ai8uPIDefault|>\tikzmarkin[set border color=martinired]{KB_PI_LEN_0}<|[KB_PI_LEN]|>\tikzmarkend{KB_PI_LEN_0}<|;
	struct rt_mutex lockPI;        |>\tikzmarkend{processImage}<|
	bool stopIO;
	piDevices *devs; |>\setcounter{lstnumber}{94}<|
} tpiControlDev;
\end{lstlisting}

Das Prozessabbild ist als Byte-Array der Länge \lstinline{KB_PI_LEN} in Listing~\ref{lst:4-spiControlDev} definiert. Konfigurationsparameter wie \lstinline{KB_PI_LEN} oder die Zykluszeit für den Datenaustausch zwischen SPS und IO-Modulen sind im folgenden Listing~\ref{lst:4-process} definiert.

\begin{lstlisting}[language={c},firstnumber=119,caption={Konfigurationsparameter des Prozessabbildes in project.h\label{lst:4-process}}]
#define INTERVAL_PI_GATE (5*1000*1000)  // 5 ms piGateCommunication |>\setcounter{lstnumber}{128}<|

#define INTERVAL_IO_COM (5*1000*1000)  // 5 ms piIoComm |>\setcounter{lstnumber}{132}<|

#define KB_PD_LEN       512
|>\tikzmarkin[set border color=martiniblue]{KB_PI_LEN_1}<|#define KB_PI_LEN       4096|>\tikzmarkend{KB_PI_LEN_1}<|
\end{lstlisting}

Das zu setzende Bit wurde zu diesem Zeitpunkt erfolgreich in das Prozessabbild der SPS geschrieben.
Es stellt sich die Frage, wie dieses nun an das IO-Modul kommuniziert wird.
Die Kommunikation mit allen angebundenen Modulen ist ebenfalls Aufgabe des piControl-Treibers.

\begin{lstlisting}[language={c},firstnumber=256,caption={Auszug der Methode \lstinline{piIoThread} in \lstinline{revpi_core.c}\label{lst:4-piIoThread}}]
static int piIoThread(void *data)
{
	//TODO int value = 0;
	ktime_t time;
	ktime_t now;
	s64 tDiff;

	hrtimer_init(&piCore_g.ioTimer, CLOCK_MONOTONIC, HRTIMER_MODE_ABS);
	piCore_g.ioTimer.function = piIoTimer;

	pr_info("piIO thread started\n");

	now = hrtimer_cb_get_time(&piCore_g.ioTimer);

	PiBridgeMaster_Reset();

	while (!kthread_should_stop()) {
		if (|>\tikzmarkin[set border color=martinired]{PiBridgeMaster}<|PiBridgeMaster_Run()|>\tikzmarkend{PiBridgeMaster}<| < 0)
			break;
	}

	RevPiDevice_finish();

	pr_info("piIO exit\n");
	return 0;
}
\end{lstlisting}

Der Kernel-Thread \lstinline{piIoThread} ist verantwortlich für den zyklischen Datenaustausch mit den IO-Modulen. In diesem wird fortlaufend die Methode \lstinline{PiBridgeMaster_Run()} aufgerufen, siehe Listing~\ref{lst:4-piIoThread}.

\begin{lstlisting}[language={c},firstnumber=262,caption={Auszug der Methode \lstinline{PiBridgeMaster_Run(void)} in \lstinline{RevPiDevice.c}\label{lst:4-PiBridgeMaster_Run}}]
int PiBridgeMaster_Run(void)
{
	static kbUT_Timer tTimeoutTimer_s;
	static kbUT_Timer tConfigTimeoutTimer_s;
	static int error_cnt;
	static INT8U last_led;
	static unsigned long last_update;
	int ret = 0;
	int i;

	my_rt_mutex_lock(&piCore_g.lockBridgeState);
	if (piCore_g.eBridgeState != piBridgeStop) {
		switch (eRunStatus_s) { |>\setcounter{lstnumber}{514}<|
		    case enPiBridgeMasterStatus_EndOfConfig:|>\setcounter{lstnumber}{621}<|
		    if (|>\tikzmarkin[set border color=martinired]{RevPiDevice}<|RevPiDevice_run()|>\tikzmarkend{RevPiDevice}<|) {
				// an error occured, check error limits |>\setcounter{lstnumber}{641}<|
			} else {
				ret = 1;
			}
			piCore_g.image.drv.i16uRS485ErrorCnt = RevPiDevice_getErrCnt();
			break;
\end{lstlisting}

Die in Listing~\ref{lst:4-PiBridgeMaster_Run} dargestellte Methode ist eine sog. State-Machine. Ist die Konfiguration der IO-Module erfolgreich abgeschlossen, so führt sie bei Aufruf lediglich die Methode \lstinline{RevPiDevice_run()} aus.

\begin{lstlisting}[language={c},firstnumber=140,caption={Auszug der Methode \lstinline{RevPiDevice_run(void)} in \lstinline{RevPiDevice.c}\label{lst:4-RevPiDevice_run}}]
int RevPiDevice_run(void)
{
	INT8U i8uDevice = 0;
	INT32U r;
	int retval = 0;

	RevPiDevices_s.i16uErrorCnt = 0;

	for (i8uDevice = 0; i8uDevice < RevPiDevice_getDevCnt(); i8uDevice++) {
		if (RevPiDevice_getDev(i8uDevice)->i8uActive) {
			switch (RevPiDevice_getDev(i8uDevice)->sId.i16uModulType) {
			case KUNBUS_FW_DESCR_TYP_PI_DIO_14:
			case KUNBUS_FW_DESCR_TYP_PI_DI_16:
			case KUNBUS_FW_DESCR_TYP_PI_DO_16:
				r = |>\tikzmarkin[set border color=martinired]{sendCyclicTelegram}<|piDIOComm_sendCyclicTelegram(i8uDevice)|>\tikzmarkend{sendCyclicTelegram}\setcounter{lstnumber}{166} <|;

				break; |>\setcounter{lstnumber}{216}<|
			}
		}
	} |>\setcounter{lstnumber}{227}<|
	return retval;
}
\end{lstlisting}

Diese iteriert wie in Listing~\ref{lst:4-RevPiDevice_run} abgebildete durch alle gegenwärtig in der SPS konfigurierten Module. Ist das aktuelle Modul als aktiv markiert, so wird anhand eines sog. Firmware-Descriptors entschieden, welche Methode für die Ansteuerung des Moduls aufzurufen ist.

\begin{lstlisting}[language={c},firstnumber=161,caption={Auszug der Methode \lstinline{piDIOComm_sendCyclicTelegram} in \lstinline{piDIOComm.c}\label{lst:4-sendCyclicTelegram}}]
INT32U piDIOComm_sendCyclicTelegram(INT8U i8uDevice_p)
{
	INT32U i32uRv_l = 0;
	SIOGeneric sRequest_l;
	SIOGeneric sResponse_l;
	INT8U len_l, data_out[18], i, p, data_in[70];
	INT8U i8uAddress;
	int ret; |>\setcounter{lstnumber}{239}<|
	
    |>\tikzmarkin[set border color=martinired]{piIoComm}<|ret = piIoComm_send((INT8U *) & sRequest_l, IOPROTOCOL_HEADER_LENGTH + len_l + 1);  |>\tikzmarkend{piIoComm}\setcounter{lstnumber}{298}<|
}
\end{lstlisting}

Im Falle des hier verwendeten DO-Moduls wird die in Listing~\ref{lst:4-sendCyclicTelegram} abgebildete Methode \lstinline{piDIOComm_sendCyclicTelegram()} aufgerufen. Dieser wird ein Zeiger auf das zu schreibende Gerät übergeben. 
Zunächst wird das Prozessabbild mittels eines proprietären, jedoch im Quellcode offen nachvollziehbaren Protokolls in ein \lstinline{sRequest_l} genanntes Byte-Array umgewandelt. Dieser Schritt ist in Listing~\ref{lst:4-sendCyclicTelegram} nicht abgebildet. Anschließend wird \lstinline{piIoComm_send()} ein Zeiger auf die so generierte Schreib-Anfrage übergeben.

\begin{lstlisting}[language={c},firstnumber=220,caption={Auszug der Methode \lstinline{piIOComm_send} in \lstinline{piIOComm.c}\label{lst:4-piIOComm_send}}]
int piIoComm_send(INT8U * buf_p, INT16U i16uLen_p)
{
	ssize_t write_l = 0;
	INT16U i16uSent_l = 0;|>\setcounter{lstnumber}{249}<|

	while (i16uSent_l < i16uLen_p) {
		write_l = vfs_write(piIoComm_fd_m, buf_p + i16uSent_l, i16uLen_p - i16uSent_l, &piIoComm_fd_m->f_pos);
		if (write_l < 0) {
			pr_info_serial("write error %d\n", (int)write_l);
			return -1;
		} 
		i16uSent_l += write_l;|>\setcounter{lstnumber}{263}<|
	}
	clear();
	vfs_fsync(piIoComm_fd_m, 1);
	return 0;
}
\end{lstlisting}

Listing~\ref{lst:4-piIOComm_send} zeigt die Implementierung von \lstinline{piIoComm_send()}. Diese Methode ist für das Schreiben der oben generierten Anfrage auf die seriellen Schnittstelle verantwortlich. Realisiert wird dies mittels der Methode \lstinline{vfs_write()}. Diese ist in \lstinline{<linux/fs.h>} definiert. Sie ermöglicht das Schreiben einer Datei im Userspace aus dem Kernel heraus. Geschrieben wird hier die Datei mit dem Deskriptor \lstinline{piIoComm_fd_m}.
Da die Funktion \lstinline{vfs_write()} durch andere Kernel-Tasks unterbrochen werden kann, ist nicht gewährleistet, dass die gesamte Anfrage mit nur einem Aufruf geschrieben wird. Die oben abgebildete while-Schleife stellt das vollständige Senden der Anfrage sicher.

\begin{lstlisting}[language={c},firstnumber=157,caption={Auszug der Methode \lstinline{piIOComm_open_serial} in \lstinline{piIOComm.c}\label{lst:4-piIOComm_open_serial}}]
int piIoComm_open_serial(void)
{   |>\setcounter{lstnumber}{167}<|
	struct file *fd;	/* Filedeskriptor */
	struct termios newtio;	/* Schnittstellenoptionen */

	|>\tikzmarkin[set border color=martiniblue]{fd}<|/* Port oeffnen - read/write, kein "controlling tty", 
	    Status von DCD ignorieren */
	fd = filp_open(|>\tikzmarkin[set border color=martinired]{tty}<|REV_PI_TTY_DEVICE|>\tikzmarkend{tty}<|, O_RDWR | O_NOCTTY, 0); |>\setcounter{lstnumber}{208}<|
	
	piIoComm_fd_m = fd;                                                      |>\tikzmarkend{fd}\setcounter{lstnumber}{217}<|

	return 0;
}
\end{lstlisting}

Der zum Schreiben auf die serielle Schnittstelle verwendete Datei-Deskriptor wird von der in Listing~\ref{lst:4-piIOComm_open_serial} abgebildeten Methode \lstinline{piIoComm_open_serial()} generiert. 

\begin{lstlisting}[language={c},firstnumber=45,caption={Definition der seriellen Schnittstelle in \lstinline{piIOComm.h}\label{lst:4-REV_PI_TTY_DEVICE}}]
#define REV_PI_TTY_DEVICE	"/dev/ttyAMA0"
\end{lstlisting}

Das in Listing~\ref{lst:4-REV_PI_TTY_DEVICE} definierte Macro verweist auf eine der seriellen Schnittstellen des RaspberryPi.
Die Implementierung des zugehörigen Schnittstellentreibers soll hier nicht weiter untersucht werden. Somit ist an dieser Stelle die Kette vom Setzen einer Variablen auf dem OPC-Server bis hin zur Aktualisierung des Prozessabbilds der IO-Module geschlossen.

% \begin{lstlisting}[language={c},firstnumber={226},caption={Setzen der Scheduler-Priorität auf SCHED\_FIFO in 
% revpi\_common.c\label{lst:2-sched_priority}}]
% param.sched_priority = ktprio->prio;
% ret = sched_setscheduler(child, SCHED_FIFO, &param);
% \end{lstlisting}
% % % Imports nur für Referenzenauflösung während des Schreibens! Vorm Kompilieren auskommentieren!
% \bibliography{0_hauptdatei}
% \input{1_einleitung}
% \input{2_grundlagen}
% \input{3_konzeption}
% \input{4_implementierung}
% \input{5_tests}
% \input{6_zusammenfassung}
% % Ende Imports

\section{Test des OPC-Servers im Gesamtsystem%
  \label{sec:5-tests}}

% % % Imports nur für Referenzenauflösung während des schreibens! Vorm Kompilieren auskommentieren!
% \bibliography{0_hauptdatei}
% \input{1_einleitung}
% \input{2_grundlagen}
% \input{3_konzeption}
% \input{4_implementierung}
% \input{5_tests}
% \input{6_zusammenfassung}
% % Ende Imports

\section{Zusammenfassung und Ausblick%
  \label{sec:6-fazit}}
Der folgende Abschnitt~\ref{sec:6-zusammenfassung} fasst die gewonnenen Erkenntnisse und den Stand der Implementierung zusammen.
Den Abschluss dieser Arbeit bildet der Ausblick in Abschnitt~\ref{sec:6-ausblick}.

\subsection{Zusammenfassung%
     \label{sec:6-zusammenfassung}}

\subsection{Ausblick%
     \label{sec:6-ausblick}}

% % Ende Imports

\section{Test des OPC-Servers im Gesamtsystem%
  \label{sec:5-tests}}

% % % Imports nur für Referenzenauflösung während des schreibens! Vorm Kompilieren auskommentieren!
% \bibliography{0_hauptdatei}
% % Mit \section{...} eröffnen wir einen neuen Abschnitt.
% Der Befehl setzt nicht nur den Text in einer größeren,
% fetten Schrift, sondern sorgt außerdem dafür, daß er im
% Inhaltsverzeichnis erscheint.
%
% Mit \label{...} erzeugen wir einen Bezeichner, mit dessen Hilfe
% wir später auf die Nummer des Abschnitts verweisen können (nämlich
% mit~\ref{...}).
%
% Das Kommentarzeichen hinter „Übersicht“ dient dazu, ein
% Leerzeichen zwischen „Übersicht“ und dem \label-Befehl
% zu vermeiden, das andernfalls sichtbar würde – z.B. im
% Inhaltsverzeichnis.
%

% % Imports nur für Referenzenauflösung während des Schreibens! Vorm Kompilieren auskommentieren!
% \bibliography{0_hauptdatei}
% \input{1_einleitung}
%\input{2_grundlagen}
%\input{3_konzeption}
%\input{4_implementierung}
%\input{5_tests}
%\input{6_zusammenfassung}
% % Ende Imports

\section{Einleitung und Motivation%
  \label{sec:1-einleitung}}
Ziel dieses Projektes ist die Integration eines OPC-Servers mit einer auf Linux
basierenden speicherprogrammierbaren Steuerung (SPS). Angeschlossen an diese SPS
ist jeweils ein digitales Ein-/\,bzw.~Ausgabemodul. Die von diesen bereitgestellten
Ein-/\, bzw.~Ausgänge (IO) sollen in der Datenstruktur des OPC-Servers abgebildet
und über diesen für OPC-Clients les-/\,und schreibar sein. Weiterhin sollen einige
Funktionen zur Überwachung und Steuerung der an die SPS angeschlossenen Aktoren
und Sensoren direkt im OPC-Server implementiert werden.
Hiermit stellt dieses Projekt eine der Grundlagen für ein übergeordnetes Projekt,
die cloudbasierte Steuerung eines miniaturisierten Produktions-Systems, dar.

Der hier verwendete OPC-Server ist Teil des sog. open62541 Projekts. Er ist in C
geschrieben und implementiert bereits einen großen Teil der im OPC-UA-Standard
spezifizierten Funktionen.
Als SPS findet ein Revolution Pi 3 der Firma Kunbus Verwendung. Dieser integriert
ein sog. Compute Module der Raspberry Pi Foundation in ein industrietaugliches
Gehäuse und erlaubt die Erweiterung mittels IO- oder Gateway-Modulen. Über diese
erfolgt die Kommunikation mit weiteren Komponenten der Automatisierungstechnik.

Motiviert ist dieses Projekt durch die Beobachtung, dass die Verbreitung offener
Standards sowie freier Software auch in der Automatisierungstechnik zunimmt.
Linux ist ein freies Betriebssystem, OPC-UA ein offen zugänglicher, aktiv gepflegter
und weit verbreiteter Standard. Der Raspberry Pi findet sowohl bei Hobby-Anwendern als
auch in den Bereichen Forschung und Entwicklung sowie bei industriellen Anwendern
Verwendung. Dieses Projekt stellt somit eine für unterschiedliche Anwender interessante
Entwicklung dar.

Im Anschluss an diese einleitende Übersicht im Abschnitt~\ref{sec:1-einleitung} folgt
die Darstellung der wichtigsten Grundlagen in Abschnitt~\ref{sec:2-grundlagen}.
Aufbauend auf diesen Grundlagen folgt die konzeptuelle Ausarbeitung im Abschnitt~\ref{sec:3-konzeption}.
Die Umsetzung wird im Abschnitt~\ref{sec:4-implementierung} erläutert.
Die Leistungsfähigkeit der Implementierung wird in Abschnitt~\ref{sec:5-tests} untersucht.
Eine Zusammenfassung und ein Ausblick schließen die Arbeit in
Abschnitt~\ref{sec:6-fazit} ab. Eventuell noch benötigte Anhänge
finden sich in den Anhängen [...] bis [...].

% % % Imports nur für Referenzenauflösung während des Schreibens! Vorm Kompilieren auskommentieren!
% \bibliography{0_hauptdatei}
% \input{1_einleitung}
% \input{2_grundlagen}
% \input{3_konzeption}
% \input{4_implementierung}
% \input{5_tests}
% \input{6_zusammenfassung}
% % Ende Imports

\section{Grundlagen%
  \label{sec:2-grundlagen}}

\subsection{Speicherprogrammierbare-Steuerung und Linux -- Revolution Pi%
     \label{sec:2-sps}}

\subsubsection{Kunbus RevolutionPi%
        \label{sec:2-revpi}}
Der RevolutionPi 3 ist eine speicherprogrammierbare Steuerung (SPS) des Herstellers
Kunbus GmbH. Kern dieser SPS ist das von der Raspberry Pi Foundation entwickelte
und vertriebene Raspberry Pi Compute Module 3. Dieses integriert ein Broadcom BCM2837
System-on-Chip (SoC) mit vier 1,2GHz Prozessorkernen, 1GB RAM, 4GB eMMC Anwendungsspeicher
und sonstige Peripherie in ein Modul im DDR2-SODIMM Formfaktor. Diese Spezifikationen
sind weitgehend identisch zu denen des ausgesprochen populären Raspberry Pi 3.
Der Revolution Pi profitiert daher von dem gleichen großen Angebot an Software
und Unterstützung wie der Raspberry Pi, ergänzt dessen Hardware jedoch um eine 24V
Spannungsversorgung, die Möglichkeit der Erweiterung durch mehrere industrietaugliche
Ein-/ Ausgabemodule und Gateways sowie ein Gehäuse zur Montage auf einer DIN-Schiene.
\begin{itemize}
  \item{Prozessor: BCM2837}
  \item{Taktfrequenz 1,2 GHz}
  \item{Anzahl Prozessorkerne: 4}
  \item{Arbeitsspeicher: 1 GByte}
  \item{eMMC Flash Speicher: 4 GByte}
  \item{Betriebssystem: Angepasstes Raspbian mit RT-Patch}
  \item{RTC mit 24h Pufferung über wartungsfreien Kondensator}
  \item{Treiber / API: Treiber schreibt zyklisch Prozessdaten in ein Prozessabbild, Zugriff auf Prozessabbild über Linux-Filesystem als API zu Fremdsoftware.}
  \item{Kommunikationsanschlüsse: 2 x USB 2.0 A (je 500 mA belastbar), 1 x Micro-USB, HDMI, Ethernet (RJ45) 10/100 Mbit/s}
  \item{Stromversorgung: min. 10,7 V, max. 28,8 V, maximal 10 Watt}
  \item{Zulässige Umgebungstemperatur: -40 bis +55 C}
  \item{Gehäuseabmessungen: (HxBxL) 96 mm x 22,5 mm x 110,5 mm (ohne gesteckte Stecker)}
  \item{ESD Schutz: 4 kV / 8 kV gemäß EN61131-2 und IEC 61000-6-2}
  \item{Surge / Burst Prüfungen: gemäß EN61131-2 und IEC 61000-6-2 eingekoppelt auf Versorgungsspannung, Ethernet und IO-Leitungen}
  \item{EMI Prüfungen: gemäß EN61131-2 und IEC 61000-6-2}
\end{itemize}

Kunbus bietet eine Auswahl an IO- und Gateway-Modulen zur Erweiterung des Revolution Pi an.
Gateways dienen der Kommunikation mit Systemen oder Komponenten der Automatisierungstechnik
über Protokolle wie PROFIBUS oder EtherCAT. IO-Module erlauben die Überwachung
und Steuerung von digitalen oder analogen Ein- und Ausgängen.

\subsubsection{Zugriff auf IO-Module%
        \label{sec:2-io}}
Der Zugriff auf die Ein- und Ausgänge der IO-Module erfolgt über ein Prozessabbild
und einen hierfür von Kunbus bereitgestellten Treiber, genannt piControl. Dieser
aktualisiert das Prozessabbild zyklisch. Die angestrebte Zykluszeit beträgt 5ms,
kann jedoch je nach Anzahl der angeschlossenen Module auch größer sein. Kunbus
garantiert bei drei IO-Modulen und zwei Gateway-Modulen eine Zykluszeit von 10 ms.
Jedes der IO-Module stellt ein eigenständiges eingebettetes System dar. Es verfügt
über einen Microcontroller, welcher die IOs bereitstellt und über einen RS485-Bus
mit dem Revolution Pi kommuniziert.
% https://revolution.kunbus.de/io-modul/

Lizenz: GPL
% https://github.com/RevolutionPi/piControl

\begin{lstlisting}[language={c},firstnumber={226},caption={Setzen der Scheduler-Priorität auf SCHED\_FIFO in revpi\_common.c\label{lst:2-sched_priority}}]
param.sched_priority = ktprio->prio;
ret = sched_setscheduler(child, SCHED_FIFO,
       &param);
\end{lstlisting}


\subsection{Echtzeit und Multithreading unter Linux -- preemptRT und posix%
     \label{sec:2-echtzeit}}


 Der Linux-Kernel verfügt über mehrere unterschiedliche Preemtion-Modelle:

\begin{itemize}
  \item No Forced Preemption (server):
  Ausgelegt auf maximal möglichen Durchsatz, lediglich Interrupts und
  System-Call-Returns bewirken Präemption.

  \item Voluntary Kernel Preemption (Desktop):
  Neben den implizit bevorrechtigten Interrupts und System-Call-Returns gibt es
  in diesem Modell weitere Abschnitte des Kernels in welchen Preämption explizit
  gestattet ist.

  \item Preemptible Kernel (Low-Latency Desktop):
  In diesem Modell ist der gesamte Kernel, mit Ausnahme sog.~kritischer Abschnitte
  präemptible. Nach jedem kritischen Abschnitt gibt es einen impliziten Präemptions-Punkt.

  \item Preemptible Kernel (Basic RT):
  Dieses Modell ist dem zuvor genannten sehr ähnlich, hier sind jedoch alle Interrupt-Handler
  als eigenständige Threads ausgeführt.

  \item Fully Preemptible Kernel (RT):
  Wie auch bei den beiden zuvor genannten Modellen ist hier der gesamte Kernel
  präemtible, die Anzahl und Dauer der nicht-präemtiblen kritischen Abschnitte
  ist auf ein notwendiges Minimum beschränkt. Alle Interrupt-Handler sind als
  eigenständige Threads ausgeführt, Spinlocks durch Sleeping-Spinlocks und Mutexe
  durch sog.~RT-Mutexe ersetzt.

\end{itemize}
\todo{Spinlocks und Mutexe sowie die RT-Varianten dieser erklären!}

Lediglich mit dem vollständig präemtiblen Kernel kann Echtzeit-Verhalten realisiert werden.

% https://wiki.linuxfoundation.org/realtime/documentation/technical_basics/preemption_models bzw kernel/Kconfig.preempt

\subsubsection{preemptRT%
        \label{sec:2-preemptRT}}
% https://wiki.linuxfoundation.org/realtime/documentation/technical_details/start
% https://wiki.linuxfoundation.org/realtime/documentation/technical_basics/start

Das dem PREEMPT RT Kernel zugrunde liegende Prinzip lässt sich in einer einfachen
Regel ausdrücken: Nur Code, welcher absolut nicht-präemtible sein darf, ist es
gestattet nicht-präemtible zu sein.
Das erklärte Ziel des PREEMPT\_RT Patches ist es folglich, die Menge des nicht-präemtiblen
Codes im Linux-Kernel auf das absolut notwendige Minimum zu reduzieren.

Dies wird durch Verwendung folgender Mechanismen erreicht:

\begin{itemize}
  \item Hochauflösende Timer
  \item Sleeping Spinlocks
  \item Threaded Interrupt Handlers
  \item rt\_mutex
  \item RCU
\end{itemize}


\subsubsection{posix%
        \label{sec:2-posix}}
Ist posix hier wirklich relevant? Debian bzw.~Raspbian sind weitgehend posix
kompatibel, aber wird es hier genutzt? -> JA, open62541 nutzt pthread.h
piControl nutzt kthread.h, und semaphore.h

\subsection{OPC-UA und open62541%
     \label{sec:2-opc}}

\subsubsection{OPC UA%
        \label{sec:2-opcua}}
Open Platform Communications (OPC) ist eine Familie von Standards zur herstellerunabhängigen
Kommunikation von Maschinen (M2M) in der Automatisierungstechnik. Die sog.~OPC Task Force, zu deren
Mitgliedern verschiedene große Firmen der Automatisierungsindustrie gehören, veröffentlichte
die OPC Specification Version 1.0 im August 1996.
Motiviert ist dieser offene Standard durch die Erkenntniss, dass die Anpassung der
zahlreichen Herstellerstandards an individuelle Infrastrukturen und Anlagen einen
großen Mehraufwand verursachen.
Die Wikipedia beschreibt das Anwendungsgebiet für OPC wie folgt:

\glqq{}OPC wird dort eingesetzt, wo Sensoren, Regler und Steuerungen verschiedener Hersteller
ein gemeinsames Netzwerk bilden. Ohne OPC benötigten zwei Geräte zum Datenaustausch
genaue Kenntnis über die Kommunikationsmöglichkeiten des Gegenübers. Erweiterungen
und Austausch gestalten sich entsprechend schwierig. Mit OPC genügt es, für jedes
Gerät genau einmal einen OPC-konformen Treiber zu schreiben. Idealerweise wird
dieser bereits vom Hersteller zur Verfügung gestellt. Ein OPC-Treiber lässt sich
ohne großen Anpassungsaufwand in beliebig große Steuer- und Überwachungssysteme
integrieren.

OPC unterteilt sich in verschiedene Unterstandards, die für den jeweiligen Anwendungsfall
unabhängig voneinander implementiert werden können. OPC lässt sich damit verwenden
für Echtzeitdaten (Überwachung), Datenarchivierung, Alarm-Meldungen und neuerdings
auch direkt zur Steuerung (Befehlsübermittlung).\grqq{}

OPC basiert in der ursprünglichen Spezifikation auf Microsofts DCOM-Spezifikation.
DCOM macht Funktionen und Objekte einer Anwendung anderen Anwendungen im Netzwerk
zugänglich. Der OPC-Standard definiert entsprechende DCOM-Objekte um mit anderen
OPC-Anwendungen Daten austauschen zu können. Die Verwendung von DCOM bindet Anwender
an Betriebssysteme von Microsoft. Die ursprüngliche OPC Spezifikation wird durch die
Entwicklung von OPC Unified Architecture (OPC UA) abgelöst.
OPC UA setzt auf einem eigenen Kommunikationionsstack auf, die Verwendung von DCOM
und damit die Bindung an Microsoft wurden aufgelöst.

Die OPC-UA-Architektur ist eine Service-orientierte Architektur (SOA), deren Struktur
aus mehreren Schichten besteht.

% Wikipedia
Das OPC-Informationsmodell ist nicht mehr nur eine Hierarchie aus Ordnern, Items
und Properties. Es ist ein sogenanntes Full-Mesh-Network aus Nodes, mit dem neben
den Nutzdaten eines Nodes auch Meta- und Diagnoseinformationen repräsentiert werden.
Ein Node ähnelt einem Objekt aus der objektorientierten Programmierung. Ein Node
kann Attribute besitzen, die gelesen werden können (Data Access (DA), Historical
Data Access (HDA)). Es ist möglich Methoden zu definieren und aufzurufen.
Eine Methode besitzt Aufrufargumente und Rückgabewerte. Sie wird durch ein Command
aufgerufen. Weiterhin werden Events unterstützt, die versendet werden können
(AE (Alarms \& Events), DA DataChange), um bestimmte Informationen zwischen Geräten
auszutauschen. Ein Event besitzt unter anderem einen Empfangszeitpunkt, eine Nachricht
und einen Schweregrad. Die o. g. Nodes werden sowohl für die Nutzdaten als auch
alle anderen Arten von Metadaten verwendet. Der damit modellierte OPC-Adressraum
beinhaltet nun auch ein Typmodell, mit dem sämtliche Datentypen spezifiziert werden.

% https://de.wikipedia.org/wiki/Open_Platform_Communications
% https://de.wikipedia.org/wiki/OPC_Unified_Architecture
% https://opcfoundation.org/developer-tools/specifications-unified-architecture
% Von Gerhard Gappmeier - ascolab GmbH, CC BY-SA 3.0, https://de.wikipedia.org/w/index.php?curid=1892069
\subsubsection{open62541%
        \label{sec:2-open62541}}
open62541 ist eine offene und freie Implementierung von OPC UA. Die in C geschriebene
Bibliothek stellt eine beständig zunehmende Anzahl der im OPC UA Standard definierten
Funktionen bereit. Sie kann sowohl zur Erstellung von OPC-Servern als auch -Clients
genutzt werden. Ergänzend zu der unter der Mozilla Public License v2.0 lizensierten
Bibliothek stellt das open62541 Projekt auch Beispielprogramme unter einer CC0 Lizenz
zur Verfügung.

Die Bibliothek eignet sich auch für die Entwicklung auf eingebetteten Systemen und
Microcontrollern. Je nach Umfang der gewünschten Funktionen und des OPC Informationsmodells
beträgt die Größe einer Server-Binary weniger als 100kb. %evtl. kürzen?

\todo{Nodes erklären! Evtl.~oben!}

Folgende Auswahl an Eigenschaften und Funktionen zeichnet die in dieser Arbeit verwendete
Version 0.3 von open62541 aus:
\begin{itemize}
  \item Kommunikationionsstack
  \begin{itemize}
      \item OPC UA Binär-Protokoll (HTTP oder SOAP werden gegenwärtig nicht unterstützt)
      \item Austauschbare Netzwerk-Schicht, welche die Verwendung eigener Netzwerk-APIs
      erlaubt.
      \item Verschlüsselte Kommunikationion
      \item Asynchrone Dienst-Anfragen im Client
  \end{itemize}
  \item Informationsmodell
  \begin{itemize}
    \item Unterstützung aller OPC UA Node-Typen, inkl.~Methoden
    \item Hinzufügen und Entfernen von Nodes und Referenzen zur Laufzeit.
    \item Vererbung und Instanziierung von Objekt- und Variablentypen
    \item Zugriffskontrolle auch für einzelne Nodes
  \end{itemize}
  \item Subscriptions
  \begin{itemize}
    \item Erlaubt die Überwachung (subscriptions / monitoreditems)
    \item Sehr geringer Ressourcenbedarf pro überwachtem Wert
  \end{itemize}
  \item Code-Generierung auf XML-Basis
  \begin{itemize}
    \item Erlaubt die Erstellung von Datentypen
    \item Erlaubt die Generierung des serverseitigen Informationsmodells
  \end{itemize}
\end{itemize}

% https://open62541.org/doc/0.3/


Mozilla Public License
CC0 Lizenz für Beispiele und Plugins

% https://open62541.org/doc/open62541-current.pdf
% https://open62541.org/

% % % Imports nur für Referenzenauflösung während des Schreibens! Vorm Kompilieren auskommentieren!
% \bibliography{0_hauptdatei}
% \input{1_einleitung}
% \input{2_grundlagen}
% \input{3_konzeption}
% \input{4_implementierung}
% \input{5_tests}
% \input{6_zusammenfassung}
% \input{anhang}
% % Ende Imports

\section{Systemkonzept%
  \label{sec:3-konzeption}}
Auf Basis der in Abschnitt \ref{sec:2-grundlagen} vorgestellten Möglichkeiten folgt nun die Ausarbeitung eines Konzepts.
In den folgenden Abschnitten soll näher auf zwei zentrale Aspekte eingegangen werden: Abschnitt~\ref{sec:3-anbindung} stellt Möglichkeiten zum Zugriff auf Variablen bzw.\,Werte im Prozessabbild des Revolution Pi vor; in Abschnitt~\ref{sec:3-integration} wird ein Konzept zur Bereitstellung dieser Variablen auf einem OPC-Server vorgestellt.

\subsection{Anbindung der IO an den OPC-Server%
     \label{sec:3-anbindung}}

Eine Webanwendung mit Bezeichnung PiCtory dient zur Konfiguration der I/O- und virtuellen Module des RevolutionPi. Die Konfiguration liegt im JSON-Format in der Datei \lstinline{/etc/revpi/config.rsc}. Der piControl-Treiber liest diese Datei beim Start. 
Der folgende Auszug aus der Manpage des piControl-Kernelmoduls beschreibt die von diesem zum Lesen und Schreiben einzelner Bits des Prozessabbildes bereitgestellten Funktionen~\citep[vgl.]{web-revpi-manpage}. Sie ist an dieser Stelle weitgehend ungekürzt zitiert, da sie die nutzbare Schnittstelle sehr kompakt beschreibt.

\begin{lstlisting}[breakindent=0pt, numbers=none, caption={Auszug aus der Revolution Pi Programmers Manual\label{lst:4-manpage}}]
KB_FIND_VARIABLE SPIVariable *argp
Find a variable in the process image by its name. A pointer to a structure of type SPIVariable must be passed as argument. [...]
The struct SPIVariable [...] is defined as 
typedef struct SPIVariableStr
{
    char strVarName[32]; // Variable name
    uint16_t i16uAddress; // Address of the byte in the process image
    uint8_t i8uBit; // 0-7 bit position, >= 8 whole byte
    uint16_t i16uLength; // length of the variable in bits.
    // Possible values are 1, 8, 16 and 32
} SPIVariable;

Set and get values of the process image
KB_GET_VALUE SPIValue *argp
[...]
KB_SET_VALUE SPIValue *argp
Write one bit or one byte to the process image [...].  This call is more efficient than the usual calls of seek and write because only one function call is necessary. If more than on application are writing bits in one output byte, this call is the only safe way to set a bit without overwriting the other bits because this call is doing a read-modify-write-cycle. 

The struct SPIValue used by this ioctl is defined as
typedef struct SPIValueStr
{
    uint16_t i16uAddress; // Address of the byte in the process image
    uint8_t i8uBit; // 0-7 bit position, >= 8 whole byte
    uint8_t i8uValue; // Value: 0/1 for bit access, whole byte otherwise
} SPIValue;
\end{lstlisting} 

Die oben beschriebenden Funtkionen \lstinline{KB_FIND_VARIABLE}, \lstinline{KB_GET_VALUE} und \lstinline{KB_SET_VALUE} ermöglichen einen einfachen und (lt.\,Manpage) effizienten Zugriff auf einzelne Bits des Prozessabbildes und damit auch auf die IO des RevolutionPi.
Der Zugriff des OPC-Servers auf das Prozessabbild soll daher mittels dieser Funktionen realisiert werden.
\lstinline{KB_FIND_VARIABLE} kann genutzt werden, um Adressen von Variablen im Prozessabbild mittels ihres Namens aufzulösen.
\lstinline{KB_GET_VALUE} und \lstinline{KB_SET_VALUE} ermöglichen den Zugriff auf die Werte dieser Variablen.


\subsection{Integration des OPC-Servers in das System%
     \label{sec:3-integration}}

open62541 bietet drei Möglichkeiten zum Abgleich von Variablen mit dem Prozessabbild~\citep[vgl.][Tutorials - Connecting a Variable with a Physical Process]{web-open62541}:
\begin{itemize}
    \item Manuelles oder zyklisches Aktualisieren
    \item Variable Value Callback
    \item Variable Datasource
\end{itemize}

Die zyklische Aktualisierung eines oder mehrerer Werte nimmt, abhängig von der Zykluszeit, viele Systemressourcen in Anspruch. Value Callbacks ermöglichen es, einen Variablenwert effizienter mit einer Ressource wie etwa einem Prozessabbild zu synchronisieren. An die Variable wird ein Callback angehängt, welches vor jedem Lesen und nach jedem Schreibvorgang ausgeführt wird.
Der Wert der Variablen wird weiterhin im Variablenknoten auf dem OPC-Server gespeichert, der Abgleich mit der verknüpften Ressource erfolgt durch die Callback-Methoden.

Sogenannte Datenquellen gehen noch einen Schritt weiter. Der Server leitet jede Lese- und Schreibanforderung direkt an eine Callback-Funktion weiter. Beim Lesen liefert der Rückruf eine Kopie des aktuellen Wertes. Die Datenquelle muss intern ein eigenes Speichermanagement implementieren.

Der Zugriff auf die Werte des Prozessabbildes erfolgt, wie in Abschnitt~\ref{sec:3-anbindung} beschrieben, über von piControl bereitgestellte Methoden. Um die durch open62541 gepflegte OPC-Datenstruktur und das durch piControl verwaltete Prozessabbild möglichst effektiv verknüpfen zu können, soll diese Interaktion mittels Datenquellen und den zugehörigen Callbacks implementiert werden.
% % % Imports nur für Referenzenauflösung während des Schreibens! Vorm Kompilieren auskommentieren!
% \bibliography{0_hauptdatei}
% \input{1_einleitung}
% \input{2_grundlagen}
% \input{3_konzeption}
% \input{4_implementierung}
% \input{5_tests}
% \input{6_zusammenfassung}
% \input{anhang}
% % Ende Imports

\section{Implementierung%
  \label{sec:4-implementierung}}
Das folgende Kapitel stellt in Auszügen die Implementierung des OPC-Servers sowie die Anbindung an die IO-Module
der SPS dar. Der Schwerpunkt liegt hierbei auf der Funktionsweise des piControl-Treibers und dessen Integration in das Projekt. Abschnitt~\ref{sec:4-picontrol} erklärt die zum Schreibens eines Bits verwendeten Funktionsaufrufe.
Zuvor soll jedoch in Abschnitt~\ref{sec:4-open62541} der Teil des OPC-Servers vorgestellt werden, welcher auf besagten Treiber zugreift. 

\subsection{Implementierung des OPC-Servers%
     \label{sec:4-open62541}}
Wie im vorangegangenen Abschnitt~\ref{sec:3-integration} begründet, soll die Verknüpfung zwischen dem Prozessabbild der SPS und den auf dem OPC-Server bereitgestellten Werten über sog.\,Datenquellen erfolgen. Hierzu ist zunächst eine Callback-Methode zu implementieren, welche bei einem Lese- oder Schreibzugriff auf eine Variable aufgerufen wird. Die Verknüpfung zwischen Callback-Methode und Variable muss manuell erfolgen.

\begin{lstlisting}[language={c},firstnumber=237,caption={Auszug der Methode \lstinline{linkDataSourceVariable} in \lstinline{variables.c}\label{lst:4-linkDataSourceVariable}}]
extern UA_StatusCode
 linkDataSourceVariable(UA_Server *server, UA_NodeId nodeId) {
     bool readonly = false;
     UA_DataSource dataSourceVariable;
     UA_StatusCode rc; |>\setcounter{lstnumber}{254}<|

     dataSourceVariable.read = readDataSourceVariable;
     if (!readonly)
        dataSourceVariable.write = writeDataSourceVariable;
     else
        dataSourceVariable.write = writeReadonlyDataSourceVariable;

     return UA_Server_setVariableNode_dataSource(server, nodeId, dataSourceVariable);
 }
\end{lstlisting}

\begin{figure}[h]
    \centering
    \includegraphics[width=0.42\textwidth]{doc/img/OPC_RevPiDO.pdf}
    \caption{Auszug des verwendeten Nodesets, hier Digitalausgang 1 des Versuchsaufbaus
      \label{fig:opc-do}}
\end{figure}

Die in Listing~\ref{lst:4-linkDataSourceVariable} abgebildete Methode \lstinline{linkDataSourceVariable()} erzeugt ein Struct vom Typ \lstinline{UA_DataSource}. In diesem werden dem Lesen und Schreiben einer OPC-Variablen entsprechende Callback-Methoden zugewiesen. Die Verknüpfung einer OPC-Variable, genauer ihrer NodeId, mit der zuvor definierten Datenquelle erfolgt über die von open62541 bereitgestellte Methode \lstinline{UA_Server_setVariableNode_dataSource()}. Vor dem Lesen und nach dem Schreiben dieser Variable werden von nun an die entsprechenden Callbacks aufgerufen.
     
\begin{lstlisting}[language={c},firstnumber=168,caption={Auszug des Callbacks \lstinline{writeDataSourceVariable} in \lstinline{variables.c}\label{lst:4-writeDataSourceVariable}}]  
extern UA_StatusCode
 writeDataSourceVariable(UA_Server *server,
            const UA_NodeId *sessionId, void *sessionContext,
            const UA_NodeId *nodeId, void *nodeContext,
            const UA_NumericRange *range, const UA_DataValue *dataValue) {

    UA_StatusCode retval  = UA_STATUSCODE_GOOD;
    UA_NodeId *nameNodeId = UA_malloc(sizeof(UA_NodeId));
    UA_QualifiedName nameQN = UA_QUALIFIEDNAME(1, "Name");
    UA_Variant nameVar;
    UA_Boolean bit;

    retval |= findSiblingByBrowsename(server, nodeId, &nameQN, nameNodeId);
    retval |= UA_Server_readValue(server, *nameNodeId, &nameVar);
    retval |= UA_Boolean_copy(dataValue->value.data, &bit);

    |>\tikzmarkin[set border color=martinired]{writeIO}<|PI_writeSingleIO(String_fromUA_String(nameVar.data), &bit, false);                                                 |>\tikzmarkend{writeIO}<|

    free(nameNodeId);
    return retval;
 }
\end{lstlisting}

Listing~\ref{lst:4-writeDataSourceVariable} zeigt die Callback-Methode, welche nach dem Schreiben einer Variablen auf dem OPC-Server aufgerufen wird.
Dieser Methode wird neben der NodeId der mit ihr verknüpften Variablen auch der Wert dieser in Form eines Zeigers auf ein Struct vom Typ \lstinline{UA_DataValue} übergeben.

Die Gestaltung des hier verwendeten Nodesets sieht vor, dass in einer OPC-Variablen \lstinline{"Name"} der Bezeichner des zu schreibenden Digitalausgangs hinterlegt ist, siehe Abbildung~\ref{fig:opc-do}. Dies erlaubt eine Rekonfiguration der Ein- und Ausgänge der SPS ohne Änderungen im Programmcode des OPC-Servers vornehmen zu müssen.
Es ist daher erforderlich, nach jedem Schreiben einer mit einem Digitalausgang verknüpften Variablen, hier \lstinline{"Value"}, dessen Bezeichner \lstinline{"Name"} abzufragen. 
Dies geschieht in den Zeilen 180 und 181.
Anschließend wird dieser Bezeichner sowie der zu schreibende Wert der Methode \lstinline{PI_writeSingleIO()} übergeben, welche wiederum die Interaktion mit piControl übernimmt (vgl. Abschnitt \ref{sec:4-picontrol}).
 
\subsection{Integration von piControl%
     \label{sec:4-picontrol}}
In Abschnitt~\ref{sec:2-io} wurde die Anbindung der IO-Module des Revolution Pi sowie die Funktionsweise von piControl aus Anwendersicht beschrieben. Die verfügbare Literatur beschränkt sich auch auf lediglich diese Sicht; eine weiterführende Dokumentation für Entwickler gibt es, neben der in Abschnitt~\ref{sec:3-anbindung} vorgestellten Manpage, nicht. 
In diesem Abschnitt soll daher der Quellcode von piControl sowie dessen Verwendung im Projekt genauer betrachtet werden.
Hierzu wird exemplarisch die in Abschnitt~\ref{sec:4-open62541} eingeführte Methode \lstinline{PI_writeSingleIO()} untersucht.
Diese Methode ermöglicht das Setzen eines einzelnen Bits im Prozessabbild der SPS, und damit das Schalten eines digitalen Ausgangs auf einem IO-Modul.
Die äquivalente Methode \lstinline{int piControlGetBitValue(SPIValue *pSpiValue)} zum Lesen eines Bits bzw. Eingangs funktioniert analog und soll daher an dieser Stelle nicht dediziert erörtert werden.

\begin{lstlisting}[language={c},firstnumber=97,
                   caption={Setzen eines phsikalischen, digitalen Ausgangs in \lstinline{revpi.c}
                   \label{lst:4-PI_writeSingleIO}}]
extern void PI_writeSingleIO(char *pszVariableName, bool *bit, bool verbose)
{
	int rc;
	SPIVariable sPiVariable;
	SPIValue sPIValue;

	strncpy(sPiVariable.strVarName, pszVariableName, sizeof(sPiVariable.strVarName));
	rc = piControlGetVariableInfo(&sPiVariable);
	if (rc < 0) {
		printf("Cannot find variable '%s'\n", pszVariableName);
		return;
	}

		sPIValue.i16uAddress = sPiVariable.i16uAddress;
		sPIValue.i8uBit = sPiVariable.i8uBit;
		sPIValue.i8uValue = *bit;
		rc = |>\tikzmarkin[set border color=martinired]{setBitValue}<|piControlSetBitValue(&sPIValue)|>\tikzmarkend{setBitValue}<|;
		if (rc < 0)
			printf("Set bit error %s\n", getWriteError(rc));
		else if (verbose)
			printf("Set bit %d on byte at offset %d. Value %d\n", sPIValue.i8uBit, sPIValue.i16uAddress,
			       sPIValue.i8uValue);
}
\end{lstlisting}

Der Programmcode in Listing~\ref{lst:4-PI_writeSingleIO} ist Teil des implementierten OPC-Servers. In diesem wird auf zwei Funktionen des piControl-Treibers zugegriffen. 
Beiden Methoden wird als Argument ein Zeiger auf ein Struct vom Typ \lstinline{SPIValue} übergeben. Der im Struct abgelegte Name wird mittels \lstinline{piControlGetVariableInfo(&sPIValue)} zu einer Adresse im Prozessabbild aufgelöst. Diese wird in \lstinline{sPIValue.i16uAdress} gespeichert. Der Wert der Variablen wird anschließend mittels \lstinline{piControlSetBitValue(&sPIValue)} an dieser Adresse in das Prozessabbild geschrieben.

\begin{lstlisting}[language={c},firstnumber=309,caption={Methode \lstinline{piControlSetBitValue} in \lstinline{piControlIf.c}\label{lst:4-piControlSetBitValue}}]
int |>\tikzmarkin[set border color=martiniblue]{setBitValueFcn}<|piControlSetBitValue(SPIValue *pSpiValue)|>\tikzmarkend{setBitValueFcn}<|
{
    piControlOpen();

    if (PiControlHandle_g < 0)
	    return -ENODEV;

    pSpiValue->i16uAddress += pSpiValue->i8uBit / 8;
    pSpiValue->i8uBit %= 8;

    if (|>\tikzmarkin[set border color=martinired]{ioctl}<|ioctl(PiControlHandle_g, KB_SET_VALUE, pSpiValue)|>\tikzmarkend{ioctl}<| < 0)
	    return errno;

    return 0;
}
\end{lstlisting}

Die in Listing~\ref{lst:4-piControlSetBitValue} dargestellte Methode \lstinline{piControlSetBitValue} ist lediglich eine Hüllfunktion (häufig auch als Wrapper-Funktion bezeichnet) für einen Aufruf des \lstinline{ioctl} Kernel-Moduls.
Folgende Parameter werden übergeben:
\lstinline{PiControlHandle_g} ist die Referenz auf die Geräte-Datei des piControl-Treibers. \lstinline{KB_SET_VALUE} ist das ioctl-Kommando zum Schreiben eines Bits in das Prozessabbild. Der Zeiger \lstinline{pSpiValue} verweist auf ein Struct des bereits vorgestellten Typs \lstinline{SPIValue}.

\begin{lstlisting}[language={c},firstnumber=80,caption={Methode \lstinline{piControlOpen} in \lstinline{piControlIf.c}\label{lst:4-piControlOpen}}]
void piControlOpen(void)
{
    /* open handle if needed */
    if (PiControlHandle_g < 0)
    {
	    |>\tikzmarkin[set border color=martiniblue]{PiControlHandle}<|PiControlHandle_g = open(PICONTROL_DEVICE, O_RDWR)|>\tikzmarkend{PiControlHandle}<|;
    }
}
\end{lstlisting}

Die in Listing~\ref{lst:4-piControlOpen} dargestellte Methode öffnet, sofern nicht bereits geschehen, die Geräte-Datei. Das Macro \lstinline{PICONTROL_DEVICE} verweist hierbei auf \lstinline{/dev/piControl0}.

\begin{lstlisting}[language={c},firstnumber=721,caption={Methode \lstinline{piControlIoctl} in \lstinline{piControlMain.c}\label{lst:4-piControlIoctl}}]
static long |>\tikzmarkin[set border color=martiniblue, below offset=0.9em]{piControlIoctl}<|piControlIoctl(struct file *file, unsigned int prg_nr, 
                           unsigned long usr_addr)                                      |>\tikzmarkend{piControlIoctl}<|
{
  int status = -EFAULT;
  tpiControlInst *priv;
  int timeout = 10000;	// ms

  if (prg_nr != KB_CONFIG_SEND && prg_nr != KB_CONFIG_START && !isRunning()) {
  	return -EAGAIN;
  }

  priv = (tpiControlInst *) file->private_data;

  if (prg_nr != KB_GET_LAST_MESSAGE) {
  	// clear old message
  	priv->pcErrorMessage[0] = 0;
  }

  switch (prg_nr) {|>\setcounter{lstnumber}{864}<|

    case |>\tikzmarkin[set border color=martiniblue]{KB_SET_VALUE}<|KB_SET_VALUE:|>\tikzmarkend{KB_SET_VALUE}<|
  		{
  			SPIValue *pValue = (SPIValue *) usr_addr;

  			if (!isRunning())
  				return -EFAULT;

  			if (pValue->i16uAddress >= KB_PI_LEN) {
  				status = -EFAULT;
  			} else {
  				INT8U i8uValue_l;
  				my_rt_mutex_lock(&piDev_g.lockPI);
  				i8uValue_l = piDev_g.ai8uPI[pValue->i16uAddress];

  				if (pValue->i8uBit >= 8) {
  					i8uValue_l = pValue->i8uValue;
  				} else {
  					if (pValue->i8uValue)
  						i8uValue_l |= (1 << pValue->i8uBit);
  					else
  						i8uValue_l &= ~(1 << pValue->i8uBit);
  				}

  				|>\tikzmarkin[set border color=martinired]{i8uValue}<|piDev_g.ai8uPI[pValue->i16uAddress] = i8uValue_l;|>\tikzmarkend{i8uValue}<|
  				rt_mutex_unlock(&piDev_g.lockPI);

  #ifdef VERBOSE
  				pr_info("piControlIoctl Addr=%u, bit=%u: %02x %02x\n", pValue->i16uAddress, pValue->i8uBit, pValue->i8uValue, i8uValue_l);
  #endif

  				status = 0;
  			}
  		}
  		break; |>\setcounter{lstnumber}{1314}<|

    default:
      pr_err("Invalid Ioctl");
      return (-EINVAL);
      break;

    }

    return status;
  }
\end{lstlisting}

Listing~\ref{lst:4-piControlIoctl} zeigt in Auszügen die ioctl-Methode des piControl Kernel-Treibers. Diese bekommt folgende Argumente übergeben: \lstinline{struct file *file} enthält den Verweis auf die Geräte-Datei, hier \lstinline{/dev/piControl0}. Der Wert von \lstinline{unsigned int prg_nr} beschreibt die Anfrage an den Treiber, in diesem Fall \lstinline{KB_SET_VALUE}. Das Argument \lstinline{unsigned long usr_addr} enthält einen typ-agnostischen Pointer. Dieser verweist auf einen Speicherbereich, in welchem die zur Bearbeitung der Anfrage notwendigen Daten abgelegt sind. Hier können auch vom Treiber empfangene Daten dem Anwendungsprogramm bereitgestellt werden. 

Die switch-case-Anweisung führt die über das Argument \lstinline{prg_nr} spezifizierte Aktion aus. Hier betrachten wir \lstinline{KB_SET_VALUE}:
Zunächst wird in Zeile 868 der übergebene Zeiger \lstinline{usr_addr} mittels explizitem Typecast zu einem Zeiger des Typs \lstinline{SPIValue *} konvertiert. Da dieser auf Daten im Userspace verweist, ist beim Zugriff durch den Kernel-Treiber besondere Vorsicht geboten.
In Zeile 877 wird mittels Mutex das Prozessabbild \lstinline{piDev_g} für den Zugriff durch andere Threads oder Prozesse gesperrt.
\lstinline{my_rt_mutex_lock} verweist hierbei auf die Funktion \lstinline{rt_mutex_lock} aus \lstinline{linux/sched.h}\footnote{Offenbar wurde hier auch eine alternative Implementierung vorgesehen, siehe revpi\_common.h}

In Zeile 889 wird das Byte \lstinline{i8uValue_l}, welches den zu schreibenden Wert enthält in das Prozessabbild übertragen. Anschließend wird die Mutex auf \lstinline{piDev_g} wieder entsperrt.
\newpage

\begin{lstlisting}[language={c},firstnumber=62,caption={Auszug des Struct \lstinline{spiControlDev} in \lstinline{piControlMain.h}\label{lst:4-spiControlDev}}]
|>\tikzmarkin[set border color=martiniblue]{spiControlDev}<|typedef struct spiControlDev|>\tikzmarkend{spiControlDev}<| {
	// device driver stuff
	int init_step;
	enum revpi_machine machine_type;
	void *machine;
	struct cdev cdev;	// Char device structure
	struct device *dev;
	struct thermal_zone_device *thermal_zone;

	|>\tikzmarkin[set border color=martiniblue]{processImage}<|// process image stuff
	INT8U ai8uPI[KB_PI_LEN];
	INT8U ai8uPIDefault|>\tikzmarkin[set border color=martinired]{KB_PI_LEN_0}<|[KB_PI_LEN]|>\tikzmarkend{KB_PI_LEN_0}<|;
	struct rt_mutex lockPI;        |>\tikzmarkend{processImage}<|
	bool stopIO;
	piDevices *devs; |>\setcounter{lstnumber}{94}<|
} tpiControlDev;
\end{lstlisting}

Das Prozessabbild ist als Byte-Array der Länge \lstinline{KB_PI_LEN} in Listing~\ref{lst:4-spiControlDev} definiert. Konfigurationsparameter wie \lstinline{KB_PI_LEN} oder die Zykluszeit für den Datenaustausch zwischen SPS und IO-Modulen sind im folgenden Listing~\ref{lst:4-process} definiert.

\begin{lstlisting}[language={c},firstnumber=119,caption={Konfigurationsparameter des Prozessabbildes in project.h\label{lst:4-process}}]
#define INTERVAL_PI_GATE (5*1000*1000)  // 5 ms piGateCommunication |>\setcounter{lstnumber}{128}<|

#define INTERVAL_IO_COM (5*1000*1000)  // 5 ms piIoComm |>\setcounter{lstnumber}{132}<|

#define KB_PD_LEN       512
|>\tikzmarkin[set border color=martiniblue]{KB_PI_LEN_1}<|#define KB_PI_LEN       4096|>\tikzmarkend{KB_PI_LEN_1}<|
\end{lstlisting}

Das zu setzende Bit wurde zu diesem Zeitpunkt erfolgreich in das Prozessabbild der SPS geschrieben.
Es stellt sich die Frage, wie dieses nun an das IO-Modul kommuniziert wird.
Die Kommunikation mit allen angebundenen Modulen ist ebenfalls Aufgabe des piControl-Treibers.

\begin{lstlisting}[language={c},firstnumber=256,caption={Auszug der Methode \lstinline{piIoThread} in \lstinline{revpi_core.c}\label{lst:4-piIoThread}}]
static int piIoThread(void *data)
{
	//TODO int value = 0;
	ktime_t time;
	ktime_t now;
	s64 tDiff;

	hrtimer_init(&piCore_g.ioTimer, CLOCK_MONOTONIC, HRTIMER_MODE_ABS);
	piCore_g.ioTimer.function = piIoTimer;

	pr_info("piIO thread started\n");

	now = hrtimer_cb_get_time(&piCore_g.ioTimer);

	PiBridgeMaster_Reset();

	while (!kthread_should_stop()) {
		if (|>\tikzmarkin[set border color=martinired]{PiBridgeMaster}<|PiBridgeMaster_Run()|>\tikzmarkend{PiBridgeMaster}<| < 0)
			break;
	}

	RevPiDevice_finish();

	pr_info("piIO exit\n");
	return 0;
}
\end{lstlisting}

Der Kernel-Thread \lstinline{piIoThread} ist verantwortlich für den zyklischen Datenaustausch mit den IO-Modulen. In diesem wird fortlaufend die Methode \lstinline{PiBridgeMaster_Run()} aufgerufen, siehe Listing~\ref{lst:4-piIoThread}.

\begin{lstlisting}[language={c},firstnumber=262,caption={Auszug der Methode \lstinline{PiBridgeMaster_Run(void)} in \lstinline{RevPiDevice.c}\label{lst:4-PiBridgeMaster_Run}}]
int PiBridgeMaster_Run(void)
{
	static kbUT_Timer tTimeoutTimer_s;
	static kbUT_Timer tConfigTimeoutTimer_s;
	static int error_cnt;
	static INT8U last_led;
	static unsigned long last_update;
	int ret = 0;
	int i;

	my_rt_mutex_lock(&piCore_g.lockBridgeState);
	if (piCore_g.eBridgeState != piBridgeStop) {
		switch (eRunStatus_s) { |>\setcounter{lstnumber}{514}<|
		    case enPiBridgeMasterStatus_EndOfConfig:|>\setcounter{lstnumber}{621}<|
		    if (|>\tikzmarkin[set border color=martinired]{RevPiDevice}<|RevPiDevice_run()|>\tikzmarkend{RevPiDevice}<|) {
				// an error occured, check error limits |>\setcounter{lstnumber}{641}<|
			} else {
				ret = 1;
			}
			piCore_g.image.drv.i16uRS485ErrorCnt = RevPiDevice_getErrCnt();
			break;
\end{lstlisting}

Die in Listing~\ref{lst:4-PiBridgeMaster_Run} dargestellte Methode ist eine sog. State-Machine. Ist die Konfiguration der IO-Module erfolgreich abgeschlossen, so führt sie bei Aufruf lediglich die Methode \lstinline{RevPiDevice_run()} aus.

\begin{lstlisting}[language={c},firstnumber=140,caption={Auszug der Methode \lstinline{RevPiDevice_run(void)} in \lstinline{RevPiDevice.c}\label{lst:4-RevPiDevice_run}}]
int RevPiDevice_run(void)
{
	INT8U i8uDevice = 0;
	INT32U r;
	int retval = 0;

	RevPiDevices_s.i16uErrorCnt = 0;

	for (i8uDevice = 0; i8uDevice < RevPiDevice_getDevCnt(); i8uDevice++) {
		if (RevPiDevice_getDev(i8uDevice)->i8uActive) {
			switch (RevPiDevice_getDev(i8uDevice)->sId.i16uModulType) {
			case KUNBUS_FW_DESCR_TYP_PI_DIO_14:
			case KUNBUS_FW_DESCR_TYP_PI_DI_16:
			case KUNBUS_FW_DESCR_TYP_PI_DO_16:
				r = |>\tikzmarkin[set border color=martinired]{sendCyclicTelegram}<|piDIOComm_sendCyclicTelegram(i8uDevice)|>\tikzmarkend{sendCyclicTelegram}\setcounter{lstnumber}{166} <|;

				break; |>\setcounter{lstnumber}{216}<|
			}
		}
	} |>\setcounter{lstnumber}{227}<|
	return retval;
}
\end{lstlisting}

Diese iteriert wie in Listing~\ref{lst:4-RevPiDevice_run} abgebildete durch alle gegenwärtig in der SPS konfigurierten Module. Ist das aktuelle Modul als aktiv markiert, so wird anhand eines sog. Firmware-Descriptors entschieden, welche Methode für die Ansteuerung des Moduls aufzurufen ist.

\begin{lstlisting}[language={c},firstnumber=161,caption={Auszug der Methode \lstinline{piDIOComm_sendCyclicTelegram} in \lstinline{piDIOComm.c}\label{lst:4-sendCyclicTelegram}}]
INT32U piDIOComm_sendCyclicTelegram(INT8U i8uDevice_p)
{
	INT32U i32uRv_l = 0;
	SIOGeneric sRequest_l;
	SIOGeneric sResponse_l;
	INT8U len_l, data_out[18], i, p, data_in[70];
	INT8U i8uAddress;
	int ret; |>\setcounter{lstnumber}{239}<|
	
    |>\tikzmarkin[set border color=martinired]{piIoComm}<|ret = piIoComm_send((INT8U *) & sRequest_l, IOPROTOCOL_HEADER_LENGTH + len_l + 1);  |>\tikzmarkend{piIoComm}\setcounter{lstnumber}{298}<|
}
\end{lstlisting}

Im Falle des hier verwendeten DO-Moduls wird die in Listing~\ref{lst:4-sendCyclicTelegram} abgebildete Methode \lstinline{piDIOComm_sendCyclicTelegram()} aufgerufen. Dieser wird ein Zeiger auf das zu schreibende Gerät übergeben. 
Zunächst wird das Prozessabbild mittels eines proprietären, jedoch im Quellcode offen nachvollziehbaren Protokolls in ein \lstinline{sRequest_l} genanntes Byte-Array umgewandelt. Dieser Schritt ist in Listing~\ref{lst:4-sendCyclicTelegram} nicht abgebildet. Anschließend wird \lstinline{piIoComm_send()} ein Zeiger auf die so generierte Schreib-Anfrage übergeben.

\begin{lstlisting}[language={c},firstnumber=220,caption={Auszug der Methode \lstinline{piIOComm_send} in \lstinline{piIOComm.c}\label{lst:4-piIOComm_send}}]
int piIoComm_send(INT8U * buf_p, INT16U i16uLen_p)
{
	ssize_t write_l = 0;
	INT16U i16uSent_l = 0;|>\setcounter{lstnumber}{249}<|

	while (i16uSent_l < i16uLen_p) {
		write_l = vfs_write(piIoComm_fd_m, buf_p + i16uSent_l, i16uLen_p - i16uSent_l, &piIoComm_fd_m->f_pos);
		if (write_l < 0) {
			pr_info_serial("write error %d\n", (int)write_l);
			return -1;
		} 
		i16uSent_l += write_l;|>\setcounter{lstnumber}{263}<|
	}
	clear();
	vfs_fsync(piIoComm_fd_m, 1);
	return 0;
}
\end{lstlisting}

Listing~\ref{lst:4-piIOComm_send} zeigt die Implementierung von \lstinline{piIoComm_send()}. Diese Methode ist für das Schreiben der oben generierten Anfrage auf die seriellen Schnittstelle verantwortlich. Realisiert wird dies mittels der Methode \lstinline{vfs_write()}. Diese ist in \lstinline{<linux/fs.h>} definiert. Sie ermöglicht das Schreiben einer Datei im Userspace aus dem Kernel heraus. Geschrieben wird hier die Datei mit dem Deskriptor \lstinline{piIoComm_fd_m}.
Da die Funktion \lstinline{vfs_write()} durch andere Kernel-Tasks unterbrochen werden kann, ist nicht gewährleistet, dass die gesamte Anfrage mit nur einem Aufruf geschrieben wird. Die oben abgebildete while-Schleife stellt das vollständige Senden der Anfrage sicher.

\begin{lstlisting}[language={c},firstnumber=157,caption={Auszug der Methode \lstinline{piIOComm_open_serial} in \lstinline{piIOComm.c}\label{lst:4-piIOComm_open_serial}}]
int piIoComm_open_serial(void)
{   |>\setcounter{lstnumber}{167}<|
	struct file *fd;	/* Filedeskriptor */
	struct termios newtio;	/* Schnittstellenoptionen */

	|>\tikzmarkin[set border color=martiniblue]{fd}<|/* Port oeffnen - read/write, kein "controlling tty", 
	    Status von DCD ignorieren */
	fd = filp_open(|>\tikzmarkin[set border color=martinired]{tty}<|REV_PI_TTY_DEVICE|>\tikzmarkend{tty}<|, O_RDWR | O_NOCTTY, 0); |>\setcounter{lstnumber}{208}<|
	
	piIoComm_fd_m = fd;                                                      |>\tikzmarkend{fd}\setcounter{lstnumber}{217}<|

	return 0;
}
\end{lstlisting}

Der zum Schreiben auf die serielle Schnittstelle verwendete Datei-Deskriptor wird von der in Listing~\ref{lst:4-piIOComm_open_serial} abgebildeten Methode \lstinline{piIoComm_open_serial()} generiert. 

\begin{lstlisting}[language={c},firstnumber=45,caption={Definition der seriellen Schnittstelle in \lstinline{piIOComm.h}\label{lst:4-REV_PI_TTY_DEVICE}}]
#define REV_PI_TTY_DEVICE	"/dev/ttyAMA0"
\end{lstlisting}

Das in Listing~\ref{lst:4-REV_PI_TTY_DEVICE} definierte Macro verweist auf eine der seriellen Schnittstellen des RaspberryPi.
Die Implementierung des zugehörigen Schnittstellentreibers soll hier nicht weiter untersucht werden. Somit ist an dieser Stelle die Kette vom Setzen einer Variablen auf dem OPC-Server bis hin zur Aktualisierung des Prozessabbilds der IO-Module geschlossen.

% \begin{lstlisting}[language={c},firstnumber={226},caption={Setzen der Scheduler-Priorität auf SCHED\_FIFO in 
% revpi\_common.c\label{lst:2-sched_priority}}]
% param.sched_priority = ktprio->prio;
% ret = sched_setscheduler(child, SCHED_FIFO, &param);
% \end{lstlisting}
% % % Imports nur für Referenzenauflösung während des Schreibens! Vorm Kompilieren auskommentieren!
% \bibliography{0_hauptdatei}
% \input{1_einleitung}
% \input{2_grundlagen}
% \input{3_konzeption}
% \input{4_implementierung}
% \input{5_tests}
% \input{6_zusammenfassung}
% % Ende Imports

\section{Test des OPC-Servers im Gesamtsystem%
  \label{sec:5-tests}}

% % % Imports nur für Referenzenauflösung während des schreibens! Vorm Kompilieren auskommentieren!
% \bibliography{0_hauptdatei}
% \input{1_einleitung}
% \input{2_grundlagen}
% \input{3_konzeption}
% \input{4_implementierung}
% \input{5_tests}
% \input{6_zusammenfassung}
% % Ende Imports

\section{Zusammenfassung und Ausblick%
  \label{sec:6-fazit}}
Der folgende Abschnitt~\ref{sec:6-zusammenfassung} fasst die gewonnenen Erkenntnisse und den Stand der Implementierung zusammen.
Den Abschluss dieser Arbeit bildet der Ausblick in Abschnitt~\ref{sec:6-ausblick}.

\subsection{Zusammenfassung%
     \label{sec:6-zusammenfassung}}

\subsection{Ausblick%
     \label{sec:6-ausblick}}

% % Ende Imports

\section{Zusammenfassung und Ausblick%
  \label{sec:6-fazit}}
Der folgende Abschnitt~\ref{sec:6-zusammenfassung} fasst die gewonnenen Erkenntnisse und den Stand der Implementierung zusammen.
Den Abschluss dieser Arbeit bildet der Ausblick in Abschnitt~\ref{sec:6-ausblick}.

\subsection{Zusammenfassung%
     \label{sec:6-zusammenfassung}}

\subsection{Ausblick%
     \label{sec:6-ausblick}}

% % Ende Imports

\section{Grundlagen%
  \label{sec:2-grundlagen}}

\subsection{Speicherprogrammierbare-Steuerung und Linux -- Revolution Pi%
     \label{sec:2-sps}}

\subsubsection{Kunbus RevolutionPi%
        \label{sec:2-revpi}}
Der RevolutionPi 3 ist eine speicherprogrammierbare Steuerung (SPS) des Herstellers
Kunbus GmbH. Kern dieser SPS ist das von der Raspberry Pi Foundation entwickelte
und vertriebene Raspberry Pi Compute Module 3. Dieses integriert ein Broadcom BCM2837
System-on-Chip (SoC) mit vier 1,2GHz Prozessorkernen, 1GB RAM, 4GB eMMC Anwendungsspeicher
und sonstige Peripherie in ein Modul im DDR2-SODIMM Formfaktor. Diese Spezifikationen
sind weitgehend identisch zu denen des ausgesprochen populären Raspberry Pi 3.
Der Revolution Pi profitiert daher von dem gleichen großen Angebot an Software
und Unterstützung wie der Raspberry Pi, ergänzt dessen Hardware jedoch um eine 24V
Spannungsversorgung, die Möglichkeit der Erweiterung durch mehrere industrietaugliche
Ein-/ Ausgabemodule und Gateways sowie ein Gehäuse zur Montage auf einer DIN-Schiene.
\begin{itemize}
  \item{Prozessor: BCM2837}
  \item{Taktfrequenz 1,2 GHz}
  \item{Anzahl Prozessorkerne: 4}
  \item{Arbeitsspeicher: 1 GByte}
  \item{eMMC Flash Speicher: 4 GByte}
  \item{Betriebssystem: Angepasstes Raspbian mit RT-Patch}
  \item{RTC mit 24h Pufferung über wartungsfreien Kondensator}
  \item{Treiber / API: Treiber schreibt zyklisch Prozessdaten in ein Prozessabbild, Zugriff auf Prozessabbild über Linux-Filesystem als API zu Fremdsoftware.}
  \item{Kommunikationsanschlüsse: 2 x USB 2.0 A (je 500 mA belastbar), 1 x Micro-USB, HDMI, Ethernet (RJ45) 10/100 Mbit/s}
  \item{Stromversorgung: min. 10,7 V, max. 28,8 V, maximal 10 Watt}
  \item{Zulässige Umgebungstemperatur: -40 bis +55 C}
  \item{Gehäuseabmessungen: (HxBxL) 96 mm x 22,5 mm x 110,5 mm (ohne gesteckte Stecker)}
  \item{ESD Schutz: 4 kV / 8 kV gemäß EN61131-2 und IEC 61000-6-2}
  \item{Surge / Burst Prüfungen: gemäß EN61131-2 und IEC 61000-6-2 eingekoppelt auf Versorgungsspannung, Ethernet und IO-Leitungen}
  \item{EMI Prüfungen: gemäß EN61131-2 und IEC 61000-6-2}
\end{itemize}

Kunbus bietet eine Auswahl an IO- und Gateway-Modulen zur Erweiterung des Revolution Pi an.
Gateways dienen der Kommunikation mit Systemen oder Komponenten der Automatisierungstechnik
über Protokolle wie PROFIBUS oder EtherCAT. IO-Module erlauben die Überwachung
und Steuerung von digitalen oder analogen Ein- und Ausgängen.

\subsubsection{Zugriff auf IO-Module%
        \label{sec:2-io}}
Der Zugriff auf die Ein- und Ausgänge der IO-Module erfolgt über ein Prozessabbild
und einen hierfür von Kunbus bereitgestellten Treiber, genannt piControl. Dieser
aktualisiert das Prozessabbild zyklisch. Die angestrebte Zykluszeit beträgt 5ms,
kann jedoch je nach Anzahl der angeschlossenen Module auch größer sein. Kunbus
garantiert bei drei IO-Modulen und zwei Gateway-Modulen eine Zykluszeit von 10 ms.
Jedes der IO-Module stellt ein eigenständiges eingebettetes System dar. Es verfügt
über einen Microcontroller, welcher die IOs bereitstellt und über einen RS485-Bus
mit dem Revolution Pi kommuniziert.
% https://revolution.kunbus.de/io-modul/

Lizenz: GPL
% https://github.com/RevolutionPi/piControl

\begin{lstlisting}[language={c},firstnumber={226},caption={Setzen der Scheduler-Priorität auf SCHED\_FIFO in revpi\_common.c\label{lst:2-sched_priority}}]
param.sched_priority = ktprio->prio;
ret = sched_setscheduler(child, SCHED_FIFO,
       &param);
\end{lstlisting}


\subsection{Echtzeit und Multithreading unter Linux -- preemptRT und posix%
     \label{sec:2-echtzeit}}


 Der Linux-Kernel verfügt über mehrere unterschiedliche Preemtion-Modelle:

\begin{itemize}
  \item No Forced Preemption (server):
  Ausgelegt auf maximal möglichen Durchsatz, lediglich Interrupts und
  System-Call-Returns bewirken Präemption.

  \item Voluntary Kernel Preemption (Desktop):
  Neben den implizit bevorrechtigten Interrupts und System-Call-Returns gibt es
  in diesem Modell weitere Abschnitte des Kernels in welchen Preämption explizit
  gestattet ist.

  \item Preemptible Kernel (Low-Latency Desktop):
  In diesem Modell ist der gesamte Kernel, mit Ausnahme sog.~kritischer Abschnitte
  präemptible. Nach jedem kritischen Abschnitt gibt es einen impliziten Präemptions-Punkt.

  \item Preemptible Kernel (Basic RT):
  Dieses Modell ist dem zuvor genannten sehr ähnlich, hier sind jedoch alle Interrupt-Handler
  als eigenständige Threads ausgeführt.

  \item Fully Preemptible Kernel (RT):
  Wie auch bei den beiden zuvor genannten Modellen ist hier der gesamte Kernel
  präemtible, die Anzahl und Dauer der nicht-präemtiblen kritischen Abschnitte
  ist auf ein notwendiges Minimum beschränkt. Alle Interrupt-Handler sind als
  eigenständige Threads ausgeführt, Spinlocks durch Sleeping-Spinlocks und Mutexe
  durch sog.~RT-Mutexe ersetzt.

\end{itemize}
\todo{Spinlocks und Mutexe sowie die RT-Varianten dieser erklären!}

Lediglich mit dem vollständig präemtiblen Kernel kann Echtzeit-Verhalten realisiert werden.

% https://wiki.linuxfoundation.org/realtime/documentation/technical_basics/preemption_models bzw kernel/Kconfig.preempt

\subsubsection{preemptRT%
        \label{sec:2-preemptRT}}
% https://wiki.linuxfoundation.org/realtime/documentation/technical_details/start
% https://wiki.linuxfoundation.org/realtime/documentation/technical_basics/start

Das dem PREEMPT RT Kernel zugrunde liegende Prinzip lässt sich in einer einfachen
Regel ausdrücken: Nur Code, welcher absolut nicht-präemtible sein darf, ist es
gestattet nicht-präemtible zu sein.
Das erklärte Ziel des PREEMPT\_RT Patches ist es folglich, die Menge des nicht-präemtiblen
Codes im Linux-Kernel auf das absolut notwendige Minimum zu reduzieren.

Dies wird durch Verwendung folgender Mechanismen erreicht:

\begin{itemize}
  \item Hochauflösende Timer
  \item Sleeping Spinlocks
  \item Threaded Interrupt Handlers
  \item rt\_mutex
  \item RCU
\end{itemize}


\subsubsection{posix%
        \label{sec:2-posix}}
Ist posix hier wirklich relevant? Debian bzw.~Raspbian sind weitgehend posix
kompatibel, aber wird es hier genutzt? -> JA, open62541 nutzt pthread.h
piControl nutzt kthread.h, und semaphore.h

\subsection{OPC-UA und open62541%
     \label{sec:2-opc}}

\subsubsection{OPC UA%
        \label{sec:2-opcua}}
Open Platform Communications (OPC) ist eine Familie von Standards zur herstellerunabhängigen
Kommunikation von Maschinen (M2M) in der Automatisierungstechnik. Die sog.~OPC Task Force, zu deren
Mitgliedern verschiedene große Firmen der Automatisierungsindustrie gehören, veröffentlichte
die OPC Specification Version 1.0 im August 1996.
Motiviert ist dieser offene Standard durch die Erkenntniss, dass die Anpassung der
zahlreichen Herstellerstandards an individuelle Infrastrukturen und Anlagen einen
großen Mehraufwand verursachen.
Die Wikipedia beschreibt das Anwendungsgebiet für OPC wie folgt:

\glqq{}OPC wird dort eingesetzt, wo Sensoren, Regler und Steuerungen verschiedener Hersteller
ein gemeinsames Netzwerk bilden. Ohne OPC benötigten zwei Geräte zum Datenaustausch
genaue Kenntnis über die Kommunikationsmöglichkeiten des Gegenübers. Erweiterungen
und Austausch gestalten sich entsprechend schwierig. Mit OPC genügt es, für jedes
Gerät genau einmal einen OPC-konformen Treiber zu schreiben. Idealerweise wird
dieser bereits vom Hersteller zur Verfügung gestellt. Ein OPC-Treiber lässt sich
ohne großen Anpassungsaufwand in beliebig große Steuer- und Überwachungssysteme
integrieren.

OPC unterteilt sich in verschiedene Unterstandards, die für den jeweiligen Anwendungsfall
unabhängig voneinander implementiert werden können. OPC lässt sich damit verwenden
für Echtzeitdaten (Überwachung), Datenarchivierung, Alarm-Meldungen und neuerdings
auch direkt zur Steuerung (Befehlsübermittlung).\grqq{}

OPC basiert in der ursprünglichen Spezifikation auf Microsofts DCOM-Spezifikation.
DCOM macht Funktionen und Objekte einer Anwendung anderen Anwendungen im Netzwerk
zugänglich. Der OPC-Standard definiert entsprechende DCOM-Objekte um mit anderen
OPC-Anwendungen Daten austauschen zu können. Die Verwendung von DCOM bindet Anwender
an Betriebssysteme von Microsoft. Die ursprüngliche OPC Spezifikation wird durch die
Entwicklung von OPC Unified Architecture (OPC UA) abgelöst.
OPC UA setzt auf einem eigenen Kommunikationionsstack auf, die Verwendung von DCOM
und damit die Bindung an Microsoft wurden aufgelöst.

Die OPC-UA-Architektur ist eine Service-orientierte Architektur (SOA), deren Struktur
aus mehreren Schichten besteht.

% Wikipedia
Das OPC-Informationsmodell ist nicht mehr nur eine Hierarchie aus Ordnern, Items
und Properties. Es ist ein sogenanntes Full-Mesh-Network aus Nodes, mit dem neben
den Nutzdaten eines Nodes auch Meta- und Diagnoseinformationen repräsentiert werden.
Ein Node ähnelt einem Objekt aus der objektorientierten Programmierung. Ein Node
kann Attribute besitzen, die gelesen werden können (Data Access (DA), Historical
Data Access (HDA)). Es ist möglich Methoden zu definieren und aufzurufen.
Eine Methode besitzt Aufrufargumente und Rückgabewerte. Sie wird durch ein Command
aufgerufen. Weiterhin werden Events unterstützt, die versendet werden können
(AE (Alarms \& Events), DA DataChange), um bestimmte Informationen zwischen Geräten
auszutauschen. Ein Event besitzt unter anderem einen Empfangszeitpunkt, eine Nachricht
und einen Schweregrad. Die o. g. Nodes werden sowohl für die Nutzdaten als auch
alle anderen Arten von Metadaten verwendet. Der damit modellierte OPC-Adressraum
beinhaltet nun auch ein Typmodell, mit dem sämtliche Datentypen spezifiziert werden.

% https://de.wikipedia.org/wiki/Open_Platform_Communications
% https://de.wikipedia.org/wiki/OPC_Unified_Architecture
% https://opcfoundation.org/developer-tools/specifications-unified-architecture
% Von Gerhard Gappmeier - ascolab GmbH, CC BY-SA 3.0, https://de.wikipedia.org/w/index.php?curid=1892069
\subsubsection{open62541%
        \label{sec:2-open62541}}
open62541 ist eine offene und freie Implementierung von OPC UA. Die in C geschriebene
Bibliothek stellt eine beständig zunehmende Anzahl der im OPC UA Standard definierten
Funktionen bereit. Sie kann sowohl zur Erstellung von OPC-Servern als auch -Clients
genutzt werden. Ergänzend zu der unter der Mozilla Public License v2.0 lizensierten
Bibliothek stellt das open62541 Projekt auch Beispielprogramme unter einer CC0 Lizenz
zur Verfügung.

Die Bibliothek eignet sich auch für die Entwicklung auf eingebetteten Systemen und
Microcontrollern. Je nach Umfang der gewünschten Funktionen und des OPC Informationsmodells
beträgt die Größe einer Server-Binary weniger als 100kb. %evtl. kürzen?

\todo{Nodes erklären! Evtl.~oben!}

Folgende Auswahl an Eigenschaften und Funktionen zeichnet die in dieser Arbeit verwendete
Version 0.3 von open62541 aus:
\begin{itemize}
  \item Kommunikationionsstack
  \begin{itemize}
      \item OPC UA Binär-Protokoll (HTTP oder SOAP werden gegenwärtig nicht unterstützt)
      \item Austauschbare Netzwerk-Schicht, welche die Verwendung eigener Netzwerk-APIs
      erlaubt.
      \item Verschlüsselte Kommunikationion
      \item Asynchrone Dienst-Anfragen im Client
  \end{itemize}
  \item Informationsmodell
  \begin{itemize}
    \item Unterstützung aller OPC UA Node-Typen, inkl.~Methoden
    \item Hinzufügen und Entfernen von Nodes und Referenzen zur Laufzeit.
    \item Vererbung und Instanziierung von Objekt- und Variablentypen
    \item Zugriffskontrolle auch für einzelne Nodes
  \end{itemize}
  \item Subscriptions
  \begin{itemize}
    \item Erlaubt die Überwachung (subscriptions / monitoreditems)
    \item Sehr geringer Ressourcenbedarf pro überwachtem Wert
  \end{itemize}
  \item Code-Generierung auf XML-Basis
  \begin{itemize}
    \item Erlaubt die Erstellung von Datentypen
    \item Erlaubt die Generierung des serverseitigen Informationsmodells
  \end{itemize}
\end{itemize}

% https://open62541.org/doc/0.3/


Mozilla Public License
CC0 Lizenz für Beispiele und Plugins

% https://open62541.org/doc/open62541-current.pdf
% https://open62541.org/

% % % Imports nur für Referenzenauflösung während des Schreibens! Vorm Kompilieren auskommentieren!
% \bibliography{0_hauptdatei}
% % Mit \section{...} eröffnen wir einen neuen Abschnitt.
% Der Befehl setzt nicht nur den Text in einer größeren,
% fetten Schrift, sondern sorgt außerdem dafür, daß er im
% Inhaltsverzeichnis erscheint.
%
% Mit \label{...} erzeugen wir einen Bezeichner, mit dessen Hilfe
% wir später auf die Nummer des Abschnitts verweisen können (nämlich
% mit~\ref{...}).
%
% Das Kommentarzeichen hinter „Übersicht“ dient dazu, ein
% Leerzeichen zwischen „Übersicht“ und dem \label-Befehl
% zu vermeiden, das andernfalls sichtbar würde – z.B. im
% Inhaltsverzeichnis.
%

% % Imports nur für Referenzenauflösung während des Schreibens! Vorm Kompilieren auskommentieren!
% \bibliography{0_hauptdatei}
% % Mit \section{...} eröffnen wir einen neuen Abschnitt.
% Der Befehl setzt nicht nur den Text in einer größeren,
% fetten Schrift, sondern sorgt außerdem dafür, daß er im
% Inhaltsverzeichnis erscheint.
%
% Mit \label{...} erzeugen wir einen Bezeichner, mit dessen Hilfe
% wir später auf die Nummer des Abschnitts verweisen können (nämlich
% mit~\ref{...}).
%
% Das Kommentarzeichen hinter „Übersicht“ dient dazu, ein
% Leerzeichen zwischen „Übersicht“ und dem \label-Befehl
% zu vermeiden, das andernfalls sichtbar würde – z.B. im
% Inhaltsverzeichnis.
%

% % Imports nur für Referenzenauflösung während des Schreibens! Vorm Kompilieren auskommentieren!
% \bibliography{0_hauptdatei}
% \input{1_einleitung}
%\input{2_grundlagen}
%\input{3_konzeption}
%\input{4_implementierung}
%\input{5_tests}
%\input{6_zusammenfassung}
% % Ende Imports

\section{Einleitung und Motivation%
  \label{sec:1-einleitung}}
Ziel dieses Projektes ist die Integration eines OPC-Servers mit einer auf Linux
basierenden speicherprogrammierbaren Steuerung (SPS). Angeschlossen an diese SPS
ist jeweils ein digitales Ein-/\,bzw.~Ausgabemodul. Die von diesen bereitgestellten
Ein-/\, bzw.~Ausgänge (IO) sollen in der Datenstruktur des OPC-Servers abgebildet
und über diesen für OPC-Clients les-/\,und schreibar sein. Weiterhin sollen einige
Funktionen zur Überwachung und Steuerung der an die SPS angeschlossenen Aktoren
und Sensoren direkt im OPC-Server implementiert werden.
Hiermit stellt dieses Projekt eine der Grundlagen für ein übergeordnetes Projekt,
die cloudbasierte Steuerung eines miniaturisierten Produktions-Systems, dar.

Der hier verwendete OPC-Server ist Teil des sog. open62541 Projekts. Er ist in C
geschrieben und implementiert bereits einen großen Teil der im OPC-UA-Standard
spezifizierten Funktionen.
Als SPS findet ein Revolution Pi 3 der Firma Kunbus Verwendung. Dieser integriert
ein sog. Compute Module der Raspberry Pi Foundation in ein industrietaugliches
Gehäuse und erlaubt die Erweiterung mittels IO- oder Gateway-Modulen. Über diese
erfolgt die Kommunikation mit weiteren Komponenten der Automatisierungstechnik.

Motiviert ist dieses Projekt durch die Beobachtung, dass die Verbreitung offener
Standards sowie freier Software auch in der Automatisierungstechnik zunimmt.
Linux ist ein freies Betriebssystem, OPC-UA ein offen zugänglicher, aktiv gepflegter
und weit verbreiteter Standard. Der Raspberry Pi findet sowohl bei Hobby-Anwendern als
auch in den Bereichen Forschung und Entwicklung sowie bei industriellen Anwendern
Verwendung. Dieses Projekt stellt somit eine für unterschiedliche Anwender interessante
Entwicklung dar.

Im Anschluss an diese einleitende Übersicht im Abschnitt~\ref{sec:1-einleitung} folgt
die Darstellung der wichtigsten Grundlagen in Abschnitt~\ref{sec:2-grundlagen}.
Aufbauend auf diesen Grundlagen folgt die konzeptuelle Ausarbeitung im Abschnitt~\ref{sec:3-konzeption}.
Die Umsetzung wird im Abschnitt~\ref{sec:4-implementierung} erläutert.
Die Leistungsfähigkeit der Implementierung wird in Abschnitt~\ref{sec:5-tests} untersucht.
Eine Zusammenfassung und ein Ausblick schließen die Arbeit in
Abschnitt~\ref{sec:6-fazit} ab. Eventuell noch benötigte Anhänge
finden sich in den Anhängen [...] bis [...].

%% % Imports nur für Referenzenauflösung während des Schreibens! Vorm Kompilieren auskommentieren!
% \bibliography{0_hauptdatei}
% \input{1_einleitung}
% \input{2_grundlagen}
% \input{3_konzeption}
% \input{4_implementierung}
% \input{5_tests}
% \input{6_zusammenfassung}
% % Ende Imports

\section{Grundlagen%
  \label{sec:2-grundlagen}}

\subsection{Speicherprogrammierbare-Steuerung und Linux -- Revolution Pi%
     \label{sec:2-sps}}

\subsubsection{Kunbus RevolutionPi%
        \label{sec:2-revpi}}
Der RevolutionPi 3 ist eine speicherprogrammierbare Steuerung (SPS) des Herstellers
Kunbus GmbH. Kern dieser SPS ist das von der Raspberry Pi Foundation entwickelte
und vertriebene Raspberry Pi Compute Module 3. Dieses integriert ein Broadcom BCM2837
System-on-Chip (SoC) mit vier 1,2GHz Prozessorkernen, 1GB RAM, 4GB eMMC Anwendungsspeicher
und sonstige Peripherie in ein Modul im DDR2-SODIMM Formfaktor. Diese Spezifikationen
sind weitgehend identisch zu denen des ausgesprochen populären Raspberry Pi 3.
Der Revolution Pi profitiert daher von dem gleichen großen Angebot an Software
und Unterstützung wie der Raspberry Pi, ergänzt dessen Hardware jedoch um eine 24V
Spannungsversorgung, die Möglichkeit der Erweiterung durch mehrere industrietaugliche
Ein-/ Ausgabemodule und Gateways sowie ein Gehäuse zur Montage auf einer DIN-Schiene.
\begin{itemize}
  \item{Prozessor: BCM2837}
  \item{Taktfrequenz 1,2 GHz}
  \item{Anzahl Prozessorkerne: 4}
  \item{Arbeitsspeicher: 1 GByte}
  \item{eMMC Flash Speicher: 4 GByte}
  \item{Betriebssystem: Angepasstes Raspbian mit RT-Patch}
  \item{RTC mit 24h Pufferung über wartungsfreien Kondensator}
  \item{Treiber / API: Treiber schreibt zyklisch Prozessdaten in ein Prozessabbild, Zugriff auf Prozessabbild über Linux-Filesystem als API zu Fremdsoftware.}
  \item{Kommunikationsanschlüsse: 2 x USB 2.0 A (je 500 mA belastbar), 1 x Micro-USB, HDMI, Ethernet (RJ45) 10/100 Mbit/s}
  \item{Stromversorgung: min. 10,7 V, max. 28,8 V, maximal 10 Watt}
  \item{Zulässige Umgebungstemperatur: -40 bis +55 C}
  \item{Gehäuseabmessungen: (HxBxL) 96 mm x 22,5 mm x 110,5 mm (ohne gesteckte Stecker)}
  \item{ESD Schutz: 4 kV / 8 kV gemäß EN61131-2 und IEC 61000-6-2}
  \item{Surge / Burst Prüfungen: gemäß EN61131-2 und IEC 61000-6-2 eingekoppelt auf Versorgungsspannung, Ethernet und IO-Leitungen}
  \item{EMI Prüfungen: gemäß EN61131-2 und IEC 61000-6-2}
\end{itemize}

Kunbus bietet eine Auswahl an IO- und Gateway-Modulen zur Erweiterung des Revolution Pi an.
Gateways dienen der Kommunikation mit Systemen oder Komponenten der Automatisierungstechnik
über Protokolle wie PROFIBUS oder EtherCAT. IO-Module erlauben die Überwachung
und Steuerung von digitalen oder analogen Ein- und Ausgängen.

\subsubsection{Zugriff auf IO-Module%
        \label{sec:2-io}}
Der Zugriff auf die Ein- und Ausgänge der IO-Module erfolgt über ein Prozessabbild
und einen hierfür von Kunbus bereitgestellten Treiber, genannt piControl. Dieser
aktualisiert das Prozessabbild zyklisch. Die angestrebte Zykluszeit beträgt 5ms,
kann jedoch je nach Anzahl der angeschlossenen Module auch größer sein. Kunbus
garantiert bei drei IO-Modulen und zwei Gateway-Modulen eine Zykluszeit von 10 ms.
Jedes der IO-Module stellt ein eigenständiges eingebettetes System dar. Es verfügt
über einen Microcontroller, welcher die IOs bereitstellt und über einen RS485-Bus
mit dem Revolution Pi kommuniziert.
% https://revolution.kunbus.de/io-modul/

Lizenz: GPL
% https://github.com/RevolutionPi/piControl

\begin{lstlisting}[language={c},firstnumber={226},caption={Setzen der Scheduler-Priorität auf SCHED\_FIFO in revpi\_common.c\label{lst:2-sched_priority}}]
param.sched_priority = ktprio->prio;
ret = sched_setscheduler(child, SCHED_FIFO,
       &param);
\end{lstlisting}


\subsection{Echtzeit und Multithreading unter Linux -- preemptRT und posix%
     \label{sec:2-echtzeit}}


 Der Linux-Kernel verfügt über mehrere unterschiedliche Preemtion-Modelle:

\begin{itemize}
  \item No Forced Preemption (server):
  Ausgelegt auf maximal möglichen Durchsatz, lediglich Interrupts und
  System-Call-Returns bewirken Präemption.

  \item Voluntary Kernel Preemption (Desktop):
  Neben den implizit bevorrechtigten Interrupts und System-Call-Returns gibt es
  in diesem Modell weitere Abschnitte des Kernels in welchen Preämption explizit
  gestattet ist.

  \item Preemptible Kernel (Low-Latency Desktop):
  In diesem Modell ist der gesamte Kernel, mit Ausnahme sog.~kritischer Abschnitte
  präemptible. Nach jedem kritischen Abschnitt gibt es einen impliziten Präemptions-Punkt.

  \item Preemptible Kernel (Basic RT):
  Dieses Modell ist dem zuvor genannten sehr ähnlich, hier sind jedoch alle Interrupt-Handler
  als eigenständige Threads ausgeführt.

  \item Fully Preemptible Kernel (RT):
  Wie auch bei den beiden zuvor genannten Modellen ist hier der gesamte Kernel
  präemtible, die Anzahl und Dauer der nicht-präemtiblen kritischen Abschnitte
  ist auf ein notwendiges Minimum beschränkt. Alle Interrupt-Handler sind als
  eigenständige Threads ausgeführt, Spinlocks durch Sleeping-Spinlocks und Mutexe
  durch sog.~RT-Mutexe ersetzt.

\end{itemize}
\todo{Spinlocks und Mutexe sowie die RT-Varianten dieser erklären!}

Lediglich mit dem vollständig präemtiblen Kernel kann Echtzeit-Verhalten realisiert werden.

% https://wiki.linuxfoundation.org/realtime/documentation/technical_basics/preemption_models bzw kernel/Kconfig.preempt

\subsubsection{preemptRT%
        \label{sec:2-preemptRT}}
% https://wiki.linuxfoundation.org/realtime/documentation/technical_details/start
% https://wiki.linuxfoundation.org/realtime/documentation/technical_basics/start

Das dem PREEMPT RT Kernel zugrunde liegende Prinzip lässt sich in einer einfachen
Regel ausdrücken: Nur Code, welcher absolut nicht-präemtible sein darf, ist es
gestattet nicht-präemtible zu sein.
Das erklärte Ziel des PREEMPT\_RT Patches ist es folglich, die Menge des nicht-präemtiblen
Codes im Linux-Kernel auf das absolut notwendige Minimum zu reduzieren.

Dies wird durch Verwendung folgender Mechanismen erreicht:

\begin{itemize}
  \item Hochauflösende Timer
  \item Sleeping Spinlocks
  \item Threaded Interrupt Handlers
  \item rt\_mutex
  \item RCU
\end{itemize}


\subsubsection{posix%
        \label{sec:2-posix}}
Ist posix hier wirklich relevant? Debian bzw.~Raspbian sind weitgehend posix
kompatibel, aber wird es hier genutzt? -> JA, open62541 nutzt pthread.h
piControl nutzt kthread.h, und semaphore.h

\subsection{OPC-UA und open62541%
     \label{sec:2-opc}}

\subsubsection{OPC UA%
        \label{sec:2-opcua}}
Open Platform Communications (OPC) ist eine Familie von Standards zur herstellerunabhängigen
Kommunikation von Maschinen (M2M) in der Automatisierungstechnik. Die sog.~OPC Task Force, zu deren
Mitgliedern verschiedene große Firmen der Automatisierungsindustrie gehören, veröffentlichte
die OPC Specification Version 1.0 im August 1996.
Motiviert ist dieser offene Standard durch die Erkenntniss, dass die Anpassung der
zahlreichen Herstellerstandards an individuelle Infrastrukturen und Anlagen einen
großen Mehraufwand verursachen.
Die Wikipedia beschreibt das Anwendungsgebiet für OPC wie folgt:

\glqq{}OPC wird dort eingesetzt, wo Sensoren, Regler und Steuerungen verschiedener Hersteller
ein gemeinsames Netzwerk bilden. Ohne OPC benötigten zwei Geräte zum Datenaustausch
genaue Kenntnis über die Kommunikationsmöglichkeiten des Gegenübers. Erweiterungen
und Austausch gestalten sich entsprechend schwierig. Mit OPC genügt es, für jedes
Gerät genau einmal einen OPC-konformen Treiber zu schreiben. Idealerweise wird
dieser bereits vom Hersteller zur Verfügung gestellt. Ein OPC-Treiber lässt sich
ohne großen Anpassungsaufwand in beliebig große Steuer- und Überwachungssysteme
integrieren.

OPC unterteilt sich in verschiedene Unterstandards, die für den jeweiligen Anwendungsfall
unabhängig voneinander implementiert werden können. OPC lässt sich damit verwenden
für Echtzeitdaten (Überwachung), Datenarchivierung, Alarm-Meldungen und neuerdings
auch direkt zur Steuerung (Befehlsübermittlung).\grqq{}

OPC basiert in der ursprünglichen Spezifikation auf Microsofts DCOM-Spezifikation.
DCOM macht Funktionen und Objekte einer Anwendung anderen Anwendungen im Netzwerk
zugänglich. Der OPC-Standard definiert entsprechende DCOM-Objekte um mit anderen
OPC-Anwendungen Daten austauschen zu können. Die Verwendung von DCOM bindet Anwender
an Betriebssysteme von Microsoft. Die ursprüngliche OPC Spezifikation wird durch die
Entwicklung von OPC Unified Architecture (OPC UA) abgelöst.
OPC UA setzt auf einem eigenen Kommunikationionsstack auf, die Verwendung von DCOM
und damit die Bindung an Microsoft wurden aufgelöst.

Die OPC-UA-Architektur ist eine Service-orientierte Architektur (SOA), deren Struktur
aus mehreren Schichten besteht.

% Wikipedia
Das OPC-Informationsmodell ist nicht mehr nur eine Hierarchie aus Ordnern, Items
und Properties. Es ist ein sogenanntes Full-Mesh-Network aus Nodes, mit dem neben
den Nutzdaten eines Nodes auch Meta- und Diagnoseinformationen repräsentiert werden.
Ein Node ähnelt einem Objekt aus der objektorientierten Programmierung. Ein Node
kann Attribute besitzen, die gelesen werden können (Data Access (DA), Historical
Data Access (HDA)). Es ist möglich Methoden zu definieren und aufzurufen.
Eine Methode besitzt Aufrufargumente und Rückgabewerte. Sie wird durch ein Command
aufgerufen. Weiterhin werden Events unterstützt, die versendet werden können
(AE (Alarms \& Events), DA DataChange), um bestimmte Informationen zwischen Geräten
auszutauschen. Ein Event besitzt unter anderem einen Empfangszeitpunkt, eine Nachricht
und einen Schweregrad. Die o. g. Nodes werden sowohl für die Nutzdaten als auch
alle anderen Arten von Metadaten verwendet. Der damit modellierte OPC-Adressraum
beinhaltet nun auch ein Typmodell, mit dem sämtliche Datentypen spezifiziert werden.

% https://de.wikipedia.org/wiki/Open_Platform_Communications
% https://de.wikipedia.org/wiki/OPC_Unified_Architecture
% https://opcfoundation.org/developer-tools/specifications-unified-architecture
% Von Gerhard Gappmeier - ascolab GmbH, CC BY-SA 3.0, https://de.wikipedia.org/w/index.php?curid=1892069
\subsubsection{open62541%
        \label{sec:2-open62541}}
open62541 ist eine offene und freie Implementierung von OPC UA. Die in C geschriebene
Bibliothek stellt eine beständig zunehmende Anzahl der im OPC UA Standard definierten
Funktionen bereit. Sie kann sowohl zur Erstellung von OPC-Servern als auch -Clients
genutzt werden. Ergänzend zu der unter der Mozilla Public License v2.0 lizensierten
Bibliothek stellt das open62541 Projekt auch Beispielprogramme unter einer CC0 Lizenz
zur Verfügung.

Die Bibliothek eignet sich auch für die Entwicklung auf eingebetteten Systemen und
Microcontrollern. Je nach Umfang der gewünschten Funktionen und des OPC Informationsmodells
beträgt die Größe einer Server-Binary weniger als 100kb. %evtl. kürzen?

\todo{Nodes erklären! Evtl.~oben!}

Folgende Auswahl an Eigenschaften und Funktionen zeichnet die in dieser Arbeit verwendete
Version 0.3 von open62541 aus:
\begin{itemize}
  \item Kommunikationionsstack
  \begin{itemize}
      \item OPC UA Binär-Protokoll (HTTP oder SOAP werden gegenwärtig nicht unterstützt)
      \item Austauschbare Netzwerk-Schicht, welche die Verwendung eigener Netzwerk-APIs
      erlaubt.
      \item Verschlüsselte Kommunikationion
      \item Asynchrone Dienst-Anfragen im Client
  \end{itemize}
  \item Informationsmodell
  \begin{itemize}
    \item Unterstützung aller OPC UA Node-Typen, inkl.~Methoden
    \item Hinzufügen und Entfernen von Nodes und Referenzen zur Laufzeit.
    \item Vererbung und Instanziierung von Objekt- und Variablentypen
    \item Zugriffskontrolle auch für einzelne Nodes
  \end{itemize}
  \item Subscriptions
  \begin{itemize}
    \item Erlaubt die Überwachung (subscriptions / monitoreditems)
    \item Sehr geringer Ressourcenbedarf pro überwachtem Wert
  \end{itemize}
  \item Code-Generierung auf XML-Basis
  \begin{itemize}
    \item Erlaubt die Erstellung von Datentypen
    \item Erlaubt die Generierung des serverseitigen Informationsmodells
  \end{itemize}
\end{itemize}

% https://open62541.org/doc/0.3/


Mozilla Public License
CC0 Lizenz für Beispiele und Plugins

% https://open62541.org/doc/open62541-current.pdf
% https://open62541.org/

%% % Imports nur für Referenzenauflösung während des Schreibens! Vorm Kompilieren auskommentieren!
% \bibliography{0_hauptdatei}
% \input{1_einleitung}
% \input{2_grundlagen}
% \input{3_konzeption}
% \input{4_implementierung}
% \input{5_tests}
% \input{6_zusammenfassung}
% \input{anhang}
% % Ende Imports

\section{Systemkonzept%
  \label{sec:3-konzeption}}
Auf Basis der in Abschnitt \ref{sec:2-grundlagen} vorgestellten Möglichkeiten folgt nun die Ausarbeitung eines Konzepts.
In den folgenden Abschnitten soll näher auf zwei zentrale Aspekte eingegangen werden: Abschnitt~\ref{sec:3-anbindung} stellt Möglichkeiten zum Zugriff auf Variablen bzw.\,Werte im Prozessabbild des Revolution Pi vor; in Abschnitt~\ref{sec:3-integration} wird ein Konzept zur Bereitstellung dieser Variablen auf einem OPC-Server vorgestellt.

\subsection{Anbindung der IO an den OPC-Server%
     \label{sec:3-anbindung}}

Eine Webanwendung mit Bezeichnung PiCtory dient zur Konfiguration der I/O- und virtuellen Module des RevolutionPi. Die Konfiguration liegt im JSON-Format in der Datei \lstinline{/etc/revpi/config.rsc}. Der piControl-Treiber liest diese Datei beim Start. 
Der folgende Auszug aus der Manpage des piControl-Kernelmoduls beschreibt die von diesem zum Lesen und Schreiben einzelner Bits des Prozessabbildes bereitgestellten Funktionen~\citep[vgl.]{web-revpi-manpage}. Sie ist an dieser Stelle weitgehend ungekürzt zitiert, da sie die nutzbare Schnittstelle sehr kompakt beschreibt.

\begin{lstlisting}[breakindent=0pt, numbers=none, caption={Auszug aus der Revolution Pi Programmers Manual\label{lst:4-manpage}}]
KB_FIND_VARIABLE SPIVariable *argp
Find a variable in the process image by its name. A pointer to a structure of type SPIVariable must be passed as argument. [...]
The struct SPIVariable [...] is defined as 
typedef struct SPIVariableStr
{
    char strVarName[32]; // Variable name
    uint16_t i16uAddress; // Address of the byte in the process image
    uint8_t i8uBit; // 0-7 bit position, >= 8 whole byte
    uint16_t i16uLength; // length of the variable in bits.
    // Possible values are 1, 8, 16 and 32
} SPIVariable;

Set and get values of the process image
KB_GET_VALUE SPIValue *argp
[...]
KB_SET_VALUE SPIValue *argp
Write one bit or one byte to the process image [...].  This call is more efficient than the usual calls of seek and write because only one function call is necessary. If more than on application are writing bits in one output byte, this call is the only safe way to set a bit without overwriting the other bits because this call is doing a read-modify-write-cycle. 

The struct SPIValue used by this ioctl is defined as
typedef struct SPIValueStr
{
    uint16_t i16uAddress; // Address of the byte in the process image
    uint8_t i8uBit; // 0-7 bit position, >= 8 whole byte
    uint8_t i8uValue; // Value: 0/1 for bit access, whole byte otherwise
} SPIValue;
\end{lstlisting} 

Die oben beschriebenden Funtkionen \lstinline{KB_FIND_VARIABLE}, \lstinline{KB_GET_VALUE} und \lstinline{KB_SET_VALUE} ermöglichen einen einfachen und (lt.\,Manpage) effizienten Zugriff auf einzelne Bits des Prozessabbildes und damit auch auf die IO des RevolutionPi.
Der Zugriff des OPC-Servers auf das Prozessabbild soll daher mittels dieser Funktionen realisiert werden.
\lstinline{KB_FIND_VARIABLE} kann genutzt werden, um Adressen von Variablen im Prozessabbild mittels ihres Namens aufzulösen.
\lstinline{KB_GET_VALUE} und \lstinline{KB_SET_VALUE} ermöglichen den Zugriff auf die Werte dieser Variablen.


\subsection{Integration des OPC-Servers in das System%
     \label{sec:3-integration}}

open62541 bietet drei Möglichkeiten zum Abgleich von Variablen mit dem Prozessabbild~\citep[vgl.][Tutorials - Connecting a Variable with a Physical Process]{web-open62541}:
\begin{itemize}
    \item Manuelles oder zyklisches Aktualisieren
    \item Variable Value Callback
    \item Variable Datasource
\end{itemize}

Die zyklische Aktualisierung eines oder mehrerer Werte nimmt, abhängig von der Zykluszeit, viele Systemressourcen in Anspruch. Value Callbacks ermöglichen es, einen Variablenwert effizienter mit einer Ressource wie etwa einem Prozessabbild zu synchronisieren. An die Variable wird ein Callback angehängt, welches vor jedem Lesen und nach jedem Schreibvorgang ausgeführt wird.
Der Wert der Variablen wird weiterhin im Variablenknoten auf dem OPC-Server gespeichert, der Abgleich mit der verknüpften Ressource erfolgt durch die Callback-Methoden.

Sogenannte Datenquellen gehen noch einen Schritt weiter. Der Server leitet jede Lese- und Schreibanforderung direkt an eine Callback-Funktion weiter. Beim Lesen liefert der Rückruf eine Kopie des aktuellen Wertes. Die Datenquelle muss intern ein eigenes Speichermanagement implementieren.

Der Zugriff auf die Werte des Prozessabbildes erfolgt, wie in Abschnitt~\ref{sec:3-anbindung} beschrieben, über von piControl bereitgestellte Methoden. Um die durch open62541 gepflegte OPC-Datenstruktur und das durch piControl verwaltete Prozessabbild möglichst effektiv verknüpfen zu können, soll diese Interaktion mittels Datenquellen und den zugehörigen Callbacks implementiert werden.
%% % Imports nur für Referenzenauflösung während des Schreibens! Vorm Kompilieren auskommentieren!
% \bibliography{0_hauptdatei}
% \input{1_einleitung}
% \input{2_grundlagen}
% \input{3_konzeption}
% \input{4_implementierung}
% \input{5_tests}
% \input{6_zusammenfassung}
% \input{anhang}
% % Ende Imports

\section{Implementierung%
  \label{sec:4-implementierung}}
Das folgende Kapitel stellt in Auszügen die Implementierung des OPC-Servers sowie die Anbindung an die IO-Module
der SPS dar. Der Schwerpunkt liegt hierbei auf der Funktionsweise des piControl-Treibers und dessen Integration in das Projekt. Abschnitt~\ref{sec:4-picontrol} erklärt die zum Schreibens eines Bits verwendeten Funktionsaufrufe.
Zuvor soll jedoch in Abschnitt~\ref{sec:4-open62541} der Teil des OPC-Servers vorgestellt werden, welcher auf besagten Treiber zugreift. 

\subsection{Implementierung des OPC-Servers%
     \label{sec:4-open62541}}
Wie im vorangegangenen Abschnitt~\ref{sec:3-integration} begründet, soll die Verknüpfung zwischen dem Prozessabbild der SPS und den auf dem OPC-Server bereitgestellten Werten über sog.\,Datenquellen erfolgen. Hierzu ist zunächst eine Callback-Methode zu implementieren, welche bei einem Lese- oder Schreibzugriff auf eine Variable aufgerufen wird. Die Verknüpfung zwischen Callback-Methode und Variable muss manuell erfolgen.

\begin{lstlisting}[language={c},firstnumber=237,caption={Auszug der Methode \lstinline{linkDataSourceVariable} in \lstinline{variables.c}\label{lst:4-linkDataSourceVariable}}]
extern UA_StatusCode
 linkDataSourceVariable(UA_Server *server, UA_NodeId nodeId) {
     bool readonly = false;
     UA_DataSource dataSourceVariable;
     UA_StatusCode rc; |>\setcounter{lstnumber}{254}<|

     dataSourceVariable.read = readDataSourceVariable;
     if (!readonly)
        dataSourceVariable.write = writeDataSourceVariable;
     else
        dataSourceVariable.write = writeReadonlyDataSourceVariable;

     return UA_Server_setVariableNode_dataSource(server, nodeId, dataSourceVariable);
 }
\end{lstlisting}

\begin{figure}[h]
    \centering
    \includegraphics[width=0.42\textwidth]{doc/img/OPC_RevPiDO.pdf}
    \caption{Auszug des verwendeten Nodesets, hier Digitalausgang 1 des Versuchsaufbaus
      \label{fig:opc-do}}
\end{figure}

Die in Listing~\ref{lst:4-linkDataSourceVariable} abgebildete Methode \lstinline{linkDataSourceVariable()} erzeugt ein Struct vom Typ \lstinline{UA_DataSource}. In diesem werden dem Lesen und Schreiben einer OPC-Variablen entsprechende Callback-Methoden zugewiesen. Die Verknüpfung einer OPC-Variable, genauer ihrer NodeId, mit der zuvor definierten Datenquelle erfolgt über die von open62541 bereitgestellte Methode \lstinline{UA_Server_setVariableNode_dataSource()}. Vor dem Lesen und nach dem Schreiben dieser Variable werden von nun an die entsprechenden Callbacks aufgerufen.
     
\begin{lstlisting}[language={c},firstnumber=168,caption={Auszug des Callbacks \lstinline{writeDataSourceVariable} in \lstinline{variables.c}\label{lst:4-writeDataSourceVariable}}]  
extern UA_StatusCode
 writeDataSourceVariable(UA_Server *server,
            const UA_NodeId *sessionId, void *sessionContext,
            const UA_NodeId *nodeId, void *nodeContext,
            const UA_NumericRange *range, const UA_DataValue *dataValue) {

    UA_StatusCode retval  = UA_STATUSCODE_GOOD;
    UA_NodeId *nameNodeId = UA_malloc(sizeof(UA_NodeId));
    UA_QualifiedName nameQN = UA_QUALIFIEDNAME(1, "Name");
    UA_Variant nameVar;
    UA_Boolean bit;

    retval |= findSiblingByBrowsename(server, nodeId, &nameQN, nameNodeId);
    retval |= UA_Server_readValue(server, *nameNodeId, &nameVar);
    retval |= UA_Boolean_copy(dataValue->value.data, &bit);

    |>\tikzmarkin[set border color=martinired]{writeIO}<|PI_writeSingleIO(String_fromUA_String(nameVar.data), &bit, false);                                                 |>\tikzmarkend{writeIO}<|

    free(nameNodeId);
    return retval;
 }
\end{lstlisting}

Listing~\ref{lst:4-writeDataSourceVariable} zeigt die Callback-Methode, welche nach dem Schreiben einer Variablen auf dem OPC-Server aufgerufen wird.
Dieser Methode wird neben der NodeId der mit ihr verknüpften Variablen auch der Wert dieser in Form eines Zeigers auf ein Struct vom Typ \lstinline{UA_DataValue} übergeben.

Die Gestaltung des hier verwendeten Nodesets sieht vor, dass in einer OPC-Variablen \lstinline{"Name"} der Bezeichner des zu schreibenden Digitalausgangs hinterlegt ist, siehe Abbildung~\ref{fig:opc-do}. Dies erlaubt eine Rekonfiguration der Ein- und Ausgänge der SPS ohne Änderungen im Programmcode des OPC-Servers vornehmen zu müssen.
Es ist daher erforderlich, nach jedem Schreiben einer mit einem Digitalausgang verknüpften Variablen, hier \lstinline{"Value"}, dessen Bezeichner \lstinline{"Name"} abzufragen. 
Dies geschieht in den Zeilen 180 und 181.
Anschließend wird dieser Bezeichner sowie der zu schreibende Wert der Methode \lstinline{PI_writeSingleIO()} übergeben, welche wiederum die Interaktion mit piControl übernimmt (vgl. Abschnitt \ref{sec:4-picontrol}).
 
\subsection{Integration von piControl%
     \label{sec:4-picontrol}}
In Abschnitt~\ref{sec:2-io} wurde die Anbindung der IO-Module des Revolution Pi sowie die Funktionsweise von piControl aus Anwendersicht beschrieben. Die verfügbare Literatur beschränkt sich auch auf lediglich diese Sicht; eine weiterführende Dokumentation für Entwickler gibt es, neben der in Abschnitt~\ref{sec:3-anbindung} vorgestellten Manpage, nicht. 
In diesem Abschnitt soll daher der Quellcode von piControl sowie dessen Verwendung im Projekt genauer betrachtet werden.
Hierzu wird exemplarisch die in Abschnitt~\ref{sec:4-open62541} eingeführte Methode \lstinline{PI_writeSingleIO()} untersucht.
Diese Methode ermöglicht das Setzen eines einzelnen Bits im Prozessabbild der SPS, und damit das Schalten eines digitalen Ausgangs auf einem IO-Modul.
Die äquivalente Methode \lstinline{int piControlGetBitValue(SPIValue *pSpiValue)} zum Lesen eines Bits bzw. Eingangs funktioniert analog und soll daher an dieser Stelle nicht dediziert erörtert werden.

\begin{lstlisting}[language={c},firstnumber=97,
                   caption={Setzen eines phsikalischen, digitalen Ausgangs in \lstinline{revpi.c}
                   \label{lst:4-PI_writeSingleIO}}]
extern void PI_writeSingleIO(char *pszVariableName, bool *bit, bool verbose)
{
	int rc;
	SPIVariable sPiVariable;
	SPIValue sPIValue;

	strncpy(sPiVariable.strVarName, pszVariableName, sizeof(sPiVariable.strVarName));
	rc = piControlGetVariableInfo(&sPiVariable);
	if (rc < 0) {
		printf("Cannot find variable '%s'\n", pszVariableName);
		return;
	}

		sPIValue.i16uAddress = sPiVariable.i16uAddress;
		sPIValue.i8uBit = sPiVariable.i8uBit;
		sPIValue.i8uValue = *bit;
		rc = |>\tikzmarkin[set border color=martinired]{setBitValue}<|piControlSetBitValue(&sPIValue)|>\tikzmarkend{setBitValue}<|;
		if (rc < 0)
			printf("Set bit error %s\n", getWriteError(rc));
		else if (verbose)
			printf("Set bit %d on byte at offset %d. Value %d\n", sPIValue.i8uBit, sPIValue.i16uAddress,
			       sPIValue.i8uValue);
}
\end{lstlisting}

Der Programmcode in Listing~\ref{lst:4-PI_writeSingleIO} ist Teil des implementierten OPC-Servers. In diesem wird auf zwei Funktionen des piControl-Treibers zugegriffen. 
Beiden Methoden wird als Argument ein Zeiger auf ein Struct vom Typ \lstinline{SPIValue} übergeben. Der im Struct abgelegte Name wird mittels \lstinline{piControlGetVariableInfo(&sPIValue)} zu einer Adresse im Prozessabbild aufgelöst. Diese wird in \lstinline{sPIValue.i16uAdress} gespeichert. Der Wert der Variablen wird anschließend mittels \lstinline{piControlSetBitValue(&sPIValue)} an dieser Adresse in das Prozessabbild geschrieben.

\begin{lstlisting}[language={c},firstnumber=309,caption={Methode \lstinline{piControlSetBitValue} in \lstinline{piControlIf.c}\label{lst:4-piControlSetBitValue}}]
int |>\tikzmarkin[set border color=martiniblue]{setBitValueFcn}<|piControlSetBitValue(SPIValue *pSpiValue)|>\tikzmarkend{setBitValueFcn}<|
{
    piControlOpen();

    if (PiControlHandle_g < 0)
	    return -ENODEV;

    pSpiValue->i16uAddress += pSpiValue->i8uBit / 8;
    pSpiValue->i8uBit %= 8;

    if (|>\tikzmarkin[set border color=martinired]{ioctl}<|ioctl(PiControlHandle_g, KB_SET_VALUE, pSpiValue)|>\tikzmarkend{ioctl}<| < 0)
	    return errno;

    return 0;
}
\end{lstlisting}

Die in Listing~\ref{lst:4-piControlSetBitValue} dargestellte Methode \lstinline{piControlSetBitValue} ist lediglich eine Hüllfunktion (häufig auch als Wrapper-Funktion bezeichnet) für einen Aufruf des \lstinline{ioctl} Kernel-Moduls.
Folgende Parameter werden übergeben:
\lstinline{PiControlHandle_g} ist die Referenz auf die Geräte-Datei des piControl-Treibers. \lstinline{KB_SET_VALUE} ist das ioctl-Kommando zum Schreiben eines Bits in das Prozessabbild. Der Zeiger \lstinline{pSpiValue} verweist auf ein Struct des bereits vorgestellten Typs \lstinline{SPIValue}.

\begin{lstlisting}[language={c},firstnumber=80,caption={Methode \lstinline{piControlOpen} in \lstinline{piControlIf.c}\label{lst:4-piControlOpen}}]
void piControlOpen(void)
{
    /* open handle if needed */
    if (PiControlHandle_g < 0)
    {
	    |>\tikzmarkin[set border color=martiniblue]{PiControlHandle}<|PiControlHandle_g = open(PICONTROL_DEVICE, O_RDWR)|>\tikzmarkend{PiControlHandle}<|;
    }
}
\end{lstlisting}

Die in Listing~\ref{lst:4-piControlOpen} dargestellte Methode öffnet, sofern nicht bereits geschehen, die Geräte-Datei. Das Macro \lstinline{PICONTROL_DEVICE} verweist hierbei auf \lstinline{/dev/piControl0}.

\begin{lstlisting}[language={c},firstnumber=721,caption={Methode \lstinline{piControlIoctl} in \lstinline{piControlMain.c}\label{lst:4-piControlIoctl}}]
static long |>\tikzmarkin[set border color=martiniblue, below offset=0.9em]{piControlIoctl}<|piControlIoctl(struct file *file, unsigned int prg_nr, 
                           unsigned long usr_addr)                                      |>\tikzmarkend{piControlIoctl}<|
{
  int status = -EFAULT;
  tpiControlInst *priv;
  int timeout = 10000;	// ms

  if (prg_nr != KB_CONFIG_SEND && prg_nr != KB_CONFIG_START && !isRunning()) {
  	return -EAGAIN;
  }

  priv = (tpiControlInst *) file->private_data;

  if (prg_nr != KB_GET_LAST_MESSAGE) {
  	// clear old message
  	priv->pcErrorMessage[0] = 0;
  }

  switch (prg_nr) {|>\setcounter{lstnumber}{864}<|

    case |>\tikzmarkin[set border color=martiniblue]{KB_SET_VALUE}<|KB_SET_VALUE:|>\tikzmarkend{KB_SET_VALUE}<|
  		{
  			SPIValue *pValue = (SPIValue *) usr_addr;

  			if (!isRunning())
  				return -EFAULT;

  			if (pValue->i16uAddress >= KB_PI_LEN) {
  				status = -EFAULT;
  			} else {
  				INT8U i8uValue_l;
  				my_rt_mutex_lock(&piDev_g.lockPI);
  				i8uValue_l = piDev_g.ai8uPI[pValue->i16uAddress];

  				if (pValue->i8uBit >= 8) {
  					i8uValue_l = pValue->i8uValue;
  				} else {
  					if (pValue->i8uValue)
  						i8uValue_l |= (1 << pValue->i8uBit);
  					else
  						i8uValue_l &= ~(1 << pValue->i8uBit);
  				}

  				|>\tikzmarkin[set border color=martinired]{i8uValue}<|piDev_g.ai8uPI[pValue->i16uAddress] = i8uValue_l;|>\tikzmarkend{i8uValue}<|
  				rt_mutex_unlock(&piDev_g.lockPI);

  #ifdef VERBOSE
  				pr_info("piControlIoctl Addr=%u, bit=%u: %02x %02x\n", pValue->i16uAddress, pValue->i8uBit, pValue->i8uValue, i8uValue_l);
  #endif

  				status = 0;
  			}
  		}
  		break; |>\setcounter{lstnumber}{1314}<|

    default:
      pr_err("Invalid Ioctl");
      return (-EINVAL);
      break;

    }

    return status;
  }
\end{lstlisting}

Listing~\ref{lst:4-piControlIoctl} zeigt in Auszügen die ioctl-Methode des piControl Kernel-Treibers. Diese bekommt folgende Argumente übergeben: \lstinline{struct file *file} enthält den Verweis auf die Geräte-Datei, hier \lstinline{/dev/piControl0}. Der Wert von \lstinline{unsigned int prg_nr} beschreibt die Anfrage an den Treiber, in diesem Fall \lstinline{KB_SET_VALUE}. Das Argument \lstinline{unsigned long usr_addr} enthält einen typ-agnostischen Pointer. Dieser verweist auf einen Speicherbereich, in welchem die zur Bearbeitung der Anfrage notwendigen Daten abgelegt sind. Hier können auch vom Treiber empfangene Daten dem Anwendungsprogramm bereitgestellt werden. 

Die switch-case-Anweisung führt die über das Argument \lstinline{prg_nr} spezifizierte Aktion aus. Hier betrachten wir \lstinline{KB_SET_VALUE}:
Zunächst wird in Zeile 868 der übergebene Zeiger \lstinline{usr_addr} mittels explizitem Typecast zu einem Zeiger des Typs \lstinline{SPIValue *} konvertiert. Da dieser auf Daten im Userspace verweist, ist beim Zugriff durch den Kernel-Treiber besondere Vorsicht geboten.
In Zeile 877 wird mittels Mutex das Prozessabbild \lstinline{piDev_g} für den Zugriff durch andere Threads oder Prozesse gesperrt.
\lstinline{my_rt_mutex_lock} verweist hierbei auf die Funktion \lstinline{rt_mutex_lock} aus \lstinline{linux/sched.h}\footnote{Offenbar wurde hier auch eine alternative Implementierung vorgesehen, siehe revpi\_common.h}

In Zeile 889 wird das Byte \lstinline{i8uValue_l}, welches den zu schreibenden Wert enthält in das Prozessabbild übertragen. Anschließend wird die Mutex auf \lstinline{piDev_g} wieder entsperrt.
\newpage

\begin{lstlisting}[language={c},firstnumber=62,caption={Auszug des Struct \lstinline{spiControlDev} in \lstinline{piControlMain.h}\label{lst:4-spiControlDev}}]
|>\tikzmarkin[set border color=martiniblue]{spiControlDev}<|typedef struct spiControlDev|>\tikzmarkend{spiControlDev}<| {
	// device driver stuff
	int init_step;
	enum revpi_machine machine_type;
	void *machine;
	struct cdev cdev;	// Char device structure
	struct device *dev;
	struct thermal_zone_device *thermal_zone;

	|>\tikzmarkin[set border color=martiniblue]{processImage}<|// process image stuff
	INT8U ai8uPI[KB_PI_LEN];
	INT8U ai8uPIDefault|>\tikzmarkin[set border color=martinired]{KB_PI_LEN_0}<|[KB_PI_LEN]|>\tikzmarkend{KB_PI_LEN_0}<|;
	struct rt_mutex lockPI;        |>\tikzmarkend{processImage}<|
	bool stopIO;
	piDevices *devs; |>\setcounter{lstnumber}{94}<|
} tpiControlDev;
\end{lstlisting}

Das Prozessabbild ist als Byte-Array der Länge \lstinline{KB_PI_LEN} in Listing~\ref{lst:4-spiControlDev} definiert. Konfigurationsparameter wie \lstinline{KB_PI_LEN} oder die Zykluszeit für den Datenaustausch zwischen SPS und IO-Modulen sind im folgenden Listing~\ref{lst:4-process} definiert.

\begin{lstlisting}[language={c},firstnumber=119,caption={Konfigurationsparameter des Prozessabbildes in project.h\label{lst:4-process}}]
#define INTERVAL_PI_GATE (5*1000*1000)  // 5 ms piGateCommunication |>\setcounter{lstnumber}{128}<|

#define INTERVAL_IO_COM (5*1000*1000)  // 5 ms piIoComm |>\setcounter{lstnumber}{132}<|

#define KB_PD_LEN       512
|>\tikzmarkin[set border color=martiniblue]{KB_PI_LEN_1}<|#define KB_PI_LEN       4096|>\tikzmarkend{KB_PI_LEN_1}<|
\end{lstlisting}

Das zu setzende Bit wurde zu diesem Zeitpunkt erfolgreich in das Prozessabbild der SPS geschrieben.
Es stellt sich die Frage, wie dieses nun an das IO-Modul kommuniziert wird.
Die Kommunikation mit allen angebundenen Modulen ist ebenfalls Aufgabe des piControl-Treibers.

\begin{lstlisting}[language={c},firstnumber=256,caption={Auszug der Methode \lstinline{piIoThread} in \lstinline{revpi_core.c}\label{lst:4-piIoThread}}]
static int piIoThread(void *data)
{
	//TODO int value = 0;
	ktime_t time;
	ktime_t now;
	s64 tDiff;

	hrtimer_init(&piCore_g.ioTimer, CLOCK_MONOTONIC, HRTIMER_MODE_ABS);
	piCore_g.ioTimer.function = piIoTimer;

	pr_info("piIO thread started\n");

	now = hrtimer_cb_get_time(&piCore_g.ioTimer);

	PiBridgeMaster_Reset();

	while (!kthread_should_stop()) {
		if (|>\tikzmarkin[set border color=martinired]{PiBridgeMaster}<|PiBridgeMaster_Run()|>\tikzmarkend{PiBridgeMaster}<| < 0)
			break;
	}

	RevPiDevice_finish();

	pr_info("piIO exit\n");
	return 0;
}
\end{lstlisting}

Der Kernel-Thread \lstinline{piIoThread} ist verantwortlich für den zyklischen Datenaustausch mit den IO-Modulen. In diesem wird fortlaufend die Methode \lstinline{PiBridgeMaster_Run()} aufgerufen, siehe Listing~\ref{lst:4-piIoThread}.

\begin{lstlisting}[language={c},firstnumber=262,caption={Auszug der Methode \lstinline{PiBridgeMaster_Run(void)} in \lstinline{RevPiDevice.c}\label{lst:4-PiBridgeMaster_Run}}]
int PiBridgeMaster_Run(void)
{
	static kbUT_Timer tTimeoutTimer_s;
	static kbUT_Timer tConfigTimeoutTimer_s;
	static int error_cnt;
	static INT8U last_led;
	static unsigned long last_update;
	int ret = 0;
	int i;

	my_rt_mutex_lock(&piCore_g.lockBridgeState);
	if (piCore_g.eBridgeState != piBridgeStop) {
		switch (eRunStatus_s) { |>\setcounter{lstnumber}{514}<|
		    case enPiBridgeMasterStatus_EndOfConfig:|>\setcounter{lstnumber}{621}<|
		    if (|>\tikzmarkin[set border color=martinired]{RevPiDevice}<|RevPiDevice_run()|>\tikzmarkend{RevPiDevice}<|) {
				// an error occured, check error limits |>\setcounter{lstnumber}{641}<|
			} else {
				ret = 1;
			}
			piCore_g.image.drv.i16uRS485ErrorCnt = RevPiDevice_getErrCnt();
			break;
\end{lstlisting}

Die in Listing~\ref{lst:4-PiBridgeMaster_Run} dargestellte Methode ist eine sog. State-Machine. Ist die Konfiguration der IO-Module erfolgreich abgeschlossen, so führt sie bei Aufruf lediglich die Methode \lstinline{RevPiDevice_run()} aus.

\begin{lstlisting}[language={c},firstnumber=140,caption={Auszug der Methode \lstinline{RevPiDevice_run(void)} in \lstinline{RevPiDevice.c}\label{lst:4-RevPiDevice_run}}]
int RevPiDevice_run(void)
{
	INT8U i8uDevice = 0;
	INT32U r;
	int retval = 0;

	RevPiDevices_s.i16uErrorCnt = 0;

	for (i8uDevice = 0; i8uDevice < RevPiDevice_getDevCnt(); i8uDevice++) {
		if (RevPiDevice_getDev(i8uDevice)->i8uActive) {
			switch (RevPiDevice_getDev(i8uDevice)->sId.i16uModulType) {
			case KUNBUS_FW_DESCR_TYP_PI_DIO_14:
			case KUNBUS_FW_DESCR_TYP_PI_DI_16:
			case KUNBUS_FW_DESCR_TYP_PI_DO_16:
				r = |>\tikzmarkin[set border color=martinired]{sendCyclicTelegram}<|piDIOComm_sendCyclicTelegram(i8uDevice)|>\tikzmarkend{sendCyclicTelegram}\setcounter{lstnumber}{166} <|;

				break; |>\setcounter{lstnumber}{216}<|
			}
		}
	} |>\setcounter{lstnumber}{227}<|
	return retval;
}
\end{lstlisting}

Diese iteriert wie in Listing~\ref{lst:4-RevPiDevice_run} abgebildete durch alle gegenwärtig in der SPS konfigurierten Module. Ist das aktuelle Modul als aktiv markiert, so wird anhand eines sog. Firmware-Descriptors entschieden, welche Methode für die Ansteuerung des Moduls aufzurufen ist.

\begin{lstlisting}[language={c},firstnumber=161,caption={Auszug der Methode \lstinline{piDIOComm_sendCyclicTelegram} in \lstinline{piDIOComm.c}\label{lst:4-sendCyclicTelegram}}]
INT32U piDIOComm_sendCyclicTelegram(INT8U i8uDevice_p)
{
	INT32U i32uRv_l = 0;
	SIOGeneric sRequest_l;
	SIOGeneric sResponse_l;
	INT8U len_l, data_out[18], i, p, data_in[70];
	INT8U i8uAddress;
	int ret; |>\setcounter{lstnumber}{239}<|
	
    |>\tikzmarkin[set border color=martinired]{piIoComm}<|ret = piIoComm_send((INT8U *) & sRequest_l, IOPROTOCOL_HEADER_LENGTH + len_l + 1);  |>\tikzmarkend{piIoComm}\setcounter{lstnumber}{298}<|
}
\end{lstlisting}

Im Falle des hier verwendeten DO-Moduls wird die in Listing~\ref{lst:4-sendCyclicTelegram} abgebildete Methode \lstinline{piDIOComm_sendCyclicTelegram()} aufgerufen. Dieser wird ein Zeiger auf das zu schreibende Gerät übergeben. 
Zunächst wird das Prozessabbild mittels eines proprietären, jedoch im Quellcode offen nachvollziehbaren Protokolls in ein \lstinline{sRequest_l} genanntes Byte-Array umgewandelt. Dieser Schritt ist in Listing~\ref{lst:4-sendCyclicTelegram} nicht abgebildet. Anschließend wird \lstinline{piIoComm_send()} ein Zeiger auf die so generierte Schreib-Anfrage übergeben.

\begin{lstlisting}[language={c},firstnumber=220,caption={Auszug der Methode \lstinline{piIOComm_send} in \lstinline{piIOComm.c}\label{lst:4-piIOComm_send}}]
int piIoComm_send(INT8U * buf_p, INT16U i16uLen_p)
{
	ssize_t write_l = 0;
	INT16U i16uSent_l = 0;|>\setcounter{lstnumber}{249}<|

	while (i16uSent_l < i16uLen_p) {
		write_l = vfs_write(piIoComm_fd_m, buf_p + i16uSent_l, i16uLen_p - i16uSent_l, &piIoComm_fd_m->f_pos);
		if (write_l < 0) {
			pr_info_serial("write error %d\n", (int)write_l);
			return -1;
		} 
		i16uSent_l += write_l;|>\setcounter{lstnumber}{263}<|
	}
	clear();
	vfs_fsync(piIoComm_fd_m, 1);
	return 0;
}
\end{lstlisting}

Listing~\ref{lst:4-piIOComm_send} zeigt die Implementierung von \lstinline{piIoComm_send()}. Diese Methode ist für das Schreiben der oben generierten Anfrage auf die seriellen Schnittstelle verantwortlich. Realisiert wird dies mittels der Methode \lstinline{vfs_write()}. Diese ist in \lstinline{<linux/fs.h>} definiert. Sie ermöglicht das Schreiben einer Datei im Userspace aus dem Kernel heraus. Geschrieben wird hier die Datei mit dem Deskriptor \lstinline{piIoComm_fd_m}.
Da die Funktion \lstinline{vfs_write()} durch andere Kernel-Tasks unterbrochen werden kann, ist nicht gewährleistet, dass die gesamte Anfrage mit nur einem Aufruf geschrieben wird. Die oben abgebildete while-Schleife stellt das vollständige Senden der Anfrage sicher.

\begin{lstlisting}[language={c},firstnumber=157,caption={Auszug der Methode \lstinline{piIOComm_open_serial} in \lstinline{piIOComm.c}\label{lst:4-piIOComm_open_serial}}]
int piIoComm_open_serial(void)
{   |>\setcounter{lstnumber}{167}<|
	struct file *fd;	/* Filedeskriptor */
	struct termios newtio;	/* Schnittstellenoptionen */

	|>\tikzmarkin[set border color=martiniblue]{fd}<|/* Port oeffnen - read/write, kein "controlling tty", 
	    Status von DCD ignorieren */
	fd = filp_open(|>\tikzmarkin[set border color=martinired]{tty}<|REV_PI_TTY_DEVICE|>\tikzmarkend{tty}<|, O_RDWR | O_NOCTTY, 0); |>\setcounter{lstnumber}{208}<|
	
	piIoComm_fd_m = fd;                                                      |>\tikzmarkend{fd}\setcounter{lstnumber}{217}<|

	return 0;
}
\end{lstlisting}

Der zum Schreiben auf die serielle Schnittstelle verwendete Datei-Deskriptor wird von der in Listing~\ref{lst:4-piIOComm_open_serial} abgebildeten Methode \lstinline{piIoComm_open_serial()} generiert. 

\begin{lstlisting}[language={c},firstnumber=45,caption={Definition der seriellen Schnittstelle in \lstinline{piIOComm.h}\label{lst:4-REV_PI_TTY_DEVICE}}]
#define REV_PI_TTY_DEVICE	"/dev/ttyAMA0"
\end{lstlisting}

Das in Listing~\ref{lst:4-REV_PI_TTY_DEVICE} definierte Macro verweist auf eine der seriellen Schnittstellen des RaspberryPi.
Die Implementierung des zugehörigen Schnittstellentreibers soll hier nicht weiter untersucht werden. Somit ist an dieser Stelle die Kette vom Setzen einer Variablen auf dem OPC-Server bis hin zur Aktualisierung des Prozessabbilds der IO-Module geschlossen.

% \begin{lstlisting}[language={c},firstnumber={226},caption={Setzen der Scheduler-Priorität auf SCHED\_FIFO in 
% revpi\_common.c\label{lst:2-sched_priority}}]
% param.sched_priority = ktprio->prio;
% ret = sched_setscheduler(child, SCHED_FIFO, &param);
% \end{lstlisting}
%% % Imports nur für Referenzenauflösung während des Schreibens! Vorm Kompilieren auskommentieren!
% \bibliography{0_hauptdatei}
% \input{1_einleitung}
% \input{2_grundlagen}
% \input{3_konzeption}
% \input{4_implementierung}
% \input{5_tests}
% \input{6_zusammenfassung}
% % Ende Imports

\section{Test des OPC-Servers im Gesamtsystem%
  \label{sec:5-tests}}

%% % Imports nur für Referenzenauflösung während des schreibens! Vorm Kompilieren auskommentieren!
% \bibliography{0_hauptdatei}
% \input{1_einleitung}
% \input{2_grundlagen}
% \input{3_konzeption}
% \input{4_implementierung}
% \input{5_tests}
% \input{6_zusammenfassung}
% % Ende Imports

\section{Zusammenfassung und Ausblick%
  \label{sec:6-fazit}}
Der folgende Abschnitt~\ref{sec:6-zusammenfassung} fasst die gewonnenen Erkenntnisse und den Stand der Implementierung zusammen.
Den Abschluss dieser Arbeit bildet der Ausblick in Abschnitt~\ref{sec:6-ausblick}.

\subsection{Zusammenfassung%
     \label{sec:6-zusammenfassung}}

\subsection{Ausblick%
     \label{sec:6-ausblick}}

% % Ende Imports

\section{Einleitung und Motivation%
  \label{sec:1-einleitung}}
Ziel dieses Projektes ist die Integration eines OPC-Servers mit einer auf Linux
basierenden speicherprogrammierbaren Steuerung (SPS). Angeschlossen an diese SPS
ist jeweils ein digitales Ein-/\,bzw.~Ausgabemodul. Die von diesen bereitgestellten
Ein-/\, bzw.~Ausgänge (IO) sollen in der Datenstruktur des OPC-Servers abgebildet
und über diesen für OPC-Clients les-/\,und schreibar sein. Weiterhin sollen einige
Funktionen zur Überwachung und Steuerung der an die SPS angeschlossenen Aktoren
und Sensoren direkt im OPC-Server implementiert werden.
Hiermit stellt dieses Projekt eine der Grundlagen für ein übergeordnetes Projekt,
die cloudbasierte Steuerung eines miniaturisierten Produktions-Systems, dar.

Der hier verwendete OPC-Server ist Teil des sog. open62541 Projekts. Er ist in C
geschrieben und implementiert bereits einen großen Teil der im OPC-UA-Standard
spezifizierten Funktionen.
Als SPS findet ein Revolution Pi 3 der Firma Kunbus Verwendung. Dieser integriert
ein sog. Compute Module der Raspberry Pi Foundation in ein industrietaugliches
Gehäuse und erlaubt die Erweiterung mittels IO- oder Gateway-Modulen. Über diese
erfolgt die Kommunikation mit weiteren Komponenten der Automatisierungstechnik.

Motiviert ist dieses Projekt durch die Beobachtung, dass die Verbreitung offener
Standards sowie freier Software auch in der Automatisierungstechnik zunimmt.
Linux ist ein freies Betriebssystem, OPC-UA ein offen zugänglicher, aktiv gepflegter
und weit verbreiteter Standard. Der Raspberry Pi findet sowohl bei Hobby-Anwendern als
auch in den Bereichen Forschung und Entwicklung sowie bei industriellen Anwendern
Verwendung. Dieses Projekt stellt somit eine für unterschiedliche Anwender interessante
Entwicklung dar.

Im Anschluss an diese einleitende Übersicht im Abschnitt~\ref{sec:1-einleitung} folgt
die Darstellung der wichtigsten Grundlagen in Abschnitt~\ref{sec:2-grundlagen}.
Aufbauend auf diesen Grundlagen folgt die konzeptuelle Ausarbeitung im Abschnitt~\ref{sec:3-konzeption}.
Die Umsetzung wird im Abschnitt~\ref{sec:4-implementierung} erläutert.
Die Leistungsfähigkeit der Implementierung wird in Abschnitt~\ref{sec:5-tests} untersucht.
Eine Zusammenfassung und ein Ausblick schließen die Arbeit in
Abschnitt~\ref{sec:6-fazit} ab. Eventuell noch benötigte Anhänge
finden sich in den Anhängen [...] bis [...].

% % % Imports nur für Referenzenauflösung während des Schreibens! Vorm Kompilieren auskommentieren!
% \bibliography{0_hauptdatei}
% % Mit \section{...} eröffnen wir einen neuen Abschnitt.
% Der Befehl setzt nicht nur den Text in einer größeren,
% fetten Schrift, sondern sorgt außerdem dafür, daß er im
% Inhaltsverzeichnis erscheint.
%
% Mit \label{...} erzeugen wir einen Bezeichner, mit dessen Hilfe
% wir später auf die Nummer des Abschnitts verweisen können (nämlich
% mit~\ref{...}).
%
% Das Kommentarzeichen hinter „Übersicht“ dient dazu, ein
% Leerzeichen zwischen „Übersicht“ und dem \label-Befehl
% zu vermeiden, das andernfalls sichtbar würde – z.B. im
% Inhaltsverzeichnis.
%

% % Imports nur für Referenzenauflösung während des Schreibens! Vorm Kompilieren auskommentieren!
% \bibliography{0_hauptdatei}
% \input{1_einleitung}
%\input{2_grundlagen}
%\input{3_konzeption}
%\input{4_implementierung}
%\input{5_tests}
%\input{6_zusammenfassung}
% % Ende Imports

\section{Einleitung und Motivation%
  \label{sec:1-einleitung}}
Ziel dieses Projektes ist die Integration eines OPC-Servers mit einer auf Linux
basierenden speicherprogrammierbaren Steuerung (SPS). Angeschlossen an diese SPS
ist jeweils ein digitales Ein-/\,bzw.~Ausgabemodul. Die von diesen bereitgestellten
Ein-/\, bzw.~Ausgänge (IO) sollen in der Datenstruktur des OPC-Servers abgebildet
und über diesen für OPC-Clients les-/\,und schreibar sein. Weiterhin sollen einige
Funktionen zur Überwachung und Steuerung der an die SPS angeschlossenen Aktoren
und Sensoren direkt im OPC-Server implementiert werden.
Hiermit stellt dieses Projekt eine der Grundlagen für ein übergeordnetes Projekt,
die cloudbasierte Steuerung eines miniaturisierten Produktions-Systems, dar.

Der hier verwendete OPC-Server ist Teil des sog. open62541 Projekts. Er ist in C
geschrieben und implementiert bereits einen großen Teil der im OPC-UA-Standard
spezifizierten Funktionen.
Als SPS findet ein Revolution Pi 3 der Firma Kunbus Verwendung. Dieser integriert
ein sog. Compute Module der Raspberry Pi Foundation in ein industrietaugliches
Gehäuse und erlaubt die Erweiterung mittels IO- oder Gateway-Modulen. Über diese
erfolgt die Kommunikation mit weiteren Komponenten der Automatisierungstechnik.

Motiviert ist dieses Projekt durch die Beobachtung, dass die Verbreitung offener
Standards sowie freier Software auch in der Automatisierungstechnik zunimmt.
Linux ist ein freies Betriebssystem, OPC-UA ein offen zugänglicher, aktiv gepflegter
und weit verbreiteter Standard. Der Raspberry Pi findet sowohl bei Hobby-Anwendern als
auch in den Bereichen Forschung und Entwicklung sowie bei industriellen Anwendern
Verwendung. Dieses Projekt stellt somit eine für unterschiedliche Anwender interessante
Entwicklung dar.

Im Anschluss an diese einleitende Übersicht im Abschnitt~\ref{sec:1-einleitung} folgt
die Darstellung der wichtigsten Grundlagen in Abschnitt~\ref{sec:2-grundlagen}.
Aufbauend auf diesen Grundlagen folgt die konzeptuelle Ausarbeitung im Abschnitt~\ref{sec:3-konzeption}.
Die Umsetzung wird im Abschnitt~\ref{sec:4-implementierung} erläutert.
Die Leistungsfähigkeit der Implementierung wird in Abschnitt~\ref{sec:5-tests} untersucht.
Eine Zusammenfassung und ein Ausblick schließen die Arbeit in
Abschnitt~\ref{sec:6-fazit} ab. Eventuell noch benötigte Anhänge
finden sich in den Anhängen [...] bis [...].

% % % Imports nur für Referenzenauflösung während des Schreibens! Vorm Kompilieren auskommentieren!
% \bibliography{0_hauptdatei}
% \input{1_einleitung}
% \input{2_grundlagen}
% \input{3_konzeption}
% \input{4_implementierung}
% \input{5_tests}
% \input{6_zusammenfassung}
% % Ende Imports

\section{Grundlagen%
  \label{sec:2-grundlagen}}

\subsection{Speicherprogrammierbare-Steuerung und Linux -- Revolution Pi%
     \label{sec:2-sps}}

\subsubsection{Kunbus RevolutionPi%
        \label{sec:2-revpi}}
Der RevolutionPi 3 ist eine speicherprogrammierbare Steuerung (SPS) des Herstellers
Kunbus GmbH. Kern dieser SPS ist das von der Raspberry Pi Foundation entwickelte
und vertriebene Raspberry Pi Compute Module 3. Dieses integriert ein Broadcom BCM2837
System-on-Chip (SoC) mit vier 1,2GHz Prozessorkernen, 1GB RAM, 4GB eMMC Anwendungsspeicher
und sonstige Peripherie in ein Modul im DDR2-SODIMM Formfaktor. Diese Spezifikationen
sind weitgehend identisch zu denen des ausgesprochen populären Raspberry Pi 3.
Der Revolution Pi profitiert daher von dem gleichen großen Angebot an Software
und Unterstützung wie der Raspberry Pi, ergänzt dessen Hardware jedoch um eine 24V
Spannungsversorgung, die Möglichkeit der Erweiterung durch mehrere industrietaugliche
Ein-/ Ausgabemodule und Gateways sowie ein Gehäuse zur Montage auf einer DIN-Schiene.
\begin{itemize}
  \item{Prozessor: BCM2837}
  \item{Taktfrequenz 1,2 GHz}
  \item{Anzahl Prozessorkerne: 4}
  \item{Arbeitsspeicher: 1 GByte}
  \item{eMMC Flash Speicher: 4 GByte}
  \item{Betriebssystem: Angepasstes Raspbian mit RT-Patch}
  \item{RTC mit 24h Pufferung über wartungsfreien Kondensator}
  \item{Treiber / API: Treiber schreibt zyklisch Prozessdaten in ein Prozessabbild, Zugriff auf Prozessabbild über Linux-Filesystem als API zu Fremdsoftware.}
  \item{Kommunikationsanschlüsse: 2 x USB 2.0 A (je 500 mA belastbar), 1 x Micro-USB, HDMI, Ethernet (RJ45) 10/100 Mbit/s}
  \item{Stromversorgung: min. 10,7 V, max. 28,8 V, maximal 10 Watt}
  \item{Zulässige Umgebungstemperatur: -40 bis +55 C}
  \item{Gehäuseabmessungen: (HxBxL) 96 mm x 22,5 mm x 110,5 mm (ohne gesteckte Stecker)}
  \item{ESD Schutz: 4 kV / 8 kV gemäß EN61131-2 und IEC 61000-6-2}
  \item{Surge / Burst Prüfungen: gemäß EN61131-2 und IEC 61000-6-2 eingekoppelt auf Versorgungsspannung, Ethernet und IO-Leitungen}
  \item{EMI Prüfungen: gemäß EN61131-2 und IEC 61000-6-2}
\end{itemize}

Kunbus bietet eine Auswahl an IO- und Gateway-Modulen zur Erweiterung des Revolution Pi an.
Gateways dienen der Kommunikation mit Systemen oder Komponenten der Automatisierungstechnik
über Protokolle wie PROFIBUS oder EtherCAT. IO-Module erlauben die Überwachung
und Steuerung von digitalen oder analogen Ein- und Ausgängen.

\subsubsection{Zugriff auf IO-Module%
        \label{sec:2-io}}
Der Zugriff auf die Ein- und Ausgänge der IO-Module erfolgt über ein Prozessabbild
und einen hierfür von Kunbus bereitgestellten Treiber, genannt piControl. Dieser
aktualisiert das Prozessabbild zyklisch. Die angestrebte Zykluszeit beträgt 5ms,
kann jedoch je nach Anzahl der angeschlossenen Module auch größer sein. Kunbus
garantiert bei drei IO-Modulen und zwei Gateway-Modulen eine Zykluszeit von 10 ms.
Jedes der IO-Module stellt ein eigenständiges eingebettetes System dar. Es verfügt
über einen Microcontroller, welcher die IOs bereitstellt und über einen RS485-Bus
mit dem Revolution Pi kommuniziert.
% https://revolution.kunbus.de/io-modul/

Lizenz: GPL
% https://github.com/RevolutionPi/piControl

\begin{lstlisting}[language={c},firstnumber={226},caption={Setzen der Scheduler-Priorität auf SCHED\_FIFO in revpi\_common.c\label{lst:2-sched_priority}}]
param.sched_priority = ktprio->prio;
ret = sched_setscheduler(child, SCHED_FIFO,
       &param);
\end{lstlisting}


\subsection{Echtzeit und Multithreading unter Linux -- preemptRT und posix%
     \label{sec:2-echtzeit}}


 Der Linux-Kernel verfügt über mehrere unterschiedliche Preemtion-Modelle:

\begin{itemize}
  \item No Forced Preemption (server):
  Ausgelegt auf maximal möglichen Durchsatz, lediglich Interrupts und
  System-Call-Returns bewirken Präemption.

  \item Voluntary Kernel Preemption (Desktop):
  Neben den implizit bevorrechtigten Interrupts und System-Call-Returns gibt es
  in diesem Modell weitere Abschnitte des Kernels in welchen Preämption explizit
  gestattet ist.

  \item Preemptible Kernel (Low-Latency Desktop):
  In diesem Modell ist der gesamte Kernel, mit Ausnahme sog.~kritischer Abschnitte
  präemptible. Nach jedem kritischen Abschnitt gibt es einen impliziten Präemptions-Punkt.

  \item Preemptible Kernel (Basic RT):
  Dieses Modell ist dem zuvor genannten sehr ähnlich, hier sind jedoch alle Interrupt-Handler
  als eigenständige Threads ausgeführt.

  \item Fully Preemptible Kernel (RT):
  Wie auch bei den beiden zuvor genannten Modellen ist hier der gesamte Kernel
  präemtible, die Anzahl und Dauer der nicht-präemtiblen kritischen Abschnitte
  ist auf ein notwendiges Minimum beschränkt. Alle Interrupt-Handler sind als
  eigenständige Threads ausgeführt, Spinlocks durch Sleeping-Spinlocks und Mutexe
  durch sog.~RT-Mutexe ersetzt.

\end{itemize}
\todo{Spinlocks und Mutexe sowie die RT-Varianten dieser erklären!}

Lediglich mit dem vollständig präemtiblen Kernel kann Echtzeit-Verhalten realisiert werden.

% https://wiki.linuxfoundation.org/realtime/documentation/technical_basics/preemption_models bzw kernel/Kconfig.preempt

\subsubsection{preemptRT%
        \label{sec:2-preemptRT}}
% https://wiki.linuxfoundation.org/realtime/documentation/technical_details/start
% https://wiki.linuxfoundation.org/realtime/documentation/technical_basics/start

Das dem PREEMPT RT Kernel zugrunde liegende Prinzip lässt sich in einer einfachen
Regel ausdrücken: Nur Code, welcher absolut nicht-präemtible sein darf, ist es
gestattet nicht-präemtible zu sein.
Das erklärte Ziel des PREEMPT\_RT Patches ist es folglich, die Menge des nicht-präemtiblen
Codes im Linux-Kernel auf das absolut notwendige Minimum zu reduzieren.

Dies wird durch Verwendung folgender Mechanismen erreicht:

\begin{itemize}
  \item Hochauflösende Timer
  \item Sleeping Spinlocks
  \item Threaded Interrupt Handlers
  \item rt\_mutex
  \item RCU
\end{itemize}


\subsubsection{posix%
        \label{sec:2-posix}}
Ist posix hier wirklich relevant? Debian bzw.~Raspbian sind weitgehend posix
kompatibel, aber wird es hier genutzt? -> JA, open62541 nutzt pthread.h
piControl nutzt kthread.h, und semaphore.h

\subsection{OPC-UA und open62541%
     \label{sec:2-opc}}

\subsubsection{OPC UA%
        \label{sec:2-opcua}}
Open Platform Communications (OPC) ist eine Familie von Standards zur herstellerunabhängigen
Kommunikation von Maschinen (M2M) in der Automatisierungstechnik. Die sog.~OPC Task Force, zu deren
Mitgliedern verschiedene große Firmen der Automatisierungsindustrie gehören, veröffentlichte
die OPC Specification Version 1.0 im August 1996.
Motiviert ist dieser offene Standard durch die Erkenntniss, dass die Anpassung der
zahlreichen Herstellerstandards an individuelle Infrastrukturen und Anlagen einen
großen Mehraufwand verursachen.
Die Wikipedia beschreibt das Anwendungsgebiet für OPC wie folgt:

\glqq{}OPC wird dort eingesetzt, wo Sensoren, Regler und Steuerungen verschiedener Hersteller
ein gemeinsames Netzwerk bilden. Ohne OPC benötigten zwei Geräte zum Datenaustausch
genaue Kenntnis über die Kommunikationsmöglichkeiten des Gegenübers. Erweiterungen
und Austausch gestalten sich entsprechend schwierig. Mit OPC genügt es, für jedes
Gerät genau einmal einen OPC-konformen Treiber zu schreiben. Idealerweise wird
dieser bereits vom Hersteller zur Verfügung gestellt. Ein OPC-Treiber lässt sich
ohne großen Anpassungsaufwand in beliebig große Steuer- und Überwachungssysteme
integrieren.

OPC unterteilt sich in verschiedene Unterstandards, die für den jeweiligen Anwendungsfall
unabhängig voneinander implementiert werden können. OPC lässt sich damit verwenden
für Echtzeitdaten (Überwachung), Datenarchivierung, Alarm-Meldungen und neuerdings
auch direkt zur Steuerung (Befehlsübermittlung).\grqq{}

OPC basiert in der ursprünglichen Spezifikation auf Microsofts DCOM-Spezifikation.
DCOM macht Funktionen und Objekte einer Anwendung anderen Anwendungen im Netzwerk
zugänglich. Der OPC-Standard definiert entsprechende DCOM-Objekte um mit anderen
OPC-Anwendungen Daten austauschen zu können. Die Verwendung von DCOM bindet Anwender
an Betriebssysteme von Microsoft. Die ursprüngliche OPC Spezifikation wird durch die
Entwicklung von OPC Unified Architecture (OPC UA) abgelöst.
OPC UA setzt auf einem eigenen Kommunikationionsstack auf, die Verwendung von DCOM
und damit die Bindung an Microsoft wurden aufgelöst.

Die OPC-UA-Architektur ist eine Service-orientierte Architektur (SOA), deren Struktur
aus mehreren Schichten besteht.

% Wikipedia
Das OPC-Informationsmodell ist nicht mehr nur eine Hierarchie aus Ordnern, Items
und Properties. Es ist ein sogenanntes Full-Mesh-Network aus Nodes, mit dem neben
den Nutzdaten eines Nodes auch Meta- und Diagnoseinformationen repräsentiert werden.
Ein Node ähnelt einem Objekt aus der objektorientierten Programmierung. Ein Node
kann Attribute besitzen, die gelesen werden können (Data Access (DA), Historical
Data Access (HDA)). Es ist möglich Methoden zu definieren und aufzurufen.
Eine Methode besitzt Aufrufargumente und Rückgabewerte. Sie wird durch ein Command
aufgerufen. Weiterhin werden Events unterstützt, die versendet werden können
(AE (Alarms \& Events), DA DataChange), um bestimmte Informationen zwischen Geräten
auszutauschen. Ein Event besitzt unter anderem einen Empfangszeitpunkt, eine Nachricht
und einen Schweregrad. Die o. g. Nodes werden sowohl für die Nutzdaten als auch
alle anderen Arten von Metadaten verwendet. Der damit modellierte OPC-Adressraum
beinhaltet nun auch ein Typmodell, mit dem sämtliche Datentypen spezifiziert werden.

% https://de.wikipedia.org/wiki/Open_Platform_Communications
% https://de.wikipedia.org/wiki/OPC_Unified_Architecture
% https://opcfoundation.org/developer-tools/specifications-unified-architecture
% Von Gerhard Gappmeier - ascolab GmbH, CC BY-SA 3.0, https://de.wikipedia.org/w/index.php?curid=1892069
\subsubsection{open62541%
        \label{sec:2-open62541}}
open62541 ist eine offene und freie Implementierung von OPC UA. Die in C geschriebene
Bibliothek stellt eine beständig zunehmende Anzahl der im OPC UA Standard definierten
Funktionen bereit. Sie kann sowohl zur Erstellung von OPC-Servern als auch -Clients
genutzt werden. Ergänzend zu der unter der Mozilla Public License v2.0 lizensierten
Bibliothek stellt das open62541 Projekt auch Beispielprogramme unter einer CC0 Lizenz
zur Verfügung.

Die Bibliothek eignet sich auch für die Entwicklung auf eingebetteten Systemen und
Microcontrollern. Je nach Umfang der gewünschten Funktionen und des OPC Informationsmodells
beträgt die Größe einer Server-Binary weniger als 100kb. %evtl. kürzen?

\todo{Nodes erklären! Evtl.~oben!}

Folgende Auswahl an Eigenschaften und Funktionen zeichnet die in dieser Arbeit verwendete
Version 0.3 von open62541 aus:
\begin{itemize}
  \item Kommunikationionsstack
  \begin{itemize}
      \item OPC UA Binär-Protokoll (HTTP oder SOAP werden gegenwärtig nicht unterstützt)
      \item Austauschbare Netzwerk-Schicht, welche die Verwendung eigener Netzwerk-APIs
      erlaubt.
      \item Verschlüsselte Kommunikationion
      \item Asynchrone Dienst-Anfragen im Client
  \end{itemize}
  \item Informationsmodell
  \begin{itemize}
    \item Unterstützung aller OPC UA Node-Typen, inkl.~Methoden
    \item Hinzufügen und Entfernen von Nodes und Referenzen zur Laufzeit.
    \item Vererbung und Instanziierung von Objekt- und Variablentypen
    \item Zugriffskontrolle auch für einzelne Nodes
  \end{itemize}
  \item Subscriptions
  \begin{itemize}
    \item Erlaubt die Überwachung (subscriptions / monitoreditems)
    \item Sehr geringer Ressourcenbedarf pro überwachtem Wert
  \end{itemize}
  \item Code-Generierung auf XML-Basis
  \begin{itemize}
    \item Erlaubt die Erstellung von Datentypen
    \item Erlaubt die Generierung des serverseitigen Informationsmodells
  \end{itemize}
\end{itemize}

% https://open62541.org/doc/0.3/


Mozilla Public License
CC0 Lizenz für Beispiele und Plugins

% https://open62541.org/doc/open62541-current.pdf
% https://open62541.org/

% % % Imports nur für Referenzenauflösung während des Schreibens! Vorm Kompilieren auskommentieren!
% \bibliography{0_hauptdatei}
% \input{1_einleitung}
% \input{2_grundlagen}
% \input{3_konzeption}
% \input{4_implementierung}
% \input{5_tests}
% \input{6_zusammenfassung}
% \input{anhang}
% % Ende Imports

\section{Systemkonzept%
  \label{sec:3-konzeption}}
Auf Basis der in Abschnitt \ref{sec:2-grundlagen} vorgestellten Möglichkeiten folgt nun die Ausarbeitung eines Konzepts.
In den folgenden Abschnitten soll näher auf zwei zentrale Aspekte eingegangen werden: Abschnitt~\ref{sec:3-anbindung} stellt Möglichkeiten zum Zugriff auf Variablen bzw.\,Werte im Prozessabbild des Revolution Pi vor; in Abschnitt~\ref{sec:3-integration} wird ein Konzept zur Bereitstellung dieser Variablen auf einem OPC-Server vorgestellt.

\subsection{Anbindung der IO an den OPC-Server%
     \label{sec:3-anbindung}}

Eine Webanwendung mit Bezeichnung PiCtory dient zur Konfiguration der I/O- und virtuellen Module des RevolutionPi. Die Konfiguration liegt im JSON-Format in der Datei \lstinline{/etc/revpi/config.rsc}. Der piControl-Treiber liest diese Datei beim Start. 
Der folgende Auszug aus der Manpage des piControl-Kernelmoduls beschreibt die von diesem zum Lesen und Schreiben einzelner Bits des Prozessabbildes bereitgestellten Funktionen~\citep[vgl.]{web-revpi-manpage}. Sie ist an dieser Stelle weitgehend ungekürzt zitiert, da sie die nutzbare Schnittstelle sehr kompakt beschreibt.

\begin{lstlisting}[breakindent=0pt, numbers=none, caption={Auszug aus der Revolution Pi Programmers Manual\label{lst:4-manpage}}]
KB_FIND_VARIABLE SPIVariable *argp
Find a variable in the process image by its name. A pointer to a structure of type SPIVariable must be passed as argument. [...]
The struct SPIVariable [...] is defined as 
typedef struct SPIVariableStr
{
    char strVarName[32]; // Variable name
    uint16_t i16uAddress; // Address of the byte in the process image
    uint8_t i8uBit; // 0-7 bit position, >= 8 whole byte
    uint16_t i16uLength; // length of the variable in bits.
    // Possible values are 1, 8, 16 and 32
} SPIVariable;

Set and get values of the process image
KB_GET_VALUE SPIValue *argp
[...]
KB_SET_VALUE SPIValue *argp
Write one bit or one byte to the process image [...].  This call is more efficient than the usual calls of seek and write because only one function call is necessary. If more than on application are writing bits in one output byte, this call is the only safe way to set a bit without overwriting the other bits because this call is doing a read-modify-write-cycle. 

The struct SPIValue used by this ioctl is defined as
typedef struct SPIValueStr
{
    uint16_t i16uAddress; // Address of the byte in the process image
    uint8_t i8uBit; // 0-7 bit position, >= 8 whole byte
    uint8_t i8uValue; // Value: 0/1 for bit access, whole byte otherwise
} SPIValue;
\end{lstlisting} 

Die oben beschriebenden Funtkionen \lstinline{KB_FIND_VARIABLE}, \lstinline{KB_GET_VALUE} und \lstinline{KB_SET_VALUE} ermöglichen einen einfachen und (lt.\,Manpage) effizienten Zugriff auf einzelne Bits des Prozessabbildes und damit auch auf die IO des RevolutionPi.
Der Zugriff des OPC-Servers auf das Prozessabbild soll daher mittels dieser Funktionen realisiert werden.
\lstinline{KB_FIND_VARIABLE} kann genutzt werden, um Adressen von Variablen im Prozessabbild mittels ihres Namens aufzulösen.
\lstinline{KB_GET_VALUE} und \lstinline{KB_SET_VALUE} ermöglichen den Zugriff auf die Werte dieser Variablen.


\subsection{Integration des OPC-Servers in das System%
     \label{sec:3-integration}}

open62541 bietet drei Möglichkeiten zum Abgleich von Variablen mit dem Prozessabbild~\citep[vgl.][Tutorials - Connecting a Variable with a Physical Process]{web-open62541}:
\begin{itemize}
    \item Manuelles oder zyklisches Aktualisieren
    \item Variable Value Callback
    \item Variable Datasource
\end{itemize}

Die zyklische Aktualisierung eines oder mehrerer Werte nimmt, abhängig von der Zykluszeit, viele Systemressourcen in Anspruch. Value Callbacks ermöglichen es, einen Variablenwert effizienter mit einer Ressource wie etwa einem Prozessabbild zu synchronisieren. An die Variable wird ein Callback angehängt, welches vor jedem Lesen und nach jedem Schreibvorgang ausgeführt wird.
Der Wert der Variablen wird weiterhin im Variablenknoten auf dem OPC-Server gespeichert, der Abgleich mit der verknüpften Ressource erfolgt durch die Callback-Methoden.

Sogenannte Datenquellen gehen noch einen Schritt weiter. Der Server leitet jede Lese- und Schreibanforderung direkt an eine Callback-Funktion weiter. Beim Lesen liefert der Rückruf eine Kopie des aktuellen Wertes. Die Datenquelle muss intern ein eigenes Speichermanagement implementieren.

Der Zugriff auf die Werte des Prozessabbildes erfolgt, wie in Abschnitt~\ref{sec:3-anbindung} beschrieben, über von piControl bereitgestellte Methoden. Um die durch open62541 gepflegte OPC-Datenstruktur und das durch piControl verwaltete Prozessabbild möglichst effektiv verknüpfen zu können, soll diese Interaktion mittels Datenquellen und den zugehörigen Callbacks implementiert werden.
% % % Imports nur für Referenzenauflösung während des Schreibens! Vorm Kompilieren auskommentieren!
% \bibliography{0_hauptdatei}
% \input{1_einleitung}
% \input{2_grundlagen}
% \input{3_konzeption}
% \input{4_implementierung}
% \input{5_tests}
% \input{6_zusammenfassung}
% \input{anhang}
% % Ende Imports

\section{Implementierung%
  \label{sec:4-implementierung}}
Das folgende Kapitel stellt in Auszügen die Implementierung des OPC-Servers sowie die Anbindung an die IO-Module
der SPS dar. Der Schwerpunkt liegt hierbei auf der Funktionsweise des piControl-Treibers und dessen Integration in das Projekt. Abschnitt~\ref{sec:4-picontrol} erklärt die zum Schreibens eines Bits verwendeten Funktionsaufrufe.
Zuvor soll jedoch in Abschnitt~\ref{sec:4-open62541} der Teil des OPC-Servers vorgestellt werden, welcher auf besagten Treiber zugreift. 

\subsection{Implementierung des OPC-Servers%
     \label{sec:4-open62541}}
Wie im vorangegangenen Abschnitt~\ref{sec:3-integration} begründet, soll die Verknüpfung zwischen dem Prozessabbild der SPS und den auf dem OPC-Server bereitgestellten Werten über sog.\,Datenquellen erfolgen. Hierzu ist zunächst eine Callback-Methode zu implementieren, welche bei einem Lese- oder Schreibzugriff auf eine Variable aufgerufen wird. Die Verknüpfung zwischen Callback-Methode und Variable muss manuell erfolgen.

\begin{lstlisting}[language={c},firstnumber=237,caption={Auszug der Methode \lstinline{linkDataSourceVariable} in \lstinline{variables.c}\label{lst:4-linkDataSourceVariable}}]
extern UA_StatusCode
 linkDataSourceVariable(UA_Server *server, UA_NodeId nodeId) {
     bool readonly = false;
     UA_DataSource dataSourceVariable;
     UA_StatusCode rc; |>\setcounter{lstnumber}{254}<|

     dataSourceVariable.read = readDataSourceVariable;
     if (!readonly)
        dataSourceVariable.write = writeDataSourceVariable;
     else
        dataSourceVariable.write = writeReadonlyDataSourceVariable;

     return UA_Server_setVariableNode_dataSource(server, nodeId, dataSourceVariable);
 }
\end{lstlisting}

\begin{figure}[h]
    \centering
    \includegraphics[width=0.42\textwidth]{doc/img/OPC_RevPiDO.pdf}
    \caption{Auszug des verwendeten Nodesets, hier Digitalausgang 1 des Versuchsaufbaus
      \label{fig:opc-do}}
\end{figure}

Die in Listing~\ref{lst:4-linkDataSourceVariable} abgebildete Methode \lstinline{linkDataSourceVariable()} erzeugt ein Struct vom Typ \lstinline{UA_DataSource}. In diesem werden dem Lesen und Schreiben einer OPC-Variablen entsprechende Callback-Methoden zugewiesen. Die Verknüpfung einer OPC-Variable, genauer ihrer NodeId, mit der zuvor definierten Datenquelle erfolgt über die von open62541 bereitgestellte Methode \lstinline{UA_Server_setVariableNode_dataSource()}. Vor dem Lesen und nach dem Schreiben dieser Variable werden von nun an die entsprechenden Callbacks aufgerufen.
     
\begin{lstlisting}[language={c},firstnumber=168,caption={Auszug des Callbacks \lstinline{writeDataSourceVariable} in \lstinline{variables.c}\label{lst:4-writeDataSourceVariable}}]  
extern UA_StatusCode
 writeDataSourceVariable(UA_Server *server,
            const UA_NodeId *sessionId, void *sessionContext,
            const UA_NodeId *nodeId, void *nodeContext,
            const UA_NumericRange *range, const UA_DataValue *dataValue) {

    UA_StatusCode retval  = UA_STATUSCODE_GOOD;
    UA_NodeId *nameNodeId = UA_malloc(sizeof(UA_NodeId));
    UA_QualifiedName nameQN = UA_QUALIFIEDNAME(1, "Name");
    UA_Variant nameVar;
    UA_Boolean bit;

    retval |= findSiblingByBrowsename(server, nodeId, &nameQN, nameNodeId);
    retval |= UA_Server_readValue(server, *nameNodeId, &nameVar);
    retval |= UA_Boolean_copy(dataValue->value.data, &bit);

    |>\tikzmarkin[set border color=martinired]{writeIO}<|PI_writeSingleIO(String_fromUA_String(nameVar.data), &bit, false);                                                 |>\tikzmarkend{writeIO}<|

    free(nameNodeId);
    return retval;
 }
\end{lstlisting}

Listing~\ref{lst:4-writeDataSourceVariable} zeigt die Callback-Methode, welche nach dem Schreiben einer Variablen auf dem OPC-Server aufgerufen wird.
Dieser Methode wird neben der NodeId der mit ihr verknüpften Variablen auch der Wert dieser in Form eines Zeigers auf ein Struct vom Typ \lstinline{UA_DataValue} übergeben.

Die Gestaltung des hier verwendeten Nodesets sieht vor, dass in einer OPC-Variablen \lstinline{"Name"} der Bezeichner des zu schreibenden Digitalausgangs hinterlegt ist, siehe Abbildung~\ref{fig:opc-do}. Dies erlaubt eine Rekonfiguration der Ein- und Ausgänge der SPS ohne Änderungen im Programmcode des OPC-Servers vornehmen zu müssen.
Es ist daher erforderlich, nach jedem Schreiben einer mit einem Digitalausgang verknüpften Variablen, hier \lstinline{"Value"}, dessen Bezeichner \lstinline{"Name"} abzufragen. 
Dies geschieht in den Zeilen 180 und 181.
Anschließend wird dieser Bezeichner sowie der zu schreibende Wert der Methode \lstinline{PI_writeSingleIO()} übergeben, welche wiederum die Interaktion mit piControl übernimmt (vgl. Abschnitt \ref{sec:4-picontrol}).
 
\subsection{Integration von piControl%
     \label{sec:4-picontrol}}
In Abschnitt~\ref{sec:2-io} wurde die Anbindung der IO-Module des Revolution Pi sowie die Funktionsweise von piControl aus Anwendersicht beschrieben. Die verfügbare Literatur beschränkt sich auch auf lediglich diese Sicht; eine weiterführende Dokumentation für Entwickler gibt es, neben der in Abschnitt~\ref{sec:3-anbindung} vorgestellten Manpage, nicht. 
In diesem Abschnitt soll daher der Quellcode von piControl sowie dessen Verwendung im Projekt genauer betrachtet werden.
Hierzu wird exemplarisch die in Abschnitt~\ref{sec:4-open62541} eingeführte Methode \lstinline{PI_writeSingleIO()} untersucht.
Diese Methode ermöglicht das Setzen eines einzelnen Bits im Prozessabbild der SPS, und damit das Schalten eines digitalen Ausgangs auf einem IO-Modul.
Die äquivalente Methode \lstinline{int piControlGetBitValue(SPIValue *pSpiValue)} zum Lesen eines Bits bzw. Eingangs funktioniert analog und soll daher an dieser Stelle nicht dediziert erörtert werden.

\begin{lstlisting}[language={c},firstnumber=97,
                   caption={Setzen eines phsikalischen, digitalen Ausgangs in \lstinline{revpi.c}
                   \label{lst:4-PI_writeSingleIO}}]
extern void PI_writeSingleIO(char *pszVariableName, bool *bit, bool verbose)
{
	int rc;
	SPIVariable sPiVariable;
	SPIValue sPIValue;

	strncpy(sPiVariable.strVarName, pszVariableName, sizeof(sPiVariable.strVarName));
	rc = piControlGetVariableInfo(&sPiVariable);
	if (rc < 0) {
		printf("Cannot find variable '%s'\n", pszVariableName);
		return;
	}

		sPIValue.i16uAddress = sPiVariable.i16uAddress;
		sPIValue.i8uBit = sPiVariable.i8uBit;
		sPIValue.i8uValue = *bit;
		rc = |>\tikzmarkin[set border color=martinired]{setBitValue}<|piControlSetBitValue(&sPIValue)|>\tikzmarkend{setBitValue}<|;
		if (rc < 0)
			printf("Set bit error %s\n", getWriteError(rc));
		else if (verbose)
			printf("Set bit %d on byte at offset %d. Value %d\n", sPIValue.i8uBit, sPIValue.i16uAddress,
			       sPIValue.i8uValue);
}
\end{lstlisting}

Der Programmcode in Listing~\ref{lst:4-PI_writeSingleIO} ist Teil des implementierten OPC-Servers. In diesem wird auf zwei Funktionen des piControl-Treibers zugegriffen. 
Beiden Methoden wird als Argument ein Zeiger auf ein Struct vom Typ \lstinline{SPIValue} übergeben. Der im Struct abgelegte Name wird mittels \lstinline{piControlGetVariableInfo(&sPIValue)} zu einer Adresse im Prozessabbild aufgelöst. Diese wird in \lstinline{sPIValue.i16uAdress} gespeichert. Der Wert der Variablen wird anschließend mittels \lstinline{piControlSetBitValue(&sPIValue)} an dieser Adresse in das Prozessabbild geschrieben.

\begin{lstlisting}[language={c},firstnumber=309,caption={Methode \lstinline{piControlSetBitValue} in \lstinline{piControlIf.c}\label{lst:4-piControlSetBitValue}}]
int |>\tikzmarkin[set border color=martiniblue]{setBitValueFcn}<|piControlSetBitValue(SPIValue *pSpiValue)|>\tikzmarkend{setBitValueFcn}<|
{
    piControlOpen();

    if (PiControlHandle_g < 0)
	    return -ENODEV;

    pSpiValue->i16uAddress += pSpiValue->i8uBit / 8;
    pSpiValue->i8uBit %= 8;

    if (|>\tikzmarkin[set border color=martinired]{ioctl}<|ioctl(PiControlHandle_g, KB_SET_VALUE, pSpiValue)|>\tikzmarkend{ioctl}<| < 0)
	    return errno;

    return 0;
}
\end{lstlisting}

Die in Listing~\ref{lst:4-piControlSetBitValue} dargestellte Methode \lstinline{piControlSetBitValue} ist lediglich eine Hüllfunktion (häufig auch als Wrapper-Funktion bezeichnet) für einen Aufruf des \lstinline{ioctl} Kernel-Moduls.
Folgende Parameter werden übergeben:
\lstinline{PiControlHandle_g} ist die Referenz auf die Geräte-Datei des piControl-Treibers. \lstinline{KB_SET_VALUE} ist das ioctl-Kommando zum Schreiben eines Bits in das Prozessabbild. Der Zeiger \lstinline{pSpiValue} verweist auf ein Struct des bereits vorgestellten Typs \lstinline{SPIValue}.

\begin{lstlisting}[language={c},firstnumber=80,caption={Methode \lstinline{piControlOpen} in \lstinline{piControlIf.c}\label{lst:4-piControlOpen}}]
void piControlOpen(void)
{
    /* open handle if needed */
    if (PiControlHandle_g < 0)
    {
	    |>\tikzmarkin[set border color=martiniblue]{PiControlHandle}<|PiControlHandle_g = open(PICONTROL_DEVICE, O_RDWR)|>\tikzmarkend{PiControlHandle}<|;
    }
}
\end{lstlisting}

Die in Listing~\ref{lst:4-piControlOpen} dargestellte Methode öffnet, sofern nicht bereits geschehen, die Geräte-Datei. Das Macro \lstinline{PICONTROL_DEVICE} verweist hierbei auf \lstinline{/dev/piControl0}.

\begin{lstlisting}[language={c},firstnumber=721,caption={Methode \lstinline{piControlIoctl} in \lstinline{piControlMain.c}\label{lst:4-piControlIoctl}}]
static long |>\tikzmarkin[set border color=martiniblue, below offset=0.9em]{piControlIoctl}<|piControlIoctl(struct file *file, unsigned int prg_nr, 
                           unsigned long usr_addr)                                      |>\tikzmarkend{piControlIoctl}<|
{
  int status = -EFAULT;
  tpiControlInst *priv;
  int timeout = 10000;	// ms

  if (prg_nr != KB_CONFIG_SEND && prg_nr != KB_CONFIG_START && !isRunning()) {
  	return -EAGAIN;
  }

  priv = (tpiControlInst *) file->private_data;

  if (prg_nr != KB_GET_LAST_MESSAGE) {
  	// clear old message
  	priv->pcErrorMessage[0] = 0;
  }

  switch (prg_nr) {|>\setcounter{lstnumber}{864}<|

    case |>\tikzmarkin[set border color=martiniblue]{KB_SET_VALUE}<|KB_SET_VALUE:|>\tikzmarkend{KB_SET_VALUE}<|
  		{
  			SPIValue *pValue = (SPIValue *) usr_addr;

  			if (!isRunning())
  				return -EFAULT;

  			if (pValue->i16uAddress >= KB_PI_LEN) {
  				status = -EFAULT;
  			} else {
  				INT8U i8uValue_l;
  				my_rt_mutex_lock(&piDev_g.lockPI);
  				i8uValue_l = piDev_g.ai8uPI[pValue->i16uAddress];

  				if (pValue->i8uBit >= 8) {
  					i8uValue_l = pValue->i8uValue;
  				} else {
  					if (pValue->i8uValue)
  						i8uValue_l |= (1 << pValue->i8uBit);
  					else
  						i8uValue_l &= ~(1 << pValue->i8uBit);
  				}

  				|>\tikzmarkin[set border color=martinired]{i8uValue}<|piDev_g.ai8uPI[pValue->i16uAddress] = i8uValue_l;|>\tikzmarkend{i8uValue}<|
  				rt_mutex_unlock(&piDev_g.lockPI);

  #ifdef VERBOSE
  				pr_info("piControlIoctl Addr=%u, bit=%u: %02x %02x\n", pValue->i16uAddress, pValue->i8uBit, pValue->i8uValue, i8uValue_l);
  #endif

  				status = 0;
  			}
  		}
  		break; |>\setcounter{lstnumber}{1314}<|

    default:
      pr_err("Invalid Ioctl");
      return (-EINVAL);
      break;

    }

    return status;
  }
\end{lstlisting}

Listing~\ref{lst:4-piControlIoctl} zeigt in Auszügen die ioctl-Methode des piControl Kernel-Treibers. Diese bekommt folgende Argumente übergeben: \lstinline{struct file *file} enthält den Verweis auf die Geräte-Datei, hier \lstinline{/dev/piControl0}. Der Wert von \lstinline{unsigned int prg_nr} beschreibt die Anfrage an den Treiber, in diesem Fall \lstinline{KB_SET_VALUE}. Das Argument \lstinline{unsigned long usr_addr} enthält einen typ-agnostischen Pointer. Dieser verweist auf einen Speicherbereich, in welchem die zur Bearbeitung der Anfrage notwendigen Daten abgelegt sind. Hier können auch vom Treiber empfangene Daten dem Anwendungsprogramm bereitgestellt werden. 

Die switch-case-Anweisung führt die über das Argument \lstinline{prg_nr} spezifizierte Aktion aus. Hier betrachten wir \lstinline{KB_SET_VALUE}:
Zunächst wird in Zeile 868 der übergebene Zeiger \lstinline{usr_addr} mittels explizitem Typecast zu einem Zeiger des Typs \lstinline{SPIValue *} konvertiert. Da dieser auf Daten im Userspace verweist, ist beim Zugriff durch den Kernel-Treiber besondere Vorsicht geboten.
In Zeile 877 wird mittels Mutex das Prozessabbild \lstinline{piDev_g} für den Zugriff durch andere Threads oder Prozesse gesperrt.
\lstinline{my_rt_mutex_lock} verweist hierbei auf die Funktion \lstinline{rt_mutex_lock} aus \lstinline{linux/sched.h}\footnote{Offenbar wurde hier auch eine alternative Implementierung vorgesehen, siehe revpi\_common.h}

In Zeile 889 wird das Byte \lstinline{i8uValue_l}, welches den zu schreibenden Wert enthält in das Prozessabbild übertragen. Anschließend wird die Mutex auf \lstinline{piDev_g} wieder entsperrt.
\newpage

\begin{lstlisting}[language={c},firstnumber=62,caption={Auszug des Struct \lstinline{spiControlDev} in \lstinline{piControlMain.h}\label{lst:4-spiControlDev}}]
|>\tikzmarkin[set border color=martiniblue]{spiControlDev}<|typedef struct spiControlDev|>\tikzmarkend{spiControlDev}<| {
	// device driver stuff
	int init_step;
	enum revpi_machine machine_type;
	void *machine;
	struct cdev cdev;	// Char device structure
	struct device *dev;
	struct thermal_zone_device *thermal_zone;

	|>\tikzmarkin[set border color=martiniblue]{processImage}<|// process image stuff
	INT8U ai8uPI[KB_PI_LEN];
	INT8U ai8uPIDefault|>\tikzmarkin[set border color=martinired]{KB_PI_LEN_0}<|[KB_PI_LEN]|>\tikzmarkend{KB_PI_LEN_0}<|;
	struct rt_mutex lockPI;        |>\tikzmarkend{processImage}<|
	bool stopIO;
	piDevices *devs; |>\setcounter{lstnumber}{94}<|
} tpiControlDev;
\end{lstlisting}

Das Prozessabbild ist als Byte-Array der Länge \lstinline{KB_PI_LEN} in Listing~\ref{lst:4-spiControlDev} definiert. Konfigurationsparameter wie \lstinline{KB_PI_LEN} oder die Zykluszeit für den Datenaustausch zwischen SPS und IO-Modulen sind im folgenden Listing~\ref{lst:4-process} definiert.

\begin{lstlisting}[language={c},firstnumber=119,caption={Konfigurationsparameter des Prozessabbildes in project.h\label{lst:4-process}}]
#define INTERVAL_PI_GATE (5*1000*1000)  // 5 ms piGateCommunication |>\setcounter{lstnumber}{128}<|

#define INTERVAL_IO_COM (5*1000*1000)  // 5 ms piIoComm |>\setcounter{lstnumber}{132}<|

#define KB_PD_LEN       512
|>\tikzmarkin[set border color=martiniblue]{KB_PI_LEN_1}<|#define KB_PI_LEN       4096|>\tikzmarkend{KB_PI_LEN_1}<|
\end{lstlisting}

Das zu setzende Bit wurde zu diesem Zeitpunkt erfolgreich in das Prozessabbild der SPS geschrieben.
Es stellt sich die Frage, wie dieses nun an das IO-Modul kommuniziert wird.
Die Kommunikation mit allen angebundenen Modulen ist ebenfalls Aufgabe des piControl-Treibers.

\begin{lstlisting}[language={c},firstnumber=256,caption={Auszug der Methode \lstinline{piIoThread} in \lstinline{revpi_core.c}\label{lst:4-piIoThread}}]
static int piIoThread(void *data)
{
	//TODO int value = 0;
	ktime_t time;
	ktime_t now;
	s64 tDiff;

	hrtimer_init(&piCore_g.ioTimer, CLOCK_MONOTONIC, HRTIMER_MODE_ABS);
	piCore_g.ioTimer.function = piIoTimer;

	pr_info("piIO thread started\n");

	now = hrtimer_cb_get_time(&piCore_g.ioTimer);

	PiBridgeMaster_Reset();

	while (!kthread_should_stop()) {
		if (|>\tikzmarkin[set border color=martinired]{PiBridgeMaster}<|PiBridgeMaster_Run()|>\tikzmarkend{PiBridgeMaster}<| < 0)
			break;
	}

	RevPiDevice_finish();

	pr_info("piIO exit\n");
	return 0;
}
\end{lstlisting}

Der Kernel-Thread \lstinline{piIoThread} ist verantwortlich für den zyklischen Datenaustausch mit den IO-Modulen. In diesem wird fortlaufend die Methode \lstinline{PiBridgeMaster_Run()} aufgerufen, siehe Listing~\ref{lst:4-piIoThread}.

\begin{lstlisting}[language={c},firstnumber=262,caption={Auszug der Methode \lstinline{PiBridgeMaster_Run(void)} in \lstinline{RevPiDevice.c}\label{lst:4-PiBridgeMaster_Run}}]
int PiBridgeMaster_Run(void)
{
	static kbUT_Timer tTimeoutTimer_s;
	static kbUT_Timer tConfigTimeoutTimer_s;
	static int error_cnt;
	static INT8U last_led;
	static unsigned long last_update;
	int ret = 0;
	int i;

	my_rt_mutex_lock(&piCore_g.lockBridgeState);
	if (piCore_g.eBridgeState != piBridgeStop) {
		switch (eRunStatus_s) { |>\setcounter{lstnumber}{514}<|
		    case enPiBridgeMasterStatus_EndOfConfig:|>\setcounter{lstnumber}{621}<|
		    if (|>\tikzmarkin[set border color=martinired]{RevPiDevice}<|RevPiDevice_run()|>\tikzmarkend{RevPiDevice}<|) {
				// an error occured, check error limits |>\setcounter{lstnumber}{641}<|
			} else {
				ret = 1;
			}
			piCore_g.image.drv.i16uRS485ErrorCnt = RevPiDevice_getErrCnt();
			break;
\end{lstlisting}

Die in Listing~\ref{lst:4-PiBridgeMaster_Run} dargestellte Methode ist eine sog. State-Machine. Ist die Konfiguration der IO-Module erfolgreich abgeschlossen, so führt sie bei Aufruf lediglich die Methode \lstinline{RevPiDevice_run()} aus.

\begin{lstlisting}[language={c},firstnumber=140,caption={Auszug der Methode \lstinline{RevPiDevice_run(void)} in \lstinline{RevPiDevice.c}\label{lst:4-RevPiDevice_run}}]
int RevPiDevice_run(void)
{
	INT8U i8uDevice = 0;
	INT32U r;
	int retval = 0;

	RevPiDevices_s.i16uErrorCnt = 0;

	for (i8uDevice = 0; i8uDevice < RevPiDevice_getDevCnt(); i8uDevice++) {
		if (RevPiDevice_getDev(i8uDevice)->i8uActive) {
			switch (RevPiDevice_getDev(i8uDevice)->sId.i16uModulType) {
			case KUNBUS_FW_DESCR_TYP_PI_DIO_14:
			case KUNBUS_FW_DESCR_TYP_PI_DI_16:
			case KUNBUS_FW_DESCR_TYP_PI_DO_16:
				r = |>\tikzmarkin[set border color=martinired]{sendCyclicTelegram}<|piDIOComm_sendCyclicTelegram(i8uDevice)|>\tikzmarkend{sendCyclicTelegram}\setcounter{lstnumber}{166} <|;

				break; |>\setcounter{lstnumber}{216}<|
			}
		}
	} |>\setcounter{lstnumber}{227}<|
	return retval;
}
\end{lstlisting}

Diese iteriert wie in Listing~\ref{lst:4-RevPiDevice_run} abgebildete durch alle gegenwärtig in der SPS konfigurierten Module. Ist das aktuelle Modul als aktiv markiert, so wird anhand eines sog. Firmware-Descriptors entschieden, welche Methode für die Ansteuerung des Moduls aufzurufen ist.

\begin{lstlisting}[language={c},firstnumber=161,caption={Auszug der Methode \lstinline{piDIOComm_sendCyclicTelegram} in \lstinline{piDIOComm.c}\label{lst:4-sendCyclicTelegram}}]
INT32U piDIOComm_sendCyclicTelegram(INT8U i8uDevice_p)
{
	INT32U i32uRv_l = 0;
	SIOGeneric sRequest_l;
	SIOGeneric sResponse_l;
	INT8U len_l, data_out[18], i, p, data_in[70];
	INT8U i8uAddress;
	int ret; |>\setcounter{lstnumber}{239}<|
	
    |>\tikzmarkin[set border color=martinired]{piIoComm}<|ret = piIoComm_send((INT8U *) & sRequest_l, IOPROTOCOL_HEADER_LENGTH + len_l + 1);  |>\tikzmarkend{piIoComm}\setcounter{lstnumber}{298}<|
}
\end{lstlisting}

Im Falle des hier verwendeten DO-Moduls wird die in Listing~\ref{lst:4-sendCyclicTelegram} abgebildete Methode \lstinline{piDIOComm_sendCyclicTelegram()} aufgerufen. Dieser wird ein Zeiger auf das zu schreibende Gerät übergeben. 
Zunächst wird das Prozessabbild mittels eines proprietären, jedoch im Quellcode offen nachvollziehbaren Protokolls in ein \lstinline{sRequest_l} genanntes Byte-Array umgewandelt. Dieser Schritt ist in Listing~\ref{lst:4-sendCyclicTelegram} nicht abgebildet. Anschließend wird \lstinline{piIoComm_send()} ein Zeiger auf die so generierte Schreib-Anfrage übergeben.

\begin{lstlisting}[language={c},firstnumber=220,caption={Auszug der Methode \lstinline{piIOComm_send} in \lstinline{piIOComm.c}\label{lst:4-piIOComm_send}}]
int piIoComm_send(INT8U * buf_p, INT16U i16uLen_p)
{
	ssize_t write_l = 0;
	INT16U i16uSent_l = 0;|>\setcounter{lstnumber}{249}<|

	while (i16uSent_l < i16uLen_p) {
		write_l = vfs_write(piIoComm_fd_m, buf_p + i16uSent_l, i16uLen_p - i16uSent_l, &piIoComm_fd_m->f_pos);
		if (write_l < 0) {
			pr_info_serial("write error %d\n", (int)write_l);
			return -1;
		} 
		i16uSent_l += write_l;|>\setcounter{lstnumber}{263}<|
	}
	clear();
	vfs_fsync(piIoComm_fd_m, 1);
	return 0;
}
\end{lstlisting}

Listing~\ref{lst:4-piIOComm_send} zeigt die Implementierung von \lstinline{piIoComm_send()}. Diese Methode ist für das Schreiben der oben generierten Anfrage auf die seriellen Schnittstelle verantwortlich. Realisiert wird dies mittels der Methode \lstinline{vfs_write()}. Diese ist in \lstinline{<linux/fs.h>} definiert. Sie ermöglicht das Schreiben einer Datei im Userspace aus dem Kernel heraus. Geschrieben wird hier die Datei mit dem Deskriptor \lstinline{piIoComm_fd_m}.
Da die Funktion \lstinline{vfs_write()} durch andere Kernel-Tasks unterbrochen werden kann, ist nicht gewährleistet, dass die gesamte Anfrage mit nur einem Aufruf geschrieben wird. Die oben abgebildete while-Schleife stellt das vollständige Senden der Anfrage sicher.

\begin{lstlisting}[language={c},firstnumber=157,caption={Auszug der Methode \lstinline{piIOComm_open_serial} in \lstinline{piIOComm.c}\label{lst:4-piIOComm_open_serial}}]
int piIoComm_open_serial(void)
{   |>\setcounter{lstnumber}{167}<|
	struct file *fd;	/* Filedeskriptor */
	struct termios newtio;	/* Schnittstellenoptionen */

	|>\tikzmarkin[set border color=martiniblue]{fd}<|/* Port oeffnen - read/write, kein "controlling tty", 
	    Status von DCD ignorieren */
	fd = filp_open(|>\tikzmarkin[set border color=martinired]{tty}<|REV_PI_TTY_DEVICE|>\tikzmarkend{tty}<|, O_RDWR | O_NOCTTY, 0); |>\setcounter{lstnumber}{208}<|
	
	piIoComm_fd_m = fd;                                                      |>\tikzmarkend{fd}\setcounter{lstnumber}{217}<|

	return 0;
}
\end{lstlisting}

Der zum Schreiben auf die serielle Schnittstelle verwendete Datei-Deskriptor wird von der in Listing~\ref{lst:4-piIOComm_open_serial} abgebildeten Methode \lstinline{piIoComm_open_serial()} generiert. 

\begin{lstlisting}[language={c},firstnumber=45,caption={Definition der seriellen Schnittstelle in \lstinline{piIOComm.h}\label{lst:4-REV_PI_TTY_DEVICE}}]
#define REV_PI_TTY_DEVICE	"/dev/ttyAMA0"
\end{lstlisting}

Das in Listing~\ref{lst:4-REV_PI_TTY_DEVICE} definierte Macro verweist auf eine der seriellen Schnittstellen des RaspberryPi.
Die Implementierung des zugehörigen Schnittstellentreibers soll hier nicht weiter untersucht werden. Somit ist an dieser Stelle die Kette vom Setzen einer Variablen auf dem OPC-Server bis hin zur Aktualisierung des Prozessabbilds der IO-Module geschlossen.

% \begin{lstlisting}[language={c},firstnumber={226},caption={Setzen der Scheduler-Priorität auf SCHED\_FIFO in 
% revpi\_common.c\label{lst:2-sched_priority}}]
% param.sched_priority = ktprio->prio;
% ret = sched_setscheduler(child, SCHED_FIFO, &param);
% \end{lstlisting}
% % % Imports nur für Referenzenauflösung während des Schreibens! Vorm Kompilieren auskommentieren!
% \bibliography{0_hauptdatei}
% \input{1_einleitung}
% \input{2_grundlagen}
% \input{3_konzeption}
% \input{4_implementierung}
% \input{5_tests}
% \input{6_zusammenfassung}
% % Ende Imports

\section{Test des OPC-Servers im Gesamtsystem%
  \label{sec:5-tests}}

% % % Imports nur für Referenzenauflösung während des schreibens! Vorm Kompilieren auskommentieren!
% \bibliography{0_hauptdatei}
% \input{1_einleitung}
% \input{2_grundlagen}
% \input{3_konzeption}
% \input{4_implementierung}
% \input{5_tests}
% \input{6_zusammenfassung}
% % Ende Imports

\section{Zusammenfassung und Ausblick%
  \label{sec:6-fazit}}
Der folgende Abschnitt~\ref{sec:6-zusammenfassung} fasst die gewonnenen Erkenntnisse und den Stand der Implementierung zusammen.
Den Abschluss dieser Arbeit bildet der Ausblick in Abschnitt~\ref{sec:6-ausblick}.

\subsection{Zusammenfassung%
     \label{sec:6-zusammenfassung}}

\subsection{Ausblick%
     \label{sec:6-ausblick}}

% % Ende Imports

\section{Grundlagen%
  \label{sec:2-grundlagen}}

\subsection{Speicherprogrammierbare-Steuerung und Linux -- Revolution Pi%
     \label{sec:2-sps}}

\subsubsection{Kunbus RevolutionPi%
        \label{sec:2-revpi}}
Der RevolutionPi 3 ist eine speicherprogrammierbare Steuerung (SPS) des Herstellers
Kunbus GmbH. Kern dieser SPS ist das von der Raspberry Pi Foundation entwickelte
und vertriebene Raspberry Pi Compute Module 3. Dieses integriert ein Broadcom BCM2837
System-on-Chip (SoC) mit vier 1,2GHz Prozessorkernen, 1GB RAM, 4GB eMMC Anwendungsspeicher
und sonstige Peripherie in ein Modul im DDR2-SODIMM Formfaktor. Diese Spezifikationen
sind weitgehend identisch zu denen des ausgesprochen populären Raspberry Pi 3.
Der Revolution Pi profitiert daher von dem gleichen großen Angebot an Software
und Unterstützung wie der Raspberry Pi, ergänzt dessen Hardware jedoch um eine 24V
Spannungsversorgung, die Möglichkeit der Erweiterung durch mehrere industrietaugliche
Ein-/ Ausgabemodule und Gateways sowie ein Gehäuse zur Montage auf einer DIN-Schiene.
\begin{itemize}
  \item{Prozessor: BCM2837}
  \item{Taktfrequenz 1,2 GHz}
  \item{Anzahl Prozessorkerne: 4}
  \item{Arbeitsspeicher: 1 GByte}
  \item{eMMC Flash Speicher: 4 GByte}
  \item{Betriebssystem: Angepasstes Raspbian mit RT-Patch}
  \item{RTC mit 24h Pufferung über wartungsfreien Kondensator}
  \item{Treiber / API: Treiber schreibt zyklisch Prozessdaten in ein Prozessabbild, Zugriff auf Prozessabbild über Linux-Filesystem als API zu Fremdsoftware.}
  \item{Kommunikationsanschlüsse: 2 x USB 2.0 A (je 500 mA belastbar), 1 x Micro-USB, HDMI, Ethernet (RJ45) 10/100 Mbit/s}
  \item{Stromversorgung: min. 10,7 V, max. 28,8 V, maximal 10 Watt}
  \item{Zulässige Umgebungstemperatur: -40 bis +55 C}
  \item{Gehäuseabmessungen: (HxBxL) 96 mm x 22,5 mm x 110,5 mm (ohne gesteckte Stecker)}
  \item{ESD Schutz: 4 kV / 8 kV gemäß EN61131-2 und IEC 61000-6-2}
  \item{Surge / Burst Prüfungen: gemäß EN61131-2 und IEC 61000-6-2 eingekoppelt auf Versorgungsspannung, Ethernet und IO-Leitungen}
  \item{EMI Prüfungen: gemäß EN61131-2 und IEC 61000-6-2}
\end{itemize}

Kunbus bietet eine Auswahl an IO- und Gateway-Modulen zur Erweiterung des Revolution Pi an.
Gateways dienen der Kommunikation mit Systemen oder Komponenten der Automatisierungstechnik
über Protokolle wie PROFIBUS oder EtherCAT. IO-Module erlauben die Überwachung
und Steuerung von digitalen oder analogen Ein- und Ausgängen.

\subsubsection{Zugriff auf IO-Module%
        \label{sec:2-io}}
Der Zugriff auf die Ein- und Ausgänge der IO-Module erfolgt über ein Prozessabbild
und einen hierfür von Kunbus bereitgestellten Treiber, genannt piControl. Dieser
aktualisiert das Prozessabbild zyklisch. Die angestrebte Zykluszeit beträgt 5ms,
kann jedoch je nach Anzahl der angeschlossenen Module auch größer sein. Kunbus
garantiert bei drei IO-Modulen und zwei Gateway-Modulen eine Zykluszeit von 10 ms.
Jedes der IO-Module stellt ein eigenständiges eingebettetes System dar. Es verfügt
über einen Microcontroller, welcher die IOs bereitstellt und über einen RS485-Bus
mit dem Revolution Pi kommuniziert.
% https://revolution.kunbus.de/io-modul/

Lizenz: GPL
% https://github.com/RevolutionPi/piControl

\begin{lstlisting}[language={c},firstnumber={226},caption={Setzen der Scheduler-Priorität auf SCHED\_FIFO in revpi\_common.c\label{lst:2-sched_priority}}]
param.sched_priority = ktprio->prio;
ret = sched_setscheduler(child, SCHED_FIFO,
       &param);
\end{lstlisting}


\subsection{Echtzeit und Multithreading unter Linux -- preemptRT und posix%
     \label{sec:2-echtzeit}}


 Der Linux-Kernel verfügt über mehrere unterschiedliche Preemtion-Modelle:

\begin{itemize}
  \item No Forced Preemption (server):
  Ausgelegt auf maximal möglichen Durchsatz, lediglich Interrupts und
  System-Call-Returns bewirken Präemption.

  \item Voluntary Kernel Preemption (Desktop):
  Neben den implizit bevorrechtigten Interrupts und System-Call-Returns gibt es
  in diesem Modell weitere Abschnitte des Kernels in welchen Preämption explizit
  gestattet ist.

  \item Preemptible Kernel (Low-Latency Desktop):
  In diesem Modell ist der gesamte Kernel, mit Ausnahme sog.~kritischer Abschnitte
  präemptible. Nach jedem kritischen Abschnitt gibt es einen impliziten Präemptions-Punkt.

  \item Preemptible Kernel (Basic RT):
  Dieses Modell ist dem zuvor genannten sehr ähnlich, hier sind jedoch alle Interrupt-Handler
  als eigenständige Threads ausgeführt.

  \item Fully Preemptible Kernel (RT):
  Wie auch bei den beiden zuvor genannten Modellen ist hier der gesamte Kernel
  präemtible, die Anzahl und Dauer der nicht-präemtiblen kritischen Abschnitte
  ist auf ein notwendiges Minimum beschränkt. Alle Interrupt-Handler sind als
  eigenständige Threads ausgeführt, Spinlocks durch Sleeping-Spinlocks und Mutexe
  durch sog.~RT-Mutexe ersetzt.

\end{itemize}
\todo{Spinlocks und Mutexe sowie die RT-Varianten dieser erklären!}

Lediglich mit dem vollständig präemtiblen Kernel kann Echtzeit-Verhalten realisiert werden.

% https://wiki.linuxfoundation.org/realtime/documentation/technical_basics/preemption_models bzw kernel/Kconfig.preempt

\subsubsection{preemptRT%
        \label{sec:2-preemptRT}}
% https://wiki.linuxfoundation.org/realtime/documentation/technical_details/start
% https://wiki.linuxfoundation.org/realtime/documentation/technical_basics/start

Das dem PREEMPT RT Kernel zugrunde liegende Prinzip lässt sich in einer einfachen
Regel ausdrücken: Nur Code, welcher absolut nicht-präemtible sein darf, ist es
gestattet nicht-präemtible zu sein.
Das erklärte Ziel des PREEMPT\_RT Patches ist es folglich, die Menge des nicht-präemtiblen
Codes im Linux-Kernel auf das absolut notwendige Minimum zu reduzieren.

Dies wird durch Verwendung folgender Mechanismen erreicht:

\begin{itemize}
  \item Hochauflösende Timer
  \item Sleeping Spinlocks
  \item Threaded Interrupt Handlers
  \item rt\_mutex
  \item RCU
\end{itemize}


\subsubsection{posix%
        \label{sec:2-posix}}
Ist posix hier wirklich relevant? Debian bzw.~Raspbian sind weitgehend posix
kompatibel, aber wird es hier genutzt? -> JA, open62541 nutzt pthread.h
piControl nutzt kthread.h, und semaphore.h

\subsection{OPC-UA und open62541%
     \label{sec:2-opc}}

\subsubsection{OPC UA%
        \label{sec:2-opcua}}
Open Platform Communications (OPC) ist eine Familie von Standards zur herstellerunabhängigen
Kommunikation von Maschinen (M2M) in der Automatisierungstechnik. Die sog.~OPC Task Force, zu deren
Mitgliedern verschiedene große Firmen der Automatisierungsindustrie gehören, veröffentlichte
die OPC Specification Version 1.0 im August 1996.
Motiviert ist dieser offene Standard durch die Erkenntniss, dass die Anpassung der
zahlreichen Herstellerstandards an individuelle Infrastrukturen und Anlagen einen
großen Mehraufwand verursachen.
Die Wikipedia beschreibt das Anwendungsgebiet für OPC wie folgt:

\glqq{}OPC wird dort eingesetzt, wo Sensoren, Regler und Steuerungen verschiedener Hersteller
ein gemeinsames Netzwerk bilden. Ohne OPC benötigten zwei Geräte zum Datenaustausch
genaue Kenntnis über die Kommunikationsmöglichkeiten des Gegenübers. Erweiterungen
und Austausch gestalten sich entsprechend schwierig. Mit OPC genügt es, für jedes
Gerät genau einmal einen OPC-konformen Treiber zu schreiben. Idealerweise wird
dieser bereits vom Hersteller zur Verfügung gestellt. Ein OPC-Treiber lässt sich
ohne großen Anpassungsaufwand in beliebig große Steuer- und Überwachungssysteme
integrieren.

OPC unterteilt sich in verschiedene Unterstandards, die für den jeweiligen Anwendungsfall
unabhängig voneinander implementiert werden können. OPC lässt sich damit verwenden
für Echtzeitdaten (Überwachung), Datenarchivierung, Alarm-Meldungen und neuerdings
auch direkt zur Steuerung (Befehlsübermittlung).\grqq{}

OPC basiert in der ursprünglichen Spezifikation auf Microsofts DCOM-Spezifikation.
DCOM macht Funktionen und Objekte einer Anwendung anderen Anwendungen im Netzwerk
zugänglich. Der OPC-Standard definiert entsprechende DCOM-Objekte um mit anderen
OPC-Anwendungen Daten austauschen zu können. Die Verwendung von DCOM bindet Anwender
an Betriebssysteme von Microsoft. Die ursprüngliche OPC Spezifikation wird durch die
Entwicklung von OPC Unified Architecture (OPC UA) abgelöst.
OPC UA setzt auf einem eigenen Kommunikationionsstack auf, die Verwendung von DCOM
und damit die Bindung an Microsoft wurden aufgelöst.

Die OPC-UA-Architektur ist eine Service-orientierte Architektur (SOA), deren Struktur
aus mehreren Schichten besteht.

% Wikipedia
Das OPC-Informationsmodell ist nicht mehr nur eine Hierarchie aus Ordnern, Items
und Properties. Es ist ein sogenanntes Full-Mesh-Network aus Nodes, mit dem neben
den Nutzdaten eines Nodes auch Meta- und Diagnoseinformationen repräsentiert werden.
Ein Node ähnelt einem Objekt aus der objektorientierten Programmierung. Ein Node
kann Attribute besitzen, die gelesen werden können (Data Access (DA), Historical
Data Access (HDA)). Es ist möglich Methoden zu definieren und aufzurufen.
Eine Methode besitzt Aufrufargumente und Rückgabewerte. Sie wird durch ein Command
aufgerufen. Weiterhin werden Events unterstützt, die versendet werden können
(AE (Alarms \& Events), DA DataChange), um bestimmte Informationen zwischen Geräten
auszutauschen. Ein Event besitzt unter anderem einen Empfangszeitpunkt, eine Nachricht
und einen Schweregrad. Die o. g. Nodes werden sowohl für die Nutzdaten als auch
alle anderen Arten von Metadaten verwendet. Der damit modellierte OPC-Adressraum
beinhaltet nun auch ein Typmodell, mit dem sämtliche Datentypen spezifiziert werden.

% https://de.wikipedia.org/wiki/Open_Platform_Communications
% https://de.wikipedia.org/wiki/OPC_Unified_Architecture
% https://opcfoundation.org/developer-tools/specifications-unified-architecture
% Von Gerhard Gappmeier - ascolab GmbH, CC BY-SA 3.0, https://de.wikipedia.org/w/index.php?curid=1892069
\subsubsection{open62541%
        \label{sec:2-open62541}}
open62541 ist eine offene und freie Implementierung von OPC UA. Die in C geschriebene
Bibliothek stellt eine beständig zunehmende Anzahl der im OPC UA Standard definierten
Funktionen bereit. Sie kann sowohl zur Erstellung von OPC-Servern als auch -Clients
genutzt werden. Ergänzend zu der unter der Mozilla Public License v2.0 lizensierten
Bibliothek stellt das open62541 Projekt auch Beispielprogramme unter einer CC0 Lizenz
zur Verfügung.

Die Bibliothek eignet sich auch für die Entwicklung auf eingebetteten Systemen und
Microcontrollern. Je nach Umfang der gewünschten Funktionen und des OPC Informationsmodells
beträgt die Größe einer Server-Binary weniger als 100kb. %evtl. kürzen?

\todo{Nodes erklären! Evtl.~oben!}

Folgende Auswahl an Eigenschaften und Funktionen zeichnet die in dieser Arbeit verwendete
Version 0.3 von open62541 aus:
\begin{itemize}
  \item Kommunikationionsstack
  \begin{itemize}
      \item OPC UA Binär-Protokoll (HTTP oder SOAP werden gegenwärtig nicht unterstützt)
      \item Austauschbare Netzwerk-Schicht, welche die Verwendung eigener Netzwerk-APIs
      erlaubt.
      \item Verschlüsselte Kommunikationion
      \item Asynchrone Dienst-Anfragen im Client
  \end{itemize}
  \item Informationsmodell
  \begin{itemize}
    \item Unterstützung aller OPC UA Node-Typen, inkl.~Methoden
    \item Hinzufügen und Entfernen von Nodes und Referenzen zur Laufzeit.
    \item Vererbung und Instanziierung von Objekt- und Variablentypen
    \item Zugriffskontrolle auch für einzelne Nodes
  \end{itemize}
  \item Subscriptions
  \begin{itemize}
    \item Erlaubt die Überwachung (subscriptions / monitoreditems)
    \item Sehr geringer Ressourcenbedarf pro überwachtem Wert
  \end{itemize}
  \item Code-Generierung auf XML-Basis
  \begin{itemize}
    \item Erlaubt die Erstellung von Datentypen
    \item Erlaubt die Generierung des serverseitigen Informationsmodells
  \end{itemize}
\end{itemize}

% https://open62541.org/doc/0.3/


Mozilla Public License
CC0 Lizenz für Beispiele und Plugins

% https://open62541.org/doc/open62541-current.pdf
% https://open62541.org/

% % % Imports nur für Referenzenauflösung während des Schreibens! Vorm Kompilieren auskommentieren!
% \bibliography{0_hauptdatei}
% % Mit \section{...} eröffnen wir einen neuen Abschnitt.
% Der Befehl setzt nicht nur den Text in einer größeren,
% fetten Schrift, sondern sorgt außerdem dafür, daß er im
% Inhaltsverzeichnis erscheint.
%
% Mit \label{...} erzeugen wir einen Bezeichner, mit dessen Hilfe
% wir später auf die Nummer des Abschnitts verweisen können (nämlich
% mit~\ref{...}).
%
% Das Kommentarzeichen hinter „Übersicht“ dient dazu, ein
% Leerzeichen zwischen „Übersicht“ und dem \label-Befehl
% zu vermeiden, das andernfalls sichtbar würde – z.B. im
% Inhaltsverzeichnis.
%

% % Imports nur für Referenzenauflösung während des Schreibens! Vorm Kompilieren auskommentieren!
% \bibliography{0_hauptdatei}
% \input{1_einleitung}
%\input{2_grundlagen}
%\input{3_konzeption}
%\input{4_implementierung}
%\input{5_tests}
%\input{6_zusammenfassung}
% % Ende Imports

\section{Einleitung und Motivation%
  \label{sec:1-einleitung}}
Ziel dieses Projektes ist die Integration eines OPC-Servers mit einer auf Linux
basierenden speicherprogrammierbaren Steuerung (SPS). Angeschlossen an diese SPS
ist jeweils ein digitales Ein-/\,bzw.~Ausgabemodul. Die von diesen bereitgestellten
Ein-/\, bzw.~Ausgänge (IO) sollen in der Datenstruktur des OPC-Servers abgebildet
und über diesen für OPC-Clients les-/\,und schreibar sein. Weiterhin sollen einige
Funktionen zur Überwachung und Steuerung der an die SPS angeschlossenen Aktoren
und Sensoren direkt im OPC-Server implementiert werden.
Hiermit stellt dieses Projekt eine der Grundlagen für ein übergeordnetes Projekt,
die cloudbasierte Steuerung eines miniaturisierten Produktions-Systems, dar.

Der hier verwendete OPC-Server ist Teil des sog. open62541 Projekts. Er ist in C
geschrieben und implementiert bereits einen großen Teil der im OPC-UA-Standard
spezifizierten Funktionen.
Als SPS findet ein Revolution Pi 3 der Firma Kunbus Verwendung. Dieser integriert
ein sog. Compute Module der Raspberry Pi Foundation in ein industrietaugliches
Gehäuse und erlaubt die Erweiterung mittels IO- oder Gateway-Modulen. Über diese
erfolgt die Kommunikation mit weiteren Komponenten der Automatisierungstechnik.

Motiviert ist dieses Projekt durch die Beobachtung, dass die Verbreitung offener
Standards sowie freier Software auch in der Automatisierungstechnik zunimmt.
Linux ist ein freies Betriebssystem, OPC-UA ein offen zugänglicher, aktiv gepflegter
und weit verbreiteter Standard. Der Raspberry Pi findet sowohl bei Hobby-Anwendern als
auch in den Bereichen Forschung und Entwicklung sowie bei industriellen Anwendern
Verwendung. Dieses Projekt stellt somit eine für unterschiedliche Anwender interessante
Entwicklung dar.

Im Anschluss an diese einleitende Übersicht im Abschnitt~\ref{sec:1-einleitung} folgt
die Darstellung der wichtigsten Grundlagen in Abschnitt~\ref{sec:2-grundlagen}.
Aufbauend auf diesen Grundlagen folgt die konzeptuelle Ausarbeitung im Abschnitt~\ref{sec:3-konzeption}.
Die Umsetzung wird im Abschnitt~\ref{sec:4-implementierung} erläutert.
Die Leistungsfähigkeit der Implementierung wird in Abschnitt~\ref{sec:5-tests} untersucht.
Eine Zusammenfassung und ein Ausblick schließen die Arbeit in
Abschnitt~\ref{sec:6-fazit} ab. Eventuell noch benötigte Anhänge
finden sich in den Anhängen [...] bis [...].

% % % Imports nur für Referenzenauflösung während des Schreibens! Vorm Kompilieren auskommentieren!
% \bibliography{0_hauptdatei}
% \input{1_einleitung}
% \input{2_grundlagen}
% \input{3_konzeption}
% \input{4_implementierung}
% \input{5_tests}
% \input{6_zusammenfassung}
% % Ende Imports

\section{Grundlagen%
  \label{sec:2-grundlagen}}

\subsection{Speicherprogrammierbare-Steuerung und Linux -- Revolution Pi%
     \label{sec:2-sps}}

\subsubsection{Kunbus RevolutionPi%
        \label{sec:2-revpi}}
Der RevolutionPi 3 ist eine speicherprogrammierbare Steuerung (SPS) des Herstellers
Kunbus GmbH. Kern dieser SPS ist das von der Raspberry Pi Foundation entwickelte
und vertriebene Raspberry Pi Compute Module 3. Dieses integriert ein Broadcom BCM2837
System-on-Chip (SoC) mit vier 1,2GHz Prozessorkernen, 1GB RAM, 4GB eMMC Anwendungsspeicher
und sonstige Peripherie in ein Modul im DDR2-SODIMM Formfaktor. Diese Spezifikationen
sind weitgehend identisch zu denen des ausgesprochen populären Raspberry Pi 3.
Der Revolution Pi profitiert daher von dem gleichen großen Angebot an Software
und Unterstützung wie der Raspberry Pi, ergänzt dessen Hardware jedoch um eine 24V
Spannungsversorgung, die Möglichkeit der Erweiterung durch mehrere industrietaugliche
Ein-/ Ausgabemodule und Gateways sowie ein Gehäuse zur Montage auf einer DIN-Schiene.
\begin{itemize}
  \item{Prozessor: BCM2837}
  \item{Taktfrequenz 1,2 GHz}
  \item{Anzahl Prozessorkerne: 4}
  \item{Arbeitsspeicher: 1 GByte}
  \item{eMMC Flash Speicher: 4 GByte}
  \item{Betriebssystem: Angepasstes Raspbian mit RT-Patch}
  \item{RTC mit 24h Pufferung über wartungsfreien Kondensator}
  \item{Treiber / API: Treiber schreibt zyklisch Prozessdaten in ein Prozessabbild, Zugriff auf Prozessabbild über Linux-Filesystem als API zu Fremdsoftware.}
  \item{Kommunikationsanschlüsse: 2 x USB 2.0 A (je 500 mA belastbar), 1 x Micro-USB, HDMI, Ethernet (RJ45) 10/100 Mbit/s}
  \item{Stromversorgung: min. 10,7 V, max. 28,8 V, maximal 10 Watt}
  \item{Zulässige Umgebungstemperatur: -40 bis +55 C}
  \item{Gehäuseabmessungen: (HxBxL) 96 mm x 22,5 mm x 110,5 mm (ohne gesteckte Stecker)}
  \item{ESD Schutz: 4 kV / 8 kV gemäß EN61131-2 und IEC 61000-6-2}
  \item{Surge / Burst Prüfungen: gemäß EN61131-2 und IEC 61000-6-2 eingekoppelt auf Versorgungsspannung, Ethernet und IO-Leitungen}
  \item{EMI Prüfungen: gemäß EN61131-2 und IEC 61000-6-2}
\end{itemize}

Kunbus bietet eine Auswahl an IO- und Gateway-Modulen zur Erweiterung des Revolution Pi an.
Gateways dienen der Kommunikation mit Systemen oder Komponenten der Automatisierungstechnik
über Protokolle wie PROFIBUS oder EtherCAT. IO-Module erlauben die Überwachung
und Steuerung von digitalen oder analogen Ein- und Ausgängen.

\subsubsection{Zugriff auf IO-Module%
        \label{sec:2-io}}
Der Zugriff auf die Ein- und Ausgänge der IO-Module erfolgt über ein Prozessabbild
und einen hierfür von Kunbus bereitgestellten Treiber, genannt piControl. Dieser
aktualisiert das Prozessabbild zyklisch. Die angestrebte Zykluszeit beträgt 5ms,
kann jedoch je nach Anzahl der angeschlossenen Module auch größer sein. Kunbus
garantiert bei drei IO-Modulen und zwei Gateway-Modulen eine Zykluszeit von 10 ms.
Jedes der IO-Module stellt ein eigenständiges eingebettetes System dar. Es verfügt
über einen Microcontroller, welcher die IOs bereitstellt und über einen RS485-Bus
mit dem Revolution Pi kommuniziert.
% https://revolution.kunbus.de/io-modul/

Lizenz: GPL
% https://github.com/RevolutionPi/piControl

\begin{lstlisting}[language={c},firstnumber={226},caption={Setzen der Scheduler-Priorität auf SCHED\_FIFO in revpi\_common.c\label{lst:2-sched_priority}}]
param.sched_priority = ktprio->prio;
ret = sched_setscheduler(child, SCHED_FIFO,
       &param);
\end{lstlisting}


\subsection{Echtzeit und Multithreading unter Linux -- preemptRT und posix%
     \label{sec:2-echtzeit}}


 Der Linux-Kernel verfügt über mehrere unterschiedliche Preemtion-Modelle:

\begin{itemize}
  \item No Forced Preemption (server):
  Ausgelegt auf maximal möglichen Durchsatz, lediglich Interrupts und
  System-Call-Returns bewirken Präemption.

  \item Voluntary Kernel Preemption (Desktop):
  Neben den implizit bevorrechtigten Interrupts und System-Call-Returns gibt es
  in diesem Modell weitere Abschnitte des Kernels in welchen Preämption explizit
  gestattet ist.

  \item Preemptible Kernel (Low-Latency Desktop):
  In diesem Modell ist der gesamte Kernel, mit Ausnahme sog.~kritischer Abschnitte
  präemptible. Nach jedem kritischen Abschnitt gibt es einen impliziten Präemptions-Punkt.

  \item Preemptible Kernel (Basic RT):
  Dieses Modell ist dem zuvor genannten sehr ähnlich, hier sind jedoch alle Interrupt-Handler
  als eigenständige Threads ausgeführt.

  \item Fully Preemptible Kernel (RT):
  Wie auch bei den beiden zuvor genannten Modellen ist hier der gesamte Kernel
  präemtible, die Anzahl und Dauer der nicht-präemtiblen kritischen Abschnitte
  ist auf ein notwendiges Minimum beschränkt. Alle Interrupt-Handler sind als
  eigenständige Threads ausgeführt, Spinlocks durch Sleeping-Spinlocks und Mutexe
  durch sog.~RT-Mutexe ersetzt.

\end{itemize}
\todo{Spinlocks und Mutexe sowie die RT-Varianten dieser erklären!}

Lediglich mit dem vollständig präemtiblen Kernel kann Echtzeit-Verhalten realisiert werden.

% https://wiki.linuxfoundation.org/realtime/documentation/technical_basics/preemption_models bzw kernel/Kconfig.preempt

\subsubsection{preemptRT%
        \label{sec:2-preemptRT}}
% https://wiki.linuxfoundation.org/realtime/documentation/technical_details/start
% https://wiki.linuxfoundation.org/realtime/documentation/technical_basics/start

Das dem PREEMPT RT Kernel zugrunde liegende Prinzip lässt sich in einer einfachen
Regel ausdrücken: Nur Code, welcher absolut nicht-präemtible sein darf, ist es
gestattet nicht-präemtible zu sein.
Das erklärte Ziel des PREEMPT\_RT Patches ist es folglich, die Menge des nicht-präemtiblen
Codes im Linux-Kernel auf das absolut notwendige Minimum zu reduzieren.

Dies wird durch Verwendung folgender Mechanismen erreicht:

\begin{itemize}
  \item Hochauflösende Timer
  \item Sleeping Spinlocks
  \item Threaded Interrupt Handlers
  \item rt\_mutex
  \item RCU
\end{itemize}


\subsubsection{posix%
        \label{sec:2-posix}}
Ist posix hier wirklich relevant? Debian bzw.~Raspbian sind weitgehend posix
kompatibel, aber wird es hier genutzt? -> JA, open62541 nutzt pthread.h
piControl nutzt kthread.h, und semaphore.h

\subsection{OPC-UA und open62541%
     \label{sec:2-opc}}

\subsubsection{OPC UA%
        \label{sec:2-opcua}}
Open Platform Communications (OPC) ist eine Familie von Standards zur herstellerunabhängigen
Kommunikation von Maschinen (M2M) in der Automatisierungstechnik. Die sog.~OPC Task Force, zu deren
Mitgliedern verschiedene große Firmen der Automatisierungsindustrie gehören, veröffentlichte
die OPC Specification Version 1.0 im August 1996.
Motiviert ist dieser offene Standard durch die Erkenntniss, dass die Anpassung der
zahlreichen Herstellerstandards an individuelle Infrastrukturen und Anlagen einen
großen Mehraufwand verursachen.
Die Wikipedia beschreibt das Anwendungsgebiet für OPC wie folgt:

\glqq{}OPC wird dort eingesetzt, wo Sensoren, Regler und Steuerungen verschiedener Hersteller
ein gemeinsames Netzwerk bilden. Ohne OPC benötigten zwei Geräte zum Datenaustausch
genaue Kenntnis über die Kommunikationsmöglichkeiten des Gegenübers. Erweiterungen
und Austausch gestalten sich entsprechend schwierig. Mit OPC genügt es, für jedes
Gerät genau einmal einen OPC-konformen Treiber zu schreiben. Idealerweise wird
dieser bereits vom Hersteller zur Verfügung gestellt. Ein OPC-Treiber lässt sich
ohne großen Anpassungsaufwand in beliebig große Steuer- und Überwachungssysteme
integrieren.

OPC unterteilt sich in verschiedene Unterstandards, die für den jeweiligen Anwendungsfall
unabhängig voneinander implementiert werden können. OPC lässt sich damit verwenden
für Echtzeitdaten (Überwachung), Datenarchivierung, Alarm-Meldungen und neuerdings
auch direkt zur Steuerung (Befehlsübermittlung).\grqq{}

OPC basiert in der ursprünglichen Spezifikation auf Microsofts DCOM-Spezifikation.
DCOM macht Funktionen und Objekte einer Anwendung anderen Anwendungen im Netzwerk
zugänglich. Der OPC-Standard definiert entsprechende DCOM-Objekte um mit anderen
OPC-Anwendungen Daten austauschen zu können. Die Verwendung von DCOM bindet Anwender
an Betriebssysteme von Microsoft. Die ursprüngliche OPC Spezifikation wird durch die
Entwicklung von OPC Unified Architecture (OPC UA) abgelöst.
OPC UA setzt auf einem eigenen Kommunikationionsstack auf, die Verwendung von DCOM
und damit die Bindung an Microsoft wurden aufgelöst.

Die OPC-UA-Architektur ist eine Service-orientierte Architektur (SOA), deren Struktur
aus mehreren Schichten besteht.

% Wikipedia
Das OPC-Informationsmodell ist nicht mehr nur eine Hierarchie aus Ordnern, Items
und Properties. Es ist ein sogenanntes Full-Mesh-Network aus Nodes, mit dem neben
den Nutzdaten eines Nodes auch Meta- und Diagnoseinformationen repräsentiert werden.
Ein Node ähnelt einem Objekt aus der objektorientierten Programmierung. Ein Node
kann Attribute besitzen, die gelesen werden können (Data Access (DA), Historical
Data Access (HDA)). Es ist möglich Methoden zu definieren und aufzurufen.
Eine Methode besitzt Aufrufargumente und Rückgabewerte. Sie wird durch ein Command
aufgerufen. Weiterhin werden Events unterstützt, die versendet werden können
(AE (Alarms \& Events), DA DataChange), um bestimmte Informationen zwischen Geräten
auszutauschen. Ein Event besitzt unter anderem einen Empfangszeitpunkt, eine Nachricht
und einen Schweregrad. Die o. g. Nodes werden sowohl für die Nutzdaten als auch
alle anderen Arten von Metadaten verwendet. Der damit modellierte OPC-Adressraum
beinhaltet nun auch ein Typmodell, mit dem sämtliche Datentypen spezifiziert werden.

% https://de.wikipedia.org/wiki/Open_Platform_Communications
% https://de.wikipedia.org/wiki/OPC_Unified_Architecture
% https://opcfoundation.org/developer-tools/specifications-unified-architecture
% Von Gerhard Gappmeier - ascolab GmbH, CC BY-SA 3.0, https://de.wikipedia.org/w/index.php?curid=1892069
\subsubsection{open62541%
        \label{sec:2-open62541}}
open62541 ist eine offene und freie Implementierung von OPC UA. Die in C geschriebene
Bibliothek stellt eine beständig zunehmende Anzahl der im OPC UA Standard definierten
Funktionen bereit. Sie kann sowohl zur Erstellung von OPC-Servern als auch -Clients
genutzt werden. Ergänzend zu der unter der Mozilla Public License v2.0 lizensierten
Bibliothek stellt das open62541 Projekt auch Beispielprogramme unter einer CC0 Lizenz
zur Verfügung.

Die Bibliothek eignet sich auch für die Entwicklung auf eingebetteten Systemen und
Microcontrollern. Je nach Umfang der gewünschten Funktionen und des OPC Informationsmodells
beträgt die Größe einer Server-Binary weniger als 100kb. %evtl. kürzen?

\todo{Nodes erklären! Evtl.~oben!}

Folgende Auswahl an Eigenschaften und Funktionen zeichnet die in dieser Arbeit verwendete
Version 0.3 von open62541 aus:
\begin{itemize}
  \item Kommunikationionsstack
  \begin{itemize}
      \item OPC UA Binär-Protokoll (HTTP oder SOAP werden gegenwärtig nicht unterstützt)
      \item Austauschbare Netzwerk-Schicht, welche die Verwendung eigener Netzwerk-APIs
      erlaubt.
      \item Verschlüsselte Kommunikationion
      \item Asynchrone Dienst-Anfragen im Client
  \end{itemize}
  \item Informationsmodell
  \begin{itemize}
    \item Unterstützung aller OPC UA Node-Typen, inkl.~Methoden
    \item Hinzufügen und Entfernen von Nodes und Referenzen zur Laufzeit.
    \item Vererbung und Instanziierung von Objekt- und Variablentypen
    \item Zugriffskontrolle auch für einzelne Nodes
  \end{itemize}
  \item Subscriptions
  \begin{itemize}
    \item Erlaubt die Überwachung (subscriptions / monitoreditems)
    \item Sehr geringer Ressourcenbedarf pro überwachtem Wert
  \end{itemize}
  \item Code-Generierung auf XML-Basis
  \begin{itemize}
    \item Erlaubt die Erstellung von Datentypen
    \item Erlaubt die Generierung des serverseitigen Informationsmodells
  \end{itemize}
\end{itemize}

% https://open62541.org/doc/0.3/


Mozilla Public License
CC0 Lizenz für Beispiele und Plugins

% https://open62541.org/doc/open62541-current.pdf
% https://open62541.org/

% % % Imports nur für Referenzenauflösung während des Schreibens! Vorm Kompilieren auskommentieren!
% \bibliography{0_hauptdatei}
% \input{1_einleitung}
% \input{2_grundlagen}
% \input{3_konzeption}
% \input{4_implementierung}
% \input{5_tests}
% \input{6_zusammenfassung}
% \input{anhang}
% % Ende Imports

\section{Systemkonzept%
  \label{sec:3-konzeption}}
Auf Basis der in Abschnitt \ref{sec:2-grundlagen} vorgestellten Möglichkeiten folgt nun die Ausarbeitung eines Konzepts.
In den folgenden Abschnitten soll näher auf zwei zentrale Aspekte eingegangen werden: Abschnitt~\ref{sec:3-anbindung} stellt Möglichkeiten zum Zugriff auf Variablen bzw.\,Werte im Prozessabbild des Revolution Pi vor; in Abschnitt~\ref{sec:3-integration} wird ein Konzept zur Bereitstellung dieser Variablen auf einem OPC-Server vorgestellt.

\subsection{Anbindung der IO an den OPC-Server%
     \label{sec:3-anbindung}}

Eine Webanwendung mit Bezeichnung PiCtory dient zur Konfiguration der I/O- und virtuellen Module des RevolutionPi. Die Konfiguration liegt im JSON-Format in der Datei \lstinline{/etc/revpi/config.rsc}. Der piControl-Treiber liest diese Datei beim Start. 
Der folgende Auszug aus der Manpage des piControl-Kernelmoduls beschreibt die von diesem zum Lesen und Schreiben einzelner Bits des Prozessabbildes bereitgestellten Funktionen~\citep[vgl.]{web-revpi-manpage}. Sie ist an dieser Stelle weitgehend ungekürzt zitiert, da sie die nutzbare Schnittstelle sehr kompakt beschreibt.

\begin{lstlisting}[breakindent=0pt, numbers=none, caption={Auszug aus der Revolution Pi Programmers Manual\label{lst:4-manpage}}]
KB_FIND_VARIABLE SPIVariable *argp
Find a variable in the process image by its name. A pointer to a structure of type SPIVariable must be passed as argument. [...]
The struct SPIVariable [...] is defined as 
typedef struct SPIVariableStr
{
    char strVarName[32]; // Variable name
    uint16_t i16uAddress; // Address of the byte in the process image
    uint8_t i8uBit; // 0-7 bit position, >= 8 whole byte
    uint16_t i16uLength; // length of the variable in bits.
    // Possible values are 1, 8, 16 and 32
} SPIVariable;

Set and get values of the process image
KB_GET_VALUE SPIValue *argp
[...]
KB_SET_VALUE SPIValue *argp
Write one bit or one byte to the process image [...].  This call is more efficient than the usual calls of seek and write because only one function call is necessary. If more than on application are writing bits in one output byte, this call is the only safe way to set a bit without overwriting the other bits because this call is doing a read-modify-write-cycle. 

The struct SPIValue used by this ioctl is defined as
typedef struct SPIValueStr
{
    uint16_t i16uAddress; // Address of the byte in the process image
    uint8_t i8uBit; // 0-7 bit position, >= 8 whole byte
    uint8_t i8uValue; // Value: 0/1 for bit access, whole byte otherwise
} SPIValue;
\end{lstlisting} 

Die oben beschriebenden Funtkionen \lstinline{KB_FIND_VARIABLE}, \lstinline{KB_GET_VALUE} und \lstinline{KB_SET_VALUE} ermöglichen einen einfachen und (lt.\,Manpage) effizienten Zugriff auf einzelne Bits des Prozessabbildes und damit auch auf die IO des RevolutionPi.
Der Zugriff des OPC-Servers auf das Prozessabbild soll daher mittels dieser Funktionen realisiert werden.
\lstinline{KB_FIND_VARIABLE} kann genutzt werden, um Adressen von Variablen im Prozessabbild mittels ihres Namens aufzulösen.
\lstinline{KB_GET_VALUE} und \lstinline{KB_SET_VALUE} ermöglichen den Zugriff auf die Werte dieser Variablen.


\subsection{Integration des OPC-Servers in das System%
     \label{sec:3-integration}}

open62541 bietet drei Möglichkeiten zum Abgleich von Variablen mit dem Prozessabbild~\citep[vgl.][Tutorials - Connecting a Variable with a Physical Process]{web-open62541}:
\begin{itemize}
    \item Manuelles oder zyklisches Aktualisieren
    \item Variable Value Callback
    \item Variable Datasource
\end{itemize}

Die zyklische Aktualisierung eines oder mehrerer Werte nimmt, abhängig von der Zykluszeit, viele Systemressourcen in Anspruch. Value Callbacks ermöglichen es, einen Variablenwert effizienter mit einer Ressource wie etwa einem Prozessabbild zu synchronisieren. An die Variable wird ein Callback angehängt, welches vor jedem Lesen und nach jedem Schreibvorgang ausgeführt wird.
Der Wert der Variablen wird weiterhin im Variablenknoten auf dem OPC-Server gespeichert, der Abgleich mit der verknüpften Ressource erfolgt durch die Callback-Methoden.

Sogenannte Datenquellen gehen noch einen Schritt weiter. Der Server leitet jede Lese- und Schreibanforderung direkt an eine Callback-Funktion weiter. Beim Lesen liefert der Rückruf eine Kopie des aktuellen Wertes. Die Datenquelle muss intern ein eigenes Speichermanagement implementieren.

Der Zugriff auf die Werte des Prozessabbildes erfolgt, wie in Abschnitt~\ref{sec:3-anbindung} beschrieben, über von piControl bereitgestellte Methoden. Um die durch open62541 gepflegte OPC-Datenstruktur und das durch piControl verwaltete Prozessabbild möglichst effektiv verknüpfen zu können, soll diese Interaktion mittels Datenquellen und den zugehörigen Callbacks implementiert werden.
% % % Imports nur für Referenzenauflösung während des Schreibens! Vorm Kompilieren auskommentieren!
% \bibliography{0_hauptdatei}
% \input{1_einleitung}
% \input{2_grundlagen}
% \input{3_konzeption}
% \input{4_implementierung}
% \input{5_tests}
% \input{6_zusammenfassung}
% \input{anhang}
% % Ende Imports

\section{Implementierung%
  \label{sec:4-implementierung}}
Das folgende Kapitel stellt in Auszügen die Implementierung des OPC-Servers sowie die Anbindung an die IO-Module
der SPS dar. Der Schwerpunkt liegt hierbei auf der Funktionsweise des piControl-Treibers und dessen Integration in das Projekt. Abschnitt~\ref{sec:4-picontrol} erklärt die zum Schreibens eines Bits verwendeten Funktionsaufrufe.
Zuvor soll jedoch in Abschnitt~\ref{sec:4-open62541} der Teil des OPC-Servers vorgestellt werden, welcher auf besagten Treiber zugreift. 

\subsection{Implementierung des OPC-Servers%
     \label{sec:4-open62541}}
Wie im vorangegangenen Abschnitt~\ref{sec:3-integration} begründet, soll die Verknüpfung zwischen dem Prozessabbild der SPS und den auf dem OPC-Server bereitgestellten Werten über sog.\,Datenquellen erfolgen. Hierzu ist zunächst eine Callback-Methode zu implementieren, welche bei einem Lese- oder Schreibzugriff auf eine Variable aufgerufen wird. Die Verknüpfung zwischen Callback-Methode und Variable muss manuell erfolgen.

\begin{lstlisting}[language={c},firstnumber=237,caption={Auszug der Methode \lstinline{linkDataSourceVariable} in \lstinline{variables.c}\label{lst:4-linkDataSourceVariable}}]
extern UA_StatusCode
 linkDataSourceVariable(UA_Server *server, UA_NodeId nodeId) {
     bool readonly = false;
     UA_DataSource dataSourceVariable;
     UA_StatusCode rc; |>\setcounter{lstnumber}{254}<|

     dataSourceVariable.read = readDataSourceVariable;
     if (!readonly)
        dataSourceVariable.write = writeDataSourceVariable;
     else
        dataSourceVariable.write = writeReadonlyDataSourceVariable;

     return UA_Server_setVariableNode_dataSource(server, nodeId, dataSourceVariable);
 }
\end{lstlisting}

\begin{figure}[h]
    \centering
    \includegraphics[width=0.42\textwidth]{doc/img/OPC_RevPiDO.pdf}
    \caption{Auszug des verwendeten Nodesets, hier Digitalausgang 1 des Versuchsaufbaus
      \label{fig:opc-do}}
\end{figure}

Die in Listing~\ref{lst:4-linkDataSourceVariable} abgebildete Methode \lstinline{linkDataSourceVariable()} erzeugt ein Struct vom Typ \lstinline{UA_DataSource}. In diesem werden dem Lesen und Schreiben einer OPC-Variablen entsprechende Callback-Methoden zugewiesen. Die Verknüpfung einer OPC-Variable, genauer ihrer NodeId, mit der zuvor definierten Datenquelle erfolgt über die von open62541 bereitgestellte Methode \lstinline{UA_Server_setVariableNode_dataSource()}. Vor dem Lesen und nach dem Schreiben dieser Variable werden von nun an die entsprechenden Callbacks aufgerufen.
     
\begin{lstlisting}[language={c},firstnumber=168,caption={Auszug des Callbacks \lstinline{writeDataSourceVariable} in \lstinline{variables.c}\label{lst:4-writeDataSourceVariable}}]  
extern UA_StatusCode
 writeDataSourceVariable(UA_Server *server,
            const UA_NodeId *sessionId, void *sessionContext,
            const UA_NodeId *nodeId, void *nodeContext,
            const UA_NumericRange *range, const UA_DataValue *dataValue) {

    UA_StatusCode retval  = UA_STATUSCODE_GOOD;
    UA_NodeId *nameNodeId = UA_malloc(sizeof(UA_NodeId));
    UA_QualifiedName nameQN = UA_QUALIFIEDNAME(1, "Name");
    UA_Variant nameVar;
    UA_Boolean bit;

    retval |= findSiblingByBrowsename(server, nodeId, &nameQN, nameNodeId);
    retval |= UA_Server_readValue(server, *nameNodeId, &nameVar);
    retval |= UA_Boolean_copy(dataValue->value.data, &bit);

    |>\tikzmarkin[set border color=martinired]{writeIO}<|PI_writeSingleIO(String_fromUA_String(nameVar.data), &bit, false);                                                 |>\tikzmarkend{writeIO}<|

    free(nameNodeId);
    return retval;
 }
\end{lstlisting}

Listing~\ref{lst:4-writeDataSourceVariable} zeigt die Callback-Methode, welche nach dem Schreiben einer Variablen auf dem OPC-Server aufgerufen wird.
Dieser Methode wird neben der NodeId der mit ihr verknüpften Variablen auch der Wert dieser in Form eines Zeigers auf ein Struct vom Typ \lstinline{UA_DataValue} übergeben.

Die Gestaltung des hier verwendeten Nodesets sieht vor, dass in einer OPC-Variablen \lstinline{"Name"} der Bezeichner des zu schreibenden Digitalausgangs hinterlegt ist, siehe Abbildung~\ref{fig:opc-do}. Dies erlaubt eine Rekonfiguration der Ein- und Ausgänge der SPS ohne Änderungen im Programmcode des OPC-Servers vornehmen zu müssen.
Es ist daher erforderlich, nach jedem Schreiben einer mit einem Digitalausgang verknüpften Variablen, hier \lstinline{"Value"}, dessen Bezeichner \lstinline{"Name"} abzufragen. 
Dies geschieht in den Zeilen 180 und 181.
Anschließend wird dieser Bezeichner sowie der zu schreibende Wert der Methode \lstinline{PI_writeSingleIO()} übergeben, welche wiederum die Interaktion mit piControl übernimmt (vgl. Abschnitt \ref{sec:4-picontrol}).
 
\subsection{Integration von piControl%
     \label{sec:4-picontrol}}
In Abschnitt~\ref{sec:2-io} wurde die Anbindung der IO-Module des Revolution Pi sowie die Funktionsweise von piControl aus Anwendersicht beschrieben. Die verfügbare Literatur beschränkt sich auch auf lediglich diese Sicht; eine weiterführende Dokumentation für Entwickler gibt es, neben der in Abschnitt~\ref{sec:3-anbindung} vorgestellten Manpage, nicht. 
In diesem Abschnitt soll daher der Quellcode von piControl sowie dessen Verwendung im Projekt genauer betrachtet werden.
Hierzu wird exemplarisch die in Abschnitt~\ref{sec:4-open62541} eingeführte Methode \lstinline{PI_writeSingleIO()} untersucht.
Diese Methode ermöglicht das Setzen eines einzelnen Bits im Prozessabbild der SPS, und damit das Schalten eines digitalen Ausgangs auf einem IO-Modul.
Die äquivalente Methode \lstinline{int piControlGetBitValue(SPIValue *pSpiValue)} zum Lesen eines Bits bzw. Eingangs funktioniert analog und soll daher an dieser Stelle nicht dediziert erörtert werden.

\begin{lstlisting}[language={c},firstnumber=97,
                   caption={Setzen eines phsikalischen, digitalen Ausgangs in \lstinline{revpi.c}
                   \label{lst:4-PI_writeSingleIO}}]
extern void PI_writeSingleIO(char *pszVariableName, bool *bit, bool verbose)
{
	int rc;
	SPIVariable sPiVariable;
	SPIValue sPIValue;

	strncpy(sPiVariable.strVarName, pszVariableName, sizeof(sPiVariable.strVarName));
	rc = piControlGetVariableInfo(&sPiVariable);
	if (rc < 0) {
		printf("Cannot find variable '%s'\n", pszVariableName);
		return;
	}

		sPIValue.i16uAddress = sPiVariable.i16uAddress;
		sPIValue.i8uBit = sPiVariable.i8uBit;
		sPIValue.i8uValue = *bit;
		rc = |>\tikzmarkin[set border color=martinired]{setBitValue}<|piControlSetBitValue(&sPIValue)|>\tikzmarkend{setBitValue}<|;
		if (rc < 0)
			printf("Set bit error %s\n", getWriteError(rc));
		else if (verbose)
			printf("Set bit %d on byte at offset %d. Value %d\n", sPIValue.i8uBit, sPIValue.i16uAddress,
			       sPIValue.i8uValue);
}
\end{lstlisting}

Der Programmcode in Listing~\ref{lst:4-PI_writeSingleIO} ist Teil des implementierten OPC-Servers. In diesem wird auf zwei Funktionen des piControl-Treibers zugegriffen. 
Beiden Methoden wird als Argument ein Zeiger auf ein Struct vom Typ \lstinline{SPIValue} übergeben. Der im Struct abgelegte Name wird mittels \lstinline{piControlGetVariableInfo(&sPIValue)} zu einer Adresse im Prozessabbild aufgelöst. Diese wird in \lstinline{sPIValue.i16uAdress} gespeichert. Der Wert der Variablen wird anschließend mittels \lstinline{piControlSetBitValue(&sPIValue)} an dieser Adresse in das Prozessabbild geschrieben.

\begin{lstlisting}[language={c},firstnumber=309,caption={Methode \lstinline{piControlSetBitValue} in \lstinline{piControlIf.c}\label{lst:4-piControlSetBitValue}}]
int |>\tikzmarkin[set border color=martiniblue]{setBitValueFcn}<|piControlSetBitValue(SPIValue *pSpiValue)|>\tikzmarkend{setBitValueFcn}<|
{
    piControlOpen();

    if (PiControlHandle_g < 0)
	    return -ENODEV;

    pSpiValue->i16uAddress += pSpiValue->i8uBit / 8;
    pSpiValue->i8uBit %= 8;

    if (|>\tikzmarkin[set border color=martinired]{ioctl}<|ioctl(PiControlHandle_g, KB_SET_VALUE, pSpiValue)|>\tikzmarkend{ioctl}<| < 0)
	    return errno;

    return 0;
}
\end{lstlisting}

Die in Listing~\ref{lst:4-piControlSetBitValue} dargestellte Methode \lstinline{piControlSetBitValue} ist lediglich eine Hüllfunktion (häufig auch als Wrapper-Funktion bezeichnet) für einen Aufruf des \lstinline{ioctl} Kernel-Moduls.
Folgende Parameter werden übergeben:
\lstinline{PiControlHandle_g} ist die Referenz auf die Geräte-Datei des piControl-Treibers. \lstinline{KB_SET_VALUE} ist das ioctl-Kommando zum Schreiben eines Bits in das Prozessabbild. Der Zeiger \lstinline{pSpiValue} verweist auf ein Struct des bereits vorgestellten Typs \lstinline{SPIValue}.

\begin{lstlisting}[language={c},firstnumber=80,caption={Methode \lstinline{piControlOpen} in \lstinline{piControlIf.c}\label{lst:4-piControlOpen}}]
void piControlOpen(void)
{
    /* open handle if needed */
    if (PiControlHandle_g < 0)
    {
	    |>\tikzmarkin[set border color=martiniblue]{PiControlHandle}<|PiControlHandle_g = open(PICONTROL_DEVICE, O_RDWR)|>\tikzmarkend{PiControlHandle}<|;
    }
}
\end{lstlisting}

Die in Listing~\ref{lst:4-piControlOpen} dargestellte Methode öffnet, sofern nicht bereits geschehen, die Geräte-Datei. Das Macro \lstinline{PICONTROL_DEVICE} verweist hierbei auf \lstinline{/dev/piControl0}.

\begin{lstlisting}[language={c},firstnumber=721,caption={Methode \lstinline{piControlIoctl} in \lstinline{piControlMain.c}\label{lst:4-piControlIoctl}}]
static long |>\tikzmarkin[set border color=martiniblue, below offset=0.9em]{piControlIoctl}<|piControlIoctl(struct file *file, unsigned int prg_nr, 
                           unsigned long usr_addr)                                      |>\tikzmarkend{piControlIoctl}<|
{
  int status = -EFAULT;
  tpiControlInst *priv;
  int timeout = 10000;	// ms

  if (prg_nr != KB_CONFIG_SEND && prg_nr != KB_CONFIG_START && !isRunning()) {
  	return -EAGAIN;
  }

  priv = (tpiControlInst *) file->private_data;

  if (prg_nr != KB_GET_LAST_MESSAGE) {
  	// clear old message
  	priv->pcErrorMessage[0] = 0;
  }

  switch (prg_nr) {|>\setcounter{lstnumber}{864}<|

    case |>\tikzmarkin[set border color=martiniblue]{KB_SET_VALUE}<|KB_SET_VALUE:|>\tikzmarkend{KB_SET_VALUE}<|
  		{
  			SPIValue *pValue = (SPIValue *) usr_addr;

  			if (!isRunning())
  				return -EFAULT;

  			if (pValue->i16uAddress >= KB_PI_LEN) {
  				status = -EFAULT;
  			} else {
  				INT8U i8uValue_l;
  				my_rt_mutex_lock(&piDev_g.lockPI);
  				i8uValue_l = piDev_g.ai8uPI[pValue->i16uAddress];

  				if (pValue->i8uBit >= 8) {
  					i8uValue_l = pValue->i8uValue;
  				} else {
  					if (pValue->i8uValue)
  						i8uValue_l |= (1 << pValue->i8uBit);
  					else
  						i8uValue_l &= ~(1 << pValue->i8uBit);
  				}

  				|>\tikzmarkin[set border color=martinired]{i8uValue}<|piDev_g.ai8uPI[pValue->i16uAddress] = i8uValue_l;|>\tikzmarkend{i8uValue}<|
  				rt_mutex_unlock(&piDev_g.lockPI);

  #ifdef VERBOSE
  				pr_info("piControlIoctl Addr=%u, bit=%u: %02x %02x\n", pValue->i16uAddress, pValue->i8uBit, pValue->i8uValue, i8uValue_l);
  #endif

  				status = 0;
  			}
  		}
  		break; |>\setcounter{lstnumber}{1314}<|

    default:
      pr_err("Invalid Ioctl");
      return (-EINVAL);
      break;

    }

    return status;
  }
\end{lstlisting}

Listing~\ref{lst:4-piControlIoctl} zeigt in Auszügen die ioctl-Methode des piControl Kernel-Treibers. Diese bekommt folgende Argumente übergeben: \lstinline{struct file *file} enthält den Verweis auf die Geräte-Datei, hier \lstinline{/dev/piControl0}. Der Wert von \lstinline{unsigned int prg_nr} beschreibt die Anfrage an den Treiber, in diesem Fall \lstinline{KB_SET_VALUE}. Das Argument \lstinline{unsigned long usr_addr} enthält einen typ-agnostischen Pointer. Dieser verweist auf einen Speicherbereich, in welchem die zur Bearbeitung der Anfrage notwendigen Daten abgelegt sind. Hier können auch vom Treiber empfangene Daten dem Anwendungsprogramm bereitgestellt werden. 

Die switch-case-Anweisung führt die über das Argument \lstinline{prg_nr} spezifizierte Aktion aus. Hier betrachten wir \lstinline{KB_SET_VALUE}:
Zunächst wird in Zeile 868 der übergebene Zeiger \lstinline{usr_addr} mittels explizitem Typecast zu einem Zeiger des Typs \lstinline{SPIValue *} konvertiert. Da dieser auf Daten im Userspace verweist, ist beim Zugriff durch den Kernel-Treiber besondere Vorsicht geboten.
In Zeile 877 wird mittels Mutex das Prozessabbild \lstinline{piDev_g} für den Zugriff durch andere Threads oder Prozesse gesperrt.
\lstinline{my_rt_mutex_lock} verweist hierbei auf die Funktion \lstinline{rt_mutex_lock} aus \lstinline{linux/sched.h}\footnote{Offenbar wurde hier auch eine alternative Implementierung vorgesehen, siehe revpi\_common.h}

In Zeile 889 wird das Byte \lstinline{i8uValue_l}, welches den zu schreibenden Wert enthält in das Prozessabbild übertragen. Anschließend wird die Mutex auf \lstinline{piDev_g} wieder entsperrt.
\newpage

\begin{lstlisting}[language={c},firstnumber=62,caption={Auszug des Struct \lstinline{spiControlDev} in \lstinline{piControlMain.h}\label{lst:4-spiControlDev}}]
|>\tikzmarkin[set border color=martiniblue]{spiControlDev}<|typedef struct spiControlDev|>\tikzmarkend{spiControlDev}<| {
	// device driver stuff
	int init_step;
	enum revpi_machine machine_type;
	void *machine;
	struct cdev cdev;	// Char device structure
	struct device *dev;
	struct thermal_zone_device *thermal_zone;

	|>\tikzmarkin[set border color=martiniblue]{processImage}<|// process image stuff
	INT8U ai8uPI[KB_PI_LEN];
	INT8U ai8uPIDefault|>\tikzmarkin[set border color=martinired]{KB_PI_LEN_0}<|[KB_PI_LEN]|>\tikzmarkend{KB_PI_LEN_0}<|;
	struct rt_mutex lockPI;        |>\tikzmarkend{processImage}<|
	bool stopIO;
	piDevices *devs; |>\setcounter{lstnumber}{94}<|
} tpiControlDev;
\end{lstlisting}

Das Prozessabbild ist als Byte-Array der Länge \lstinline{KB_PI_LEN} in Listing~\ref{lst:4-spiControlDev} definiert. Konfigurationsparameter wie \lstinline{KB_PI_LEN} oder die Zykluszeit für den Datenaustausch zwischen SPS und IO-Modulen sind im folgenden Listing~\ref{lst:4-process} definiert.

\begin{lstlisting}[language={c},firstnumber=119,caption={Konfigurationsparameter des Prozessabbildes in project.h\label{lst:4-process}}]
#define INTERVAL_PI_GATE (5*1000*1000)  // 5 ms piGateCommunication |>\setcounter{lstnumber}{128}<|

#define INTERVAL_IO_COM (5*1000*1000)  // 5 ms piIoComm |>\setcounter{lstnumber}{132}<|

#define KB_PD_LEN       512
|>\tikzmarkin[set border color=martiniblue]{KB_PI_LEN_1}<|#define KB_PI_LEN       4096|>\tikzmarkend{KB_PI_LEN_1}<|
\end{lstlisting}

Das zu setzende Bit wurde zu diesem Zeitpunkt erfolgreich in das Prozessabbild der SPS geschrieben.
Es stellt sich die Frage, wie dieses nun an das IO-Modul kommuniziert wird.
Die Kommunikation mit allen angebundenen Modulen ist ebenfalls Aufgabe des piControl-Treibers.

\begin{lstlisting}[language={c},firstnumber=256,caption={Auszug der Methode \lstinline{piIoThread} in \lstinline{revpi_core.c}\label{lst:4-piIoThread}}]
static int piIoThread(void *data)
{
	//TODO int value = 0;
	ktime_t time;
	ktime_t now;
	s64 tDiff;

	hrtimer_init(&piCore_g.ioTimer, CLOCK_MONOTONIC, HRTIMER_MODE_ABS);
	piCore_g.ioTimer.function = piIoTimer;

	pr_info("piIO thread started\n");

	now = hrtimer_cb_get_time(&piCore_g.ioTimer);

	PiBridgeMaster_Reset();

	while (!kthread_should_stop()) {
		if (|>\tikzmarkin[set border color=martinired]{PiBridgeMaster}<|PiBridgeMaster_Run()|>\tikzmarkend{PiBridgeMaster}<| < 0)
			break;
	}

	RevPiDevice_finish();

	pr_info("piIO exit\n");
	return 0;
}
\end{lstlisting}

Der Kernel-Thread \lstinline{piIoThread} ist verantwortlich für den zyklischen Datenaustausch mit den IO-Modulen. In diesem wird fortlaufend die Methode \lstinline{PiBridgeMaster_Run()} aufgerufen, siehe Listing~\ref{lst:4-piIoThread}.

\begin{lstlisting}[language={c},firstnumber=262,caption={Auszug der Methode \lstinline{PiBridgeMaster_Run(void)} in \lstinline{RevPiDevice.c}\label{lst:4-PiBridgeMaster_Run}}]
int PiBridgeMaster_Run(void)
{
	static kbUT_Timer tTimeoutTimer_s;
	static kbUT_Timer tConfigTimeoutTimer_s;
	static int error_cnt;
	static INT8U last_led;
	static unsigned long last_update;
	int ret = 0;
	int i;

	my_rt_mutex_lock(&piCore_g.lockBridgeState);
	if (piCore_g.eBridgeState != piBridgeStop) {
		switch (eRunStatus_s) { |>\setcounter{lstnumber}{514}<|
		    case enPiBridgeMasterStatus_EndOfConfig:|>\setcounter{lstnumber}{621}<|
		    if (|>\tikzmarkin[set border color=martinired]{RevPiDevice}<|RevPiDevice_run()|>\tikzmarkend{RevPiDevice}<|) {
				// an error occured, check error limits |>\setcounter{lstnumber}{641}<|
			} else {
				ret = 1;
			}
			piCore_g.image.drv.i16uRS485ErrorCnt = RevPiDevice_getErrCnt();
			break;
\end{lstlisting}

Die in Listing~\ref{lst:4-PiBridgeMaster_Run} dargestellte Methode ist eine sog. State-Machine. Ist die Konfiguration der IO-Module erfolgreich abgeschlossen, so führt sie bei Aufruf lediglich die Methode \lstinline{RevPiDevice_run()} aus.

\begin{lstlisting}[language={c},firstnumber=140,caption={Auszug der Methode \lstinline{RevPiDevice_run(void)} in \lstinline{RevPiDevice.c}\label{lst:4-RevPiDevice_run}}]
int RevPiDevice_run(void)
{
	INT8U i8uDevice = 0;
	INT32U r;
	int retval = 0;

	RevPiDevices_s.i16uErrorCnt = 0;

	for (i8uDevice = 0; i8uDevice < RevPiDevice_getDevCnt(); i8uDevice++) {
		if (RevPiDevice_getDev(i8uDevice)->i8uActive) {
			switch (RevPiDevice_getDev(i8uDevice)->sId.i16uModulType) {
			case KUNBUS_FW_DESCR_TYP_PI_DIO_14:
			case KUNBUS_FW_DESCR_TYP_PI_DI_16:
			case KUNBUS_FW_DESCR_TYP_PI_DO_16:
				r = |>\tikzmarkin[set border color=martinired]{sendCyclicTelegram}<|piDIOComm_sendCyclicTelegram(i8uDevice)|>\tikzmarkend{sendCyclicTelegram}\setcounter{lstnumber}{166} <|;

				break; |>\setcounter{lstnumber}{216}<|
			}
		}
	} |>\setcounter{lstnumber}{227}<|
	return retval;
}
\end{lstlisting}

Diese iteriert wie in Listing~\ref{lst:4-RevPiDevice_run} abgebildete durch alle gegenwärtig in der SPS konfigurierten Module. Ist das aktuelle Modul als aktiv markiert, so wird anhand eines sog. Firmware-Descriptors entschieden, welche Methode für die Ansteuerung des Moduls aufzurufen ist.

\begin{lstlisting}[language={c},firstnumber=161,caption={Auszug der Methode \lstinline{piDIOComm_sendCyclicTelegram} in \lstinline{piDIOComm.c}\label{lst:4-sendCyclicTelegram}}]
INT32U piDIOComm_sendCyclicTelegram(INT8U i8uDevice_p)
{
	INT32U i32uRv_l = 0;
	SIOGeneric sRequest_l;
	SIOGeneric sResponse_l;
	INT8U len_l, data_out[18], i, p, data_in[70];
	INT8U i8uAddress;
	int ret; |>\setcounter{lstnumber}{239}<|
	
    |>\tikzmarkin[set border color=martinired]{piIoComm}<|ret = piIoComm_send((INT8U *) & sRequest_l, IOPROTOCOL_HEADER_LENGTH + len_l + 1);  |>\tikzmarkend{piIoComm}\setcounter{lstnumber}{298}<|
}
\end{lstlisting}

Im Falle des hier verwendeten DO-Moduls wird die in Listing~\ref{lst:4-sendCyclicTelegram} abgebildete Methode \lstinline{piDIOComm_sendCyclicTelegram()} aufgerufen. Dieser wird ein Zeiger auf das zu schreibende Gerät übergeben. 
Zunächst wird das Prozessabbild mittels eines proprietären, jedoch im Quellcode offen nachvollziehbaren Protokolls in ein \lstinline{sRequest_l} genanntes Byte-Array umgewandelt. Dieser Schritt ist in Listing~\ref{lst:4-sendCyclicTelegram} nicht abgebildet. Anschließend wird \lstinline{piIoComm_send()} ein Zeiger auf die so generierte Schreib-Anfrage übergeben.

\begin{lstlisting}[language={c},firstnumber=220,caption={Auszug der Methode \lstinline{piIOComm_send} in \lstinline{piIOComm.c}\label{lst:4-piIOComm_send}}]
int piIoComm_send(INT8U * buf_p, INT16U i16uLen_p)
{
	ssize_t write_l = 0;
	INT16U i16uSent_l = 0;|>\setcounter{lstnumber}{249}<|

	while (i16uSent_l < i16uLen_p) {
		write_l = vfs_write(piIoComm_fd_m, buf_p + i16uSent_l, i16uLen_p - i16uSent_l, &piIoComm_fd_m->f_pos);
		if (write_l < 0) {
			pr_info_serial("write error %d\n", (int)write_l);
			return -1;
		} 
		i16uSent_l += write_l;|>\setcounter{lstnumber}{263}<|
	}
	clear();
	vfs_fsync(piIoComm_fd_m, 1);
	return 0;
}
\end{lstlisting}

Listing~\ref{lst:4-piIOComm_send} zeigt die Implementierung von \lstinline{piIoComm_send()}. Diese Methode ist für das Schreiben der oben generierten Anfrage auf die seriellen Schnittstelle verantwortlich. Realisiert wird dies mittels der Methode \lstinline{vfs_write()}. Diese ist in \lstinline{<linux/fs.h>} definiert. Sie ermöglicht das Schreiben einer Datei im Userspace aus dem Kernel heraus. Geschrieben wird hier die Datei mit dem Deskriptor \lstinline{piIoComm_fd_m}.
Da die Funktion \lstinline{vfs_write()} durch andere Kernel-Tasks unterbrochen werden kann, ist nicht gewährleistet, dass die gesamte Anfrage mit nur einem Aufruf geschrieben wird. Die oben abgebildete while-Schleife stellt das vollständige Senden der Anfrage sicher.

\begin{lstlisting}[language={c},firstnumber=157,caption={Auszug der Methode \lstinline{piIOComm_open_serial} in \lstinline{piIOComm.c}\label{lst:4-piIOComm_open_serial}}]
int piIoComm_open_serial(void)
{   |>\setcounter{lstnumber}{167}<|
	struct file *fd;	/* Filedeskriptor */
	struct termios newtio;	/* Schnittstellenoptionen */

	|>\tikzmarkin[set border color=martiniblue]{fd}<|/* Port oeffnen - read/write, kein "controlling tty", 
	    Status von DCD ignorieren */
	fd = filp_open(|>\tikzmarkin[set border color=martinired]{tty}<|REV_PI_TTY_DEVICE|>\tikzmarkend{tty}<|, O_RDWR | O_NOCTTY, 0); |>\setcounter{lstnumber}{208}<|
	
	piIoComm_fd_m = fd;                                                      |>\tikzmarkend{fd}\setcounter{lstnumber}{217}<|

	return 0;
}
\end{lstlisting}

Der zum Schreiben auf die serielle Schnittstelle verwendete Datei-Deskriptor wird von der in Listing~\ref{lst:4-piIOComm_open_serial} abgebildeten Methode \lstinline{piIoComm_open_serial()} generiert. 

\begin{lstlisting}[language={c},firstnumber=45,caption={Definition der seriellen Schnittstelle in \lstinline{piIOComm.h}\label{lst:4-REV_PI_TTY_DEVICE}}]
#define REV_PI_TTY_DEVICE	"/dev/ttyAMA0"
\end{lstlisting}

Das in Listing~\ref{lst:4-REV_PI_TTY_DEVICE} definierte Macro verweist auf eine der seriellen Schnittstellen des RaspberryPi.
Die Implementierung des zugehörigen Schnittstellentreibers soll hier nicht weiter untersucht werden. Somit ist an dieser Stelle die Kette vom Setzen einer Variablen auf dem OPC-Server bis hin zur Aktualisierung des Prozessabbilds der IO-Module geschlossen.

% \begin{lstlisting}[language={c},firstnumber={226},caption={Setzen der Scheduler-Priorität auf SCHED\_FIFO in 
% revpi\_common.c\label{lst:2-sched_priority}}]
% param.sched_priority = ktprio->prio;
% ret = sched_setscheduler(child, SCHED_FIFO, &param);
% \end{lstlisting}
% % % Imports nur für Referenzenauflösung während des Schreibens! Vorm Kompilieren auskommentieren!
% \bibliography{0_hauptdatei}
% \input{1_einleitung}
% \input{2_grundlagen}
% \input{3_konzeption}
% \input{4_implementierung}
% \input{5_tests}
% \input{6_zusammenfassung}
% % Ende Imports

\section{Test des OPC-Servers im Gesamtsystem%
  \label{sec:5-tests}}

% % % Imports nur für Referenzenauflösung während des schreibens! Vorm Kompilieren auskommentieren!
% \bibliography{0_hauptdatei}
% \input{1_einleitung}
% \input{2_grundlagen}
% \input{3_konzeption}
% \input{4_implementierung}
% \input{5_tests}
% \input{6_zusammenfassung}
% % Ende Imports

\section{Zusammenfassung und Ausblick%
  \label{sec:6-fazit}}
Der folgende Abschnitt~\ref{sec:6-zusammenfassung} fasst die gewonnenen Erkenntnisse und den Stand der Implementierung zusammen.
Den Abschluss dieser Arbeit bildet der Ausblick in Abschnitt~\ref{sec:6-ausblick}.

\subsection{Zusammenfassung%
     \label{sec:6-zusammenfassung}}

\subsection{Ausblick%
     \label{sec:6-ausblick}}

% \input{anhang}
% % Ende Imports

\section{Systemkonzept%
  \label{sec:3-konzeption}}
Auf Basis der in Abschnitt \ref{sec:2-grundlagen} vorgestellten Möglichkeiten folgt nun die Ausarbeitung eines Konzepts.
In den folgenden Abschnitten soll näher auf zwei zentrale Aspekte eingegangen werden: Abschnitt~\ref{sec:3-anbindung} stellt Möglichkeiten zum Zugriff auf Variablen bzw.\,Werte im Prozessabbild des Revolution Pi vor; in Abschnitt~\ref{sec:3-integration} wird ein Konzept zur Bereitstellung dieser Variablen auf einem OPC-Server vorgestellt.

\subsection{Anbindung der IO an den OPC-Server%
     \label{sec:3-anbindung}}

Eine Webanwendung mit Bezeichnung PiCtory dient zur Konfiguration der I/O- und virtuellen Module des RevolutionPi. Die Konfiguration liegt im JSON-Format in der Datei \lstinline{/etc/revpi/config.rsc}. Der piControl-Treiber liest diese Datei beim Start. 
Der folgende Auszug aus der Manpage des piControl-Kernelmoduls beschreibt die von diesem zum Lesen und Schreiben einzelner Bits des Prozessabbildes bereitgestellten Funktionen~\citep[vgl.]{web-revpi-manpage}. Sie ist an dieser Stelle weitgehend ungekürzt zitiert, da sie die nutzbare Schnittstelle sehr kompakt beschreibt.

\begin{lstlisting}[breakindent=0pt, numbers=none, caption={Auszug aus der Revolution Pi Programmers Manual\label{lst:4-manpage}}]
KB_FIND_VARIABLE SPIVariable *argp
Find a variable in the process image by its name. A pointer to a structure of type SPIVariable must be passed as argument. [...]
The struct SPIVariable [...] is defined as 
typedef struct SPIVariableStr
{
    char strVarName[32]; // Variable name
    uint16_t i16uAddress; // Address of the byte in the process image
    uint8_t i8uBit; // 0-7 bit position, >= 8 whole byte
    uint16_t i16uLength; // length of the variable in bits.
    // Possible values are 1, 8, 16 and 32
} SPIVariable;

Set and get values of the process image
KB_GET_VALUE SPIValue *argp
[...]
KB_SET_VALUE SPIValue *argp
Write one bit or one byte to the process image [...].  This call is more efficient than the usual calls of seek and write because only one function call is necessary. If more than on application are writing bits in one output byte, this call is the only safe way to set a bit without overwriting the other bits because this call is doing a read-modify-write-cycle. 

The struct SPIValue used by this ioctl is defined as
typedef struct SPIValueStr
{
    uint16_t i16uAddress; // Address of the byte in the process image
    uint8_t i8uBit; // 0-7 bit position, >= 8 whole byte
    uint8_t i8uValue; // Value: 0/1 for bit access, whole byte otherwise
} SPIValue;
\end{lstlisting} 

Die oben beschriebenden Funtkionen \lstinline{KB_FIND_VARIABLE}, \lstinline{KB_GET_VALUE} und \lstinline{KB_SET_VALUE} ermöglichen einen einfachen und (lt.\,Manpage) effizienten Zugriff auf einzelne Bits des Prozessabbildes und damit auch auf die IO des RevolutionPi.
Der Zugriff des OPC-Servers auf das Prozessabbild soll daher mittels dieser Funktionen realisiert werden.
\lstinline{KB_FIND_VARIABLE} kann genutzt werden, um Adressen von Variablen im Prozessabbild mittels ihres Namens aufzulösen.
\lstinline{KB_GET_VALUE} und \lstinline{KB_SET_VALUE} ermöglichen den Zugriff auf die Werte dieser Variablen.


\subsection{Integration des OPC-Servers in das System%
     \label{sec:3-integration}}

open62541 bietet drei Möglichkeiten zum Abgleich von Variablen mit dem Prozessabbild~\citep[vgl.][Tutorials - Connecting a Variable with a Physical Process]{web-open62541}:
\begin{itemize}
    \item Manuelles oder zyklisches Aktualisieren
    \item Variable Value Callback
    \item Variable Datasource
\end{itemize}

Die zyklische Aktualisierung eines oder mehrerer Werte nimmt, abhängig von der Zykluszeit, viele Systemressourcen in Anspruch. Value Callbacks ermöglichen es, einen Variablenwert effizienter mit einer Ressource wie etwa einem Prozessabbild zu synchronisieren. An die Variable wird ein Callback angehängt, welches vor jedem Lesen und nach jedem Schreibvorgang ausgeführt wird.
Der Wert der Variablen wird weiterhin im Variablenknoten auf dem OPC-Server gespeichert, der Abgleich mit der verknüpften Ressource erfolgt durch die Callback-Methoden.

Sogenannte Datenquellen gehen noch einen Schritt weiter. Der Server leitet jede Lese- und Schreibanforderung direkt an eine Callback-Funktion weiter. Beim Lesen liefert der Rückruf eine Kopie des aktuellen Wertes. Die Datenquelle muss intern ein eigenes Speichermanagement implementieren.

Der Zugriff auf die Werte des Prozessabbildes erfolgt, wie in Abschnitt~\ref{sec:3-anbindung} beschrieben, über von piControl bereitgestellte Methoden. Um die durch open62541 gepflegte OPC-Datenstruktur und das durch piControl verwaltete Prozessabbild möglichst effektiv verknüpfen zu können, soll diese Interaktion mittels Datenquellen und den zugehörigen Callbacks implementiert werden.
% % % Imports nur für Referenzenauflösung während des Schreibens! Vorm Kompilieren auskommentieren!
% \bibliography{0_hauptdatei}
% % Mit \section{...} eröffnen wir einen neuen Abschnitt.
% Der Befehl setzt nicht nur den Text in einer größeren,
% fetten Schrift, sondern sorgt außerdem dafür, daß er im
% Inhaltsverzeichnis erscheint.
%
% Mit \label{...} erzeugen wir einen Bezeichner, mit dessen Hilfe
% wir später auf die Nummer des Abschnitts verweisen können (nämlich
% mit~\ref{...}).
%
% Das Kommentarzeichen hinter „Übersicht“ dient dazu, ein
% Leerzeichen zwischen „Übersicht“ und dem \label-Befehl
% zu vermeiden, das andernfalls sichtbar würde – z.B. im
% Inhaltsverzeichnis.
%

% % Imports nur für Referenzenauflösung während des Schreibens! Vorm Kompilieren auskommentieren!
% \bibliography{0_hauptdatei}
% \input{1_einleitung}
%\input{2_grundlagen}
%\input{3_konzeption}
%\input{4_implementierung}
%\input{5_tests}
%\input{6_zusammenfassung}
% % Ende Imports

\section{Einleitung und Motivation%
  \label{sec:1-einleitung}}
Ziel dieses Projektes ist die Integration eines OPC-Servers mit einer auf Linux
basierenden speicherprogrammierbaren Steuerung (SPS). Angeschlossen an diese SPS
ist jeweils ein digitales Ein-/\,bzw.~Ausgabemodul. Die von diesen bereitgestellten
Ein-/\, bzw.~Ausgänge (IO) sollen in der Datenstruktur des OPC-Servers abgebildet
und über diesen für OPC-Clients les-/\,und schreibar sein. Weiterhin sollen einige
Funktionen zur Überwachung und Steuerung der an die SPS angeschlossenen Aktoren
und Sensoren direkt im OPC-Server implementiert werden.
Hiermit stellt dieses Projekt eine der Grundlagen für ein übergeordnetes Projekt,
die cloudbasierte Steuerung eines miniaturisierten Produktions-Systems, dar.

Der hier verwendete OPC-Server ist Teil des sog. open62541 Projekts. Er ist in C
geschrieben und implementiert bereits einen großen Teil der im OPC-UA-Standard
spezifizierten Funktionen.
Als SPS findet ein Revolution Pi 3 der Firma Kunbus Verwendung. Dieser integriert
ein sog. Compute Module der Raspberry Pi Foundation in ein industrietaugliches
Gehäuse und erlaubt die Erweiterung mittels IO- oder Gateway-Modulen. Über diese
erfolgt die Kommunikation mit weiteren Komponenten der Automatisierungstechnik.

Motiviert ist dieses Projekt durch die Beobachtung, dass die Verbreitung offener
Standards sowie freier Software auch in der Automatisierungstechnik zunimmt.
Linux ist ein freies Betriebssystem, OPC-UA ein offen zugänglicher, aktiv gepflegter
und weit verbreiteter Standard. Der Raspberry Pi findet sowohl bei Hobby-Anwendern als
auch in den Bereichen Forschung und Entwicklung sowie bei industriellen Anwendern
Verwendung. Dieses Projekt stellt somit eine für unterschiedliche Anwender interessante
Entwicklung dar.

Im Anschluss an diese einleitende Übersicht im Abschnitt~\ref{sec:1-einleitung} folgt
die Darstellung der wichtigsten Grundlagen in Abschnitt~\ref{sec:2-grundlagen}.
Aufbauend auf diesen Grundlagen folgt die konzeptuelle Ausarbeitung im Abschnitt~\ref{sec:3-konzeption}.
Die Umsetzung wird im Abschnitt~\ref{sec:4-implementierung} erläutert.
Die Leistungsfähigkeit der Implementierung wird in Abschnitt~\ref{sec:5-tests} untersucht.
Eine Zusammenfassung und ein Ausblick schließen die Arbeit in
Abschnitt~\ref{sec:6-fazit} ab. Eventuell noch benötigte Anhänge
finden sich in den Anhängen [...] bis [...].

% % % Imports nur für Referenzenauflösung während des Schreibens! Vorm Kompilieren auskommentieren!
% \bibliography{0_hauptdatei}
% \input{1_einleitung}
% \input{2_grundlagen}
% \input{3_konzeption}
% \input{4_implementierung}
% \input{5_tests}
% \input{6_zusammenfassung}
% % Ende Imports

\section{Grundlagen%
  \label{sec:2-grundlagen}}

\subsection{Speicherprogrammierbare-Steuerung und Linux -- Revolution Pi%
     \label{sec:2-sps}}

\subsubsection{Kunbus RevolutionPi%
        \label{sec:2-revpi}}
Der RevolutionPi 3 ist eine speicherprogrammierbare Steuerung (SPS) des Herstellers
Kunbus GmbH. Kern dieser SPS ist das von der Raspberry Pi Foundation entwickelte
und vertriebene Raspberry Pi Compute Module 3. Dieses integriert ein Broadcom BCM2837
System-on-Chip (SoC) mit vier 1,2GHz Prozessorkernen, 1GB RAM, 4GB eMMC Anwendungsspeicher
und sonstige Peripherie in ein Modul im DDR2-SODIMM Formfaktor. Diese Spezifikationen
sind weitgehend identisch zu denen des ausgesprochen populären Raspberry Pi 3.
Der Revolution Pi profitiert daher von dem gleichen großen Angebot an Software
und Unterstützung wie der Raspberry Pi, ergänzt dessen Hardware jedoch um eine 24V
Spannungsversorgung, die Möglichkeit der Erweiterung durch mehrere industrietaugliche
Ein-/ Ausgabemodule und Gateways sowie ein Gehäuse zur Montage auf einer DIN-Schiene.
\begin{itemize}
  \item{Prozessor: BCM2837}
  \item{Taktfrequenz 1,2 GHz}
  \item{Anzahl Prozessorkerne: 4}
  \item{Arbeitsspeicher: 1 GByte}
  \item{eMMC Flash Speicher: 4 GByte}
  \item{Betriebssystem: Angepasstes Raspbian mit RT-Patch}
  \item{RTC mit 24h Pufferung über wartungsfreien Kondensator}
  \item{Treiber / API: Treiber schreibt zyklisch Prozessdaten in ein Prozessabbild, Zugriff auf Prozessabbild über Linux-Filesystem als API zu Fremdsoftware.}
  \item{Kommunikationsanschlüsse: 2 x USB 2.0 A (je 500 mA belastbar), 1 x Micro-USB, HDMI, Ethernet (RJ45) 10/100 Mbit/s}
  \item{Stromversorgung: min. 10,7 V, max. 28,8 V, maximal 10 Watt}
  \item{Zulässige Umgebungstemperatur: -40 bis +55 C}
  \item{Gehäuseabmessungen: (HxBxL) 96 mm x 22,5 mm x 110,5 mm (ohne gesteckte Stecker)}
  \item{ESD Schutz: 4 kV / 8 kV gemäß EN61131-2 und IEC 61000-6-2}
  \item{Surge / Burst Prüfungen: gemäß EN61131-2 und IEC 61000-6-2 eingekoppelt auf Versorgungsspannung, Ethernet und IO-Leitungen}
  \item{EMI Prüfungen: gemäß EN61131-2 und IEC 61000-6-2}
\end{itemize}

Kunbus bietet eine Auswahl an IO- und Gateway-Modulen zur Erweiterung des Revolution Pi an.
Gateways dienen der Kommunikation mit Systemen oder Komponenten der Automatisierungstechnik
über Protokolle wie PROFIBUS oder EtherCAT. IO-Module erlauben die Überwachung
und Steuerung von digitalen oder analogen Ein- und Ausgängen.

\subsubsection{Zugriff auf IO-Module%
        \label{sec:2-io}}
Der Zugriff auf die Ein- und Ausgänge der IO-Module erfolgt über ein Prozessabbild
und einen hierfür von Kunbus bereitgestellten Treiber, genannt piControl. Dieser
aktualisiert das Prozessabbild zyklisch. Die angestrebte Zykluszeit beträgt 5ms,
kann jedoch je nach Anzahl der angeschlossenen Module auch größer sein. Kunbus
garantiert bei drei IO-Modulen und zwei Gateway-Modulen eine Zykluszeit von 10 ms.
Jedes der IO-Module stellt ein eigenständiges eingebettetes System dar. Es verfügt
über einen Microcontroller, welcher die IOs bereitstellt und über einen RS485-Bus
mit dem Revolution Pi kommuniziert.
% https://revolution.kunbus.de/io-modul/

Lizenz: GPL
% https://github.com/RevolutionPi/piControl

\begin{lstlisting}[language={c},firstnumber={226},caption={Setzen der Scheduler-Priorität auf SCHED\_FIFO in revpi\_common.c\label{lst:2-sched_priority}}]
param.sched_priority = ktprio->prio;
ret = sched_setscheduler(child, SCHED_FIFO,
       &param);
\end{lstlisting}


\subsection{Echtzeit und Multithreading unter Linux -- preemptRT und posix%
     \label{sec:2-echtzeit}}


 Der Linux-Kernel verfügt über mehrere unterschiedliche Preemtion-Modelle:

\begin{itemize}
  \item No Forced Preemption (server):
  Ausgelegt auf maximal möglichen Durchsatz, lediglich Interrupts und
  System-Call-Returns bewirken Präemption.

  \item Voluntary Kernel Preemption (Desktop):
  Neben den implizit bevorrechtigten Interrupts und System-Call-Returns gibt es
  in diesem Modell weitere Abschnitte des Kernels in welchen Preämption explizit
  gestattet ist.

  \item Preemptible Kernel (Low-Latency Desktop):
  In diesem Modell ist der gesamte Kernel, mit Ausnahme sog.~kritischer Abschnitte
  präemptible. Nach jedem kritischen Abschnitt gibt es einen impliziten Präemptions-Punkt.

  \item Preemptible Kernel (Basic RT):
  Dieses Modell ist dem zuvor genannten sehr ähnlich, hier sind jedoch alle Interrupt-Handler
  als eigenständige Threads ausgeführt.

  \item Fully Preemptible Kernel (RT):
  Wie auch bei den beiden zuvor genannten Modellen ist hier der gesamte Kernel
  präemtible, die Anzahl und Dauer der nicht-präemtiblen kritischen Abschnitte
  ist auf ein notwendiges Minimum beschränkt. Alle Interrupt-Handler sind als
  eigenständige Threads ausgeführt, Spinlocks durch Sleeping-Spinlocks und Mutexe
  durch sog.~RT-Mutexe ersetzt.

\end{itemize}
\todo{Spinlocks und Mutexe sowie die RT-Varianten dieser erklären!}

Lediglich mit dem vollständig präemtiblen Kernel kann Echtzeit-Verhalten realisiert werden.

% https://wiki.linuxfoundation.org/realtime/documentation/technical_basics/preemption_models bzw kernel/Kconfig.preempt

\subsubsection{preemptRT%
        \label{sec:2-preemptRT}}
% https://wiki.linuxfoundation.org/realtime/documentation/technical_details/start
% https://wiki.linuxfoundation.org/realtime/documentation/technical_basics/start

Das dem PREEMPT RT Kernel zugrunde liegende Prinzip lässt sich in einer einfachen
Regel ausdrücken: Nur Code, welcher absolut nicht-präemtible sein darf, ist es
gestattet nicht-präemtible zu sein.
Das erklärte Ziel des PREEMPT\_RT Patches ist es folglich, die Menge des nicht-präemtiblen
Codes im Linux-Kernel auf das absolut notwendige Minimum zu reduzieren.

Dies wird durch Verwendung folgender Mechanismen erreicht:

\begin{itemize}
  \item Hochauflösende Timer
  \item Sleeping Spinlocks
  \item Threaded Interrupt Handlers
  \item rt\_mutex
  \item RCU
\end{itemize}


\subsubsection{posix%
        \label{sec:2-posix}}
Ist posix hier wirklich relevant? Debian bzw.~Raspbian sind weitgehend posix
kompatibel, aber wird es hier genutzt? -> JA, open62541 nutzt pthread.h
piControl nutzt kthread.h, und semaphore.h

\subsection{OPC-UA und open62541%
     \label{sec:2-opc}}

\subsubsection{OPC UA%
        \label{sec:2-opcua}}
Open Platform Communications (OPC) ist eine Familie von Standards zur herstellerunabhängigen
Kommunikation von Maschinen (M2M) in der Automatisierungstechnik. Die sog.~OPC Task Force, zu deren
Mitgliedern verschiedene große Firmen der Automatisierungsindustrie gehören, veröffentlichte
die OPC Specification Version 1.0 im August 1996.
Motiviert ist dieser offene Standard durch die Erkenntniss, dass die Anpassung der
zahlreichen Herstellerstandards an individuelle Infrastrukturen und Anlagen einen
großen Mehraufwand verursachen.
Die Wikipedia beschreibt das Anwendungsgebiet für OPC wie folgt:

\glqq{}OPC wird dort eingesetzt, wo Sensoren, Regler und Steuerungen verschiedener Hersteller
ein gemeinsames Netzwerk bilden. Ohne OPC benötigten zwei Geräte zum Datenaustausch
genaue Kenntnis über die Kommunikationsmöglichkeiten des Gegenübers. Erweiterungen
und Austausch gestalten sich entsprechend schwierig. Mit OPC genügt es, für jedes
Gerät genau einmal einen OPC-konformen Treiber zu schreiben. Idealerweise wird
dieser bereits vom Hersteller zur Verfügung gestellt. Ein OPC-Treiber lässt sich
ohne großen Anpassungsaufwand in beliebig große Steuer- und Überwachungssysteme
integrieren.

OPC unterteilt sich in verschiedene Unterstandards, die für den jeweiligen Anwendungsfall
unabhängig voneinander implementiert werden können. OPC lässt sich damit verwenden
für Echtzeitdaten (Überwachung), Datenarchivierung, Alarm-Meldungen und neuerdings
auch direkt zur Steuerung (Befehlsübermittlung).\grqq{}

OPC basiert in der ursprünglichen Spezifikation auf Microsofts DCOM-Spezifikation.
DCOM macht Funktionen und Objekte einer Anwendung anderen Anwendungen im Netzwerk
zugänglich. Der OPC-Standard definiert entsprechende DCOM-Objekte um mit anderen
OPC-Anwendungen Daten austauschen zu können. Die Verwendung von DCOM bindet Anwender
an Betriebssysteme von Microsoft. Die ursprüngliche OPC Spezifikation wird durch die
Entwicklung von OPC Unified Architecture (OPC UA) abgelöst.
OPC UA setzt auf einem eigenen Kommunikationionsstack auf, die Verwendung von DCOM
und damit die Bindung an Microsoft wurden aufgelöst.

Die OPC-UA-Architektur ist eine Service-orientierte Architektur (SOA), deren Struktur
aus mehreren Schichten besteht.

% Wikipedia
Das OPC-Informationsmodell ist nicht mehr nur eine Hierarchie aus Ordnern, Items
und Properties. Es ist ein sogenanntes Full-Mesh-Network aus Nodes, mit dem neben
den Nutzdaten eines Nodes auch Meta- und Diagnoseinformationen repräsentiert werden.
Ein Node ähnelt einem Objekt aus der objektorientierten Programmierung. Ein Node
kann Attribute besitzen, die gelesen werden können (Data Access (DA), Historical
Data Access (HDA)). Es ist möglich Methoden zu definieren und aufzurufen.
Eine Methode besitzt Aufrufargumente und Rückgabewerte. Sie wird durch ein Command
aufgerufen. Weiterhin werden Events unterstützt, die versendet werden können
(AE (Alarms \& Events), DA DataChange), um bestimmte Informationen zwischen Geräten
auszutauschen. Ein Event besitzt unter anderem einen Empfangszeitpunkt, eine Nachricht
und einen Schweregrad. Die o. g. Nodes werden sowohl für die Nutzdaten als auch
alle anderen Arten von Metadaten verwendet. Der damit modellierte OPC-Adressraum
beinhaltet nun auch ein Typmodell, mit dem sämtliche Datentypen spezifiziert werden.

% https://de.wikipedia.org/wiki/Open_Platform_Communications
% https://de.wikipedia.org/wiki/OPC_Unified_Architecture
% https://opcfoundation.org/developer-tools/specifications-unified-architecture
% Von Gerhard Gappmeier - ascolab GmbH, CC BY-SA 3.0, https://de.wikipedia.org/w/index.php?curid=1892069
\subsubsection{open62541%
        \label{sec:2-open62541}}
open62541 ist eine offene und freie Implementierung von OPC UA. Die in C geschriebene
Bibliothek stellt eine beständig zunehmende Anzahl der im OPC UA Standard definierten
Funktionen bereit. Sie kann sowohl zur Erstellung von OPC-Servern als auch -Clients
genutzt werden. Ergänzend zu der unter der Mozilla Public License v2.0 lizensierten
Bibliothek stellt das open62541 Projekt auch Beispielprogramme unter einer CC0 Lizenz
zur Verfügung.

Die Bibliothek eignet sich auch für die Entwicklung auf eingebetteten Systemen und
Microcontrollern. Je nach Umfang der gewünschten Funktionen und des OPC Informationsmodells
beträgt die Größe einer Server-Binary weniger als 100kb. %evtl. kürzen?

\todo{Nodes erklären! Evtl.~oben!}

Folgende Auswahl an Eigenschaften und Funktionen zeichnet die in dieser Arbeit verwendete
Version 0.3 von open62541 aus:
\begin{itemize}
  \item Kommunikationionsstack
  \begin{itemize}
      \item OPC UA Binär-Protokoll (HTTP oder SOAP werden gegenwärtig nicht unterstützt)
      \item Austauschbare Netzwerk-Schicht, welche die Verwendung eigener Netzwerk-APIs
      erlaubt.
      \item Verschlüsselte Kommunikationion
      \item Asynchrone Dienst-Anfragen im Client
  \end{itemize}
  \item Informationsmodell
  \begin{itemize}
    \item Unterstützung aller OPC UA Node-Typen, inkl.~Methoden
    \item Hinzufügen und Entfernen von Nodes und Referenzen zur Laufzeit.
    \item Vererbung und Instanziierung von Objekt- und Variablentypen
    \item Zugriffskontrolle auch für einzelne Nodes
  \end{itemize}
  \item Subscriptions
  \begin{itemize}
    \item Erlaubt die Überwachung (subscriptions / monitoreditems)
    \item Sehr geringer Ressourcenbedarf pro überwachtem Wert
  \end{itemize}
  \item Code-Generierung auf XML-Basis
  \begin{itemize}
    \item Erlaubt die Erstellung von Datentypen
    \item Erlaubt die Generierung des serverseitigen Informationsmodells
  \end{itemize}
\end{itemize}

% https://open62541.org/doc/0.3/


Mozilla Public License
CC0 Lizenz für Beispiele und Plugins

% https://open62541.org/doc/open62541-current.pdf
% https://open62541.org/

% % % Imports nur für Referenzenauflösung während des Schreibens! Vorm Kompilieren auskommentieren!
% \bibliography{0_hauptdatei}
% \input{1_einleitung}
% \input{2_grundlagen}
% \input{3_konzeption}
% \input{4_implementierung}
% \input{5_tests}
% \input{6_zusammenfassung}
% \input{anhang}
% % Ende Imports

\section{Systemkonzept%
  \label{sec:3-konzeption}}
Auf Basis der in Abschnitt \ref{sec:2-grundlagen} vorgestellten Möglichkeiten folgt nun die Ausarbeitung eines Konzepts.
In den folgenden Abschnitten soll näher auf zwei zentrale Aspekte eingegangen werden: Abschnitt~\ref{sec:3-anbindung} stellt Möglichkeiten zum Zugriff auf Variablen bzw.\,Werte im Prozessabbild des Revolution Pi vor; in Abschnitt~\ref{sec:3-integration} wird ein Konzept zur Bereitstellung dieser Variablen auf einem OPC-Server vorgestellt.

\subsection{Anbindung der IO an den OPC-Server%
     \label{sec:3-anbindung}}

Eine Webanwendung mit Bezeichnung PiCtory dient zur Konfiguration der I/O- und virtuellen Module des RevolutionPi. Die Konfiguration liegt im JSON-Format in der Datei \lstinline{/etc/revpi/config.rsc}. Der piControl-Treiber liest diese Datei beim Start. 
Der folgende Auszug aus der Manpage des piControl-Kernelmoduls beschreibt die von diesem zum Lesen und Schreiben einzelner Bits des Prozessabbildes bereitgestellten Funktionen~\citep[vgl.]{web-revpi-manpage}. Sie ist an dieser Stelle weitgehend ungekürzt zitiert, da sie die nutzbare Schnittstelle sehr kompakt beschreibt.

\begin{lstlisting}[breakindent=0pt, numbers=none, caption={Auszug aus der Revolution Pi Programmers Manual\label{lst:4-manpage}}]
KB_FIND_VARIABLE SPIVariable *argp
Find a variable in the process image by its name. A pointer to a structure of type SPIVariable must be passed as argument. [...]
The struct SPIVariable [...] is defined as 
typedef struct SPIVariableStr
{
    char strVarName[32]; // Variable name
    uint16_t i16uAddress; // Address of the byte in the process image
    uint8_t i8uBit; // 0-7 bit position, >= 8 whole byte
    uint16_t i16uLength; // length of the variable in bits.
    // Possible values are 1, 8, 16 and 32
} SPIVariable;

Set and get values of the process image
KB_GET_VALUE SPIValue *argp
[...]
KB_SET_VALUE SPIValue *argp
Write one bit or one byte to the process image [...].  This call is more efficient than the usual calls of seek and write because only one function call is necessary. If more than on application are writing bits in one output byte, this call is the only safe way to set a bit without overwriting the other bits because this call is doing a read-modify-write-cycle. 

The struct SPIValue used by this ioctl is defined as
typedef struct SPIValueStr
{
    uint16_t i16uAddress; // Address of the byte in the process image
    uint8_t i8uBit; // 0-7 bit position, >= 8 whole byte
    uint8_t i8uValue; // Value: 0/1 for bit access, whole byte otherwise
} SPIValue;
\end{lstlisting} 

Die oben beschriebenden Funtkionen \lstinline{KB_FIND_VARIABLE}, \lstinline{KB_GET_VALUE} und \lstinline{KB_SET_VALUE} ermöglichen einen einfachen und (lt.\,Manpage) effizienten Zugriff auf einzelne Bits des Prozessabbildes und damit auch auf die IO des RevolutionPi.
Der Zugriff des OPC-Servers auf das Prozessabbild soll daher mittels dieser Funktionen realisiert werden.
\lstinline{KB_FIND_VARIABLE} kann genutzt werden, um Adressen von Variablen im Prozessabbild mittels ihres Namens aufzulösen.
\lstinline{KB_GET_VALUE} und \lstinline{KB_SET_VALUE} ermöglichen den Zugriff auf die Werte dieser Variablen.


\subsection{Integration des OPC-Servers in das System%
     \label{sec:3-integration}}

open62541 bietet drei Möglichkeiten zum Abgleich von Variablen mit dem Prozessabbild~\citep[vgl.][Tutorials - Connecting a Variable with a Physical Process]{web-open62541}:
\begin{itemize}
    \item Manuelles oder zyklisches Aktualisieren
    \item Variable Value Callback
    \item Variable Datasource
\end{itemize}

Die zyklische Aktualisierung eines oder mehrerer Werte nimmt, abhängig von der Zykluszeit, viele Systemressourcen in Anspruch. Value Callbacks ermöglichen es, einen Variablenwert effizienter mit einer Ressource wie etwa einem Prozessabbild zu synchronisieren. An die Variable wird ein Callback angehängt, welches vor jedem Lesen und nach jedem Schreibvorgang ausgeführt wird.
Der Wert der Variablen wird weiterhin im Variablenknoten auf dem OPC-Server gespeichert, der Abgleich mit der verknüpften Ressource erfolgt durch die Callback-Methoden.

Sogenannte Datenquellen gehen noch einen Schritt weiter. Der Server leitet jede Lese- und Schreibanforderung direkt an eine Callback-Funktion weiter. Beim Lesen liefert der Rückruf eine Kopie des aktuellen Wertes. Die Datenquelle muss intern ein eigenes Speichermanagement implementieren.

Der Zugriff auf die Werte des Prozessabbildes erfolgt, wie in Abschnitt~\ref{sec:3-anbindung} beschrieben, über von piControl bereitgestellte Methoden. Um die durch open62541 gepflegte OPC-Datenstruktur und das durch piControl verwaltete Prozessabbild möglichst effektiv verknüpfen zu können, soll diese Interaktion mittels Datenquellen und den zugehörigen Callbacks implementiert werden.
% % % Imports nur für Referenzenauflösung während des Schreibens! Vorm Kompilieren auskommentieren!
% \bibliography{0_hauptdatei}
% \input{1_einleitung}
% \input{2_grundlagen}
% \input{3_konzeption}
% \input{4_implementierung}
% \input{5_tests}
% \input{6_zusammenfassung}
% \input{anhang}
% % Ende Imports

\section{Implementierung%
  \label{sec:4-implementierung}}
Das folgende Kapitel stellt in Auszügen die Implementierung des OPC-Servers sowie die Anbindung an die IO-Module
der SPS dar. Der Schwerpunkt liegt hierbei auf der Funktionsweise des piControl-Treibers und dessen Integration in das Projekt. Abschnitt~\ref{sec:4-picontrol} erklärt die zum Schreibens eines Bits verwendeten Funktionsaufrufe.
Zuvor soll jedoch in Abschnitt~\ref{sec:4-open62541} der Teil des OPC-Servers vorgestellt werden, welcher auf besagten Treiber zugreift. 

\subsection{Implementierung des OPC-Servers%
     \label{sec:4-open62541}}
Wie im vorangegangenen Abschnitt~\ref{sec:3-integration} begründet, soll die Verknüpfung zwischen dem Prozessabbild der SPS und den auf dem OPC-Server bereitgestellten Werten über sog.\,Datenquellen erfolgen. Hierzu ist zunächst eine Callback-Methode zu implementieren, welche bei einem Lese- oder Schreibzugriff auf eine Variable aufgerufen wird. Die Verknüpfung zwischen Callback-Methode und Variable muss manuell erfolgen.

\begin{lstlisting}[language={c},firstnumber=237,caption={Auszug der Methode \lstinline{linkDataSourceVariable} in \lstinline{variables.c}\label{lst:4-linkDataSourceVariable}}]
extern UA_StatusCode
 linkDataSourceVariable(UA_Server *server, UA_NodeId nodeId) {
     bool readonly = false;
     UA_DataSource dataSourceVariable;
     UA_StatusCode rc; |>\setcounter{lstnumber}{254}<|

     dataSourceVariable.read = readDataSourceVariable;
     if (!readonly)
        dataSourceVariable.write = writeDataSourceVariable;
     else
        dataSourceVariable.write = writeReadonlyDataSourceVariable;

     return UA_Server_setVariableNode_dataSource(server, nodeId, dataSourceVariable);
 }
\end{lstlisting}

\begin{figure}[h]
    \centering
    \includegraphics[width=0.42\textwidth]{doc/img/OPC_RevPiDO.pdf}
    \caption{Auszug des verwendeten Nodesets, hier Digitalausgang 1 des Versuchsaufbaus
      \label{fig:opc-do}}
\end{figure}

Die in Listing~\ref{lst:4-linkDataSourceVariable} abgebildete Methode \lstinline{linkDataSourceVariable()} erzeugt ein Struct vom Typ \lstinline{UA_DataSource}. In diesem werden dem Lesen und Schreiben einer OPC-Variablen entsprechende Callback-Methoden zugewiesen. Die Verknüpfung einer OPC-Variable, genauer ihrer NodeId, mit der zuvor definierten Datenquelle erfolgt über die von open62541 bereitgestellte Methode \lstinline{UA_Server_setVariableNode_dataSource()}. Vor dem Lesen und nach dem Schreiben dieser Variable werden von nun an die entsprechenden Callbacks aufgerufen.
     
\begin{lstlisting}[language={c},firstnumber=168,caption={Auszug des Callbacks \lstinline{writeDataSourceVariable} in \lstinline{variables.c}\label{lst:4-writeDataSourceVariable}}]  
extern UA_StatusCode
 writeDataSourceVariable(UA_Server *server,
            const UA_NodeId *sessionId, void *sessionContext,
            const UA_NodeId *nodeId, void *nodeContext,
            const UA_NumericRange *range, const UA_DataValue *dataValue) {

    UA_StatusCode retval  = UA_STATUSCODE_GOOD;
    UA_NodeId *nameNodeId = UA_malloc(sizeof(UA_NodeId));
    UA_QualifiedName nameQN = UA_QUALIFIEDNAME(1, "Name");
    UA_Variant nameVar;
    UA_Boolean bit;

    retval |= findSiblingByBrowsename(server, nodeId, &nameQN, nameNodeId);
    retval |= UA_Server_readValue(server, *nameNodeId, &nameVar);
    retval |= UA_Boolean_copy(dataValue->value.data, &bit);

    |>\tikzmarkin[set border color=martinired]{writeIO}<|PI_writeSingleIO(String_fromUA_String(nameVar.data), &bit, false);                                                 |>\tikzmarkend{writeIO}<|

    free(nameNodeId);
    return retval;
 }
\end{lstlisting}

Listing~\ref{lst:4-writeDataSourceVariable} zeigt die Callback-Methode, welche nach dem Schreiben einer Variablen auf dem OPC-Server aufgerufen wird.
Dieser Methode wird neben der NodeId der mit ihr verknüpften Variablen auch der Wert dieser in Form eines Zeigers auf ein Struct vom Typ \lstinline{UA_DataValue} übergeben.

Die Gestaltung des hier verwendeten Nodesets sieht vor, dass in einer OPC-Variablen \lstinline{"Name"} der Bezeichner des zu schreibenden Digitalausgangs hinterlegt ist, siehe Abbildung~\ref{fig:opc-do}. Dies erlaubt eine Rekonfiguration der Ein- und Ausgänge der SPS ohne Änderungen im Programmcode des OPC-Servers vornehmen zu müssen.
Es ist daher erforderlich, nach jedem Schreiben einer mit einem Digitalausgang verknüpften Variablen, hier \lstinline{"Value"}, dessen Bezeichner \lstinline{"Name"} abzufragen. 
Dies geschieht in den Zeilen 180 und 181.
Anschließend wird dieser Bezeichner sowie der zu schreibende Wert der Methode \lstinline{PI_writeSingleIO()} übergeben, welche wiederum die Interaktion mit piControl übernimmt (vgl. Abschnitt \ref{sec:4-picontrol}).
 
\subsection{Integration von piControl%
     \label{sec:4-picontrol}}
In Abschnitt~\ref{sec:2-io} wurde die Anbindung der IO-Module des Revolution Pi sowie die Funktionsweise von piControl aus Anwendersicht beschrieben. Die verfügbare Literatur beschränkt sich auch auf lediglich diese Sicht; eine weiterführende Dokumentation für Entwickler gibt es, neben der in Abschnitt~\ref{sec:3-anbindung} vorgestellten Manpage, nicht. 
In diesem Abschnitt soll daher der Quellcode von piControl sowie dessen Verwendung im Projekt genauer betrachtet werden.
Hierzu wird exemplarisch die in Abschnitt~\ref{sec:4-open62541} eingeführte Methode \lstinline{PI_writeSingleIO()} untersucht.
Diese Methode ermöglicht das Setzen eines einzelnen Bits im Prozessabbild der SPS, und damit das Schalten eines digitalen Ausgangs auf einem IO-Modul.
Die äquivalente Methode \lstinline{int piControlGetBitValue(SPIValue *pSpiValue)} zum Lesen eines Bits bzw. Eingangs funktioniert analog und soll daher an dieser Stelle nicht dediziert erörtert werden.

\begin{lstlisting}[language={c},firstnumber=97,
                   caption={Setzen eines phsikalischen, digitalen Ausgangs in \lstinline{revpi.c}
                   \label{lst:4-PI_writeSingleIO}}]
extern void PI_writeSingleIO(char *pszVariableName, bool *bit, bool verbose)
{
	int rc;
	SPIVariable sPiVariable;
	SPIValue sPIValue;

	strncpy(sPiVariable.strVarName, pszVariableName, sizeof(sPiVariable.strVarName));
	rc = piControlGetVariableInfo(&sPiVariable);
	if (rc < 0) {
		printf("Cannot find variable '%s'\n", pszVariableName);
		return;
	}

		sPIValue.i16uAddress = sPiVariable.i16uAddress;
		sPIValue.i8uBit = sPiVariable.i8uBit;
		sPIValue.i8uValue = *bit;
		rc = |>\tikzmarkin[set border color=martinired]{setBitValue}<|piControlSetBitValue(&sPIValue)|>\tikzmarkend{setBitValue}<|;
		if (rc < 0)
			printf("Set bit error %s\n", getWriteError(rc));
		else if (verbose)
			printf("Set bit %d on byte at offset %d. Value %d\n", sPIValue.i8uBit, sPIValue.i16uAddress,
			       sPIValue.i8uValue);
}
\end{lstlisting}

Der Programmcode in Listing~\ref{lst:4-PI_writeSingleIO} ist Teil des implementierten OPC-Servers. In diesem wird auf zwei Funktionen des piControl-Treibers zugegriffen. 
Beiden Methoden wird als Argument ein Zeiger auf ein Struct vom Typ \lstinline{SPIValue} übergeben. Der im Struct abgelegte Name wird mittels \lstinline{piControlGetVariableInfo(&sPIValue)} zu einer Adresse im Prozessabbild aufgelöst. Diese wird in \lstinline{sPIValue.i16uAdress} gespeichert. Der Wert der Variablen wird anschließend mittels \lstinline{piControlSetBitValue(&sPIValue)} an dieser Adresse in das Prozessabbild geschrieben.

\begin{lstlisting}[language={c},firstnumber=309,caption={Methode \lstinline{piControlSetBitValue} in \lstinline{piControlIf.c}\label{lst:4-piControlSetBitValue}}]
int |>\tikzmarkin[set border color=martiniblue]{setBitValueFcn}<|piControlSetBitValue(SPIValue *pSpiValue)|>\tikzmarkend{setBitValueFcn}<|
{
    piControlOpen();

    if (PiControlHandle_g < 0)
	    return -ENODEV;

    pSpiValue->i16uAddress += pSpiValue->i8uBit / 8;
    pSpiValue->i8uBit %= 8;

    if (|>\tikzmarkin[set border color=martinired]{ioctl}<|ioctl(PiControlHandle_g, KB_SET_VALUE, pSpiValue)|>\tikzmarkend{ioctl}<| < 0)
	    return errno;

    return 0;
}
\end{lstlisting}

Die in Listing~\ref{lst:4-piControlSetBitValue} dargestellte Methode \lstinline{piControlSetBitValue} ist lediglich eine Hüllfunktion (häufig auch als Wrapper-Funktion bezeichnet) für einen Aufruf des \lstinline{ioctl} Kernel-Moduls.
Folgende Parameter werden übergeben:
\lstinline{PiControlHandle_g} ist die Referenz auf die Geräte-Datei des piControl-Treibers. \lstinline{KB_SET_VALUE} ist das ioctl-Kommando zum Schreiben eines Bits in das Prozessabbild. Der Zeiger \lstinline{pSpiValue} verweist auf ein Struct des bereits vorgestellten Typs \lstinline{SPIValue}.

\begin{lstlisting}[language={c},firstnumber=80,caption={Methode \lstinline{piControlOpen} in \lstinline{piControlIf.c}\label{lst:4-piControlOpen}}]
void piControlOpen(void)
{
    /* open handle if needed */
    if (PiControlHandle_g < 0)
    {
	    |>\tikzmarkin[set border color=martiniblue]{PiControlHandle}<|PiControlHandle_g = open(PICONTROL_DEVICE, O_RDWR)|>\tikzmarkend{PiControlHandle}<|;
    }
}
\end{lstlisting}

Die in Listing~\ref{lst:4-piControlOpen} dargestellte Methode öffnet, sofern nicht bereits geschehen, die Geräte-Datei. Das Macro \lstinline{PICONTROL_DEVICE} verweist hierbei auf \lstinline{/dev/piControl0}.

\begin{lstlisting}[language={c},firstnumber=721,caption={Methode \lstinline{piControlIoctl} in \lstinline{piControlMain.c}\label{lst:4-piControlIoctl}}]
static long |>\tikzmarkin[set border color=martiniblue, below offset=0.9em]{piControlIoctl}<|piControlIoctl(struct file *file, unsigned int prg_nr, 
                           unsigned long usr_addr)                                      |>\tikzmarkend{piControlIoctl}<|
{
  int status = -EFAULT;
  tpiControlInst *priv;
  int timeout = 10000;	// ms

  if (prg_nr != KB_CONFIG_SEND && prg_nr != KB_CONFIG_START && !isRunning()) {
  	return -EAGAIN;
  }

  priv = (tpiControlInst *) file->private_data;

  if (prg_nr != KB_GET_LAST_MESSAGE) {
  	// clear old message
  	priv->pcErrorMessage[0] = 0;
  }

  switch (prg_nr) {|>\setcounter{lstnumber}{864}<|

    case |>\tikzmarkin[set border color=martiniblue]{KB_SET_VALUE}<|KB_SET_VALUE:|>\tikzmarkend{KB_SET_VALUE}<|
  		{
  			SPIValue *pValue = (SPIValue *) usr_addr;

  			if (!isRunning())
  				return -EFAULT;

  			if (pValue->i16uAddress >= KB_PI_LEN) {
  				status = -EFAULT;
  			} else {
  				INT8U i8uValue_l;
  				my_rt_mutex_lock(&piDev_g.lockPI);
  				i8uValue_l = piDev_g.ai8uPI[pValue->i16uAddress];

  				if (pValue->i8uBit >= 8) {
  					i8uValue_l = pValue->i8uValue;
  				} else {
  					if (pValue->i8uValue)
  						i8uValue_l |= (1 << pValue->i8uBit);
  					else
  						i8uValue_l &= ~(1 << pValue->i8uBit);
  				}

  				|>\tikzmarkin[set border color=martinired]{i8uValue}<|piDev_g.ai8uPI[pValue->i16uAddress] = i8uValue_l;|>\tikzmarkend{i8uValue}<|
  				rt_mutex_unlock(&piDev_g.lockPI);

  #ifdef VERBOSE
  				pr_info("piControlIoctl Addr=%u, bit=%u: %02x %02x\n", pValue->i16uAddress, pValue->i8uBit, pValue->i8uValue, i8uValue_l);
  #endif

  				status = 0;
  			}
  		}
  		break; |>\setcounter{lstnumber}{1314}<|

    default:
      pr_err("Invalid Ioctl");
      return (-EINVAL);
      break;

    }

    return status;
  }
\end{lstlisting}

Listing~\ref{lst:4-piControlIoctl} zeigt in Auszügen die ioctl-Methode des piControl Kernel-Treibers. Diese bekommt folgende Argumente übergeben: \lstinline{struct file *file} enthält den Verweis auf die Geräte-Datei, hier \lstinline{/dev/piControl0}. Der Wert von \lstinline{unsigned int prg_nr} beschreibt die Anfrage an den Treiber, in diesem Fall \lstinline{KB_SET_VALUE}. Das Argument \lstinline{unsigned long usr_addr} enthält einen typ-agnostischen Pointer. Dieser verweist auf einen Speicherbereich, in welchem die zur Bearbeitung der Anfrage notwendigen Daten abgelegt sind. Hier können auch vom Treiber empfangene Daten dem Anwendungsprogramm bereitgestellt werden. 

Die switch-case-Anweisung führt die über das Argument \lstinline{prg_nr} spezifizierte Aktion aus. Hier betrachten wir \lstinline{KB_SET_VALUE}:
Zunächst wird in Zeile 868 der übergebene Zeiger \lstinline{usr_addr} mittels explizitem Typecast zu einem Zeiger des Typs \lstinline{SPIValue *} konvertiert. Da dieser auf Daten im Userspace verweist, ist beim Zugriff durch den Kernel-Treiber besondere Vorsicht geboten.
In Zeile 877 wird mittels Mutex das Prozessabbild \lstinline{piDev_g} für den Zugriff durch andere Threads oder Prozesse gesperrt.
\lstinline{my_rt_mutex_lock} verweist hierbei auf die Funktion \lstinline{rt_mutex_lock} aus \lstinline{linux/sched.h}\footnote{Offenbar wurde hier auch eine alternative Implementierung vorgesehen, siehe revpi\_common.h}

In Zeile 889 wird das Byte \lstinline{i8uValue_l}, welches den zu schreibenden Wert enthält in das Prozessabbild übertragen. Anschließend wird die Mutex auf \lstinline{piDev_g} wieder entsperrt.
\newpage

\begin{lstlisting}[language={c},firstnumber=62,caption={Auszug des Struct \lstinline{spiControlDev} in \lstinline{piControlMain.h}\label{lst:4-spiControlDev}}]
|>\tikzmarkin[set border color=martiniblue]{spiControlDev}<|typedef struct spiControlDev|>\tikzmarkend{spiControlDev}<| {
	// device driver stuff
	int init_step;
	enum revpi_machine machine_type;
	void *machine;
	struct cdev cdev;	// Char device structure
	struct device *dev;
	struct thermal_zone_device *thermal_zone;

	|>\tikzmarkin[set border color=martiniblue]{processImage}<|// process image stuff
	INT8U ai8uPI[KB_PI_LEN];
	INT8U ai8uPIDefault|>\tikzmarkin[set border color=martinired]{KB_PI_LEN_0}<|[KB_PI_LEN]|>\tikzmarkend{KB_PI_LEN_0}<|;
	struct rt_mutex lockPI;        |>\tikzmarkend{processImage}<|
	bool stopIO;
	piDevices *devs; |>\setcounter{lstnumber}{94}<|
} tpiControlDev;
\end{lstlisting}

Das Prozessabbild ist als Byte-Array der Länge \lstinline{KB_PI_LEN} in Listing~\ref{lst:4-spiControlDev} definiert. Konfigurationsparameter wie \lstinline{KB_PI_LEN} oder die Zykluszeit für den Datenaustausch zwischen SPS und IO-Modulen sind im folgenden Listing~\ref{lst:4-process} definiert.

\begin{lstlisting}[language={c},firstnumber=119,caption={Konfigurationsparameter des Prozessabbildes in project.h\label{lst:4-process}}]
#define INTERVAL_PI_GATE (5*1000*1000)  // 5 ms piGateCommunication |>\setcounter{lstnumber}{128}<|

#define INTERVAL_IO_COM (5*1000*1000)  // 5 ms piIoComm |>\setcounter{lstnumber}{132}<|

#define KB_PD_LEN       512
|>\tikzmarkin[set border color=martiniblue]{KB_PI_LEN_1}<|#define KB_PI_LEN       4096|>\tikzmarkend{KB_PI_LEN_1}<|
\end{lstlisting}

Das zu setzende Bit wurde zu diesem Zeitpunkt erfolgreich in das Prozessabbild der SPS geschrieben.
Es stellt sich die Frage, wie dieses nun an das IO-Modul kommuniziert wird.
Die Kommunikation mit allen angebundenen Modulen ist ebenfalls Aufgabe des piControl-Treibers.

\begin{lstlisting}[language={c},firstnumber=256,caption={Auszug der Methode \lstinline{piIoThread} in \lstinline{revpi_core.c}\label{lst:4-piIoThread}}]
static int piIoThread(void *data)
{
	//TODO int value = 0;
	ktime_t time;
	ktime_t now;
	s64 tDiff;

	hrtimer_init(&piCore_g.ioTimer, CLOCK_MONOTONIC, HRTIMER_MODE_ABS);
	piCore_g.ioTimer.function = piIoTimer;

	pr_info("piIO thread started\n");

	now = hrtimer_cb_get_time(&piCore_g.ioTimer);

	PiBridgeMaster_Reset();

	while (!kthread_should_stop()) {
		if (|>\tikzmarkin[set border color=martinired]{PiBridgeMaster}<|PiBridgeMaster_Run()|>\tikzmarkend{PiBridgeMaster}<| < 0)
			break;
	}

	RevPiDevice_finish();

	pr_info("piIO exit\n");
	return 0;
}
\end{lstlisting}

Der Kernel-Thread \lstinline{piIoThread} ist verantwortlich für den zyklischen Datenaustausch mit den IO-Modulen. In diesem wird fortlaufend die Methode \lstinline{PiBridgeMaster_Run()} aufgerufen, siehe Listing~\ref{lst:4-piIoThread}.

\begin{lstlisting}[language={c},firstnumber=262,caption={Auszug der Methode \lstinline{PiBridgeMaster_Run(void)} in \lstinline{RevPiDevice.c}\label{lst:4-PiBridgeMaster_Run}}]
int PiBridgeMaster_Run(void)
{
	static kbUT_Timer tTimeoutTimer_s;
	static kbUT_Timer tConfigTimeoutTimer_s;
	static int error_cnt;
	static INT8U last_led;
	static unsigned long last_update;
	int ret = 0;
	int i;

	my_rt_mutex_lock(&piCore_g.lockBridgeState);
	if (piCore_g.eBridgeState != piBridgeStop) {
		switch (eRunStatus_s) { |>\setcounter{lstnumber}{514}<|
		    case enPiBridgeMasterStatus_EndOfConfig:|>\setcounter{lstnumber}{621}<|
		    if (|>\tikzmarkin[set border color=martinired]{RevPiDevice}<|RevPiDevice_run()|>\tikzmarkend{RevPiDevice}<|) {
				// an error occured, check error limits |>\setcounter{lstnumber}{641}<|
			} else {
				ret = 1;
			}
			piCore_g.image.drv.i16uRS485ErrorCnt = RevPiDevice_getErrCnt();
			break;
\end{lstlisting}

Die in Listing~\ref{lst:4-PiBridgeMaster_Run} dargestellte Methode ist eine sog. State-Machine. Ist die Konfiguration der IO-Module erfolgreich abgeschlossen, so führt sie bei Aufruf lediglich die Methode \lstinline{RevPiDevice_run()} aus.

\begin{lstlisting}[language={c},firstnumber=140,caption={Auszug der Methode \lstinline{RevPiDevice_run(void)} in \lstinline{RevPiDevice.c}\label{lst:4-RevPiDevice_run}}]
int RevPiDevice_run(void)
{
	INT8U i8uDevice = 0;
	INT32U r;
	int retval = 0;

	RevPiDevices_s.i16uErrorCnt = 0;

	for (i8uDevice = 0; i8uDevice < RevPiDevice_getDevCnt(); i8uDevice++) {
		if (RevPiDevice_getDev(i8uDevice)->i8uActive) {
			switch (RevPiDevice_getDev(i8uDevice)->sId.i16uModulType) {
			case KUNBUS_FW_DESCR_TYP_PI_DIO_14:
			case KUNBUS_FW_DESCR_TYP_PI_DI_16:
			case KUNBUS_FW_DESCR_TYP_PI_DO_16:
				r = |>\tikzmarkin[set border color=martinired]{sendCyclicTelegram}<|piDIOComm_sendCyclicTelegram(i8uDevice)|>\tikzmarkend{sendCyclicTelegram}\setcounter{lstnumber}{166} <|;

				break; |>\setcounter{lstnumber}{216}<|
			}
		}
	} |>\setcounter{lstnumber}{227}<|
	return retval;
}
\end{lstlisting}

Diese iteriert wie in Listing~\ref{lst:4-RevPiDevice_run} abgebildete durch alle gegenwärtig in der SPS konfigurierten Module. Ist das aktuelle Modul als aktiv markiert, so wird anhand eines sog. Firmware-Descriptors entschieden, welche Methode für die Ansteuerung des Moduls aufzurufen ist.

\begin{lstlisting}[language={c},firstnumber=161,caption={Auszug der Methode \lstinline{piDIOComm_sendCyclicTelegram} in \lstinline{piDIOComm.c}\label{lst:4-sendCyclicTelegram}}]
INT32U piDIOComm_sendCyclicTelegram(INT8U i8uDevice_p)
{
	INT32U i32uRv_l = 0;
	SIOGeneric sRequest_l;
	SIOGeneric sResponse_l;
	INT8U len_l, data_out[18], i, p, data_in[70];
	INT8U i8uAddress;
	int ret; |>\setcounter{lstnumber}{239}<|
	
    |>\tikzmarkin[set border color=martinired]{piIoComm}<|ret = piIoComm_send((INT8U *) & sRequest_l, IOPROTOCOL_HEADER_LENGTH + len_l + 1);  |>\tikzmarkend{piIoComm}\setcounter{lstnumber}{298}<|
}
\end{lstlisting}

Im Falle des hier verwendeten DO-Moduls wird die in Listing~\ref{lst:4-sendCyclicTelegram} abgebildete Methode \lstinline{piDIOComm_sendCyclicTelegram()} aufgerufen. Dieser wird ein Zeiger auf das zu schreibende Gerät übergeben. 
Zunächst wird das Prozessabbild mittels eines proprietären, jedoch im Quellcode offen nachvollziehbaren Protokolls in ein \lstinline{sRequest_l} genanntes Byte-Array umgewandelt. Dieser Schritt ist in Listing~\ref{lst:4-sendCyclicTelegram} nicht abgebildet. Anschließend wird \lstinline{piIoComm_send()} ein Zeiger auf die so generierte Schreib-Anfrage übergeben.

\begin{lstlisting}[language={c},firstnumber=220,caption={Auszug der Methode \lstinline{piIOComm_send} in \lstinline{piIOComm.c}\label{lst:4-piIOComm_send}}]
int piIoComm_send(INT8U * buf_p, INT16U i16uLen_p)
{
	ssize_t write_l = 0;
	INT16U i16uSent_l = 0;|>\setcounter{lstnumber}{249}<|

	while (i16uSent_l < i16uLen_p) {
		write_l = vfs_write(piIoComm_fd_m, buf_p + i16uSent_l, i16uLen_p - i16uSent_l, &piIoComm_fd_m->f_pos);
		if (write_l < 0) {
			pr_info_serial("write error %d\n", (int)write_l);
			return -1;
		} 
		i16uSent_l += write_l;|>\setcounter{lstnumber}{263}<|
	}
	clear();
	vfs_fsync(piIoComm_fd_m, 1);
	return 0;
}
\end{lstlisting}

Listing~\ref{lst:4-piIOComm_send} zeigt die Implementierung von \lstinline{piIoComm_send()}. Diese Methode ist für das Schreiben der oben generierten Anfrage auf die seriellen Schnittstelle verantwortlich. Realisiert wird dies mittels der Methode \lstinline{vfs_write()}. Diese ist in \lstinline{<linux/fs.h>} definiert. Sie ermöglicht das Schreiben einer Datei im Userspace aus dem Kernel heraus. Geschrieben wird hier die Datei mit dem Deskriptor \lstinline{piIoComm_fd_m}.
Da die Funktion \lstinline{vfs_write()} durch andere Kernel-Tasks unterbrochen werden kann, ist nicht gewährleistet, dass die gesamte Anfrage mit nur einem Aufruf geschrieben wird. Die oben abgebildete while-Schleife stellt das vollständige Senden der Anfrage sicher.

\begin{lstlisting}[language={c},firstnumber=157,caption={Auszug der Methode \lstinline{piIOComm_open_serial} in \lstinline{piIOComm.c}\label{lst:4-piIOComm_open_serial}}]
int piIoComm_open_serial(void)
{   |>\setcounter{lstnumber}{167}<|
	struct file *fd;	/* Filedeskriptor */
	struct termios newtio;	/* Schnittstellenoptionen */

	|>\tikzmarkin[set border color=martiniblue]{fd}<|/* Port oeffnen - read/write, kein "controlling tty", 
	    Status von DCD ignorieren */
	fd = filp_open(|>\tikzmarkin[set border color=martinired]{tty}<|REV_PI_TTY_DEVICE|>\tikzmarkend{tty}<|, O_RDWR | O_NOCTTY, 0); |>\setcounter{lstnumber}{208}<|
	
	piIoComm_fd_m = fd;                                                      |>\tikzmarkend{fd}\setcounter{lstnumber}{217}<|

	return 0;
}
\end{lstlisting}

Der zum Schreiben auf die serielle Schnittstelle verwendete Datei-Deskriptor wird von der in Listing~\ref{lst:4-piIOComm_open_serial} abgebildeten Methode \lstinline{piIoComm_open_serial()} generiert. 

\begin{lstlisting}[language={c},firstnumber=45,caption={Definition der seriellen Schnittstelle in \lstinline{piIOComm.h}\label{lst:4-REV_PI_TTY_DEVICE}}]
#define REV_PI_TTY_DEVICE	"/dev/ttyAMA0"
\end{lstlisting}

Das in Listing~\ref{lst:4-REV_PI_TTY_DEVICE} definierte Macro verweist auf eine der seriellen Schnittstellen des RaspberryPi.
Die Implementierung des zugehörigen Schnittstellentreibers soll hier nicht weiter untersucht werden. Somit ist an dieser Stelle die Kette vom Setzen einer Variablen auf dem OPC-Server bis hin zur Aktualisierung des Prozessabbilds der IO-Module geschlossen.

% \begin{lstlisting}[language={c},firstnumber={226},caption={Setzen der Scheduler-Priorität auf SCHED\_FIFO in 
% revpi\_common.c\label{lst:2-sched_priority}}]
% param.sched_priority = ktprio->prio;
% ret = sched_setscheduler(child, SCHED_FIFO, &param);
% \end{lstlisting}
% % % Imports nur für Referenzenauflösung während des Schreibens! Vorm Kompilieren auskommentieren!
% \bibliography{0_hauptdatei}
% \input{1_einleitung}
% \input{2_grundlagen}
% \input{3_konzeption}
% \input{4_implementierung}
% \input{5_tests}
% \input{6_zusammenfassung}
% % Ende Imports

\section{Test des OPC-Servers im Gesamtsystem%
  \label{sec:5-tests}}

% % % Imports nur für Referenzenauflösung während des schreibens! Vorm Kompilieren auskommentieren!
% \bibliography{0_hauptdatei}
% \input{1_einleitung}
% \input{2_grundlagen}
% \input{3_konzeption}
% \input{4_implementierung}
% \input{5_tests}
% \input{6_zusammenfassung}
% % Ende Imports

\section{Zusammenfassung und Ausblick%
  \label{sec:6-fazit}}
Der folgende Abschnitt~\ref{sec:6-zusammenfassung} fasst die gewonnenen Erkenntnisse und den Stand der Implementierung zusammen.
Den Abschluss dieser Arbeit bildet der Ausblick in Abschnitt~\ref{sec:6-ausblick}.

\subsection{Zusammenfassung%
     \label{sec:6-zusammenfassung}}

\subsection{Ausblick%
     \label{sec:6-ausblick}}

% \input{anhang}
% % Ende Imports

\section{Implementierung%
  \label{sec:4-implementierung}}
Das folgende Kapitel stellt in Auszügen die Implementierung des OPC-Servers sowie die Anbindung an die IO-Module
der SPS dar. Der Schwerpunkt liegt hierbei auf der Funktionsweise des piControl-Treibers und dessen Integration in das Projekt. Abschnitt~\ref{sec:4-picontrol} erklärt die zum Schreibens eines Bits verwendeten Funktionsaufrufe.
Zuvor soll jedoch in Abschnitt~\ref{sec:4-open62541} der Teil des OPC-Servers vorgestellt werden, welcher auf besagten Treiber zugreift. 

\subsection{Implementierung des OPC-Servers%
     \label{sec:4-open62541}}
Wie im vorangegangenen Abschnitt~\ref{sec:3-integration} begründet, soll die Verknüpfung zwischen dem Prozessabbild der SPS und den auf dem OPC-Server bereitgestellten Werten über sog.\,Datenquellen erfolgen. Hierzu ist zunächst eine Callback-Methode zu implementieren, welche bei einem Lese- oder Schreibzugriff auf eine Variable aufgerufen wird. Die Verknüpfung zwischen Callback-Methode und Variable muss manuell erfolgen.

\begin{lstlisting}[language={c},firstnumber=237,caption={Auszug der Methode \lstinline{linkDataSourceVariable} in \lstinline{variables.c}\label{lst:4-linkDataSourceVariable}}]
extern UA_StatusCode
 linkDataSourceVariable(UA_Server *server, UA_NodeId nodeId) {
     bool readonly = false;
     UA_DataSource dataSourceVariable;
     UA_StatusCode rc; |>\setcounter{lstnumber}{254}<|

     dataSourceVariable.read = readDataSourceVariable;
     if (!readonly)
        dataSourceVariable.write = writeDataSourceVariable;
     else
        dataSourceVariable.write = writeReadonlyDataSourceVariable;

     return UA_Server_setVariableNode_dataSource(server, nodeId, dataSourceVariable);
 }
\end{lstlisting}

\begin{figure}[h]
    \centering
    \includegraphics[width=0.42\textwidth]{doc/img/OPC_RevPiDO.pdf}
    \caption{Auszug des verwendeten Nodesets, hier Digitalausgang 1 des Versuchsaufbaus
      \label{fig:opc-do}}
\end{figure}

Die in Listing~\ref{lst:4-linkDataSourceVariable} abgebildete Methode \lstinline{linkDataSourceVariable()} erzeugt ein Struct vom Typ \lstinline{UA_DataSource}. In diesem werden dem Lesen und Schreiben einer OPC-Variablen entsprechende Callback-Methoden zugewiesen. Die Verknüpfung einer OPC-Variable, genauer ihrer NodeId, mit der zuvor definierten Datenquelle erfolgt über die von open62541 bereitgestellte Methode \lstinline{UA_Server_setVariableNode_dataSource()}. Vor dem Lesen und nach dem Schreiben dieser Variable werden von nun an die entsprechenden Callbacks aufgerufen.
     
\begin{lstlisting}[language={c},firstnumber=168,caption={Auszug des Callbacks \lstinline{writeDataSourceVariable} in \lstinline{variables.c}\label{lst:4-writeDataSourceVariable}}]  
extern UA_StatusCode
 writeDataSourceVariable(UA_Server *server,
            const UA_NodeId *sessionId, void *sessionContext,
            const UA_NodeId *nodeId, void *nodeContext,
            const UA_NumericRange *range, const UA_DataValue *dataValue) {

    UA_StatusCode retval  = UA_STATUSCODE_GOOD;
    UA_NodeId *nameNodeId = UA_malloc(sizeof(UA_NodeId));
    UA_QualifiedName nameQN = UA_QUALIFIEDNAME(1, "Name");
    UA_Variant nameVar;
    UA_Boolean bit;

    retval |= findSiblingByBrowsename(server, nodeId, &nameQN, nameNodeId);
    retval |= UA_Server_readValue(server, *nameNodeId, &nameVar);
    retval |= UA_Boolean_copy(dataValue->value.data, &bit);

    |>\tikzmarkin[set border color=martinired]{writeIO}<|PI_writeSingleIO(String_fromUA_String(nameVar.data), &bit, false);                                                 |>\tikzmarkend{writeIO}<|

    free(nameNodeId);
    return retval;
 }
\end{lstlisting}

Listing~\ref{lst:4-writeDataSourceVariable} zeigt die Callback-Methode, welche nach dem Schreiben einer Variablen auf dem OPC-Server aufgerufen wird.
Dieser Methode wird neben der NodeId der mit ihr verknüpften Variablen auch der Wert dieser in Form eines Zeigers auf ein Struct vom Typ \lstinline{UA_DataValue} übergeben.

Die Gestaltung des hier verwendeten Nodesets sieht vor, dass in einer OPC-Variablen \lstinline{"Name"} der Bezeichner des zu schreibenden Digitalausgangs hinterlegt ist, siehe Abbildung~\ref{fig:opc-do}. Dies erlaubt eine Rekonfiguration der Ein- und Ausgänge der SPS ohne Änderungen im Programmcode des OPC-Servers vornehmen zu müssen.
Es ist daher erforderlich, nach jedem Schreiben einer mit einem Digitalausgang verknüpften Variablen, hier \lstinline{"Value"}, dessen Bezeichner \lstinline{"Name"} abzufragen. 
Dies geschieht in den Zeilen 180 und 181.
Anschließend wird dieser Bezeichner sowie der zu schreibende Wert der Methode \lstinline{PI_writeSingleIO()} übergeben, welche wiederum die Interaktion mit piControl übernimmt (vgl. Abschnitt \ref{sec:4-picontrol}).
 
\subsection{Integration von piControl%
     \label{sec:4-picontrol}}
In Abschnitt~\ref{sec:2-io} wurde die Anbindung der IO-Module des Revolution Pi sowie die Funktionsweise von piControl aus Anwendersicht beschrieben. Die verfügbare Literatur beschränkt sich auch auf lediglich diese Sicht; eine weiterführende Dokumentation für Entwickler gibt es, neben der in Abschnitt~\ref{sec:3-anbindung} vorgestellten Manpage, nicht. 
In diesem Abschnitt soll daher der Quellcode von piControl sowie dessen Verwendung im Projekt genauer betrachtet werden.
Hierzu wird exemplarisch die in Abschnitt~\ref{sec:4-open62541} eingeführte Methode \lstinline{PI_writeSingleIO()} untersucht.
Diese Methode ermöglicht das Setzen eines einzelnen Bits im Prozessabbild der SPS, und damit das Schalten eines digitalen Ausgangs auf einem IO-Modul.
Die äquivalente Methode \lstinline{int piControlGetBitValue(SPIValue *pSpiValue)} zum Lesen eines Bits bzw. Eingangs funktioniert analog und soll daher an dieser Stelle nicht dediziert erörtert werden.

\begin{lstlisting}[language={c},firstnumber=97,
                   caption={Setzen eines phsikalischen, digitalen Ausgangs in \lstinline{revpi.c}
                   \label{lst:4-PI_writeSingleIO}}]
extern void PI_writeSingleIO(char *pszVariableName, bool *bit, bool verbose)
{
	int rc;
	SPIVariable sPiVariable;
	SPIValue sPIValue;

	strncpy(sPiVariable.strVarName, pszVariableName, sizeof(sPiVariable.strVarName));
	rc = piControlGetVariableInfo(&sPiVariable);
	if (rc < 0) {
		printf("Cannot find variable '%s'\n", pszVariableName);
		return;
	}

		sPIValue.i16uAddress = sPiVariable.i16uAddress;
		sPIValue.i8uBit = sPiVariable.i8uBit;
		sPIValue.i8uValue = *bit;
		rc = |>\tikzmarkin[set border color=martinired]{setBitValue}<|piControlSetBitValue(&sPIValue)|>\tikzmarkend{setBitValue}<|;
		if (rc < 0)
			printf("Set bit error %s\n", getWriteError(rc));
		else if (verbose)
			printf("Set bit %d on byte at offset %d. Value %d\n", sPIValue.i8uBit, sPIValue.i16uAddress,
			       sPIValue.i8uValue);
}
\end{lstlisting}

Der Programmcode in Listing~\ref{lst:4-PI_writeSingleIO} ist Teil des implementierten OPC-Servers. In diesem wird auf zwei Funktionen des piControl-Treibers zugegriffen. 
Beiden Methoden wird als Argument ein Zeiger auf ein Struct vom Typ \lstinline{SPIValue} übergeben. Der im Struct abgelegte Name wird mittels \lstinline{piControlGetVariableInfo(&sPIValue)} zu einer Adresse im Prozessabbild aufgelöst. Diese wird in \lstinline{sPIValue.i16uAdress} gespeichert. Der Wert der Variablen wird anschließend mittels \lstinline{piControlSetBitValue(&sPIValue)} an dieser Adresse in das Prozessabbild geschrieben.

\begin{lstlisting}[language={c},firstnumber=309,caption={Methode \lstinline{piControlSetBitValue} in \lstinline{piControlIf.c}\label{lst:4-piControlSetBitValue}}]
int |>\tikzmarkin[set border color=martiniblue]{setBitValueFcn}<|piControlSetBitValue(SPIValue *pSpiValue)|>\tikzmarkend{setBitValueFcn}<|
{
    piControlOpen();

    if (PiControlHandle_g < 0)
	    return -ENODEV;

    pSpiValue->i16uAddress += pSpiValue->i8uBit / 8;
    pSpiValue->i8uBit %= 8;

    if (|>\tikzmarkin[set border color=martinired]{ioctl}<|ioctl(PiControlHandle_g, KB_SET_VALUE, pSpiValue)|>\tikzmarkend{ioctl}<| < 0)
	    return errno;

    return 0;
}
\end{lstlisting}

Die in Listing~\ref{lst:4-piControlSetBitValue} dargestellte Methode \lstinline{piControlSetBitValue} ist lediglich eine Hüllfunktion (häufig auch als Wrapper-Funktion bezeichnet) für einen Aufruf des \lstinline{ioctl} Kernel-Moduls.
Folgende Parameter werden übergeben:
\lstinline{PiControlHandle_g} ist die Referenz auf die Geräte-Datei des piControl-Treibers. \lstinline{KB_SET_VALUE} ist das ioctl-Kommando zum Schreiben eines Bits in das Prozessabbild. Der Zeiger \lstinline{pSpiValue} verweist auf ein Struct des bereits vorgestellten Typs \lstinline{SPIValue}.

\begin{lstlisting}[language={c},firstnumber=80,caption={Methode \lstinline{piControlOpen} in \lstinline{piControlIf.c}\label{lst:4-piControlOpen}}]
void piControlOpen(void)
{
    /* open handle if needed */
    if (PiControlHandle_g < 0)
    {
	    |>\tikzmarkin[set border color=martiniblue]{PiControlHandle}<|PiControlHandle_g = open(PICONTROL_DEVICE, O_RDWR)|>\tikzmarkend{PiControlHandle}<|;
    }
}
\end{lstlisting}

Die in Listing~\ref{lst:4-piControlOpen} dargestellte Methode öffnet, sofern nicht bereits geschehen, die Geräte-Datei. Das Macro \lstinline{PICONTROL_DEVICE} verweist hierbei auf \lstinline{/dev/piControl0}.

\begin{lstlisting}[language={c},firstnumber=721,caption={Methode \lstinline{piControlIoctl} in \lstinline{piControlMain.c}\label{lst:4-piControlIoctl}}]
static long |>\tikzmarkin[set border color=martiniblue, below offset=0.9em]{piControlIoctl}<|piControlIoctl(struct file *file, unsigned int prg_nr, 
                           unsigned long usr_addr)                                      |>\tikzmarkend{piControlIoctl}<|
{
  int status = -EFAULT;
  tpiControlInst *priv;
  int timeout = 10000;	// ms

  if (prg_nr != KB_CONFIG_SEND && prg_nr != KB_CONFIG_START && !isRunning()) {
  	return -EAGAIN;
  }

  priv = (tpiControlInst *) file->private_data;

  if (prg_nr != KB_GET_LAST_MESSAGE) {
  	// clear old message
  	priv->pcErrorMessage[0] = 0;
  }

  switch (prg_nr) {|>\setcounter{lstnumber}{864}<|

    case |>\tikzmarkin[set border color=martiniblue]{KB_SET_VALUE}<|KB_SET_VALUE:|>\tikzmarkend{KB_SET_VALUE}<|
  		{
  			SPIValue *pValue = (SPIValue *) usr_addr;

  			if (!isRunning())
  				return -EFAULT;

  			if (pValue->i16uAddress >= KB_PI_LEN) {
  				status = -EFAULT;
  			} else {
  				INT8U i8uValue_l;
  				my_rt_mutex_lock(&piDev_g.lockPI);
  				i8uValue_l = piDev_g.ai8uPI[pValue->i16uAddress];

  				if (pValue->i8uBit >= 8) {
  					i8uValue_l = pValue->i8uValue;
  				} else {
  					if (pValue->i8uValue)
  						i8uValue_l |= (1 << pValue->i8uBit);
  					else
  						i8uValue_l &= ~(1 << pValue->i8uBit);
  				}

  				|>\tikzmarkin[set border color=martinired]{i8uValue}<|piDev_g.ai8uPI[pValue->i16uAddress] = i8uValue_l;|>\tikzmarkend{i8uValue}<|
  				rt_mutex_unlock(&piDev_g.lockPI);

  #ifdef VERBOSE
  				pr_info("piControlIoctl Addr=%u, bit=%u: %02x %02x\n", pValue->i16uAddress, pValue->i8uBit, pValue->i8uValue, i8uValue_l);
  #endif

  				status = 0;
  			}
  		}
  		break; |>\setcounter{lstnumber}{1314}<|

    default:
      pr_err("Invalid Ioctl");
      return (-EINVAL);
      break;

    }

    return status;
  }
\end{lstlisting}

Listing~\ref{lst:4-piControlIoctl} zeigt in Auszügen die ioctl-Methode des piControl Kernel-Treibers. Diese bekommt folgende Argumente übergeben: \lstinline{struct file *file} enthält den Verweis auf die Geräte-Datei, hier \lstinline{/dev/piControl0}. Der Wert von \lstinline{unsigned int prg_nr} beschreibt die Anfrage an den Treiber, in diesem Fall \lstinline{KB_SET_VALUE}. Das Argument \lstinline{unsigned long usr_addr} enthält einen typ-agnostischen Pointer. Dieser verweist auf einen Speicherbereich, in welchem die zur Bearbeitung der Anfrage notwendigen Daten abgelegt sind. Hier können auch vom Treiber empfangene Daten dem Anwendungsprogramm bereitgestellt werden. 

Die switch-case-Anweisung führt die über das Argument \lstinline{prg_nr} spezifizierte Aktion aus. Hier betrachten wir \lstinline{KB_SET_VALUE}:
Zunächst wird in Zeile 868 der übergebene Zeiger \lstinline{usr_addr} mittels explizitem Typecast zu einem Zeiger des Typs \lstinline{SPIValue *} konvertiert. Da dieser auf Daten im Userspace verweist, ist beim Zugriff durch den Kernel-Treiber besondere Vorsicht geboten.
In Zeile 877 wird mittels Mutex das Prozessabbild \lstinline{piDev_g} für den Zugriff durch andere Threads oder Prozesse gesperrt.
\lstinline{my_rt_mutex_lock} verweist hierbei auf die Funktion \lstinline{rt_mutex_lock} aus \lstinline{linux/sched.h}\footnote{Offenbar wurde hier auch eine alternative Implementierung vorgesehen, siehe revpi\_common.h}

In Zeile 889 wird das Byte \lstinline{i8uValue_l}, welches den zu schreibenden Wert enthält in das Prozessabbild übertragen. Anschließend wird die Mutex auf \lstinline{piDev_g} wieder entsperrt.
\newpage

\begin{lstlisting}[language={c},firstnumber=62,caption={Auszug des Struct \lstinline{spiControlDev} in \lstinline{piControlMain.h}\label{lst:4-spiControlDev}}]
|>\tikzmarkin[set border color=martiniblue]{spiControlDev}<|typedef struct spiControlDev|>\tikzmarkend{spiControlDev}<| {
	// device driver stuff
	int init_step;
	enum revpi_machine machine_type;
	void *machine;
	struct cdev cdev;	// Char device structure
	struct device *dev;
	struct thermal_zone_device *thermal_zone;

	|>\tikzmarkin[set border color=martiniblue]{processImage}<|// process image stuff
	INT8U ai8uPI[KB_PI_LEN];
	INT8U ai8uPIDefault|>\tikzmarkin[set border color=martinired]{KB_PI_LEN_0}<|[KB_PI_LEN]|>\tikzmarkend{KB_PI_LEN_0}<|;
	struct rt_mutex lockPI;        |>\tikzmarkend{processImage}<|
	bool stopIO;
	piDevices *devs; |>\setcounter{lstnumber}{94}<|
} tpiControlDev;
\end{lstlisting}

Das Prozessabbild ist als Byte-Array der Länge \lstinline{KB_PI_LEN} in Listing~\ref{lst:4-spiControlDev} definiert. Konfigurationsparameter wie \lstinline{KB_PI_LEN} oder die Zykluszeit für den Datenaustausch zwischen SPS und IO-Modulen sind im folgenden Listing~\ref{lst:4-process} definiert.

\begin{lstlisting}[language={c},firstnumber=119,caption={Konfigurationsparameter des Prozessabbildes in project.h\label{lst:4-process}}]
#define INTERVAL_PI_GATE (5*1000*1000)  // 5 ms piGateCommunication |>\setcounter{lstnumber}{128}<|

#define INTERVAL_IO_COM (5*1000*1000)  // 5 ms piIoComm |>\setcounter{lstnumber}{132}<|

#define KB_PD_LEN       512
|>\tikzmarkin[set border color=martiniblue]{KB_PI_LEN_1}<|#define KB_PI_LEN       4096|>\tikzmarkend{KB_PI_LEN_1}<|
\end{lstlisting}

Das zu setzende Bit wurde zu diesem Zeitpunkt erfolgreich in das Prozessabbild der SPS geschrieben.
Es stellt sich die Frage, wie dieses nun an das IO-Modul kommuniziert wird.
Die Kommunikation mit allen angebundenen Modulen ist ebenfalls Aufgabe des piControl-Treibers.

\begin{lstlisting}[language={c},firstnumber=256,caption={Auszug der Methode \lstinline{piIoThread} in \lstinline{revpi_core.c}\label{lst:4-piIoThread}}]
static int piIoThread(void *data)
{
	//TODO int value = 0;
	ktime_t time;
	ktime_t now;
	s64 tDiff;

	hrtimer_init(&piCore_g.ioTimer, CLOCK_MONOTONIC, HRTIMER_MODE_ABS);
	piCore_g.ioTimer.function = piIoTimer;

	pr_info("piIO thread started\n");

	now = hrtimer_cb_get_time(&piCore_g.ioTimer);

	PiBridgeMaster_Reset();

	while (!kthread_should_stop()) {
		if (|>\tikzmarkin[set border color=martinired]{PiBridgeMaster}<|PiBridgeMaster_Run()|>\tikzmarkend{PiBridgeMaster}<| < 0)
			break;
	}

	RevPiDevice_finish();

	pr_info("piIO exit\n");
	return 0;
}
\end{lstlisting}

Der Kernel-Thread \lstinline{piIoThread} ist verantwortlich für den zyklischen Datenaustausch mit den IO-Modulen. In diesem wird fortlaufend die Methode \lstinline{PiBridgeMaster_Run()} aufgerufen, siehe Listing~\ref{lst:4-piIoThread}.

\begin{lstlisting}[language={c},firstnumber=262,caption={Auszug der Methode \lstinline{PiBridgeMaster_Run(void)} in \lstinline{RevPiDevice.c}\label{lst:4-PiBridgeMaster_Run}}]
int PiBridgeMaster_Run(void)
{
	static kbUT_Timer tTimeoutTimer_s;
	static kbUT_Timer tConfigTimeoutTimer_s;
	static int error_cnt;
	static INT8U last_led;
	static unsigned long last_update;
	int ret = 0;
	int i;

	my_rt_mutex_lock(&piCore_g.lockBridgeState);
	if (piCore_g.eBridgeState != piBridgeStop) {
		switch (eRunStatus_s) { |>\setcounter{lstnumber}{514}<|
		    case enPiBridgeMasterStatus_EndOfConfig:|>\setcounter{lstnumber}{621}<|
		    if (|>\tikzmarkin[set border color=martinired]{RevPiDevice}<|RevPiDevice_run()|>\tikzmarkend{RevPiDevice}<|) {
				// an error occured, check error limits |>\setcounter{lstnumber}{641}<|
			} else {
				ret = 1;
			}
			piCore_g.image.drv.i16uRS485ErrorCnt = RevPiDevice_getErrCnt();
			break;
\end{lstlisting}

Die in Listing~\ref{lst:4-PiBridgeMaster_Run} dargestellte Methode ist eine sog. State-Machine. Ist die Konfiguration der IO-Module erfolgreich abgeschlossen, so führt sie bei Aufruf lediglich die Methode \lstinline{RevPiDevice_run()} aus.

\begin{lstlisting}[language={c},firstnumber=140,caption={Auszug der Methode \lstinline{RevPiDevice_run(void)} in \lstinline{RevPiDevice.c}\label{lst:4-RevPiDevice_run}}]
int RevPiDevice_run(void)
{
	INT8U i8uDevice = 0;
	INT32U r;
	int retval = 0;

	RevPiDevices_s.i16uErrorCnt = 0;

	for (i8uDevice = 0; i8uDevice < RevPiDevice_getDevCnt(); i8uDevice++) {
		if (RevPiDevice_getDev(i8uDevice)->i8uActive) {
			switch (RevPiDevice_getDev(i8uDevice)->sId.i16uModulType) {
			case KUNBUS_FW_DESCR_TYP_PI_DIO_14:
			case KUNBUS_FW_DESCR_TYP_PI_DI_16:
			case KUNBUS_FW_DESCR_TYP_PI_DO_16:
				r = |>\tikzmarkin[set border color=martinired]{sendCyclicTelegram}<|piDIOComm_sendCyclicTelegram(i8uDevice)|>\tikzmarkend{sendCyclicTelegram}\setcounter{lstnumber}{166} <|;

				break; |>\setcounter{lstnumber}{216}<|
			}
		}
	} |>\setcounter{lstnumber}{227}<|
	return retval;
}
\end{lstlisting}

Diese iteriert wie in Listing~\ref{lst:4-RevPiDevice_run} abgebildete durch alle gegenwärtig in der SPS konfigurierten Module. Ist das aktuelle Modul als aktiv markiert, so wird anhand eines sog. Firmware-Descriptors entschieden, welche Methode für die Ansteuerung des Moduls aufzurufen ist.

\begin{lstlisting}[language={c},firstnumber=161,caption={Auszug der Methode \lstinline{piDIOComm_sendCyclicTelegram} in \lstinline{piDIOComm.c}\label{lst:4-sendCyclicTelegram}}]
INT32U piDIOComm_sendCyclicTelegram(INT8U i8uDevice_p)
{
	INT32U i32uRv_l = 0;
	SIOGeneric sRequest_l;
	SIOGeneric sResponse_l;
	INT8U len_l, data_out[18], i, p, data_in[70];
	INT8U i8uAddress;
	int ret; |>\setcounter{lstnumber}{239}<|
	
    |>\tikzmarkin[set border color=martinired]{piIoComm}<|ret = piIoComm_send((INT8U *) & sRequest_l, IOPROTOCOL_HEADER_LENGTH + len_l + 1);  |>\tikzmarkend{piIoComm}\setcounter{lstnumber}{298}<|
}
\end{lstlisting}

Im Falle des hier verwendeten DO-Moduls wird die in Listing~\ref{lst:4-sendCyclicTelegram} abgebildete Methode \lstinline{piDIOComm_sendCyclicTelegram()} aufgerufen. Dieser wird ein Zeiger auf das zu schreibende Gerät übergeben. 
Zunächst wird das Prozessabbild mittels eines proprietären, jedoch im Quellcode offen nachvollziehbaren Protokolls in ein \lstinline{sRequest_l} genanntes Byte-Array umgewandelt. Dieser Schritt ist in Listing~\ref{lst:4-sendCyclicTelegram} nicht abgebildet. Anschließend wird \lstinline{piIoComm_send()} ein Zeiger auf die so generierte Schreib-Anfrage übergeben.

\begin{lstlisting}[language={c},firstnumber=220,caption={Auszug der Methode \lstinline{piIOComm_send} in \lstinline{piIOComm.c}\label{lst:4-piIOComm_send}}]
int piIoComm_send(INT8U * buf_p, INT16U i16uLen_p)
{
	ssize_t write_l = 0;
	INT16U i16uSent_l = 0;|>\setcounter{lstnumber}{249}<|

	while (i16uSent_l < i16uLen_p) {
		write_l = vfs_write(piIoComm_fd_m, buf_p + i16uSent_l, i16uLen_p - i16uSent_l, &piIoComm_fd_m->f_pos);
		if (write_l < 0) {
			pr_info_serial("write error %d\n", (int)write_l);
			return -1;
		} 
		i16uSent_l += write_l;|>\setcounter{lstnumber}{263}<|
	}
	clear();
	vfs_fsync(piIoComm_fd_m, 1);
	return 0;
}
\end{lstlisting}

Listing~\ref{lst:4-piIOComm_send} zeigt die Implementierung von \lstinline{piIoComm_send()}. Diese Methode ist für das Schreiben der oben generierten Anfrage auf die seriellen Schnittstelle verantwortlich. Realisiert wird dies mittels der Methode \lstinline{vfs_write()}. Diese ist in \lstinline{<linux/fs.h>} definiert. Sie ermöglicht das Schreiben einer Datei im Userspace aus dem Kernel heraus. Geschrieben wird hier die Datei mit dem Deskriptor \lstinline{piIoComm_fd_m}.
Da die Funktion \lstinline{vfs_write()} durch andere Kernel-Tasks unterbrochen werden kann, ist nicht gewährleistet, dass die gesamte Anfrage mit nur einem Aufruf geschrieben wird. Die oben abgebildete while-Schleife stellt das vollständige Senden der Anfrage sicher.

\begin{lstlisting}[language={c},firstnumber=157,caption={Auszug der Methode \lstinline{piIOComm_open_serial} in \lstinline{piIOComm.c}\label{lst:4-piIOComm_open_serial}}]
int piIoComm_open_serial(void)
{   |>\setcounter{lstnumber}{167}<|
	struct file *fd;	/* Filedeskriptor */
	struct termios newtio;	/* Schnittstellenoptionen */

	|>\tikzmarkin[set border color=martiniblue]{fd}<|/* Port oeffnen - read/write, kein "controlling tty", 
	    Status von DCD ignorieren */
	fd = filp_open(|>\tikzmarkin[set border color=martinired]{tty}<|REV_PI_TTY_DEVICE|>\tikzmarkend{tty}<|, O_RDWR | O_NOCTTY, 0); |>\setcounter{lstnumber}{208}<|
	
	piIoComm_fd_m = fd;                                                      |>\tikzmarkend{fd}\setcounter{lstnumber}{217}<|

	return 0;
}
\end{lstlisting}

Der zum Schreiben auf die serielle Schnittstelle verwendete Datei-Deskriptor wird von der in Listing~\ref{lst:4-piIOComm_open_serial} abgebildeten Methode \lstinline{piIoComm_open_serial()} generiert. 

\begin{lstlisting}[language={c},firstnumber=45,caption={Definition der seriellen Schnittstelle in \lstinline{piIOComm.h}\label{lst:4-REV_PI_TTY_DEVICE}}]
#define REV_PI_TTY_DEVICE	"/dev/ttyAMA0"
\end{lstlisting}

Das in Listing~\ref{lst:4-REV_PI_TTY_DEVICE} definierte Macro verweist auf eine der seriellen Schnittstellen des RaspberryPi.
Die Implementierung des zugehörigen Schnittstellentreibers soll hier nicht weiter untersucht werden. Somit ist an dieser Stelle die Kette vom Setzen einer Variablen auf dem OPC-Server bis hin zur Aktualisierung des Prozessabbilds der IO-Module geschlossen.

% \begin{lstlisting}[language={c},firstnumber={226},caption={Setzen der Scheduler-Priorität auf SCHED\_FIFO in 
% revpi\_common.c\label{lst:2-sched_priority}}]
% param.sched_priority = ktprio->prio;
% ret = sched_setscheduler(child, SCHED_FIFO, &param);
% \end{lstlisting}
% % % Imports nur für Referenzenauflösung während des Schreibens! Vorm Kompilieren auskommentieren!
% \bibliography{0_hauptdatei}
% % Mit \section{...} eröffnen wir einen neuen Abschnitt.
% Der Befehl setzt nicht nur den Text in einer größeren,
% fetten Schrift, sondern sorgt außerdem dafür, daß er im
% Inhaltsverzeichnis erscheint.
%
% Mit \label{...} erzeugen wir einen Bezeichner, mit dessen Hilfe
% wir später auf die Nummer des Abschnitts verweisen können (nämlich
% mit~\ref{...}).
%
% Das Kommentarzeichen hinter „Übersicht“ dient dazu, ein
% Leerzeichen zwischen „Übersicht“ und dem \label-Befehl
% zu vermeiden, das andernfalls sichtbar würde – z.B. im
% Inhaltsverzeichnis.
%

% % Imports nur für Referenzenauflösung während des Schreibens! Vorm Kompilieren auskommentieren!
% \bibliography{0_hauptdatei}
% \input{1_einleitung}
%\input{2_grundlagen}
%\input{3_konzeption}
%\input{4_implementierung}
%\input{5_tests}
%\input{6_zusammenfassung}
% % Ende Imports

\section{Einleitung und Motivation%
  \label{sec:1-einleitung}}
Ziel dieses Projektes ist die Integration eines OPC-Servers mit einer auf Linux
basierenden speicherprogrammierbaren Steuerung (SPS). Angeschlossen an diese SPS
ist jeweils ein digitales Ein-/\,bzw.~Ausgabemodul. Die von diesen bereitgestellten
Ein-/\, bzw.~Ausgänge (IO) sollen in der Datenstruktur des OPC-Servers abgebildet
und über diesen für OPC-Clients les-/\,und schreibar sein. Weiterhin sollen einige
Funktionen zur Überwachung und Steuerung der an die SPS angeschlossenen Aktoren
und Sensoren direkt im OPC-Server implementiert werden.
Hiermit stellt dieses Projekt eine der Grundlagen für ein übergeordnetes Projekt,
die cloudbasierte Steuerung eines miniaturisierten Produktions-Systems, dar.

Der hier verwendete OPC-Server ist Teil des sog. open62541 Projekts. Er ist in C
geschrieben und implementiert bereits einen großen Teil der im OPC-UA-Standard
spezifizierten Funktionen.
Als SPS findet ein Revolution Pi 3 der Firma Kunbus Verwendung. Dieser integriert
ein sog. Compute Module der Raspberry Pi Foundation in ein industrietaugliches
Gehäuse und erlaubt die Erweiterung mittels IO- oder Gateway-Modulen. Über diese
erfolgt die Kommunikation mit weiteren Komponenten der Automatisierungstechnik.

Motiviert ist dieses Projekt durch die Beobachtung, dass die Verbreitung offener
Standards sowie freier Software auch in der Automatisierungstechnik zunimmt.
Linux ist ein freies Betriebssystem, OPC-UA ein offen zugänglicher, aktiv gepflegter
und weit verbreiteter Standard. Der Raspberry Pi findet sowohl bei Hobby-Anwendern als
auch in den Bereichen Forschung und Entwicklung sowie bei industriellen Anwendern
Verwendung. Dieses Projekt stellt somit eine für unterschiedliche Anwender interessante
Entwicklung dar.

Im Anschluss an diese einleitende Übersicht im Abschnitt~\ref{sec:1-einleitung} folgt
die Darstellung der wichtigsten Grundlagen in Abschnitt~\ref{sec:2-grundlagen}.
Aufbauend auf diesen Grundlagen folgt die konzeptuelle Ausarbeitung im Abschnitt~\ref{sec:3-konzeption}.
Die Umsetzung wird im Abschnitt~\ref{sec:4-implementierung} erläutert.
Die Leistungsfähigkeit der Implementierung wird in Abschnitt~\ref{sec:5-tests} untersucht.
Eine Zusammenfassung und ein Ausblick schließen die Arbeit in
Abschnitt~\ref{sec:6-fazit} ab. Eventuell noch benötigte Anhänge
finden sich in den Anhängen [...] bis [...].

% % % Imports nur für Referenzenauflösung während des Schreibens! Vorm Kompilieren auskommentieren!
% \bibliography{0_hauptdatei}
% \input{1_einleitung}
% \input{2_grundlagen}
% \input{3_konzeption}
% \input{4_implementierung}
% \input{5_tests}
% \input{6_zusammenfassung}
% % Ende Imports

\section{Grundlagen%
  \label{sec:2-grundlagen}}

\subsection{Speicherprogrammierbare-Steuerung und Linux -- Revolution Pi%
     \label{sec:2-sps}}

\subsubsection{Kunbus RevolutionPi%
        \label{sec:2-revpi}}
Der RevolutionPi 3 ist eine speicherprogrammierbare Steuerung (SPS) des Herstellers
Kunbus GmbH. Kern dieser SPS ist das von der Raspberry Pi Foundation entwickelte
und vertriebene Raspberry Pi Compute Module 3. Dieses integriert ein Broadcom BCM2837
System-on-Chip (SoC) mit vier 1,2GHz Prozessorkernen, 1GB RAM, 4GB eMMC Anwendungsspeicher
und sonstige Peripherie in ein Modul im DDR2-SODIMM Formfaktor. Diese Spezifikationen
sind weitgehend identisch zu denen des ausgesprochen populären Raspberry Pi 3.
Der Revolution Pi profitiert daher von dem gleichen großen Angebot an Software
und Unterstützung wie der Raspberry Pi, ergänzt dessen Hardware jedoch um eine 24V
Spannungsversorgung, die Möglichkeit der Erweiterung durch mehrere industrietaugliche
Ein-/ Ausgabemodule und Gateways sowie ein Gehäuse zur Montage auf einer DIN-Schiene.
\begin{itemize}
  \item{Prozessor: BCM2837}
  \item{Taktfrequenz 1,2 GHz}
  \item{Anzahl Prozessorkerne: 4}
  \item{Arbeitsspeicher: 1 GByte}
  \item{eMMC Flash Speicher: 4 GByte}
  \item{Betriebssystem: Angepasstes Raspbian mit RT-Patch}
  \item{RTC mit 24h Pufferung über wartungsfreien Kondensator}
  \item{Treiber / API: Treiber schreibt zyklisch Prozessdaten in ein Prozessabbild, Zugriff auf Prozessabbild über Linux-Filesystem als API zu Fremdsoftware.}
  \item{Kommunikationsanschlüsse: 2 x USB 2.0 A (je 500 mA belastbar), 1 x Micro-USB, HDMI, Ethernet (RJ45) 10/100 Mbit/s}
  \item{Stromversorgung: min. 10,7 V, max. 28,8 V, maximal 10 Watt}
  \item{Zulässige Umgebungstemperatur: -40 bis +55 C}
  \item{Gehäuseabmessungen: (HxBxL) 96 mm x 22,5 mm x 110,5 mm (ohne gesteckte Stecker)}
  \item{ESD Schutz: 4 kV / 8 kV gemäß EN61131-2 und IEC 61000-6-2}
  \item{Surge / Burst Prüfungen: gemäß EN61131-2 und IEC 61000-6-2 eingekoppelt auf Versorgungsspannung, Ethernet und IO-Leitungen}
  \item{EMI Prüfungen: gemäß EN61131-2 und IEC 61000-6-2}
\end{itemize}

Kunbus bietet eine Auswahl an IO- und Gateway-Modulen zur Erweiterung des Revolution Pi an.
Gateways dienen der Kommunikation mit Systemen oder Komponenten der Automatisierungstechnik
über Protokolle wie PROFIBUS oder EtherCAT. IO-Module erlauben die Überwachung
und Steuerung von digitalen oder analogen Ein- und Ausgängen.

\subsubsection{Zugriff auf IO-Module%
        \label{sec:2-io}}
Der Zugriff auf die Ein- und Ausgänge der IO-Module erfolgt über ein Prozessabbild
und einen hierfür von Kunbus bereitgestellten Treiber, genannt piControl. Dieser
aktualisiert das Prozessabbild zyklisch. Die angestrebte Zykluszeit beträgt 5ms,
kann jedoch je nach Anzahl der angeschlossenen Module auch größer sein. Kunbus
garantiert bei drei IO-Modulen und zwei Gateway-Modulen eine Zykluszeit von 10 ms.
Jedes der IO-Module stellt ein eigenständiges eingebettetes System dar. Es verfügt
über einen Microcontroller, welcher die IOs bereitstellt und über einen RS485-Bus
mit dem Revolution Pi kommuniziert.
% https://revolution.kunbus.de/io-modul/

Lizenz: GPL
% https://github.com/RevolutionPi/piControl

\begin{lstlisting}[language={c},firstnumber={226},caption={Setzen der Scheduler-Priorität auf SCHED\_FIFO in revpi\_common.c\label{lst:2-sched_priority}}]
param.sched_priority = ktprio->prio;
ret = sched_setscheduler(child, SCHED_FIFO,
       &param);
\end{lstlisting}


\subsection{Echtzeit und Multithreading unter Linux -- preemptRT und posix%
     \label{sec:2-echtzeit}}


 Der Linux-Kernel verfügt über mehrere unterschiedliche Preemtion-Modelle:

\begin{itemize}
  \item No Forced Preemption (server):
  Ausgelegt auf maximal möglichen Durchsatz, lediglich Interrupts und
  System-Call-Returns bewirken Präemption.

  \item Voluntary Kernel Preemption (Desktop):
  Neben den implizit bevorrechtigten Interrupts und System-Call-Returns gibt es
  in diesem Modell weitere Abschnitte des Kernels in welchen Preämption explizit
  gestattet ist.

  \item Preemptible Kernel (Low-Latency Desktop):
  In diesem Modell ist der gesamte Kernel, mit Ausnahme sog.~kritischer Abschnitte
  präemptible. Nach jedem kritischen Abschnitt gibt es einen impliziten Präemptions-Punkt.

  \item Preemptible Kernel (Basic RT):
  Dieses Modell ist dem zuvor genannten sehr ähnlich, hier sind jedoch alle Interrupt-Handler
  als eigenständige Threads ausgeführt.

  \item Fully Preemptible Kernel (RT):
  Wie auch bei den beiden zuvor genannten Modellen ist hier der gesamte Kernel
  präemtible, die Anzahl und Dauer der nicht-präemtiblen kritischen Abschnitte
  ist auf ein notwendiges Minimum beschränkt. Alle Interrupt-Handler sind als
  eigenständige Threads ausgeführt, Spinlocks durch Sleeping-Spinlocks und Mutexe
  durch sog.~RT-Mutexe ersetzt.

\end{itemize}
\todo{Spinlocks und Mutexe sowie die RT-Varianten dieser erklären!}

Lediglich mit dem vollständig präemtiblen Kernel kann Echtzeit-Verhalten realisiert werden.

% https://wiki.linuxfoundation.org/realtime/documentation/technical_basics/preemption_models bzw kernel/Kconfig.preempt

\subsubsection{preemptRT%
        \label{sec:2-preemptRT}}
% https://wiki.linuxfoundation.org/realtime/documentation/technical_details/start
% https://wiki.linuxfoundation.org/realtime/documentation/technical_basics/start

Das dem PREEMPT RT Kernel zugrunde liegende Prinzip lässt sich in einer einfachen
Regel ausdrücken: Nur Code, welcher absolut nicht-präemtible sein darf, ist es
gestattet nicht-präemtible zu sein.
Das erklärte Ziel des PREEMPT\_RT Patches ist es folglich, die Menge des nicht-präemtiblen
Codes im Linux-Kernel auf das absolut notwendige Minimum zu reduzieren.

Dies wird durch Verwendung folgender Mechanismen erreicht:

\begin{itemize}
  \item Hochauflösende Timer
  \item Sleeping Spinlocks
  \item Threaded Interrupt Handlers
  \item rt\_mutex
  \item RCU
\end{itemize}


\subsubsection{posix%
        \label{sec:2-posix}}
Ist posix hier wirklich relevant? Debian bzw.~Raspbian sind weitgehend posix
kompatibel, aber wird es hier genutzt? -> JA, open62541 nutzt pthread.h
piControl nutzt kthread.h, und semaphore.h

\subsection{OPC-UA und open62541%
     \label{sec:2-opc}}

\subsubsection{OPC UA%
        \label{sec:2-opcua}}
Open Platform Communications (OPC) ist eine Familie von Standards zur herstellerunabhängigen
Kommunikation von Maschinen (M2M) in der Automatisierungstechnik. Die sog.~OPC Task Force, zu deren
Mitgliedern verschiedene große Firmen der Automatisierungsindustrie gehören, veröffentlichte
die OPC Specification Version 1.0 im August 1996.
Motiviert ist dieser offene Standard durch die Erkenntniss, dass die Anpassung der
zahlreichen Herstellerstandards an individuelle Infrastrukturen und Anlagen einen
großen Mehraufwand verursachen.
Die Wikipedia beschreibt das Anwendungsgebiet für OPC wie folgt:

\glqq{}OPC wird dort eingesetzt, wo Sensoren, Regler und Steuerungen verschiedener Hersteller
ein gemeinsames Netzwerk bilden. Ohne OPC benötigten zwei Geräte zum Datenaustausch
genaue Kenntnis über die Kommunikationsmöglichkeiten des Gegenübers. Erweiterungen
und Austausch gestalten sich entsprechend schwierig. Mit OPC genügt es, für jedes
Gerät genau einmal einen OPC-konformen Treiber zu schreiben. Idealerweise wird
dieser bereits vom Hersteller zur Verfügung gestellt. Ein OPC-Treiber lässt sich
ohne großen Anpassungsaufwand in beliebig große Steuer- und Überwachungssysteme
integrieren.

OPC unterteilt sich in verschiedene Unterstandards, die für den jeweiligen Anwendungsfall
unabhängig voneinander implementiert werden können. OPC lässt sich damit verwenden
für Echtzeitdaten (Überwachung), Datenarchivierung, Alarm-Meldungen und neuerdings
auch direkt zur Steuerung (Befehlsübermittlung).\grqq{}

OPC basiert in der ursprünglichen Spezifikation auf Microsofts DCOM-Spezifikation.
DCOM macht Funktionen und Objekte einer Anwendung anderen Anwendungen im Netzwerk
zugänglich. Der OPC-Standard definiert entsprechende DCOM-Objekte um mit anderen
OPC-Anwendungen Daten austauschen zu können. Die Verwendung von DCOM bindet Anwender
an Betriebssysteme von Microsoft. Die ursprüngliche OPC Spezifikation wird durch die
Entwicklung von OPC Unified Architecture (OPC UA) abgelöst.
OPC UA setzt auf einem eigenen Kommunikationionsstack auf, die Verwendung von DCOM
und damit die Bindung an Microsoft wurden aufgelöst.

Die OPC-UA-Architektur ist eine Service-orientierte Architektur (SOA), deren Struktur
aus mehreren Schichten besteht.

% Wikipedia
Das OPC-Informationsmodell ist nicht mehr nur eine Hierarchie aus Ordnern, Items
und Properties. Es ist ein sogenanntes Full-Mesh-Network aus Nodes, mit dem neben
den Nutzdaten eines Nodes auch Meta- und Diagnoseinformationen repräsentiert werden.
Ein Node ähnelt einem Objekt aus der objektorientierten Programmierung. Ein Node
kann Attribute besitzen, die gelesen werden können (Data Access (DA), Historical
Data Access (HDA)). Es ist möglich Methoden zu definieren und aufzurufen.
Eine Methode besitzt Aufrufargumente und Rückgabewerte. Sie wird durch ein Command
aufgerufen. Weiterhin werden Events unterstützt, die versendet werden können
(AE (Alarms \& Events), DA DataChange), um bestimmte Informationen zwischen Geräten
auszutauschen. Ein Event besitzt unter anderem einen Empfangszeitpunkt, eine Nachricht
und einen Schweregrad. Die o. g. Nodes werden sowohl für die Nutzdaten als auch
alle anderen Arten von Metadaten verwendet. Der damit modellierte OPC-Adressraum
beinhaltet nun auch ein Typmodell, mit dem sämtliche Datentypen spezifiziert werden.

% https://de.wikipedia.org/wiki/Open_Platform_Communications
% https://de.wikipedia.org/wiki/OPC_Unified_Architecture
% https://opcfoundation.org/developer-tools/specifications-unified-architecture
% Von Gerhard Gappmeier - ascolab GmbH, CC BY-SA 3.0, https://de.wikipedia.org/w/index.php?curid=1892069
\subsubsection{open62541%
        \label{sec:2-open62541}}
open62541 ist eine offene und freie Implementierung von OPC UA. Die in C geschriebene
Bibliothek stellt eine beständig zunehmende Anzahl der im OPC UA Standard definierten
Funktionen bereit. Sie kann sowohl zur Erstellung von OPC-Servern als auch -Clients
genutzt werden. Ergänzend zu der unter der Mozilla Public License v2.0 lizensierten
Bibliothek stellt das open62541 Projekt auch Beispielprogramme unter einer CC0 Lizenz
zur Verfügung.

Die Bibliothek eignet sich auch für die Entwicklung auf eingebetteten Systemen und
Microcontrollern. Je nach Umfang der gewünschten Funktionen und des OPC Informationsmodells
beträgt die Größe einer Server-Binary weniger als 100kb. %evtl. kürzen?

\todo{Nodes erklären! Evtl.~oben!}

Folgende Auswahl an Eigenschaften und Funktionen zeichnet die in dieser Arbeit verwendete
Version 0.3 von open62541 aus:
\begin{itemize}
  \item Kommunikationionsstack
  \begin{itemize}
      \item OPC UA Binär-Protokoll (HTTP oder SOAP werden gegenwärtig nicht unterstützt)
      \item Austauschbare Netzwerk-Schicht, welche die Verwendung eigener Netzwerk-APIs
      erlaubt.
      \item Verschlüsselte Kommunikationion
      \item Asynchrone Dienst-Anfragen im Client
  \end{itemize}
  \item Informationsmodell
  \begin{itemize}
    \item Unterstützung aller OPC UA Node-Typen, inkl.~Methoden
    \item Hinzufügen und Entfernen von Nodes und Referenzen zur Laufzeit.
    \item Vererbung und Instanziierung von Objekt- und Variablentypen
    \item Zugriffskontrolle auch für einzelne Nodes
  \end{itemize}
  \item Subscriptions
  \begin{itemize}
    \item Erlaubt die Überwachung (subscriptions / monitoreditems)
    \item Sehr geringer Ressourcenbedarf pro überwachtem Wert
  \end{itemize}
  \item Code-Generierung auf XML-Basis
  \begin{itemize}
    \item Erlaubt die Erstellung von Datentypen
    \item Erlaubt die Generierung des serverseitigen Informationsmodells
  \end{itemize}
\end{itemize}

% https://open62541.org/doc/0.3/


Mozilla Public License
CC0 Lizenz für Beispiele und Plugins

% https://open62541.org/doc/open62541-current.pdf
% https://open62541.org/

% % % Imports nur für Referenzenauflösung während des Schreibens! Vorm Kompilieren auskommentieren!
% \bibliography{0_hauptdatei}
% \input{1_einleitung}
% \input{2_grundlagen}
% \input{3_konzeption}
% \input{4_implementierung}
% \input{5_tests}
% \input{6_zusammenfassung}
% \input{anhang}
% % Ende Imports

\section{Systemkonzept%
  \label{sec:3-konzeption}}
Auf Basis der in Abschnitt \ref{sec:2-grundlagen} vorgestellten Möglichkeiten folgt nun die Ausarbeitung eines Konzepts.
In den folgenden Abschnitten soll näher auf zwei zentrale Aspekte eingegangen werden: Abschnitt~\ref{sec:3-anbindung} stellt Möglichkeiten zum Zugriff auf Variablen bzw.\,Werte im Prozessabbild des Revolution Pi vor; in Abschnitt~\ref{sec:3-integration} wird ein Konzept zur Bereitstellung dieser Variablen auf einem OPC-Server vorgestellt.

\subsection{Anbindung der IO an den OPC-Server%
     \label{sec:3-anbindung}}

Eine Webanwendung mit Bezeichnung PiCtory dient zur Konfiguration der I/O- und virtuellen Module des RevolutionPi. Die Konfiguration liegt im JSON-Format in der Datei \lstinline{/etc/revpi/config.rsc}. Der piControl-Treiber liest diese Datei beim Start. 
Der folgende Auszug aus der Manpage des piControl-Kernelmoduls beschreibt die von diesem zum Lesen und Schreiben einzelner Bits des Prozessabbildes bereitgestellten Funktionen~\citep[vgl.]{web-revpi-manpage}. Sie ist an dieser Stelle weitgehend ungekürzt zitiert, da sie die nutzbare Schnittstelle sehr kompakt beschreibt.

\begin{lstlisting}[breakindent=0pt, numbers=none, caption={Auszug aus der Revolution Pi Programmers Manual\label{lst:4-manpage}}]
KB_FIND_VARIABLE SPIVariable *argp
Find a variable in the process image by its name. A pointer to a structure of type SPIVariable must be passed as argument. [...]
The struct SPIVariable [...] is defined as 
typedef struct SPIVariableStr
{
    char strVarName[32]; // Variable name
    uint16_t i16uAddress; // Address of the byte in the process image
    uint8_t i8uBit; // 0-7 bit position, >= 8 whole byte
    uint16_t i16uLength; // length of the variable in bits.
    // Possible values are 1, 8, 16 and 32
} SPIVariable;

Set and get values of the process image
KB_GET_VALUE SPIValue *argp
[...]
KB_SET_VALUE SPIValue *argp
Write one bit or one byte to the process image [...].  This call is more efficient than the usual calls of seek and write because only one function call is necessary. If more than on application are writing bits in one output byte, this call is the only safe way to set a bit without overwriting the other bits because this call is doing a read-modify-write-cycle. 

The struct SPIValue used by this ioctl is defined as
typedef struct SPIValueStr
{
    uint16_t i16uAddress; // Address of the byte in the process image
    uint8_t i8uBit; // 0-7 bit position, >= 8 whole byte
    uint8_t i8uValue; // Value: 0/1 for bit access, whole byte otherwise
} SPIValue;
\end{lstlisting} 

Die oben beschriebenden Funtkionen \lstinline{KB_FIND_VARIABLE}, \lstinline{KB_GET_VALUE} und \lstinline{KB_SET_VALUE} ermöglichen einen einfachen und (lt.\,Manpage) effizienten Zugriff auf einzelne Bits des Prozessabbildes und damit auch auf die IO des RevolutionPi.
Der Zugriff des OPC-Servers auf das Prozessabbild soll daher mittels dieser Funktionen realisiert werden.
\lstinline{KB_FIND_VARIABLE} kann genutzt werden, um Adressen von Variablen im Prozessabbild mittels ihres Namens aufzulösen.
\lstinline{KB_GET_VALUE} und \lstinline{KB_SET_VALUE} ermöglichen den Zugriff auf die Werte dieser Variablen.


\subsection{Integration des OPC-Servers in das System%
     \label{sec:3-integration}}

open62541 bietet drei Möglichkeiten zum Abgleich von Variablen mit dem Prozessabbild~\citep[vgl.][Tutorials - Connecting a Variable with a Physical Process]{web-open62541}:
\begin{itemize}
    \item Manuelles oder zyklisches Aktualisieren
    \item Variable Value Callback
    \item Variable Datasource
\end{itemize}

Die zyklische Aktualisierung eines oder mehrerer Werte nimmt, abhängig von der Zykluszeit, viele Systemressourcen in Anspruch. Value Callbacks ermöglichen es, einen Variablenwert effizienter mit einer Ressource wie etwa einem Prozessabbild zu synchronisieren. An die Variable wird ein Callback angehängt, welches vor jedem Lesen und nach jedem Schreibvorgang ausgeführt wird.
Der Wert der Variablen wird weiterhin im Variablenknoten auf dem OPC-Server gespeichert, der Abgleich mit der verknüpften Ressource erfolgt durch die Callback-Methoden.

Sogenannte Datenquellen gehen noch einen Schritt weiter. Der Server leitet jede Lese- und Schreibanforderung direkt an eine Callback-Funktion weiter. Beim Lesen liefert der Rückruf eine Kopie des aktuellen Wertes. Die Datenquelle muss intern ein eigenes Speichermanagement implementieren.

Der Zugriff auf die Werte des Prozessabbildes erfolgt, wie in Abschnitt~\ref{sec:3-anbindung} beschrieben, über von piControl bereitgestellte Methoden. Um die durch open62541 gepflegte OPC-Datenstruktur und das durch piControl verwaltete Prozessabbild möglichst effektiv verknüpfen zu können, soll diese Interaktion mittels Datenquellen und den zugehörigen Callbacks implementiert werden.
% % % Imports nur für Referenzenauflösung während des Schreibens! Vorm Kompilieren auskommentieren!
% \bibliography{0_hauptdatei}
% \input{1_einleitung}
% \input{2_grundlagen}
% \input{3_konzeption}
% \input{4_implementierung}
% \input{5_tests}
% \input{6_zusammenfassung}
% \input{anhang}
% % Ende Imports

\section{Implementierung%
  \label{sec:4-implementierung}}
Das folgende Kapitel stellt in Auszügen die Implementierung des OPC-Servers sowie die Anbindung an die IO-Module
der SPS dar. Der Schwerpunkt liegt hierbei auf der Funktionsweise des piControl-Treibers und dessen Integration in das Projekt. Abschnitt~\ref{sec:4-picontrol} erklärt die zum Schreibens eines Bits verwendeten Funktionsaufrufe.
Zuvor soll jedoch in Abschnitt~\ref{sec:4-open62541} der Teil des OPC-Servers vorgestellt werden, welcher auf besagten Treiber zugreift. 

\subsection{Implementierung des OPC-Servers%
     \label{sec:4-open62541}}
Wie im vorangegangenen Abschnitt~\ref{sec:3-integration} begründet, soll die Verknüpfung zwischen dem Prozessabbild der SPS und den auf dem OPC-Server bereitgestellten Werten über sog.\,Datenquellen erfolgen. Hierzu ist zunächst eine Callback-Methode zu implementieren, welche bei einem Lese- oder Schreibzugriff auf eine Variable aufgerufen wird. Die Verknüpfung zwischen Callback-Methode und Variable muss manuell erfolgen.

\begin{lstlisting}[language={c},firstnumber=237,caption={Auszug der Methode \lstinline{linkDataSourceVariable} in \lstinline{variables.c}\label{lst:4-linkDataSourceVariable}}]
extern UA_StatusCode
 linkDataSourceVariable(UA_Server *server, UA_NodeId nodeId) {
     bool readonly = false;
     UA_DataSource dataSourceVariable;
     UA_StatusCode rc; |>\setcounter{lstnumber}{254}<|

     dataSourceVariable.read = readDataSourceVariable;
     if (!readonly)
        dataSourceVariable.write = writeDataSourceVariable;
     else
        dataSourceVariable.write = writeReadonlyDataSourceVariable;

     return UA_Server_setVariableNode_dataSource(server, nodeId, dataSourceVariable);
 }
\end{lstlisting}

\begin{figure}[h]
    \centering
    \includegraphics[width=0.42\textwidth]{doc/img/OPC_RevPiDO.pdf}
    \caption{Auszug des verwendeten Nodesets, hier Digitalausgang 1 des Versuchsaufbaus
      \label{fig:opc-do}}
\end{figure}

Die in Listing~\ref{lst:4-linkDataSourceVariable} abgebildete Methode \lstinline{linkDataSourceVariable()} erzeugt ein Struct vom Typ \lstinline{UA_DataSource}. In diesem werden dem Lesen und Schreiben einer OPC-Variablen entsprechende Callback-Methoden zugewiesen. Die Verknüpfung einer OPC-Variable, genauer ihrer NodeId, mit der zuvor definierten Datenquelle erfolgt über die von open62541 bereitgestellte Methode \lstinline{UA_Server_setVariableNode_dataSource()}. Vor dem Lesen und nach dem Schreiben dieser Variable werden von nun an die entsprechenden Callbacks aufgerufen.
     
\begin{lstlisting}[language={c},firstnumber=168,caption={Auszug des Callbacks \lstinline{writeDataSourceVariable} in \lstinline{variables.c}\label{lst:4-writeDataSourceVariable}}]  
extern UA_StatusCode
 writeDataSourceVariable(UA_Server *server,
            const UA_NodeId *sessionId, void *sessionContext,
            const UA_NodeId *nodeId, void *nodeContext,
            const UA_NumericRange *range, const UA_DataValue *dataValue) {

    UA_StatusCode retval  = UA_STATUSCODE_GOOD;
    UA_NodeId *nameNodeId = UA_malloc(sizeof(UA_NodeId));
    UA_QualifiedName nameQN = UA_QUALIFIEDNAME(1, "Name");
    UA_Variant nameVar;
    UA_Boolean bit;

    retval |= findSiblingByBrowsename(server, nodeId, &nameQN, nameNodeId);
    retval |= UA_Server_readValue(server, *nameNodeId, &nameVar);
    retval |= UA_Boolean_copy(dataValue->value.data, &bit);

    |>\tikzmarkin[set border color=martinired]{writeIO}<|PI_writeSingleIO(String_fromUA_String(nameVar.data), &bit, false);                                                 |>\tikzmarkend{writeIO}<|

    free(nameNodeId);
    return retval;
 }
\end{lstlisting}

Listing~\ref{lst:4-writeDataSourceVariable} zeigt die Callback-Methode, welche nach dem Schreiben einer Variablen auf dem OPC-Server aufgerufen wird.
Dieser Methode wird neben der NodeId der mit ihr verknüpften Variablen auch der Wert dieser in Form eines Zeigers auf ein Struct vom Typ \lstinline{UA_DataValue} übergeben.

Die Gestaltung des hier verwendeten Nodesets sieht vor, dass in einer OPC-Variablen \lstinline{"Name"} der Bezeichner des zu schreibenden Digitalausgangs hinterlegt ist, siehe Abbildung~\ref{fig:opc-do}. Dies erlaubt eine Rekonfiguration der Ein- und Ausgänge der SPS ohne Änderungen im Programmcode des OPC-Servers vornehmen zu müssen.
Es ist daher erforderlich, nach jedem Schreiben einer mit einem Digitalausgang verknüpften Variablen, hier \lstinline{"Value"}, dessen Bezeichner \lstinline{"Name"} abzufragen. 
Dies geschieht in den Zeilen 180 und 181.
Anschließend wird dieser Bezeichner sowie der zu schreibende Wert der Methode \lstinline{PI_writeSingleIO()} übergeben, welche wiederum die Interaktion mit piControl übernimmt (vgl. Abschnitt \ref{sec:4-picontrol}).
 
\subsection{Integration von piControl%
     \label{sec:4-picontrol}}
In Abschnitt~\ref{sec:2-io} wurde die Anbindung der IO-Module des Revolution Pi sowie die Funktionsweise von piControl aus Anwendersicht beschrieben. Die verfügbare Literatur beschränkt sich auch auf lediglich diese Sicht; eine weiterführende Dokumentation für Entwickler gibt es, neben der in Abschnitt~\ref{sec:3-anbindung} vorgestellten Manpage, nicht. 
In diesem Abschnitt soll daher der Quellcode von piControl sowie dessen Verwendung im Projekt genauer betrachtet werden.
Hierzu wird exemplarisch die in Abschnitt~\ref{sec:4-open62541} eingeführte Methode \lstinline{PI_writeSingleIO()} untersucht.
Diese Methode ermöglicht das Setzen eines einzelnen Bits im Prozessabbild der SPS, und damit das Schalten eines digitalen Ausgangs auf einem IO-Modul.
Die äquivalente Methode \lstinline{int piControlGetBitValue(SPIValue *pSpiValue)} zum Lesen eines Bits bzw. Eingangs funktioniert analog und soll daher an dieser Stelle nicht dediziert erörtert werden.

\begin{lstlisting}[language={c},firstnumber=97,
                   caption={Setzen eines phsikalischen, digitalen Ausgangs in \lstinline{revpi.c}
                   \label{lst:4-PI_writeSingleIO}}]
extern void PI_writeSingleIO(char *pszVariableName, bool *bit, bool verbose)
{
	int rc;
	SPIVariable sPiVariable;
	SPIValue sPIValue;

	strncpy(sPiVariable.strVarName, pszVariableName, sizeof(sPiVariable.strVarName));
	rc = piControlGetVariableInfo(&sPiVariable);
	if (rc < 0) {
		printf("Cannot find variable '%s'\n", pszVariableName);
		return;
	}

		sPIValue.i16uAddress = sPiVariable.i16uAddress;
		sPIValue.i8uBit = sPiVariable.i8uBit;
		sPIValue.i8uValue = *bit;
		rc = |>\tikzmarkin[set border color=martinired]{setBitValue}<|piControlSetBitValue(&sPIValue)|>\tikzmarkend{setBitValue}<|;
		if (rc < 0)
			printf("Set bit error %s\n", getWriteError(rc));
		else if (verbose)
			printf("Set bit %d on byte at offset %d. Value %d\n", sPIValue.i8uBit, sPIValue.i16uAddress,
			       sPIValue.i8uValue);
}
\end{lstlisting}

Der Programmcode in Listing~\ref{lst:4-PI_writeSingleIO} ist Teil des implementierten OPC-Servers. In diesem wird auf zwei Funktionen des piControl-Treibers zugegriffen. 
Beiden Methoden wird als Argument ein Zeiger auf ein Struct vom Typ \lstinline{SPIValue} übergeben. Der im Struct abgelegte Name wird mittels \lstinline{piControlGetVariableInfo(&sPIValue)} zu einer Adresse im Prozessabbild aufgelöst. Diese wird in \lstinline{sPIValue.i16uAdress} gespeichert. Der Wert der Variablen wird anschließend mittels \lstinline{piControlSetBitValue(&sPIValue)} an dieser Adresse in das Prozessabbild geschrieben.

\begin{lstlisting}[language={c},firstnumber=309,caption={Methode \lstinline{piControlSetBitValue} in \lstinline{piControlIf.c}\label{lst:4-piControlSetBitValue}}]
int |>\tikzmarkin[set border color=martiniblue]{setBitValueFcn}<|piControlSetBitValue(SPIValue *pSpiValue)|>\tikzmarkend{setBitValueFcn}<|
{
    piControlOpen();

    if (PiControlHandle_g < 0)
	    return -ENODEV;

    pSpiValue->i16uAddress += pSpiValue->i8uBit / 8;
    pSpiValue->i8uBit %= 8;

    if (|>\tikzmarkin[set border color=martinired]{ioctl}<|ioctl(PiControlHandle_g, KB_SET_VALUE, pSpiValue)|>\tikzmarkend{ioctl}<| < 0)
	    return errno;

    return 0;
}
\end{lstlisting}

Die in Listing~\ref{lst:4-piControlSetBitValue} dargestellte Methode \lstinline{piControlSetBitValue} ist lediglich eine Hüllfunktion (häufig auch als Wrapper-Funktion bezeichnet) für einen Aufruf des \lstinline{ioctl} Kernel-Moduls.
Folgende Parameter werden übergeben:
\lstinline{PiControlHandle_g} ist die Referenz auf die Geräte-Datei des piControl-Treibers. \lstinline{KB_SET_VALUE} ist das ioctl-Kommando zum Schreiben eines Bits in das Prozessabbild. Der Zeiger \lstinline{pSpiValue} verweist auf ein Struct des bereits vorgestellten Typs \lstinline{SPIValue}.

\begin{lstlisting}[language={c},firstnumber=80,caption={Methode \lstinline{piControlOpen} in \lstinline{piControlIf.c}\label{lst:4-piControlOpen}}]
void piControlOpen(void)
{
    /* open handle if needed */
    if (PiControlHandle_g < 0)
    {
	    |>\tikzmarkin[set border color=martiniblue]{PiControlHandle}<|PiControlHandle_g = open(PICONTROL_DEVICE, O_RDWR)|>\tikzmarkend{PiControlHandle}<|;
    }
}
\end{lstlisting}

Die in Listing~\ref{lst:4-piControlOpen} dargestellte Methode öffnet, sofern nicht bereits geschehen, die Geräte-Datei. Das Macro \lstinline{PICONTROL_DEVICE} verweist hierbei auf \lstinline{/dev/piControl0}.

\begin{lstlisting}[language={c},firstnumber=721,caption={Methode \lstinline{piControlIoctl} in \lstinline{piControlMain.c}\label{lst:4-piControlIoctl}}]
static long |>\tikzmarkin[set border color=martiniblue, below offset=0.9em]{piControlIoctl}<|piControlIoctl(struct file *file, unsigned int prg_nr, 
                           unsigned long usr_addr)                                      |>\tikzmarkend{piControlIoctl}<|
{
  int status = -EFAULT;
  tpiControlInst *priv;
  int timeout = 10000;	// ms

  if (prg_nr != KB_CONFIG_SEND && prg_nr != KB_CONFIG_START && !isRunning()) {
  	return -EAGAIN;
  }

  priv = (tpiControlInst *) file->private_data;

  if (prg_nr != KB_GET_LAST_MESSAGE) {
  	// clear old message
  	priv->pcErrorMessage[0] = 0;
  }

  switch (prg_nr) {|>\setcounter{lstnumber}{864}<|

    case |>\tikzmarkin[set border color=martiniblue]{KB_SET_VALUE}<|KB_SET_VALUE:|>\tikzmarkend{KB_SET_VALUE}<|
  		{
  			SPIValue *pValue = (SPIValue *) usr_addr;

  			if (!isRunning())
  				return -EFAULT;

  			if (pValue->i16uAddress >= KB_PI_LEN) {
  				status = -EFAULT;
  			} else {
  				INT8U i8uValue_l;
  				my_rt_mutex_lock(&piDev_g.lockPI);
  				i8uValue_l = piDev_g.ai8uPI[pValue->i16uAddress];

  				if (pValue->i8uBit >= 8) {
  					i8uValue_l = pValue->i8uValue;
  				} else {
  					if (pValue->i8uValue)
  						i8uValue_l |= (1 << pValue->i8uBit);
  					else
  						i8uValue_l &= ~(1 << pValue->i8uBit);
  				}

  				|>\tikzmarkin[set border color=martinired]{i8uValue}<|piDev_g.ai8uPI[pValue->i16uAddress] = i8uValue_l;|>\tikzmarkend{i8uValue}<|
  				rt_mutex_unlock(&piDev_g.lockPI);

  #ifdef VERBOSE
  				pr_info("piControlIoctl Addr=%u, bit=%u: %02x %02x\n", pValue->i16uAddress, pValue->i8uBit, pValue->i8uValue, i8uValue_l);
  #endif

  				status = 0;
  			}
  		}
  		break; |>\setcounter{lstnumber}{1314}<|

    default:
      pr_err("Invalid Ioctl");
      return (-EINVAL);
      break;

    }

    return status;
  }
\end{lstlisting}

Listing~\ref{lst:4-piControlIoctl} zeigt in Auszügen die ioctl-Methode des piControl Kernel-Treibers. Diese bekommt folgende Argumente übergeben: \lstinline{struct file *file} enthält den Verweis auf die Geräte-Datei, hier \lstinline{/dev/piControl0}. Der Wert von \lstinline{unsigned int prg_nr} beschreibt die Anfrage an den Treiber, in diesem Fall \lstinline{KB_SET_VALUE}. Das Argument \lstinline{unsigned long usr_addr} enthält einen typ-agnostischen Pointer. Dieser verweist auf einen Speicherbereich, in welchem die zur Bearbeitung der Anfrage notwendigen Daten abgelegt sind. Hier können auch vom Treiber empfangene Daten dem Anwendungsprogramm bereitgestellt werden. 

Die switch-case-Anweisung führt die über das Argument \lstinline{prg_nr} spezifizierte Aktion aus. Hier betrachten wir \lstinline{KB_SET_VALUE}:
Zunächst wird in Zeile 868 der übergebene Zeiger \lstinline{usr_addr} mittels explizitem Typecast zu einem Zeiger des Typs \lstinline{SPIValue *} konvertiert. Da dieser auf Daten im Userspace verweist, ist beim Zugriff durch den Kernel-Treiber besondere Vorsicht geboten.
In Zeile 877 wird mittels Mutex das Prozessabbild \lstinline{piDev_g} für den Zugriff durch andere Threads oder Prozesse gesperrt.
\lstinline{my_rt_mutex_lock} verweist hierbei auf die Funktion \lstinline{rt_mutex_lock} aus \lstinline{linux/sched.h}\footnote{Offenbar wurde hier auch eine alternative Implementierung vorgesehen, siehe revpi\_common.h}

In Zeile 889 wird das Byte \lstinline{i8uValue_l}, welches den zu schreibenden Wert enthält in das Prozessabbild übertragen. Anschließend wird die Mutex auf \lstinline{piDev_g} wieder entsperrt.
\newpage

\begin{lstlisting}[language={c},firstnumber=62,caption={Auszug des Struct \lstinline{spiControlDev} in \lstinline{piControlMain.h}\label{lst:4-spiControlDev}}]
|>\tikzmarkin[set border color=martiniblue]{spiControlDev}<|typedef struct spiControlDev|>\tikzmarkend{spiControlDev}<| {
	// device driver stuff
	int init_step;
	enum revpi_machine machine_type;
	void *machine;
	struct cdev cdev;	// Char device structure
	struct device *dev;
	struct thermal_zone_device *thermal_zone;

	|>\tikzmarkin[set border color=martiniblue]{processImage}<|// process image stuff
	INT8U ai8uPI[KB_PI_LEN];
	INT8U ai8uPIDefault|>\tikzmarkin[set border color=martinired]{KB_PI_LEN_0}<|[KB_PI_LEN]|>\tikzmarkend{KB_PI_LEN_0}<|;
	struct rt_mutex lockPI;        |>\tikzmarkend{processImage}<|
	bool stopIO;
	piDevices *devs; |>\setcounter{lstnumber}{94}<|
} tpiControlDev;
\end{lstlisting}

Das Prozessabbild ist als Byte-Array der Länge \lstinline{KB_PI_LEN} in Listing~\ref{lst:4-spiControlDev} definiert. Konfigurationsparameter wie \lstinline{KB_PI_LEN} oder die Zykluszeit für den Datenaustausch zwischen SPS und IO-Modulen sind im folgenden Listing~\ref{lst:4-process} definiert.

\begin{lstlisting}[language={c},firstnumber=119,caption={Konfigurationsparameter des Prozessabbildes in project.h\label{lst:4-process}}]
#define INTERVAL_PI_GATE (5*1000*1000)  // 5 ms piGateCommunication |>\setcounter{lstnumber}{128}<|

#define INTERVAL_IO_COM (5*1000*1000)  // 5 ms piIoComm |>\setcounter{lstnumber}{132}<|

#define KB_PD_LEN       512
|>\tikzmarkin[set border color=martiniblue]{KB_PI_LEN_1}<|#define KB_PI_LEN       4096|>\tikzmarkend{KB_PI_LEN_1}<|
\end{lstlisting}

Das zu setzende Bit wurde zu diesem Zeitpunkt erfolgreich in das Prozessabbild der SPS geschrieben.
Es stellt sich die Frage, wie dieses nun an das IO-Modul kommuniziert wird.
Die Kommunikation mit allen angebundenen Modulen ist ebenfalls Aufgabe des piControl-Treibers.

\begin{lstlisting}[language={c},firstnumber=256,caption={Auszug der Methode \lstinline{piIoThread} in \lstinline{revpi_core.c}\label{lst:4-piIoThread}}]
static int piIoThread(void *data)
{
	//TODO int value = 0;
	ktime_t time;
	ktime_t now;
	s64 tDiff;

	hrtimer_init(&piCore_g.ioTimer, CLOCK_MONOTONIC, HRTIMER_MODE_ABS);
	piCore_g.ioTimer.function = piIoTimer;

	pr_info("piIO thread started\n");

	now = hrtimer_cb_get_time(&piCore_g.ioTimer);

	PiBridgeMaster_Reset();

	while (!kthread_should_stop()) {
		if (|>\tikzmarkin[set border color=martinired]{PiBridgeMaster}<|PiBridgeMaster_Run()|>\tikzmarkend{PiBridgeMaster}<| < 0)
			break;
	}

	RevPiDevice_finish();

	pr_info("piIO exit\n");
	return 0;
}
\end{lstlisting}

Der Kernel-Thread \lstinline{piIoThread} ist verantwortlich für den zyklischen Datenaustausch mit den IO-Modulen. In diesem wird fortlaufend die Methode \lstinline{PiBridgeMaster_Run()} aufgerufen, siehe Listing~\ref{lst:4-piIoThread}.

\begin{lstlisting}[language={c},firstnumber=262,caption={Auszug der Methode \lstinline{PiBridgeMaster_Run(void)} in \lstinline{RevPiDevice.c}\label{lst:4-PiBridgeMaster_Run}}]
int PiBridgeMaster_Run(void)
{
	static kbUT_Timer tTimeoutTimer_s;
	static kbUT_Timer tConfigTimeoutTimer_s;
	static int error_cnt;
	static INT8U last_led;
	static unsigned long last_update;
	int ret = 0;
	int i;

	my_rt_mutex_lock(&piCore_g.lockBridgeState);
	if (piCore_g.eBridgeState != piBridgeStop) {
		switch (eRunStatus_s) { |>\setcounter{lstnumber}{514}<|
		    case enPiBridgeMasterStatus_EndOfConfig:|>\setcounter{lstnumber}{621}<|
		    if (|>\tikzmarkin[set border color=martinired]{RevPiDevice}<|RevPiDevice_run()|>\tikzmarkend{RevPiDevice}<|) {
				// an error occured, check error limits |>\setcounter{lstnumber}{641}<|
			} else {
				ret = 1;
			}
			piCore_g.image.drv.i16uRS485ErrorCnt = RevPiDevice_getErrCnt();
			break;
\end{lstlisting}

Die in Listing~\ref{lst:4-PiBridgeMaster_Run} dargestellte Methode ist eine sog. State-Machine. Ist die Konfiguration der IO-Module erfolgreich abgeschlossen, so führt sie bei Aufruf lediglich die Methode \lstinline{RevPiDevice_run()} aus.

\begin{lstlisting}[language={c},firstnumber=140,caption={Auszug der Methode \lstinline{RevPiDevice_run(void)} in \lstinline{RevPiDevice.c}\label{lst:4-RevPiDevice_run}}]
int RevPiDevice_run(void)
{
	INT8U i8uDevice = 0;
	INT32U r;
	int retval = 0;

	RevPiDevices_s.i16uErrorCnt = 0;

	for (i8uDevice = 0; i8uDevice < RevPiDevice_getDevCnt(); i8uDevice++) {
		if (RevPiDevice_getDev(i8uDevice)->i8uActive) {
			switch (RevPiDevice_getDev(i8uDevice)->sId.i16uModulType) {
			case KUNBUS_FW_DESCR_TYP_PI_DIO_14:
			case KUNBUS_FW_DESCR_TYP_PI_DI_16:
			case KUNBUS_FW_DESCR_TYP_PI_DO_16:
				r = |>\tikzmarkin[set border color=martinired]{sendCyclicTelegram}<|piDIOComm_sendCyclicTelegram(i8uDevice)|>\tikzmarkend{sendCyclicTelegram}\setcounter{lstnumber}{166} <|;

				break; |>\setcounter{lstnumber}{216}<|
			}
		}
	} |>\setcounter{lstnumber}{227}<|
	return retval;
}
\end{lstlisting}

Diese iteriert wie in Listing~\ref{lst:4-RevPiDevice_run} abgebildete durch alle gegenwärtig in der SPS konfigurierten Module. Ist das aktuelle Modul als aktiv markiert, so wird anhand eines sog. Firmware-Descriptors entschieden, welche Methode für die Ansteuerung des Moduls aufzurufen ist.

\begin{lstlisting}[language={c},firstnumber=161,caption={Auszug der Methode \lstinline{piDIOComm_sendCyclicTelegram} in \lstinline{piDIOComm.c}\label{lst:4-sendCyclicTelegram}}]
INT32U piDIOComm_sendCyclicTelegram(INT8U i8uDevice_p)
{
	INT32U i32uRv_l = 0;
	SIOGeneric sRequest_l;
	SIOGeneric sResponse_l;
	INT8U len_l, data_out[18], i, p, data_in[70];
	INT8U i8uAddress;
	int ret; |>\setcounter{lstnumber}{239}<|
	
    |>\tikzmarkin[set border color=martinired]{piIoComm}<|ret = piIoComm_send((INT8U *) & sRequest_l, IOPROTOCOL_HEADER_LENGTH + len_l + 1);  |>\tikzmarkend{piIoComm}\setcounter{lstnumber}{298}<|
}
\end{lstlisting}

Im Falle des hier verwendeten DO-Moduls wird die in Listing~\ref{lst:4-sendCyclicTelegram} abgebildete Methode \lstinline{piDIOComm_sendCyclicTelegram()} aufgerufen. Dieser wird ein Zeiger auf das zu schreibende Gerät übergeben. 
Zunächst wird das Prozessabbild mittels eines proprietären, jedoch im Quellcode offen nachvollziehbaren Protokolls in ein \lstinline{sRequest_l} genanntes Byte-Array umgewandelt. Dieser Schritt ist in Listing~\ref{lst:4-sendCyclicTelegram} nicht abgebildet. Anschließend wird \lstinline{piIoComm_send()} ein Zeiger auf die so generierte Schreib-Anfrage übergeben.

\begin{lstlisting}[language={c},firstnumber=220,caption={Auszug der Methode \lstinline{piIOComm_send} in \lstinline{piIOComm.c}\label{lst:4-piIOComm_send}}]
int piIoComm_send(INT8U * buf_p, INT16U i16uLen_p)
{
	ssize_t write_l = 0;
	INT16U i16uSent_l = 0;|>\setcounter{lstnumber}{249}<|

	while (i16uSent_l < i16uLen_p) {
		write_l = vfs_write(piIoComm_fd_m, buf_p + i16uSent_l, i16uLen_p - i16uSent_l, &piIoComm_fd_m->f_pos);
		if (write_l < 0) {
			pr_info_serial("write error %d\n", (int)write_l);
			return -1;
		} 
		i16uSent_l += write_l;|>\setcounter{lstnumber}{263}<|
	}
	clear();
	vfs_fsync(piIoComm_fd_m, 1);
	return 0;
}
\end{lstlisting}

Listing~\ref{lst:4-piIOComm_send} zeigt die Implementierung von \lstinline{piIoComm_send()}. Diese Methode ist für das Schreiben der oben generierten Anfrage auf die seriellen Schnittstelle verantwortlich. Realisiert wird dies mittels der Methode \lstinline{vfs_write()}. Diese ist in \lstinline{<linux/fs.h>} definiert. Sie ermöglicht das Schreiben einer Datei im Userspace aus dem Kernel heraus. Geschrieben wird hier die Datei mit dem Deskriptor \lstinline{piIoComm_fd_m}.
Da die Funktion \lstinline{vfs_write()} durch andere Kernel-Tasks unterbrochen werden kann, ist nicht gewährleistet, dass die gesamte Anfrage mit nur einem Aufruf geschrieben wird. Die oben abgebildete while-Schleife stellt das vollständige Senden der Anfrage sicher.

\begin{lstlisting}[language={c},firstnumber=157,caption={Auszug der Methode \lstinline{piIOComm_open_serial} in \lstinline{piIOComm.c}\label{lst:4-piIOComm_open_serial}}]
int piIoComm_open_serial(void)
{   |>\setcounter{lstnumber}{167}<|
	struct file *fd;	/* Filedeskriptor */
	struct termios newtio;	/* Schnittstellenoptionen */

	|>\tikzmarkin[set border color=martiniblue]{fd}<|/* Port oeffnen - read/write, kein "controlling tty", 
	    Status von DCD ignorieren */
	fd = filp_open(|>\tikzmarkin[set border color=martinired]{tty}<|REV_PI_TTY_DEVICE|>\tikzmarkend{tty}<|, O_RDWR | O_NOCTTY, 0); |>\setcounter{lstnumber}{208}<|
	
	piIoComm_fd_m = fd;                                                      |>\tikzmarkend{fd}\setcounter{lstnumber}{217}<|

	return 0;
}
\end{lstlisting}

Der zum Schreiben auf die serielle Schnittstelle verwendete Datei-Deskriptor wird von der in Listing~\ref{lst:4-piIOComm_open_serial} abgebildeten Methode \lstinline{piIoComm_open_serial()} generiert. 

\begin{lstlisting}[language={c},firstnumber=45,caption={Definition der seriellen Schnittstelle in \lstinline{piIOComm.h}\label{lst:4-REV_PI_TTY_DEVICE}}]
#define REV_PI_TTY_DEVICE	"/dev/ttyAMA0"
\end{lstlisting}

Das in Listing~\ref{lst:4-REV_PI_TTY_DEVICE} definierte Macro verweist auf eine der seriellen Schnittstellen des RaspberryPi.
Die Implementierung des zugehörigen Schnittstellentreibers soll hier nicht weiter untersucht werden. Somit ist an dieser Stelle die Kette vom Setzen einer Variablen auf dem OPC-Server bis hin zur Aktualisierung des Prozessabbilds der IO-Module geschlossen.

% \begin{lstlisting}[language={c},firstnumber={226},caption={Setzen der Scheduler-Priorität auf SCHED\_FIFO in 
% revpi\_common.c\label{lst:2-sched_priority}}]
% param.sched_priority = ktprio->prio;
% ret = sched_setscheduler(child, SCHED_FIFO, &param);
% \end{lstlisting}
% % % Imports nur für Referenzenauflösung während des Schreibens! Vorm Kompilieren auskommentieren!
% \bibliography{0_hauptdatei}
% \input{1_einleitung}
% \input{2_grundlagen}
% \input{3_konzeption}
% \input{4_implementierung}
% \input{5_tests}
% \input{6_zusammenfassung}
% % Ende Imports

\section{Test des OPC-Servers im Gesamtsystem%
  \label{sec:5-tests}}

% % % Imports nur für Referenzenauflösung während des schreibens! Vorm Kompilieren auskommentieren!
% \bibliography{0_hauptdatei}
% \input{1_einleitung}
% \input{2_grundlagen}
% \input{3_konzeption}
% \input{4_implementierung}
% \input{5_tests}
% \input{6_zusammenfassung}
% % Ende Imports

\section{Zusammenfassung und Ausblick%
  \label{sec:6-fazit}}
Der folgende Abschnitt~\ref{sec:6-zusammenfassung} fasst die gewonnenen Erkenntnisse und den Stand der Implementierung zusammen.
Den Abschluss dieser Arbeit bildet der Ausblick in Abschnitt~\ref{sec:6-ausblick}.

\subsection{Zusammenfassung%
     \label{sec:6-zusammenfassung}}

\subsection{Ausblick%
     \label{sec:6-ausblick}}

% % Ende Imports

\section{Test des OPC-Servers im Gesamtsystem%
  \label{sec:5-tests}}

% % % Imports nur für Referenzenauflösung während des schreibens! Vorm Kompilieren auskommentieren!
% \bibliography{0_hauptdatei}
% % Mit \section{...} eröffnen wir einen neuen Abschnitt.
% Der Befehl setzt nicht nur den Text in einer größeren,
% fetten Schrift, sondern sorgt außerdem dafür, daß er im
% Inhaltsverzeichnis erscheint.
%
% Mit \label{...} erzeugen wir einen Bezeichner, mit dessen Hilfe
% wir später auf die Nummer des Abschnitts verweisen können (nämlich
% mit~\ref{...}).
%
% Das Kommentarzeichen hinter „Übersicht“ dient dazu, ein
% Leerzeichen zwischen „Übersicht“ und dem \label-Befehl
% zu vermeiden, das andernfalls sichtbar würde – z.B. im
% Inhaltsverzeichnis.
%

% % Imports nur für Referenzenauflösung während des Schreibens! Vorm Kompilieren auskommentieren!
% \bibliography{0_hauptdatei}
% \input{1_einleitung}
%\input{2_grundlagen}
%\input{3_konzeption}
%\input{4_implementierung}
%\input{5_tests}
%\input{6_zusammenfassung}
% % Ende Imports

\section{Einleitung und Motivation%
  \label{sec:1-einleitung}}
Ziel dieses Projektes ist die Integration eines OPC-Servers mit einer auf Linux
basierenden speicherprogrammierbaren Steuerung (SPS). Angeschlossen an diese SPS
ist jeweils ein digitales Ein-/\,bzw.~Ausgabemodul. Die von diesen bereitgestellten
Ein-/\, bzw.~Ausgänge (IO) sollen in der Datenstruktur des OPC-Servers abgebildet
und über diesen für OPC-Clients les-/\,und schreibar sein. Weiterhin sollen einige
Funktionen zur Überwachung und Steuerung der an die SPS angeschlossenen Aktoren
und Sensoren direkt im OPC-Server implementiert werden.
Hiermit stellt dieses Projekt eine der Grundlagen für ein übergeordnetes Projekt,
die cloudbasierte Steuerung eines miniaturisierten Produktions-Systems, dar.

Der hier verwendete OPC-Server ist Teil des sog. open62541 Projekts. Er ist in C
geschrieben und implementiert bereits einen großen Teil der im OPC-UA-Standard
spezifizierten Funktionen.
Als SPS findet ein Revolution Pi 3 der Firma Kunbus Verwendung. Dieser integriert
ein sog. Compute Module der Raspberry Pi Foundation in ein industrietaugliches
Gehäuse und erlaubt die Erweiterung mittels IO- oder Gateway-Modulen. Über diese
erfolgt die Kommunikation mit weiteren Komponenten der Automatisierungstechnik.

Motiviert ist dieses Projekt durch die Beobachtung, dass die Verbreitung offener
Standards sowie freier Software auch in der Automatisierungstechnik zunimmt.
Linux ist ein freies Betriebssystem, OPC-UA ein offen zugänglicher, aktiv gepflegter
und weit verbreiteter Standard. Der Raspberry Pi findet sowohl bei Hobby-Anwendern als
auch in den Bereichen Forschung und Entwicklung sowie bei industriellen Anwendern
Verwendung. Dieses Projekt stellt somit eine für unterschiedliche Anwender interessante
Entwicklung dar.

Im Anschluss an diese einleitende Übersicht im Abschnitt~\ref{sec:1-einleitung} folgt
die Darstellung der wichtigsten Grundlagen in Abschnitt~\ref{sec:2-grundlagen}.
Aufbauend auf diesen Grundlagen folgt die konzeptuelle Ausarbeitung im Abschnitt~\ref{sec:3-konzeption}.
Die Umsetzung wird im Abschnitt~\ref{sec:4-implementierung} erläutert.
Die Leistungsfähigkeit der Implementierung wird in Abschnitt~\ref{sec:5-tests} untersucht.
Eine Zusammenfassung und ein Ausblick schließen die Arbeit in
Abschnitt~\ref{sec:6-fazit} ab. Eventuell noch benötigte Anhänge
finden sich in den Anhängen [...] bis [...].

% % % Imports nur für Referenzenauflösung während des Schreibens! Vorm Kompilieren auskommentieren!
% \bibliography{0_hauptdatei}
% \input{1_einleitung}
% \input{2_grundlagen}
% \input{3_konzeption}
% \input{4_implementierung}
% \input{5_tests}
% \input{6_zusammenfassung}
% % Ende Imports

\section{Grundlagen%
  \label{sec:2-grundlagen}}

\subsection{Speicherprogrammierbare-Steuerung und Linux -- Revolution Pi%
     \label{sec:2-sps}}

\subsubsection{Kunbus RevolutionPi%
        \label{sec:2-revpi}}
Der RevolutionPi 3 ist eine speicherprogrammierbare Steuerung (SPS) des Herstellers
Kunbus GmbH. Kern dieser SPS ist das von der Raspberry Pi Foundation entwickelte
und vertriebene Raspberry Pi Compute Module 3. Dieses integriert ein Broadcom BCM2837
System-on-Chip (SoC) mit vier 1,2GHz Prozessorkernen, 1GB RAM, 4GB eMMC Anwendungsspeicher
und sonstige Peripherie in ein Modul im DDR2-SODIMM Formfaktor. Diese Spezifikationen
sind weitgehend identisch zu denen des ausgesprochen populären Raspberry Pi 3.
Der Revolution Pi profitiert daher von dem gleichen großen Angebot an Software
und Unterstützung wie der Raspberry Pi, ergänzt dessen Hardware jedoch um eine 24V
Spannungsversorgung, die Möglichkeit der Erweiterung durch mehrere industrietaugliche
Ein-/ Ausgabemodule und Gateways sowie ein Gehäuse zur Montage auf einer DIN-Schiene.
\begin{itemize}
  \item{Prozessor: BCM2837}
  \item{Taktfrequenz 1,2 GHz}
  \item{Anzahl Prozessorkerne: 4}
  \item{Arbeitsspeicher: 1 GByte}
  \item{eMMC Flash Speicher: 4 GByte}
  \item{Betriebssystem: Angepasstes Raspbian mit RT-Patch}
  \item{RTC mit 24h Pufferung über wartungsfreien Kondensator}
  \item{Treiber / API: Treiber schreibt zyklisch Prozessdaten in ein Prozessabbild, Zugriff auf Prozessabbild über Linux-Filesystem als API zu Fremdsoftware.}
  \item{Kommunikationsanschlüsse: 2 x USB 2.0 A (je 500 mA belastbar), 1 x Micro-USB, HDMI, Ethernet (RJ45) 10/100 Mbit/s}
  \item{Stromversorgung: min. 10,7 V, max. 28,8 V, maximal 10 Watt}
  \item{Zulässige Umgebungstemperatur: -40 bis +55 C}
  \item{Gehäuseabmessungen: (HxBxL) 96 mm x 22,5 mm x 110,5 mm (ohne gesteckte Stecker)}
  \item{ESD Schutz: 4 kV / 8 kV gemäß EN61131-2 und IEC 61000-6-2}
  \item{Surge / Burst Prüfungen: gemäß EN61131-2 und IEC 61000-6-2 eingekoppelt auf Versorgungsspannung, Ethernet und IO-Leitungen}
  \item{EMI Prüfungen: gemäß EN61131-2 und IEC 61000-6-2}
\end{itemize}

Kunbus bietet eine Auswahl an IO- und Gateway-Modulen zur Erweiterung des Revolution Pi an.
Gateways dienen der Kommunikation mit Systemen oder Komponenten der Automatisierungstechnik
über Protokolle wie PROFIBUS oder EtherCAT. IO-Module erlauben die Überwachung
und Steuerung von digitalen oder analogen Ein- und Ausgängen.

\subsubsection{Zugriff auf IO-Module%
        \label{sec:2-io}}
Der Zugriff auf die Ein- und Ausgänge der IO-Module erfolgt über ein Prozessabbild
und einen hierfür von Kunbus bereitgestellten Treiber, genannt piControl. Dieser
aktualisiert das Prozessabbild zyklisch. Die angestrebte Zykluszeit beträgt 5ms,
kann jedoch je nach Anzahl der angeschlossenen Module auch größer sein. Kunbus
garantiert bei drei IO-Modulen und zwei Gateway-Modulen eine Zykluszeit von 10 ms.
Jedes der IO-Module stellt ein eigenständiges eingebettetes System dar. Es verfügt
über einen Microcontroller, welcher die IOs bereitstellt und über einen RS485-Bus
mit dem Revolution Pi kommuniziert.
% https://revolution.kunbus.de/io-modul/

Lizenz: GPL
% https://github.com/RevolutionPi/piControl

\begin{lstlisting}[language={c},firstnumber={226},caption={Setzen der Scheduler-Priorität auf SCHED\_FIFO in revpi\_common.c\label{lst:2-sched_priority}}]
param.sched_priority = ktprio->prio;
ret = sched_setscheduler(child, SCHED_FIFO,
       &param);
\end{lstlisting}


\subsection{Echtzeit und Multithreading unter Linux -- preemptRT und posix%
     \label{sec:2-echtzeit}}


 Der Linux-Kernel verfügt über mehrere unterschiedliche Preemtion-Modelle:

\begin{itemize}
  \item No Forced Preemption (server):
  Ausgelegt auf maximal möglichen Durchsatz, lediglich Interrupts und
  System-Call-Returns bewirken Präemption.

  \item Voluntary Kernel Preemption (Desktop):
  Neben den implizit bevorrechtigten Interrupts und System-Call-Returns gibt es
  in diesem Modell weitere Abschnitte des Kernels in welchen Preämption explizit
  gestattet ist.

  \item Preemptible Kernel (Low-Latency Desktop):
  In diesem Modell ist der gesamte Kernel, mit Ausnahme sog.~kritischer Abschnitte
  präemptible. Nach jedem kritischen Abschnitt gibt es einen impliziten Präemptions-Punkt.

  \item Preemptible Kernel (Basic RT):
  Dieses Modell ist dem zuvor genannten sehr ähnlich, hier sind jedoch alle Interrupt-Handler
  als eigenständige Threads ausgeführt.

  \item Fully Preemptible Kernel (RT):
  Wie auch bei den beiden zuvor genannten Modellen ist hier der gesamte Kernel
  präemtible, die Anzahl und Dauer der nicht-präemtiblen kritischen Abschnitte
  ist auf ein notwendiges Minimum beschränkt. Alle Interrupt-Handler sind als
  eigenständige Threads ausgeführt, Spinlocks durch Sleeping-Spinlocks und Mutexe
  durch sog.~RT-Mutexe ersetzt.

\end{itemize}
\todo{Spinlocks und Mutexe sowie die RT-Varianten dieser erklären!}

Lediglich mit dem vollständig präemtiblen Kernel kann Echtzeit-Verhalten realisiert werden.

% https://wiki.linuxfoundation.org/realtime/documentation/technical_basics/preemption_models bzw kernel/Kconfig.preempt

\subsubsection{preemptRT%
        \label{sec:2-preemptRT}}
% https://wiki.linuxfoundation.org/realtime/documentation/technical_details/start
% https://wiki.linuxfoundation.org/realtime/documentation/technical_basics/start

Das dem PREEMPT RT Kernel zugrunde liegende Prinzip lässt sich in einer einfachen
Regel ausdrücken: Nur Code, welcher absolut nicht-präemtible sein darf, ist es
gestattet nicht-präemtible zu sein.
Das erklärte Ziel des PREEMPT\_RT Patches ist es folglich, die Menge des nicht-präemtiblen
Codes im Linux-Kernel auf das absolut notwendige Minimum zu reduzieren.

Dies wird durch Verwendung folgender Mechanismen erreicht:

\begin{itemize}
  \item Hochauflösende Timer
  \item Sleeping Spinlocks
  \item Threaded Interrupt Handlers
  \item rt\_mutex
  \item RCU
\end{itemize}


\subsubsection{posix%
        \label{sec:2-posix}}
Ist posix hier wirklich relevant? Debian bzw.~Raspbian sind weitgehend posix
kompatibel, aber wird es hier genutzt? -> JA, open62541 nutzt pthread.h
piControl nutzt kthread.h, und semaphore.h

\subsection{OPC-UA und open62541%
     \label{sec:2-opc}}

\subsubsection{OPC UA%
        \label{sec:2-opcua}}
Open Platform Communications (OPC) ist eine Familie von Standards zur herstellerunabhängigen
Kommunikation von Maschinen (M2M) in der Automatisierungstechnik. Die sog.~OPC Task Force, zu deren
Mitgliedern verschiedene große Firmen der Automatisierungsindustrie gehören, veröffentlichte
die OPC Specification Version 1.0 im August 1996.
Motiviert ist dieser offene Standard durch die Erkenntniss, dass die Anpassung der
zahlreichen Herstellerstandards an individuelle Infrastrukturen und Anlagen einen
großen Mehraufwand verursachen.
Die Wikipedia beschreibt das Anwendungsgebiet für OPC wie folgt:

\glqq{}OPC wird dort eingesetzt, wo Sensoren, Regler und Steuerungen verschiedener Hersteller
ein gemeinsames Netzwerk bilden. Ohne OPC benötigten zwei Geräte zum Datenaustausch
genaue Kenntnis über die Kommunikationsmöglichkeiten des Gegenübers. Erweiterungen
und Austausch gestalten sich entsprechend schwierig. Mit OPC genügt es, für jedes
Gerät genau einmal einen OPC-konformen Treiber zu schreiben. Idealerweise wird
dieser bereits vom Hersteller zur Verfügung gestellt. Ein OPC-Treiber lässt sich
ohne großen Anpassungsaufwand in beliebig große Steuer- und Überwachungssysteme
integrieren.

OPC unterteilt sich in verschiedene Unterstandards, die für den jeweiligen Anwendungsfall
unabhängig voneinander implementiert werden können. OPC lässt sich damit verwenden
für Echtzeitdaten (Überwachung), Datenarchivierung, Alarm-Meldungen und neuerdings
auch direkt zur Steuerung (Befehlsübermittlung).\grqq{}

OPC basiert in der ursprünglichen Spezifikation auf Microsofts DCOM-Spezifikation.
DCOM macht Funktionen und Objekte einer Anwendung anderen Anwendungen im Netzwerk
zugänglich. Der OPC-Standard definiert entsprechende DCOM-Objekte um mit anderen
OPC-Anwendungen Daten austauschen zu können. Die Verwendung von DCOM bindet Anwender
an Betriebssysteme von Microsoft. Die ursprüngliche OPC Spezifikation wird durch die
Entwicklung von OPC Unified Architecture (OPC UA) abgelöst.
OPC UA setzt auf einem eigenen Kommunikationionsstack auf, die Verwendung von DCOM
und damit die Bindung an Microsoft wurden aufgelöst.

Die OPC-UA-Architektur ist eine Service-orientierte Architektur (SOA), deren Struktur
aus mehreren Schichten besteht.

% Wikipedia
Das OPC-Informationsmodell ist nicht mehr nur eine Hierarchie aus Ordnern, Items
und Properties. Es ist ein sogenanntes Full-Mesh-Network aus Nodes, mit dem neben
den Nutzdaten eines Nodes auch Meta- und Diagnoseinformationen repräsentiert werden.
Ein Node ähnelt einem Objekt aus der objektorientierten Programmierung. Ein Node
kann Attribute besitzen, die gelesen werden können (Data Access (DA), Historical
Data Access (HDA)). Es ist möglich Methoden zu definieren und aufzurufen.
Eine Methode besitzt Aufrufargumente und Rückgabewerte. Sie wird durch ein Command
aufgerufen. Weiterhin werden Events unterstützt, die versendet werden können
(AE (Alarms \& Events), DA DataChange), um bestimmte Informationen zwischen Geräten
auszutauschen. Ein Event besitzt unter anderem einen Empfangszeitpunkt, eine Nachricht
und einen Schweregrad. Die o. g. Nodes werden sowohl für die Nutzdaten als auch
alle anderen Arten von Metadaten verwendet. Der damit modellierte OPC-Adressraum
beinhaltet nun auch ein Typmodell, mit dem sämtliche Datentypen spezifiziert werden.

% https://de.wikipedia.org/wiki/Open_Platform_Communications
% https://de.wikipedia.org/wiki/OPC_Unified_Architecture
% https://opcfoundation.org/developer-tools/specifications-unified-architecture
% Von Gerhard Gappmeier - ascolab GmbH, CC BY-SA 3.0, https://de.wikipedia.org/w/index.php?curid=1892069
\subsubsection{open62541%
        \label{sec:2-open62541}}
open62541 ist eine offene und freie Implementierung von OPC UA. Die in C geschriebene
Bibliothek stellt eine beständig zunehmende Anzahl der im OPC UA Standard definierten
Funktionen bereit. Sie kann sowohl zur Erstellung von OPC-Servern als auch -Clients
genutzt werden. Ergänzend zu der unter der Mozilla Public License v2.0 lizensierten
Bibliothek stellt das open62541 Projekt auch Beispielprogramme unter einer CC0 Lizenz
zur Verfügung.

Die Bibliothek eignet sich auch für die Entwicklung auf eingebetteten Systemen und
Microcontrollern. Je nach Umfang der gewünschten Funktionen und des OPC Informationsmodells
beträgt die Größe einer Server-Binary weniger als 100kb. %evtl. kürzen?

\todo{Nodes erklären! Evtl.~oben!}

Folgende Auswahl an Eigenschaften und Funktionen zeichnet die in dieser Arbeit verwendete
Version 0.3 von open62541 aus:
\begin{itemize}
  \item Kommunikationionsstack
  \begin{itemize}
      \item OPC UA Binär-Protokoll (HTTP oder SOAP werden gegenwärtig nicht unterstützt)
      \item Austauschbare Netzwerk-Schicht, welche die Verwendung eigener Netzwerk-APIs
      erlaubt.
      \item Verschlüsselte Kommunikationion
      \item Asynchrone Dienst-Anfragen im Client
  \end{itemize}
  \item Informationsmodell
  \begin{itemize}
    \item Unterstützung aller OPC UA Node-Typen, inkl.~Methoden
    \item Hinzufügen und Entfernen von Nodes und Referenzen zur Laufzeit.
    \item Vererbung und Instanziierung von Objekt- und Variablentypen
    \item Zugriffskontrolle auch für einzelne Nodes
  \end{itemize}
  \item Subscriptions
  \begin{itemize}
    \item Erlaubt die Überwachung (subscriptions / monitoreditems)
    \item Sehr geringer Ressourcenbedarf pro überwachtem Wert
  \end{itemize}
  \item Code-Generierung auf XML-Basis
  \begin{itemize}
    \item Erlaubt die Erstellung von Datentypen
    \item Erlaubt die Generierung des serverseitigen Informationsmodells
  \end{itemize}
\end{itemize}

% https://open62541.org/doc/0.3/


Mozilla Public License
CC0 Lizenz für Beispiele und Plugins

% https://open62541.org/doc/open62541-current.pdf
% https://open62541.org/

% % % Imports nur für Referenzenauflösung während des Schreibens! Vorm Kompilieren auskommentieren!
% \bibliography{0_hauptdatei}
% \input{1_einleitung}
% \input{2_grundlagen}
% \input{3_konzeption}
% \input{4_implementierung}
% \input{5_tests}
% \input{6_zusammenfassung}
% \input{anhang}
% % Ende Imports

\section{Systemkonzept%
  \label{sec:3-konzeption}}
Auf Basis der in Abschnitt \ref{sec:2-grundlagen} vorgestellten Möglichkeiten folgt nun die Ausarbeitung eines Konzepts.
In den folgenden Abschnitten soll näher auf zwei zentrale Aspekte eingegangen werden: Abschnitt~\ref{sec:3-anbindung} stellt Möglichkeiten zum Zugriff auf Variablen bzw.\,Werte im Prozessabbild des Revolution Pi vor; in Abschnitt~\ref{sec:3-integration} wird ein Konzept zur Bereitstellung dieser Variablen auf einem OPC-Server vorgestellt.

\subsection{Anbindung der IO an den OPC-Server%
     \label{sec:3-anbindung}}

Eine Webanwendung mit Bezeichnung PiCtory dient zur Konfiguration der I/O- und virtuellen Module des RevolutionPi. Die Konfiguration liegt im JSON-Format in der Datei \lstinline{/etc/revpi/config.rsc}. Der piControl-Treiber liest diese Datei beim Start. 
Der folgende Auszug aus der Manpage des piControl-Kernelmoduls beschreibt die von diesem zum Lesen und Schreiben einzelner Bits des Prozessabbildes bereitgestellten Funktionen~\citep[vgl.]{web-revpi-manpage}. Sie ist an dieser Stelle weitgehend ungekürzt zitiert, da sie die nutzbare Schnittstelle sehr kompakt beschreibt.

\begin{lstlisting}[breakindent=0pt, numbers=none, caption={Auszug aus der Revolution Pi Programmers Manual\label{lst:4-manpage}}]
KB_FIND_VARIABLE SPIVariable *argp
Find a variable in the process image by its name. A pointer to a structure of type SPIVariable must be passed as argument. [...]
The struct SPIVariable [...] is defined as 
typedef struct SPIVariableStr
{
    char strVarName[32]; // Variable name
    uint16_t i16uAddress; // Address of the byte in the process image
    uint8_t i8uBit; // 0-7 bit position, >= 8 whole byte
    uint16_t i16uLength; // length of the variable in bits.
    // Possible values are 1, 8, 16 and 32
} SPIVariable;

Set and get values of the process image
KB_GET_VALUE SPIValue *argp
[...]
KB_SET_VALUE SPIValue *argp
Write one bit or one byte to the process image [...].  This call is more efficient than the usual calls of seek and write because only one function call is necessary. If more than on application are writing bits in one output byte, this call is the only safe way to set a bit without overwriting the other bits because this call is doing a read-modify-write-cycle. 

The struct SPIValue used by this ioctl is defined as
typedef struct SPIValueStr
{
    uint16_t i16uAddress; // Address of the byte in the process image
    uint8_t i8uBit; // 0-7 bit position, >= 8 whole byte
    uint8_t i8uValue; // Value: 0/1 for bit access, whole byte otherwise
} SPIValue;
\end{lstlisting} 

Die oben beschriebenden Funtkionen \lstinline{KB_FIND_VARIABLE}, \lstinline{KB_GET_VALUE} und \lstinline{KB_SET_VALUE} ermöglichen einen einfachen und (lt.\,Manpage) effizienten Zugriff auf einzelne Bits des Prozessabbildes und damit auch auf die IO des RevolutionPi.
Der Zugriff des OPC-Servers auf das Prozessabbild soll daher mittels dieser Funktionen realisiert werden.
\lstinline{KB_FIND_VARIABLE} kann genutzt werden, um Adressen von Variablen im Prozessabbild mittels ihres Namens aufzulösen.
\lstinline{KB_GET_VALUE} und \lstinline{KB_SET_VALUE} ermöglichen den Zugriff auf die Werte dieser Variablen.


\subsection{Integration des OPC-Servers in das System%
     \label{sec:3-integration}}

open62541 bietet drei Möglichkeiten zum Abgleich von Variablen mit dem Prozessabbild~\citep[vgl.][Tutorials - Connecting a Variable with a Physical Process]{web-open62541}:
\begin{itemize}
    \item Manuelles oder zyklisches Aktualisieren
    \item Variable Value Callback
    \item Variable Datasource
\end{itemize}

Die zyklische Aktualisierung eines oder mehrerer Werte nimmt, abhängig von der Zykluszeit, viele Systemressourcen in Anspruch. Value Callbacks ermöglichen es, einen Variablenwert effizienter mit einer Ressource wie etwa einem Prozessabbild zu synchronisieren. An die Variable wird ein Callback angehängt, welches vor jedem Lesen und nach jedem Schreibvorgang ausgeführt wird.
Der Wert der Variablen wird weiterhin im Variablenknoten auf dem OPC-Server gespeichert, der Abgleich mit der verknüpften Ressource erfolgt durch die Callback-Methoden.

Sogenannte Datenquellen gehen noch einen Schritt weiter. Der Server leitet jede Lese- und Schreibanforderung direkt an eine Callback-Funktion weiter. Beim Lesen liefert der Rückruf eine Kopie des aktuellen Wertes. Die Datenquelle muss intern ein eigenes Speichermanagement implementieren.

Der Zugriff auf die Werte des Prozessabbildes erfolgt, wie in Abschnitt~\ref{sec:3-anbindung} beschrieben, über von piControl bereitgestellte Methoden. Um die durch open62541 gepflegte OPC-Datenstruktur und das durch piControl verwaltete Prozessabbild möglichst effektiv verknüpfen zu können, soll diese Interaktion mittels Datenquellen und den zugehörigen Callbacks implementiert werden.
% % % Imports nur für Referenzenauflösung während des Schreibens! Vorm Kompilieren auskommentieren!
% \bibliography{0_hauptdatei}
% \input{1_einleitung}
% \input{2_grundlagen}
% \input{3_konzeption}
% \input{4_implementierung}
% \input{5_tests}
% \input{6_zusammenfassung}
% \input{anhang}
% % Ende Imports

\section{Implementierung%
  \label{sec:4-implementierung}}
Das folgende Kapitel stellt in Auszügen die Implementierung des OPC-Servers sowie die Anbindung an die IO-Module
der SPS dar. Der Schwerpunkt liegt hierbei auf der Funktionsweise des piControl-Treibers und dessen Integration in das Projekt. Abschnitt~\ref{sec:4-picontrol} erklärt die zum Schreibens eines Bits verwendeten Funktionsaufrufe.
Zuvor soll jedoch in Abschnitt~\ref{sec:4-open62541} der Teil des OPC-Servers vorgestellt werden, welcher auf besagten Treiber zugreift. 

\subsection{Implementierung des OPC-Servers%
     \label{sec:4-open62541}}
Wie im vorangegangenen Abschnitt~\ref{sec:3-integration} begründet, soll die Verknüpfung zwischen dem Prozessabbild der SPS und den auf dem OPC-Server bereitgestellten Werten über sog.\,Datenquellen erfolgen. Hierzu ist zunächst eine Callback-Methode zu implementieren, welche bei einem Lese- oder Schreibzugriff auf eine Variable aufgerufen wird. Die Verknüpfung zwischen Callback-Methode und Variable muss manuell erfolgen.

\begin{lstlisting}[language={c},firstnumber=237,caption={Auszug der Methode \lstinline{linkDataSourceVariable} in \lstinline{variables.c}\label{lst:4-linkDataSourceVariable}}]
extern UA_StatusCode
 linkDataSourceVariable(UA_Server *server, UA_NodeId nodeId) {
     bool readonly = false;
     UA_DataSource dataSourceVariable;
     UA_StatusCode rc; |>\setcounter{lstnumber}{254}<|

     dataSourceVariable.read = readDataSourceVariable;
     if (!readonly)
        dataSourceVariable.write = writeDataSourceVariable;
     else
        dataSourceVariable.write = writeReadonlyDataSourceVariable;

     return UA_Server_setVariableNode_dataSource(server, nodeId, dataSourceVariable);
 }
\end{lstlisting}

\begin{figure}[h]
    \centering
    \includegraphics[width=0.42\textwidth]{doc/img/OPC_RevPiDO.pdf}
    \caption{Auszug des verwendeten Nodesets, hier Digitalausgang 1 des Versuchsaufbaus
      \label{fig:opc-do}}
\end{figure}

Die in Listing~\ref{lst:4-linkDataSourceVariable} abgebildete Methode \lstinline{linkDataSourceVariable()} erzeugt ein Struct vom Typ \lstinline{UA_DataSource}. In diesem werden dem Lesen und Schreiben einer OPC-Variablen entsprechende Callback-Methoden zugewiesen. Die Verknüpfung einer OPC-Variable, genauer ihrer NodeId, mit der zuvor definierten Datenquelle erfolgt über die von open62541 bereitgestellte Methode \lstinline{UA_Server_setVariableNode_dataSource()}. Vor dem Lesen und nach dem Schreiben dieser Variable werden von nun an die entsprechenden Callbacks aufgerufen.
     
\begin{lstlisting}[language={c},firstnumber=168,caption={Auszug des Callbacks \lstinline{writeDataSourceVariable} in \lstinline{variables.c}\label{lst:4-writeDataSourceVariable}}]  
extern UA_StatusCode
 writeDataSourceVariable(UA_Server *server,
            const UA_NodeId *sessionId, void *sessionContext,
            const UA_NodeId *nodeId, void *nodeContext,
            const UA_NumericRange *range, const UA_DataValue *dataValue) {

    UA_StatusCode retval  = UA_STATUSCODE_GOOD;
    UA_NodeId *nameNodeId = UA_malloc(sizeof(UA_NodeId));
    UA_QualifiedName nameQN = UA_QUALIFIEDNAME(1, "Name");
    UA_Variant nameVar;
    UA_Boolean bit;

    retval |= findSiblingByBrowsename(server, nodeId, &nameQN, nameNodeId);
    retval |= UA_Server_readValue(server, *nameNodeId, &nameVar);
    retval |= UA_Boolean_copy(dataValue->value.data, &bit);

    |>\tikzmarkin[set border color=martinired]{writeIO}<|PI_writeSingleIO(String_fromUA_String(nameVar.data), &bit, false);                                                 |>\tikzmarkend{writeIO}<|

    free(nameNodeId);
    return retval;
 }
\end{lstlisting}

Listing~\ref{lst:4-writeDataSourceVariable} zeigt die Callback-Methode, welche nach dem Schreiben einer Variablen auf dem OPC-Server aufgerufen wird.
Dieser Methode wird neben der NodeId der mit ihr verknüpften Variablen auch der Wert dieser in Form eines Zeigers auf ein Struct vom Typ \lstinline{UA_DataValue} übergeben.

Die Gestaltung des hier verwendeten Nodesets sieht vor, dass in einer OPC-Variablen \lstinline{"Name"} der Bezeichner des zu schreibenden Digitalausgangs hinterlegt ist, siehe Abbildung~\ref{fig:opc-do}. Dies erlaubt eine Rekonfiguration der Ein- und Ausgänge der SPS ohne Änderungen im Programmcode des OPC-Servers vornehmen zu müssen.
Es ist daher erforderlich, nach jedem Schreiben einer mit einem Digitalausgang verknüpften Variablen, hier \lstinline{"Value"}, dessen Bezeichner \lstinline{"Name"} abzufragen. 
Dies geschieht in den Zeilen 180 und 181.
Anschließend wird dieser Bezeichner sowie der zu schreibende Wert der Methode \lstinline{PI_writeSingleIO()} übergeben, welche wiederum die Interaktion mit piControl übernimmt (vgl. Abschnitt \ref{sec:4-picontrol}).
 
\subsection{Integration von piControl%
     \label{sec:4-picontrol}}
In Abschnitt~\ref{sec:2-io} wurde die Anbindung der IO-Module des Revolution Pi sowie die Funktionsweise von piControl aus Anwendersicht beschrieben. Die verfügbare Literatur beschränkt sich auch auf lediglich diese Sicht; eine weiterführende Dokumentation für Entwickler gibt es, neben der in Abschnitt~\ref{sec:3-anbindung} vorgestellten Manpage, nicht. 
In diesem Abschnitt soll daher der Quellcode von piControl sowie dessen Verwendung im Projekt genauer betrachtet werden.
Hierzu wird exemplarisch die in Abschnitt~\ref{sec:4-open62541} eingeführte Methode \lstinline{PI_writeSingleIO()} untersucht.
Diese Methode ermöglicht das Setzen eines einzelnen Bits im Prozessabbild der SPS, und damit das Schalten eines digitalen Ausgangs auf einem IO-Modul.
Die äquivalente Methode \lstinline{int piControlGetBitValue(SPIValue *pSpiValue)} zum Lesen eines Bits bzw. Eingangs funktioniert analog und soll daher an dieser Stelle nicht dediziert erörtert werden.

\begin{lstlisting}[language={c},firstnumber=97,
                   caption={Setzen eines phsikalischen, digitalen Ausgangs in \lstinline{revpi.c}
                   \label{lst:4-PI_writeSingleIO}}]
extern void PI_writeSingleIO(char *pszVariableName, bool *bit, bool verbose)
{
	int rc;
	SPIVariable sPiVariable;
	SPIValue sPIValue;

	strncpy(sPiVariable.strVarName, pszVariableName, sizeof(sPiVariable.strVarName));
	rc = piControlGetVariableInfo(&sPiVariable);
	if (rc < 0) {
		printf("Cannot find variable '%s'\n", pszVariableName);
		return;
	}

		sPIValue.i16uAddress = sPiVariable.i16uAddress;
		sPIValue.i8uBit = sPiVariable.i8uBit;
		sPIValue.i8uValue = *bit;
		rc = |>\tikzmarkin[set border color=martinired]{setBitValue}<|piControlSetBitValue(&sPIValue)|>\tikzmarkend{setBitValue}<|;
		if (rc < 0)
			printf("Set bit error %s\n", getWriteError(rc));
		else if (verbose)
			printf("Set bit %d on byte at offset %d. Value %d\n", sPIValue.i8uBit, sPIValue.i16uAddress,
			       sPIValue.i8uValue);
}
\end{lstlisting}

Der Programmcode in Listing~\ref{lst:4-PI_writeSingleIO} ist Teil des implementierten OPC-Servers. In diesem wird auf zwei Funktionen des piControl-Treibers zugegriffen. 
Beiden Methoden wird als Argument ein Zeiger auf ein Struct vom Typ \lstinline{SPIValue} übergeben. Der im Struct abgelegte Name wird mittels \lstinline{piControlGetVariableInfo(&sPIValue)} zu einer Adresse im Prozessabbild aufgelöst. Diese wird in \lstinline{sPIValue.i16uAdress} gespeichert. Der Wert der Variablen wird anschließend mittels \lstinline{piControlSetBitValue(&sPIValue)} an dieser Adresse in das Prozessabbild geschrieben.

\begin{lstlisting}[language={c},firstnumber=309,caption={Methode \lstinline{piControlSetBitValue} in \lstinline{piControlIf.c}\label{lst:4-piControlSetBitValue}}]
int |>\tikzmarkin[set border color=martiniblue]{setBitValueFcn}<|piControlSetBitValue(SPIValue *pSpiValue)|>\tikzmarkend{setBitValueFcn}<|
{
    piControlOpen();

    if (PiControlHandle_g < 0)
	    return -ENODEV;

    pSpiValue->i16uAddress += pSpiValue->i8uBit / 8;
    pSpiValue->i8uBit %= 8;

    if (|>\tikzmarkin[set border color=martinired]{ioctl}<|ioctl(PiControlHandle_g, KB_SET_VALUE, pSpiValue)|>\tikzmarkend{ioctl}<| < 0)
	    return errno;

    return 0;
}
\end{lstlisting}

Die in Listing~\ref{lst:4-piControlSetBitValue} dargestellte Methode \lstinline{piControlSetBitValue} ist lediglich eine Hüllfunktion (häufig auch als Wrapper-Funktion bezeichnet) für einen Aufruf des \lstinline{ioctl} Kernel-Moduls.
Folgende Parameter werden übergeben:
\lstinline{PiControlHandle_g} ist die Referenz auf die Geräte-Datei des piControl-Treibers. \lstinline{KB_SET_VALUE} ist das ioctl-Kommando zum Schreiben eines Bits in das Prozessabbild. Der Zeiger \lstinline{pSpiValue} verweist auf ein Struct des bereits vorgestellten Typs \lstinline{SPIValue}.

\begin{lstlisting}[language={c},firstnumber=80,caption={Methode \lstinline{piControlOpen} in \lstinline{piControlIf.c}\label{lst:4-piControlOpen}}]
void piControlOpen(void)
{
    /* open handle if needed */
    if (PiControlHandle_g < 0)
    {
	    |>\tikzmarkin[set border color=martiniblue]{PiControlHandle}<|PiControlHandle_g = open(PICONTROL_DEVICE, O_RDWR)|>\tikzmarkend{PiControlHandle}<|;
    }
}
\end{lstlisting}

Die in Listing~\ref{lst:4-piControlOpen} dargestellte Methode öffnet, sofern nicht bereits geschehen, die Geräte-Datei. Das Macro \lstinline{PICONTROL_DEVICE} verweist hierbei auf \lstinline{/dev/piControl0}.

\begin{lstlisting}[language={c},firstnumber=721,caption={Methode \lstinline{piControlIoctl} in \lstinline{piControlMain.c}\label{lst:4-piControlIoctl}}]
static long |>\tikzmarkin[set border color=martiniblue, below offset=0.9em]{piControlIoctl}<|piControlIoctl(struct file *file, unsigned int prg_nr, 
                           unsigned long usr_addr)                                      |>\tikzmarkend{piControlIoctl}<|
{
  int status = -EFAULT;
  tpiControlInst *priv;
  int timeout = 10000;	// ms

  if (prg_nr != KB_CONFIG_SEND && prg_nr != KB_CONFIG_START && !isRunning()) {
  	return -EAGAIN;
  }

  priv = (tpiControlInst *) file->private_data;

  if (prg_nr != KB_GET_LAST_MESSAGE) {
  	// clear old message
  	priv->pcErrorMessage[0] = 0;
  }

  switch (prg_nr) {|>\setcounter{lstnumber}{864}<|

    case |>\tikzmarkin[set border color=martiniblue]{KB_SET_VALUE}<|KB_SET_VALUE:|>\tikzmarkend{KB_SET_VALUE}<|
  		{
  			SPIValue *pValue = (SPIValue *) usr_addr;

  			if (!isRunning())
  				return -EFAULT;

  			if (pValue->i16uAddress >= KB_PI_LEN) {
  				status = -EFAULT;
  			} else {
  				INT8U i8uValue_l;
  				my_rt_mutex_lock(&piDev_g.lockPI);
  				i8uValue_l = piDev_g.ai8uPI[pValue->i16uAddress];

  				if (pValue->i8uBit >= 8) {
  					i8uValue_l = pValue->i8uValue;
  				} else {
  					if (pValue->i8uValue)
  						i8uValue_l |= (1 << pValue->i8uBit);
  					else
  						i8uValue_l &= ~(1 << pValue->i8uBit);
  				}

  				|>\tikzmarkin[set border color=martinired]{i8uValue}<|piDev_g.ai8uPI[pValue->i16uAddress] = i8uValue_l;|>\tikzmarkend{i8uValue}<|
  				rt_mutex_unlock(&piDev_g.lockPI);

  #ifdef VERBOSE
  				pr_info("piControlIoctl Addr=%u, bit=%u: %02x %02x\n", pValue->i16uAddress, pValue->i8uBit, pValue->i8uValue, i8uValue_l);
  #endif

  				status = 0;
  			}
  		}
  		break; |>\setcounter{lstnumber}{1314}<|

    default:
      pr_err("Invalid Ioctl");
      return (-EINVAL);
      break;

    }

    return status;
  }
\end{lstlisting}

Listing~\ref{lst:4-piControlIoctl} zeigt in Auszügen die ioctl-Methode des piControl Kernel-Treibers. Diese bekommt folgende Argumente übergeben: \lstinline{struct file *file} enthält den Verweis auf die Geräte-Datei, hier \lstinline{/dev/piControl0}. Der Wert von \lstinline{unsigned int prg_nr} beschreibt die Anfrage an den Treiber, in diesem Fall \lstinline{KB_SET_VALUE}. Das Argument \lstinline{unsigned long usr_addr} enthält einen typ-agnostischen Pointer. Dieser verweist auf einen Speicherbereich, in welchem die zur Bearbeitung der Anfrage notwendigen Daten abgelegt sind. Hier können auch vom Treiber empfangene Daten dem Anwendungsprogramm bereitgestellt werden. 

Die switch-case-Anweisung führt die über das Argument \lstinline{prg_nr} spezifizierte Aktion aus. Hier betrachten wir \lstinline{KB_SET_VALUE}:
Zunächst wird in Zeile 868 der übergebene Zeiger \lstinline{usr_addr} mittels explizitem Typecast zu einem Zeiger des Typs \lstinline{SPIValue *} konvertiert. Da dieser auf Daten im Userspace verweist, ist beim Zugriff durch den Kernel-Treiber besondere Vorsicht geboten.
In Zeile 877 wird mittels Mutex das Prozessabbild \lstinline{piDev_g} für den Zugriff durch andere Threads oder Prozesse gesperrt.
\lstinline{my_rt_mutex_lock} verweist hierbei auf die Funktion \lstinline{rt_mutex_lock} aus \lstinline{linux/sched.h}\footnote{Offenbar wurde hier auch eine alternative Implementierung vorgesehen, siehe revpi\_common.h}

In Zeile 889 wird das Byte \lstinline{i8uValue_l}, welches den zu schreibenden Wert enthält in das Prozessabbild übertragen. Anschließend wird die Mutex auf \lstinline{piDev_g} wieder entsperrt.
\newpage

\begin{lstlisting}[language={c},firstnumber=62,caption={Auszug des Struct \lstinline{spiControlDev} in \lstinline{piControlMain.h}\label{lst:4-spiControlDev}}]
|>\tikzmarkin[set border color=martiniblue]{spiControlDev}<|typedef struct spiControlDev|>\tikzmarkend{spiControlDev}<| {
	// device driver stuff
	int init_step;
	enum revpi_machine machine_type;
	void *machine;
	struct cdev cdev;	// Char device structure
	struct device *dev;
	struct thermal_zone_device *thermal_zone;

	|>\tikzmarkin[set border color=martiniblue]{processImage}<|// process image stuff
	INT8U ai8uPI[KB_PI_LEN];
	INT8U ai8uPIDefault|>\tikzmarkin[set border color=martinired]{KB_PI_LEN_0}<|[KB_PI_LEN]|>\tikzmarkend{KB_PI_LEN_0}<|;
	struct rt_mutex lockPI;        |>\tikzmarkend{processImage}<|
	bool stopIO;
	piDevices *devs; |>\setcounter{lstnumber}{94}<|
} tpiControlDev;
\end{lstlisting}

Das Prozessabbild ist als Byte-Array der Länge \lstinline{KB_PI_LEN} in Listing~\ref{lst:4-spiControlDev} definiert. Konfigurationsparameter wie \lstinline{KB_PI_LEN} oder die Zykluszeit für den Datenaustausch zwischen SPS und IO-Modulen sind im folgenden Listing~\ref{lst:4-process} definiert.

\begin{lstlisting}[language={c},firstnumber=119,caption={Konfigurationsparameter des Prozessabbildes in project.h\label{lst:4-process}}]
#define INTERVAL_PI_GATE (5*1000*1000)  // 5 ms piGateCommunication |>\setcounter{lstnumber}{128}<|

#define INTERVAL_IO_COM (5*1000*1000)  // 5 ms piIoComm |>\setcounter{lstnumber}{132}<|

#define KB_PD_LEN       512
|>\tikzmarkin[set border color=martiniblue]{KB_PI_LEN_1}<|#define KB_PI_LEN       4096|>\tikzmarkend{KB_PI_LEN_1}<|
\end{lstlisting}

Das zu setzende Bit wurde zu diesem Zeitpunkt erfolgreich in das Prozessabbild der SPS geschrieben.
Es stellt sich die Frage, wie dieses nun an das IO-Modul kommuniziert wird.
Die Kommunikation mit allen angebundenen Modulen ist ebenfalls Aufgabe des piControl-Treibers.

\begin{lstlisting}[language={c},firstnumber=256,caption={Auszug der Methode \lstinline{piIoThread} in \lstinline{revpi_core.c}\label{lst:4-piIoThread}}]
static int piIoThread(void *data)
{
	//TODO int value = 0;
	ktime_t time;
	ktime_t now;
	s64 tDiff;

	hrtimer_init(&piCore_g.ioTimer, CLOCK_MONOTONIC, HRTIMER_MODE_ABS);
	piCore_g.ioTimer.function = piIoTimer;

	pr_info("piIO thread started\n");

	now = hrtimer_cb_get_time(&piCore_g.ioTimer);

	PiBridgeMaster_Reset();

	while (!kthread_should_stop()) {
		if (|>\tikzmarkin[set border color=martinired]{PiBridgeMaster}<|PiBridgeMaster_Run()|>\tikzmarkend{PiBridgeMaster}<| < 0)
			break;
	}

	RevPiDevice_finish();

	pr_info("piIO exit\n");
	return 0;
}
\end{lstlisting}

Der Kernel-Thread \lstinline{piIoThread} ist verantwortlich für den zyklischen Datenaustausch mit den IO-Modulen. In diesem wird fortlaufend die Methode \lstinline{PiBridgeMaster_Run()} aufgerufen, siehe Listing~\ref{lst:4-piIoThread}.

\begin{lstlisting}[language={c},firstnumber=262,caption={Auszug der Methode \lstinline{PiBridgeMaster_Run(void)} in \lstinline{RevPiDevice.c}\label{lst:4-PiBridgeMaster_Run}}]
int PiBridgeMaster_Run(void)
{
	static kbUT_Timer tTimeoutTimer_s;
	static kbUT_Timer tConfigTimeoutTimer_s;
	static int error_cnt;
	static INT8U last_led;
	static unsigned long last_update;
	int ret = 0;
	int i;

	my_rt_mutex_lock(&piCore_g.lockBridgeState);
	if (piCore_g.eBridgeState != piBridgeStop) {
		switch (eRunStatus_s) { |>\setcounter{lstnumber}{514}<|
		    case enPiBridgeMasterStatus_EndOfConfig:|>\setcounter{lstnumber}{621}<|
		    if (|>\tikzmarkin[set border color=martinired]{RevPiDevice}<|RevPiDevice_run()|>\tikzmarkend{RevPiDevice}<|) {
				// an error occured, check error limits |>\setcounter{lstnumber}{641}<|
			} else {
				ret = 1;
			}
			piCore_g.image.drv.i16uRS485ErrorCnt = RevPiDevice_getErrCnt();
			break;
\end{lstlisting}

Die in Listing~\ref{lst:4-PiBridgeMaster_Run} dargestellte Methode ist eine sog. State-Machine. Ist die Konfiguration der IO-Module erfolgreich abgeschlossen, so führt sie bei Aufruf lediglich die Methode \lstinline{RevPiDevice_run()} aus.

\begin{lstlisting}[language={c},firstnumber=140,caption={Auszug der Methode \lstinline{RevPiDevice_run(void)} in \lstinline{RevPiDevice.c}\label{lst:4-RevPiDevice_run}}]
int RevPiDevice_run(void)
{
	INT8U i8uDevice = 0;
	INT32U r;
	int retval = 0;

	RevPiDevices_s.i16uErrorCnt = 0;

	for (i8uDevice = 0; i8uDevice < RevPiDevice_getDevCnt(); i8uDevice++) {
		if (RevPiDevice_getDev(i8uDevice)->i8uActive) {
			switch (RevPiDevice_getDev(i8uDevice)->sId.i16uModulType) {
			case KUNBUS_FW_DESCR_TYP_PI_DIO_14:
			case KUNBUS_FW_DESCR_TYP_PI_DI_16:
			case KUNBUS_FW_DESCR_TYP_PI_DO_16:
				r = |>\tikzmarkin[set border color=martinired]{sendCyclicTelegram}<|piDIOComm_sendCyclicTelegram(i8uDevice)|>\tikzmarkend{sendCyclicTelegram}\setcounter{lstnumber}{166} <|;

				break; |>\setcounter{lstnumber}{216}<|
			}
		}
	} |>\setcounter{lstnumber}{227}<|
	return retval;
}
\end{lstlisting}

Diese iteriert wie in Listing~\ref{lst:4-RevPiDevice_run} abgebildete durch alle gegenwärtig in der SPS konfigurierten Module. Ist das aktuelle Modul als aktiv markiert, so wird anhand eines sog. Firmware-Descriptors entschieden, welche Methode für die Ansteuerung des Moduls aufzurufen ist.

\begin{lstlisting}[language={c},firstnumber=161,caption={Auszug der Methode \lstinline{piDIOComm_sendCyclicTelegram} in \lstinline{piDIOComm.c}\label{lst:4-sendCyclicTelegram}}]
INT32U piDIOComm_sendCyclicTelegram(INT8U i8uDevice_p)
{
	INT32U i32uRv_l = 0;
	SIOGeneric sRequest_l;
	SIOGeneric sResponse_l;
	INT8U len_l, data_out[18], i, p, data_in[70];
	INT8U i8uAddress;
	int ret; |>\setcounter{lstnumber}{239}<|
	
    |>\tikzmarkin[set border color=martinired]{piIoComm}<|ret = piIoComm_send((INT8U *) & sRequest_l, IOPROTOCOL_HEADER_LENGTH + len_l + 1);  |>\tikzmarkend{piIoComm}\setcounter{lstnumber}{298}<|
}
\end{lstlisting}

Im Falle des hier verwendeten DO-Moduls wird die in Listing~\ref{lst:4-sendCyclicTelegram} abgebildete Methode \lstinline{piDIOComm_sendCyclicTelegram()} aufgerufen. Dieser wird ein Zeiger auf das zu schreibende Gerät übergeben. 
Zunächst wird das Prozessabbild mittels eines proprietären, jedoch im Quellcode offen nachvollziehbaren Protokolls in ein \lstinline{sRequest_l} genanntes Byte-Array umgewandelt. Dieser Schritt ist in Listing~\ref{lst:4-sendCyclicTelegram} nicht abgebildet. Anschließend wird \lstinline{piIoComm_send()} ein Zeiger auf die so generierte Schreib-Anfrage übergeben.

\begin{lstlisting}[language={c},firstnumber=220,caption={Auszug der Methode \lstinline{piIOComm_send} in \lstinline{piIOComm.c}\label{lst:4-piIOComm_send}}]
int piIoComm_send(INT8U * buf_p, INT16U i16uLen_p)
{
	ssize_t write_l = 0;
	INT16U i16uSent_l = 0;|>\setcounter{lstnumber}{249}<|

	while (i16uSent_l < i16uLen_p) {
		write_l = vfs_write(piIoComm_fd_m, buf_p + i16uSent_l, i16uLen_p - i16uSent_l, &piIoComm_fd_m->f_pos);
		if (write_l < 0) {
			pr_info_serial("write error %d\n", (int)write_l);
			return -1;
		} 
		i16uSent_l += write_l;|>\setcounter{lstnumber}{263}<|
	}
	clear();
	vfs_fsync(piIoComm_fd_m, 1);
	return 0;
}
\end{lstlisting}

Listing~\ref{lst:4-piIOComm_send} zeigt die Implementierung von \lstinline{piIoComm_send()}. Diese Methode ist für das Schreiben der oben generierten Anfrage auf die seriellen Schnittstelle verantwortlich. Realisiert wird dies mittels der Methode \lstinline{vfs_write()}. Diese ist in \lstinline{<linux/fs.h>} definiert. Sie ermöglicht das Schreiben einer Datei im Userspace aus dem Kernel heraus. Geschrieben wird hier die Datei mit dem Deskriptor \lstinline{piIoComm_fd_m}.
Da die Funktion \lstinline{vfs_write()} durch andere Kernel-Tasks unterbrochen werden kann, ist nicht gewährleistet, dass die gesamte Anfrage mit nur einem Aufruf geschrieben wird. Die oben abgebildete while-Schleife stellt das vollständige Senden der Anfrage sicher.

\begin{lstlisting}[language={c},firstnumber=157,caption={Auszug der Methode \lstinline{piIOComm_open_serial} in \lstinline{piIOComm.c}\label{lst:4-piIOComm_open_serial}}]
int piIoComm_open_serial(void)
{   |>\setcounter{lstnumber}{167}<|
	struct file *fd;	/* Filedeskriptor */
	struct termios newtio;	/* Schnittstellenoptionen */

	|>\tikzmarkin[set border color=martiniblue]{fd}<|/* Port oeffnen - read/write, kein "controlling tty", 
	    Status von DCD ignorieren */
	fd = filp_open(|>\tikzmarkin[set border color=martinired]{tty}<|REV_PI_TTY_DEVICE|>\tikzmarkend{tty}<|, O_RDWR | O_NOCTTY, 0); |>\setcounter{lstnumber}{208}<|
	
	piIoComm_fd_m = fd;                                                      |>\tikzmarkend{fd}\setcounter{lstnumber}{217}<|

	return 0;
}
\end{lstlisting}

Der zum Schreiben auf die serielle Schnittstelle verwendete Datei-Deskriptor wird von der in Listing~\ref{lst:4-piIOComm_open_serial} abgebildeten Methode \lstinline{piIoComm_open_serial()} generiert. 

\begin{lstlisting}[language={c},firstnumber=45,caption={Definition der seriellen Schnittstelle in \lstinline{piIOComm.h}\label{lst:4-REV_PI_TTY_DEVICE}}]
#define REV_PI_TTY_DEVICE	"/dev/ttyAMA0"
\end{lstlisting}

Das in Listing~\ref{lst:4-REV_PI_TTY_DEVICE} definierte Macro verweist auf eine der seriellen Schnittstellen des RaspberryPi.
Die Implementierung des zugehörigen Schnittstellentreibers soll hier nicht weiter untersucht werden. Somit ist an dieser Stelle die Kette vom Setzen einer Variablen auf dem OPC-Server bis hin zur Aktualisierung des Prozessabbilds der IO-Module geschlossen.

% \begin{lstlisting}[language={c},firstnumber={226},caption={Setzen der Scheduler-Priorität auf SCHED\_FIFO in 
% revpi\_common.c\label{lst:2-sched_priority}}]
% param.sched_priority = ktprio->prio;
% ret = sched_setscheduler(child, SCHED_FIFO, &param);
% \end{lstlisting}
% % % Imports nur für Referenzenauflösung während des Schreibens! Vorm Kompilieren auskommentieren!
% \bibliography{0_hauptdatei}
% \input{1_einleitung}
% \input{2_grundlagen}
% \input{3_konzeption}
% \input{4_implementierung}
% \input{5_tests}
% \input{6_zusammenfassung}
% % Ende Imports

\section{Test des OPC-Servers im Gesamtsystem%
  \label{sec:5-tests}}

% % % Imports nur für Referenzenauflösung während des schreibens! Vorm Kompilieren auskommentieren!
% \bibliography{0_hauptdatei}
% \input{1_einleitung}
% \input{2_grundlagen}
% \input{3_konzeption}
% \input{4_implementierung}
% \input{5_tests}
% \input{6_zusammenfassung}
% % Ende Imports

\section{Zusammenfassung und Ausblick%
  \label{sec:6-fazit}}
Der folgende Abschnitt~\ref{sec:6-zusammenfassung} fasst die gewonnenen Erkenntnisse und den Stand der Implementierung zusammen.
Den Abschluss dieser Arbeit bildet der Ausblick in Abschnitt~\ref{sec:6-ausblick}.

\subsection{Zusammenfassung%
     \label{sec:6-zusammenfassung}}

\subsection{Ausblick%
     \label{sec:6-ausblick}}

% % Ende Imports

\section{Zusammenfassung und Ausblick%
  \label{sec:6-fazit}}
Der folgende Abschnitt~\ref{sec:6-zusammenfassung} fasst die gewonnenen Erkenntnisse und den Stand der Implementierung zusammen.
Den Abschluss dieser Arbeit bildet der Ausblick in Abschnitt~\ref{sec:6-ausblick}.

\subsection{Zusammenfassung%
     \label{sec:6-zusammenfassung}}

\subsection{Ausblick%
     \label{sec:6-ausblick}}

% \input{anhang}
% % Ende Imports

\section{Systemkonzept%
  \label{sec:3-konzeption}}
Auf Basis der in Abschnitt \ref{sec:2-grundlagen} vorgestellten Möglichkeiten folgt nun die Ausarbeitung eines Konzepts.
In den folgenden Abschnitten soll näher auf zwei zentrale Aspekte eingegangen werden: Abschnitt~\ref{sec:3-anbindung} stellt Möglichkeiten zum Zugriff auf Variablen bzw.\,Werte im Prozessabbild des Revolution Pi vor; in Abschnitt~\ref{sec:3-integration} wird ein Konzept zur Bereitstellung dieser Variablen auf einem OPC-Server vorgestellt.

\subsection{Anbindung der IO an den OPC-Server%
     \label{sec:3-anbindung}}

Eine Webanwendung mit Bezeichnung PiCtory dient zur Konfiguration der I/O- und virtuellen Module des RevolutionPi. Die Konfiguration liegt im JSON-Format in der Datei \lstinline{/etc/revpi/config.rsc}. Der piControl-Treiber liest diese Datei beim Start. 
Der folgende Auszug aus der Manpage des piControl-Kernelmoduls beschreibt die von diesem zum Lesen und Schreiben einzelner Bits des Prozessabbildes bereitgestellten Funktionen~\citep[vgl.]{web-revpi-manpage}. Sie ist an dieser Stelle weitgehend ungekürzt zitiert, da sie die nutzbare Schnittstelle sehr kompakt beschreibt.

\begin{lstlisting}[breakindent=0pt, numbers=none, caption={Auszug aus der Revolution Pi Programmers Manual\label{lst:4-manpage}}]
KB_FIND_VARIABLE SPIVariable *argp
Find a variable in the process image by its name. A pointer to a structure of type SPIVariable must be passed as argument. [...]
The struct SPIVariable [...] is defined as 
typedef struct SPIVariableStr
{
    char strVarName[32]; // Variable name
    uint16_t i16uAddress; // Address of the byte in the process image
    uint8_t i8uBit; // 0-7 bit position, >= 8 whole byte
    uint16_t i16uLength; // length of the variable in bits.
    // Possible values are 1, 8, 16 and 32
} SPIVariable;

Set and get values of the process image
KB_GET_VALUE SPIValue *argp
[...]
KB_SET_VALUE SPIValue *argp
Write one bit or one byte to the process image [...].  This call is more efficient than the usual calls of seek and write because only one function call is necessary. If more than on application are writing bits in one output byte, this call is the only safe way to set a bit without overwriting the other bits because this call is doing a read-modify-write-cycle. 

The struct SPIValue used by this ioctl is defined as
typedef struct SPIValueStr
{
    uint16_t i16uAddress; // Address of the byte in the process image
    uint8_t i8uBit; // 0-7 bit position, >= 8 whole byte
    uint8_t i8uValue; // Value: 0/1 for bit access, whole byte otherwise
} SPIValue;
\end{lstlisting} 

Die oben beschriebenden Funtkionen \lstinline{KB_FIND_VARIABLE}, \lstinline{KB_GET_VALUE} und \lstinline{KB_SET_VALUE} ermöglichen einen einfachen und (lt.\,Manpage) effizienten Zugriff auf einzelne Bits des Prozessabbildes und damit auch auf die IO des RevolutionPi.
Der Zugriff des OPC-Servers auf das Prozessabbild soll daher mittels dieser Funktionen realisiert werden.
\lstinline{KB_FIND_VARIABLE} kann genutzt werden, um Adressen von Variablen im Prozessabbild mittels ihres Namens aufzulösen.
\lstinline{KB_GET_VALUE} und \lstinline{KB_SET_VALUE} ermöglichen den Zugriff auf die Werte dieser Variablen.


\subsection{Integration des OPC-Servers in das System%
     \label{sec:3-integration}}

open62541 bietet drei Möglichkeiten zum Abgleich von Variablen mit dem Prozessabbild~\citep[vgl.][Tutorials - Connecting a Variable with a Physical Process]{web-open62541}:
\begin{itemize}
    \item Manuelles oder zyklisches Aktualisieren
    \item Variable Value Callback
    \item Variable Datasource
\end{itemize}

Die zyklische Aktualisierung eines oder mehrerer Werte nimmt, abhängig von der Zykluszeit, viele Systemressourcen in Anspruch. Value Callbacks ermöglichen es, einen Variablenwert effizienter mit einer Ressource wie etwa einem Prozessabbild zu synchronisieren. An die Variable wird ein Callback angehängt, welches vor jedem Lesen und nach jedem Schreibvorgang ausgeführt wird.
Der Wert der Variablen wird weiterhin im Variablenknoten auf dem OPC-Server gespeichert, der Abgleich mit der verknüpften Ressource erfolgt durch die Callback-Methoden.

Sogenannte Datenquellen gehen noch einen Schritt weiter. Der Server leitet jede Lese- und Schreibanforderung direkt an eine Callback-Funktion weiter. Beim Lesen liefert der Rückruf eine Kopie des aktuellen Wertes. Die Datenquelle muss intern ein eigenes Speichermanagement implementieren.

Der Zugriff auf die Werte des Prozessabbildes erfolgt, wie in Abschnitt~\ref{sec:3-anbindung} beschrieben, über von piControl bereitgestellte Methoden. Um die durch open62541 gepflegte OPC-Datenstruktur und das durch piControl verwaltete Prozessabbild möglichst effektiv verknüpfen zu können, soll diese Interaktion mittels Datenquellen und den zugehörigen Callbacks implementiert werden.
% % % Imports nur für Referenzenauflösung während des Schreibens! Vorm Kompilieren auskommentieren!
% \bibliography{0_hauptdatei}
% % Mit \section{...} eröffnen wir einen neuen Abschnitt.
% Der Befehl setzt nicht nur den Text in einer größeren,
% fetten Schrift, sondern sorgt außerdem dafür, daß er im
% Inhaltsverzeichnis erscheint.
%
% Mit \label{...} erzeugen wir einen Bezeichner, mit dessen Hilfe
% wir später auf die Nummer des Abschnitts verweisen können (nämlich
% mit~\ref{...}).
%
% Das Kommentarzeichen hinter „Übersicht“ dient dazu, ein
% Leerzeichen zwischen „Übersicht“ und dem \label-Befehl
% zu vermeiden, das andernfalls sichtbar würde – z.B. im
% Inhaltsverzeichnis.
%

% % Imports nur für Referenzenauflösung während des Schreibens! Vorm Kompilieren auskommentieren!
% \bibliography{0_hauptdatei}
% % Mit \section{...} eröffnen wir einen neuen Abschnitt.
% Der Befehl setzt nicht nur den Text in einer größeren,
% fetten Schrift, sondern sorgt außerdem dafür, daß er im
% Inhaltsverzeichnis erscheint.
%
% Mit \label{...} erzeugen wir einen Bezeichner, mit dessen Hilfe
% wir später auf die Nummer des Abschnitts verweisen können (nämlich
% mit~\ref{...}).
%
% Das Kommentarzeichen hinter „Übersicht“ dient dazu, ein
% Leerzeichen zwischen „Übersicht“ und dem \label-Befehl
% zu vermeiden, das andernfalls sichtbar würde – z.B. im
% Inhaltsverzeichnis.
%

% % Imports nur für Referenzenauflösung während des Schreibens! Vorm Kompilieren auskommentieren!
% \bibliography{0_hauptdatei}
% \input{1_einleitung}
%\input{2_grundlagen}
%\input{3_konzeption}
%\input{4_implementierung}
%\input{5_tests}
%\input{6_zusammenfassung}
% % Ende Imports

\section{Einleitung und Motivation%
  \label{sec:1-einleitung}}
Ziel dieses Projektes ist die Integration eines OPC-Servers mit einer auf Linux
basierenden speicherprogrammierbaren Steuerung (SPS). Angeschlossen an diese SPS
ist jeweils ein digitales Ein-/\,bzw.~Ausgabemodul. Die von diesen bereitgestellten
Ein-/\, bzw.~Ausgänge (IO) sollen in der Datenstruktur des OPC-Servers abgebildet
und über diesen für OPC-Clients les-/\,und schreibar sein. Weiterhin sollen einige
Funktionen zur Überwachung und Steuerung der an die SPS angeschlossenen Aktoren
und Sensoren direkt im OPC-Server implementiert werden.
Hiermit stellt dieses Projekt eine der Grundlagen für ein übergeordnetes Projekt,
die cloudbasierte Steuerung eines miniaturisierten Produktions-Systems, dar.

Der hier verwendete OPC-Server ist Teil des sog. open62541 Projekts. Er ist in C
geschrieben und implementiert bereits einen großen Teil der im OPC-UA-Standard
spezifizierten Funktionen.
Als SPS findet ein Revolution Pi 3 der Firma Kunbus Verwendung. Dieser integriert
ein sog. Compute Module der Raspberry Pi Foundation in ein industrietaugliches
Gehäuse und erlaubt die Erweiterung mittels IO- oder Gateway-Modulen. Über diese
erfolgt die Kommunikation mit weiteren Komponenten der Automatisierungstechnik.

Motiviert ist dieses Projekt durch die Beobachtung, dass die Verbreitung offener
Standards sowie freier Software auch in der Automatisierungstechnik zunimmt.
Linux ist ein freies Betriebssystem, OPC-UA ein offen zugänglicher, aktiv gepflegter
und weit verbreiteter Standard. Der Raspberry Pi findet sowohl bei Hobby-Anwendern als
auch in den Bereichen Forschung und Entwicklung sowie bei industriellen Anwendern
Verwendung. Dieses Projekt stellt somit eine für unterschiedliche Anwender interessante
Entwicklung dar.

Im Anschluss an diese einleitende Übersicht im Abschnitt~\ref{sec:1-einleitung} folgt
die Darstellung der wichtigsten Grundlagen in Abschnitt~\ref{sec:2-grundlagen}.
Aufbauend auf diesen Grundlagen folgt die konzeptuelle Ausarbeitung im Abschnitt~\ref{sec:3-konzeption}.
Die Umsetzung wird im Abschnitt~\ref{sec:4-implementierung} erläutert.
Die Leistungsfähigkeit der Implementierung wird in Abschnitt~\ref{sec:5-tests} untersucht.
Eine Zusammenfassung und ein Ausblick schließen die Arbeit in
Abschnitt~\ref{sec:6-fazit} ab. Eventuell noch benötigte Anhänge
finden sich in den Anhängen [...] bis [...].

%% % Imports nur für Referenzenauflösung während des Schreibens! Vorm Kompilieren auskommentieren!
% \bibliography{0_hauptdatei}
% \input{1_einleitung}
% \input{2_grundlagen}
% \input{3_konzeption}
% \input{4_implementierung}
% \input{5_tests}
% \input{6_zusammenfassung}
% % Ende Imports

\section{Grundlagen%
  \label{sec:2-grundlagen}}

\subsection{Speicherprogrammierbare-Steuerung und Linux -- Revolution Pi%
     \label{sec:2-sps}}

\subsubsection{Kunbus RevolutionPi%
        \label{sec:2-revpi}}
Der RevolutionPi 3 ist eine speicherprogrammierbare Steuerung (SPS) des Herstellers
Kunbus GmbH. Kern dieser SPS ist das von der Raspberry Pi Foundation entwickelte
und vertriebene Raspberry Pi Compute Module 3. Dieses integriert ein Broadcom BCM2837
System-on-Chip (SoC) mit vier 1,2GHz Prozessorkernen, 1GB RAM, 4GB eMMC Anwendungsspeicher
und sonstige Peripherie in ein Modul im DDR2-SODIMM Formfaktor. Diese Spezifikationen
sind weitgehend identisch zu denen des ausgesprochen populären Raspberry Pi 3.
Der Revolution Pi profitiert daher von dem gleichen großen Angebot an Software
und Unterstützung wie der Raspberry Pi, ergänzt dessen Hardware jedoch um eine 24V
Spannungsversorgung, die Möglichkeit der Erweiterung durch mehrere industrietaugliche
Ein-/ Ausgabemodule und Gateways sowie ein Gehäuse zur Montage auf einer DIN-Schiene.
\begin{itemize}
  \item{Prozessor: BCM2837}
  \item{Taktfrequenz 1,2 GHz}
  \item{Anzahl Prozessorkerne: 4}
  \item{Arbeitsspeicher: 1 GByte}
  \item{eMMC Flash Speicher: 4 GByte}
  \item{Betriebssystem: Angepasstes Raspbian mit RT-Patch}
  \item{RTC mit 24h Pufferung über wartungsfreien Kondensator}
  \item{Treiber / API: Treiber schreibt zyklisch Prozessdaten in ein Prozessabbild, Zugriff auf Prozessabbild über Linux-Filesystem als API zu Fremdsoftware.}
  \item{Kommunikationsanschlüsse: 2 x USB 2.0 A (je 500 mA belastbar), 1 x Micro-USB, HDMI, Ethernet (RJ45) 10/100 Mbit/s}
  \item{Stromversorgung: min. 10,7 V, max. 28,8 V, maximal 10 Watt}
  \item{Zulässige Umgebungstemperatur: -40 bis +55 C}
  \item{Gehäuseabmessungen: (HxBxL) 96 mm x 22,5 mm x 110,5 mm (ohne gesteckte Stecker)}
  \item{ESD Schutz: 4 kV / 8 kV gemäß EN61131-2 und IEC 61000-6-2}
  \item{Surge / Burst Prüfungen: gemäß EN61131-2 und IEC 61000-6-2 eingekoppelt auf Versorgungsspannung, Ethernet und IO-Leitungen}
  \item{EMI Prüfungen: gemäß EN61131-2 und IEC 61000-6-2}
\end{itemize}

Kunbus bietet eine Auswahl an IO- und Gateway-Modulen zur Erweiterung des Revolution Pi an.
Gateways dienen der Kommunikation mit Systemen oder Komponenten der Automatisierungstechnik
über Protokolle wie PROFIBUS oder EtherCAT. IO-Module erlauben die Überwachung
und Steuerung von digitalen oder analogen Ein- und Ausgängen.

\subsubsection{Zugriff auf IO-Module%
        \label{sec:2-io}}
Der Zugriff auf die Ein- und Ausgänge der IO-Module erfolgt über ein Prozessabbild
und einen hierfür von Kunbus bereitgestellten Treiber, genannt piControl. Dieser
aktualisiert das Prozessabbild zyklisch. Die angestrebte Zykluszeit beträgt 5ms,
kann jedoch je nach Anzahl der angeschlossenen Module auch größer sein. Kunbus
garantiert bei drei IO-Modulen und zwei Gateway-Modulen eine Zykluszeit von 10 ms.
Jedes der IO-Module stellt ein eigenständiges eingebettetes System dar. Es verfügt
über einen Microcontroller, welcher die IOs bereitstellt und über einen RS485-Bus
mit dem Revolution Pi kommuniziert.
% https://revolution.kunbus.de/io-modul/

Lizenz: GPL
% https://github.com/RevolutionPi/piControl

\begin{lstlisting}[language={c},firstnumber={226},caption={Setzen der Scheduler-Priorität auf SCHED\_FIFO in revpi\_common.c\label{lst:2-sched_priority}}]
param.sched_priority = ktprio->prio;
ret = sched_setscheduler(child, SCHED_FIFO,
       &param);
\end{lstlisting}


\subsection{Echtzeit und Multithreading unter Linux -- preemptRT und posix%
     \label{sec:2-echtzeit}}


 Der Linux-Kernel verfügt über mehrere unterschiedliche Preemtion-Modelle:

\begin{itemize}
  \item No Forced Preemption (server):
  Ausgelegt auf maximal möglichen Durchsatz, lediglich Interrupts und
  System-Call-Returns bewirken Präemption.

  \item Voluntary Kernel Preemption (Desktop):
  Neben den implizit bevorrechtigten Interrupts und System-Call-Returns gibt es
  in diesem Modell weitere Abschnitte des Kernels in welchen Preämption explizit
  gestattet ist.

  \item Preemptible Kernel (Low-Latency Desktop):
  In diesem Modell ist der gesamte Kernel, mit Ausnahme sog.~kritischer Abschnitte
  präemptible. Nach jedem kritischen Abschnitt gibt es einen impliziten Präemptions-Punkt.

  \item Preemptible Kernel (Basic RT):
  Dieses Modell ist dem zuvor genannten sehr ähnlich, hier sind jedoch alle Interrupt-Handler
  als eigenständige Threads ausgeführt.

  \item Fully Preemptible Kernel (RT):
  Wie auch bei den beiden zuvor genannten Modellen ist hier der gesamte Kernel
  präemtible, die Anzahl und Dauer der nicht-präemtiblen kritischen Abschnitte
  ist auf ein notwendiges Minimum beschränkt. Alle Interrupt-Handler sind als
  eigenständige Threads ausgeführt, Spinlocks durch Sleeping-Spinlocks und Mutexe
  durch sog.~RT-Mutexe ersetzt.

\end{itemize}
\todo{Spinlocks und Mutexe sowie die RT-Varianten dieser erklären!}

Lediglich mit dem vollständig präemtiblen Kernel kann Echtzeit-Verhalten realisiert werden.

% https://wiki.linuxfoundation.org/realtime/documentation/technical_basics/preemption_models bzw kernel/Kconfig.preempt

\subsubsection{preemptRT%
        \label{sec:2-preemptRT}}
% https://wiki.linuxfoundation.org/realtime/documentation/technical_details/start
% https://wiki.linuxfoundation.org/realtime/documentation/technical_basics/start

Das dem PREEMPT RT Kernel zugrunde liegende Prinzip lässt sich in einer einfachen
Regel ausdrücken: Nur Code, welcher absolut nicht-präemtible sein darf, ist es
gestattet nicht-präemtible zu sein.
Das erklärte Ziel des PREEMPT\_RT Patches ist es folglich, die Menge des nicht-präemtiblen
Codes im Linux-Kernel auf das absolut notwendige Minimum zu reduzieren.

Dies wird durch Verwendung folgender Mechanismen erreicht:

\begin{itemize}
  \item Hochauflösende Timer
  \item Sleeping Spinlocks
  \item Threaded Interrupt Handlers
  \item rt\_mutex
  \item RCU
\end{itemize}


\subsubsection{posix%
        \label{sec:2-posix}}
Ist posix hier wirklich relevant? Debian bzw.~Raspbian sind weitgehend posix
kompatibel, aber wird es hier genutzt? -> JA, open62541 nutzt pthread.h
piControl nutzt kthread.h, und semaphore.h

\subsection{OPC-UA und open62541%
     \label{sec:2-opc}}

\subsubsection{OPC UA%
        \label{sec:2-opcua}}
Open Platform Communications (OPC) ist eine Familie von Standards zur herstellerunabhängigen
Kommunikation von Maschinen (M2M) in der Automatisierungstechnik. Die sog.~OPC Task Force, zu deren
Mitgliedern verschiedene große Firmen der Automatisierungsindustrie gehören, veröffentlichte
die OPC Specification Version 1.0 im August 1996.
Motiviert ist dieser offene Standard durch die Erkenntniss, dass die Anpassung der
zahlreichen Herstellerstandards an individuelle Infrastrukturen und Anlagen einen
großen Mehraufwand verursachen.
Die Wikipedia beschreibt das Anwendungsgebiet für OPC wie folgt:

\glqq{}OPC wird dort eingesetzt, wo Sensoren, Regler und Steuerungen verschiedener Hersteller
ein gemeinsames Netzwerk bilden. Ohne OPC benötigten zwei Geräte zum Datenaustausch
genaue Kenntnis über die Kommunikationsmöglichkeiten des Gegenübers. Erweiterungen
und Austausch gestalten sich entsprechend schwierig. Mit OPC genügt es, für jedes
Gerät genau einmal einen OPC-konformen Treiber zu schreiben. Idealerweise wird
dieser bereits vom Hersteller zur Verfügung gestellt. Ein OPC-Treiber lässt sich
ohne großen Anpassungsaufwand in beliebig große Steuer- und Überwachungssysteme
integrieren.

OPC unterteilt sich in verschiedene Unterstandards, die für den jeweiligen Anwendungsfall
unabhängig voneinander implementiert werden können. OPC lässt sich damit verwenden
für Echtzeitdaten (Überwachung), Datenarchivierung, Alarm-Meldungen und neuerdings
auch direkt zur Steuerung (Befehlsübermittlung).\grqq{}

OPC basiert in der ursprünglichen Spezifikation auf Microsofts DCOM-Spezifikation.
DCOM macht Funktionen und Objekte einer Anwendung anderen Anwendungen im Netzwerk
zugänglich. Der OPC-Standard definiert entsprechende DCOM-Objekte um mit anderen
OPC-Anwendungen Daten austauschen zu können. Die Verwendung von DCOM bindet Anwender
an Betriebssysteme von Microsoft. Die ursprüngliche OPC Spezifikation wird durch die
Entwicklung von OPC Unified Architecture (OPC UA) abgelöst.
OPC UA setzt auf einem eigenen Kommunikationionsstack auf, die Verwendung von DCOM
und damit die Bindung an Microsoft wurden aufgelöst.

Die OPC-UA-Architektur ist eine Service-orientierte Architektur (SOA), deren Struktur
aus mehreren Schichten besteht.

% Wikipedia
Das OPC-Informationsmodell ist nicht mehr nur eine Hierarchie aus Ordnern, Items
und Properties. Es ist ein sogenanntes Full-Mesh-Network aus Nodes, mit dem neben
den Nutzdaten eines Nodes auch Meta- und Diagnoseinformationen repräsentiert werden.
Ein Node ähnelt einem Objekt aus der objektorientierten Programmierung. Ein Node
kann Attribute besitzen, die gelesen werden können (Data Access (DA), Historical
Data Access (HDA)). Es ist möglich Methoden zu definieren und aufzurufen.
Eine Methode besitzt Aufrufargumente und Rückgabewerte. Sie wird durch ein Command
aufgerufen. Weiterhin werden Events unterstützt, die versendet werden können
(AE (Alarms \& Events), DA DataChange), um bestimmte Informationen zwischen Geräten
auszutauschen. Ein Event besitzt unter anderem einen Empfangszeitpunkt, eine Nachricht
und einen Schweregrad. Die o. g. Nodes werden sowohl für die Nutzdaten als auch
alle anderen Arten von Metadaten verwendet. Der damit modellierte OPC-Adressraum
beinhaltet nun auch ein Typmodell, mit dem sämtliche Datentypen spezifiziert werden.

% https://de.wikipedia.org/wiki/Open_Platform_Communications
% https://de.wikipedia.org/wiki/OPC_Unified_Architecture
% https://opcfoundation.org/developer-tools/specifications-unified-architecture
% Von Gerhard Gappmeier - ascolab GmbH, CC BY-SA 3.0, https://de.wikipedia.org/w/index.php?curid=1892069
\subsubsection{open62541%
        \label{sec:2-open62541}}
open62541 ist eine offene und freie Implementierung von OPC UA. Die in C geschriebene
Bibliothek stellt eine beständig zunehmende Anzahl der im OPC UA Standard definierten
Funktionen bereit. Sie kann sowohl zur Erstellung von OPC-Servern als auch -Clients
genutzt werden. Ergänzend zu der unter der Mozilla Public License v2.0 lizensierten
Bibliothek stellt das open62541 Projekt auch Beispielprogramme unter einer CC0 Lizenz
zur Verfügung.

Die Bibliothek eignet sich auch für die Entwicklung auf eingebetteten Systemen und
Microcontrollern. Je nach Umfang der gewünschten Funktionen und des OPC Informationsmodells
beträgt die Größe einer Server-Binary weniger als 100kb. %evtl. kürzen?

\todo{Nodes erklären! Evtl.~oben!}

Folgende Auswahl an Eigenschaften und Funktionen zeichnet die in dieser Arbeit verwendete
Version 0.3 von open62541 aus:
\begin{itemize}
  \item Kommunikationionsstack
  \begin{itemize}
      \item OPC UA Binär-Protokoll (HTTP oder SOAP werden gegenwärtig nicht unterstützt)
      \item Austauschbare Netzwerk-Schicht, welche die Verwendung eigener Netzwerk-APIs
      erlaubt.
      \item Verschlüsselte Kommunikationion
      \item Asynchrone Dienst-Anfragen im Client
  \end{itemize}
  \item Informationsmodell
  \begin{itemize}
    \item Unterstützung aller OPC UA Node-Typen, inkl.~Methoden
    \item Hinzufügen und Entfernen von Nodes und Referenzen zur Laufzeit.
    \item Vererbung und Instanziierung von Objekt- und Variablentypen
    \item Zugriffskontrolle auch für einzelne Nodes
  \end{itemize}
  \item Subscriptions
  \begin{itemize}
    \item Erlaubt die Überwachung (subscriptions / monitoreditems)
    \item Sehr geringer Ressourcenbedarf pro überwachtem Wert
  \end{itemize}
  \item Code-Generierung auf XML-Basis
  \begin{itemize}
    \item Erlaubt die Erstellung von Datentypen
    \item Erlaubt die Generierung des serverseitigen Informationsmodells
  \end{itemize}
\end{itemize}

% https://open62541.org/doc/0.3/


Mozilla Public License
CC0 Lizenz für Beispiele und Plugins

% https://open62541.org/doc/open62541-current.pdf
% https://open62541.org/

%% % Imports nur für Referenzenauflösung während des Schreibens! Vorm Kompilieren auskommentieren!
% \bibliography{0_hauptdatei}
% \input{1_einleitung}
% \input{2_grundlagen}
% \input{3_konzeption}
% \input{4_implementierung}
% \input{5_tests}
% \input{6_zusammenfassung}
% \input{anhang}
% % Ende Imports

\section{Systemkonzept%
  \label{sec:3-konzeption}}
Auf Basis der in Abschnitt \ref{sec:2-grundlagen} vorgestellten Möglichkeiten folgt nun die Ausarbeitung eines Konzepts.
In den folgenden Abschnitten soll näher auf zwei zentrale Aspekte eingegangen werden: Abschnitt~\ref{sec:3-anbindung} stellt Möglichkeiten zum Zugriff auf Variablen bzw.\,Werte im Prozessabbild des Revolution Pi vor; in Abschnitt~\ref{sec:3-integration} wird ein Konzept zur Bereitstellung dieser Variablen auf einem OPC-Server vorgestellt.

\subsection{Anbindung der IO an den OPC-Server%
     \label{sec:3-anbindung}}

Eine Webanwendung mit Bezeichnung PiCtory dient zur Konfiguration der I/O- und virtuellen Module des RevolutionPi. Die Konfiguration liegt im JSON-Format in der Datei \lstinline{/etc/revpi/config.rsc}. Der piControl-Treiber liest diese Datei beim Start. 
Der folgende Auszug aus der Manpage des piControl-Kernelmoduls beschreibt die von diesem zum Lesen und Schreiben einzelner Bits des Prozessabbildes bereitgestellten Funktionen~\citep[vgl.]{web-revpi-manpage}. Sie ist an dieser Stelle weitgehend ungekürzt zitiert, da sie die nutzbare Schnittstelle sehr kompakt beschreibt.

\begin{lstlisting}[breakindent=0pt, numbers=none, caption={Auszug aus der Revolution Pi Programmers Manual\label{lst:4-manpage}}]
KB_FIND_VARIABLE SPIVariable *argp
Find a variable in the process image by its name. A pointer to a structure of type SPIVariable must be passed as argument. [...]
The struct SPIVariable [...] is defined as 
typedef struct SPIVariableStr
{
    char strVarName[32]; // Variable name
    uint16_t i16uAddress; // Address of the byte in the process image
    uint8_t i8uBit; // 0-7 bit position, >= 8 whole byte
    uint16_t i16uLength; // length of the variable in bits.
    // Possible values are 1, 8, 16 and 32
} SPIVariable;

Set and get values of the process image
KB_GET_VALUE SPIValue *argp
[...]
KB_SET_VALUE SPIValue *argp
Write one bit or one byte to the process image [...].  This call is more efficient than the usual calls of seek and write because only one function call is necessary. If more than on application are writing bits in one output byte, this call is the only safe way to set a bit without overwriting the other bits because this call is doing a read-modify-write-cycle. 

The struct SPIValue used by this ioctl is defined as
typedef struct SPIValueStr
{
    uint16_t i16uAddress; // Address of the byte in the process image
    uint8_t i8uBit; // 0-7 bit position, >= 8 whole byte
    uint8_t i8uValue; // Value: 0/1 for bit access, whole byte otherwise
} SPIValue;
\end{lstlisting} 

Die oben beschriebenden Funtkionen \lstinline{KB_FIND_VARIABLE}, \lstinline{KB_GET_VALUE} und \lstinline{KB_SET_VALUE} ermöglichen einen einfachen und (lt.\,Manpage) effizienten Zugriff auf einzelne Bits des Prozessabbildes und damit auch auf die IO des RevolutionPi.
Der Zugriff des OPC-Servers auf das Prozessabbild soll daher mittels dieser Funktionen realisiert werden.
\lstinline{KB_FIND_VARIABLE} kann genutzt werden, um Adressen von Variablen im Prozessabbild mittels ihres Namens aufzulösen.
\lstinline{KB_GET_VALUE} und \lstinline{KB_SET_VALUE} ermöglichen den Zugriff auf die Werte dieser Variablen.


\subsection{Integration des OPC-Servers in das System%
     \label{sec:3-integration}}

open62541 bietet drei Möglichkeiten zum Abgleich von Variablen mit dem Prozessabbild~\citep[vgl.][Tutorials - Connecting a Variable with a Physical Process]{web-open62541}:
\begin{itemize}
    \item Manuelles oder zyklisches Aktualisieren
    \item Variable Value Callback
    \item Variable Datasource
\end{itemize}

Die zyklische Aktualisierung eines oder mehrerer Werte nimmt, abhängig von der Zykluszeit, viele Systemressourcen in Anspruch. Value Callbacks ermöglichen es, einen Variablenwert effizienter mit einer Ressource wie etwa einem Prozessabbild zu synchronisieren. An die Variable wird ein Callback angehängt, welches vor jedem Lesen und nach jedem Schreibvorgang ausgeführt wird.
Der Wert der Variablen wird weiterhin im Variablenknoten auf dem OPC-Server gespeichert, der Abgleich mit der verknüpften Ressource erfolgt durch die Callback-Methoden.

Sogenannte Datenquellen gehen noch einen Schritt weiter. Der Server leitet jede Lese- und Schreibanforderung direkt an eine Callback-Funktion weiter. Beim Lesen liefert der Rückruf eine Kopie des aktuellen Wertes. Die Datenquelle muss intern ein eigenes Speichermanagement implementieren.

Der Zugriff auf die Werte des Prozessabbildes erfolgt, wie in Abschnitt~\ref{sec:3-anbindung} beschrieben, über von piControl bereitgestellte Methoden. Um die durch open62541 gepflegte OPC-Datenstruktur und das durch piControl verwaltete Prozessabbild möglichst effektiv verknüpfen zu können, soll diese Interaktion mittels Datenquellen und den zugehörigen Callbacks implementiert werden.
%% % Imports nur für Referenzenauflösung während des Schreibens! Vorm Kompilieren auskommentieren!
% \bibliography{0_hauptdatei}
% \input{1_einleitung}
% \input{2_grundlagen}
% \input{3_konzeption}
% \input{4_implementierung}
% \input{5_tests}
% \input{6_zusammenfassung}
% \input{anhang}
% % Ende Imports

\section{Implementierung%
  \label{sec:4-implementierung}}
Das folgende Kapitel stellt in Auszügen die Implementierung des OPC-Servers sowie die Anbindung an die IO-Module
der SPS dar. Der Schwerpunkt liegt hierbei auf der Funktionsweise des piControl-Treibers und dessen Integration in das Projekt. Abschnitt~\ref{sec:4-picontrol} erklärt die zum Schreibens eines Bits verwendeten Funktionsaufrufe.
Zuvor soll jedoch in Abschnitt~\ref{sec:4-open62541} der Teil des OPC-Servers vorgestellt werden, welcher auf besagten Treiber zugreift. 

\subsection{Implementierung des OPC-Servers%
     \label{sec:4-open62541}}
Wie im vorangegangenen Abschnitt~\ref{sec:3-integration} begründet, soll die Verknüpfung zwischen dem Prozessabbild der SPS und den auf dem OPC-Server bereitgestellten Werten über sog.\,Datenquellen erfolgen. Hierzu ist zunächst eine Callback-Methode zu implementieren, welche bei einem Lese- oder Schreibzugriff auf eine Variable aufgerufen wird. Die Verknüpfung zwischen Callback-Methode und Variable muss manuell erfolgen.

\begin{lstlisting}[language={c},firstnumber=237,caption={Auszug der Methode \lstinline{linkDataSourceVariable} in \lstinline{variables.c}\label{lst:4-linkDataSourceVariable}}]
extern UA_StatusCode
 linkDataSourceVariable(UA_Server *server, UA_NodeId nodeId) {
     bool readonly = false;
     UA_DataSource dataSourceVariable;
     UA_StatusCode rc; |>\setcounter{lstnumber}{254}<|

     dataSourceVariable.read = readDataSourceVariable;
     if (!readonly)
        dataSourceVariable.write = writeDataSourceVariable;
     else
        dataSourceVariable.write = writeReadonlyDataSourceVariable;

     return UA_Server_setVariableNode_dataSource(server, nodeId, dataSourceVariable);
 }
\end{lstlisting}

\begin{figure}[h]
    \centering
    \includegraphics[width=0.42\textwidth]{doc/img/OPC_RevPiDO.pdf}
    \caption{Auszug des verwendeten Nodesets, hier Digitalausgang 1 des Versuchsaufbaus
      \label{fig:opc-do}}
\end{figure}

Die in Listing~\ref{lst:4-linkDataSourceVariable} abgebildete Methode \lstinline{linkDataSourceVariable()} erzeugt ein Struct vom Typ \lstinline{UA_DataSource}. In diesem werden dem Lesen und Schreiben einer OPC-Variablen entsprechende Callback-Methoden zugewiesen. Die Verknüpfung einer OPC-Variable, genauer ihrer NodeId, mit der zuvor definierten Datenquelle erfolgt über die von open62541 bereitgestellte Methode \lstinline{UA_Server_setVariableNode_dataSource()}. Vor dem Lesen und nach dem Schreiben dieser Variable werden von nun an die entsprechenden Callbacks aufgerufen.
     
\begin{lstlisting}[language={c},firstnumber=168,caption={Auszug des Callbacks \lstinline{writeDataSourceVariable} in \lstinline{variables.c}\label{lst:4-writeDataSourceVariable}}]  
extern UA_StatusCode
 writeDataSourceVariable(UA_Server *server,
            const UA_NodeId *sessionId, void *sessionContext,
            const UA_NodeId *nodeId, void *nodeContext,
            const UA_NumericRange *range, const UA_DataValue *dataValue) {

    UA_StatusCode retval  = UA_STATUSCODE_GOOD;
    UA_NodeId *nameNodeId = UA_malloc(sizeof(UA_NodeId));
    UA_QualifiedName nameQN = UA_QUALIFIEDNAME(1, "Name");
    UA_Variant nameVar;
    UA_Boolean bit;

    retval |= findSiblingByBrowsename(server, nodeId, &nameQN, nameNodeId);
    retval |= UA_Server_readValue(server, *nameNodeId, &nameVar);
    retval |= UA_Boolean_copy(dataValue->value.data, &bit);

    |>\tikzmarkin[set border color=martinired]{writeIO}<|PI_writeSingleIO(String_fromUA_String(nameVar.data), &bit, false);                                                 |>\tikzmarkend{writeIO}<|

    free(nameNodeId);
    return retval;
 }
\end{lstlisting}

Listing~\ref{lst:4-writeDataSourceVariable} zeigt die Callback-Methode, welche nach dem Schreiben einer Variablen auf dem OPC-Server aufgerufen wird.
Dieser Methode wird neben der NodeId der mit ihr verknüpften Variablen auch der Wert dieser in Form eines Zeigers auf ein Struct vom Typ \lstinline{UA_DataValue} übergeben.

Die Gestaltung des hier verwendeten Nodesets sieht vor, dass in einer OPC-Variablen \lstinline{"Name"} der Bezeichner des zu schreibenden Digitalausgangs hinterlegt ist, siehe Abbildung~\ref{fig:opc-do}. Dies erlaubt eine Rekonfiguration der Ein- und Ausgänge der SPS ohne Änderungen im Programmcode des OPC-Servers vornehmen zu müssen.
Es ist daher erforderlich, nach jedem Schreiben einer mit einem Digitalausgang verknüpften Variablen, hier \lstinline{"Value"}, dessen Bezeichner \lstinline{"Name"} abzufragen. 
Dies geschieht in den Zeilen 180 und 181.
Anschließend wird dieser Bezeichner sowie der zu schreibende Wert der Methode \lstinline{PI_writeSingleIO()} übergeben, welche wiederum die Interaktion mit piControl übernimmt (vgl. Abschnitt \ref{sec:4-picontrol}).
 
\subsection{Integration von piControl%
     \label{sec:4-picontrol}}
In Abschnitt~\ref{sec:2-io} wurde die Anbindung der IO-Module des Revolution Pi sowie die Funktionsweise von piControl aus Anwendersicht beschrieben. Die verfügbare Literatur beschränkt sich auch auf lediglich diese Sicht; eine weiterführende Dokumentation für Entwickler gibt es, neben der in Abschnitt~\ref{sec:3-anbindung} vorgestellten Manpage, nicht. 
In diesem Abschnitt soll daher der Quellcode von piControl sowie dessen Verwendung im Projekt genauer betrachtet werden.
Hierzu wird exemplarisch die in Abschnitt~\ref{sec:4-open62541} eingeführte Methode \lstinline{PI_writeSingleIO()} untersucht.
Diese Methode ermöglicht das Setzen eines einzelnen Bits im Prozessabbild der SPS, und damit das Schalten eines digitalen Ausgangs auf einem IO-Modul.
Die äquivalente Methode \lstinline{int piControlGetBitValue(SPIValue *pSpiValue)} zum Lesen eines Bits bzw. Eingangs funktioniert analog und soll daher an dieser Stelle nicht dediziert erörtert werden.

\begin{lstlisting}[language={c},firstnumber=97,
                   caption={Setzen eines phsikalischen, digitalen Ausgangs in \lstinline{revpi.c}
                   \label{lst:4-PI_writeSingleIO}}]
extern void PI_writeSingleIO(char *pszVariableName, bool *bit, bool verbose)
{
	int rc;
	SPIVariable sPiVariable;
	SPIValue sPIValue;

	strncpy(sPiVariable.strVarName, pszVariableName, sizeof(sPiVariable.strVarName));
	rc = piControlGetVariableInfo(&sPiVariable);
	if (rc < 0) {
		printf("Cannot find variable '%s'\n", pszVariableName);
		return;
	}

		sPIValue.i16uAddress = sPiVariable.i16uAddress;
		sPIValue.i8uBit = sPiVariable.i8uBit;
		sPIValue.i8uValue = *bit;
		rc = |>\tikzmarkin[set border color=martinired]{setBitValue}<|piControlSetBitValue(&sPIValue)|>\tikzmarkend{setBitValue}<|;
		if (rc < 0)
			printf("Set bit error %s\n", getWriteError(rc));
		else if (verbose)
			printf("Set bit %d on byte at offset %d. Value %d\n", sPIValue.i8uBit, sPIValue.i16uAddress,
			       sPIValue.i8uValue);
}
\end{lstlisting}

Der Programmcode in Listing~\ref{lst:4-PI_writeSingleIO} ist Teil des implementierten OPC-Servers. In diesem wird auf zwei Funktionen des piControl-Treibers zugegriffen. 
Beiden Methoden wird als Argument ein Zeiger auf ein Struct vom Typ \lstinline{SPIValue} übergeben. Der im Struct abgelegte Name wird mittels \lstinline{piControlGetVariableInfo(&sPIValue)} zu einer Adresse im Prozessabbild aufgelöst. Diese wird in \lstinline{sPIValue.i16uAdress} gespeichert. Der Wert der Variablen wird anschließend mittels \lstinline{piControlSetBitValue(&sPIValue)} an dieser Adresse in das Prozessabbild geschrieben.

\begin{lstlisting}[language={c},firstnumber=309,caption={Methode \lstinline{piControlSetBitValue} in \lstinline{piControlIf.c}\label{lst:4-piControlSetBitValue}}]
int |>\tikzmarkin[set border color=martiniblue]{setBitValueFcn}<|piControlSetBitValue(SPIValue *pSpiValue)|>\tikzmarkend{setBitValueFcn}<|
{
    piControlOpen();

    if (PiControlHandle_g < 0)
	    return -ENODEV;

    pSpiValue->i16uAddress += pSpiValue->i8uBit / 8;
    pSpiValue->i8uBit %= 8;

    if (|>\tikzmarkin[set border color=martinired]{ioctl}<|ioctl(PiControlHandle_g, KB_SET_VALUE, pSpiValue)|>\tikzmarkend{ioctl}<| < 0)
	    return errno;

    return 0;
}
\end{lstlisting}

Die in Listing~\ref{lst:4-piControlSetBitValue} dargestellte Methode \lstinline{piControlSetBitValue} ist lediglich eine Hüllfunktion (häufig auch als Wrapper-Funktion bezeichnet) für einen Aufruf des \lstinline{ioctl} Kernel-Moduls.
Folgende Parameter werden übergeben:
\lstinline{PiControlHandle_g} ist die Referenz auf die Geräte-Datei des piControl-Treibers. \lstinline{KB_SET_VALUE} ist das ioctl-Kommando zum Schreiben eines Bits in das Prozessabbild. Der Zeiger \lstinline{pSpiValue} verweist auf ein Struct des bereits vorgestellten Typs \lstinline{SPIValue}.

\begin{lstlisting}[language={c},firstnumber=80,caption={Methode \lstinline{piControlOpen} in \lstinline{piControlIf.c}\label{lst:4-piControlOpen}}]
void piControlOpen(void)
{
    /* open handle if needed */
    if (PiControlHandle_g < 0)
    {
	    |>\tikzmarkin[set border color=martiniblue]{PiControlHandle}<|PiControlHandle_g = open(PICONTROL_DEVICE, O_RDWR)|>\tikzmarkend{PiControlHandle}<|;
    }
}
\end{lstlisting}

Die in Listing~\ref{lst:4-piControlOpen} dargestellte Methode öffnet, sofern nicht bereits geschehen, die Geräte-Datei. Das Macro \lstinline{PICONTROL_DEVICE} verweist hierbei auf \lstinline{/dev/piControl0}.

\begin{lstlisting}[language={c},firstnumber=721,caption={Methode \lstinline{piControlIoctl} in \lstinline{piControlMain.c}\label{lst:4-piControlIoctl}}]
static long |>\tikzmarkin[set border color=martiniblue, below offset=0.9em]{piControlIoctl}<|piControlIoctl(struct file *file, unsigned int prg_nr, 
                           unsigned long usr_addr)                                      |>\tikzmarkend{piControlIoctl}<|
{
  int status = -EFAULT;
  tpiControlInst *priv;
  int timeout = 10000;	// ms

  if (prg_nr != KB_CONFIG_SEND && prg_nr != KB_CONFIG_START && !isRunning()) {
  	return -EAGAIN;
  }

  priv = (tpiControlInst *) file->private_data;

  if (prg_nr != KB_GET_LAST_MESSAGE) {
  	// clear old message
  	priv->pcErrorMessage[0] = 0;
  }

  switch (prg_nr) {|>\setcounter{lstnumber}{864}<|

    case |>\tikzmarkin[set border color=martiniblue]{KB_SET_VALUE}<|KB_SET_VALUE:|>\tikzmarkend{KB_SET_VALUE}<|
  		{
  			SPIValue *pValue = (SPIValue *) usr_addr;

  			if (!isRunning())
  				return -EFAULT;

  			if (pValue->i16uAddress >= KB_PI_LEN) {
  				status = -EFAULT;
  			} else {
  				INT8U i8uValue_l;
  				my_rt_mutex_lock(&piDev_g.lockPI);
  				i8uValue_l = piDev_g.ai8uPI[pValue->i16uAddress];

  				if (pValue->i8uBit >= 8) {
  					i8uValue_l = pValue->i8uValue;
  				} else {
  					if (pValue->i8uValue)
  						i8uValue_l |= (1 << pValue->i8uBit);
  					else
  						i8uValue_l &= ~(1 << pValue->i8uBit);
  				}

  				|>\tikzmarkin[set border color=martinired]{i8uValue}<|piDev_g.ai8uPI[pValue->i16uAddress] = i8uValue_l;|>\tikzmarkend{i8uValue}<|
  				rt_mutex_unlock(&piDev_g.lockPI);

  #ifdef VERBOSE
  				pr_info("piControlIoctl Addr=%u, bit=%u: %02x %02x\n", pValue->i16uAddress, pValue->i8uBit, pValue->i8uValue, i8uValue_l);
  #endif

  				status = 0;
  			}
  		}
  		break; |>\setcounter{lstnumber}{1314}<|

    default:
      pr_err("Invalid Ioctl");
      return (-EINVAL);
      break;

    }

    return status;
  }
\end{lstlisting}

Listing~\ref{lst:4-piControlIoctl} zeigt in Auszügen die ioctl-Methode des piControl Kernel-Treibers. Diese bekommt folgende Argumente übergeben: \lstinline{struct file *file} enthält den Verweis auf die Geräte-Datei, hier \lstinline{/dev/piControl0}. Der Wert von \lstinline{unsigned int prg_nr} beschreibt die Anfrage an den Treiber, in diesem Fall \lstinline{KB_SET_VALUE}. Das Argument \lstinline{unsigned long usr_addr} enthält einen typ-agnostischen Pointer. Dieser verweist auf einen Speicherbereich, in welchem die zur Bearbeitung der Anfrage notwendigen Daten abgelegt sind. Hier können auch vom Treiber empfangene Daten dem Anwendungsprogramm bereitgestellt werden. 

Die switch-case-Anweisung führt die über das Argument \lstinline{prg_nr} spezifizierte Aktion aus. Hier betrachten wir \lstinline{KB_SET_VALUE}:
Zunächst wird in Zeile 868 der übergebene Zeiger \lstinline{usr_addr} mittels explizitem Typecast zu einem Zeiger des Typs \lstinline{SPIValue *} konvertiert. Da dieser auf Daten im Userspace verweist, ist beim Zugriff durch den Kernel-Treiber besondere Vorsicht geboten.
In Zeile 877 wird mittels Mutex das Prozessabbild \lstinline{piDev_g} für den Zugriff durch andere Threads oder Prozesse gesperrt.
\lstinline{my_rt_mutex_lock} verweist hierbei auf die Funktion \lstinline{rt_mutex_lock} aus \lstinline{linux/sched.h}\footnote{Offenbar wurde hier auch eine alternative Implementierung vorgesehen, siehe revpi\_common.h}

In Zeile 889 wird das Byte \lstinline{i8uValue_l}, welches den zu schreibenden Wert enthält in das Prozessabbild übertragen. Anschließend wird die Mutex auf \lstinline{piDev_g} wieder entsperrt.
\newpage

\begin{lstlisting}[language={c},firstnumber=62,caption={Auszug des Struct \lstinline{spiControlDev} in \lstinline{piControlMain.h}\label{lst:4-spiControlDev}}]
|>\tikzmarkin[set border color=martiniblue]{spiControlDev}<|typedef struct spiControlDev|>\tikzmarkend{spiControlDev}<| {
	// device driver stuff
	int init_step;
	enum revpi_machine machine_type;
	void *machine;
	struct cdev cdev;	// Char device structure
	struct device *dev;
	struct thermal_zone_device *thermal_zone;

	|>\tikzmarkin[set border color=martiniblue]{processImage}<|// process image stuff
	INT8U ai8uPI[KB_PI_LEN];
	INT8U ai8uPIDefault|>\tikzmarkin[set border color=martinired]{KB_PI_LEN_0}<|[KB_PI_LEN]|>\tikzmarkend{KB_PI_LEN_0}<|;
	struct rt_mutex lockPI;        |>\tikzmarkend{processImage}<|
	bool stopIO;
	piDevices *devs; |>\setcounter{lstnumber}{94}<|
} tpiControlDev;
\end{lstlisting}

Das Prozessabbild ist als Byte-Array der Länge \lstinline{KB_PI_LEN} in Listing~\ref{lst:4-spiControlDev} definiert. Konfigurationsparameter wie \lstinline{KB_PI_LEN} oder die Zykluszeit für den Datenaustausch zwischen SPS und IO-Modulen sind im folgenden Listing~\ref{lst:4-process} definiert.

\begin{lstlisting}[language={c},firstnumber=119,caption={Konfigurationsparameter des Prozessabbildes in project.h\label{lst:4-process}}]
#define INTERVAL_PI_GATE (5*1000*1000)  // 5 ms piGateCommunication |>\setcounter{lstnumber}{128}<|

#define INTERVAL_IO_COM (5*1000*1000)  // 5 ms piIoComm |>\setcounter{lstnumber}{132}<|

#define KB_PD_LEN       512
|>\tikzmarkin[set border color=martiniblue]{KB_PI_LEN_1}<|#define KB_PI_LEN       4096|>\tikzmarkend{KB_PI_LEN_1}<|
\end{lstlisting}

Das zu setzende Bit wurde zu diesem Zeitpunkt erfolgreich in das Prozessabbild der SPS geschrieben.
Es stellt sich die Frage, wie dieses nun an das IO-Modul kommuniziert wird.
Die Kommunikation mit allen angebundenen Modulen ist ebenfalls Aufgabe des piControl-Treibers.

\begin{lstlisting}[language={c},firstnumber=256,caption={Auszug der Methode \lstinline{piIoThread} in \lstinline{revpi_core.c}\label{lst:4-piIoThread}}]
static int piIoThread(void *data)
{
	//TODO int value = 0;
	ktime_t time;
	ktime_t now;
	s64 tDiff;

	hrtimer_init(&piCore_g.ioTimer, CLOCK_MONOTONIC, HRTIMER_MODE_ABS);
	piCore_g.ioTimer.function = piIoTimer;

	pr_info("piIO thread started\n");

	now = hrtimer_cb_get_time(&piCore_g.ioTimer);

	PiBridgeMaster_Reset();

	while (!kthread_should_stop()) {
		if (|>\tikzmarkin[set border color=martinired]{PiBridgeMaster}<|PiBridgeMaster_Run()|>\tikzmarkend{PiBridgeMaster}<| < 0)
			break;
	}

	RevPiDevice_finish();

	pr_info("piIO exit\n");
	return 0;
}
\end{lstlisting}

Der Kernel-Thread \lstinline{piIoThread} ist verantwortlich für den zyklischen Datenaustausch mit den IO-Modulen. In diesem wird fortlaufend die Methode \lstinline{PiBridgeMaster_Run()} aufgerufen, siehe Listing~\ref{lst:4-piIoThread}.

\begin{lstlisting}[language={c},firstnumber=262,caption={Auszug der Methode \lstinline{PiBridgeMaster_Run(void)} in \lstinline{RevPiDevice.c}\label{lst:4-PiBridgeMaster_Run}}]
int PiBridgeMaster_Run(void)
{
	static kbUT_Timer tTimeoutTimer_s;
	static kbUT_Timer tConfigTimeoutTimer_s;
	static int error_cnt;
	static INT8U last_led;
	static unsigned long last_update;
	int ret = 0;
	int i;

	my_rt_mutex_lock(&piCore_g.lockBridgeState);
	if (piCore_g.eBridgeState != piBridgeStop) {
		switch (eRunStatus_s) { |>\setcounter{lstnumber}{514}<|
		    case enPiBridgeMasterStatus_EndOfConfig:|>\setcounter{lstnumber}{621}<|
		    if (|>\tikzmarkin[set border color=martinired]{RevPiDevice}<|RevPiDevice_run()|>\tikzmarkend{RevPiDevice}<|) {
				// an error occured, check error limits |>\setcounter{lstnumber}{641}<|
			} else {
				ret = 1;
			}
			piCore_g.image.drv.i16uRS485ErrorCnt = RevPiDevice_getErrCnt();
			break;
\end{lstlisting}

Die in Listing~\ref{lst:4-PiBridgeMaster_Run} dargestellte Methode ist eine sog. State-Machine. Ist die Konfiguration der IO-Module erfolgreich abgeschlossen, so führt sie bei Aufruf lediglich die Methode \lstinline{RevPiDevice_run()} aus.

\begin{lstlisting}[language={c},firstnumber=140,caption={Auszug der Methode \lstinline{RevPiDevice_run(void)} in \lstinline{RevPiDevice.c}\label{lst:4-RevPiDevice_run}}]
int RevPiDevice_run(void)
{
	INT8U i8uDevice = 0;
	INT32U r;
	int retval = 0;

	RevPiDevices_s.i16uErrorCnt = 0;

	for (i8uDevice = 0; i8uDevice < RevPiDevice_getDevCnt(); i8uDevice++) {
		if (RevPiDevice_getDev(i8uDevice)->i8uActive) {
			switch (RevPiDevice_getDev(i8uDevice)->sId.i16uModulType) {
			case KUNBUS_FW_DESCR_TYP_PI_DIO_14:
			case KUNBUS_FW_DESCR_TYP_PI_DI_16:
			case KUNBUS_FW_DESCR_TYP_PI_DO_16:
				r = |>\tikzmarkin[set border color=martinired]{sendCyclicTelegram}<|piDIOComm_sendCyclicTelegram(i8uDevice)|>\tikzmarkend{sendCyclicTelegram}\setcounter{lstnumber}{166} <|;

				break; |>\setcounter{lstnumber}{216}<|
			}
		}
	} |>\setcounter{lstnumber}{227}<|
	return retval;
}
\end{lstlisting}

Diese iteriert wie in Listing~\ref{lst:4-RevPiDevice_run} abgebildete durch alle gegenwärtig in der SPS konfigurierten Module. Ist das aktuelle Modul als aktiv markiert, so wird anhand eines sog. Firmware-Descriptors entschieden, welche Methode für die Ansteuerung des Moduls aufzurufen ist.

\begin{lstlisting}[language={c},firstnumber=161,caption={Auszug der Methode \lstinline{piDIOComm_sendCyclicTelegram} in \lstinline{piDIOComm.c}\label{lst:4-sendCyclicTelegram}}]
INT32U piDIOComm_sendCyclicTelegram(INT8U i8uDevice_p)
{
	INT32U i32uRv_l = 0;
	SIOGeneric sRequest_l;
	SIOGeneric sResponse_l;
	INT8U len_l, data_out[18], i, p, data_in[70];
	INT8U i8uAddress;
	int ret; |>\setcounter{lstnumber}{239}<|
	
    |>\tikzmarkin[set border color=martinired]{piIoComm}<|ret = piIoComm_send((INT8U *) & sRequest_l, IOPROTOCOL_HEADER_LENGTH + len_l + 1);  |>\tikzmarkend{piIoComm}\setcounter{lstnumber}{298}<|
}
\end{lstlisting}

Im Falle des hier verwendeten DO-Moduls wird die in Listing~\ref{lst:4-sendCyclicTelegram} abgebildete Methode \lstinline{piDIOComm_sendCyclicTelegram()} aufgerufen. Dieser wird ein Zeiger auf das zu schreibende Gerät übergeben. 
Zunächst wird das Prozessabbild mittels eines proprietären, jedoch im Quellcode offen nachvollziehbaren Protokolls in ein \lstinline{sRequest_l} genanntes Byte-Array umgewandelt. Dieser Schritt ist in Listing~\ref{lst:4-sendCyclicTelegram} nicht abgebildet. Anschließend wird \lstinline{piIoComm_send()} ein Zeiger auf die so generierte Schreib-Anfrage übergeben.

\begin{lstlisting}[language={c},firstnumber=220,caption={Auszug der Methode \lstinline{piIOComm_send} in \lstinline{piIOComm.c}\label{lst:4-piIOComm_send}}]
int piIoComm_send(INT8U * buf_p, INT16U i16uLen_p)
{
	ssize_t write_l = 0;
	INT16U i16uSent_l = 0;|>\setcounter{lstnumber}{249}<|

	while (i16uSent_l < i16uLen_p) {
		write_l = vfs_write(piIoComm_fd_m, buf_p + i16uSent_l, i16uLen_p - i16uSent_l, &piIoComm_fd_m->f_pos);
		if (write_l < 0) {
			pr_info_serial("write error %d\n", (int)write_l);
			return -1;
		} 
		i16uSent_l += write_l;|>\setcounter{lstnumber}{263}<|
	}
	clear();
	vfs_fsync(piIoComm_fd_m, 1);
	return 0;
}
\end{lstlisting}

Listing~\ref{lst:4-piIOComm_send} zeigt die Implementierung von \lstinline{piIoComm_send()}. Diese Methode ist für das Schreiben der oben generierten Anfrage auf die seriellen Schnittstelle verantwortlich. Realisiert wird dies mittels der Methode \lstinline{vfs_write()}. Diese ist in \lstinline{<linux/fs.h>} definiert. Sie ermöglicht das Schreiben einer Datei im Userspace aus dem Kernel heraus. Geschrieben wird hier die Datei mit dem Deskriptor \lstinline{piIoComm_fd_m}.
Da die Funktion \lstinline{vfs_write()} durch andere Kernel-Tasks unterbrochen werden kann, ist nicht gewährleistet, dass die gesamte Anfrage mit nur einem Aufruf geschrieben wird. Die oben abgebildete while-Schleife stellt das vollständige Senden der Anfrage sicher.

\begin{lstlisting}[language={c},firstnumber=157,caption={Auszug der Methode \lstinline{piIOComm_open_serial} in \lstinline{piIOComm.c}\label{lst:4-piIOComm_open_serial}}]
int piIoComm_open_serial(void)
{   |>\setcounter{lstnumber}{167}<|
	struct file *fd;	/* Filedeskriptor */
	struct termios newtio;	/* Schnittstellenoptionen */

	|>\tikzmarkin[set border color=martiniblue]{fd}<|/* Port oeffnen - read/write, kein "controlling tty", 
	    Status von DCD ignorieren */
	fd = filp_open(|>\tikzmarkin[set border color=martinired]{tty}<|REV_PI_TTY_DEVICE|>\tikzmarkend{tty}<|, O_RDWR | O_NOCTTY, 0); |>\setcounter{lstnumber}{208}<|
	
	piIoComm_fd_m = fd;                                                      |>\tikzmarkend{fd}\setcounter{lstnumber}{217}<|

	return 0;
}
\end{lstlisting}

Der zum Schreiben auf die serielle Schnittstelle verwendete Datei-Deskriptor wird von der in Listing~\ref{lst:4-piIOComm_open_serial} abgebildeten Methode \lstinline{piIoComm_open_serial()} generiert. 

\begin{lstlisting}[language={c},firstnumber=45,caption={Definition der seriellen Schnittstelle in \lstinline{piIOComm.h}\label{lst:4-REV_PI_TTY_DEVICE}}]
#define REV_PI_TTY_DEVICE	"/dev/ttyAMA0"
\end{lstlisting}

Das in Listing~\ref{lst:4-REV_PI_TTY_DEVICE} definierte Macro verweist auf eine der seriellen Schnittstellen des RaspberryPi.
Die Implementierung des zugehörigen Schnittstellentreibers soll hier nicht weiter untersucht werden. Somit ist an dieser Stelle die Kette vom Setzen einer Variablen auf dem OPC-Server bis hin zur Aktualisierung des Prozessabbilds der IO-Module geschlossen.

% \begin{lstlisting}[language={c},firstnumber={226},caption={Setzen der Scheduler-Priorität auf SCHED\_FIFO in 
% revpi\_common.c\label{lst:2-sched_priority}}]
% param.sched_priority = ktprio->prio;
% ret = sched_setscheduler(child, SCHED_FIFO, &param);
% \end{lstlisting}
%% % Imports nur für Referenzenauflösung während des Schreibens! Vorm Kompilieren auskommentieren!
% \bibliography{0_hauptdatei}
% \input{1_einleitung}
% \input{2_grundlagen}
% \input{3_konzeption}
% \input{4_implementierung}
% \input{5_tests}
% \input{6_zusammenfassung}
% % Ende Imports

\section{Test des OPC-Servers im Gesamtsystem%
  \label{sec:5-tests}}

%% % Imports nur für Referenzenauflösung während des schreibens! Vorm Kompilieren auskommentieren!
% \bibliography{0_hauptdatei}
% \input{1_einleitung}
% \input{2_grundlagen}
% \input{3_konzeption}
% \input{4_implementierung}
% \input{5_tests}
% \input{6_zusammenfassung}
% % Ende Imports

\section{Zusammenfassung und Ausblick%
  \label{sec:6-fazit}}
Der folgende Abschnitt~\ref{sec:6-zusammenfassung} fasst die gewonnenen Erkenntnisse und den Stand der Implementierung zusammen.
Den Abschluss dieser Arbeit bildet der Ausblick in Abschnitt~\ref{sec:6-ausblick}.

\subsection{Zusammenfassung%
     \label{sec:6-zusammenfassung}}

\subsection{Ausblick%
     \label{sec:6-ausblick}}

% % Ende Imports

\section{Einleitung und Motivation%
  \label{sec:1-einleitung}}
Ziel dieses Projektes ist die Integration eines OPC-Servers mit einer auf Linux
basierenden speicherprogrammierbaren Steuerung (SPS). Angeschlossen an diese SPS
ist jeweils ein digitales Ein-/\,bzw.~Ausgabemodul. Die von diesen bereitgestellten
Ein-/\, bzw.~Ausgänge (IO) sollen in der Datenstruktur des OPC-Servers abgebildet
und über diesen für OPC-Clients les-/\,und schreibar sein. Weiterhin sollen einige
Funktionen zur Überwachung und Steuerung der an die SPS angeschlossenen Aktoren
und Sensoren direkt im OPC-Server implementiert werden.
Hiermit stellt dieses Projekt eine der Grundlagen für ein übergeordnetes Projekt,
die cloudbasierte Steuerung eines miniaturisierten Produktions-Systems, dar.

Der hier verwendete OPC-Server ist Teil des sog. open62541 Projekts. Er ist in C
geschrieben und implementiert bereits einen großen Teil der im OPC-UA-Standard
spezifizierten Funktionen.
Als SPS findet ein Revolution Pi 3 der Firma Kunbus Verwendung. Dieser integriert
ein sog. Compute Module der Raspberry Pi Foundation in ein industrietaugliches
Gehäuse und erlaubt die Erweiterung mittels IO- oder Gateway-Modulen. Über diese
erfolgt die Kommunikation mit weiteren Komponenten der Automatisierungstechnik.

Motiviert ist dieses Projekt durch die Beobachtung, dass die Verbreitung offener
Standards sowie freier Software auch in der Automatisierungstechnik zunimmt.
Linux ist ein freies Betriebssystem, OPC-UA ein offen zugänglicher, aktiv gepflegter
und weit verbreiteter Standard. Der Raspberry Pi findet sowohl bei Hobby-Anwendern als
auch in den Bereichen Forschung und Entwicklung sowie bei industriellen Anwendern
Verwendung. Dieses Projekt stellt somit eine für unterschiedliche Anwender interessante
Entwicklung dar.

Im Anschluss an diese einleitende Übersicht im Abschnitt~\ref{sec:1-einleitung} folgt
die Darstellung der wichtigsten Grundlagen in Abschnitt~\ref{sec:2-grundlagen}.
Aufbauend auf diesen Grundlagen folgt die konzeptuelle Ausarbeitung im Abschnitt~\ref{sec:3-konzeption}.
Die Umsetzung wird im Abschnitt~\ref{sec:4-implementierung} erläutert.
Die Leistungsfähigkeit der Implementierung wird in Abschnitt~\ref{sec:5-tests} untersucht.
Eine Zusammenfassung und ein Ausblick schließen die Arbeit in
Abschnitt~\ref{sec:6-fazit} ab. Eventuell noch benötigte Anhänge
finden sich in den Anhängen [...] bis [...].

% % % Imports nur für Referenzenauflösung während des Schreibens! Vorm Kompilieren auskommentieren!
% \bibliography{0_hauptdatei}
% % Mit \section{...} eröffnen wir einen neuen Abschnitt.
% Der Befehl setzt nicht nur den Text in einer größeren,
% fetten Schrift, sondern sorgt außerdem dafür, daß er im
% Inhaltsverzeichnis erscheint.
%
% Mit \label{...} erzeugen wir einen Bezeichner, mit dessen Hilfe
% wir später auf die Nummer des Abschnitts verweisen können (nämlich
% mit~\ref{...}).
%
% Das Kommentarzeichen hinter „Übersicht“ dient dazu, ein
% Leerzeichen zwischen „Übersicht“ und dem \label-Befehl
% zu vermeiden, das andernfalls sichtbar würde – z.B. im
% Inhaltsverzeichnis.
%

% % Imports nur für Referenzenauflösung während des Schreibens! Vorm Kompilieren auskommentieren!
% \bibliography{0_hauptdatei}
% \input{1_einleitung}
%\input{2_grundlagen}
%\input{3_konzeption}
%\input{4_implementierung}
%\input{5_tests}
%\input{6_zusammenfassung}
% % Ende Imports

\section{Einleitung und Motivation%
  \label{sec:1-einleitung}}
Ziel dieses Projektes ist die Integration eines OPC-Servers mit einer auf Linux
basierenden speicherprogrammierbaren Steuerung (SPS). Angeschlossen an diese SPS
ist jeweils ein digitales Ein-/\,bzw.~Ausgabemodul. Die von diesen bereitgestellten
Ein-/\, bzw.~Ausgänge (IO) sollen in der Datenstruktur des OPC-Servers abgebildet
und über diesen für OPC-Clients les-/\,und schreibar sein. Weiterhin sollen einige
Funktionen zur Überwachung und Steuerung der an die SPS angeschlossenen Aktoren
und Sensoren direkt im OPC-Server implementiert werden.
Hiermit stellt dieses Projekt eine der Grundlagen für ein übergeordnetes Projekt,
die cloudbasierte Steuerung eines miniaturisierten Produktions-Systems, dar.

Der hier verwendete OPC-Server ist Teil des sog. open62541 Projekts. Er ist in C
geschrieben und implementiert bereits einen großen Teil der im OPC-UA-Standard
spezifizierten Funktionen.
Als SPS findet ein Revolution Pi 3 der Firma Kunbus Verwendung. Dieser integriert
ein sog. Compute Module der Raspberry Pi Foundation in ein industrietaugliches
Gehäuse und erlaubt die Erweiterung mittels IO- oder Gateway-Modulen. Über diese
erfolgt die Kommunikation mit weiteren Komponenten der Automatisierungstechnik.

Motiviert ist dieses Projekt durch die Beobachtung, dass die Verbreitung offener
Standards sowie freier Software auch in der Automatisierungstechnik zunimmt.
Linux ist ein freies Betriebssystem, OPC-UA ein offen zugänglicher, aktiv gepflegter
und weit verbreiteter Standard. Der Raspberry Pi findet sowohl bei Hobby-Anwendern als
auch in den Bereichen Forschung und Entwicklung sowie bei industriellen Anwendern
Verwendung. Dieses Projekt stellt somit eine für unterschiedliche Anwender interessante
Entwicklung dar.

Im Anschluss an diese einleitende Übersicht im Abschnitt~\ref{sec:1-einleitung} folgt
die Darstellung der wichtigsten Grundlagen in Abschnitt~\ref{sec:2-grundlagen}.
Aufbauend auf diesen Grundlagen folgt die konzeptuelle Ausarbeitung im Abschnitt~\ref{sec:3-konzeption}.
Die Umsetzung wird im Abschnitt~\ref{sec:4-implementierung} erläutert.
Die Leistungsfähigkeit der Implementierung wird in Abschnitt~\ref{sec:5-tests} untersucht.
Eine Zusammenfassung und ein Ausblick schließen die Arbeit in
Abschnitt~\ref{sec:6-fazit} ab. Eventuell noch benötigte Anhänge
finden sich in den Anhängen [...] bis [...].

% % % Imports nur für Referenzenauflösung während des Schreibens! Vorm Kompilieren auskommentieren!
% \bibliography{0_hauptdatei}
% \input{1_einleitung}
% \input{2_grundlagen}
% \input{3_konzeption}
% \input{4_implementierung}
% \input{5_tests}
% \input{6_zusammenfassung}
% % Ende Imports

\section{Grundlagen%
  \label{sec:2-grundlagen}}

\subsection{Speicherprogrammierbare-Steuerung und Linux -- Revolution Pi%
     \label{sec:2-sps}}

\subsubsection{Kunbus RevolutionPi%
        \label{sec:2-revpi}}
Der RevolutionPi 3 ist eine speicherprogrammierbare Steuerung (SPS) des Herstellers
Kunbus GmbH. Kern dieser SPS ist das von der Raspberry Pi Foundation entwickelte
und vertriebene Raspberry Pi Compute Module 3. Dieses integriert ein Broadcom BCM2837
System-on-Chip (SoC) mit vier 1,2GHz Prozessorkernen, 1GB RAM, 4GB eMMC Anwendungsspeicher
und sonstige Peripherie in ein Modul im DDR2-SODIMM Formfaktor. Diese Spezifikationen
sind weitgehend identisch zu denen des ausgesprochen populären Raspberry Pi 3.
Der Revolution Pi profitiert daher von dem gleichen großen Angebot an Software
und Unterstützung wie der Raspberry Pi, ergänzt dessen Hardware jedoch um eine 24V
Spannungsversorgung, die Möglichkeit der Erweiterung durch mehrere industrietaugliche
Ein-/ Ausgabemodule und Gateways sowie ein Gehäuse zur Montage auf einer DIN-Schiene.
\begin{itemize}
  \item{Prozessor: BCM2837}
  \item{Taktfrequenz 1,2 GHz}
  \item{Anzahl Prozessorkerne: 4}
  \item{Arbeitsspeicher: 1 GByte}
  \item{eMMC Flash Speicher: 4 GByte}
  \item{Betriebssystem: Angepasstes Raspbian mit RT-Patch}
  \item{RTC mit 24h Pufferung über wartungsfreien Kondensator}
  \item{Treiber / API: Treiber schreibt zyklisch Prozessdaten in ein Prozessabbild, Zugriff auf Prozessabbild über Linux-Filesystem als API zu Fremdsoftware.}
  \item{Kommunikationsanschlüsse: 2 x USB 2.0 A (je 500 mA belastbar), 1 x Micro-USB, HDMI, Ethernet (RJ45) 10/100 Mbit/s}
  \item{Stromversorgung: min. 10,7 V, max. 28,8 V, maximal 10 Watt}
  \item{Zulässige Umgebungstemperatur: -40 bis +55 C}
  \item{Gehäuseabmessungen: (HxBxL) 96 mm x 22,5 mm x 110,5 mm (ohne gesteckte Stecker)}
  \item{ESD Schutz: 4 kV / 8 kV gemäß EN61131-2 und IEC 61000-6-2}
  \item{Surge / Burst Prüfungen: gemäß EN61131-2 und IEC 61000-6-2 eingekoppelt auf Versorgungsspannung, Ethernet und IO-Leitungen}
  \item{EMI Prüfungen: gemäß EN61131-2 und IEC 61000-6-2}
\end{itemize}

Kunbus bietet eine Auswahl an IO- und Gateway-Modulen zur Erweiterung des Revolution Pi an.
Gateways dienen der Kommunikation mit Systemen oder Komponenten der Automatisierungstechnik
über Protokolle wie PROFIBUS oder EtherCAT. IO-Module erlauben die Überwachung
und Steuerung von digitalen oder analogen Ein- und Ausgängen.

\subsubsection{Zugriff auf IO-Module%
        \label{sec:2-io}}
Der Zugriff auf die Ein- und Ausgänge der IO-Module erfolgt über ein Prozessabbild
und einen hierfür von Kunbus bereitgestellten Treiber, genannt piControl. Dieser
aktualisiert das Prozessabbild zyklisch. Die angestrebte Zykluszeit beträgt 5ms,
kann jedoch je nach Anzahl der angeschlossenen Module auch größer sein. Kunbus
garantiert bei drei IO-Modulen und zwei Gateway-Modulen eine Zykluszeit von 10 ms.
Jedes der IO-Module stellt ein eigenständiges eingebettetes System dar. Es verfügt
über einen Microcontroller, welcher die IOs bereitstellt und über einen RS485-Bus
mit dem Revolution Pi kommuniziert.
% https://revolution.kunbus.de/io-modul/

Lizenz: GPL
% https://github.com/RevolutionPi/piControl

\begin{lstlisting}[language={c},firstnumber={226},caption={Setzen der Scheduler-Priorität auf SCHED\_FIFO in revpi\_common.c\label{lst:2-sched_priority}}]
param.sched_priority = ktprio->prio;
ret = sched_setscheduler(child, SCHED_FIFO,
       &param);
\end{lstlisting}


\subsection{Echtzeit und Multithreading unter Linux -- preemptRT und posix%
     \label{sec:2-echtzeit}}


 Der Linux-Kernel verfügt über mehrere unterschiedliche Preemtion-Modelle:

\begin{itemize}
  \item No Forced Preemption (server):
  Ausgelegt auf maximal möglichen Durchsatz, lediglich Interrupts und
  System-Call-Returns bewirken Präemption.

  \item Voluntary Kernel Preemption (Desktop):
  Neben den implizit bevorrechtigten Interrupts und System-Call-Returns gibt es
  in diesem Modell weitere Abschnitte des Kernels in welchen Preämption explizit
  gestattet ist.

  \item Preemptible Kernel (Low-Latency Desktop):
  In diesem Modell ist der gesamte Kernel, mit Ausnahme sog.~kritischer Abschnitte
  präemptible. Nach jedem kritischen Abschnitt gibt es einen impliziten Präemptions-Punkt.

  \item Preemptible Kernel (Basic RT):
  Dieses Modell ist dem zuvor genannten sehr ähnlich, hier sind jedoch alle Interrupt-Handler
  als eigenständige Threads ausgeführt.

  \item Fully Preemptible Kernel (RT):
  Wie auch bei den beiden zuvor genannten Modellen ist hier der gesamte Kernel
  präemtible, die Anzahl und Dauer der nicht-präemtiblen kritischen Abschnitte
  ist auf ein notwendiges Minimum beschränkt. Alle Interrupt-Handler sind als
  eigenständige Threads ausgeführt, Spinlocks durch Sleeping-Spinlocks und Mutexe
  durch sog.~RT-Mutexe ersetzt.

\end{itemize}
\todo{Spinlocks und Mutexe sowie die RT-Varianten dieser erklären!}

Lediglich mit dem vollständig präemtiblen Kernel kann Echtzeit-Verhalten realisiert werden.

% https://wiki.linuxfoundation.org/realtime/documentation/technical_basics/preemption_models bzw kernel/Kconfig.preempt

\subsubsection{preemptRT%
        \label{sec:2-preemptRT}}
% https://wiki.linuxfoundation.org/realtime/documentation/technical_details/start
% https://wiki.linuxfoundation.org/realtime/documentation/technical_basics/start

Das dem PREEMPT RT Kernel zugrunde liegende Prinzip lässt sich in einer einfachen
Regel ausdrücken: Nur Code, welcher absolut nicht-präemtible sein darf, ist es
gestattet nicht-präemtible zu sein.
Das erklärte Ziel des PREEMPT\_RT Patches ist es folglich, die Menge des nicht-präemtiblen
Codes im Linux-Kernel auf das absolut notwendige Minimum zu reduzieren.

Dies wird durch Verwendung folgender Mechanismen erreicht:

\begin{itemize}
  \item Hochauflösende Timer
  \item Sleeping Spinlocks
  \item Threaded Interrupt Handlers
  \item rt\_mutex
  \item RCU
\end{itemize}


\subsubsection{posix%
        \label{sec:2-posix}}
Ist posix hier wirklich relevant? Debian bzw.~Raspbian sind weitgehend posix
kompatibel, aber wird es hier genutzt? -> JA, open62541 nutzt pthread.h
piControl nutzt kthread.h, und semaphore.h

\subsection{OPC-UA und open62541%
     \label{sec:2-opc}}

\subsubsection{OPC UA%
        \label{sec:2-opcua}}
Open Platform Communications (OPC) ist eine Familie von Standards zur herstellerunabhängigen
Kommunikation von Maschinen (M2M) in der Automatisierungstechnik. Die sog.~OPC Task Force, zu deren
Mitgliedern verschiedene große Firmen der Automatisierungsindustrie gehören, veröffentlichte
die OPC Specification Version 1.0 im August 1996.
Motiviert ist dieser offene Standard durch die Erkenntniss, dass die Anpassung der
zahlreichen Herstellerstandards an individuelle Infrastrukturen und Anlagen einen
großen Mehraufwand verursachen.
Die Wikipedia beschreibt das Anwendungsgebiet für OPC wie folgt:

\glqq{}OPC wird dort eingesetzt, wo Sensoren, Regler und Steuerungen verschiedener Hersteller
ein gemeinsames Netzwerk bilden. Ohne OPC benötigten zwei Geräte zum Datenaustausch
genaue Kenntnis über die Kommunikationsmöglichkeiten des Gegenübers. Erweiterungen
und Austausch gestalten sich entsprechend schwierig. Mit OPC genügt es, für jedes
Gerät genau einmal einen OPC-konformen Treiber zu schreiben. Idealerweise wird
dieser bereits vom Hersteller zur Verfügung gestellt. Ein OPC-Treiber lässt sich
ohne großen Anpassungsaufwand in beliebig große Steuer- und Überwachungssysteme
integrieren.

OPC unterteilt sich in verschiedene Unterstandards, die für den jeweiligen Anwendungsfall
unabhängig voneinander implementiert werden können. OPC lässt sich damit verwenden
für Echtzeitdaten (Überwachung), Datenarchivierung, Alarm-Meldungen und neuerdings
auch direkt zur Steuerung (Befehlsübermittlung).\grqq{}

OPC basiert in der ursprünglichen Spezifikation auf Microsofts DCOM-Spezifikation.
DCOM macht Funktionen und Objekte einer Anwendung anderen Anwendungen im Netzwerk
zugänglich. Der OPC-Standard definiert entsprechende DCOM-Objekte um mit anderen
OPC-Anwendungen Daten austauschen zu können. Die Verwendung von DCOM bindet Anwender
an Betriebssysteme von Microsoft. Die ursprüngliche OPC Spezifikation wird durch die
Entwicklung von OPC Unified Architecture (OPC UA) abgelöst.
OPC UA setzt auf einem eigenen Kommunikationionsstack auf, die Verwendung von DCOM
und damit die Bindung an Microsoft wurden aufgelöst.

Die OPC-UA-Architektur ist eine Service-orientierte Architektur (SOA), deren Struktur
aus mehreren Schichten besteht.

% Wikipedia
Das OPC-Informationsmodell ist nicht mehr nur eine Hierarchie aus Ordnern, Items
und Properties. Es ist ein sogenanntes Full-Mesh-Network aus Nodes, mit dem neben
den Nutzdaten eines Nodes auch Meta- und Diagnoseinformationen repräsentiert werden.
Ein Node ähnelt einem Objekt aus der objektorientierten Programmierung. Ein Node
kann Attribute besitzen, die gelesen werden können (Data Access (DA), Historical
Data Access (HDA)). Es ist möglich Methoden zu definieren und aufzurufen.
Eine Methode besitzt Aufrufargumente und Rückgabewerte. Sie wird durch ein Command
aufgerufen. Weiterhin werden Events unterstützt, die versendet werden können
(AE (Alarms \& Events), DA DataChange), um bestimmte Informationen zwischen Geräten
auszutauschen. Ein Event besitzt unter anderem einen Empfangszeitpunkt, eine Nachricht
und einen Schweregrad. Die o. g. Nodes werden sowohl für die Nutzdaten als auch
alle anderen Arten von Metadaten verwendet. Der damit modellierte OPC-Adressraum
beinhaltet nun auch ein Typmodell, mit dem sämtliche Datentypen spezifiziert werden.

% https://de.wikipedia.org/wiki/Open_Platform_Communications
% https://de.wikipedia.org/wiki/OPC_Unified_Architecture
% https://opcfoundation.org/developer-tools/specifications-unified-architecture
% Von Gerhard Gappmeier - ascolab GmbH, CC BY-SA 3.0, https://de.wikipedia.org/w/index.php?curid=1892069
\subsubsection{open62541%
        \label{sec:2-open62541}}
open62541 ist eine offene und freie Implementierung von OPC UA. Die in C geschriebene
Bibliothek stellt eine beständig zunehmende Anzahl der im OPC UA Standard definierten
Funktionen bereit. Sie kann sowohl zur Erstellung von OPC-Servern als auch -Clients
genutzt werden. Ergänzend zu der unter der Mozilla Public License v2.0 lizensierten
Bibliothek stellt das open62541 Projekt auch Beispielprogramme unter einer CC0 Lizenz
zur Verfügung.

Die Bibliothek eignet sich auch für die Entwicklung auf eingebetteten Systemen und
Microcontrollern. Je nach Umfang der gewünschten Funktionen und des OPC Informationsmodells
beträgt die Größe einer Server-Binary weniger als 100kb. %evtl. kürzen?

\todo{Nodes erklären! Evtl.~oben!}

Folgende Auswahl an Eigenschaften und Funktionen zeichnet die in dieser Arbeit verwendete
Version 0.3 von open62541 aus:
\begin{itemize}
  \item Kommunikationionsstack
  \begin{itemize}
      \item OPC UA Binär-Protokoll (HTTP oder SOAP werden gegenwärtig nicht unterstützt)
      \item Austauschbare Netzwerk-Schicht, welche die Verwendung eigener Netzwerk-APIs
      erlaubt.
      \item Verschlüsselte Kommunikationion
      \item Asynchrone Dienst-Anfragen im Client
  \end{itemize}
  \item Informationsmodell
  \begin{itemize}
    \item Unterstützung aller OPC UA Node-Typen, inkl.~Methoden
    \item Hinzufügen und Entfernen von Nodes und Referenzen zur Laufzeit.
    \item Vererbung und Instanziierung von Objekt- und Variablentypen
    \item Zugriffskontrolle auch für einzelne Nodes
  \end{itemize}
  \item Subscriptions
  \begin{itemize}
    \item Erlaubt die Überwachung (subscriptions / monitoreditems)
    \item Sehr geringer Ressourcenbedarf pro überwachtem Wert
  \end{itemize}
  \item Code-Generierung auf XML-Basis
  \begin{itemize}
    \item Erlaubt die Erstellung von Datentypen
    \item Erlaubt die Generierung des serverseitigen Informationsmodells
  \end{itemize}
\end{itemize}

% https://open62541.org/doc/0.3/


Mozilla Public License
CC0 Lizenz für Beispiele und Plugins

% https://open62541.org/doc/open62541-current.pdf
% https://open62541.org/

% % % Imports nur für Referenzenauflösung während des Schreibens! Vorm Kompilieren auskommentieren!
% \bibliography{0_hauptdatei}
% \input{1_einleitung}
% \input{2_grundlagen}
% \input{3_konzeption}
% \input{4_implementierung}
% \input{5_tests}
% \input{6_zusammenfassung}
% \input{anhang}
% % Ende Imports

\section{Systemkonzept%
  \label{sec:3-konzeption}}
Auf Basis der in Abschnitt \ref{sec:2-grundlagen} vorgestellten Möglichkeiten folgt nun die Ausarbeitung eines Konzepts.
In den folgenden Abschnitten soll näher auf zwei zentrale Aspekte eingegangen werden: Abschnitt~\ref{sec:3-anbindung} stellt Möglichkeiten zum Zugriff auf Variablen bzw.\,Werte im Prozessabbild des Revolution Pi vor; in Abschnitt~\ref{sec:3-integration} wird ein Konzept zur Bereitstellung dieser Variablen auf einem OPC-Server vorgestellt.

\subsection{Anbindung der IO an den OPC-Server%
     \label{sec:3-anbindung}}

Eine Webanwendung mit Bezeichnung PiCtory dient zur Konfiguration der I/O- und virtuellen Module des RevolutionPi. Die Konfiguration liegt im JSON-Format in der Datei \lstinline{/etc/revpi/config.rsc}. Der piControl-Treiber liest diese Datei beim Start. 
Der folgende Auszug aus der Manpage des piControl-Kernelmoduls beschreibt die von diesem zum Lesen und Schreiben einzelner Bits des Prozessabbildes bereitgestellten Funktionen~\citep[vgl.]{web-revpi-manpage}. Sie ist an dieser Stelle weitgehend ungekürzt zitiert, da sie die nutzbare Schnittstelle sehr kompakt beschreibt.

\begin{lstlisting}[breakindent=0pt, numbers=none, caption={Auszug aus der Revolution Pi Programmers Manual\label{lst:4-manpage}}]
KB_FIND_VARIABLE SPIVariable *argp
Find a variable in the process image by its name. A pointer to a structure of type SPIVariable must be passed as argument. [...]
The struct SPIVariable [...] is defined as 
typedef struct SPIVariableStr
{
    char strVarName[32]; // Variable name
    uint16_t i16uAddress; // Address of the byte in the process image
    uint8_t i8uBit; // 0-7 bit position, >= 8 whole byte
    uint16_t i16uLength; // length of the variable in bits.
    // Possible values are 1, 8, 16 and 32
} SPIVariable;

Set and get values of the process image
KB_GET_VALUE SPIValue *argp
[...]
KB_SET_VALUE SPIValue *argp
Write one bit or one byte to the process image [...].  This call is more efficient than the usual calls of seek and write because only one function call is necessary. If more than on application are writing bits in one output byte, this call is the only safe way to set a bit without overwriting the other bits because this call is doing a read-modify-write-cycle. 

The struct SPIValue used by this ioctl is defined as
typedef struct SPIValueStr
{
    uint16_t i16uAddress; // Address of the byte in the process image
    uint8_t i8uBit; // 0-7 bit position, >= 8 whole byte
    uint8_t i8uValue; // Value: 0/1 for bit access, whole byte otherwise
} SPIValue;
\end{lstlisting} 

Die oben beschriebenden Funtkionen \lstinline{KB_FIND_VARIABLE}, \lstinline{KB_GET_VALUE} und \lstinline{KB_SET_VALUE} ermöglichen einen einfachen und (lt.\,Manpage) effizienten Zugriff auf einzelne Bits des Prozessabbildes und damit auch auf die IO des RevolutionPi.
Der Zugriff des OPC-Servers auf das Prozessabbild soll daher mittels dieser Funktionen realisiert werden.
\lstinline{KB_FIND_VARIABLE} kann genutzt werden, um Adressen von Variablen im Prozessabbild mittels ihres Namens aufzulösen.
\lstinline{KB_GET_VALUE} und \lstinline{KB_SET_VALUE} ermöglichen den Zugriff auf die Werte dieser Variablen.


\subsection{Integration des OPC-Servers in das System%
     \label{sec:3-integration}}

open62541 bietet drei Möglichkeiten zum Abgleich von Variablen mit dem Prozessabbild~\citep[vgl.][Tutorials - Connecting a Variable with a Physical Process]{web-open62541}:
\begin{itemize}
    \item Manuelles oder zyklisches Aktualisieren
    \item Variable Value Callback
    \item Variable Datasource
\end{itemize}

Die zyklische Aktualisierung eines oder mehrerer Werte nimmt, abhängig von der Zykluszeit, viele Systemressourcen in Anspruch. Value Callbacks ermöglichen es, einen Variablenwert effizienter mit einer Ressource wie etwa einem Prozessabbild zu synchronisieren. An die Variable wird ein Callback angehängt, welches vor jedem Lesen und nach jedem Schreibvorgang ausgeführt wird.
Der Wert der Variablen wird weiterhin im Variablenknoten auf dem OPC-Server gespeichert, der Abgleich mit der verknüpften Ressource erfolgt durch die Callback-Methoden.

Sogenannte Datenquellen gehen noch einen Schritt weiter. Der Server leitet jede Lese- und Schreibanforderung direkt an eine Callback-Funktion weiter. Beim Lesen liefert der Rückruf eine Kopie des aktuellen Wertes. Die Datenquelle muss intern ein eigenes Speichermanagement implementieren.

Der Zugriff auf die Werte des Prozessabbildes erfolgt, wie in Abschnitt~\ref{sec:3-anbindung} beschrieben, über von piControl bereitgestellte Methoden. Um die durch open62541 gepflegte OPC-Datenstruktur und das durch piControl verwaltete Prozessabbild möglichst effektiv verknüpfen zu können, soll diese Interaktion mittels Datenquellen und den zugehörigen Callbacks implementiert werden.
% % % Imports nur für Referenzenauflösung während des Schreibens! Vorm Kompilieren auskommentieren!
% \bibliography{0_hauptdatei}
% \input{1_einleitung}
% \input{2_grundlagen}
% \input{3_konzeption}
% \input{4_implementierung}
% \input{5_tests}
% \input{6_zusammenfassung}
% \input{anhang}
% % Ende Imports

\section{Implementierung%
  \label{sec:4-implementierung}}
Das folgende Kapitel stellt in Auszügen die Implementierung des OPC-Servers sowie die Anbindung an die IO-Module
der SPS dar. Der Schwerpunkt liegt hierbei auf der Funktionsweise des piControl-Treibers und dessen Integration in das Projekt. Abschnitt~\ref{sec:4-picontrol} erklärt die zum Schreibens eines Bits verwendeten Funktionsaufrufe.
Zuvor soll jedoch in Abschnitt~\ref{sec:4-open62541} der Teil des OPC-Servers vorgestellt werden, welcher auf besagten Treiber zugreift. 

\subsection{Implementierung des OPC-Servers%
     \label{sec:4-open62541}}
Wie im vorangegangenen Abschnitt~\ref{sec:3-integration} begründet, soll die Verknüpfung zwischen dem Prozessabbild der SPS und den auf dem OPC-Server bereitgestellten Werten über sog.\,Datenquellen erfolgen. Hierzu ist zunächst eine Callback-Methode zu implementieren, welche bei einem Lese- oder Schreibzugriff auf eine Variable aufgerufen wird. Die Verknüpfung zwischen Callback-Methode und Variable muss manuell erfolgen.

\begin{lstlisting}[language={c},firstnumber=237,caption={Auszug der Methode \lstinline{linkDataSourceVariable} in \lstinline{variables.c}\label{lst:4-linkDataSourceVariable}}]
extern UA_StatusCode
 linkDataSourceVariable(UA_Server *server, UA_NodeId nodeId) {
     bool readonly = false;
     UA_DataSource dataSourceVariable;
     UA_StatusCode rc; |>\setcounter{lstnumber}{254}<|

     dataSourceVariable.read = readDataSourceVariable;
     if (!readonly)
        dataSourceVariable.write = writeDataSourceVariable;
     else
        dataSourceVariable.write = writeReadonlyDataSourceVariable;

     return UA_Server_setVariableNode_dataSource(server, nodeId, dataSourceVariable);
 }
\end{lstlisting}

\begin{figure}[h]
    \centering
    \includegraphics[width=0.42\textwidth]{doc/img/OPC_RevPiDO.pdf}
    \caption{Auszug des verwendeten Nodesets, hier Digitalausgang 1 des Versuchsaufbaus
      \label{fig:opc-do}}
\end{figure}

Die in Listing~\ref{lst:4-linkDataSourceVariable} abgebildete Methode \lstinline{linkDataSourceVariable()} erzeugt ein Struct vom Typ \lstinline{UA_DataSource}. In diesem werden dem Lesen und Schreiben einer OPC-Variablen entsprechende Callback-Methoden zugewiesen. Die Verknüpfung einer OPC-Variable, genauer ihrer NodeId, mit der zuvor definierten Datenquelle erfolgt über die von open62541 bereitgestellte Methode \lstinline{UA_Server_setVariableNode_dataSource()}. Vor dem Lesen und nach dem Schreiben dieser Variable werden von nun an die entsprechenden Callbacks aufgerufen.
     
\begin{lstlisting}[language={c},firstnumber=168,caption={Auszug des Callbacks \lstinline{writeDataSourceVariable} in \lstinline{variables.c}\label{lst:4-writeDataSourceVariable}}]  
extern UA_StatusCode
 writeDataSourceVariable(UA_Server *server,
            const UA_NodeId *sessionId, void *sessionContext,
            const UA_NodeId *nodeId, void *nodeContext,
            const UA_NumericRange *range, const UA_DataValue *dataValue) {

    UA_StatusCode retval  = UA_STATUSCODE_GOOD;
    UA_NodeId *nameNodeId = UA_malloc(sizeof(UA_NodeId));
    UA_QualifiedName nameQN = UA_QUALIFIEDNAME(1, "Name");
    UA_Variant nameVar;
    UA_Boolean bit;

    retval |= findSiblingByBrowsename(server, nodeId, &nameQN, nameNodeId);
    retval |= UA_Server_readValue(server, *nameNodeId, &nameVar);
    retval |= UA_Boolean_copy(dataValue->value.data, &bit);

    |>\tikzmarkin[set border color=martinired]{writeIO}<|PI_writeSingleIO(String_fromUA_String(nameVar.data), &bit, false);                                                 |>\tikzmarkend{writeIO}<|

    free(nameNodeId);
    return retval;
 }
\end{lstlisting}

Listing~\ref{lst:4-writeDataSourceVariable} zeigt die Callback-Methode, welche nach dem Schreiben einer Variablen auf dem OPC-Server aufgerufen wird.
Dieser Methode wird neben der NodeId der mit ihr verknüpften Variablen auch der Wert dieser in Form eines Zeigers auf ein Struct vom Typ \lstinline{UA_DataValue} übergeben.

Die Gestaltung des hier verwendeten Nodesets sieht vor, dass in einer OPC-Variablen \lstinline{"Name"} der Bezeichner des zu schreibenden Digitalausgangs hinterlegt ist, siehe Abbildung~\ref{fig:opc-do}. Dies erlaubt eine Rekonfiguration der Ein- und Ausgänge der SPS ohne Änderungen im Programmcode des OPC-Servers vornehmen zu müssen.
Es ist daher erforderlich, nach jedem Schreiben einer mit einem Digitalausgang verknüpften Variablen, hier \lstinline{"Value"}, dessen Bezeichner \lstinline{"Name"} abzufragen. 
Dies geschieht in den Zeilen 180 und 181.
Anschließend wird dieser Bezeichner sowie der zu schreibende Wert der Methode \lstinline{PI_writeSingleIO()} übergeben, welche wiederum die Interaktion mit piControl übernimmt (vgl. Abschnitt \ref{sec:4-picontrol}).
 
\subsection{Integration von piControl%
     \label{sec:4-picontrol}}
In Abschnitt~\ref{sec:2-io} wurde die Anbindung der IO-Module des Revolution Pi sowie die Funktionsweise von piControl aus Anwendersicht beschrieben. Die verfügbare Literatur beschränkt sich auch auf lediglich diese Sicht; eine weiterführende Dokumentation für Entwickler gibt es, neben der in Abschnitt~\ref{sec:3-anbindung} vorgestellten Manpage, nicht. 
In diesem Abschnitt soll daher der Quellcode von piControl sowie dessen Verwendung im Projekt genauer betrachtet werden.
Hierzu wird exemplarisch die in Abschnitt~\ref{sec:4-open62541} eingeführte Methode \lstinline{PI_writeSingleIO()} untersucht.
Diese Methode ermöglicht das Setzen eines einzelnen Bits im Prozessabbild der SPS, und damit das Schalten eines digitalen Ausgangs auf einem IO-Modul.
Die äquivalente Methode \lstinline{int piControlGetBitValue(SPIValue *pSpiValue)} zum Lesen eines Bits bzw. Eingangs funktioniert analog und soll daher an dieser Stelle nicht dediziert erörtert werden.

\begin{lstlisting}[language={c},firstnumber=97,
                   caption={Setzen eines phsikalischen, digitalen Ausgangs in \lstinline{revpi.c}
                   \label{lst:4-PI_writeSingleIO}}]
extern void PI_writeSingleIO(char *pszVariableName, bool *bit, bool verbose)
{
	int rc;
	SPIVariable sPiVariable;
	SPIValue sPIValue;

	strncpy(sPiVariable.strVarName, pszVariableName, sizeof(sPiVariable.strVarName));
	rc = piControlGetVariableInfo(&sPiVariable);
	if (rc < 0) {
		printf("Cannot find variable '%s'\n", pszVariableName);
		return;
	}

		sPIValue.i16uAddress = sPiVariable.i16uAddress;
		sPIValue.i8uBit = sPiVariable.i8uBit;
		sPIValue.i8uValue = *bit;
		rc = |>\tikzmarkin[set border color=martinired]{setBitValue}<|piControlSetBitValue(&sPIValue)|>\tikzmarkend{setBitValue}<|;
		if (rc < 0)
			printf("Set bit error %s\n", getWriteError(rc));
		else if (verbose)
			printf("Set bit %d on byte at offset %d. Value %d\n", sPIValue.i8uBit, sPIValue.i16uAddress,
			       sPIValue.i8uValue);
}
\end{lstlisting}

Der Programmcode in Listing~\ref{lst:4-PI_writeSingleIO} ist Teil des implementierten OPC-Servers. In diesem wird auf zwei Funktionen des piControl-Treibers zugegriffen. 
Beiden Methoden wird als Argument ein Zeiger auf ein Struct vom Typ \lstinline{SPIValue} übergeben. Der im Struct abgelegte Name wird mittels \lstinline{piControlGetVariableInfo(&sPIValue)} zu einer Adresse im Prozessabbild aufgelöst. Diese wird in \lstinline{sPIValue.i16uAdress} gespeichert. Der Wert der Variablen wird anschließend mittels \lstinline{piControlSetBitValue(&sPIValue)} an dieser Adresse in das Prozessabbild geschrieben.

\begin{lstlisting}[language={c},firstnumber=309,caption={Methode \lstinline{piControlSetBitValue} in \lstinline{piControlIf.c}\label{lst:4-piControlSetBitValue}}]
int |>\tikzmarkin[set border color=martiniblue]{setBitValueFcn}<|piControlSetBitValue(SPIValue *pSpiValue)|>\tikzmarkend{setBitValueFcn}<|
{
    piControlOpen();

    if (PiControlHandle_g < 0)
	    return -ENODEV;

    pSpiValue->i16uAddress += pSpiValue->i8uBit / 8;
    pSpiValue->i8uBit %= 8;

    if (|>\tikzmarkin[set border color=martinired]{ioctl}<|ioctl(PiControlHandle_g, KB_SET_VALUE, pSpiValue)|>\tikzmarkend{ioctl}<| < 0)
	    return errno;

    return 0;
}
\end{lstlisting}

Die in Listing~\ref{lst:4-piControlSetBitValue} dargestellte Methode \lstinline{piControlSetBitValue} ist lediglich eine Hüllfunktion (häufig auch als Wrapper-Funktion bezeichnet) für einen Aufruf des \lstinline{ioctl} Kernel-Moduls.
Folgende Parameter werden übergeben:
\lstinline{PiControlHandle_g} ist die Referenz auf die Geräte-Datei des piControl-Treibers. \lstinline{KB_SET_VALUE} ist das ioctl-Kommando zum Schreiben eines Bits in das Prozessabbild. Der Zeiger \lstinline{pSpiValue} verweist auf ein Struct des bereits vorgestellten Typs \lstinline{SPIValue}.

\begin{lstlisting}[language={c},firstnumber=80,caption={Methode \lstinline{piControlOpen} in \lstinline{piControlIf.c}\label{lst:4-piControlOpen}}]
void piControlOpen(void)
{
    /* open handle if needed */
    if (PiControlHandle_g < 0)
    {
	    |>\tikzmarkin[set border color=martiniblue]{PiControlHandle}<|PiControlHandle_g = open(PICONTROL_DEVICE, O_RDWR)|>\tikzmarkend{PiControlHandle}<|;
    }
}
\end{lstlisting}

Die in Listing~\ref{lst:4-piControlOpen} dargestellte Methode öffnet, sofern nicht bereits geschehen, die Geräte-Datei. Das Macro \lstinline{PICONTROL_DEVICE} verweist hierbei auf \lstinline{/dev/piControl0}.

\begin{lstlisting}[language={c},firstnumber=721,caption={Methode \lstinline{piControlIoctl} in \lstinline{piControlMain.c}\label{lst:4-piControlIoctl}}]
static long |>\tikzmarkin[set border color=martiniblue, below offset=0.9em]{piControlIoctl}<|piControlIoctl(struct file *file, unsigned int prg_nr, 
                           unsigned long usr_addr)                                      |>\tikzmarkend{piControlIoctl}<|
{
  int status = -EFAULT;
  tpiControlInst *priv;
  int timeout = 10000;	// ms

  if (prg_nr != KB_CONFIG_SEND && prg_nr != KB_CONFIG_START && !isRunning()) {
  	return -EAGAIN;
  }

  priv = (tpiControlInst *) file->private_data;

  if (prg_nr != KB_GET_LAST_MESSAGE) {
  	// clear old message
  	priv->pcErrorMessage[0] = 0;
  }

  switch (prg_nr) {|>\setcounter{lstnumber}{864}<|

    case |>\tikzmarkin[set border color=martiniblue]{KB_SET_VALUE}<|KB_SET_VALUE:|>\tikzmarkend{KB_SET_VALUE}<|
  		{
  			SPIValue *pValue = (SPIValue *) usr_addr;

  			if (!isRunning())
  				return -EFAULT;

  			if (pValue->i16uAddress >= KB_PI_LEN) {
  				status = -EFAULT;
  			} else {
  				INT8U i8uValue_l;
  				my_rt_mutex_lock(&piDev_g.lockPI);
  				i8uValue_l = piDev_g.ai8uPI[pValue->i16uAddress];

  				if (pValue->i8uBit >= 8) {
  					i8uValue_l = pValue->i8uValue;
  				} else {
  					if (pValue->i8uValue)
  						i8uValue_l |= (1 << pValue->i8uBit);
  					else
  						i8uValue_l &= ~(1 << pValue->i8uBit);
  				}

  				|>\tikzmarkin[set border color=martinired]{i8uValue}<|piDev_g.ai8uPI[pValue->i16uAddress] = i8uValue_l;|>\tikzmarkend{i8uValue}<|
  				rt_mutex_unlock(&piDev_g.lockPI);

  #ifdef VERBOSE
  				pr_info("piControlIoctl Addr=%u, bit=%u: %02x %02x\n", pValue->i16uAddress, pValue->i8uBit, pValue->i8uValue, i8uValue_l);
  #endif

  				status = 0;
  			}
  		}
  		break; |>\setcounter{lstnumber}{1314}<|

    default:
      pr_err("Invalid Ioctl");
      return (-EINVAL);
      break;

    }

    return status;
  }
\end{lstlisting}

Listing~\ref{lst:4-piControlIoctl} zeigt in Auszügen die ioctl-Methode des piControl Kernel-Treibers. Diese bekommt folgende Argumente übergeben: \lstinline{struct file *file} enthält den Verweis auf die Geräte-Datei, hier \lstinline{/dev/piControl0}. Der Wert von \lstinline{unsigned int prg_nr} beschreibt die Anfrage an den Treiber, in diesem Fall \lstinline{KB_SET_VALUE}. Das Argument \lstinline{unsigned long usr_addr} enthält einen typ-agnostischen Pointer. Dieser verweist auf einen Speicherbereich, in welchem die zur Bearbeitung der Anfrage notwendigen Daten abgelegt sind. Hier können auch vom Treiber empfangene Daten dem Anwendungsprogramm bereitgestellt werden. 

Die switch-case-Anweisung führt die über das Argument \lstinline{prg_nr} spezifizierte Aktion aus. Hier betrachten wir \lstinline{KB_SET_VALUE}:
Zunächst wird in Zeile 868 der übergebene Zeiger \lstinline{usr_addr} mittels explizitem Typecast zu einem Zeiger des Typs \lstinline{SPIValue *} konvertiert. Da dieser auf Daten im Userspace verweist, ist beim Zugriff durch den Kernel-Treiber besondere Vorsicht geboten.
In Zeile 877 wird mittels Mutex das Prozessabbild \lstinline{piDev_g} für den Zugriff durch andere Threads oder Prozesse gesperrt.
\lstinline{my_rt_mutex_lock} verweist hierbei auf die Funktion \lstinline{rt_mutex_lock} aus \lstinline{linux/sched.h}\footnote{Offenbar wurde hier auch eine alternative Implementierung vorgesehen, siehe revpi\_common.h}

In Zeile 889 wird das Byte \lstinline{i8uValue_l}, welches den zu schreibenden Wert enthält in das Prozessabbild übertragen. Anschließend wird die Mutex auf \lstinline{piDev_g} wieder entsperrt.
\newpage

\begin{lstlisting}[language={c},firstnumber=62,caption={Auszug des Struct \lstinline{spiControlDev} in \lstinline{piControlMain.h}\label{lst:4-spiControlDev}}]
|>\tikzmarkin[set border color=martiniblue]{spiControlDev}<|typedef struct spiControlDev|>\tikzmarkend{spiControlDev}<| {
	// device driver stuff
	int init_step;
	enum revpi_machine machine_type;
	void *machine;
	struct cdev cdev;	// Char device structure
	struct device *dev;
	struct thermal_zone_device *thermal_zone;

	|>\tikzmarkin[set border color=martiniblue]{processImage}<|// process image stuff
	INT8U ai8uPI[KB_PI_LEN];
	INT8U ai8uPIDefault|>\tikzmarkin[set border color=martinired]{KB_PI_LEN_0}<|[KB_PI_LEN]|>\tikzmarkend{KB_PI_LEN_0}<|;
	struct rt_mutex lockPI;        |>\tikzmarkend{processImage}<|
	bool stopIO;
	piDevices *devs; |>\setcounter{lstnumber}{94}<|
} tpiControlDev;
\end{lstlisting}

Das Prozessabbild ist als Byte-Array der Länge \lstinline{KB_PI_LEN} in Listing~\ref{lst:4-spiControlDev} definiert. Konfigurationsparameter wie \lstinline{KB_PI_LEN} oder die Zykluszeit für den Datenaustausch zwischen SPS und IO-Modulen sind im folgenden Listing~\ref{lst:4-process} definiert.

\begin{lstlisting}[language={c},firstnumber=119,caption={Konfigurationsparameter des Prozessabbildes in project.h\label{lst:4-process}}]
#define INTERVAL_PI_GATE (5*1000*1000)  // 5 ms piGateCommunication |>\setcounter{lstnumber}{128}<|

#define INTERVAL_IO_COM (5*1000*1000)  // 5 ms piIoComm |>\setcounter{lstnumber}{132}<|

#define KB_PD_LEN       512
|>\tikzmarkin[set border color=martiniblue]{KB_PI_LEN_1}<|#define KB_PI_LEN       4096|>\tikzmarkend{KB_PI_LEN_1}<|
\end{lstlisting}

Das zu setzende Bit wurde zu diesem Zeitpunkt erfolgreich in das Prozessabbild der SPS geschrieben.
Es stellt sich die Frage, wie dieses nun an das IO-Modul kommuniziert wird.
Die Kommunikation mit allen angebundenen Modulen ist ebenfalls Aufgabe des piControl-Treibers.

\begin{lstlisting}[language={c},firstnumber=256,caption={Auszug der Methode \lstinline{piIoThread} in \lstinline{revpi_core.c}\label{lst:4-piIoThread}}]
static int piIoThread(void *data)
{
	//TODO int value = 0;
	ktime_t time;
	ktime_t now;
	s64 tDiff;

	hrtimer_init(&piCore_g.ioTimer, CLOCK_MONOTONIC, HRTIMER_MODE_ABS);
	piCore_g.ioTimer.function = piIoTimer;

	pr_info("piIO thread started\n");

	now = hrtimer_cb_get_time(&piCore_g.ioTimer);

	PiBridgeMaster_Reset();

	while (!kthread_should_stop()) {
		if (|>\tikzmarkin[set border color=martinired]{PiBridgeMaster}<|PiBridgeMaster_Run()|>\tikzmarkend{PiBridgeMaster}<| < 0)
			break;
	}

	RevPiDevice_finish();

	pr_info("piIO exit\n");
	return 0;
}
\end{lstlisting}

Der Kernel-Thread \lstinline{piIoThread} ist verantwortlich für den zyklischen Datenaustausch mit den IO-Modulen. In diesem wird fortlaufend die Methode \lstinline{PiBridgeMaster_Run()} aufgerufen, siehe Listing~\ref{lst:4-piIoThread}.

\begin{lstlisting}[language={c},firstnumber=262,caption={Auszug der Methode \lstinline{PiBridgeMaster_Run(void)} in \lstinline{RevPiDevice.c}\label{lst:4-PiBridgeMaster_Run}}]
int PiBridgeMaster_Run(void)
{
	static kbUT_Timer tTimeoutTimer_s;
	static kbUT_Timer tConfigTimeoutTimer_s;
	static int error_cnt;
	static INT8U last_led;
	static unsigned long last_update;
	int ret = 0;
	int i;

	my_rt_mutex_lock(&piCore_g.lockBridgeState);
	if (piCore_g.eBridgeState != piBridgeStop) {
		switch (eRunStatus_s) { |>\setcounter{lstnumber}{514}<|
		    case enPiBridgeMasterStatus_EndOfConfig:|>\setcounter{lstnumber}{621}<|
		    if (|>\tikzmarkin[set border color=martinired]{RevPiDevice}<|RevPiDevice_run()|>\tikzmarkend{RevPiDevice}<|) {
				// an error occured, check error limits |>\setcounter{lstnumber}{641}<|
			} else {
				ret = 1;
			}
			piCore_g.image.drv.i16uRS485ErrorCnt = RevPiDevice_getErrCnt();
			break;
\end{lstlisting}

Die in Listing~\ref{lst:4-PiBridgeMaster_Run} dargestellte Methode ist eine sog. State-Machine. Ist die Konfiguration der IO-Module erfolgreich abgeschlossen, so führt sie bei Aufruf lediglich die Methode \lstinline{RevPiDevice_run()} aus.

\begin{lstlisting}[language={c},firstnumber=140,caption={Auszug der Methode \lstinline{RevPiDevice_run(void)} in \lstinline{RevPiDevice.c}\label{lst:4-RevPiDevice_run}}]
int RevPiDevice_run(void)
{
	INT8U i8uDevice = 0;
	INT32U r;
	int retval = 0;

	RevPiDevices_s.i16uErrorCnt = 0;

	for (i8uDevice = 0; i8uDevice < RevPiDevice_getDevCnt(); i8uDevice++) {
		if (RevPiDevice_getDev(i8uDevice)->i8uActive) {
			switch (RevPiDevice_getDev(i8uDevice)->sId.i16uModulType) {
			case KUNBUS_FW_DESCR_TYP_PI_DIO_14:
			case KUNBUS_FW_DESCR_TYP_PI_DI_16:
			case KUNBUS_FW_DESCR_TYP_PI_DO_16:
				r = |>\tikzmarkin[set border color=martinired]{sendCyclicTelegram}<|piDIOComm_sendCyclicTelegram(i8uDevice)|>\tikzmarkend{sendCyclicTelegram}\setcounter{lstnumber}{166} <|;

				break; |>\setcounter{lstnumber}{216}<|
			}
		}
	} |>\setcounter{lstnumber}{227}<|
	return retval;
}
\end{lstlisting}

Diese iteriert wie in Listing~\ref{lst:4-RevPiDevice_run} abgebildete durch alle gegenwärtig in der SPS konfigurierten Module. Ist das aktuelle Modul als aktiv markiert, so wird anhand eines sog. Firmware-Descriptors entschieden, welche Methode für die Ansteuerung des Moduls aufzurufen ist.

\begin{lstlisting}[language={c},firstnumber=161,caption={Auszug der Methode \lstinline{piDIOComm_sendCyclicTelegram} in \lstinline{piDIOComm.c}\label{lst:4-sendCyclicTelegram}}]
INT32U piDIOComm_sendCyclicTelegram(INT8U i8uDevice_p)
{
	INT32U i32uRv_l = 0;
	SIOGeneric sRequest_l;
	SIOGeneric sResponse_l;
	INT8U len_l, data_out[18], i, p, data_in[70];
	INT8U i8uAddress;
	int ret; |>\setcounter{lstnumber}{239}<|
	
    |>\tikzmarkin[set border color=martinired]{piIoComm}<|ret = piIoComm_send((INT8U *) & sRequest_l, IOPROTOCOL_HEADER_LENGTH + len_l + 1);  |>\tikzmarkend{piIoComm}\setcounter{lstnumber}{298}<|
}
\end{lstlisting}

Im Falle des hier verwendeten DO-Moduls wird die in Listing~\ref{lst:4-sendCyclicTelegram} abgebildete Methode \lstinline{piDIOComm_sendCyclicTelegram()} aufgerufen. Dieser wird ein Zeiger auf das zu schreibende Gerät übergeben. 
Zunächst wird das Prozessabbild mittels eines proprietären, jedoch im Quellcode offen nachvollziehbaren Protokolls in ein \lstinline{sRequest_l} genanntes Byte-Array umgewandelt. Dieser Schritt ist in Listing~\ref{lst:4-sendCyclicTelegram} nicht abgebildet. Anschließend wird \lstinline{piIoComm_send()} ein Zeiger auf die so generierte Schreib-Anfrage übergeben.

\begin{lstlisting}[language={c},firstnumber=220,caption={Auszug der Methode \lstinline{piIOComm_send} in \lstinline{piIOComm.c}\label{lst:4-piIOComm_send}}]
int piIoComm_send(INT8U * buf_p, INT16U i16uLen_p)
{
	ssize_t write_l = 0;
	INT16U i16uSent_l = 0;|>\setcounter{lstnumber}{249}<|

	while (i16uSent_l < i16uLen_p) {
		write_l = vfs_write(piIoComm_fd_m, buf_p + i16uSent_l, i16uLen_p - i16uSent_l, &piIoComm_fd_m->f_pos);
		if (write_l < 0) {
			pr_info_serial("write error %d\n", (int)write_l);
			return -1;
		} 
		i16uSent_l += write_l;|>\setcounter{lstnumber}{263}<|
	}
	clear();
	vfs_fsync(piIoComm_fd_m, 1);
	return 0;
}
\end{lstlisting}

Listing~\ref{lst:4-piIOComm_send} zeigt die Implementierung von \lstinline{piIoComm_send()}. Diese Methode ist für das Schreiben der oben generierten Anfrage auf die seriellen Schnittstelle verantwortlich. Realisiert wird dies mittels der Methode \lstinline{vfs_write()}. Diese ist in \lstinline{<linux/fs.h>} definiert. Sie ermöglicht das Schreiben einer Datei im Userspace aus dem Kernel heraus. Geschrieben wird hier die Datei mit dem Deskriptor \lstinline{piIoComm_fd_m}.
Da die Funktion \lstinline{vfs_write()} durch andere Kernel-Tasks unterbrochen werden kann, ist nicht gewährleistet, dass die gesamte Anfrage mit nur einem Aufruf geschrieben wird. Die oben abgebildete while-Schleife stellt das vollständige Senden der Anfrage sicher.

\begin{lstlisting}[language={c},firstnumber=157,caption={Auszug der Methode \lstinline{piIOComm_open_serial} in \lstinline{piIOComm.c}\label{lst:4-piIOComm_open_serial}}]
int piIoComm_open_serial(void)
{   |>\setcounter{lstnumber}{167}<|
	struct file *fd;	/* Filedeskriptor */
	struct termios newtio;	/* Schnittstellenoptionen */

	|>\tikzmarkin[set border color=martiniblue]{fd}<|/* Port oeffnen - read/write, kein "controlling tty", 
	    Status von DCD ignorieren */
	fd = filp_open(|>\tikzmarkin[set border color=martinired]{tty}<|REV_PI_TTY_DEVICE|>\tikzmarkend{tty}<|, O_RDWR | O_NOCTTY, 0); |>\setcounter{lstnumber}{208}<|
	
	piIoComm_fd_m = fd;                                                      |>\tikzmarkend{fd}\setcounter{lstnumber}{217}<|

	return 0;
}
\end{lstlisting}

Der zum Schreiben auf die serielle Schnittstelle verwendete Datei-Deskriptor wird von der in Listing~\ref{lst:4-piIOComm_open_serial} abgebildeten Methode \lstinline{piIoComm_open_serial()} generiert. 

\begin{lstlisting}[language={c},firstnumber=45,caption={Definition der seriellen Schnittstelle in \lstinline{piIOComm.h}\label{lst:4-REV_PI_TTY_DEVICE}}]
#define REV_PI_TTY_DEVICE	"/dev/ttyAMA0"
\end{lstlisting}

Das in Listing~\ref{lst:4-REV_PI_TTY_DEVICE} definierte Macro verweist auf eine der seriellen Schnittstellen des RaspberryPi.
Die Implementierung des zugehörigen Schnittstellentreibers soll hier nicht weiter untersucht werden. Somit ist an dieser Stelle die Kette vom Setzen einer Variablen auf dem OPC-Server bis hin zur Aktualisierung des Prozessabbilds der IO-Module geschlossen.

% \begin{lstlisting}[language={c},firstnumber={226},caption={Setzen der Scheduler-Priorität auf SCHED\_FIFO in 
% revpi\_common.c\label{lst:2-sched_priority}}]
% param.sched_priority = ktprio->prio;
% ret = sched_setscheduler(child, SCHED_FIFO, &param);
% \end{lstlisting}
% % % Imports nur für Referenzenauflösung während des Schreibens! Vorm Kompilieren auskommentieren!
% \bibliography{0_hauptdatei}
% \input{1_einleitung}
% \input{2_grundlagen}
% \input{3_konzeption}
% \input{4_implementierung}
% \input{5_tests}
% \input{6_zusammenfassung}
% % Ende Imports

\section{Test des OPC-Servers im Gesamtsystem%
  \label{sec:5-tests}}

% % % Imports nur für Referenzenauflösung während des schreibens! Vorm Kompilieren auskommentieren!
% \bibliography{0_hauptdatei}
% \input{1_einleitung}
% \input{2_grundlagen}
% \input{3_konzeption}
% \input{4_implementierung}
% \input{5_tests}
% \input{6_zusammenfassung}
% % Ende Imports

\section{Zusammenfassung und Ausblick%
  \label{sec:6-fazit}}
Der folgende Abschnitt~\ref{sec:6-zusammenfassung} fasst die gewonnenen Erkenntnisse und den Stand der Implementierung zusammen.
Den Abschluss dieser Arbeit bildet der Ausblick in Abschnitt~\ref{sec:6-ausblick}.

\subsection{Zusammenfassung%
     \label{sec:6-zusammenfassung}}

\subsection{Ausblick%
     \label{sec:6-ausblick}}

% % Ende Imports

\section{Grundlagen%
  \label{sec:2-grundlagen}}

\subsection{Speicherprogrammierbare-Steuerung und Linux -- Revolution Pi%
     \label{sec:2-sps}}

\subsubsection{Kunbus RevolutionPi%
        \label{sec:2-revpi}}
Der RevolutionPi 3 ist eine speicherprogrammierbare Steuerung (SPS) des Herstellers
Kunbus GmbH. Kern dieser SPS ist das von der Raspberry Pi Foundation entwickelte
und vertriebene Raspberry Pi Compute Module 3. Dieses integriert ein Broadcom BCM2837
System-on-Chip (SoC) mit vier 1,2GHz Prozessorkernen, 1GB RAM, 4GB eMMC Anwendungsspeicher
und sonstige Peripherie in ein Modul im DDR2-SODIMM Formfaktor. Diese Spezifikationen
sind weitgehend identisch zu denen des ausgesprochen populären Raspberry Pi 3.
Der Revolution Pi profitiert daher von dem gleichen großen Angebot an Software
und Unterstützung wie der Raspberry Pi, ergänzt dessen Hardware jedoch um eine 24V
Spannungsversorgung, die Möglichkeit der Erweiterung durch mehrere industrietaugliche
Ein-/ Ausgabemodule und Gateways sowie ein Gehäuse zur Montage auf einer DIN-Schiene.
\begin{itemize}
  \item{Prozessor: BCM2837}
  \item{Taktfrequenz 1,2 GHz}
  \item{Anzahl Prozessorkerne: 4}
  \item{Arbeitsspeicher: 1 GByte}
  \item{eMMC Flash Speicher: 4 GByte}
  \item{Betriebssystem: Angepasstes Raspbian mit RT-Patch}
  \item{RTC mit 24h Pufferung über wartungsfreien Kondensator}
  \item{Treiber / API: Treiber schreibt zyklisch Prozessdaten in ein Prozessabbild, Zugriff auf Prozessabbild über Linux-Filesystem als API zu Fremdsoftware.}
  \item{Kommunikationsanschlüsse: 2 x USB 2.0 A (je 500 mA belastbar), 1 x Micro-USB, HDMI, Ethernet (RJ45) 10/100 Mbit/s}
  \item{Stromversorgung: min. 10,7 V, max. 28,8 V, maximal 10 Watt}
  \item{Zulässige Umgebungstemperatur: -40 bis +55 C}
  \item{Gehäuseabmessungen: (HxBxL) 96 mm x 22,5 mm x 110,5 mm (ohne gesteckte Stecker)}
  \item{ESD Schutz: 4 kV / 8 kV gemäß EN61131-2 und IEC 61000-6-2}
  \item{Surge / Burst Prüfungen: gemäß EN61131-2 und IEC 61000-6-2 eingekoppelt auf Versorgungsspannung, Ethernet und IO-Leitungen}
  \item{EMI Prüfungen: gemäß EN61131-2 und IEC 61000-6-2}
\end{itemize}

Kunbus bietet eine Auswahl an IO- und Gateway-Modulen zur Erweiterung des Revolution Pi an.
Gateways dienen der Kommunikation mit Systemen oder Komponenten der Automatisierungstechnik
über Protokolle wie PROFIBUS oder EtherCAT. IO-Module erlauben die Überwachung
und Steuerung von digitalen oder analogen Ein- und Ausgängen.

\subsubsection{Zugriff auf IO-Module%
        \label{sec:2-io}}
Der Zugriff auf die Ein- und Ausgänge der IO-Module erfolgt über ein Prozessabbild
und einen hierfür von Kunbus bereitgestellten Treiber, genannt piControl. Dieser
aktualisiert das Prozessabbild zyklisch. Die angestrebte Zykluszeit beträgt 5ms,
kann jedoch je nach Anzahl der angeschlossenen Module auch größer sein. Kunbus
garantiert bei drei IO-Modulen und zwei Gateway-Modulen eine Zykluszeit von 10 ms.
Jedes der IO-Module stellt ein eigenständiges eingebettetes System dar. Es verfügt
über einen Microcontroller, welcher die IOs bereitstellt und über einen RS485-Bus
mit dem Revolution Pi kommuniziert.
% https://revolution.kunbus.de/io-modul/

Lizenz: GPL
% https://github.com/RevolutionPi/piControl

\begin{lstlisting}[language={c},firstnumber={226},caption={Setzen der Scheduler-Priorität auf SCHED\_FIFO in revpi\_common.c\label{lst:2-sched_priority}}]
param.sched_priority = ktprio->prio;
ret = sched_setscheduler(child, SCHED_FIFO,
       &param);
\end{lstlisting}


\subsection{Echtzeit und Multithreading unter Linux -- preemptRT und posix%
     \label{sec:2-echtzeit}}


 Der Linux-Kernel verfügt über mehrere unterschiedliche Preemtion-Modelle:

\begin{itemize}
  \item No Forced Preemption (server):
  Ausgelegt auf maximal möglichen Durchsatz, lediglich Interrupts und
  System-Call-Returns bewirken Präemption.

  \item Voluntary Kernel Preemption (Desktop):
  Neben den implizit bevorrechtigten Interrupts und System-Call-Returns gibt es
  in diesem Modell weitere Abschnitte des Kernels in welchen Preämption explizit
  gestattet ist.

  \item Preemptible Kernel (Low-Latency Desktop):
  In diesem Modell ist der gesamte Kernel, mit Ausnahme sog.~kritischer Abschnitte
  präemptible. Nach jedem kritischen Abschnitt gibt es einen impliziten Präemptions-Punkt.

  \item Preemptible Kernel (Basic RT):
  Dieses Modell ist dem zuvor genannten sehr ähnlich, hier sind jedoch alle Interrupt-Handler
  als eigenständige Threads ausgeführt.

  \item Fully Preemptible Kernel (RT):
  Wie auch bei den beiden zuvor genannten Modellen ist hier der gesamte Kernel
  präemtible, die Anzahl und Dauer der nicht-präemtiblen kritischen Abschnitte
  ist auf ein notwendiges Minimum beschränkt. Alle Interrupt-Handler sind als
  eigenständige Threads ausgeführt, Spinlocks durch Sleeping-Spinlocks und Mutexe
  durch sog.~RT-Mutexe ersetzt.

\end{itemize}
\todo{Spinlocks und Mutexe sowie die RT-Varianten dieser erklären!}

Lediglich mit dem vollständig präemtiblen Kernel kann Echtzeit-Verhalten realisiert werden.

% https://wiki.linuxfoundation.org/realtime/documentation/technical_basics/preemption_models bzw kernel/Kconfig.preempt

\subsubsection{preemptRT%
        \label{sec:2-preemptRT}}
% https://wiki.linuxfoundation.org/realtime/documentation/technical_details/start
% https://wiki.linuxfoundation.org/realtime/documentation/technical_basics/start

Das dem PREEMPT RT Kernel zugrunde liegende Prinzip lässt sich in einer einfachen
Regel ausdrücken: Nur Code, welcher absolut nicht-präemtible sein darf, ist es
gestattet nicht-präemtible zu sein.
Das erklärte Ziel des PREEMPT\_RT Patches ist es folglich, die Menge des nicht-präemtiblen
Codes im Linux-Kernel auf das absolut notwendige Minimum zu reduzieren.

Dies wird durch Verwendung folgender Mechanismen erreicht:

\begin{itemize}
  \item Hochauflösende Timer
  \item Sleeping Spinlocks
  \item Threaded Interrupt Handlers
  \item rt\_mutex
  \item RCU
\end{itemize}


\subsubsection{posix%
        \label{sec:2-posix}}
Ist posix hier wirklich relevant? Debian bzw.~Raspbian sind weitgehend posix
kompatibel, aber wird es hier genutzt? -> JA, open62541 nutzt pthread.h
piControl nutzt kthread.h, und semaphore.h

\subsection{OPC-UA und open62541%
     \label{sec:2-opc}}

\subsubsection{OPC UA%
        \label{sec:2-opcua}}
Open Platform Communications (OPC) ist eine Familie von Standards zur herstellerunabhängigen
Kommunikation von Maschinen (M2M) in der Automatisierungstechnik. Die sog.~OPC Task Force, zu deren
Mitgliedern verschiedene große Firmen der Automatisierungsindustrie gehören, veröffentlichte
die OPC Specification Version 1.0 im August 1996.
Motiviert ist dieser offene Standard durch die Erkenntniss, dass die Anpassung der
zahlreichen Herstellerstandards an individuelle Infrastrukturen und Anlagen einen
großen Mehraufwand verursachen.
Die Wikipedia beschreibt das Anwendungsgebiet für OPC wie folgt:

\glqq{}OPC wird dort eingesetzt, wo Sensoren, Regler und Steuerungen verschiedener Hersteller
ein gemeinsames Netzwerk bilden. Ohne OPC benötigten zwei Geräte zum Datenaustausch
genaue Kenntnis über die Kommunikationsmöglichkeiten des Gegenübers. Erweiterungen
und Austausch gestalten sich entsprechend schwierig. Mit OPC genügt es, für jedes
Gerät genau einmal einen OPC-konformen Treiber zu schreiben. Idealerweise wird
dieser bereits vom Hersteller zur Verfügung gestellt. Ein OPC-Treiber lässt sich
ohne großen Anpassungsaufwand in beliebig große Steuer- und Überwachungssysteme
integrieren.

OPC unterteilt sich in verschiedene Unterstandards, die für den jeweiligen Anwendungsfall
unabhängig voneinander implementiert werden können. OPC lässt sich damit verwenden
für Echtzeitdaten (Überwachung), Datenarchivierung, Alarm-Meldungen und neuerdings
auch direkt zur Steuerung (Befehlsübermittlung).\grqq{}

OPC basiert in der ursprünglichen Spezifikation auf Microsofts DCOM-Spezifikation.
DCOM macht Funktionen und Objekte einer Anwendung anderen Anwendungen im Netzwerk
zugänglich. Der OPC-Standard definiert entsprechende DCOM-Objekte um mit anderen
OPC-Anwendungen Daten austauschen zu können. Die Verwendung von DCOM bindet Anwender
an Betriebssysteme von Microsoft. Die ursprüngliche OPC Spezifikation wird durch die
Entwicklung von OPC Unified Architecture (OPC UA) abgelöst.
OPC UA setzt auf einem eigenen Kommunikationionsstack auf, die Verwendung von DCOM
und damit die Bindung an Microsoft wurden aufgelöst.

Die OPC-UA-Architektur ist eine Service-orientierte Architektur (SOA), deren Struktur
aus mehreren Schichten besteht.

% Wikipedia
Das OPC-Informationsmodell ist nicht mehr nur eine Hierarchie aus Ordnern, Items
und Properties. Es ist ein sogenanntes Full-Mesh-Network aus Nodes, mit dem neben
den Nutzdaten eines Nodes auch Meta- und Diagnoseinformationen repräsentiert werden.
Ein Node ähnelt einem Objekt aus der objektorientierten Programmierung. Ein Node
kann Attribute besitzen, die gelesen werden können (Data Access (DA), Historical
Data Access (HDA)). Es ist möglich Methoden zu definieren und aufzurufen.
Eine Methode besitzt Aufrufargumente und Rückgabewerte. Sie wird durch ein Command
aufgerufen. Weiterhin werden Events unterstützt, die versendet werden können
(AE (Alarms \& Events), DA DataChange), um bestimmte Informationen zwischen Geräten
auszutauschen. Ein Event besitzt unter anderem einen Empfangszeitpunkt, eine Nachricht
und einen Schweregrad. Die o. g. Nodes werden sowohl für die Nutzdaten als auch
alle anderen Arten von Metadaten verwendet. Der damit modellierte OPC-Adressraum
beinhaltet nun auch ein Typmodell, mit dem sämtliche Datentypen spezifiziert werden.

% https://de.wikipedia.org/wiki/Open_Platform_Communications
% https://de.wikipedia.org/wiki/OPC_Unified_Architecture
% https://opcfoundation.org/developer-tools/specifications-unified-architecture
% Von Gerhard Gappmeier - ascolab GmbH, CC BY-SA 3.0, https://de.wikipedia.org/w/index.php?curid=1892069
\subsubsection{open62541%
        \label{sec:2-open62541}}
open62541 ist eine offene und freie Implementierung von OPC UA. Die in C geschriebene
Bibliothek stellt eine beständig zunehmende Anzahl der im OPC UA Standard definierten
Funktionen bereit. Sie kann sowohl zur Erstellung von OPC-Servern als auch -Clients
genutzt werden. Ergänzend zu der unter der Mozilla Public License v2.0 lizensierten
Bibliothek stellt das open62541 Projekt auch Beispielprogramme unter einer CC0 Lizenz
zur Verfügung.

Die Bibliothek eignet sich auch für die Entwicklung auf eingebetteten Systemen und
Microcontrollern. Je nach Umfang der gewünschten Funktionen und des OPC Informationsmodells
beträgt die Größe einer Server-Binary weniger als 100kb. %evtl. kürzen?

\todo{Nodes erklären! Evtl.~oben!}

Folgende Auswahl an Eigenschaften und Funktionen zeichnet die in dieser Arbeit verwendete
Version 0.3 von open62541 aus:
\begin{itemize}
  \item Kommunikationionsstack
  \begin{itemize}
      \item OPC UA Binär-Protokoll (HTTP oder SOAP werden gegenwärtig nicht unterstützt)
      \item Austauschbare Netzwerk-Schicht, welche die Verwendung eigener Netzwerk-APIs
      erlaubt.
      \item Verschlüsselte Kommunikationion
      \item Asynchrone Dienst-Anfragen im Client
  \end{itemize}
  \item Informationsmodell
  \begin{itemize}
    \item Unterstützung aller OPC UA Node-Typen, inkl.~Methoden
    \item Hinzufügen und Entfernen von Nodes und Referenzen zur Laufzeit.
    \item Vererbung und Instanziierung von Objekt- und Variablentypen
    \item Zugriffskontrolle auch für einzelne Nodes
  \end{itemize}
  \item Subscriptions
  \begin{itemize}
    \item Erlaubt die Überwachung (subscriptions / monitoreditems)
    \item Sehr geringer Ressourcenbedarf pro überwachtem Wert
  \end{itemize}
  \item Code-Generierung auf XML-Basis
  \begin{itemize}
    \item Erlaubt die Erstellung von Datentypen
    \item Erlaubt die Generierung des serverseitigen Informationsmodells
  \end{itemize}
\end{itemize}

% https://open62541.org/doc/0.3/


Mozilla Public License
CC0 Lizenz für Beispiele und Plugins

% https://open62541.org/doc/open62541-current.pdf
% https://open62541.org/

% % % Imports nur für Referenzenauflösung während des Schreibens! Vorm Kompilieren auskommentieren!
% \bibliography{0_hauptdatei}
% % Mit \section{...} eröffnen wir einen neuen Abschnitt.
% Der Befehl setzt nicht nur den Text in einer größeren,
% fetten Schrift, sondern sorgt außerdem dafür, daß er im
% Inhaltsverzeichnis erscheint.
%
% Mit \label{...} erzeugen wir einen Bezeichner, mit dessen Hilfe
% wir später auf die Nummer des Abschnitts verweisen können (nämlich
% mit~\ref{...}).
%
% Das Kommentarzeichen hinter „Übersicht“ dient dazu, ein
% Leerzeichen zwischen „Übersicht“ und dem \label-Befehl
% zu vermeiden, das andernfalls sichtbar würde – z.B. im
% Inhaltsverzeichnis.
%

% % Imports nur für Referenzenauflösung während des Schreibens! Vorm Kompilieren auskommentieren!
% \bibliography{0_hauptdatei}
% \input{1_einleitung}
%\input{2_grundlagen}
%\input{3_konzeption}
%\input{4_implementierung}
%\input{5_tests}
%\input{6_zusammenfassung}
% % Ende Imports

\section{Einleitung und Motivation%
  \label{sec:1-einleitung}}
Ziel dieses Projektes ist die Integration eines OPC-Servers mit einer auf Linux
basierenden speicherprogrammierbaren Steuerung (SPS). Angeschlossen an diese SPS
ist jeweils ein digitales Ein-/\,bzw.~Ausgabemodul. Die von diesen bereitgestellten
Ein-/\, bzw.~Ausgänge (IO) sollen in der Datenstruktur des OPC-Servers abgebildet
und über diesen für OPC-Clients les-/\,und schreibar sein. Weiterhin sollen einige
Funktionen zur Überwachung und Steuerung der an die SPS angeschlossenen Aktoren
und Sensoren direkt im OPC-Server implementiert werden.
Hiermit stellt dieses Projekt eine der Grundlagen für ein übergeordnetes Projekt,
die cloudbasierte Steuerung eines miniaturisierten Produktions-Systems, dar.

Der hier verwendete OPC-Server ist Teil des sog. open62541 Projekts. Er ist in C
geschrieben und implementiert bereits einen großen Teil der im OPC-UA-Standard
spezifizierten Funktionen.
Als SPS findet ein Revolution Pi 3 der Firma Kunbus Verwendung. Dieser integriert
ein sog. Compute Module der Raspberry Pi Foundation in ein industrietaugliches
Gehäuse und erlaubt die Erweiterung mittels IO- oder Gateway-Modulen. Über diese
erfolgt die Kommunikation mit weiteren Komponenten der Automatisierungstechnik.

Motiviert ist dieses Projekt durch die Beobachtung, dass die Verbreitung offener
Standards sowie freier Software auch in der Automatisierungstechnik zunimmt.
Linux ist ein freies Betriebssystem, OPC-UA ein offen zugänglicher, aktiv gepflegter
und weit verbreiteter Standard. Der Raspberry Pi findet sowohl bei Hobby-Anwendern als
auch in den Bereichen Forschung und Entwicklung sowie bei industriellen Anwendern
Verwendung. Dieses Projekt stellt somit eine für unterschiedliche Anwender interessante
Entwicklung dar.

Im Anschluss an diese einleitende Übersicht im Abschnitt~\ref{sec:1-einleitung} folgt
die Darstellung der wichtigsten Grundlagen in Abschnitt~\ref{sec:2-grundlagen}.
Aufbauend auf diesen Grundlagen folgt die konzeptuelle Ausarbeitung im Abschnitt~\ref{sec:3-konzeption}.
Die Umsetzung wird im Abschnitt~\ref{sec:4-implementierung} erläutert.
Die Leistungsfähigkeit der Implementierung wird in Abschnitt~\ref{sec:5-tests} untersucht.
Eine Zusammenfassung und ein Ausblick schließen die Arbeit in
Abschnitt~\ref{sec:6-fazit} ab. Eventuell noch benötigte Anhänge
finden sich in den Anhängen [...] bis [...].

% % % Imports nur für Referenzenauflösung während des Schreibens! Vorm Kompilieren auskommentieren!
% \bibliography{0_hauptdatei}
% \input{1_einleitung}
% \input{2_grundlagen}
% \input{3_konzeption}
% \input{4_implementierung}
% \input{5_tests}
% \input{6_zusammenfassung}
% % Ende Imports

\section{Grundlagen%
  \label{sec:2-grundlagen}}

\subsection{Speicherprogrammierbare-Steuerung und Linux -- Revolution Pi%
     \label{sec:2-sps}}

\subsubsection{Kunbus RevolutionPi%
        \label{sec:2-revpi}}
Der RevolutionPi 3 ist eine speicherprogrammierbare Steuerung (SPS) des Herstellers
Kunbus GmbH. Kern dieser SPS ist das von der Raspberry Pi Foundation entwickelte
und vertriebene Raspberry Pi Compute Module 3. Dieses integriert ein Broadcom BCM2837
System-on-Chip (SoC) mit vier 1,2GHz Prozessorkernen, 1GB RAM, 4GB eMMC Anwendungsspeicher
und sonstige Peripherie in ein Modul im DDR2-SODIMM Formfaktor. Diese Spezifikationen
sind weitgehend identisch zu denen des ausgesprochen populären Raspberry Pi 3.
Der Revolution Pi profitiert daher von dem gleichen großen Angebot an Software
und Unterstützung wie der Raspberry Pi, ergänzt dessen Hardware jedoch um eine 24V
Spannungsversorgung, die Möglichkeit der Erweiterung durch mehrere industrietaugliche
Ein-/ Ausgabemodule und Gateways sowie ein Gehäuse zur Montage auf einer DIN-Schiene.
\begin{itemize}
  \item{Prozessor: BCM2837}
  \item{Taktfrequenz 1,2 GHz}
  \item{Anzahl Prozessorkerne: 4}
  \item{Arbeitsspeicher: 1 GByte}
  \item{eMMC Flash Speicher: 4 GByte}
  \item{Betriebssystem: Angepasstes Raspbian mit RT-Patch}
  \item{RTC mit 24h Pufferung über wartungsfreien Kondensator}
  \item{Treiber / API: Treiber schreibt zyklisch Prozessdaten in ein Prozessabbild, Zugriff auf Prozessabbild über Linux-Filesystem als API zu Fremdsoftware.}
  \item{Kommunikationsanschlüsse: 2 x USB 2.0 A (je 500 mA belastbar), 1 x Micro-USB, HDMI, Ethernet (RJ45) 10/100 Mbit/s}
  \item{Stromversorgung: min. 10,7 V, max. 28,8 V, maximal 10 Watt}
  \item{Zulässige Umgebungstemperatur: -40 bis +55 C}
  \item{Gehäuseabmessungen: (HxBxL) 96 mm x 22,5 mm x 110,5 mm (ohne gesteckte Stecker)}
  \item{ESD Schutz: 4 kV / 8 kV gemäß EN61131-2 und IEC 61000-6-2}
  \item{Surge / Burst Prüfungen: gemäß EN61131-2 und IEC 61000-6-2 eingekoppelt auf Versorgungsspannung, Ethernet und IO-Leitungen}
  \item{EMI Prüfungen: gemäß EN61131-2 und IEC 61000-6-2}
\end{itemize}

Kunbus bietet eine Auswahl an IO- und Gateway-Modulen zur Erweiterung des Revolution Pi an.
Gateways dienen der Kommunikation mit Systemen oder Komponenten der Automatisierungstechnik
über Protokolle wie PROFIBUS oder EtherCAT. IO-Module erlauben die Überwachung
und Steuerung von digitalen oder analogen Ein- und Ausgängen.

\subsubsection{Zugriff auf IO-Module%
        \label{sec:2-io}}
Der Zugriff auf die Ein- und Ausgänge der IO-Module erfolgt über ein Prozessabbild
und einen hierfür von Kunbus bereitgestellten Treiber, genannt piControl. Dieser
aktualisiert das Prozessabbild zyklisch. Die angestrebte Zykluszeit beträgt 5ms,
kann jedoch je nach Anzahl der angeschlossenen Module auch größer sein. Kunbus
garantiert bei drei IO-Modulen und zwei Gateway-Modulen eine Zykluszeit von 10 ms.
Jedes der IO-Module stellt ein eigenständiges eingebettetes System dar. Es verfügt
über einen Microcontroller, welcher die IOs bereitstellt und über einen RS485-Bus
mit dem Revolution Pi kommuniziert.
% https://revolution.kunbus.de/io-modul/

Lizenz: GPL
% https://github.com/RevolutionPi/piControl

\begin{lstlisting}[language={c},firstnumber={226},caption={Setzen der Scheduler-Priorität auf SCHED\_FIFO in revpi\_common.c\label{lst:2-sched_priority}}]
param.sched_priority = ktprio->prio;
ret = sched_setscheduler(child, SCHED_FIFO,
       &param);
\end{lstlisting}


\subsection{Echtzeit und Multithreading unter Linux -- preemptRT und posix%
     \label{sec:2-echtzeit}}


 Der Linux-Kernel verfügt über mehrere unterschiedliche Preemtion-Modelle:

\begin{itemize}
  \item No Forced Preemption (server):
  Ausgelegt auf maximal möglichen Durchsatz, lediglich Interrupts und
  System-Call-Returns bewirken Präemption.

  \item Voluntary Kernel Preemption (Desktop):
  Neben den implizit bevorrechtigten Interrupts und System-Call-Returns gibt es
  in diesem Modell weitere Abschnitte des Kernels in welchen Preämption explizit
  gestattet ist.

  \item Preemptible Kernel (Low-Latency Desktop):
  In diesem Modell ist der gesamte Kernel, mit Ausnahme sog.~kritischer Abschnitte
  präemptible. Nach jedem kritischen Abschnitt gibt es einen impliziten Präemptions-Punkt.

  \item Preemptible Kernel (Basic RT):
  Dieses Modell ist dem zuvor genannten sehr ähnlich, hier sind jedoch alle Interrupt-Handler
  als eigenständige Threads ausgeführt.

  \item Fully Preemptible Kernel (RT):
  Wie auch bei den beiden zuvor genannten Modellen ist hier der gesamte Kernel
  präemtible, die Anzahl und Dauer der nicht-präemtiblen kritischen Abschnitte
  ist auf ein notwendiges Minimum beschränkt. Alle Interrupt-Handler sind als
  eigenständige Threads ausgeführt, Spinlocks durch Sleeping-Spinlocks und Mutexe
  durch sog.~RT-Mutexe ersetzt.

\end{itemize}
\todo{Spinlocks und Mutexe sowie die RT-Varianten dieser erklären!}

Lediglich mit dem vollständig präemtiblen Kernel kann Echtzeit-Verhalten realisiert werden.

% https://wiki.linuxfoundation.org/realtime/documentation/technical_basics/preemption_models bzw kernel/Kconfig.preempt

\subsubsection{preemptRT%
        \label{sec:2-preemptRT}}
% https://wiki.linuxfoundation.org/realtime/documentation/technical_details/start
% https://wiki.linuxfoundation.org/realtime/documentation/technical_basics/start

Das dem PREEMPT RT Kernel zugrunde liegende Prinzip lässt sich in einer einfachen
Regel ausdrücken: Nur Code, welcher absolut nicht-präemtible sein darf, ist es
gestattet nicht-präemtible zu sein.
Das erklärte Ziel des PREEMPT\_RT Patches ist es folglich, die Menge des nicht-präemtiblen
Codes im Linux-Kernel auf das absolut notwendige Minimum zu reduzieren.

Dies wird durch Verwendung folgender Mechanismen erreicht:

\begin{itemize}
  \item Hochauflösende Timer
  \item Sleeping Spinlocks
  \item Threaded Interrupt Handlers
  \item rt\_mutex
  \item RCU
\end{itemize}


\subsubsection{posix%
        \label{sec:2-posix}}
Ist posix hier wirklich relevant? Debian bzw.~Raspbian sind weitgehend posix
kompatibel, aber wird es hier genutzt? -> JA, open62541 nutzt pthread.h
piControl nutzt kthread.h, und semaphore.h

\subsection{OPC-UA und open62541%
     \label{sec:2-opc}}

\subsubsection{OPC UA%
        \label{sec:2-opcua}}
Open Platform Communications (OPC) ist eine Familie von Standards zur herstellerunabhängigen
Kommunikation von Maschinen (M2M) in der Automatisierungstechnik. Die sog.~OPC Task Force, zu deren
Mitgliedern verschiedene große Firmen der Automatisierungsindustrie gehören, veröffentlichte
die OPC Specification Version 1.0 im August 1996.
Motiviert ist dieser offene Standard durch die Erkenntniss, dass die Anpassung der
zahlreichen Herstellerstandards an individuelle Infrastrukturen und Anlagen einen
großen Mehraufwand verursachen.
Die Wikipedia beschreibt das Anwendungsgebiet für OPC wie folgt:

\glqq{}OPC wird dort eingesetzt, wo Sensoren, Regler und Steuerungen verschiedener Hersteller
ein gemeinsames Netzwerk bilden. Ohne OPC benötigten zwei Geräte zum Datenaustausch
genaue Kenntnis über die Kommunikationsmöglichkeiten des Gegenübers. Erweiterungen
und Austausch gestalten sich entsprechend schwierig. Mit OPC genügt es, für jedes
Gerät genau einmal einen OPC-konformen Treiber zu schreiben. Idealerweise wird
dieser bereits vom Hersteller zur Verfügung gestellt. Ein OPC-Treiber lässt sich
ohne großen Anpassungsaufwand in beliebig große Steuer- und Überwachungssysteme
integrieren.

OPC unterteilt sich in verschiedene Unterstandards, die für den jeweiligen Anwendungsfall
unabhängig voneinander implementiert werden können. OPC lässt sich damit verwenden
für Echtzeitdaten (Überwachung), Datenarchivierung, Alarm-Meldungen und neuerdings
auch direkt zur Steuerung (Befehlsübermittlung).\grqq{}

OPC basiert in der ursprünglichen Spezifikation auf Microsofts DCOM-Spezifikation.
DCOM macht Funktionen und Objekte einer Anwendung anderen Anwendungen im Netzwerk
zugänglich. Der OPC-Standard definiert entsprechende DCOM-Objekte um mit anderen
OPC-Anwendungen Daten austauschen zu können. Die Verwendung von DCOM bindet Anwender
an Betriebssysteme von Microsoft. Die ursprüngliche OPC Spezifikation wird durch die
Entwicklung von OPC Unified Architecture (OPC UA) abgelöst.
OPC UA setzt auf einem eigenen Kommunikationionsstack auf, die Verwendung von DCOM
und damit die Bindung an Microsoft wurden aufgelöst.

Die OPC-UA-Architektur ist eine Service-orientierte Architektur (SOA), deren Struktur
aus mehreren Schichten besteht.

% Wikipedia
Das OPC-Informationsmodell ist nicht mehr nur eine Hierarchie aus Ordnern, Items
und Properties. Es ist ein sogenanntes Full-Mesh-Network aus Nodes, mit dem neben
den Nutzdaten eines Nodes auch Meta- und Diagnoseinformationen repräsentiert werden.
Ein Node ähnelt einem Objekt aus der objektorientierten Programmierung. Ein Node
kann Attribute besitzen, die gelesen werden können (Data Access (DA), Historical
Data Access (HDA)). Es ist möglich Methoden zu definieren und aufzurufen.
Eine Methode besitzt Aufrufargumente und Rückgabewerte. Sie wird durch ein Command
aufgerufen. Weiterhin werden Events unterstützt, die versendet werden können
(AE (Alarms \& Events), DA DataChange), um bestimmte Informationen zwischen Geräten
auszutauschen. Ein Event besitzt unter anderem einen Empfangszeitpunkt, eine Nachricht
und einen Schweregrad. Die o. g. Nodes werden sowohl für die Nutzdaten als auch
alle anderen Arten von Metadaten verwendet. Der damit modellierte OPC-Adressraum
beinhaltet nun auch ein Typmodell, mit dem sämtliche Datentypen spezifiziert werden.

% https://de.wikipedia.org/wiki/Open_Platform_Communications
% https://de.wikipedia.org/wiki/OPC_Unified_Architecture
% https://opcfoundation.org/developer-tools/specifications-unified-architecture
% Von Gerhard Gappmeier - ascolab GmbH, CC BY-SA 3.0, https://de.wikipedia.org/w/index.php?curid=1892069
\subsubsection{open62541%
        \label{sec:2-open62541}}
open62541 ist eine offene und freie Implementierung von OPC UA. Die in C geschriebene
Bibliothek stellt eine beständig zunehmende Anzahl der im OPC UA Standard definierten
Funktionen bereit. Sie kann sowohl zur Erstellung von OPC-Servern als auch -Clients
genutzt werden. Ergänzend zu der unter der Mozilla Public License v2.0 lizensierten
Bibliothek stellt das open62541 Projekt auch Beispielprogramme unter einer CC0 Lizenz
zur Verfügung.

Die Bibliothek eignet sich auch für die Entwicklung auf eingebetteten Systemen und
Microcontrollern. Je nach Umfang der gewünschten Funktionen und des OPC Informationsmodells
beträgt die Größe einer Server-Binary weniger als 100kb. %evtl. kürzen?

\todo{Nodes erklären! Evtl.~oben!}

Folgende Auswahl an Eigenschaften und Funktionen zeichnet die in dieser Arbeit verwendete
Version 0.3 von open62541 aus:
\begin{itemize}
  \item Kommunikationionsstack
  \begin{itemize}
      \item OPC UA Binär-Protokoll (HTTP oder SOAP werden gegenwärtig nicht unterstützt)
      \item Austauschbare Netzwerk-Schicht, welche die Verwendung eigener Netzwerk-APIs
      erlaubt.
      \item Verschlüsselte Kommunikationion
      \item Asynchrone Dienst-Anfragen im Client
  \end{itemize}
  \item Informationsmodell
  \begin{itemize}
    \item Unterstützung aller OPC UA Node-Typen, inkl.~Methoden
    \item Hinzufügen und Entfernen von Nodes und Referenzen zur Laufzeit.
    \item Vererbung und Instanziierung von Objekt- und Variablentypen
    \item Zugriffskontrolle auch für einzelne Nodes
  \end{itemize}
  \item Subscriptions
  \begin{itemize}
    \item Erlaubt die Überwachung (subscriptions / monitoreditems)
    \item Sehr geringer Ressourcenbedarf pro überwachtem Wert
  \end{itemize}
  \item Code-Generierung auf XML-Basis
  \begin{itemize}
    \item Erlaubt die Erstellung von Datentypen
    \item Erlaubt die Generierung des serverseitigen Informationsmodells
  \end{itemize}
\end{itemize}

% https://open62541.org/doc/0.3/


Mozilla Public License
CC0 Lizenz für Beispiele und Plugins

% https://open62541.org/doc/open62541-current.pdf
% https://open62541.org/

% % % Imports nur für Referenzenauflösung während des Schreibens! Vorm Kompilieren auskommentieren!
% \bibliography{0_hauptdatei}
% \input{1_einleitung}
% \input{2_grundlagen}
% \input{3_konzeption}
% \input{4_implementierung}
% \input{5_tests}
% \input{6_zusammenfassung}
% \input{anhang}
% % Ende Imports

\section{Systemkonzept%
  \label{sec:3-konzeption}}
Auf Basis der in Abschnitt \ref{sec:2-grundlagen} vorgestellten Möglichkeiten folgt nun die Ausarbeitung eines Konzepts.
In den folgenden Abschnitten soll näher auf zwei zentrale Aspekte eingegangen werden: Abschnitt~\ref{sec:3-anbindung} stellt Möglichkeiten zum Zugriff auf Variablen bzw.\,Werte im Prozessabbild des Revolution Pi vor; in Abschnitt~\ref{sec:3-integration} wird ein Konzept zur Bereitstellung dieser Variablen auf einem OPC-Server vorgestellt.

\subsection{Anbindung der IO an den OPC-Server%
     \label{sec:3-anbindung}}

Eine Webanwendung mit Bezeichnung PiCtory dient zur Konfiguration der I/O- und virtuellen Module des RevolutionPi. Die Konfiguration liegt im JSON-Format in der Datei \lstinline{/etc/revpi/config.rsc}. Der piControl-Treiber liest diese Datei beim Start. 
Der folgende Auszug aus der Manpage des piControl-Kernelmoduls beschreibt die von diesem zum Lesen und Schreiben einzelner Bits des Prozessabbildes bereitgestellten Funktionen~\citep[vgl.]{web-revpi-manpage}. Sie ist an dieser Stelle weitgehend ungekürzt zitiert, da sie die nutzbare Schnittstelle sehr kompakt beschreibt.

\begin{lstlisting}[breakindent=0pt, numbers=none, caption={Auszug aus der Revolution Pi Programmers Manual\label{lst:4-manpage}}]
KB_FIND_VARIABLE SPIVariable *argp
Find a variable in the process image by its name. A pointer to a structure of type SPIVariable must be passed as argument. [...]
The struct SPIVariable [...] is defined as 
typedef struct SPIVariableStr
{
    char strVarName[32]; // Variable name
    uint16_t i16uAddress; // Address of the byte in the process image
    uint8_t i8uBit; // 0-7 bit position, >= 8 whole byte
    uint16_t i16uLength; // length of the variable in bits.
    // Possible values are 1, 8, 16 and 32
} SPIVariable;

Set and get values of the process image
KB_GET_VALUE SPIValue *argp
[...]
KB_SET_VALUE SPIValue *argp
Write one bit or one byte to the process image [...].  This call is more efficient than the usual calls of seek and write because only one function call is necessary. If more than on application are writing bits in one output byte, this call is the only safe way to set a bit without overwriting the other bits because this call is doing a read-modify-write-cycle. 

The struct SPIValue used by this ioctl is defined as
typedef struct SPIValueStr
{
    uint16_t i16uAddress; // Address of the byte in the process image
    uint8_t i8uBit; // 0-7 bit position, >= 8 whole byte
    uint8_t i8uValue; // Value: 0/1 for bit access, whole byte otherwise
} SPIValue;
\end{lstlisting} 

Die oben beschriebenden Funtkionen \lstinline{KB_FIND_VARIABLE}, \lstinline{KB_GET_VALUE} und \lstinline{KB_SET_VALUE} ermöglichen einen einfachen und (lt.\,Manpage) effizienten Zugriff auf einzelne Bits des Prozessabbildes und damit auch auf die IO des RevolutionPi.
Der Zugriff des OPC-Servers auf das Prozessabbild soll daher mittels dieser Funktionen realisiert werden.
\lstinline{KB_FIND_VARIABLE} kann genutzt werden, um Adressen von Variablen im Prozessabbild mittels ihres Namens aufzulösen.
\lstinline{KB_GET_VALUE} und \lstinline{KB_SET_VALUE} ermöglichen den Zugriff auf die Werte dieser Variablen.


\subsection{Integration des OPC-Servers in das System%
     \label{sec:3-integration}}

open62541 bietet drei Möglichkeiten zum Abgleich von Variablen mit dem Prozessabbild~\citep[vgl.][Tutorials - Connecting a Variable with a Physical Process]{web-open62541}:
\begin{itemize}
    \item Manuelles oder zyklisches Aktualisieren
    \item Variable Value Callback
    \item Variable Datasource
\end{itemize}

Die zyklische Aktualisierung eines oder mehrerer Werte nimmt, abhängig von der Zykluszeit, viele Systemressourcen in Anspruch. Value Callbacks ermöglichen es, einen Variablenwert effizienter mit einer Ressource wie etwa einem Prozessabbild zu synchronisieren. An die Variable wird ein Callback angehängt, welches vor jedem Lesen und nach jedem Schreibvorgang ausgeführt wird.
Der Wert der Variablen wird weiterhin im Variablenknoten auf dem OPC-Server gespeichert, der Abgleich mit der verknüpften Ressource erfolgt durch die Callback-Methoden.

Sogenannte Datenquellen gehen noch einen Schritt weiter. Der Server leitet jede Lese- und Schreibanforderung direkt an eine Callback-Funktion weiter. Beim Lesen liefert der Rückruf eine Kopie des aktuellen Wertes. Die Datenquelle muss intern ein eigenes Speichermanagement implementieren.

Der Zugriff auf die Werte des Prozessabbildes erfolgt, wie in Abschnitt~\ref{sec:3-anbindung} beschrieben, über von piControl bereitgestellte Methoden. Um die durch open62541 gepflegte OPC-Datenstruktur und das durch piControl verwaltete Prozessabbild möglichst effektiv verknüpfen zu können, soll diese Interaktion mittels Datenquellen und den zugehörigen Callbacks implementiert werden.
% % % Imports nur für Referenzenauflösung während des Schreibens! Vorm Kompilieren auskommentieren!
% \bibliography{0_hauptdatei}
% \input{1_einleitung}
% \input{2_grundlagen}
% \input{3_konzeption}
% \input{4_implementierung}
% \input{5_tests}
% \input{6_zusammenfassung}
% \input{anhang}
% % Ende Imports

\section{Implementierung%
  \label{sec:4-implementierung}}
Das folgende Kapitel stellt in Auszügen die Implementierung des OPC-Servers sowie die Anbindung an die IO-Module
der SPS dar. Der Schwerpunkt liegt hierbei auf der Funktionsweise des piControl-Treibers und dessen Integration in das Projekt. Abschnitt~\ref{sec:4-picontrol} erklärt die zum Schreibens eines Bits verwendeten Funktionsaufrufe.
Zuvor soll jedoch in Abschnitt~\ref{sec:4-open62541} der Teil des OPC-Servers vorgestellt werden, welcher auf besagten Treiber zugreift. 

\subsection{Implementierung des OPC-Servers%
     \label{sec:4-open62541}}
Wie im vorangegangenen Abschnitt~\ref{sec:3-integration} begründet, soll die Verknüpfung zwischen dem Prozessabbild der SPS und den auf dem OPC-Server bereitgestellten Werten über sog.\,Datenquellen erfolgen. Hierzu ist zunächst eine Callback-Methode zu implementieren, welche bei einem Lese- oder Schreibzugriff auf eine Variable aufgerufen wird. Die Verknüpfung zwischen Callback-Methode und Variable muss manuell erfolgen.

\begin{lstlisting}[language={c},firstnumber=237,caption={Auszug der Methode \lstinline{linkDataSourceVariable} in \lstinline{variables.c}\label{lst:4-linkDataSourceVariable}}]
extern UA_StatusCode
 linkDataSourceVariable(UA_Server *server, UA_NodeId nodeId) {
     bool readonly = false;
     UA_DataSource dataSourceVariable;
     UA_StatusCode rc; |>\setcounter{lstnumber}{254}<|

     dataSourceVariable.read = readDataSourceVariable;
     if (!readonly)
        dataSourceVariable.write = writeDataSourceVariable;
     else
        dataSourceVariable.write = writeReadonlyDataSourceVariable;

     return UA_Server_setVariableNode_dataSource(server, nodeId, dataSourceVariable);
 }
\end{lstlisting}

\begin{figure}[h]
    \centering
    \includegraphics[width=0.42\textwidth]{doc/img/OPC_RevPiDO.pdf}
    \caption{Auszug des verwendeten Nodesets, hier Digitalausgang 1 des Versuchsaufbaus
      \label{fig:opc-do}}
\end{figure}

Die in Listing~\ref{lst:4-linkDataSourceVariable} abgebildete Methode \lstinline{linkDataSourceVariable()} erzeugt ein Struct vom Typ \lstinline{UA_DataSource}. In diesem werden dem Lesen und Schreiben einer OPC-Variablen entsprechende Callback-Methoden zugewiesen. Die Verknüpfung einer OPC-Variable, genauer ihrer NodeId, mit der zuvor definierten Datenquelle erfolgt über die von open62541 bereitgestellte Methode \lstinline{UA_Server_setVariableNode_dataSource()}. Vor dem Lesen und nach dem Schreiben dieser Variable werden von nun an die entsprechenden Callbacks aufgerufen.
     
\begin{lstlisting}[language={c},firstnumber=168,caption={Auszug des Callbacks \lstinline{writeDataSourceVariable} in \lstinline{variables.c}\label{lst:4-writeDataSourceVariable}}]  
extern UA_StatusCode
 writeDataSourceVariable(UA_Server *server,
            const UA_NodeId *sessionId, void *sessionContext,
            const UA_NodeId *nodeId, void *nodeContext,
            const UA_NumericRange *range, const UA_DataValue *dataValue) {

    UA_StatusCode retval  = UA_STATUSCODE_GOOD;
    UA_NodeId *nameNodeId = UA_malloc(sizeof(UA_NodeId));
    UA_QualifiedName nameQN = UA_QUALIFIEDNAME(1, "Name");
    UA_Variant nameVar;
    UA_Boolean bit;

    retval |= findSiblingByBrowsename(server, nodeId, &nameQN, nameNodeId);
    retval |= UA_Server_readValue(server, *nameNodeId, &nameVar);
    retval |= UA_Boolean_copy(dataValue->value.data, &bit);

    |>\tikzmarkin[set border color=martinired]{writeIO}<|PI_writeSingleIO(String_fromUA_String(nameVar.data), &bit, false);                                                 |>\tikzmarkend{writeIO}<|

    free(nameNodeId);
    return retval;
 }
\end{lstlisting}

Listing~\ref{lst:4-writeDataSourceVariable} zeigt die Callback-Methode, welche nach dem Schreiben einer Variablen auf dem OPC-Server aufgerufen wird.
Dieser Methode wird neben der NodeId der mit ihr verknüpften Variablen auch der Wert dieser in Form eines Zeigers auf ein Struct vom Typ \lstinline{UA_DataValue} übergeben.

Die Gestaltung des hier verwendeten Nodesets sieht vor, dass in einer OPC-Variablen \lstinline{"Name"} der Bezeichner des zu schreibenden Digitalausgangs hinterlegt ist, siehe Abbildung~\ref{fig:opc-do}. Dies erlaubt eine Rekonfiguration der Ein- und Ausgänge der SPS ohne Änderungen im Programmcode des OPC-Servers vornehmen zu müssen.
Es ist daher erforderlich, nach jedem Schreiben einer mit einem Digitalausgang verknüpften Variablen, hier \lstinline{"Value"}, dessen Bezeichner \lstinline{"Name"} abzufragen. 
Dies geschieht in den Zeilen 180 und 181.
Anschließend wird dieser Bezeichner sowie der zu schreibende Wert der Methode \lstinline{PI_writeSingleIO()} übergeben, welche wiederum die Interaktion mit piControl übernimmt (vgl. Abschnitt \ref{sec:4-picontrol}).
 
\subsection{Integration von piControl%
     \label{sec:4-picontrol}}
In Abschnitt~\ref{sec:2-io} wurde die Anbindung der IO-Module des Revolution Pi sowie die Funktionsweise von piControl aus Anwendersicht beschrieben. Die verfügbare Literatur beschränkt sich auch auf lediglich diese Sicht; eine weiterführende Dokumentation für Entwickler gibt es, neben der in Abschnitt~\ref{sec:3-anbindung} vorgestellten Manpage, nicht. 
In diesem Abschnitt soll daher der Quellcode von piControl sowie dessen Verwendung im Projekt genauer betrachtet werden.
Hierzu wird exemplarisch die in Abschnitt~\ref{sec:4-open62541} eingeführte Methode \lstinline{PI_writeSingleIO()} untersucht.
Diese Methode ermöglicht das Setzen eines einzelnen Bits im Prozessabbild der SPS, und damit das Schalten eines digitalen Ausgangs auf einem IO-Modul.
Die äquivalente Methode \lstinline{int piControlGetBitValue(SPIValue *pSpiValue)} zum Lesen eines Bits bzw. Eingangs funktioniert analog und soll daher an dieser Stelle nicht dediziert erörtert werden.

\begin{lstlisting}[language={c},firstnumber=97,
                   caption={Setzen eines phsikalischen, digitalen Ausgangs in \lstinline{revpi.c}
                   \label{lst:4-PI_writeSingleIO}}]
extern void PI_writeSingleIO(char *pszVariableName, bool *bit, bool verbose)
{
	int rc;
	SPIVariable sPiVariable;
	SPIValue sPIValue;

	strncpy(sPiVariable.strVarName, pszVariableName, sizeof(sPiVariable.strVarName));
	rc = piControlGetVariableInfo(&sPiVariable);
	if (rc < 0) {
		printf("Cannot find variable '%s'\n", pszVariableName);
		return;
	}

		sPIValue.i16uAddress = sPiVariable.i16uAddress;
		sPIValue.i8uBit = sPiVariable.i8uBit;
		sPIValue.i8uValue = *bit;
		rc = |>\tikzmarkin[set border color=martinired]{setBitValue}<|piControlSetBitValue(&sPIValue)|>\tikzmarkend{setBitValue}<|;
		if (rc < 0)
			printf("Set bit error %s\n", getWriteError(rc));
		else if (verbose)
			printf("Set bit %d on byte at offset %d. Value %d\n", sPIValue.i8uBit, sPIValue.i16uAddress,
			       sPIValue.i8uValue);
}
\end{lstlisting}

Der Programmcode in Listing~\ref{lst:4-PI_writeSingleIO} ist Teil des implementierten OPC-Servers. In diesem wird auf zwei Funktionen des piControl-Treibers zugegriffen. 
Beiden Methoden wird als Argument ein Zeiger auf ein Struct vom Typ \lstinline{SPIValue} übergeben. Der im Struct abgelegte Name wird mittels \lstinline{piControlGetVariableInfo(&sPIValue)} zu einer Adresse im Prozessabbild aufgelöst. Diese wird in \lstinline{sPIValue.i16uAdress} gespeichert. Der Wert der Variablen wird anschließend mittels \lstinline{piControlSetBitValue(&sPIValue)} an dieser Adresse in das Prozessabbild geschrieben.

\begin{lstlisting}[language={c},firstnumber=309,caption={Methode \lstinline{piControlSetBitValue} in \lstinline{piControlIf.c}\label{lst:4-piControlSetBitValue}}]
int |>\tikzmarkin[set border color=martiniblue]{setBitValueFcn}<|piControlSetBitValue(SPIValue *pSpiValue)|>\tikzmarkend{setBitValueFcn}<|
{
    piControlOpen();

    if (PiControlHandle_g < 0)
	    return -ENODEV;

    pSpiValue->i16uAddress += pSpiValue->i8uBit / 8;
    pSpiValue->i8uBit %= 8;

    if (|>\tikzmarkin[set border color=martinired]{ioctl}<|ioctl(PiControlHandle_g, KB_SET_VALUE, pSpiValue)|>\tikzmarkend{ioctl}<| < 0)
	    return errno;

    return 0;
}
\end{lstlisting}

Die in Listing~\ref{lst:4-piControlSetBitValue} dargestellte Methode \lstinline{piControlSetBitValue} ist lediglich eine Hüllfunktion (häufig auch als Wrapper-Funktion bezeichnet) für einen Aufruf des \lstinline{ioctl} Kernel-Moduls.
Folgende Parameter werden übergeben:
\lstinline{PiControlHandle_g} ist die Referenz auf die Geräte-Datei des piControl-Treibers. \lstinline{KB_SET_VALUE} ist das ioctl-Kommando zum Schreiben eines Bits in das Prozessabbild. Der Zeiger \lstinline{pSpiValue} verweist auf ein Struct des bereits vorgestellten Typs \lstinline{SPIValue}.

\begin{lstlisting}[language={c},firstnumber=80,caption={Methode \lstinline{piControlOpen} in \lstinline{piControlIf.c}\label{lst:4-piControlOpen}}]
void piControlOpen(void)
{
    /* open handle if needed */
    if (PiControlHandle_g < 0)
    {
	    |>\tikzmarkin[set border color=martiniblue]{PiControlHandle}<|PiControlHandle_g = open(PICONTROL_DEVICE, O_RDWR)|>\tikzmarkend{PiControlHandle}<|;
    }
}
\end{lstlisting}

Die in Listing~\ref{lst:4-piControlOpen} dargestellte Methode öffnet, sofern nicht bereits geschehen, die Geräte-Datei. Das Macro \lstinline{PICONTROL_DEVICE} verweist hierbei auf \lstinline{/dev/piControl0}.

\begin{lstlisting}[language={c},firstnumber=721,caption={Methode \lstinline{piControlIoctl} in \lstinline{piControlMain.c}\label{lst:4-piControlIoctl}}]
static long |>\tikzmarkin[set border color=martiniblue, below offset=0.9em]{piControlIoctl}<|piControlIoctl(struct file *file, unsigned int prg_nr, 
                           unsigned long usr_addr)                                      |>\tikzmarkend{piControlIoctl}<|
{
  int status = -EFAULT;
  tpiControlInst *priv;
  int timeout = 10000;	// ms

  if (prg_nr != KB_CONFIG_SEND && prg_nr != KB_CONFIG_START && !isRunning()) {
  	return -EAGAIN;
  }

  priv = (tpiControlInst *) file->private_data;

  if (prg_nr != KB_GET_LAST_MESSAGE) {
  	// clear old message
  	priv->pcErrorMessage[0] = 0;
  }

  switch (prg_nr) {|>\setcounter{lstnumber}{864}<|

    case |>\tikzmarkin[set border color=martiniblue]{KB_SET_VALUE}<|KB_SET_VALUE:|>\tikzmarkend{KB_SET_VALUE}<|
  		{
  			SPIValue *pValue = (SPIValue *) usr_addr;

  			if (!isRunning())
  				return -EFAULT;

  			if (pValue->i16uAddress >= KB_PI_LEN) {
  				status = -EFAULT;
  			} else {
  				INT8U i8uValue_l;
  				my_rt_mutex_lock(&piDev_g.lockPI);
  				i8uValue_l = piDev_g.ai8uPI[pValue->i16uAddress];

  				if (pValue->i8uBit >= 8) {
  					i8uValue_l = pValue->i8uValue;
  				} else {
  					if (pValue->i8uValue)
  						i8uValue_l |= (1 << pValue->i8uBit);
  					else
  						i8uValue_l &= ~(1 << pValue->i8uBit);
  				}

  				|>\tikzmarkin[set border color=martinired]{i8uValue}<|piDev_g.ai8uPI[pValue->i16uAddress] = i8uValue_l;|>\tikzmarkend{i8uValue}<|
  				rt_mutex_unlock(&piDev_g.lockPI);

  #ifdef VERBOSE
  				pr_info("piControlIoctl Addr=%u, bit=%u: %02x %02x\n", pValue->i16uAddress, pValue->i8uBit, pValue->i8uValue, i8uValue_l);
  #endif

  				status = 0;
  			}
  		}
  		break; |>\setcounter{lstnumber}{1314}<|

    default:
      pr_err("Invalid Ioctl");
      return (-EINVAL);
      break;

    }

    return status;
  }
\end{lstlisting}

Listing~\ref{lst:4-piControlIoctl} zeigt in Auszügen die ioctl-Methode des piControl Kernel-Treibers. Diese bekommt folgende Argumente übergeben: \lstinline{struct file *file} enthält den Verweis auf die Geräte-Datei, hier \lstinline{/dev/piControl0}. Der Wert von \lstinline{unsigned int prg_nr} beschreibt die Anfrage an den Treiber, in diesem Fall \lstinline{KB_SET_VALUE}. Das Argument \lstinline{unsigned long usr_addr} enthält einen typ-agnostischen Pointer. Dieser verweist auf einen Speicherbereich, in welchem die zur Bearbeitung der Anfrage notwendigen Daten abgelegt sind. Hier können auch vom Treiber empfangene Daten dem Anwendungsprogramm bereitgestellt werden. 

Die switch-case-Anweisung führt die über das Argument \lstinline{prg_nr} spezifizierte Aktion aus. Hier betrachten wir \lstinline{KB_SET_VALUE}:
Zunächst wird in Zeile 868 der übergebene Zeiger \lstinline{usr_addr} mittels explizitem Typecast zu einem Zeiger des Typs \lstinline{SPIValue *} konvertiert. Da dieser auf Daten im Userspace verweist, ist beim Zugriff durch den Kernel-Treiber besondere Vorsicht geboten.
In Zeile 877 wird mittels Mutex das Prozessabbild \lstinline{piDev_g} für den Zugriff durch andere Threads oder Prozesse gesperrt.
\lstinline{my_rt_mutex_lock} verweist hierbei auf die Funktion \lstinline{rt_mutex_lock} aus \lstinline{linux/sched.h}\footnote{Offenbar wurde hier auch eine alternative Implementierung vorgesehen, siehe revpi\_common.h}

In Zeile 889 wird das Byte \lstinline{i8uValue_l}, welches den zu schreibenden Wert enthält in das Prozessabbild übertragen. Anschließend wird die Mutex auf \lstinline{piDev_g} wieder entsperrt.
\newpage

\begin{lstlisting}[language={c},firstnumber=62,caption={Auszug des Struct \lstinline{spiControlDev} in \lstinline{piControlMain.h}\label{lst:4-spiControlDev}}]
|>\tikzmarkin[set border color=martiniblue]{spiControlDev}<|typedef struct spiControlDev|>\tikzmarkend{spiControlDev}<| {
	// device driver stuff
	int init_step;
	enum revpi_machine machine_type;
	void *machine;
	struct cdev cdev;	// Char device structure
	struct device *dev;
	struct thermal_zone_device *thermal_zone;

	|>\tikzmarkin[set border color=martiniblue]{processImage}<|// process image stuff
	INT8U ai8uPI[KB_PI_LEN];
	INT8U ai8uPIDefault|>\tikzmarkin[set border color=martinired]{KB_PI_LEN_0}<|[KB_PI_LEN]|>\tikzmarkend{KB_PI_LEN_0}<|;
	struct rt_mutex lockPI;        |>\tikzmarkend{processImage}<|
	bool stopIO;
	piDevices *devs; |>\setcounter{lstnumber}{94}<|
} tpiControlDev;
\end{lstlisting}

Das Prozessabbild ist als Byte-Array der Länge \lstinline{KB_PI_LEN} in Listing~\ref{lst:4-spiControlDev} definiert. Konfigurationsparameter wie \lstinline{KB_PI_LEN} oder die Zykluszeit für den Datenaustausch zwischen SPS und IO-Modulen sind im folgenden Listing~\ref{lst:4-process} definiert.

\begin{lstlisting}[language={c},firstnumber=119,caption={Konfigurationsparameter des Prozessabbildes in project.h\label{lst:4-process}}]
#define INTERVAL_PI_GATE (5*1000*1000)  // 5 ms piGateCommunication |>\setcounter{lstnumber}{128}<|

#define INTERVAL_IO_COM (5*1000*1000)  // 5 ms piIoComm |>\setcounter{lstnumber}{132}<|

#define KB_PD_LEN       512
|>\tikzmarkin[set border color=martiniblue]{KB_PI_LEN_1}<|#define KB_PI_LEN       4096|>\tikzmarkend{KB_PI_LEN_1}<|
\end{lstlisting}

Das zu setzende Bit wurde zu diesem Zeitpunkt erfolgreich in das Prozessabbild der SPS geschrieben.
Es stellt sich die Frage, wie dieses nun an das IO-Modul kommuniziert wird.
Die Kommunikation mit allen angebundenen Modulen ist ebenfalls Aufgabe des piControl-Treibers.

\begin{lstlisting}[language={c},firstnumber=256,caption={Auszug der Methode \lstinline{piIoThread} in \lstinline{revpi_core.c}\label{lst:4-piIoThread}}]
static int piIoThread(void *data)
{
	//TODO int value = 0;
	ktime_t time;
	ktime_t now;
	s64 tDiff;

	hrtimer_init(&piCore_g.ioTimer, CLOCK_MONOTONIC, HRTIMER_MODE_ABS);
	piCore_g.ioTimer.function = piIoTimer;

	pr_info("piIO thread started\n");

	now = hrtimer_cb_get_time(&piCore_g.ioTimer);

	PiBridgeMaster_Reset();

	while (!kthread_should_stop()) {
		if (|>\tikzmarkin[set border color=martinired]{PiBridgeMaster}<|PiBridgeMaster_Run()|>\tikzmarkend{PiBridgeMaster}<| < 0)
			break;
	}

	RevPiDevice_finish();

	pr_info("piIO exit\n");
	return 0;
}
\end{lstlisting}

Der Kernel-Thread \lstinline{piIoThread} ist verantwortlich für den zyklischen Datenaustausch mit den IO-Modulen. In diesem wird fortlaufend die Methode \lstinline{PiBridgeMaster_Run()} aufgerufen, siehe Listing~\ref{lst:4-piIoThread}.

\begin{lstlisting}[language={c},firstnumber=262,caption={Auszug der Methode \lstinline{PiBridgeMaster_Run(void)} in \lstinline{RevPiDevice.c}\label{lst:4-PiBridgeMaster_Run}}]
int PiBridgeMaster_Run(void)
{
	static kbUT_Timer tTimeoutTimer_s;
	static kbUT_Timer tConfigTimeoutTimer_s;
	static int error_cnt;
	static INT8U last_led;
	static unsigned long last_update;
	int ret = 0;
	int i;

	my_rt_mutex_lock(&piCore_g.lockBridgeState);
	if (piCore_g.eBridgeState != piBridgeStop) {
		switch (eRunStatus_s) { |>\setcounter{lstnumber}{514}<|
		    case enPiBridgeMasterStatus_EndOfConfig:|>\setcounter{lstnumber}{621}<|
		    if (|>\tikzmarkin[set border color=martinired]{RevPiDevice}<|RevPiDevice_run()|>\tikzmarkend{RevPiDevice}<|) {
				// an error occured, check error limits |>\setcounter{lstnumber}{641}<|
			} else {
				ret = 1;
			}
			piCore_g.image.drv.i16uRS485ErrorCnt = RevPiDevice_getErrCnt();
			break;
\end{lstlisting}

Die in Listing~\ref{lst:4-PiBridgeMaster_Run} dargestellte Methode ist eine sog. State-Machine. Ist die Konfiguration der IO-Module erfolgreich abgeschlossen, so führt sie bei Aufruf lediglich die Methode \lstinline{RevPiDevice_run()} aus.

\begin{lstlisting}[language={c},firstnumber=140,caption={Auszug der Methode \lstinline{RevPiDevice_run(void)} in \lstinline{RevPiDevice.c}\label{lst:4-RevPiDevice_run}}]
int RevPiDevice_run(void)
{
	INT8U i8uDevice = 0;
	INT32U r;
	int retval = 0;

	RevPiDevices_s.i16uErrorCnt = 0;

	for (i8uDevice = 0; i8uDevice < RevPiDevice_getDevCnt(); i8uDevice++) {
		if (RevPiDevice_getDev(i8uDevice)->i8uActive) {
			switch (RevPiDevice_getDev(i8uDevice)->sId.i16uModulType) {
			case KUNBUS_FW_DESCR_TYP_PI_DIO_14:
			case KUNBUS_FW_DESCR_TYP_PI_DI_16:
			case KUNBUS_FW_DESCR_TYP_PI_DO_16:
				r = |>\tikzmarkin[set border color=martinired]{sendCyclicTelegram}<|piDIOComm_sendCyclicTelegram(i8uDevice)|>\tikzmarkend{sendCyclicTelegram}\setcounter{lstnumber}{166} <|;

				break; |>\setcounter{lstnumber}{216}<|
			}
		}
	} |>\setcounter{lstnumber}{227}<|
	return retval;
}
\end{lstlisting}

Diese iteriert wie in Listing~\ref{lst:4-RevPiDevice_run} abgebildete durch alle gegenwärtig in der SPS konfigurierten Module. Ist das aktuelle Modul als aktiv markiert, so wird anhand eines sog. Firmware-Descriptors entschieden, welche Methode für die Ansteuerung des Moduls aufzurufen ist.

\begin{lstlisting}[language={c},firstnumber=161,caption={Auszug der Methode \lstinline{piDIOComm_sendCyclicTelegram} in \lstinline{piDIOComm.c}\label{lst:4-sendCyclicTelegram}}]
INT32U piDIOComm_sendCyclicTelegram(INT8U i8uDevice_p)
{
	INT32U i32uRv_l = 0;
	SIOGeneric sRequest_l;
	SIOGeneric sResponse_l;
	INT8U len_l, data_out[18], i, p, data_in[70];
	INT8U i8uAddress;
	int ret; |>\setcounter{lstnumber}{239}<|
	
    |>\tikzmarkin[set border color=martinired]{piIoComm}<|ret = piIoComm_send((INT8U *) & sRequest_l, IOPROTOCOL_HEADER_LENGTH + len_l + 1);  |>\tikzmarkend{piIoComm}\setcounter{lstnumber}{298}<|
}
\end{lstlisting}

Im Falle des hier verwendeten DO-Moduls wird die in Listing~\ref{lst:4-sendCyclicTelegram} abgebildete Methode \lstinline{piDIOComm_sendCyclicTelegram()} aufgerufen. Dieser wird ein Zeiger auf das zu schreibende Gerät übergeben. 
Zunächst wird das Prozessabbild mittels eines proprietären, jedoch im Quellcode offen nachvollziehbaren Protokolls in ein \lstinline{sRequest_l} genanntes Byte-Array umgewandelt. Dieser Schritt ist in Listing~\ref{lst:4-sendCyclicTelegram} nicht abgebildet. Anschließend wird \lstinline{piIoComm_send()} ein Zeiger auf die so generierte Schreib-Anfrage übergeben.

\begin{lstlisting}[language={c},firstnumber=220,caption={Auszug der Methode \lstinline{piIOComm_send} in \lstinline{piIOComm.c}\label{lst:4-piIOComm_send}}]
int piIoComm_send(INT8U * buf_p, INT16U i16uLen_p)
{
	ssize_t write_l = 0;
	INT16U i16uSent_l = 0;|>\setcounter{lstnumber}{249}<|

	while (i16uSent_l < i16uLen_p) {
		write_l = vfs_write(piIoComm_fd_m, buf_p + i16uSent_l, i16uLen_p - i16uSent_l, &piIoComm_fd_m->f_pos);
		if (write_l < 0) {
			pr_info_serial("write error %d\n", (int)write_l);
			return -1;
		} 
		i16uSent_l += write_l;|>\setcounter{lstnumber}{263}<|
	}
	clear();
	vfs_fsync(piIoComm_fd_m, 1);
	return 0;
}
\end{lstlisting}

Listing~\ref{lst:4-piIOComm_send} zeigt die Implementierung von \lstinline{piIoComm_send()}. Diese Methode ist für das Schreiben der oben generierten Anfrage auf die seriellen Schnittstelle verantwortlich. Realisiert wird dies mittels der Methode \lstinline{vfs_write()}. Diese ist in \lstinline{<linux/fs.h>} definiert. Sie ermöglicht das Schreiben einer Datei im Userspace aus dem Kernel heraus. Geschrieben wird hier die Datei mit dem Deskriptor \lstinline{piIoComm_fd_m}.
Da die Funktion \lstinline{vfs_write()} durch andere Kernel-Tasks unterbrochen werden kann, ist nicht gewährleistet, dass die gesamte Anfrage mit nur einem Aufruf geschrieben wird. Die oben abgebildete while-Schleife stellt das vollständige Senden der Anfrage sicher.

\begin{lstlisting}[language={c},firstnumber=157,caption={Auszug der Methode \lstinline{piIOComm_open_serial} in \lstinline{piIOComm.c}\label{lst:4-piIOComm_open_serial}}]
int piIoComm_open_serial(void)
{   |>\setcounter{lstnumber}{167}<|
	struct file *fd;	/* Filedeskriptor */
	struct termios newtio;	/* Schnittstellenoptionen */

	|>\tikzmarkin[set border color=martiniblue]{fd}<|/* Port oeffnen - read/write, kein "controlling tty", 
	    Status von DCD ignorieren */
	fd = filp_open(|>\tikzmarkin[set border color=martinired]{tty}<|REV_PI_TTY_DEVICE|>\tikzmarkend{tty}<|, O_RDWR | O_NOCTTY, 0); |>\setcounter{lstnumber}{208}<|
	
	piIoComm_fd_m = fd;                                                      |>\tikzmarkend{fd}\setcounter{lstnumber}{217}<|

	return 0;
}
\end{lstlisting}

Der zum Schreiben auf die serielle Schnittstelle verwendete Datei-Deskriptor wird von der in Listing~\ref{lst:4-piIOComm_open_serial} abgebildeten Methode \lstinline{piIoComm_open_serial()} generiert. 

\begin{lstlisting}[language={c},firstnumber=45,caption={Definition der seriellen Schnittstelle in \lstinline{piIOComm.h}\label{lst:4-REV_PI_TTY_DEVICE}}]
#define REV_PI_TTY_DEVICE	"/dev/ttyAMA0"
\end{lstlisting}

Das in Listing~\ref{lst:4-REV_PI_TTY_DEVICE} definierte Macro verweist auf eine der seriellen Schnittstellen des RaspberryPi.
Die Implementierung des zugehörigen Schnittstellentreibers soll hier nicht weiter untersucht werden. Somit ist an dieser Stelle die Kette vom Setzen einer Variablen auf dem OPC-Server bis hin zur Aktualisierung des Prozessabbilds der IO-Module geschlossen.

% \begin{lstlisting}[language={c},firstnumber={226},caption={Setzen der Scheduler-Priorität auf SCHED\_FIFO in 
% revpi\_common.c\label{lst:2-sched_priority}}]
% param.sched_priority = ktprio->prio;
% ret = sched_setscheduler(child, SCHED_FIFO, &param);
% \end{lstlisting}
% % % Imports nur für Referenzenauflösung während des Schreibens! Vorm Kompilieren auskommentieren!
% \bibliography{0_hauptdatei}
% \input{1_einleitung}
% \input{2_grundlagen}
% \input{3_konzeption}
% \input{4_implementierung}
% \input{5_tests}
% \input{6_zusammenfassung}
% % Ende Imports

\section{Test des OPC-Servers im Gesamtsystem%
  \label{sec:5-tests}}

% % % Imports nur für Referenzenauflösung während des schreibens! Vorm Kompilieren auskommentieren!
% \bibliography{0_hauptdatei}
% \input{1_einleitung}
% \input{2_grundlagen}
% \input{3_konzeption}
% \input{4_implementierung}
% \input{5_tests}
% \input{6_zusammenfassung}
% % Ende Imports

\section{Zusammenfassung und Ausblick%
  \label{sec:6-fazit}}
Der folgende Abschnitt~\ref{sec:6-zusammenfassung} fasst die gewonnenen Erkenntnisse und den Stand der Implementierung zusammen.
Den Abschluss dieser Arbeit bildet der Ausblick in Abschnitt~\ref{sec:6-ausblick}.

\subsection{Zusammenfassung%
     \label{sec:6-zusammenfassung}}

\subsection{Ausblick%
     \label{sec:6-ausblick}}

% \input{anhang}
% % Ende Imports

\section{Systemkonzept%
  \label{sec:3-konzeption}}
Auf Basis der in Abschnitt \ref{sec:2-grundlagen} vorgestellten Möglichkeiten folgt nun die Ausarbeitung eines Konzepts.
In den folgenden Abschnitten soll näher auf zwei zentrale Aspekte eingegangen werden: Abschnitt~\ref{sec:3-anbindung} stellt Möglichkeiten zum Zugriff auf Variablen bzw.\,Werte im Prozessabbild des Revolution Pi vor; in Abschnitt~\ref{sec:3-integration} wird ein Konzept zur Bereitstellung dieser Variablen auf einem OPC-Server vorgestellt.

\subsection{Anbindung der IO an den OPC-Server%
     \label{sec:3-anbindung}}

Eine Webanwendung mit Bezeichnung PiCtory dient zur Konfiguration der I/O- und virtuellen Module des RevolutionPi. Die Konfiguration liegt im JSON-Format in der Datei \lstinline{/etc/revpi/config.rsc}. Der piControl-Treiber liest diese Datei beim Start. 
Der folgende Auszug aus der Manpage des piControl-Kernelmoduls beschreibt die von diesem zum Lesen und Schreiben einzelner Bits des Prozessabbildes bereitgestellten Funktionen~\citep[vgl.]{web-revpi-manpage}. Sie ist an dieser Stelle weitgehend ungekürzt zitiert, da sie die nutzbare Schnittstelle sehr kompakt beschreibt.

\begin{lstlisting}[breakindent=0pt, numbers=none, caption={Auszug aus der Revolution Pi Programmers Manual\label{lst:4-manpage}}]
KB_FIND_VARIABLE SPIVariable *argp
Find a variable in the process image by its name. A pointer to a structure of type SPIVariable must be passed as argument. [...]
The struct SPIVariable [...] is defined as 
typedef struct SPIVariableStr
{
    char strVarName[32]; // Variable name
    uint16_t i16uAddress; // Address of the byte in the process image
    uint8_t i8uBit; // 0-7 bit position, >= 8 whole byte
    uint16_t i16uLength; // length of the variable in bits.
    // Possible values are 1, 8, 16 and 32
} SPIVariable;

Set and get values of the process image
KB_GET_VALUE SPIValue *argp
[...]
KB_SET_VALUE SPIValue *argp
Write one bit or one byte to the process image [...].  This call is more efficient than the usual calls of seek and write because only one function call is necessary. If more than on application are writing bits in one output byte, this call is the only safe way to set a bit without overwriting the other bits because this call is doing a read-modify-write-cycle. 

The struct SPIValue used by this ioctl is defined as
typedef struct SPIValueStr
{
    uint16_t i16uAddress; // Address of the byte in the process image
    uint8_t i8uBit; // 0-7 bit position, >= 8 whole byte
    uint8_t i8uValue; // Value: 0/1 for bit access, whole byte otherwise
} SPIValue;
\end{lstlisting} 

Die oben beschriebenden Funtkionen \lstinline{KB_FIND_VARIABLE}, \lstinline{KB_GET_VALUE} und \lstinline{KB_SET_VALUE} ermöglichen einen einfachen und (lt.\,Manpage) effizienten Zugriff auf einzelne Bits des Prozessabbildes und damit auch auf die IO des RevolutionPi.
Der Zugriff des OPC-Servers auf das Prozessabbild soll daher mittels dieser Funktionen realisiert werden.
\lstinline{KB_FIND_VARIABLE} kann genutzt werden, um Adressen von Variablen im Prozessabbild mittels ihres Namens aufzulösen.
\lstinline{KB_GET_VALUE} und \lstinline{KB_SET_VALUE} ermöglichen den Zugriff auf die Werte dieser Variablen.


\subsection{Integration des OPC-Servers in das System%
     \label{sec:3-integration}}

open62541 bietet drei Möglichkeiten zum Abgleich von Variablen mit dem Prozessabbild~\citep[vgl.][Tutorials - Connecting a Variable with a Physical Process]{web-open62541}:
\begin{itemize}
    \item Manuelles oder zyklisches Aktualisieren
    \item Variable Value Callback
    \item Variable Datasource
\end{itemize}

Die zyklische Aktualisierung eines oder mehrerer Werte nimmt, abhängig von der Zykluszeit, viele Systemressourcen in Anspruch. Value Callbacks ermöglichen es, einen Variablenwert effizienter mit einer Ressource wie etwa einem Prozessabbild zu synchronisieren. An die Variable wird ein Callback angehängt, welches vor jedem Lesen und nach jedem Schreibvorgang ausgeführt wird.
Der Wert der Variablen wird weiterhin im Variablenknoten auf dem OPC-Server gespeichert, der Abgleich mit der verknüpften Ressource erfolgt durch die Callback-Methoden.

Sogenannte Datenquellen gehen noch einen Schritt weiter. Der Server leitet jede Lese- und Schreibanforderung direkt an eine Callback-Funktion weiter. Beim Lesen liefert der Rückruf eine Kopie des aktuellen Wertes. Die Datenquelle muss intern ein eigenes Speichermanagement implementieren.

Der Zugriff auf die Werte des Prozessabbildes erfolgt, wie in Abschnitt~\ref{sec:3-anbindung} beschrieben, über von piControl bereitgestellte Methoden. Um die durch open62541 gepflegte OPC-Datenstruktur und das durch piControl verwaltete Prozessabbild möglichst effektiv verknüpfen zu können, soll diese Interaktion mittels Datenquellen und den zugehörigen Callbacks implementiert werden.
% % % Imports nur für Referenzenauflösung während des Schreibens! Vorm Kompilieren auskommentieren!
% \bibliography{0_hauptdatei}
% % Mit \section{...} eröffnen wir einen neuen Abschnitt.
% Der Befehl setzt nicht nur den Text in einer größeren,
% fetten Schrift, sondern sorgt außerdem dafür, daß er im
% Inhaltsverzeichnis erscheint.
%
% Mit \label{...} erzeugen wir einen Bezeichner, mit dessen Hilfe
% wir später auf die Nummer des Abschnitts verweisen können (nämlich
% mit~\ref{...}).
%
% Das Kommentarzeichen hinter „Übersicht“ dient dazu, ein
% Leerzeichen zwischen „Übersicht“ und dem \label-Befehl
% zu vermeiden, das andernfalls sichtbar würde – z.B. im
% Inhaltsverzeichnis.
%

% % Imports nur für Referenzenauflösung während des Schreibens! Vorm Kompilieren auskommentieren!
% \bibliography{0_hauptdatei}
% \input{1_einleitung}
%\input{2_grundlagen}
%\input{3_konzeption}
%\input{4_implementierung}
%\input{5_tests}
%\input{6_zusammenfassung}
% % Ende Imports

\section{Einleitung und Motivation%
  \label{sec:1-einleitung}}
Ziel dieses Projektes ist die Integration eines OPC-Servers mit einer auf Linux
basierenden speicherprogrammierbaren Steuerung (SPS). Angeschlossen an diese SPS
ist jeweils ein digitales Ein-/\,bzw.~Ausgabemodul. Die von diesen bereitgestellten
Ein-/\, bzw.~Ausgänge (IO) sollen in der Datenstruktur des OPC-Servers abgebildet
und über diesen für OPC-Clients les-/\,und schreibar sein. Weiterhin sollen einige
Funktionen zur Überwachung und Steuerung der an die SPS angeschlossenen Aktoren
und Sensoren direkt im OPC-Server implementiert werden.
Hiermit stellt dieses Projekt eine der Grundlagen für ein übergeordnetes Projekt,
die cloudbasierte Steuerung eines miniaturisierten Produktions-Systems, dar.

Der hier verwendete OPC-Server ist Teil des sog. open62541 Projekts. Er ist in C
geschrieben und implementiert bereits einen großen Teil der im OPC-UA-Standard
spezifizierten Funktionen.
Als SPS findet ein Revolution Pi 3 der Firma Kunbus Verwendung. Dieser integriert
ein sog. Compute Module der Raspberry Pi Foundation in ein industrietaugliches
Gehäuse und erlaubt die Erweiterung mittels IO- oder Gateway-Modulen. Über diese
erfolgt die Kommunikation mit weiteren Komponenten der Automatisierungstechnik.

Motiviert ist dieses Projekt durch die Beobachtung, dass die Verbreitung offener
Standards sowie freier Software auch in der Automatisierungstechnik zunimmt.
Linux ist ein freies Betriebssystem, OPC-UA ein offen zugänglicher, aktiv gepflegter
und weit verbreiteter Standard. Der Raspberry Pi findet sowohl bei Hobby-Anwendern als
auch in den Bereichen Forschung und Entwicklung sowie bei industriellen Anwendern
Verwendung. Dieses Projekt stellt somit eine für unterschiedliche Anwender interessante
Entwicklung dar.

Im Anschluss an diese einleitende Übersicht im Abschnitt~\ref{sec:1-einleitung} folgt
die Darstellung der wichtigsten Grundlagen in Abschnitt~\ref{sec:2-grundlagen}.
Aufbauend auf diesen Grundlagen folgt die konzeptuelle Ausarbeitung im Abschnitt~\ref{sec:3-konzeption}.
Die Umsetzung wird im Abschnitt~\ref{sec:4-implementierung} erläutert.
Die Leistungsfähigkeit der Implementierung wird in Abschnitt~\ref{sec:5-tests} untersucht.
Eine Zusammenfassung und ein Ausblick schließen die Arbeit in
Abschnitt~\ref{sec:6-fazit} ab. Eventuell noch benötigte Anhänge
finden sich in den Anhängen [...] bis [...].

% % % Imports nur für Referenzenauflösung während des Schreibens! Vorm Kompilieren auskommentieren!
% \bibliography{0_hauptdatei}
% \input{1_einleitung}
% \input{2_grundlagen}
% \input{3_konzeption}
% \input{4_implementierung}
% \input{5_tests}
% \input{6_zusammenfassung}
% % Ende Imports

\section{Grundlagen%
  \label{sec:2-grundlagen}}

\subsection{Speicherprogrammierbare-Steuerung und Linux -- Revolution Pi%
     \label{sec:2-sps}}

\subsubsection{Kunbus RevolutionPi%
        \label{sec:2-revpi}}
Der RevolutionPi 3 ist eine speicherprogrammierbare Steuerung (SPS) des Herstellers
Kunbus GmbH. Kern dieser SPS ist das von der Raspberry Pi Foundation entwickelte
und vertriebene Raspberry Pi Compute Module 3. Dieses integriert ein Broadcom BCM2837
System-on-Chip (SoC) mit vier 1,2GHz Prozessorkernen, 1GB RAM, 4GB eMMC Anwendungsspeicher
und sonstige Peripherie in ein Modul im DDR2-SODIMM Formfaktor. Diese Spezifikationen
sind weitgehend identisch zu denen des ausgesprochen populären Raspberry Pi 3.
Der Revolution Pi profitiert daher von dem gleichen großen Angebot an Software
und Unterstützung wie der Raspberry Pi, ergänzt dessen Hardware jedoch um eine 24V
Spannungsversorgung, die Möglichkeit der Erweiterung durch mehrere industrietaugliche
Ein-/ Ausgabemodule und Gateways sowie ein Gehäuse zur Montage auf einer DIN-Schiene.
\begin{itemize}
  \item{Prozessor: BCM2837}
  \item{Taktfrequenz 1,2 GHz}
  \item{Anzahl Prozessorkerne: 4}
  \item{Arbeitsspeicher: 1 GByte}
  \item{eMMC Flash Speicher: 4 GByte}
  \item{Betriebssystem: Angepasstes Raspbian mit RT-Patch}
  \item{RTC mit 24h Pufferung über wartungsfreien Kondensator}
  \item{Treiber / API: Treiber schreibt zyklisch Prozessdaten in ein Prozessabbild, Zugriff auf Prozessabbild über Linux-Filesystem als API zu Fremdsoftware.}
  \item{Kommunikationsanschlüsse: 2 x USB 2.0 A (je 500 mA belastbar), 1 x Micro-USB, HDMI, Ethernet (RJ45) 10/100 Mbit/s}
  \item{Stromversorgung: min. 10,7 V, max. 28,8 V, maximal 10 Watt}
  \item{Zulässige Umgebungstemperatur: -40 bis +55 C}
  \item{Gehäuseabmessungen: (HxBxL) 96 mm x 22,5 mm x 110,5 mm (ohne gesteckte Stecker)}
  \item{ESD Schutz: 4 kV / 8 kV gemäß EN61131-2 und IEC 61000-6-2}
  \item{Surge / Burst Prüfungen: gemäß EN61131-2 und IEC 61000-6-2 eingekoppelt auf Versorgungsspannung, Ethernet und IO-Leitungen}
  \item{EMI Prüfungen: gemäß EN61131-2 und IEC 61000-6-2}
\end{itemize}

Kunbus bietet eine Auswahl an IO- und Gateway-Modulen zur Erweiterung des Revolution Pi an.
Gateways dienen der Kommunikation mit Systemen oder Komponenten der Automatisierungstechnik
über Protokolle wie PROFIBUS oder EtherCAT. IO-Module erlauben die Überwachung
und Steuerung von digitalen oder analogen Ein- und Ausgängen.

\subsubsection{Zugriff auf IO-Module%
        \label{sec:2-io}}
Der Zugriff auf die Ein- und Ausgänge der IO-Module erfolgt über ein Prozessabbild
und einen hierfür von Kunbus bereitgestellten Treiber, genannt piControl. Dieser
aktualisiert das Prozessabbild zyklisch. Die angestrebte Zykluszeit beträgt 5ms,
kann jedoch je nach Anzahl der angeschlossenen Module auch größer sein. Kunbus
garantiert bei drei IO-Modulen und zwei Gateway-Modulen eine Zykluszeit von 10 ms.
Jedes der IO-Module stellt ein eigenständiges eingebettetes System dar. Es verfügt
über einen Microcontroller, welcher die IOs bereitstellt und über einen RS485-Bus
mit dem Revolution Pi kommuniziert.
% https://revolution.kunbus.de/io-modul/

Lizenz: GPL
% https://github.com/RevolutionPi/piControl

\begin{lstlisting}[language={c},firstnumber={226},caption={Setzen der Scheduler-Priorität auf SCHED\_FIFO in revpi\_common.c\label{lst:2-sched_priority}}]
param.sched_priority = ktprio->prio;
ret = sched_setscheduler(child, SCHED_FIFO,
       &param);
\end{lstlisting}


\subsection{Echtzeit und Multithreading unter Linux -- preemptRT und posix%
     \label{sec:2-echtzeit}}


 Der Linux-Kernel verfügt über mehrere unterschiedliche Preemtion-Modelle:

\begin{itemize}
  \item No Forced Preemption (server):
  Ausgelegt auf maximal möglichen Durchsatz, lediglich Interrupts und
  System-Call-Returns bewirken Präemption.

  \item Voluntary Kernel Preemption (Desktop):
  Neben den implizit bevorrechtigten Interrupts und System-Call-Returns gibt es
  in diesem Modell weitere Abschnitte des Kernels in welchen Preämption explizit
  gestattet ist.

  \item Preemptible Kernel (Low-Latency Desktop):
  In diesem Modell ist der gesamte Kernel, mit Ausnahme sog.~kritischer Abschnitte
  präemptible. Nach jedem kritischen Abschnitt gibt es einen impliziten Präemptions-Punkt.

  \item Preemptible Kernel (Basic RT):
  Dieses Modell ist dem zuvor genannten sehr ähnlich, hier sind jedoch alle Interrupt-Handler
  als eigenständige Threads ausgeführt.

  \item Fully Preemptible Kernel (RT):
  Wie auch bei den beiden zuvor genannten Modellen ist hier der gesamte Kernel
  präemtible, die Anzahl und Dauer der nicht-präemtiblen kritischen Abschnitte
  ist auf ein notwendiges Minimum beschränkt. Alle Interrupt-Handler sind als
  eigenständige Threads ausgeführt, Spinlocks durch Sleeping-Spinlocks und Mutexe
  durch sog.~RT-Mutexe ersetzt.

\end{itemize}
\todo{Spinlocks und Mutexe sowie die RT-Varianten dieser erklären!}

Lediglich mit dem vollständig präemtiblen Kernel kann Echtzeit-Verhalten realisiert werden.

% https://wiki.linuxfoundation.org/realtime/documentation/technical_basics/preemption_models bzw kernel/Kconfig.preempt

\subsubsection{preemptRT%
        \label{sec:2-preemptRT}}
% https://wiki.linuxfoundation.org/realtime/documentation/technical_details/start
% https://wiki.linuxfoundation.org/realtime/documentation/technical_basics/start

Das dem PREEMPT RT Kernel zugrunde liegende Prinzip lässt sich in einer einfachen
Regel ausdrücken: Nur Code, welcher absolut nicht-präemtible sein darf, ist es
gestattet nicht-präemtible zu sein.
Das erklärte Ziel des PREEMPT\_RT Patches ist es folglich, die Menge des nicht-präemtiblen
Codes im Linux-Kernel auf das absolut notwendige Minimum zu reduzieren.

Dies wird durch Verwendung folgender Mechanismen erreicht:

\begin{itemize}
  \item Hochauflösende Timer
  \item Sleeping Spinlocks
  \item Threaded Interrupt Handlers
  \item rt\_mutex
  \item RCU
\end{itemize}


\subsubsection{posix%
        \label{sec:2-posix}}
Ist posix hier wirklich relevant? Debian bzw.~Raspbian sind weitgehend posix
kompatibel, aber wird es hier genutzt? -> JA, open62541 nutzt pthread.h
piControl nutzt kthread.h, und semaphore.h

\subsection{OPC-UA und open62541%
     \label{sec:2-opc}}

\subsubsection{OPC UA%
        \label{sec:2-opcua}}
Open Platform Communications (OPC) ist eine Familie von Standards zur herstellerunabhängigen
Kommunikation von Maschinen (M2M) in der Automatisierungstechnik. Die sog.~OPC Task Force, zu deren
Mitgliedern verschiedene große Firmen der Automatisierungsindustrie gehören, veröffentlichte
die OPC Specification Version 1.0 im August 1996.
Motiviert ist dieser offene Standard durch die Erkenntniss, dass die Anpassung der
zahlreichen Herstellerstandards an individuelle Infrastrukturen und Anlagen einen
großen Mehraufwand verursachen.
Die Wikipedia beschreibt das Anwendungsgebiet für OPC wie folgt:

\glqq{}OPC wird dort eingesetzt, wo Sensoren, Regler und Steuerungen verschiedener Hersteller
ein gemeinsames Netzwerk bilden. Ohne OPC benötigten zwei Geräte zum Datenaustausch
genaue Kenntnis über die Kommunikationsmöglichkeiten des Gegenübers. Erweiterungen
und Austausch gestalten sich entsprechend schwierig. Mit OPC genügt es, für jedes
Gerät genau einmal einen OPC-konformen Treiber zu schreiben. Idealerweise wird
dieser bereits vom Hersteller zur Verfügung gestellt. Ein OPC-Treiber lässt sich
ohne großen Anpassungsaufwand in beliebig große Steuer- und Überwachungssysteme
integrieren.

OPC unterteilt sich in verschiedene Unterstandards, die für den jeweiligen Anwendungsfall
unabhängig voneinander implementiert werden können. OPC lässt sich damit verwenden
für Echtzeitdaten (Überwachung), Datenarchivierung, Alarm-Meldungen und neuerdings
auch direkt zur Steuerung (Befehlsübermittlung).\grqq{}

OPC basiert in der ursprünglichen Spezifikation auf Microsofts DCOM-Spezifikation.
DCOM macht Funktionen und Objekte einer Anwendung anderen Anwendungen im Netzwerk
zugänglich. Der OPC-Standard definiert entsprechende DCOM-Objekte um mit anderen
OPC-Anwendungen Daten austauschen zu können. Die Verwendung von DCOM bindet Anwender
an Betriebssysteme von Microsoft. Die ursprüngliche OPC Spezifikation wird durch die
Entwicklung von OPC Unified Architecture (OPC UA) abgelöst.
OPC UA setzt auf einem eigenen Kommunikationionsstack auf, die Verwendung von DCOM
und damit die Bindung an Microsoft wurden aufgelöst.

Die OPC-UA-Architektur ist eine Service-orientierte Architektur (SOA), deren Struktur
aus mehreren Schichten besteht.

% Wikipedia
Das OPC-Informationsmodell ist nicht mehr nur eine Hierarchie aus Ordnern, Items
und Properties. Es ist ein sogenanntes Full-Mesh-Network aus Nodes, mit dem neben
den Nutzdaten eines Nodes auch Meta- und Diagnoseinformationen repräsentiert werden.
Ein Node ähnelt einem Objekt aus der objektorientierten Programmierung. Ein Node
kann Attribute besitzen, die gelesen werden können (Data Access (DA), Historical
Data Access (HDA)). Es ist möglich Methoden zu definieren und aufzurufen.
Eine Methode besitzt Aufrufargumente und Rückgabewerte. Sie wird durch ein Command
aufgerufen. Weiterhin werden Events unterstützt, die versendet werden können
(AE (Alarms \& Events), DA DataChange), um bestimmte Informationen zwischen Geräten
auszutauschen. Ein Event besitzt unter anderem einen Empfangszeitpunkt, eine Nachricht
und einen Schweregrad. Die o. g. Nodes werden sowohl für die Nutzdaten als auch
alle anderen Arten von Metadaten verwendet. Der damit modellierte OPC-Adressraum
beinhaltet nun auch ein Typmodell, mit dem sämtliche Datentypen spezifiziert werden.

% https://de.wikipedia.org/wiki/Open_Platform_Communications
% https://de.wikipedia.org/wiki/OPC_Unified_Architecture
% https://opcfoundation.org/developer-tools/specifications-unified-architecture
% Von Gerhard Gappmeier - ascolab GmbH, CC BY-SA 3.0, https://de.wikipedia.org/w/index.php?curid=1892069
\subsubsection{open62541%
        \label{sec:2-open62541}}
open62541 ist eine offene und freie Implementierung von OPC UA. Die in C geschriebene
Bibliothek stellt eine beständig zunehmende Anzahl der im OPC UA Standard definierten
Funktionen bereit. Sie kann sowohl zur Erstellung von OPC-Servern als auch -Clients
genutzt werden. Ergänzend zu der unter der Mozilla Public License v2.0 lizensierten
Bibliothek stellt das open62541 Projekt auch Beispielprogramme unter einer CC0 Lizenz
zur Verfügung.

Die Bibliothek eignet sich auch für die Entwicklung auf eingebetteten Systemen und
Microcontrollern. Je nach Umfang der gewünschten Funktionen und des OPC Informationsmodells
beträgt die Größe einer Server-Binary weniger als 100kb. %evtl. kürzen?

\todo{Nodes erklären! Evtl.~oben!}

Folgende Auswahl an Eigenschaften und Funktionen zeichnet die in dieser Arbeit verwendete
Version 0.3 von open62541 aus:
\begin{itemize}
  \item Kommunikationionsstack
  \begin{itemize}
      \item OPC UA Binär-Protokoll (HTTP oder SOAP werden gegenwärtig nicht unterstützt)
      \item Austauschbare Netzwerk-Schicht, welche die Verwendung eigener Netzwerk-APIs
      erlaubt.
      \item Verschlüsselte Kommunikationion
      \item Asynchrone Dienst-Anfragen im Client
  \end{itemize}
  \item Informationsmodell
  \begin{itemize}
    \item Unterstützung aller OPC UA Node-Typen, inkl.~Methoden
    \item Hinzufügen und Entfernen von Nodes und Referenzen zur Laufzeit.
    \item Vererbung und Instanziierung von Objekt- und Variablentypen
    \item Zugriffskontrolle auch für einzelne Nodes
  \end{itemize}
  \item Subscriptions
  \begin{itemize}
    \item Erlaubt die Überwachung (subscriptions / monitoreditems)
    \item Sehr geringer Ressourcenbedarf pro überwachtem Wert
  \end{itemize}
  \item Code-Generierung auf XML-Basis
  \begin{itemize}
    \item Erlaubt die Erstellung von Datentypen
    \item Erlaubt die Generierung des serverseitigen Informationsmodells
  \end{itemize}
\end{itemize}

% https://open62541.org/doc/0.3/


Mozilla Public License
CC0 Lizenz für Beispiele und Plugins

% https://open62541.org/doc/open62541-current.pdf
% https://open62541.org/

% % % Imports nur für Referenzenauflösung während des Schreibens! Vorm Kompilieren auskommentieren!
% \bibliography{0_hauptdatei}
% \input{1_einleitung}
% \input{2_grundlagen}
% \input{3_konzeption}
% \input{4_implementierung}
% \input{5_tests}
% \input{6_zusammenfassung}
% \input{anhang}
% % Ende Imports

\section{Systemkonzept%
  \label{sec:3-konzeption}}
Auf Basis der in Abschnitt \ref{sec:2-grundlagen} vorgestellten Möglichkeiten folgt nun die Ausarbeitung eines Konzepts.
In den folgenden Abschnitten soll näher auf zwei zentrale Aspekte eingegangen werden: Abschnitt~\ref{sec:3-anbindung} stellt Möglichkeiten zum Zugriff auf Variablen bzw.\,Werte im Prozessabbild des Revolution Pi vor; in Abschnitt~\ref{sec:3-integration} wird ein Konzept zur Bereitstellung dieser Variablen auf einem OPC-Server vorgestellt.

\subsection{Anbindung der IO an den OPC-Server%
     \label{sec:3-anbindung}}

Eine Webanwendung mit Bezeichnung PiCtory dient zur Konfiguration der I/O- und virtuellen Module des RevolutionPi. Die Konfiguration liegt im JSON-Format in der Datei \lstinline{/etc/revpi/config.rsc}. Der piControl-Treiber liest diese Datei beim Start. 
Der folgende Auszug aus der Manpage des piControl-Kernelmoduls beschreibt die von diesem zum Lesen und Schreiben einzelner Bits des Prozessabbildes bereitgestellten Funktionen~\citep[vgl.]{web-revpi-manpage}. Sie ist an dieser Stelle weitgehend ungekürzt zitiert, da sie die nutzbare Schnittstelle sehr kompakt beschreibt.

\begin{lstlisting}[breakindent=0pt, numbers=none, caption={Auszug aus der Revolution Pi Programmers Manual\label{lst:4-manpage}}]
KB_FIND_VARIABLE SPIVariable *argp
Find a variable in the process image by its name. A pointer to a structure of type SPIVariable must be passed as argument. [...]
The struct SPIVariable [...] is defined as 
typedef struct SPIVariableStr
{
    char strVarName[32]; // Variable name
    uint16_t i16uAddress; // Address of the byte in the process image
    uint8_t i8uBit; // 0-7 bit position, >= 8 whole byte
    uint16_t i16uLength; // length of the variable in bits.
    // Possible values are 1, 8, 16 and 32
} SPIVariable;

Set and get values of the process image
KB_GET_VALUE SPIValue *argp
[...]
KB_SET_VALUE SPIValue *argp
Write one bit or one byte to the process image [...].  This call is more efficient than the usual calls of seek and write because only one function call is necessary. If more than on application are writing bits in one output byte, this call is the only safe way to set a bit without overwriting the other bits because this call is doing a read-modify-write-cycle. 

The struct SPIValue used by this ioctl is defined as
typedef struct SPIValueStr
{
    uint16_t i16uAddress; // Address of the byte in the process image
    uint8_t i8uBit; // 0-7 bit position, >= 8 whole byte
    uint8_t i8uValue; // Value: 0/1 for bit access, whole byte otherwise
} SPIValue;
\end{lstlisting} 

Die oben beschriebenden Funtkionen \lstinline{KB_FIND_VARIABLE}, \lstinline{KB_GET_VALUE} und \lstinline{KB_SET_VALUE} ermöglichen einen einfachen und (lt.\,Manpage) effizienten Zugriff auf einzelne Bits des Prozessabbildes und damit auch auf die IO des RevolutionPi.
Der Zugriff des OPC-Servers auf das Prozessabbild soll daher mittels dieser Funktionen realisiert werden.
\lstinline{KB_FIND_VARIABLE} kann genutzt werden, um Adressen von Variablen im Prozessabbild mittels ihres Namens aufzulösen.
\lstinline{KB_GET_VALUE} und \lstinline{KB_SET_VALUE} ermöglichen den Zugriff auf die Werte dieser Variablen.


\subsection{Integration des OPC-Servers in das System%
     \label{sec:3-integration}}

open62541 bietet drei Möglichkeiten zum Abgleich von Variablen mit dem Prozessabbild~\citep[vgl.][Tutorials - Connecting a Variable with a Physical Process]{web-open62541}:
\begin{itemize}
    \item Manuelles oder zyklisches Aktualisieren
    \item Variable Value Callback
    \item Variable Datasource
\end{itemize}

Die zyklische Aktualisierung eines oder mehrerer Werte nimmt, abhängig von der Zykluszeit, viele Systemressourcen in Anspruch. Value Callbacks ermöglichen es, einen Variablenwert effizienter mit einer Ressource wie etwa einem Prozessabbild zu synchronisieren. An die Variable wird ein Callback angehängt, welches vor jedem Lesen und nach jedem Schreibvorgang ausgeführt wird.
Der Wert der Variablen wird weiterhin im Variablenknoten auf dem OPC-Server gespeichert, der Abgleich mit der verknüpften Ressource erfolgt durch die Callback-Methoden.

Sogenannte Datenquellen gehen noch einen Schritt weiter. Der Server leitet jede Lese- und Schreibanforderung direkt an eine Callback-Funktion weiter. Beim Lesen liefert der Rückruf eine Kopie des aktuellen Wertes. Die Datenquelle muss intern ein eigenes Speichermanagement implementieren.

Der Zugriff auf die Werte des Prozessabbildes erfolgt, wie in Abschnitt~\ref{sec:3-anbindung} beschrieben, über von piControl bereitgestellte Methoden. Um die durch open62541 gepflegte OPC-Datenstruktur und das durch piControl verwaltete Prozessabbild möglichst effektiv verknüpfen zu können, soll diese Interaktion mittels Datenquellen und den zugehörigen Callbacks implementiert werden.
% % % Imports nur für Referenzenauflösung während des Schreibens! Vorm Kompilieren auskommentieren!
% \bibliography{0_hauptdatei}
% \input{1_einleitung}
% \input{2_grundlagen}
% \input{3_konzeption}
% \input{4_implementierung}
% \input{5_tests}
% \input{6_zusammenfassung}
% \input{anhang}
% % Ende Imports

\section{Implementierung%
  \label{sec:4-implementierung}}
Das folgende Kapitel stellt in Auszügen die Implementierung des OPC-Servers sowie die Anbindung an die IO-Module
der SPS dar. Der Schwerpunkt liegt hierbei auf der Funktionsweise des piControl-Treibers und dessen Integration in das Projekt. Abschnitt~\ref{sec:4-picontrol} erklärt die zum Schreibens eines Bits verwendeten Funktionsaufrufe.
Zuvor soll jedoch in Abschnitt~\ref{sec:4-open62541} der Teil des OPC-Servers vorgestellt werden, welcher auf besagten Treiber zugreift. 

\subsection{Implementierung des OPC-Servers%
     \label{sec:4-open62541}}
Wie im vorangegangenen Abschnitt~\ref{sec:3-integration} begründet, soll die Verknüpfung zwischen dem Prozessabbild der SPS und den auf dem OPC-Server bereitgestellten Werten über sog.\,Datenquellen erfolgen. Hierzu ist zunächst eine Callback-Methode zu implementieren, welche bei einem Lese- oder Schreibzugriff auf eine Variable aufgerufen wird. Die Verknüpfung zwischen Callback-Methode und Variable muss manuell erfolgen.

\begin{lstlisting}[language={c},firstnumber=237,caption={Auszug der Methode \lstinline{linkDataSourceVariable} in \lstinline{variables.c}\label{lst:4-linkDataSourceVariable}}]
extern UA_StatusCode
 linkDataSourceVariable(UA_Server *server, UA_NodeId nodeId) {
     bool readonly = false;
     UA_DataSource dataSourceVariable;
     UA_StatusCode rc; |>\setcounter{lstnumber}{254}<|

     dataSourceVariable.read = readDataSourceVariable;
     if (!readonly)
        dataSourceVariable.write = writeDataSourceVariable;
     else
        dataSourceVariable.write = writeReadonlyDataSourceVariable;

     return UA_Server_setVariableNode_dataSource(server, nodeId, dataSourceVariable);
 }
\end{lstlisting}

\begin{figure}[h]
    \centering
    \includegraphics[width=0.42\textwidth]{doc/img/OPC_RevPiDO.pdf}
    \caption{Auszug des verwendeten Nodesets, hier Digitalausgang 1 des Versuchsaufbaus
      \label{fig:opc-do}}
\end{figure}

Die in Listing~\ref{lst:4-linkDataSourceVariable} abgebildete Methode \lstinline{linkDataSourceVariable()} erzeugt ein Struct vom Typ \lstinline{UA_DataSource}. In diesem werden dem Lesen und Schreiben einer OPC-Variablen entsprechende Callback-Methoden zugewiesen. Die Verknüpfung einer OPC-Variable, genauer ihrer NodeId, mit der zuvor definierten Datenquelle erfolgt über die von open62541 bereitgestellte Methode \lstinline{UA_Server_setVariableNode_dataSource()}. Vor dem Lesen und nach dem Schreiben dieser Variable werden von nun an die entsprechenden Callbacks aufgerufen.
     
\begin{lstlisting}[language={c},firstnumber=168,caption={Auszug des Callbacks \lstinline{writeDataSourceVariable} in \lstinline{variables.c}\label{lst:4-writeDataSourceVariable}}]  
extern UA_StatusCode
 writeDataSourceVariable(UA_Server *server,
            const UA_NodeId *sessionId, void *sessionContext,
            const UA_NodeId *nodeId, void *nodeContext,
            const UA_NumericRange *range, const UA_DataValue *dataValue) {

    UA_StatusCode retval  = UA_STATUSCODE_GOOD;
    UA_NodeId *nameNodeId = UA_malloc(sizeof(UA_NodeId));
    UA_QualifiedName nameQN = UA_QUALIFIEDNAME(1, "Name");
    UA_Variant nameVar;
    UA_Boolean bit;

    retval |= findSiblingByBrowsename(server, nodeId, &nameQN, nameNodeId);
    retval |= UA_Server_readValue(server, *nameNodeId, &nameVar);
    retval |= UA_Boolean_copy(dataValue->value.data, &bit);

    |>\tikzmarkin[set border color=martinired]{writeIO}<|PI_writeSingleIO(String_fromUA_String(nameVar.data), &bit, false);                                                 |>\tikzmarkend{writeIO}<|

    free(nameNodeId);
    return retval;
 }
\end{lstlisting}

Listing~\ref{lst:4-writeDataSourceVariable} zeigt die Callback-Methode, welche nach dem Schreiben einer Variablen auf dem OPC-Server aufgerufen wird.
Dieser Methode wird neben der NodeId der mit ihr verknüpften Variablen auch der Wert dieser in Form eines Zeigers auf ein Struct vom Typ \lstinline{UA_DataValue} übergeben.

Die Gestaltung des hier verwendeten Nodesets sieht vor, dass in einer OPC-Variablen \lstinline{"Name"} der Bezeichner des zu schreibenden Digitalausgangs hinterlegt ist, siehe Abbildung~\ref{fig:opc-do}. Dies erlaubt eine Rekonfiguration der Ein- und Ausgänge der SPS ohne Änderungen im Programmcode des OPC-Servers vornehmen zu müssen.
Es ist daher erforderlich, nach jedem Schreiben einer mit einem Digitalausgang verknüpften Variablen, hier \lstinline{"Value"}, dessen Bezeichner \lstinline{"Name"} abzufragen. 
Dies geschieht in den Zeilen 180 und 181.
Anschließend wird dieser Bezeichner sowie der zu schreibende Wert der Methode \lstinline{PI_writeSingleIO()} übergeben, welche wiederum die Interaktion mit piControl übernimmt (vgl. Abschnitt \ref{sec:4-picontrol}).
 
\subsection{Integration von piControl%
     \label{sec:4-picontrol}}
In Abschnitt~\ref{sec:2-io} wurde die Anbindung der IO-Module des Revolution Pi sowie die Funktionsweise von piControl aus Anwendersicht beschrieben. Die verfügbare Literatur beschränkt sich auch auf lediglich diese Sicht; eine weiterführende Dokumentation für Entwickler gibt es, neben der in Abschnitt~\ref{sec:3-anbindung} vorgestellten Manpage, nicht. 
In diesem Abschnitt soll daher der Quellcode von piControl sowie dessen Verwendung im Projekt genauer betrachtet werden.
Hierzu wird exemplarisch die in Abschnitt~\ref{sec:4-open62541} eingeführte Methode \lstinline{PI_writeSingleIO()} untersucht.
Diese Methode ermöglicht das Setzen eines einzelnen Bits im Prozessabbild der SPS, und damit das Schalten eines digitalen Ausgangs auf einem IO-Modul.
Die äquivalente Methode \lstinline{int piControlGetBitValue(SPIValue *pSpiValue)} zum Lesen eines Bits bzw. Eingangs funktioniert analog und soll daher an dieser Stelle nicht dediziert erörtert werden.

\begin{lstlisting}[language={c},firstnumber=97,
                   caption={Setzen eines phsikalischen, digitalen Ausgangs in \lstinline{revpi.c}
                   \label{lst:4-PI_writeSingleIO}}]
extern void PI_writeSingleIO(char *pszVariableName, bool *bit, bool verbose)
{
	int rc;
	SPIVariable sPiVariable;
	SPIValue sPIValue;

	strncpy(sPiVariable.strVarName, pszVariableName, sizeof(sPiVariable.strVarName));
	rc = piControlGetVariableInfo(&sPiVariable);
	if (rc < 0) {
		printf("Cannot find variable '%s'\n", pszVariableName);
		return;
	}

		sPIValue.i16uAddress = sPiVariable.i16uAddress;
		sPIValue.i8uBit = sPiVariable.i8uBit;
		sPIValue.i8uValue = *bit;
		rc = |>\tikzmarkin[set border color=martinired]{setBitValue}<|piControlSetBitValue(&sPIValue)|>\tikzmarkend{setBitValue}<|;
		if (rc < 0)
			printf("Set bit error %s\n", getWriteError(rc));
		else if (verbose)
			printf("Set bit %d on byte at offset %d. Value %d\n", sPIValue.i8uBit, sPIValue.i16uAddress,
			       sPIValue.i8uValue);
}
\end{lstlisting}

Der Programmcode in Listing~\ref{lst:4-PI_writeSingleIO} ist Teil des implementierten OPC-Servers. In diesem wird auf zwei Funktionen des piControl-Treibers zugegriffen. 
Beiden Methoden wird als Argument ein Zeiger auf ein Struct vom Typ \lstinline{SPIValue} übergeben. Der im Struct abgelegte Name wird mittels \lstinline{piControlGetVariableInfo(&sPIValue)} zu einer Adresse im Prozessabbild aufgelöst. Diese wird in \lstinline{sPIValue.i16uAdress} gespeichert. Der Wert der Variablen wird anschließend mittels \lstinline{piControlSetBitValue(&sPIValue)} an dieser Adresse in das Prozessabbild geschrieben.

\begin{lstlisting}[language={c},firstnumber=309,caption={Methode \lstinline{piControlSetBitValue} in \lstinline{piControlIf.c}\label{lst:4-piControlSetBitValue}}]
int |>\tikzmarkin[set border color=martiniblue]{setBitValueFcn}<|piControlSetBitValue(SPIValue *pSpiValue)|>\tikzmarkend{setBitValueFcn}<|
{
    piControlOpen();

    if (PiControlHandle_g < 0)
	    return -ENODEV;

    pSpiValue->i16uAddress += pSpiValue->i8uBit / 8;
    pSpiValue->i8uBit %= 8;

    if (|>\tikzmarkin[set border color=martinired]{ioctl}<|ioctl(PiControlHandle_g, KB_SET_VALUE, pSpiValue)|>\tikzmarkend{ioctl}<| < 0)
	    return errno;

    return 0;
}
\end{lstlisting}

Die in Listing~\ref{lst:4-piControlSetBitValue} dargestellte Methode \lstinline{piControlSetBitValue} ist lediglich eine Hüllfunktion (häufig auch als Wrapper-Funktion bezeichnet) für einen Aufruf des \lstinline{ioctl} Kernel-Moduls.
Folgende Parameter werden übergeben:
\lstinline{PiControlHandle_g} ist die Referenz auf die Geräte-Datei des piControl-Treibers. \lstinline{KB_SET_VALUE} ist das ioctl-Kommando zum Schreiben eines Bits in das Prozessabbild. Der Zeiger \lstinline{pSpiValue} verweist auf ein Struct des bereits vorgestellten Typs \lstinline{SPIValue}.

\begin{lstlisting}[language={c},firstnumber=80,caption={Methode \lstinline{piControlOpen} in \lstinline{piControlIf.c}\label{lst:4-piControlOpen}}]
void piControlOpen(void)
{
    /* open handle if needed */
    if (PiControlHandle_g < 0)
    {
	    |>\tikzmarkin[set border color=martiniblue]{PiControlHandle}<|PiControlHandle_g = open(PICONTROL_DEVICE, O_RDWR)|>\tikzmarkend{PiControlHandle}<|;
    }
}
\end{lstlisting}

Die in Listing~\ref{lst:4-piControlOpen} dargestellte Methode öffnet, sofern nicht bereits geschehen, die Geräte-Datei. Das Macro \lstinline{PICONTROL_DEVICE} verweist hierbei auf \lstinline{/dev/piControl0}.

\begin{lstlisting}[language={c},firstnumber=721,caption={Methode \lstinline{piControlIoctl} in \lstinline{piControlMain.c}\label{lst:4-piControlIoctl}}]
static long |>\tikzmarkin[set border color=martiniblue, below offset=0.9em]{piControlIoctl}<|piControlIoctl(struct file *file, unsigned int prg_nr, 
                           unsigned long usr_addr)                                      |>\tikzmarkend{piControlIoctl}<|
{
  int status = -EFAULT;
  tpiControlInst *priv;
  int timeout = 10000;	// ms

  if (prg_nr != KB_CONFIG_SEND && prg_nr != KB_CONFIG_START && !isRunning()) {
  	return -EAGAIN;
  }

  priv = (tpiControlInst *) file->private_data;

  if (prg_nr != KB_GET_LAST_MESSAGE) {
  	// clear old message
  	priv->pcErrorMessage[0] = 0;
  }

  switch (prg_nr) {|>\setcounter{lstnumber}{864}<|

    case |>\tikzmarkin[set border color=martiniblue]{KB_SET_VALUE}<|KB_SET_VALUE:|>\tikzmarkend{KB_SET_VALUE}<|
  		{
  			SPIValue *pValue = (SPIValue *) usr_addr;

  			if (!isRunning())
  				return -EFAULT;

  			if (pValue->i16uAddress >= KB_PI_LEN) {
  				status = -EFAULT;
  			} else {
  				INT8U i8uValue_l;
  				my_rt_mutex_lock(&piDev_g.lockPI);
  				i8uValue_l = piDev_g.ai8uPI[pValue->i16uAddress];

  				if (pValue->i8uBit >= 8) {
  					i8uValue_l = pValue->i8uValue;
  				} else {
  					if (pValue->i8uValue)
  						i8uValue_l |= (1 << pValue->i8uBit);
  					else
  						i8uValue_l &= ~(1 << pValue->i8uBit);
  				}

  				|>\tikzmarkin[set border color=martinired]{i8uValue}<|piDev_g.ai8uPI[pValue->i16uAddress] = i8uValue_l;|>\tikzmarkend{i8uValue}<|
  				rt_mutex_unlock(&piDev_g.lockPI);

  #ifdef VERBOSE
  				pr_info("piControlIoctl Addr=%u, bit=%u: %02x %02x\n", pValue->i16uAddress, pValue->i8uBit, pValue->i8uValue, i8uValue_l);
  #endif

  				status = 0;
  			}
  		}
  		break; |>\setcounter{lstnumber}{1314}<|

    default:
      pr_err("Invalid Ioctl");
      return (-EINVAL);
      break;

    }

    return status;
  }
\end{lstlisting}

Listing~\ref{lst:4-piControlIoctl} zeigt in Auszügen die ioctl-Methode des piControl Kernel-Treibers. Diese bekommt folgende Argumente übergeben: \lstinline{struct file *file} enthält den Verweis auf die Geräte-Datei, hier \lstinline{/dev/piControl0}. Der Wert von \lstinline{unsigned int prg_nr} beschreibt die Anfrage an den Treiber, in diesem Fall \lstinline{KB_SET_VALUE}. Das Argument \lstinline{unsigned long usr_addr} enthält einen typ-agnostischen Pointer. Dieser verweist auf einen Speicherbereich, in welchem die zur Bearbeitung der Anfrage notwendigen Daten abgelegt sind. Hier können auch vom Treiber empfangene Daten dem Anwendungsprogramm bereitgestellt werden. 

Die switch-case-Anweisung führt die über das Argument \lstinline{prg_nr} spezifizierte Aktion aus. Hier betrachten wir \lstinline{KB_SET_VALUE}:
Zunächst wird in Zeile 868 der übergebene Zeiger \lstinline{usr_addr} mittels explizitem Typecast zu einem Zeiger des Typs \lstinline{SPIValue *} konvertiert. Da dieser auf Daten im Userspace verweist, ist beim Zugriff durch den Kernel-Treiber besondere Vorsicht geboten.
In Zeile 877 wird mittels Mutex das Prozessabbild \lstinline{piDev_g} für den Zugriff durch andere Threads oder Prozesse gesperrt.
\lstinline{my_rt_mutex_lock} verweist hierbei auf die Funktion \lstinline{rt_mutex_lock} aus \lstinline{linux/sched.h}\footnote{Offenbar wurde hier auch eine alternative Implementierung vorgesehen, siehe revpi\_common.h}

In Zeile 889 wird das Byte \lstinline{i8uValue_l}, welches den zu schreibenden Wert enthält in das Prozessabbild übertragen. Anschließend wird die Mutex auf \lstinline{piDev_g} wieder entsperrt.
\newpage

\begin{lstlisting}[language={c},firstnumber=62,caption={Auszug des Struct \lstinline{spiControlDev} in \lstinline{piControlMain.h}\label{lst:4-spiControlDev}}]
|>\tikzmarkin[set border color=martiniblue]{spiControlDev}<|typedef struct spiControlDev|>\tikzmarkend{spiControlDev}<| {
	// device driver stuff
	int init_step;
	enum revpi_machine machine_type;
	void *machine;
	struct cdev cdev;	// Char device structure
	struct device *dev;
	struct thermal_zone_device *thermal_zone;

	|>\tikzmarkin[set border color=martiniblue]{processImage}<|// process image stuff
	INT8U ai8uPI[KB_PI_LEN];
	INT8U ai8uPIDefault|>\tikzmarkin[set border color=martinired]{KB_PI_LEN_0}<|[KB_PI_LEN]|>\tikzmarkend{KB_PI_LEN_0}<|;
	struct rt_mutex lockPI;        |>\tikzmarkend{processImage}<|
	bool stopIO;
	piDevices *devs; |>\setcounter{lstnumber}{94}<|
} tpiControlDev;
\end{lstlisting}

Das Prozessabbild ist als Byte-Array der Länge \lstinline{KB_PI_LEN} in Listing~\ref{lst:4-spiControlDev} definiert. Konfigurationsparameter wie \lstinline{KB_PI_LEN} oder die Zykluszeit für den Datenaustausch zwischen SPS und IO-Modulen sind im folgenden Listing~\ref{lst:4-process} definiert.

\begin{lstlisting}[language={c},firstnumber=119,caption={Konfigurationsparameter des Prozessabbildes in project.h\label{lst:4-process}}]
#define INTERVAL_PI_GATE (5*1000*1000)  // 5 ms piGateCommunication |>\setcounter{lstnumber}{128}<|

#define INTERVAL_IO_COM (5*1000*1000)  // 5 ms piIoComm |>\setcounter{lstnumber}{132}<|

#define KB_PD_LEN       512
|>\tikzmarkin[set border color=martiniblue]{KB_PI_LEN_1}<|#define KB_PI_LEN       4096|>\tikzmarkend{KB_PI_LEN_1}<|
\end{lstlisting}

Das zu setzende Bit wurde zu diesem Zeitpunkt erfolgreich in das Prozessabbild der SPS geschrieben.
Es stellt sich die Frage, wie dieses nun an das IO-Modul kommuniziert wird.
Die Kommunikation mit allen angebundenen Modulen ist ebenfalls Aufgabe des piControl-Treibers.

\begin{lstlisting}[language={c},firstnumber=256,caption={Auszug der Methode \lstinline{piIoThread} in \lstinline{revpi_core.c}\label{lst:4-piIoThread}}]
static int piIoThread(void *data)
{
	//TODO int value = 0;
	ktime_t time;
	ktime_t now;
	s64 tDiff;

	hrtimer_init(&piCore_g.ioTimer, CLOCK_MONOTONIC, HRTIMER_MODE_ABS);
	piCore_g.ioTimer.function = piIoTimer;

	pr_info("piIO thread started\n");

	now = hrtimer_cb_get_time(&piCore_g.ioTimer);

	PiBridgeMaster_Reset();

	while (!kthread_should_stop()) {
		if (|>\tikzmarkin[set border color=martinired]{PiBridgeMaster}<|PiBridgeMaster_Run()|>\tikzmarkend{PiBridgeMaster}<| < 0)
			break;
	}

	RevPiDevice_finish();

	pr_info("piIO exit\n");
	return 0;
}
\end{lstlisting}

Der Kernel-Thread \lstinline{piIoThread} ist verantwortlich für den zyklischen Datenaustausch mit den IO-Modulen. In diesem wird fortlaufend die Methode \lstinline{PiBridgeMaster_Run()} aufgerufen, siehe Listing~\ref{lst:4-piIoThread}.

\begin{lstlisting}[language={c},firstnumber=262,caption={Auszug der Methode \lstinline{PiBridgeMaster_Run(void)} in \lstinline{RevPiDevice.c}\label{lst:4-PiBridgeMaster_Run}}]
int PiBridgeMaster_Run(void)
{
	static kbUT_Timer tTimeoutTimer_s;
	static kbUT_Timer tConfigTimeoutTimer_s;
	static int error_cnt;
	static INT8U last_led;
	static unsigned long last_update;
	int ret = 0;
	int i;

	my_rt_mutex_lock(&piCore_g.lockBridgeState);
	if (piCore_g.eBridgeState != piBridgeStop) {
		switch (eRunStatus_s) { |>\setcounter{lstnumber}{514}<|
		    case enPiBridgeMasterStatus_EndOfConfig:|>\setcounter{lstnumber}{621}<|
		    if (|>\tikzmarkin[set border color=martinired]{RevPiDevice}<|RevPiDevice_run()|>\tikzmarkend{RevPiDevice}<|) {
				// an error occured, check error limits |>\setcounter{lstnumber}{641}<|
			} else {
				ret = 1;
			}
			piCore_g.image.drv.i16uRS485ErrorCnt = RevPiDevice_getErrCnt();
			break;
\end{lstlisting}

Die in Listing~\ref{lst:4-PiBridgeMaster_Run} dargestellte Methode ist eine sog. State-Machine. Ist die Konfiguration der IO-Module erfolgreich abgeschlossen, so führt sie bei Aufruf lediglich die Methode \lstinline{RevPiDevice_run()} aus.

\begin{lstlisting}[language={c},firstnumber=140,caption={Auszug der Methode \lstinline{RevPiDevice_run(void)} in \lstinline{RevPiDevice.c}\label{lst:4-RevPiDevice_run}}]
int RevPiDevice_run(void)
{
	INT8U i8uDevice = 0;
	INT32U r;
	int retval = 0;

	RevPiDevices_s.i16uErrorCnt = 0;

	for (i8uDevice = 0; i8uDevice < RevPiDevice_getDevCnt(); i8uDevice++) {
		if (RevPiDevice_getDev(i8uDevice)->i8uActive) {
			switch (RevPiDevice_getDev(i8uDevice)->sId.i16uModulType) {
			case KUNBUS_FW_DESCR_TYP_PI_DIO_14:
			case KUNBUS_FW_DESCR_TYP_PI_DI_16:
			case KUNBUS_FW_DESCR_TYP_PI_DO_16:
				r = |>\tikzmarkin[set border color=martinired]{sendCyclicTelegram}<|piDIOComm_sendCyclicTelegram(i8uDevice)|>\tikzmarkend{sendCyclicTelegram}\setcounter{lstnumber}{166} <|;

				break; |>\setcounter{lstnumber}{216}<|
			}
		}
	} |>\setcounter{lstnumber}{227}<|
	return retval;
}
\end{lstlisting}

Diese iteriert wie in Listing~\ref{lst:4-RevPiDevice_run} abgebildete durch alle gegenwärtig in der SPS konfigurierten Module. Ist das aktuelle Modul als aktiv markiert, so wird anhand eines sog. Firmware-Descriptors entschieden, welche Methode für die Ansteuerung des Moduls aufzurufen ist.

\begin{lstlisting}[language={c},firstnumber=161,caption={Auszug der Methode \lstinline{piDIOComm_sendCyclicTelegram} in \lstinline{piDIOComm.c}\label{lst:4-sendCyclicTelegram}}]
INT32U piDIOComm_sendCyclicTelegram(INT8U i8uDevice_p)
{
	INT32U i32uRv_l = 0;
	SIOGeneric sRequest_l;
	SIOGeneric sResponse_l;
	INT8U len_l, data_out[18], i, p, data_in[70];
	INT8U i8uAddress;
	int ret; |>\setcounter{lstnumber}{239}<|
	
    |>\tikzmarkin[set border color=martinired]{piIoComm}<|ret = piIoComm_send((INT8U *) & sRequest_l, IOPROTOCOL_HEADER_LENGTH + len_l + 1);  |>\tikzmarkend{piIoComm}\setcounter{lstnumber}{298}<|
}
\end{lstlisting}

Im Falle des hier verwendeten DO-Moduls wird die in Listing~\ref{lst:4-sendCyclicTelegram} abgebildete Methode \lstinline{piDIOComm_sendCyclicTelegram()} aufgerufen. Dieser wird ein Zeiger auf das zu schreibende Gerät übergeben. 
Zunächst wird das Prozessabbild mittels eines proprietären, jedoch im Quellcode offen nachvollziehbaren Protokolls in ein \lstinline{sRequest_l} genanntes Byte-Array umgewandelt. Dieser Schritt ist in Listing~\ref{lst:4-sendCyclicTelegram} nicht abgebildet. Anschließend wird \lstinline{piIoComm_send()} ein Zeiger auf die so generierte Schreib-Anfrage übergeben.

\begin{lstlisting}[language={c},firstnumber=220,caption={Auszug der Methode \lstinline{piIOComm_send} in \lstinline{piIOComm.c}\label{lst:4-piIOComm_send}}]
int piIoComm_send(INT8U * buf_p, INT16U i16uLen_p)
{
	ssize_t write_l = 0;
	INT16U i16uSent_l = 0;|>\setcounter{lstnumber}{249}<|

	while (i16uSent_l < i16uLen_p) {
		write_l = vfs_write(piIoComm_fd_m, buf_p + i16uSent_l, i16uLen_p - i16uSent_l, &piIoComm_fd_m->f_pos);
		if (write_l < 0) {
			pr_info_serial("write error %d\n", (int)write_l);
			return -1;
		} 
		i16uSent_l += write_l;|>\setcounter{lstnumber}{263}<|
	}
	clear();
	vfs_fsync(piIoComm_fd_m, 1);
	return 0;
}
\end{lstlisting}

Listing~\ref{lst:4-piIOComm_send} zeigt die Implementierung von \lstinline{piIoComm_send()}. Diese Methode ist für das Schreiben der oben generierten Anfrage auf die seriellen Schnittstelle verantwortlich. Realisiert wird dies mittels der Methode \lstinline{vfs_write()}. Diese ist in \lstinline{<linux/fs.h>} definiert. Sie ermöglicht das Schreiben einer Datei im Userspace aus dem Kernel heraus. Geschrieben wird hier die Datei mit dem Deskriptor \lstinline{piIoComm_fd_m}.
Da die Funktion \lstinline{vfs_write()} durch andere Kernel-Tasks unterbrochen werden kann, ist nicht gewährleistet, dass die gesamte Anfrage mit nur einem Aufruf geschrieben wird. Die oben abgebildete while-Schleife stellt das vollständige Senden der Anfrage sicher.

\begin{lstlisting}[language={c},firstnumber=157,caption={Auszug der Methode \lstinline{piIOComm_open_serial} in \lstinline{piIOComm.c}\label{lst:4-piIOComm_open_serial}}]
int piIoComm_open_serial(void)
{   |>\setcounter{lstnumber}{167}<|
	struct file *fd;	/* Filedeskriptor */
	struct termios newtio;	/* Schnittstellenoptionen */

	|>\tikzmarkin[set border color=martiniblue]{fd}<|/* Port oeffnen - read/write, kein "controlling tty", 
	    Status von DCD ignorieren */
	fd = filp_open(|>\tikzmarkin[set border color=martinired]{tty}<|REV_PI_TTY_DEVICE|>\tikzmarkend{tty}<|, O_RDWR | O_NOCTTY, 0); |>\setcounter{lstnumber}{208}<|
	
	piIoComm_fd_m = fd;                                                      |>\tikzmarkend{fd}\setcounter{lstnumber}{217}<|

	return 0;
}
\end{lstlisting}

Der zum Schreiben auf die serielle Schnittstelle verwendete Datei-Deskriptor wird von der in Listing~\ref{lst:4-piIOComm_open_serial} abgebildeten Methode \lstinline{piIoComm_open_serial()} generiert. 

\begin{lstlisting}[language={c},firstnumber=45,caption={Definition der seriellen Schnittstelle in \lstinline{piIOComm.h}\label{lst:4-REV_PI_TTY_DEVICE}}]
#define REV_PI_TTY_DEVICE	"/dev/ttyAMA0"
\end{lstlisting}

Das in Listing~\ref{lst:4-REV_PI_TTY_DEVICE} definierte Macro verweist auf eine der seriellen Schnittstellen des RaspberryPi.
Die Implementierung des zugehörigen Schnittstellentreibers soll hier nicht weiter untersucht werden. Somit ist an dieser Stelle die Kette vom Setzen einer Variablen auf dem OPC-Server bis hin zur Aktualisierung des Prozessabbilds der IO-Module geschlossen.

% \begin{lstlisting}[language={c},firstnumber={226},caption={Setzen der Scheduler-Priorität auf SCHED\_FIFO in 
% revpi\_common.c\label{lst:2-sched_priority}}]
% param.sched_priority = ktprio->prio;
% ret = sched_setscheduler(child, SCHED_FIFO, &param);
% \end{lstlisting}
% % % Imports nur für Referenzenauflösung während des Schreibens! Vorm Kompilieren auskommentieren!
% \bibliography{0_hauptdatei}
% \input{1_einleitung}
% \input{2_grundlagen}
% \input{3_konzeption}
% \input{4_implementierung}
% \input{5_tests}
% \input{6_zusammenfassung}
% % Ende Imports

\section{Test des OPC-Servers im Gesamtsystem%
  \label{sec:5-tests}}

% % % Imports nur für Referenzenauflösung während des schreibens! Vorm Kompilieren auskommentieren!
% \bibliography{0_hauptdatei}
% \input{1_einleitung}
% \input{2_grundlagen}
% \input{3_konzeption}
% \input{4_implementierung}
% \input{5_tests}
% \input{6_zusammenfassung}
% % Ende Imports

\section{Zusammenfassung und Ausblick%
  \label{sec:6-fazit}}
Der folgende Abschnitt~\ref{sec:6-zusammenfassung} fasst die gewonnenen Erkenntnisse und den Stand der Implementierung zusammen.
Den Abschluss dieser Arbeit bildet der Ausblick in Abschnitt~\ref{sec:6-ausblick}.

\subsection{Zusammenfassung%
     \label{sec:6-zusammenfassung}}

\subsection{Ausblick%
     \label{sec:6-ausblick}}

% \input{anhang}
% % Ende Imports

\section{Implementierung%
  \label{sec:4-implementierung}}
Das folgende Kapitel stellt in Auszügen die Implementierung des OPC-Servers sowie die Anbindung an die IO-Module
der SPS dar. Der Schwerpunkt liegt hierbei auf der Funktionsweise des piControl-Treibers und dessen Integration in das Projekt. Abschnitt~\ref{sec:4-picontrol} erklärt die zum Schreibens eines Bits verwendeten Funktionsaufrufe.
Zuvor soll jedoch in Abschnitt~\ref{sec:4-open62541} der Teil des OPC-Servers vorgestellt werden, welcher auf besagten Treiber zugreift. 

\subsection{Implementierung des OPC-Servers%
     \label{sec:4-open62541}}
Wie im vorangegangenen Abschnitt~\ref{sec:3-integration} begründet, soll die Verknüpfung zwischen dem Prozessabbild der SPS und den auf dem OPC-Server bereitgestellten Werten über sog.\,Datenquellen erfolgen. Hierzu ist zunächst eine Callback-Methode zu implementieren, welche bei einem Lese- oder Schreibzugriff auf eine Variable aufgerufen wird. Die Verknüpfung zwischen Callback-Methode und Variable muss manuell erfolgen.

\begin{lstlisting}[language={c},firstnumber=237,caption={Auszug der Methode \lstinline{linkDataSourceVariable} in \lstinline{variables.c}\label{lst:4-linkDataSourceVariable}}]
extern UA_StatusCode
 linkDataSourceVariable(UA_Server *server, UA_NodeId nodeId) {
     bool readonly = false;
     UA_DataSource dataSourceVariable;
     UA_StatusCode rc; |>\setcounter{lstnumber}{254}<|

     dataSourceVariable.read = readDataSourceVariable;
     if (!readonly)
        dataSourceVariable.write = writeDataSourceVariable;
     else
        dataSourceVariable.write = writeReadonlyDataSourceVariable;

     return UA_Server_setVariableNode_dataSource(server, nodeId, dataSourceVariable);
 }
\end{lstlisting}

\begin{figure}[h]
    \centering
    \includegraphics[width=0.42\textwidth]{doc/img/OPC_RevPiDO.pdf}
    \caption{Auszug des verwendeten Nodesets, hier Digitalausgang 1 des Versuchsaufbaus
      \label{fig:opc-do}}
\end{figure}

Die in Listing~\ref{lst:4-linkDataSourceVariable} abgebildete Methode \lstinline{linkDataSourceVariable()} erzeugt ein Struct vom Typ \lstinline{UA_DataSource}. In diesem werden dem Lesen und Schreiben einer OPC-Variablen entsprechende Callback-Methoden zugewiesen. Die Verknüpfung einer OPC-Variable, genauer ihrer NodeId, mit der zuvor definierten Datenquelle erfolgt über die von open62541 bereitgestellte Methode \lstinline{UA_Server_setVariableNode_dataSource()}. Vor dem Lesen und nach dem Schreiben dieser Variable werden von nun an die entsprechenden Callbacks aufgerufen.
     
\begin{lstlisting}[language={c},firstnumber=168,caption={Auszug des Callbacks \lstinline{writeDataSourceVariable} in \lstinline{variables.c}\label{lst:4-writeDataSourceVariable}}]  
extern UA_StatusCode
 writeDataSourceVariable(UA_Server *server,
            const UA_NodeId *sessionId, void *sessionContext,
            const UA_NodeId *nodeId, void *nodeContext,
            const UA_NumericRange *range, const UA_DataValue *dataValue) {

    UA_StatusCode retval  = UA_STATUSCODE_GOOD;
    UA_NodeId *nameNodeId = UA_malloc(sizeof(UA_NodeId));
    UA_QualifiedName nameQN = UA_QUALIFIEDNAME(1, "Name");
    UA_Variant nameVar;
    UA_Boolean bit;

    retval |= findSiblingByBrowsename(server, nodeId, &nameQN, nameNodeId);
    retval |= UA_Server_readValue(server, *nameNodeId, &nameVar);
    retval |= UA_Boolean_copy(dataValue->value.data, &bit);

    |>\tikzmarkin[set border color=martinired]{writeIO}<|PI_writeSingleIO(String_fromUA_String(nameVar.data), &bit, false);                                                 |>\tikzmarkend{writeIO}<|

    free(nameNodeId);
    return retval;
 }
\end{lstlisting}

Listing~\ref{lst:4-writeDataSourceVariable} zeigt die Callback-Methode, welche nach dem Schreiben einer Variablen auf dem OPC-Server aufgerufen wird.
Dieser Methode wird neben der NodeId der mit ihr verknüpften Variablen auch der Wert dieser in Form eines Zeigers auf ein Struct vom Typ \lstinline{UA_DataValue} übergeben.

Die Gestaltung des hier verwendeten Nodesets sieht vor, dass in einer OPC-Variablen \lstinline{"Name"} der Bezeichner des zu schreibenden Digitalausgangs hinterlegt ist, siehe Abbildung~\ref{fig:opc-do}. Dies erlaubt eine Rekonfiguration der Ein- und Ausgänge der SPS ohne Änderungen im Programmcode des OPC-Servers vornehmen zu müssen.
Es ist daher erforderlich, nach jedem Schreiben einer mit einem Digitalausgang verknüpften Variablen, hier \lstinline{"Value"}, dessen Bezeichner \lstinline{"Name"} abzufragen. 
Dies geschieht in den Zeilen 180 und 181.
Anschließend wird dieser Bezeichner sowie der zu schreibende Wert der Methode \lstinline{PI_writeSingleIO()} übergeben, welche wiederum die Interaktion mit piControl übernimmt (vgl. Abschnitt \ref{sec:4-picontrol}).
 
\subsection{Integration von piControl%
     \label{sec:4-picontrol}}
In Abschnitt~\ref{sec:2-io} wurde die Anbindung der IO-Module des Revolution Pi sowie die Funktionsweise von piControl aus Anwendersicht beschrieben. Die verfügbare Literatur beschränkt sich auch auf lediglich diese Sicht; eine weiterführende Dokumentation für Entwickler gibt es, neben der in Abschnitt~\ref{sec:3-anbindung} vorgestellten Manpage, nicht. 
In diesem Abschnitt soll daher der Quellcode von piControl sowie dessen Verwendung im Projekt genauer betrachtet werden.
Hierzu wird exemplarisch die in Abschnitt~\ref{sec:4-open62541} eingeführte Methode \lstinline{PI_writeSingleIO()} untersucht.
Diese Methode ermöglicht das Setzen eines einzelnen Bits im Prozessabbild der SPS, und damit das Schalten eines digitalen Ausgangs auf einem IO-Modul.
Die äquivalente Methode \lstinline{int piControlGetBitValue(SPIValue *pSpiValue)} zum Lesen eines Bits bzw. Eingangs funktioniert analog und soll daher an dieser Stelle nicht dediziert erörtert werden.

\begin{lstlisting}[language={c},firstnumber=97,
                   caption={Setzen eines phsikalischen, digitalen Ausgangs in \lstinline{revpi.c}
                   \label{lst:4-PI_writeSingleIO}}]
extern void PI_writeSingleIO(char *pszVariableName, bool *bit, bool verbose)
{
	int rc;
	SPIVariable sPiVariable;
	SPIValue sPIValue;

	strncpy(sPiVariable.strVarName, pszVariableName, sizeof(sPiVariable.strVarName));
	rc = piControlGetVariableInfo(&sPiVariable);
	if (rc < 0) {
		printf("Cannot find variable '%s'\n", pszVariableName);
		return;
	}

		sPIValue.i16uAddress = sPiVariable.i16uAddress;
		sPIValue.i8uBit = sPiVariable.i8uBit;
		sPIValue.i8uValue = *bit;
		rc = |>\tikzmarkin[set border color=martinired]{setBitValue}<|piControlSetBitValue(&sPIValue)|>\tikzmarkend{setBitValue}<|;
		if (rc < 0)
			printf("Set bit error %s\n", getWriteError(rc));
		else if (verbose)
			printf("Set bit %d on byte at offset %d. Value %d\n", sPIValue.i8uBit, sPIValue.i16uAddress,
			       sPIValue.i8uValue);
}
\end{lstlisting}

Der Programmcode in Listing~\ref{lst:4-PI_writeSingleIO} ist Teil des implementierten OPC-Servers. In diesem wird auf zwei Funktionen des piControl-Treibers zugegriffen. 
Beiden Methoden wird als Argument ein Zeiger auf ein Struct vom Typ \lstinline{SPIValue} übergeben. Der im Struct abgelegte Name wird mittels \lstinline{piControlGetVariableInfo(&sPIValue)} zu einer Adresse im Prozessabbild aufgelöst. Diese wird in \lstinline{sPIValue.i16uAdress} gespeichert. Der Wert der Variablen wird anschließend mittels \lstinline{piControlSetBitValue(&sPIValue)} an dieser Adresse in das Prozessabbild geschrieben.

\begin{lstlisting}[language={c},firstnumber=309,caption={Methode \lstinline{piControlSetBitValue} in \lstinline{piControlIf.c}\label{lst:4-piControlSetBitValue}}]
int |>\tikzmarkin[set border color=martiniblue]{setBitValueFcn}<|piControlSetBitValue(SPIValue *pSpiValue)|>\tikzmarkend{setBitValueFcn}<|
{
    piControlOpen();

    if (PiControlHandle_g < 0)
	    return -ENODEV;

    pSpiValue->i16uAddress += pSpiValue->i8uBit / 8;
    pSpiValue->i8uBit %= 8;

    if (|>\tikzmarkin[set border color=martinired]{ioctl}<|ioctl(PiControlHandle_g, KB_SET_VALUE, pSpiValue)|>\tikzmarkend{ioctl}<| < 0)
	    return errno;

    return 0;
}
\end{lstlisting}

Die in Listing~\ref{lst:4-piControlSetBitValue} dargestellte Methode \lstinline{piControlSetBitValue} ist lediglich eine Hüllfunktion (häufig auch als Wrapper-Funktion bezeichnet) für einen Aufruf des \lstinline{ioctl} Kernel-Moduls.
Folgende Parameter werden übergeben:
\lstinline{PiControlHandle_g} ist die Referenz auf die Geräte-Datei des piControl-Treibers. \lstinline{KB_SET_VALUE} ist das ioctl-Kommando zum Schreiben eines Bits in das Prozessabbild. Der Zeiger \lstinline{pSpiValue} verweist auf ein Struct des bereits vorgestellten Typs \lstinline{SPIValue}.

\begin{lstlisting}[language={c},firstnumber=80,caption={Methode \lstinline{piControlOpen} in \lstinline{piControlIf.c}\label{lst:4-piControlOpen}}]
void piControlOpen(void)
{
    /* open handle if needed */
    if (PiControlHandle_g < 0)
    {
	    |>\tikzmarkin[set border color=martiniblue]{PiControlHandle}<|PiControlHandle_g = open(PICONTROL_DEVICE, O_RDWR)|>\tikzmarkend{PiControlHandle}<|;
    }
}
\end{lstlisting}

Die in Listing~\ref{lst:4-piControlOpen} dargestellte Methode öffnet, sofern nicht bereits geschehen, die Geräte-Datei. Das Macro \lstinline{PICONTROL_DEVICE} verweist hierbei auf \lstinline{/dev/piControl0}.

\begin{lstlisting}[language={c},firstnumber=721,caption={Methode \lstinline{piControlIoctl} in \lstinline{piControlMain.c}\label{lst:4-piControlIoctl}}]
static long |>\tikzmarkin[set border color=martiniblue, below offset=0.9em]{piControlIoctl}<|piControlIoctl(struct file *file, unsigned int prg_nr, 
                           unsigned long usr_addr)                                      |>\tikzmarkend{piControlIoctl}<|
{
  int status = -EFAULT;
  tpiControlInst *priv;
  int timeout = 10000;	// ms

  if (prg_nr != KB_CONFIG_SEND && prg_nr != KB_CONFIG_START && !isRunning()) {
  	return -EAGAIN;
  }

  priv = (tpiControlInst *) file->private_data;

  if (prg_nr != KB_GET_LAST_MESSAGE) {
  	// clear old message
  	priv->pcErrorMessage[0] = 0;
  }

  switch (prg_nr) {|>\setcounter{lstnumber}{864}<|

    case |>\tikzmarkin[set border color=martiniblue]{KB_SET_VALUE}<|KB_SET_VALUE:|>\tikzmarkend{KB_SET_VALUE}<|
  		{
  			SPIValue *pValue = (SPIValue *) usr_addr;

  			if (!isRunning())
  				return -EFAULT;

  			if (pValue->i16uAddress >= KB_PI_LEN) {
  				status = -EFAULT;
  			} else {
  				INT8U i8uValue_l;
  				my_rt_mutex_lock(&piDev_g.lockPI);
  				i8uValue_l = piDev_g.ai8uPI[pValue->i16uAddress];

  				if (pValue->i8uBit >= 8) {
  					i8uValue_l = pValue->i8uValue;
  				} else {
  					if (pValue->i8uValue)
  						i8uValue_l |= (1 << pValue->i8uBit);
  					else
  						i8uValue_l &= ~(1 << pValue->i8uBit);
  				}

  				|>\tikzmarkin[set border color=martinired]{i8uValue}<|piDev_g.ai8uPI[pValue->i16uAddress] = i8uValue_l;|>\tikzmarkend{i8uValue}<|
  				rt_mutex_unlock(&piDev_g.lockPI);

  #ifdef VERBOSE
  				pr_info("piControlIoctl Addr=%u, bit=%u: %02x %02x\n", pValue->i16uAddress, pValue->i8uBit, pValue->i8uValue, i8uValue_l);
  #endif

  				status = 0;
  			}
  		}
  		break; |>\setcounter{lstnumber}{1314}<|

    default:
      pr_err("Invalid Ioctl");
      return (-EINVAL);
      break;

    }

    return status;
  }
\end{lstlisting}

Listing~\ref{lst:4-piControlIoctl} zeigt in Auszügen die ioctl-Methode des piControl Kernel-Treibers. Diese bekommt folgende Argumente übergeben: \lstinline{struct file *file} enthält den Verweis auf die Geräte-Datei, hier \lstinline{/dev/piControl0}. Der Wert von \lstinline{unsigned int prg_nr} beschreibt die Anfrage an den Treiber, in diesem Fall \lstinline{KB_SET_VALUE}. Das Argument \lstinline{unsigned long usr_addr} enthält einen typ-agnostischen Pointer. Dieser verweist auf einen Speicherbereich, in welchem die zur Bearbeitung der Anfrage notwendigen Daten abgelegt sind. Hier können auch vom Treiber empfangene Daten dem Anwendungsprogramm bereitgestellt werden. 

Die switch-case-Anweisung führt die über das Argument \lstinline{prg_nr} spezifizierte Aktion aus. Hier betrachten wir \lstinline{KB_SET_VALUE}:
Zunächst wird in Zeile 868 der übergebene Zeiger \lstinline{usr_addr} mittels explizitem Typecast zu einem Zeiger des Typs \lstinline{SPIValue *} konvertiert. Da dieser auf Daten im Userspace verweist, ist beim Zugriff durch den Kernel-Treiber besondere Vorsicht geboten.
In Zeile 877 wird mittels Mutex das Prozessabbild \lstinline{piDev_g} für den Zugriff durch andere Threads oder Prozesse gesperrt.
\lstinline{my_rt_mutex_lock} verweist hierbei auf die Funktion \lstinline{rt_mutex_lock} aus \lstinline{linux/sched.h}\footnote{Offenbar wurde hier auch eine alternative Implementierung vorgesehen, siehe revpi\_common.h}

In Zeile 889 wird das Byte \lstinline{i8uValue_l}, welches den zu schreibenden Wert enthält in das Prozessabbild übertragen. Anschließend wird die Mutex auf \lstinline{piDev_g} wieder entsperrt.
\newpage

\begin{lstlisting}[language={c},firstnumber=62,caption={Auszug des Struct \lstinline{spiControlDev} in \lstinline{piControlMain.h}\label{lst:4-spiControlDev}}]
|>\tikzmarkin[set border color=martiniblue]{spiControlDev}<|typedef struct spiControlDev|>\tikzmarkend{spiControlDev}<| {
	// device driver stuff
	int init_step;
	enum revpi_machine machine_type;
	void *machine;
	struct cdev cdev;	// Char device structure
	struct device *dev;
	struct thermal_zone_device *thermal_zone;

	|>\tikzmarkin[set border color=martiniblue]{processImage}<|// process image stuff
	INT8U ai8uPI[KB_PI_LEN];
	INT8U ai8uPIDefault|>\tikzmarkin[set border color=martinired]{KB_PI_LEN_0}<|[KB_PI_LEN]|>\tikzmarkend{KB_PI_LEN_0}<|;
	struct rt_mutex lockPI;        |>\tikzmarkend{processImage}<|
	bool stopIO;
	piDevices *devs; |>\setcounter{lstnumber}{94}<|
} tpiControlDev;
\end{lstlisting}

Das Prozessabbild ist als Byte-Array der Länge \lstinline{KB_PI_LEN} in Listing~\ref{lst:4-spiControlDev} definiert. Konfigurationsparameter wie \lstinline{KB_PI_LEN} oder die Zykluszeit für den Datenaustausch zwischen SPS und IO-Modulen sind im folgenden Listing~\ref{lst:4-process} definiert.

\begin{lstlisting}[language={c},firstnumber=119,caption={Konfigurationsparameter des Prozessabbildes in project.h\label{lst:4-process}}]
#define INTERVAL_PI_GATE (5*1000*1000)  // 5 ms piGateCommunication |>\setcounter{lstnumber}{128}<|

#define INTERVAL_IO_COM (5*1000*1000)  // 5 ms piIoComm |>\setcounter{lstnumber}{132}<|

#define KB_PD_LEN       512
|>\tikzmarkin[set border color=martiniblue]{KB_PI_LEN_1}<|#define KB_PI_LEN       4096|>\tikzmarkend{KB_PI_LEN_1}<|
\end{lstlisting}

Das zu setzende Bit wurde zu diesem Zeitpunkt erfolgreich in das Prozessabbild der SPS geschrieben.
Es stellt sich die Frage, wie dieses nun an das IO-Modul kommuniziert wird.
Die Kommunikation mit allen angebundenen Modulen ist ebenfalls Aufgabe des piControl-Treibers.

\begin{lstlisting}[language={c},firstnumber=256,caption={Auszug der Methode \lstinline{piIoThread} in \lstinline{revpi_core.c}\label{lst:4-piIoThread}}]
static int piIoThread(void *data)
{
	//TODO int value = 0;
	ktime_t time;
	ktime_t now;
	s64 tDiff;

	hrtimer_init(&piCore_g.ioTimer, CLOCK_MONOTONIC, HRTIMER_MODE_ABS);
	piCore_g.ioTimer.function = piIoTimer;

	pr_info("piIO thread started\n");

	now = hrtimer_cb_get_time(&piCore_g.ioTimer);

	PiBridgeMaster_Reset();

	while (!kthread_should_stop()) {
		if (|>\tikzmarkin[set border color=martinired]{PiBridgeMaster}<|PiBridgeMaster_Run()|>\tikzmarkend{PiBridgeMaster}<| < 0)
			break;
	}

	RevPiDevice_finish();

	pr_info("piIO exit\n");
	return 0;
}
\end{lstlisting}

Der Kernel-Thread \lstinline{piIoThread} ist verantwortlich für den zyklischen Datenaustausch mit den IO-Modulen. In diesem wird fortlaufend die Methode \lstinline{PiBridgeMaster_Run()} aufgerufen, siehe Listing~\ref{lst:4-piIoThread}.

\begin{lstlisting}[language={c},firstnumber=262,caption={Auszug der Methode \lstinline{PiBridgeMaster_Run(void)} in \lstinline{RevPiDevice.c}\label{lst:4-PiBridgeMaster_Run}}]
int PiBridgeMaster_Run(void)
{
	static kbUT_Timer tTimeoutTimer_s;
	static kbUT_Timer tConfigTimeoutTimer_s;
	static int error_cnt;
	static INT8U last_led;
	static unsigned long last_update;
	int ret = 0;
	int i;

	my_rt_mutex_lock(&piCore_g.lockBridgeState);
	if (piCore_g.eBridgeState != piBridgeStop) {
		switch (eRunStatus_s) { |>\setcounter{lstnumber}{514}<|
		    case enPiBridgeMasterStatus_EndOfConfig:|>\setcounter{lstnumber}{621}<|
		    if (|>\tikzmarkin[set border color=martinired]{RevPiDevice}<|RevPiDevice_run()|>\tikzmarkend{RevPiDevice}<|) {
				// an error occured, check error limits |>\setcounter{lstnumber}{641}<|
			} else {
				ret = 1;
			}
			piCore_g.image.drv.i16uRS485ErrorCnt = RevPiDevice_getErrCnt();
			break;
\end{lstlisting}

Die in Listing~\ref{lst:4-PiBridgeMaster_Run} dargestellte Methode ist eine sog. State-Machine. Ist die Konfiguration der IO-Module erfolgreich abgeschlossen, so führt sie bei Aufruf lediglich die Methode \lstinline{RevPiDevice_run()} aus.

\begin{lstlisting}[language={c},firstnumber=140,caption={Auszug der Methode \lstinline{RevPiDevice_run(void)} in \lstinline{RevPiDevice.c}\label{lst:4-RevPiDevice_run}}]
int RevPiDevice_run(void)
{
	INT8U i8uDevice = 0;
	INT32U r;
	int retval = 0;

	RevPiDevices_s.i16uErrorCnt = 0;

	for (i8uDevice = 0; i8uDevice < RevPiDevice_getDevCnt(); i8uDevice++) {
		if (RevPiDevice_getDev(i8uDevice)->i8uActive) {
			switch (RevPiDevice_getDev(i8uDevice)->sId.i16uModulType) {
			case KUNBUS_FW_DESCR_TYP_PI_DIO_14:
			case KUNBUS_FW_DESCR_TYP_PI_DI_16:
			case KUNBUS_FW_DESCR_TYP_PI_DO_16:
				r = |>\tikzmarkin[set border color=martinired]{sendCyclicTelegram}<|piDIOComm_sendCyclicTelegram(i8uDevice)|>\tikzmarkend{sendCyclicTelegram}\setcounter{lstnumber}{166} <|;

				break; |>\setcounter{lstnumber}{216}<|
			}
		}
	} |>\setcounter{lstnumber}{227}<|
	return retval;
}
\end{lstlisting}

Diese iteriert wie in Listing~\ref{lst:4-RevPiDevice_run} abgebildete durch alle gegenwärtig in der SPS konfigurierten Module. Ist das aktuelle Modul als aktiv markiert, so wird anhand eines sog. Firmware-Descriptors entschieden, welche Methode für die Ansteuerung des Moduls aufzurufen ist.

\begin{lstlisting}[language={c},firstnumber=161,caption={Auszug der Methode \lstinline{piDIOComm_sendCyclicTelegram} in \lstinline{piDIOComm.c}\label{lst:4-sendCyclicTelegram}}]
INT32U piDIOComm_sendCyclicTelegram(INT8U i8uDevice_p)
{
	INT32U i32uRv_l = 0;
	SIOGeneric sRequest_l;
	SIOGeneric sResponse_l;
	INT8U len_l, data_out[18], i, p, data_in[70];
	INT8U i8uAddress;
	int ret; |>\setcounter{lstnumber}{239}<|
	
    |>\tikzmarkin[set border color=martinired]{piIoComm}<|ret = piIoComm_send((INT8U *) & sRequest_l, IOPROTOCOL_HEADER_LENGTH + len_l + 1);  |>\tikzmarkend{piIoComm}\setcounter{lstnumber}{298}<|
}
\end{lstlisting}

Im Falle des hier verwendeten DO-Moduls wird die in Listing~\ref{lst:4-sendCyclicTelegram} abgebildete Methode \lstinline{piDIOComm_sendCyclicTelegram()} aufgerufen. Dieser wird ein Zeiger auf das zu schreibende Gerät übergeben. 
Zunächst wird das Prozessabbild mittels eines proprietären, jedoch im Quellcode offen nachvollziehbaren Protokolls in ein \lstinline{sRequest_l} genanntes Byte-Array umgewandelt. Dieser Schritt ist in Listing~\ref{lst:4-sendCyclicTelegram} nicht abgebildet. Anschließend wird \lstinline{piIoComm_send()} ein Zeiger auf die so generierte Schreib-Anfrage übergeben.

\begin{lstlisting}[language={c},firstnumber=220,caption={Auszug der Methode \lstinline{piIOComm_send} in \lstinline{piIOComm.c}\label{lst:4-piIOComm_send}}]
int piIoComm_send(INT8U * buf_p, INT16U i16uLen_p)
{
	ssize_t write_l = 0;
	INT16U i16uSent_l = 0;|>\setcounter{lstnumber}{249}<|

	while (i16uSent_l < i16uLen_p) {
		write_l = vfs_write(piIoComm_fd_m, buf_p + i16uSent_l, i16uLen_p - i16uSent_l, &piIoComm_fd_m->f_pos);
		if (write_l < 0) {
			pr_info_serial("write error %d\n", (int)write_l);
			return -1;
		} 
		i16uSent_l += write_l;|>\setcounter{lstnumber}{263}<|
	}
	clear();
	vfs_fsync(piIoComm_fd_m, 1);
	return 0;
}
\end{lstlisting}

Listing~\ref{lst:4-piIOComm_send} zeigt die Implementierung von \lstinline{piIoComm_send()}. Diese Methode ist für das Schreiben der oben generierten Anfrage auf die seriellen Schnittstelle verantwortlich. Realisiert wird dies mittels der Methode \lstinline{vfs_write()}. Diese ist in \lstinline{<linux/fs.h>} definiert. Sie ermöglicht das Schreiben einer Datei im Userspace aus dem Kernel heraus. Geschrieben wird hier die Datei mit dem Deskriptor \lstinline{piIoComm_fd_m}.
Da die Funktion \lstinline{vfs_write()} durch andere Kernel-Tasks unterbrochen werden kann, ist nicht gewährleistet, dass die gesamte Anfrage mit nur einem Aufruf geschrieben wird. Die oben abgebildete while-Schleife stellt das vollständige Senden der Anfrage sicher.

\begin{lstlisting}[language={c},firstnumber=157,caption={Auszug der Methode \lstinline{piIOComm_open_serial} in \lstinline{piIOComm.c}\label{lst:4-piIOComm_open_serial}}]
int piIoComm_open_serial(void)
{   |>\setcounter{lstnumber}{167}<|
	struct file *fd;	/* Filedeskriptor */
	struct termios newtio;	/* Schnittstellenoptionen */

	|>\tikzmarkin[set border color=martiniblue]{fd}<|/* Port oeffnen - read/write, kein "controlling tty", 
	    Status von DCD ignorieren */
	fd = filp_open(|>\tikzmarkin[set border color=martinired]{tty}<|REV_PI_TTY_DEVICE|>\tikzmarkend{tty}<|, O_RDWR | O_NOCTTY, 0); |>\setcounter{lstnumber}{208}<|
	
	piIoComm_fd_m = fd;                                                      |>\tikzmarkend{fd}\setcounter{lstnumber}{217}<|

	return 0;
}
\end{lstlisting}

Der zum Schreiben auf die serielle Schnittstelle verwendete Datei-Deskriptor wird von der in Listing~\ref{lst:4-piIOComm_open_serial} abgebildeten Methode \lstinline{piIoComm_open_serial()} generiert. 

\begin{lstlisting}[language={c},firstnumber=45,caption={Definition der seriellen Schnittstelle in \lstinline{piIOComm.h}\label{lst:4-REV_PI_TTY_DEVICE}}]
#define REV_PI_TTY_DEVICE	"/dev/ttyAMA0"
\end{lstlisting}

Das in Listing~\ref{lst:4-REV_PI_TTY_DEVICE} definierte Macro verweist auf eine der seriellen Schnittstellen des RaspberryPi.
Die Implementierung des zugehörigen Schnittstellentreibers soll hier nicht weiter untersucht werden. Somit ist an dieser Stelle die Kette vom Setzen einer Variablen auf dem OPC-Server bis hin zur Aktualisierung des Prozessabbilds der IO-Module geschlossen.

% \begin{lstlisting}[language={c},firstnumber={226},caption={Setzen der Scheduler-Priorität auf SCHED\_FIFO in 
% revpi\_common.c\label{lst:2-sched_priority}}]
% param.sched_priority = ktprio->prio;
% ret = sched_setscheduler(child, SCHED_FIFO, &param);
% \end{lstlisting}
% % % Imports nur für Referenzenauflösung während des Schreibens! Vorm Kompilieren auskommentieren!
% \bibliography{0_hauptdatei}
% % Mit \section{...} eröffnen wir einen neuen Abschnitt.
% Der Befehl setzt nicht nur den Text in einer größeren,
% fetten Schrift, sondern sorgt außerdem dafür, daß er im
% Inhaltsverzeichnis erscheint.
%
% Mit \label{...} erzeugen wir einen Bezeichner, mit dessen Hilfe
% wir später auf die Nummer des Abschnitts verweisen können (nämlich
% mit~\ref{...}).
%
% Das Kommentarzeichen hinter „Übersicht“ dient dazu, ein
% Leerzeichen zwischen „Übersicht“ und dem \label-Befehl
% zu vermeiden, das andernfalls sichtbar würde – z.B. im
% Inhaltsverzeichnis.
%

% % Imports nur für Referenzenauflösung während des Schreibens! Vorm Kompilieren auskommentieren!
% \bibliography{0_hauptdatei}
% \input{1_einleitung}
%\input{2_grundlagen}
%\input{3_konzeption}
%\input{4_implementierung}
%\input{5_tests}
%\input{6_zusammenfassung}
% % Ende Imports

\section{Einleitung und Motivation%
  \label{sec:1-einleitung}}
Ziel dieses Projektes ist die Integration eines OPC-Servers mit einer auf Linux
basierenden speicherprogrammierbaren Steuerung (SPS). Angeschlossen an diese SPS
ist jeweils ein digitales Ein-/\,bzw.~Ausgabemodul. Die von diesen bereitgestellten
Ein-/\, bzw.~Ausgänge (IO) sollen in der Datenstruktur des OPC-Servers abgebildet
und über diesen für OPC-Clients les-/\,und schreibar sein. Weiterhin sollen einige
Funktionen zur Überwachung und Steuerung der an die SPS angeschlossenen Aktoren
und Sensoren direkt im OPC-Server implementiert werden.
Hiermit stellt dieses Projekt eine der Grundlagen für ein übergeordnetes Projekt,
die cloudbasierte Steuerung eines miniaturisierten Produktions-Systems, dar.

Der hier verwendete OPC-Server ist Teil des sog. open62541 Projekts. Er ist in C
geschrieben und implementiert bereits einen großen Teil der im OPC-UA-Standard
spezifizierten Funktionen.
Als SPS findet ein Revolution Pi 3 der Firma Kunbus Verwendung. Dieser integriert
ein sog. Compute Module der Raspberry Pi Foundation in ein industrietaugliches
Gehäuse und erlaubt die Erweiterung mittels IO- oder Gateway-Modulen. Über diese
erfolgt die Kommunikation mit weiteren Komponenten der Automatisierungstechnik.

Motiviert ist dieses Projekt durch die Beobachtung, dass die Verbreitung offener
Standards sowie freier Software auch in der Automatisierungstechnik zunimmt.
Linux ist ein freies Betriebssystem, OPC-UA ein offen zugänglicher, aktiv gepflegter
und weit verbreiteter Standard. Der Raspberry Pi findet sowohl bei Hobby-Anwendern als
auch in den Bereichen Forschung und Entwicklung sowie bei industriellen Anwendern
Verwendung. Dieses Projekt stellt somit eine für unterschiedliche Anwender interessante
Entwicklung dar.

Im Anschluss an diese einleitende Übersicht im Abschnitt~\ref{sec:1-einleitung} folgt
die Darstellung der wichtigsten Grundlagen in Abschnitt~\ref{sec:2-grundlagen}.
Aufbauend auf diesen Grundlagen folgt die konzeptuelle Ausarbeitung im Abschnitt~\ref{sec:3-konzeption}.
Die Umsetzung wird im Abschnitt~\ref{sec:4-implementierung} erläutert.
Die Leistungsfähigkeit der Implementierung wird in Abschnitt~\ref{sec:5-tests} untersucht.
Eine Zusammenfassung und ein Ausblick schließen die Arbeit in
Abschnitt~\ref{sec:6-fazit} ab. Eventuell noch benötigte Anhänge
finden sich in den Anhängen [...] bis [...].

% % % Imports nur für Referenzenauflösung während des Schreibens! Vorm Kompilieren auskommentieren!
% \bibliography{0_hauptdatei}
% \input{1_einleitung}
% \input{2_grundlagen}
% \input{3_konzeption}
% \input{4_implementierung}
% \input{5_tests}
% \input{6_zusammenfassung}
% % Ende Imports

\section{Grundlagen%
  \label{sec:2-grundlagen}}

\subsection{Speicherprogrammierbare-Steuerung und Linux -- Revolution Pi%
     \label{sec:2-sps}}

\subsubsection{Kunbus RevolutionPi%
        \label{sec:2-revpi}}
Der RevolutionPi 3 ist eine speicherprogrammierbare Steuerung (SPS) des Herstellers
Kunbus GmbH. Kern dieser SPS ist das von der Raspberry Pi Foundation entwickelte
und vertriebene Raspberry Pi Compute Module 3. Dieses integriert ein Broadcom BCM2837
System-on-Chip (SoC) mit vier 1,2GHz Prozessorkernen, 1GB RAM, 4GB eMMC Anwendungsspeicher
und sonstige Peripherie in ein Modul im DDR2-SODIMM Formfaktor. Diese Spezifikationen
sind weitgehend identisch zu denen des ausgesprochen populären Raspberry Pi 3.
Der Revolution Pi profitiert daher von dem gleichen großen Angebot an Software
und Unterstützung wie der Raspberry Pi, ergänzt dessen Hardware jedoch um eine 24V
Spannungsversorgung, die Möglichkeit der Erweiterung durch mehrere industrietaugliche
Ein-/ Ausgabemodule und Gateways sowie ein Gehäuse zur Montage auf einer DIN-Schiene.
\begin{itemize}
  \item{Prozessor: BCM2837}
  \item{Taktfrequenz 1,2 GHz}
  \item{Anzahl Prozessorkerne: 4}
  \item{Arbeitsspeicher: 1 GByte}
  \item{eMMC Flash Speicher: 4 GByte}
  \item{Betriebssystem: Angepasstes Raspbian mit RT-Patch}
  \item{RTC mit 24h Pufferung über wartungsfreien Kondensator}
  \item{Treiber / API: Treiber schreibt zyklisch Prozessdaten in ein Prozessabbild, Zugriff auf Prozessabbild über Linux-Filesystem als API zu Fremdsoftware.}
  \item{Kommunikationsanschlüsse: 2 x USB 2.0 A (je 500 mA belastbar), 1 x Micro-USB, HDMI, Ethernet (RJ45) 10/100 Mbit/s}
  \item{Stromversorgung: min. 10,7 V, max. 28,8 V, maximal 10 Watt}
  \item{Zulässige Umgebungstemperatur: -40 bis +55 C}
  \item{Gehäuseabmessungen: (HxBxL) 96 mm x 22,5 mm x 110,5 mm (ohne gesteckte Stecker)}
  \item{ESD Schutz: 4 kV / 8 kV gemäß EN61131-2 und IEC 61000-6-2}
  \item{Surge / Burst Prüfungen: gemäß EN61131-2 und IEC 61000-6-2 eingekoppelt auf Versorgungsspannung, Ethernet und IO-Leitungen}
  \item{EMI Prüfungen: gemäß EN61131-2 und IEC 61000-6-2}
\end{itemize}

Kunbus bietet eine Auswahl an IO- und Gateway-Modulen zur Erweiterung des Revolution Pi an.
Gateways dienen der Kommunikation mit Systemen oder Komponenten der Automatisierungstechnik
über Protokolle wie PROFIBUS oder EtherCAT. IO-Module erlauben die Überwachung
und Steuerung von digitalen oder analogen Ein- und Ausgängen.

\subsubsection{Zugriff auf IO-Module%
        \label{sec:2-io}}
Der Zugriff auf die Ein- und Ausgänge der IO-Module erfolgt über ein Prozessabbild
und einen hierfür von Kunbus bereitgestellten Treiber, genannt piControl. Dieser
aktualisiert das Prozessabbild zyklisch. Die angestrebte Zykluszeit beträgt 5ms,
kann jedoch je nach Anzahl der angeschlossenen Module auch größer sein. Kunbus
garantiert bei drei IO-Modulen und zwei Gateway-Modulen eine Zykluszeit von 10 ms.
Jedes der IO-Module stellt ein eigenständiges eingebettetes System dar. Es verfügt
über einen Microcontroller, welcher die IOs bereitstellt und über einen RS485-Bus
mit dem Revolution Pi kommuniziert.
% https://revolution.kunbus.de/io-modul/

Lizenz: GPL
% https://github.com/RevolutionPi/piControl

\begin{lstlisting}[language={c},firstnumber={226},caption={Setzen der Scheduler-Priorität auf SCHED\_FIFO in revpi\_common.c\label{lst:2-sched_priority}}]
param.sched_priority = ktprio->prio;
ret = sched_setscheduler(child, SCHED_FIFO,
       &param);
\end{lstlisting}


\subsection{Echtzeit und Multithreading unter Linux -- preemptRT und posix%
     \label{sec:2-echtzeit}}


 Der Linux-Kernel verfügt über mehrere unterschiedliche Preemtion-Modelle:

\begin{itemize}
  \item No Forced Preemption (server):
  Ausgelegt auf maximal möglichen Durchsatz, lediglich Interrupts und
  System-Call-Returns bewirken Präemption.

  \item Voluntary Kernel Preemption (Desktop):
  Neben den implizit bevorrechtigten Interrupts und System-Call-Returns gibt es
  in diesem Modell weitere Abschnitte des Kernels in welchen Preämption explizit
  gestattet ist.

  \item Preemptible Kernel (Low-Latency Desktop):
  In diesem Modell ist der gesamte Kernel, mit Ausnahme sog.~kritischer Abschnitte
  präemptible. Nach jedem kritischen Abschnitt gibt es einen impliziten Präemptions-Punkt.

  \item Preemptible Kernel (Basic RT):
  Dieses Modell ist dem zuvor genannten sehr ähnlich, hier sind jedoch alle Interrupt-Handler
  als eigenständige Threads ausgeführt.

  \item Fully Preemptible Kernel (RT):
  Wie auch bei den beiden zuvor genannten Modellen ist hier der gesamte Kernel
  präemtible, die Anzahl und Dauer der nicht-präemtiblen kritischen Abschnitte
  ist auf ein notwendiges Minimum beschränkt. Alle Interrupt-Handler sind als
  eigenständige Threads ausgeführt, Spinlocks durch Sleeping-Spinlocks und Mutexe
  durch sog.~RT-Mutexe ersetzt.

\end{itemize}
\todo{Spinlocks und Mutexe sowie die RT-Varianten dieser erklären!}

Lediglich mit dem vollständig präemtiblen Kernel kann Echtzeit-Verhalten realisiert werden.

% https://wiki.linuxfoundation.org/realtime/documentation/technical_basics/preemption_models bzw kernel/Kconfig.preempt

\subsubsection{preemptRT%
        \label{sec:2-preemptRT}}
% https://wiki.linuxfoundation.org/realtime/documentation/technical_details/start
% https://wiki.linuxfoundation.org/realtime/documentation/technical_basics/start

Das dem PREEMPT RT Kernel zugrunde liegende Prinzip lässt sich in einer einfachen
Regel ausdrücken: Nur Code, welcher absolut nicht-präemtible sein darf, ist es
gestattet nicht-präemtible zu sein.
Das erklärte Ziel des PREEMPT\_RT Patches ist es folglich, die Menge des nicht-präemtiblen
Codes im Linux-Kernel auf das absolut notwendige Minimum zu reduzieren.

Dies wird durch Verwendung folgender Mechanismen erreicht:

\begin{itemize}
  \item Hochauflösende Timer
  \item Sleeping Spinlocks
  \item Threaded Interrupt Handlers
  \item rt\_mutex
  \item RCU
\end{itemize}


\subsubsection{posix%
        \label{sec:2-posix}}
Ist posix hier wirklich relevant? Debian bzw.~Raspbian sind weitgehend posix
kompatibel, aber wird es hier genutzt? -> JA, open62541 nutzt pthread.h
piControl nutzt kthread.h, und semaphore.h

\subsection{OPC-UA und open62541%
     \label{sec:2-opc}}

\subsubsection{OPC UA%
        \label{sec:2-opcua}}
Open Platform Communications (OPC) ist eine Familie von Standards zur herstellerunabhängigen
Kommunikation von Maschinen (M2M) in der Automatisierungstechnik. Die sog.~OPC Task Force, zu deren
Mitgliedern verschiedene große Firmen der Automatisierungsindustrie gehören, veröffentlichte
die OPC Specification Version 1.0 im August 1996.
Motiviert ist dieser offene Standard durch die Erkenntniss, dass die Anpassung der
zahlreichen Herstellerstandards an individuelle Infrastrukturen und Anlagen einen
großen Mehraufwand verursachen.
Die Wikipedia beschreibt das Anwendungsgebiet für OPC wie folgt:

\glqq{}OPC wird dort eingesetzt, wo Sensoren, Regler und Steuerungen verschiedener Hersteller
ein gemeinsames Netzwerk bilden. Ohne OPC benötigten zwei Geräte zum Datenaustausch
genaue Kenntnis über die Kommunikationsmöglichkeiten des Gegenübers. Erweiterungen
und Austausch gestalten sich entsprechend schwierig. Mit OPC genügt es, für jedes
Gerät genau einmal einen OPC-konformen Treiber zu schreiben. Idealerweise wird
dieser bereits vom Hersteller zur Verfügung gestellt. Ein OPC-Treiber lässt sich
ohne großen Anpassungsaufwand in beliebig große Steuer- und Überwachungssysteme
integrieren.

OPC unterteilt sich in verschiedene Unterstandards, die für den jeweiligen Anwendungsfall
unabhängig voneinander implementiert werden können. OPC lässt sich damit verwenden
für Echtzeitdaten (Überwachung), Datenarchivierung, Alarm-Meldungen und neuerdings
auch direkt zur Steuerung (Befehlsübermittlung).\grqq{}

OPC basiert in der ursprünglichen Spezifikation auf Microsofts DCOM-Spezifikation.
DCOM macht Funktionen und Objekte einer Anwendung anderen Anwendungen im Netzwerk
zugänglich. Der OPC-Standard definiert entsprechende DCOM-Objekte um mit anderen
OPC-Anwendungen Daten austauschen zu können. Die Verwendung von DCOM bindet Anwender
an Betriebssysteme von Microsoft. Die ursprüngliche OPC Spezifikation wird durch die
Entwicklung von OPC Unified Architecture (OPC UA) abgelöst.
OPC UA setzt auf einem eigenen Kommunikationionsstack auf, die Verwendung von DCOM
und damit die Bindung an Microsoft wurden aufgelöst.

Die OPC-UA-Architektur ist eine Service-orientierte Architektur (SOA), deren Struktur
aus mehreren Schichten besteht.

% Wikipedia
Das OPC-Informationsmodell ist nicht mehr nur eine Hierarchie aus Ordnern, Items
und Properties. Es ist ein sogenanntes Full-Mesh-Network aus Nodes, mit dem neben
den Nutzdaten eines Nodes auch Meta- und Diagnoseinformationen repräsentiert werden.
Ein Node ähnelt einem Objekt aus der objektorientierten Programmierung. Ein Node
kann Attribute besitzen, die gelesen werden können (Data Access (DA), Historical
Data Access (HDA)). Es ist möglich Methoden zu definieren und aufzurufen.
Eine Methode besitzt Aufrufargumente und Rückgabewerte. Sie wird durch ein Command
aufgerufen. Weiterhin werden Events unterstützt, die versendet werden können
(AE (Alarms \& Events), DA DataChange), um bestimmte Informationen zwischen Geräten
auszutauschen. Ein Event besitzt unter anderem einen Empfangszeitpunkt, eine Nachricht
und einen Schweregrad. Die o. g. Nodes werden sowohl für die Nutzdaten als auch
alle anderen Arten von Metadaten verwendet. Der damit modellierte OPC-Adressraum
beinhaltet nun auch ein Typmodell, mit dem sämtliche Datentypen spezifiziert werden.

% https://de.wikipedia.org/wiki/Open_Platform_Communications
% https://de.wikipedia.org/wiki/OPC_Unified_Architecture
% https://opcfoundation.org/developer-tools/specifications-unified-architecture
% Von Gerhard Gappmeier - ascolab GmbH, CC BY-SA 3.0, https://de.wikipedia.org/w/index.php?curid=1892069
\subsubsection{open62541%
        \label{sec:2-open62541}}
open62541 ist eine offene und freie Implementierung von OPC UA. Die in C geschriebene
Bibliothek stellt eine beständig zunehmende Anzahl der im OPC UA Standard definierten
Funktionen bereit. Sie kann sowohl zur Erstellung von OPC-Servern als auch -Clients
genutzt werden. Ergänzend zu der unter der Mozilla Public License v2.0 lizensierten
Bibliothek stellt das open62541 Projekt auch Beispielprogramme unter einer CC0 Lizenz
zur Verfügung.

Die Bibliothek eignet sich auch für die Entwicklung auf eingebetteten Systemen und
Microcontrollern. Je nach Umfang der gewünschten Funktionen und des OPC Informationsmodells
beträgt die Größe einer Server-Binary weniger als 100kb. %evtl. kürzen?

\todo{Nodes erklären! Evtl.~oben!}

Folgende Auswahl an Eigenschaften und Funktionen zeichnet die in dieser Arbeit verwendete
Version 0.3 von open62541 aus:
\begin{itemize}
  \item Kommunikationionsstack
  \begin{itemize}
      \item OPC UA Binär-Protokoll (HTTP oder SOAP werden gegenwärtig nicht unterstützt)
      \item Austauschbare Netzwerk-Schicht, welche die Verwendung eigener Netzwerk-APIs
      erlaubt.
      \item Verschlüsselte Kommunikationion
      \item Asynchrone Dienst-Anfragen im Client
  \end{itemize}
  \item Informationsmodell
  \begin{itemize}
    \item Unterstützung aller OPC UA Node-Typen, inkl.~Methoden
    \item Hinzufügen und Entfernen von Nodes und Referenzen zur Laufzeit.
    \item Vererbung und Instanziierung von Objekt- und Variablentypen
    \item Zugriffskontrolle auch für einzelne Nodes
  \end{itemize}
  \item Subscriptions
  \begin{itemize}
    \item Erlaubt die Überwachung (subscriptions / monitoreditems)
    \item Sehr geringer Ressourcenbedarf pro überwachtem Wert
  \end{itemize}
  \item Code-Generierung auf XML-Basis
  \begin{itemize}
    \item Erlaubt die Erstellung von Datentypen
    \item Erlaubt die Generierung des serverseitigen Informationsmodells
  \end{itemize}
\end{itemize}

% https://open62541.org/doc/0.3/


Mozilla Public License
CC0 Lizenz für Beispiele und Plugins

% https://open62541.org/doc/open62541-current.pdf
% https://open62541.org/

% % % Imports nur für Referenzenauflösung während des Schreibens! Vorm Kompilieren auskommentieren!
% \bibliography{0_hauptdatei}
% \input{1_einleitung}
% \input{2_grundlagen}
% \input{3_konzeption}
% \input{4_implementierung}
% \input{5_tests}
% \input{6_zusammenfassung}
% \input{anhang}
% % Ende Imports

\section{Systemkonzept%
  \label{sec:3-konzeption}}
Auf Basis der in Abschnitt \ref{sec:2-grundlagen} vorgestellten Möglichkeiten folgt nun die Ausarbeitung eines Konzepts.
In den folgenden Abschnitten soll näher auf zwei zentrale Aspekte eingegangen werden: Abschnitt~\ref{sec:3-anbindung} stellt Möglichkeiten zum Zugriff auf Variablen bzw.\,Werte im Prozessabbild des Revolution Pi vor; in Abschnitt~\ref{sec:3-integration} wird ein Konzept zur Bereitstellung dieser Variablen auf einem OPC-Server vorgestellt.

\subsection{Anbindung der IO an den OPC-Server%
     \label{sec:3-anbindung}}

Eine Webanwendung mit Bezeichnung PiCtory dient zur Konfiguration der I/O- und virtuellen Module des RevolutionPi. Die Konfiguration liegt im JSON-Format in der Datei \lstinline{/etc/revpi/config.rsc}. Der piControl-Treiber liest diese Datei beim Start. 
Der folgende Auszug aus der Manpage des piControl-Kernelmoduls beschreibt die von diesem zum Lesen und Schreiben einzelner Bits des Prozessabbildes bereitgestellten Funktionen~\citep[vgl.]{web-revpi-manpage}. Sie ist an dieser Stelle weitgehend ungekürzt zitiert, da sie die nutzbare Schnittstelle sehr kompakt beschreibt.

\begin{lstlisting}[breakindent=0pt, numbers=none, caption={Auszug aus der Revolution Pi Programmers Manual\label{lst:4-manpage}}]
KB_FIND_VARIABLE SPIVariable *argp
Find a variable in the process image by its name. A pointer to a structure of type SPIVariable must be passed as argument. [...]
The struct SPIVariable [...] is defined as 
typedef struct SPIVariableStr
{
    char strVarName[32]; // Variable name
    uint16_t i16uAddress; // Address of the byte in the process image
    uint8_t i8uBit; // 0-7 bit position, >= 8 whole byte
    uint16_t i16uLength; // length of the variable in bits.
    // Possible values are 1, 8, 16 and 32
} SPIVariable;

Set and get values of the process image
KB_GET_VALUE SPIValue *argp
[...]
KB_SET_VALUE SPIValue *argp
Write one bit or one byte to the process image [...].  This call is more efficient than the usual calls of seek and write because only one function call is necessary. If more than on application are writing bits in one output byte, this call is the only safe way to set a bit without overwriting the other bits because this call is doing a read-modify-write-cycle. 

The struct SPIValue used by this ioctl is defined as
typedef struct SPIValueStr
{
    uint16_t i16uAddress; // Address of the byte in the process image
    uint8_t i8uBit; // 0-7 bit position, >= 8 whole byte
    uint8_t i8uValue; // Value: 0/1 for bit access, whole byte otherwise
} SPIValue;
\end{lstlisting} 

Die oben beschriebenden Funtkionen \lstinline{KB_FIND_VARIABLE}, \lstinline{KB_GET_VALUE} und \lstinline{KB_SET_VALUE} ermöglichen einen einfachen und (lt.\,Manpage) effizienten Zugriff auf einzelne Bits des Prozessabbildes und damit auch auf die IO des RevolutionPi.
Der Zugriff des OPC-Servers auf das Prozessabbild soll daher mittels dieser Funktionen realisiert werden.
\lstinline{KB_FIND_VARIABLE} kann genutzt werden, um Adressen von Variablen im Prozessabbild mittels ihres Namens aufzulösen.
\lstinline{KB_GET_VALUE} und \lstinline{KB_SET_VALUE} ermöglichen den Zugriff auf die Werte dieser Variablen.


\subsection{Integration des OPC-Servers in das System%
     \label{sec:3-integration}}

open62541 bietet drei Möglichkeiten zum Abgleich von Variablen mit dem Prozessabbild~\citep[vgl.][Tutorials - Connecting a Variable with a Physical Process]{web-open62541}:
\begin{itemize}
    \item Manuelles oder zyklisches Aktualisieren
    \item Variable Value Callback
    \item Variable Datasource
\end{itemize}

Die zyklische Aktualisierung eines oder mehrerer Werte nimmt, abhängig von der Zykluszeit, viele Systemressourcen in Anspruch. Value Callbacks ermöglichen es, einen Variablenwert effizienter mit einer Ressource wie etwa einem Prozessabbild zu synchronisieren. An die Variable wird ein Callback angehängt, welches vor jedem Lesen und nach jedem Schreibvorgang ausgeführt wird.
Der Wert der Variablen wird weiterhin im Variablenknoten auf dem OPC-Server gespeichert, der Abgleich mit der verknüpften Ressource erfolgt durch die Callback-Methoden.

Sogenannte Datenquellen gehen noch einen Schritt weiter. Der Server leitet jede Lese- und Schreibanforderung direkt an eine Callback-Funktion weiter. Beim Lesen liefert der Rückruf eine Kopie des aktuellen Wertes. Die Datenquelle muss intern ein eigenes Speichermanagement implementieren.

Der Zugriff auf die Werte des Prozessabbildes erfolgt, wie in Abschnitt~\ref{sec:3-anbindung} beschrieben, über von piControl bereitgestellte Methoden. Um die durch open62541 gepflegte OPC-Datenstruktur und das durch piControl verwaltete Prozessabbild möglichst effektiv verknüpfen zu können, soll diese Interaktion mittels Datenquellen und den zugehörigen Callbacks implementiert werden.
% % % Imports nur für Referenzenauflösung während des Schreibens! Vorm Kompilieren auskommentieren!
% \bibliography{0_hauptdatei}
% \input{1_einleitung}
% \input{2_grundlagen}
% \input{3_konzeption}
% \input{4_implementierung}
% \input{5_tests}
% \input{6_zusammenfassung}
% \input{anhang}
% % Ende Imports

\section{Implementierung%
  \label{sec:4-implementierung}}
Das folgende Kapitel stellt in Auszügen die Implementierung des OPC-Servers sowie die Anbindung an die IO-Module
der SPS dar. Der Schwerpunkt liegt hierbei auf der Funktionsweise des piControl-Treibers und dessen Integration in das Projekt. Abschnitt~\ref{sec:4-picontrol} erklärt die zum Schreibens eines Bits verwendeten Funktionsaufrufe.
Zuvor soll jedoch in Abschnitt~\ref{sec:4-open62541} der Teil des OPC-Servers vorgestellt werden, welcher auf besagten Treiber zugreift. 

\subsection{Implementierung des OPC-Servers%
     \label{sec:4-open62541}}
Wie im vorangegangenen Abschnitt~\ref{sec:3-integration} begründet, soll die Verknüpfung zwischen dem Prozessabbild der SPS und den auf dem OPC-Server bereitgestellten Werten über sog.\,Datenquellen erfolgen. Hierzu ist zunächst eine Callback-Methode zu implementieren, welche bei einem Lese- oder Schreibzugriff auf eine Variable aufgerufen wird. Die Verknüpfung zwischen Callback-Methode und Variable muss manuell erfolgen.

\begin{lstlisting}[language={c},firstnumber=237,caption={Auszug der Methode \lstinline{linkDataSourceVariable} in \lstinline{variables.c}\label{lst:4-linkDataSourceVariable}}]
extern UA_StatusCode
 linkDataSourceVariable(UA_Server *server, UA_NodeId nodeId) {
     bool readonly = false;
     UA_DataSource dataSourceVariable;
     UA_StatusCode rc; |>\setcounter{lstnumber}{254}<|

     dataSourceVariable.read = readDataSourceVariable;
     if (!readonly)
        dataSourceVariable.write = writeDataSourceVariable;
     else
        dataSourceVariable.write = writeReadonlyDataSourceVariable;

     return UA_Server_setVariableNode_dataSource(server, nodeId, dataSourceVariable);
 }
\end{lstlisting}

\begin{figure}[h]
    \centering
    \includegraphics[width=0.42\textwidth]{doc/img/OPC_RevPiDO.pdf}
    \caption{Auszug des verwendeten Nodesets, hier Digitalausgang 1 des Versuchsaufbaus
      \label{fig:opc-do}}
\end{figure}

Die in Listing~\ref{lst:4-linkDataSourceVariable} abgebildete Methode \lstinline{linkDataSourceVariable()} erzeugt ein Struct vom Typ \lstinline{UA_DataSource}. In diesem werden dem Lesen und Schreiben einer OPC-Variablen entsprechende Callback-Methoden zugewiesen. Die Verknüpfung einer OPC-Variable, genauer ihrer NodeId, mit der zuvor definierten Datenquelle erfolgt über die von open62541 bereitgestellte Methode \lstinline{UA_Server_setVariableNode_dataSource()}. Vor dem Lesen und nach dem Schreiben dieser Variable werden von nun an die entsprechenden Callbacks aufgerufen.
     
\begin{lstlisting}[language={c},firstnumber=168,caption={Auszug des Callbacks \lstinline{writeDataSourceVariable} in \lstinline{variables.c}\label{lst:4-writeDataSourceVariable}}]  
extern UA_StatusCode
 writeDataSourceVariable(UA_Server *server,
            const UA_NodeId *sessionId, void *sessionContext,
            const UA_NodeId *nodeId, void *nodeContext,
            const UA_NumericRange *range, const UA_DataValue *dataValue) {

    UA_StatusCode retval  = UA_STATUSCODE_GOOD;
    UA_NodeId *nameNodeId = UA_malloc(sizeof(UA_NodeId));
    UA_QualifiedName nameQN = UA_QUALIFIEDNAME(1, "Name");
    UA_Variant nameVar;
    UA_Boolean bit;

    retval |= findSiblingByBrowsename(server, nodeId, &nameQN, nameNodeId);
    retval |= UA_Server_readValue(server, *nameNodeId, &nameVar);
    retval |= UA_Boolean_copy(dataValue->value.data, &bit);

    |>\tikzmarkin[set border color=martinired]{writeIO}<|PI_writeSingleIO(String_fromUA_String(nameVar.data), &bit, false);                                                 |>\tikzmarkend{writeIO}<|

    free(nameNodeId);
    return retval;
 }
\end{lstlisting}

Listing~\ref{lst:4-writeDataSourceVariable} zeigt die Callback-Methode, welche nach dem Schreiben einer Variablen auf dem OPC-Server aufgerufen wird.
Dieser Methode wird neben der NodeId der mit ihr verknüpften Variablen auch der Wert dieser in Form eines Zeigers auf ein Struct vom Typ \lstinline{UA_DataValue} übergeben.

Die Gestaltung des hier verwendeten Nodesets sieht vor, dass in einer OPC-Variablen \lstinline{"Name"} der Bezeichner des zu schreibenden Digitalausgangs hinterlegt ist, siehe Abbildung~\ref{fig:opc-do}. Dies erlaubt eine Rekonfiguration der Ein- und Ausgänge der SPS ohne Änderungen im Programmcode des OPC-Servers vornehmen zu müssen.
Es ist daher erforderlich, nach jedem Schreiben einer mit einem Digitalausgang verknüpften Variablen, hier \lstinline{"Value"}, dessen Bezeichner \lstinline{"Name"} abzufragen. 
Dies geschieht in den Zeilen 180 und 181.
Anschließend wird dieser Bezeichner sowie der zu schreibende Wert der Methode \lstinline{PI_writeSingleIO()} übergeben, welche wiederum die Interaktion mit piControl übernimmt (vgl. Abschnitt \ref{sec:4-picontrol}).
 
\subsection{Integration von piControl%
     \label{sec:4-picontrol}}
In Abschnitt~\ref{sec:2-io} wurde die Anbindung der IO-Module des Revolution Pi sowie die Funktionsweise von piControl aus Anwendersicht beschrieben. Die verfügbare Literatur beschränkt sich auch auf lediglich diese Sicht; eine weiterführende Dokumentation für Entwickler gibt es, neben der in Abschnitt~\ref{sec:3-anbindung} vorgestellten Manpage, nicht. 
In diesem Abschnitt soll daher der Quellcode von piControl sowie dessen Verwendung im Projekt genauer betrachtet werden.
Hierzu wird exemplarisch die in Abschnitt~\ref{sec:4-open62541} eingeführte Methode \lstinline{PI_writeSingleIO()} untersucht.
Diese Methode ermöglicht das Setzen eines einzelnen Bits im Prozessabbild der SPS, und damit das Schalten eines digitalen Ausgangs auf einem IO-Modul.
Die äquivalente Methode \lstinline{int piControlGetBitValue(SPIValue *pSpiValue)} zum Lesen eines Bits bzw. Eingangs funktioniert analog und soll daher an dieser Stelle nicht dediziert erörtert werden.

\begin{lstlisting}[language={c},firstnumber=97,
                   caption={Setzen eines phsikalischen, digitalen Ausgangs in \lstinline{revpi.c}
                   \label{lst:4-PI_writeSingleIO}}]
extern void PI_writeSingleIO(char *pszVariableName, bool *bit, bool verbose)
{
	int rc;
	SPIVariable sPiVariable;
	SPIValue sPIValue;

	strncpy(sPiVariable.strVarName, pszVariableName, sizeof(sPiVariable.strVarName));
	rc = piControlGetVariableInfo(&sPiVariable);
	if (rc < 0) {
		printf("Cannot find variable '%s'\n", pszVariableName);
		return;
	}

		sPIValue.i16uAddress = sPiVariable.i16uAddress;
		sPIValue.i8uBit = sPiVariable.i8uBit;
		sPIValue.i8uValue = *bit;
		rc = |>\tikzmarkin[set border color=martinired]{setBitValue}<|piControlSetBitValue(&sPIValue)|>\tikzmarkend{setBitValue}<|;
		if (rc < 0)
			printf("Set bit error %s\n", getWriteError(rc));
		else if (verbose)
			printf("Set bit %d on byte at offset %d. Value %d\n", sPIValue.i8uBit, sPIValue.i16uAddress,
			       sPIValue.i8uValue);
}
\end{lstlisting}

Der Programmcode in Listing~\ref{lst:4-PI_writeSingleIO} ist Teil des implementierten OPC-Servers. In diesem wird auf zwei Funktionen des piControl-Treibers zugegriffen. 
Beiden Methoden wird als Argument ein Zeiger auf ein Struct vom Typ \lstinline{SPIValue} übergeben. Der im Struct abgelegte Name wird mittels \lstinline{piControlGetVariableInfo(&sPIValue)} zu einer Adresse im Prozessabbild aufgelöst. Diese wird in \lstinline{sPIValue.i16uAdress} gespeichert. Der Wert der Variablen wird anschließend mittels \lstinline{piControlSetBitValue(&sPIValue)} an dieser Adresse in das Prozessabbild geschrieben.

\begin{lstlisting}[language={c},firstnumber=309,caption={Methode \lstinline{piControlSetBitValue} in \lstinline{piControlIf.c}\label{lst:4-piControlSetBitValue}}]
int |>\tikzmarkin[set border color=martiniblue]{setBitValueFcn}<|piControlSetBitValue(SPIValue *pSpiValue)|>\tikzmarkend{setBitValueFcn}<|
{
    piControlOpen();

    if (PiControlHandle_g < 0)
	    return -ENODEV;

    pSpiValue->i16uAddress += pSpiValue->i8uBit / 8;
    pSpiValue->i8uBit %= 8;

    if (|>\tikzmarkin[set border color=martinired]{ioctl}<|ioctl(PiControlHandle_g, KB_SET_VALUE, pSpiValue)|>\tikzmarkend{ioctl}<| < 0)
	    return errno;

    return 0;
}
\end{lstlisting}

Die in Listing~\ref{lst:4-piControlSetBitValue} dargestellte Methode \lstinline{piControlSetBitValue} ist lediglich eine Hüllfunktion (häufig auch als Wrapper-Funktion bezeichnet) für einen Aufruf des \lstinline{ioctl} Kernel-Moduls.
Folgende Parameter werden übergeben:
\lstinline{PiControlHandle_g} ist die Referenz auf die Geräte-Datei des piControl-Treibers. \lstinline{KB_SET_VALUE} ist das ioctl-Kommando zum Schreiben eines Bits in das Prozessabbild. Der Zeiger \lstinline{pSpiValue} verweist auf ein Struct des bereits vorgestellten Typs \lstinline{SPIValue}.

\begin{lstlisting}[language={c},firstnumber=80,caption={Methode \lstinline{piControlOpen} in \lstinline{piControlIf.c}\label{lst:4-piControlOpen}}]
void piControlOpen(void)
{
    /* open handle if needed */
    if (PiControlHandle_g < 0)
    {
	    |>\tikzmarkin[set border color=martiniblue]{PiControlHandle}<|PiControlHandle_g = open(PICONTROL_DEVICE, O_RDWR)|>\tikzmarkend{PiControlHandle}<|;
    }
}
\end{lstlisting}

Die in Listing~\ref{lst:4-piControlOpen} dargestellte Methode öffnet, sofern nicht bereits geschehen, die Geräte-Datei. Das Macro \lstinline{PICONTROL_DEVICE} verweist hierbei auf \lstinline{/dev/piControl0}.

\begin{lstlisting}[language={c},firstnumber=721,caption={Methode \lstinline{piControlIoctl} in \lstinline{piControlMain.c}\label{lst:4-piControlIoctl}}]
static long |>\tikzmarkin[set border color=martiniblue, below offset=0.9em]{piControlIoctl}<|piControlIoctl(struct file *file, unsigned int prg_nr, 
                           unsigned long usr_addr)                                      |>\tikzmarkend{piControlIoctl}<|
{
  int status = -EFAULT;
  tpiControlInst *priv;
  int timeout = 10000;	// ms

  if (prg_nr != KB_CONFIG_SEND && prg_nr != KB_CONFIG_START && !isRunning()) {
  	return -EAGAIN;
  }

  priv = (tpiControlInst *) file->private_data;

  if (prg_nr != KB_GET_LAST_MESSAGE) {
  	// clear old message
  	priv->pcErrorMessage[0] = 0;
  }

  switch (prg_nr) {|>\setcounter{lstnumber}{864}<|

    case |>\tikzmarkin[set border color=martiniblue]{KB_SET_VALUE}<|KB_SET_VALUE:|>\tikzmarkend{KB_SET_VALUE}<|
  		{
  			SPIValue *pValue = (SPIValue *) usr_addr;

  			if (!isRunning())
  				return -EFAULT;

  			if (pValue->i16uAddress >= KB_PI_LEN) {
  				status = -EFAULT;
  			} else {
  				INT8U i8uValue_l;
  				my_rt_mutex_lock(&piDev_g.lockPI);
  				i8uValue_l = piDev_g.ai8uPI[pValue->i16uAddress];

  				if (pValue->i8uBit >= 8) {
  					i8uValue_l = pValue->i8uValue;
  				} else {
  					if (pValue->i8uValue)
  						i8uValue_l |= (1 << pValue->i8uBit);
  					else
  						i8uValue_l &= ~(1 << pValue->i8uBit);
  				}

  				|>\tikzmarkin[set border color=martinired]{i8uValue}<|piDev_g.ai8uPI[pValue->i16uAddress] = i8uValue_l;|>\tikzmarkend{i8uValue}<|
  				rt_mutex_unlock(&piDev_g.lockPI);

  #ifdef VERBOSE
  				pr_info("piControlIoctl Addr=%u, bit=%u: %02x %02x\n", pValue->i16uAddress, pValue->i8uBit, pValue->i8uValue, i8uValue_l);
  #endif

  				status = 0;
  			}
  		}
  		break; |>\setcounter{lstnumber}{1314}<|

    default:
      pr_err("Invalid Ioctl");
      return (-EINVAL);
      break;

    }

    return status;
  }
\end{lstlisting}

Listing~\ref{lst:4-piControlIoctl} zeigt in Auszügen die ioctl-Methode des piControl Kernel-Treibers. Diese bekommt folgende Argumente übergeben: \lstinline{struct file *file} enthält den Verweis auf die Geräte-Datei, hier \lstinline{/dev/piControl0}. Der Wert von \lstinline{unsigned int prg_nr} beschreibt die Anfrage an den Treiber, in diesem Fall \lstinline{KB_SET_VALUE}. Das Argument \lstinline{unsigned long usr_addr} enthält einen typ-agnostischen Pointer. Dieser verweist auf einen Speicherbereich, in welchem die zur Bearbeitung der Anfrage notwendigen Daten abgelegt sind. Hier können auch vom Treiber empfangene Daten dem Anwendungsprogramm bereitgestellt werden. 

Die switch-case-Anweisung führt die über das Argument \lstinline{prg_nr} spezifizierte Aktion aus. Hier betrachten wir \lstinline{KB_SET_VALUE}:
Zunächst wird in Zeile 868 der übergebene Zeiger \lstinline{usr_addr} mittels explizitem Typecast zu einem Zeiger des Typs \lstinline{SPIValue *} konvertiert. Da dieser auf Daten im Userspace verweist, ist beim Zugriff durch den Kernel-Treiber besondere Vorsicht geboten.
In Zeile 877 wird mittels Mutex das Prozessabbild \lstinline{piDev_g} für den Zugriff durch andere Threads oder Prozesse gesperrt.
\lstinline{my_rt_mutex_lock} verweist hierbei auf die Funktion \lstinline{rt_mutex_lock} aus \lstinline{linux/sched.h}\footnote{Offenbar wurde hier auch eine alternative Implementierung vorgesehen, siehe revpi\_common.h}

In Zeile 889 wird das Byte \lstinline{i8uValue_l}, welches den zu schreibenden Wert enthält in das Prozessabbild übertragen. Anschließend wird die Mutex auf \lstinline{piDev_g} wieder entsperrt.
\newpage

\begin{lstlisting}[language={c},firstnumber=62,caption={Auszug des Struct \lstinline{spiControlDev} in \lstinline{piControlMain.h}\label{lst:4-spiControlDev}}]
|>\tikzmarkin[set border color=martiniblue]{spiControlDev}<|typedef struct spiControlDev|>\tikzmarkend{spiControlDev}<| {
	// device driver stuff
	int init_step;
	enum revpi_machine machine_type;
	void *machine;
	struct cdev cdev;	// Char device structure
	struct device *dev;
	struct thermal_zone_device *thermal_zone;

	|>\tikzmarkin[set border color=martiniblue]{processImage}<|// process image stuff
	INT8U ai8uPI[KB_PI_LEN];
	INT8U ai8uPIDefault|>\tikzmarkin[set border color=martinired]{KB_PI_LEN_0}<|[KB_PI_LEN]|>\tikzmarkend{KB_PI_LEN_0}<|;
	struct rt_mutex lockPI;        |>\tikzmarkend{processImage}<|
	bool stopIO;
	piDevices *devs; |>\setcounter{lstnumber}{94}<|
} tpiControlDev;
\end{lstlisting}

Das Prozessabbild ist als Byte-Array der Länge \lstinline{KB_PI_LEN} in Listing~\ref{lst:4-spiControlDev} definiert. Konfigurationsparameter wie \lstinline{KB_PI_LEN} oder die Zykluszeit für den Datenaustausch zwischen SPS und IO-Modulen sind im folgenden Listing~\ref{lst:4-process} definiert.

\begin{lstlisting}[language={c},firstnumber=119,caption={Konfigurationsparameter des Prozessabbildes in project.h\label{lst:4-process}}]
#define INTERVAL_PI_GATE (5*1000*1000)  // 5 ms piGateCommunication |>\setcounter{lstnumber}{128}<|

#define INTERVAL_IO_COM (5*1000*1000)  // 5 ms piIoComm |>\setcounter{lstnumber}{132}<|

#define KB_PD_LEN       512
|>\tikzmarkin[set border color=martiniblue]{KB_PI_LEN_1}<|#define KB_PI_LEN       4096|>\tikzmarkend{KB_PI_LEN_1}<|
\end{lstlisting}

Das zu setzende Bit wurde zu diesem Zeitpunkt erfolgreich in das Prozessabbild der SPS geschrieben.
Es stellt sich die Frage, wie dieses nun an das IO-Modul kommuniziert wird.
Die Kommunikation mit allen angebundenen Modulen ist ebenfalls Aufgabe des piControl-Treibers.

\begin{lstlisting}[language={c},firstnumber=256,caption={Auszug der Methode \lstinline{piIoThread} in \lstinline{revpi_core.c}\label{lst:4-piIoThread}}]
static int piIoThread(void *data)
{
	//TODO int value = 0;
	ktime_t time;
	ktime_t now;
	s64 tDiff;

	hrtimer_init(&piCore_g.ioTimer, CLOCK_MONOTONIC, HRTIMER_MODE_ABS);
	piCore_g.ioTimer.function = piIoTimer;

	pr_info("piIO thread started\n");

	now = hrtimer_cb_get_time(&piCore_g.ioTimer);

	PiBridgeMaster_Reset();

	while (!kthread_should_stop()) {
		if (|>\tikzmarkin[set border color=martinired]{PiBridgeMaster}<|PiBridgeMaster_Run()|>\tikzmarkend{PiBridgeMaster}<| < 0)
			break;
	}

	RevPiDevice_finish();

	pr_info("piIO exit\n");
	return 0;
}
\end{lstlisting}

Der Kernel-Thread \lstinline{piIoThread} ist verantwortlich für den zyklischen Datenaustausch mit den IO-Modulen. In diesem wird fortlaufend die Methode \lstinline{PiBridgeMaster_Run()} aufgerufen, siehe Listing~\ref{lst:4-piIoThread}.

\begin{lstlisting}[language={c},firstnumber=262,caption={Auszug der Methode \lstinline{PiBridgeMaster_Run(void)} in \lstinline{RevPiDevice.c}\label{lst:4-PiBridgeMaster_Run}}]
int PiBridgeMaster_Run(void)
{
	static kbUT_Timer tTimeoutTimer_s;
	static kbUT_Timer tConfigTimeoutTimer_s;
	static int error_cnt;
	static INT8U last_led;
	static unsigned long last_update;
	int ret = 0;
	int i;

	my_rt_mutex_lock(&piCore_g.lockBridgeState);
	if (piCore_g.eBridgeState != piBridgeStop) {
		switch (eRunStatus_s) { |>\setcounter{lstnumber}{514}<|
		    case enPiBridgeMasterStatus_EndOfConfig:|>\setcounter{lstnumber}{621}<|
		    if (|>\tikzmarkin[set border color=martinired]{RevPiDevice}<|RevPiDevice_run()|>\tikzmarkend{RevPiDevice}<|) {
				// an error occured, check error limits |>\setcounter{lstnumber}{641}<|
			} else {
				ret = 1;
			}
			piCore_g.image.drv.i16uRS485ErrorCnt = RevPiDevice_getErrCnt();
			break;
\end{lstlisting}

Die in Listing~\ref{lst:4-PiBridgeMaster_Run} dargestellte Methode ist eine sog. State-Machine. Ist die Konfiguration der IO-Module erfolgreich abgeschlossen, so führt sie bei Aufruf lediglich die Methode \lstinline{RevPiDevice_run()} aus.

\begin{lstlisting}[language={c},firstnumber=140,caption={Auszug der Methode \lstinline{RevPiDevice_run(void)} in \lstinline{RevPiDevice.c}\label{lst:4-RevPiDevice_run}}]
int RevPiDevice_run(void)
{
	INT8U i8uDevice = 0;
	INT32U r;
	int retval = 0;

	RevPiDevices_s.i16uErrorCnt = 0;

	for (i8uDevice = 0; i8uDevice < RevPiDevice_getDevCnt(); i8uDevice++) {
		if (RevPiDevice_getDev(i8uDevice)->i8uActive) {
			switch (RevPiDevice_getDev(i8uDevice)->sId.i16uModulType) {
			case KUNBUS_FW_DESCR_TYP_PI_DIO_14:
			case KUNBUS_FW_DESCR_TYP_PI_DI_16:
			case KUNBUS_FW_DESCR_TYP_PI_DO_16:
				r = |>\tikzmarkin[set border color=martinired]{sendCyclicTelegram}<|piDIOComm_sendCyclicTelegram(i8uDevice)|>\tikzmarkend{sendCyclicTelegram}\setcounter{lstnumber}{166} <|;

				break; |>\setcounter{lstnumber}{216}<|
			}
		}
	} |>\setcounter{lstnumber}{227}<|
	return retval;
}
\end{lstlisting}

Diese iteriert wie in Listing~\ref{lst:4-RevPiDevice_run} abgebildete durch alle gegenwärtig in der SPS konfigurierten Module. Ist das aktuelle Modul als aktiv markiert, so wird anhand eines sog. Firmware-Descriptors entschieden, welche Methode für die Ansteuerung des Moduls aufzurufen ist.

\begin{lstlisting}[language={c},firstnumber=161,caption={Auszug der Methode \lstinline{piDIOComm_sendCyclicTelegram} in \lstinline{piDIOComm.c}\label{lst:4-sendCyclicTelegram}}]
INT32U piDIOComm_sendCyclicTelegram(INT8U i8uDevice_p)
{
	INT32U i32uRv_l = 0;
	SIOGeneric sRequest_l;
	SIOGeneric sResponse_l;
	INT8U len_l, data_out[18], i, p, data_in[70];
	INT8U i8uAddress;
	int ret; |>\setcounter{lstnumber}{239}<|
	
    |>\tikzmarkin[set border color=martinired]{piIoComm}<|ret = piIoComm_send((INT8U *) & sRequest_l, IOPROTOCOL_HEADER_LENGTH + len_l + 1);  |>\tikzmarkend{piIoComm}\setcounter{lstnumber}{298}<|
}
\end{lstlisting}

Im Falle des hier verwendeten DO-Moduls wird die in Listing~\ref{lst:4-sendCyclicTelegram} abgebildete Methode \lstinline{piDIOComm_sendCyclicTelegram()} aufgerufen. Dieser wird ein Zeiger auf das zu schreibende Gerät übergeben. 
Zunächst wird das Prozessabbild mittels eines proprietären, jedoch im Quellcode offen nachvollziehbaren Protokolls in ein \lstinline{sRequest_l} genanntes Byte-Array umgewandelt. Dieser Schritt ist in Listing~\ref{lst:4-sendCyclicTelegram} nicht abgebildet. Anschließend wird \lstinline{piIoComm_send()} ein Zeiger auf die so generierte Schreib-Anfrage übergeben.

\begin{lstlisting}[language={c},firstnumber=220,caption={Auszug der Methode \lstinline{piIOComm_send} in \lstinline{piIOComm.c}\label{lst:4-piIOComm_send}}]
int piIoComm_send(INT8U * buf_p, INT16U i16uLen_p)
{
	ssize_t write_l = 0;
	INT16U i16uSent_l = 0;|>\setcounter{lstnumber}{249}<|

	while (i16uSent_l < i16uLen_p) {
		write_l = vfs_write(piIoComm_fd_m, buf_p + i16uSent_l, i16uLen_p - i16uSent_l, &piIoComm_fd_m->f_pos);
		if (write_l < 0) {
			pr_info_serial("write error %d\n", (int)write_l);
			return -1;
		} 
		i16uSent_l += write_l;|>\setcounter{lstnumber}{263}<|
	}
	clear();
	vfs_fsync(piIoComm_fd_m, 1);
	return 0;
}
\end{lstlisting}

Listing~\ref{lst:4-piIOComm_send} zeigt die Implementierung von \lstinline{piIoComm_send()}. Diese Methode ist für das Schreiben der oben generierten Anfrage auf die seriellen Schnittstelle verantwortlich. Realisiert wird dies mittels der Methode \lstinline{vfs_write()}. Diese ist in \lstinline{<linux/fs.h>} definiert. Sie ermöglicht das Schreiben einer Datei im Userspace aus dem Kernel heraus. Geschrieben wird hier die Datei mit dem Deskriptor \lstinline{piIoComm_fd_m}.
Da die Funktion \lstinline{vfs_write()} durch andere Kernel-Tasks unterbrochen werden kann, ist nicht gewährleistet, dass die gesamte Anfrage mit nur einem Aufruf geschrieben wird. Die oben abgebildete while-Schleife stellt das vollständige Senden der Anfrage sicher.

\begin{lstlisting}[language={c},firstnumber=157,caption={Auszug der Methode \lstinline{piIOComm_open_serial} in \lstinline{piIOComm.c}\label{lst:4-piIOComm_open_serial}}]
int piIoComm_open_serial(void)
{   |>\setcounter{lstnumber}{167}<|
	struct file *fd;	/* Filedeskriptor */
	struct termios newtio;	/* Schnittstellenoptionen */

	|>\tikzmarkin[set border color=martiniblue]{fd}<|/* Port oeffnen - read/write, kein "controlling tty", 
	    Status von DCD ignorieren */
	fd = filp_open(|>\tikzmarkin[set border color=martinired]{tty}<|REV_PI_TTY_DEVICE|>\tikzmarkend{tty}<|, O_RDWR | O_NOCTTY, 0); |>\setcounter{lstnumber}{208}<|
	
	piIoComm_fd_m = fd;                                                      |>\tikzmarkend{fd}\setcounter{lstnumber}{217}<|

	return 0;
}
\end{lstlisting}

Der zum Schreiben auf die serielle Schnittstelle verwendete Datei-Deskriptor wird von der in Listing~\ref{lst:4-piIOComm_open_serial} abgebildeten Methode \lstinline{piIoComm_open_serial()} generiert. 

\begin{lstlisting}[language={c},firstnumber=45,caption={Definition der seriellen Schnittstelle in \lstinline{piIOComm.h}\label{lst:4-REV_PI_TTY_DEVICE}}]
#define REV_PI_TTY_DEVICE	"/dev/ttyAMA0"
\end{lstlisting}

Das in Listing~\ref{lst:4-REV_PI_TTY_DEVICE} definierte Macro verweist auf eine der seriellen Schnittstellen des RaspberryPi.
Die Implementierung des zugehörigen Schnittstellentreibers soll hier nicht weiter untersucht werden. Somit ist an dieser Stelle die Kette vom Setzen einer Variablen auf dem OPC-Server bis hin zur Aktualisierung des Prozessabbilds der IO-Module geschlossen.

% \begin{lstlisting}[language={c},firstnumber={226},caption={Setzen der Scheduler-Priorität auf SCHED\_FIFO in 
% revpi\_common.c\label{lst:2-sched_priority}}]
% param.sched_priority = ktprio->prio;
% ret = sched_setscheduler(child, SCHED_FIFO, &param);
% \end{lstlisting}
% % % Imports nur für Referenzenauflösung während des Schreibens! Vorm Kompilieren auskommentieren!
% \bibliography{0_hauptdatei}
% \input{1_einleitung}
% \input{2_grundlagen}
% \input{3_konzeption}
% \input{4_implementierung}
% \input{5_tests}
% \input{6_zusammenfassung}
% % Ende Imports

\section{Test des OPC-Servers im Gesamtsystem%
  \label{sec:5-tests}}

% % % Imports nur für Referenzenauflösung während des schreibens! Vorm Kompilieren auskommentieren!
% \bibliography{0_hauptdatei}
% \input{1_einleitung}
% \input{2_grundlagen}
% \input{3_konzeption}
% \input{4_implementierung}
% \input{5_tests}
% \input{6_zusammenfassung}
% % Ende Imports

\section{Zusammenfassung und Ausblick%
  \label{sec:6-fazit}}
Der folgende Abschnitt~\ref{sec:6-zusammenfassung} fasst die gewonnenen Erkenntnisse und den Stand der Implementierung zusammen.
Den Abschluss dieser Arbeit bildet der Ausblick in Abschnitt~\ref{sec:6-ausblick}.

\subsection{Zusammenfassung%
     \label{sec:6-zusammenfassung}}

\subsection{Ausblick%
     \label{sec:6-ausblick}}

% % Ende Imports

\section{Test des OPC-Servers im Gesamtsystem%
  \label{sec:5-tests}}

% % % Imports nur für Referenzenauflösung während des schreibens! Vorm Kompilieren auskommentieren!
% \bibliography{0_hauptdatei}
% % Mit \section{...} eröffnen wir einen neuen Abschnitt.
% Der Befehl setzt nicht nur den Text in einer größeren,
% fetten Schrift, sondern sorgt außerdem dafür, daß er im
% Inhaltsverzeichnis erscheint.
%
% Mit \label{...} erzeugen wir einen Bezeichner, mit dessen Hilfe
% wir später auf die Nummer des Abschnitts verweisen können (nämlich
% mit~\ref{...}).
%
% Das Kommentarzeichen hinter „Übersicht“ dient dazu, ein
% Leerzeichen zwischen „Übersicht“ und dem \label-Befehl
% zu vermeiden, das andernfalls sichtbar würde – z.B. im
% Inhaltsverzeichnis.
%

% % Imports nur für Referenzenauflösung während des Schreibens! Vorm Kompilieren auskommentieren!
% \bibliography{0_hauptdatei}
% \input{1_einleitung}
%\input{2_grundlagen}
%\input{3_konzeption}
%\input{4_implementierung}
%\input{5_tests}
%\input{6_zusammenfassung}
% % Ende Imports

\section{Einleitung und Motivation%
  \label{sec:1-einleitung}}
Ziel dieses Projektes ist die Integration eines OPC-Servers mit einer auf Linux
basierenden speicherprogrammierbaren Steuerung (SPS). Angeschlossen an diese SPS
ist jeweils ein digitales Ein-/\,bzw.~Ausgabemodul. Die von diesen bereitgestellten
Ein-/\, bzw.~Ausgänge (IO) sollen in der Datenstruktur des OPC-Servers abgebildet
und über diesen für OPC-Clients les-/\,und schreibar sein. Weiterhin sollen einige
Funktionen zur Überwachung und Steuerung der an die SPS angeschlossenen Aktoren
und Sensoren direkt im OPC-Server implementiert werden.
Hiermit stellt dieses Projekt eine der Grundlagen für ein übergeordnetes Projekt,
die cloudbasierte Steuerung eines miniaturisierten Produktions-Systems, dar.

Der hier verwendete OPC-Server ist Teil des sog. open62541 Projekts. Er ist in C
geschrieben und implementiert bereits einen großen Teil der im OPC-UA-Standard
spezifizierten Funktionen.
Als SPS findet ein Revolution Pi 3 der Firma Kunbus Verwendung. Dieser integriert
ein sog. Compute Module der Raspberry Pi Foundation in ein industrietaugliches
Gehäuse und erlaubt die Erweiterung mittels IO- oder Gateway-Modulen. Über diese
erfolgt die Kommunikation mit weiteren Komponenten der Automatisierungstechnik.

Motiviert ist dieses Projekt durch die Beobachtung, dass die Verbreitung offener
Standards sowie freier Software auch in der Automatisierungstechnik zunimmt.
Linux ist ein freies Betriebssystem, OPC-UA ein offen zugänglicher, aktiv gepflegter
und weit verbreiteter Standard. Der Raspberry Pi findet sowohl bei Hobby-Anwendern als
auch in den Bereichen Forschung und Entwicklung sowie bei industriellen Anwendern
Verwendung. Dieses Projekt stellt somit eine für unterschiedliche Anwender interessante
Entwicklung dar.

Im Anschluss an diese einleitende Übersicht im Abschnitt~\ref{sec:1-einleitung} folgt
die Darstellung der wichtigsten Grundlagen in Abschnitt~\ref{sec:2-grundlagen}.
Aufbauend auf diesen Grundlagen folgt die konzeptuelle Ausarbeitung im Abschnitt~\ref{sec:3-konzeption}.
Die Umsetzung wird im Abschnitt~\ref{sec:4-implementierung} erläutert.
Die Leistungsfähigkeit der Implementierung wird in Abschnitt~\ref{sec:5-tests} untersucht.
Eine Zusammenfassung und ein Ausblick schließen die Arbeit in
Abschnitt~\ref{sec:6-fazit} ab. Eventuell noch benötigte Anhänge
finden sich in den Anhängen [...] bis [...].

% % % Imports nur für Referenzenauflösung während des Schreibens! Vorm Kompilieren auskommentieren!
% \bibliography{0_hauptdatei}
% \input{1_einleitung}
% \input{2_grundlagen}
% \input{3_konzeption}
% \input{4_implementierung}
% \input{5_tests}
% \input{6_zusammenfassung}
% % Ende Imports

\section{Grundlagen%
  \label{sec:2-grundlagen}}

\subsection{Speicherprogrammierbare-Steuerung und Linux -- Revolution Pi%
     \label{sec:2-sps}}

\subsubsection{Kunbus RevolutionPi%
        \label{sec:2-revpi}}
Der RevolutionPi 3 ist eine speicherprogrammierbare Steuerung (SPS) des Herstellers
Kunbus GmbH. Kern dieser SPS ist das von der Raspberry Pi Foundation entwickelte
und vertriebene Raspberry Pi Compute Module 3. Dieses integriert ein Broadcom BCM2837
System-on-Chip (SoC) mit vier 1,2GHz Prozessorkernen, 1GB RAM, 4GB eMMC Anwendungsspeicher
und sonstige Peripherie in ein Modul im DDR2-SODIMM Formfaktor. Diese Spezifikationen
sind weitgehend identisch zu denen des ausgesprochen populären Raspberry Pi 3.
Der Revolution Pi profitiert daher von dem gleichen großen Angebot an Software
und Unterstützung wie der Raspberry Pi, ergänzt dessen Hardware jedoch um eine 24V
Spannungsversorgung, die Möglichkeit der Erweiterung durch mehrere industrietaugliche
Ein-/ Ausgabemodule und Gateways sowie ein Gehäuse zur Montage auf einer DIN-Schiene.
\begin{itemize}
  \item{Prozessor: BCM2837}
  \item{Taktfrequenz 1,2 GHz}
  \item{Anzahl Prozessorkerne: 4}
  \item{Arbeitsspeicher: 1 GByte}
  \item{eMMC Flash Speicher: 4 GByte}
  \item{Betriebssystem: Angepasstes Raspbian mit RT-Patch}
  \item{RTC mit 24h Pufferung über wartungsfreien Kondensator}
  \item{Treiber / API: Treiber schreibt zyklisch Prozessdaten in ein Prozessabbild, Zugriff auf Prozessabbild über Linux-Filesystem als API zu Fremdsoftware.}
  \item{Kommunikationsanschlüsse: 2 x USB 2.0 A (je 500 mA belastbar), 1 x Micro-USB, HDMI, Ethernet (RJ45) 10/100 Mbit/s}
  \item{Stromversorgung: min. 10,7 V, max. 28,8 V, maximal 10 Watt}
  \item{Zulässige Umgebungstemperatur: -40 bis +55 C}
  \item{Gehäuseabmessungen: (HxBxL) 96 mm x 22,5 mm x 110,5 mm (ohne gesteckte Stecker)}
  \item{ESD Schutz: 4 kV / 8 kV gemäß EN61131-2 und IEC 61000-6-2}
  \item{Surge / Burst Prüfungen: gemäß EN61131-2 und IEC 61000-6-2 eingekoppelt auf Versorgungsspannung, Ethernet und IO-Leitungen}
  \item{EMI Prüfungen: gemäß EN61131-2 und IEC 61000-6-2}
\end{itemize}

Kunbus bietet eine Auswahl an IO- und Gateway-Modulen zur Erweiterung des Revolution Pi an.
Gateways dienen der Kommunikation mit Systemen oder Komponenten der Automatisierungstechnik
über Protokolle wie PROFIBUS oder EtherCAT. IO-Module erlauben die Überwachung
und Steuerung von digitalen oder analogen Ein- und Ausgängen.

\subsubsection{Zugriff auf IO-Module%
        \label{sec:2-io}}
Der Zugriff auf die Ein- und Ausgänge der IO-Module erfolgt über ein Prozessabbild
und einen hierfür von Kunbus bereitgestellten Treiber, genannt piControl. Dieser
aktualisiert das Prozessabbild zyklisch. Die angestrebte Zykluszeit beträgt 5ms,
kann jedoch je nach Anzahl der angeschlossenen Module auch größer sein. Kunbus
garantiert bei drei IO-Modulen und zwei Gateway-Modulen eine Zykluszeit von 10 ms.
Jedes der IO-Module stellt ein eigenständiges eingebettetes System dar. Es verfügt
über einen Microcontroller, welcher die IOs bereitstellt und über einen RS485-Bus
mit dem Revolution Pi kommuniziert.
% https://revolution.kunbus.de/io-modul/

Lizenz: GPL
% https://github.com/RevolutionPi/piControl

\begin{lstlisting}[language={c},firstnumber={226},caption={Setzen der Scheduler-Priorität auf SCHED\_FIFO in revpi\_common.c\label{lst:2-sched_priority}}]
param.sched_priority = ktprio->prio;
ret = sched_setscheduler(child, SCHED_FIFO,
       &param);
\end{lstlisting}


\subsection{Echtzeit und Multithreading unter Linux -- preemptRT und posix%
     \label{sec:2-echtzeit}}


 Der Linux-Kernel verfügt über mehrere unterschiedliche Preemtion-Modelle:

\begin{itemize}
  \item No Forced Preemption (server):
  Ausgelegt auf maximal möglichen Durchsatz, lediglich Interrupts und
  System-Call-Returns bewirken Präemption.

  \item Voluntary Kernel Preemption (Desktop):
  Neben den implizit bevorrechtigten Interrupts und System-Call-Returns gibt es
  in diesem Modell weitere Abschnitte des Kernels in welchen Preämption explizit
  gestattet ist.

  \item Preemptible Kernel (Low-Latency Desktop):
  In diesem Modell ist der gesamte Kernel, mit Ausnahme sog.~kritischer Abschnitte
  präemptible. Nach jedem kritischen Abschnitt gibt es einen impliziten Präemptions-Punkt.

  \item Preemptible Kernel (Basic RT):
  Dieses Modell ist dem zuvor genannten sehr ähnlich, hier sind jedoch alle Interrupt-Handler
  als eigenständige Threads ausgeführt.

  \item Fully Preemptible Kernel (RT):
  Wie auch bei den beiden zuvor genannten Modellen ist hier der gesamte Kernel
  präemtible, die Anzahl und Dauer der nicht-präemtiblen kritischen Abschnitte
  ist auf ein notwendiges Minimum beschränkt. Alle Interrupt-Handler sind als
  eigenständige Threads ausgeführt, Spinlocks durch Sleeping-Spinlocks und Mutexe
  durch sog.~RT-Mutexe ersetzt.

\end{itemize}
\todo{Spinlocks und Mutexe sowie die RT-Varianten dieser erklären!}

Lediglich mit dem vollständig präemtiblen Kernel kann Echtzeit-Verhalten realisiert werden.

% https://wiki.linuxfoundation.org/realtime/documentation/technical_basics/preemption_models bzw kernel/Kconfig.preempt

\subsubsection{preemptRT%
        \label{sec:2-preemptRT}}
% https://wiki.linuxfoundation.org/realtime/documentation/technical_details/start
% https://wiki.linuxfoundation.org/realtime/documentation/technical_basics/start

Das dem PREEMPT RT Kernel zugrunde liegende Prinzip lässt sich in einer einfachen
Regel ausdrücken: Nur Code, welcher absolut nicht-präemtible sein darf, ist es
gestattet nicht-präemtible zu sein.
Das erklärte Ziel des PREEMPT\_RT Patches ist es folglich, die Menge des nicht-präemtiblen
Codes im Linux-Kernel auf das absolut notwendige Minimum zu reduzieren.

Dies wird durch Verwendung folgender Mechanismen erreicht:

\begin{itemize}
  \item Hochauflösende Timer
  \item Sleeping Spinlocks
  \item Threaded Interrupt Handlers
  \item rt\_mutex
  \item RCU
\end{itemize}


\subsubsection{posix%
        \label{sec:2-posix}}
Ist posix hier wirklich relevant? Debian bzw.~Raspbian sind weitgehend posix
kompatibel, aber wird es hier genutzt? -> JA, open62541 nutzt pthread.h
piControl nutzt kthread.h, und semaphore.h

\subsection{OPC-UA und open62541%
     \label{sec:2-opc}}

\subsubsection{OPC UA%
        \label{sec:2-opcua}}
Open Platform Communications (OPC) ist eine Familie von Standards zur herstellerunabhängigen
Kommunikation von Maschinen (M2M) in der Automatisierungstechnik. Die sog.~OPC Task Force, zu deren
Mitgliedern verschiedene große Firmen der Automatisierungsindustrie gehören, veröffentlichte
die OPC Specification Version 1.0 im August 1996.
Motiviert ist dieser offene Standard durch die Erkenntniss, dass die Anpassung der
zahlreichen Herstellerstandards an individuelle Infrastrukturen und Anlagen einen
großen Mehraufwand verursachen.
Die Wikipedia beschreibt das Anwendungsgebiet für OPC wie folgt:

\glqq{}OPC wird dort eingesetzt, wo Sensoren, Regler und Steuerungen verschiedener Hersteller
ein gemeinsames Netzwerk bilden. Ohne OPC benötigten zwei Geräte zum Datenaustausch
genaue Kenntnis über die Kommunikationsmöglichkeiten des Gegenübers. Erweiterungen
und Austausch gestalten sich entsprechend schwierig. Mit OPC genügt es, für jedes
Gerät genau einmal einen OPC-konformen Treiber zu schreiben. Idealerweise wird
dieser bereits vom Hersteller zur Verfügung gestellt. Ein OPC-Treiber lässt sich
ohne großen Anpassungsaufwand in beliebig große Steuer- und Überwachungssysteme
integrieren.

OPC unterteilt sich in verschiedene Unterstandards, die für den jeweiligen Anwendungsfall
unabhängig voneinander implementiert werden können. OPC lässt sich damit verwenden
für Echtzeitdaten (Überwachung), Datenarchivierung, Alarm-Meldungen und neuerdings
auch direkt zur Steuerung (Befehlsübermittlung).\grqq{}

OPC basiert in der ursprünglichen Spezifikation auf Microsofts DCOM-Spezifikation.
DCOM macht Funktionen und Objekte einer Anwendung anderen Anwendungen im Netzwerk
zugänglich. Der OPC-Standard definiert entsprechende DCOM-Objekte um mit anderen
OPC-Anwendungen Daten austauschen zu können. Die Verwendung von DCOM bindet Anwender
an Betriebssysteme von Microsoft. Die ursprüngliche OPC Spezifikation wird durch die
Entwicklung von OPC Unified Architecture (OPC UA) abgelöst.
OPC UA setzt auf einem eigenen Kommunikationionsstack auf, die Verwendung von DCOM
und damit die Bindung an Microsoft wurden aufgelöst.

Die OPC-UA-Architektur ist eine Service-orientierte Architektur (SOA), deren Struktur
aus mehreren Schichten besteht.

% Wikipedia
Das OPC-Informationsmodell ist nicht mehr nur eine Hierarchie aus Ordnern, Items
und Properties. Es ist ein sogenanntes Full-Mesh-Network aus Nodes, mit dem neben
den Nutzdaten eines Nodes auch Meta- und Diagnoseinformationen repräsentiert werden.
Ein Node ähnelt einem Objekt aus der objektorientierten Programmierung. Ein Node
kann Attribute besitzen, die gelesen werden können (Data Access (DA), Historical
Data Access (HDA)). Es ist möglich Methoden zu definieren und aufzurufen.
Eine Methode besitzt Aufrufargumente und Rückgabewerte. Sie wird durch ein Command
aufgerufen. Weiterhin werden Events unterstützt, die versendet werden können
(AE (Alarms \& Events), DA DataChange), um bestimmte Informationen zwischen Geräten
auszutauschen. Ein Event besitzt unter anderem einen Empfangszeitpunkt, eine Nachricht
und einen Schweregrad. Die o. g. Nodes werden sowohl für die Nutzdaten als auch
alle anderen Arten von Metadaten verwendet. Der damit modellierte OPC-Adressraum
beinhaltet nun auch ein Typmodell, mit dem sämtliche Datentypen spezifiziert werden.

% https://de.wikipedia.org/wiki/Open_Platform_Communications
% https://de.wikipedia.org/wiki/OPC_Unified_Architecture
% https://opcfoundation.org/developer-tools/specifications-unified-architecture
% Von Gerhard Gappmeier - ascolab GmbH, CC BY-SA 3.0, https://de.wikipedia.org/w/index.php?curid=1892069
\subsubsection{open62541%
        \label{sec:2-open62541}}
open62541 ist eine offene und freie Implementierung von OPC UA. Die in C geschriebene
Bibliothek stellt eine beständig zunehmende Anzahl der im OPC UA Standard definierten
Funktionen bereit. Sie kann sowohl zur Erstellung von OPC-Servern als auch -Clients
genutzt werden. Ergänzend zu der unter der Mozilla Public License v2.0 lizensierten
Bibliothek stellt das open62541 Projekt auch Beispielprogramme unter einer CC0 Lizenz
zur Verfügung.

Die Bibliothek eignet sich auch für die Entwicklung auf eingebetteten Systemen und
Microcontrollern. Je nach Umfang der gewünschten Funktionen und des OPC Informationsmodells
beträgt die Größe einer Server-Binary weniger als 100kb. %evtl. kürzen?

\todo{Nodes erklären! Evtl.~oben!}

Folgende Auswahl an Eigenschaften und Funktionen zeichnet die in dieser Arbeit verwendete
Version 0.3 von open62541 aus:
\begin{itemize}
  \item Kommunikationionsstack
  \begin{itemize}
      \item OPC UA Binär-Protokoll (HTTP oder SOAP werden gegenwärtig nicht unterstützt)
      \item Austauschbare Netzwerk-Schicht, welche die Verwendung eigener Netzwerk-APIs
      erlaubt.
      \item Verschlüsselte Kommunikationion
      \item Asynchrone Dienst-Anfragen im Client
  \end{itemize}
  \item Informationsmodell
  \begin{itemize}
    \item Unterstützung aller OPC UA Node-Typen, inkl.~Methoden
    \item Hinzufügen und Entfernen von Nodes und Referenzen zur Laufzeit.
    \item Vererbung und Instanziierung von Objekt- und Variablentypen
    \item Zugriffskontrolle auch für einzelne Nodes
  \end{itemize}
  \item Subscriptions
  \begin{itemize}
    \item Erlaubt die Überwachung (subscriptions / monitoreditems)
    \item Sehr geringer Ressourcenbedarf pro überwachtem Wert
  \end{itemize}
  \item Code-Generierung auf XML-Basis
  \begin{itemize}
    \item Erlaubt die Erstellung von Datentypen
    \item Erlaubt die Generierung des serverseitigen Informationsmodells
  \end{itemize}
\end{itemize}

% https://open62541.org/doc/0.3/


Mozilla Public License
CC0 Lizenz für Beispiele und Plugins

% https://open62541.org/doc/open62541-current.pdf
% https://open62541.org/

% % % Imports nur für Referenzenauflösung während des Schreibens! Vorm Kompilieren auskommentieren!
% \bibliography{0_hauptdatei}
% \input{1_einleitung}
% \input{2_grundlagen}
% \input{3_konzeption}
% \input{4_implementierung}
% \input{5_tests}
% \input{6_zusammenfassung}
% \input{anhang}
% % Ende Imports

\section{Systemkonzept%
  \label{sec:3-konzeption}}
Auf Basis der in Abschnitt \ref{sec:2-grundlagen} vorgestellten Möglichkeiten folgt nun die Ausarbeitung eines Konzepts.
In den folgenden Abschnitten soll näher auf zwei zentrale Aspekte eingegangen werden: Abschnitt~\ref{sec:3-anbindung} stellt Möglichkeiten zum Zugriff auf Variablen bzw.\,Werte im Prozessabbild des Revolution Pi vor; in Abschnitt~\ref{sec:3-integration} wird ein Konzept zur Bereitstellung dieser Variablen auf einem OPC-Server vorgestellt.

\subsection{Anbindung der IO an den OPC-Server%
     \label{sec:3-anbindung}}

Eine Webanwendung mit Bezeichnung PiCtory dient zur Konfiguration der I/O- und virtuellen Module des RevolutionPi. Die Konfiguration liegt im JSON-Format in der Datei \lstinline{/etc/revpi/config.rsc}. Der piControl-Treiber liest diese Datei beim Start. 
Der folgende Auszug aus der Manpage des piControl-Kernelmoduls beschreibt die von diesem zum Lesen und Schreiben einzelner Bits des Prozessabbildes bereitgestellten Funktionen~\citep[vgl.]{web-revpi-manpage}. Sie ist an dieser Stelle weitgehend ungekürzt zitiert, da sie die nutzbare Schnittstelle sehr kompakt beschreibt.

\begin{lstlisting}[breakindent=0pt, numbers=none, caption={Auszug aus der Revolution Pi Programmers Manual\label{lst:4-manpage}}]
KB_FIND_VARIABLE SPIVariable *argp
Find a variable in the process image by its name. A pointer to a structure of type SPIVariable must be passed as argument. [...]
The struct SPIVariable [...] is defined as 
typedef struct SPIVariableStr
{
    char strVarName[32]; // Variable name
    uint16_t i16uAddress; // Address of the byte in the process image
    uint8_t i8uBit; // 0-7 bit position, >= 8 whole byte
    uint16_t i16uLength; // length of the variable in bits.
    // Possible values are 1, 8, 16 and 32
} SPIVariable;

Set and get values of the process image
KB_GET_VALUE SPIValue *argp
[...]
KB_SET_VALUE SPIValue *argp
Write one bit or one byte to the process image [...].  This call is more efficient than the usual calls of seek and write because only one function call is necessary. If more than on application are writing bits in one output byte, this call is the only safe way to set a bit without overwriting the other bits because this call is doing a read-modify-write-cycle. 

The struct SPIValue used by this ioctl is defined as
typedef struct SPIValueStr
{
    uint16_t i16uAddress; // Address of the byte in the process image
    uint8_t i8uBit; // 0-7 bit position, >= 8 whole byte
    uint8_t i8uValue; // Value: 0/1 for bit access, whole byte otherwise
} SPIValue;
\end{lstlisting} 

Die oben beschriebenden Funtkionen \lstinline{KB_FIND_VARIABLE}, \lstinline{KB_GET_VALUE} und \lstinline{KB_SET_VALUE} ermöglichen einen einfachen und (lt.\,Manpage) effizienten Zugriff auf einzelne Bits des Prozessabbildes und damit auch auf die IO des RevolutionPi.
Der Zugriff des OPC-Servers auf das Prozessabbild soll daher mittels dieser Funktionen realisiert werden.
\lstinline{KB_FIND_VARIABLE} kann genutzt werden, um Adressen von Variablen im Prozessabbild mittels ihres Namens aufzulösen.
\lstinline{KB_GET_VALUE} und \lstinline{KB_SET_VALUE} ermöglichen den Zugriff auf die Werte dieser Variablen.


\subsection{Integration des OPC-Servers in das System%
     \label{sec:3-integration}}

open62541 bietet drei Möglichkeiten zum Abgleich von Variablen mit dem Prozessabbild~\citep[vgl.][Tutorials - Connecting a Variable with a Physical Process]{web-open62541}:
\begin{itemize}
    \item Manuelles oder zyklisches Aktualisieren
    \item Variable Value Callback
    \item Variable Datasource
\end{itemize}

Die zyklische Aktualisierung eines oder mehrerer Werte nimmt, abhängig von der Zykluszeit, viele Systemressourcen in Anspruch. Value Callbacks ermöglichen es, einen Variablenwert effizienter mit einer Ressource wie etwa einem Prozessabbild zu synchronisieren. An die Variable wird ein Callback angehängt, welches vor jedem Lesen und nach jedem Schreibvorgang ausgeführt wird.
Der Wert der Variablen wird weiterhin im Variablenknoten auf dem OPC-Server gespeichert, der Abgleich mit der verknüpften Ressource erfolgt durch die Callback-Methoden.

Sogenannte Datenquellen gehen noch einen Schritt weiter. Der Server leitet jede Lese- und Schreibanforderung direkt an eine Callback-Funktion weiter. Beim Lesen liefert der Rückruf eine Kopie des aktuellen Wertes. Die Datenquelle muss intern ein eigenes Speichermanagement implementieren.

Der Zugriff auf die Werte des Prozessabbildes erfolgt, wie in Abschnitt~\ref{sec:3-anbindung} beschrieben, über von piControl bereitgestellte Methoden. Um die durch open62541 gepflegte OPC-Datenstruktur und das durch piControl verwaltete Prozessabbild möglichst effektiv verknüpfen zu können, soll diese Interaktion mittels Datenquellen und den zugehörigen Callbacks implementiert werden.
% % % Imports nur für Referenzenauflösung während des Schreibens! Vorm Kompilieren auskommentieren!
% \bibliography{0_hauptdatei}
% \input{1_einleitung}
% \input{2_grundlagen}
% \input{3_konzeption}
% \input{4_implementierung}
% \input{5_tests}
% \input{6_zusammenfassung}
% \input{anhang}
% % Ende Imports

\section{Implementierung%
  \label{sec:4-implementierung}}
Das folgende Kapitel stellt in Auszügen die Implementierung des OPC-Servers sowie die Anbindung an die IO-Module
der SPS dar. Der Schwerpunkt liegt hierbei auf der Funktionsweise des piControl-Treibers und dessen Integration in das Projekt. Abschnitt~\ref{sec:4-picontrol} erklärt die zum Schreibens eines Bits verwendeten Funktionsaufrufe.
Zuvor soll jedoch in Abschnitt~\ref{sec:4-open62541} der Teil des OPC-Servers vorgestellt werden, welcher auf besagten Treiber zugreift. 

\subsection{Implementierung des OPC-Servers%
     \label{sec:4-open62541}}
Wie im vorangegangenen Abschnitt~\ref{sec:3-integration} begründet, soll die Verknüpfung zwischen dem Prozessabbild der SPS und den auf dem OPC-Server bereitgestellten Werten über sog.\,Datenquellen erfolgen. Hierzu ist zunächst eine Callback-Methode zu implementieren, welche bei einem Lese- oder Schreibzugriff auf eine Variable aufgerufen wird. Die Verknüpfung zwischen Callback-Methode und Variable muss manuell erfolgen.

\begin{lstlisting}[language={c},firstnumber=237,caption={Auszug der Methode \lstinline{linkDataSourceVariable} in \lstinline{variables.c}\label{lst:4-linkDataSourceVariable}}]
extern UA_StatusCode
 linkDataSourceVariable(UA_Server *server, UA_NodeId nodeId) {
     bool readonly = false;
     UA_DataSource dataSourceVariable;
     UA_StatusCode rc; |>\setcounter{lstnumber}{254}<|

     dataSourceVariable.read = readDataSourceVariable;
     if (!readonly)
        dataSourceVariable.write = writeDataSourceVariable;
     else
        dataSourceVariable.write = writeReadonlyDataSourceVariable;

     return UA_Server_setVariableNode_dataSource(server, nodeId, dataSourceVariable);
 }
\end{lstlisting}

\begin{figure}[h]
    \centering
    \includegraphics[width=0.42\textwidth]{doc/img/OPC_RevPiDO.pdf}
    \caption{Auszug des verwendeten Nodesets, hier Digitalausgang 1 des Versuchsaufbaus
      \label{fig:opc-do}}
\end{figure}

Die in Listing~\ref{lst:4-linkDataSourceVariable} abgebildete Methode \lstinline{linkDataSourceVariable()} erzeugt ein Struct vom Typ \lstinline{UA_DataSource}. In diesem werden dem Lesen und Schreiben einer OPC-Variablen entsprechende Callback-Methoden zugewiesen. Die Verknüpfung einer OPC-Variable, genauer ihrer NodeId, mit der zuvor definierten Datenquelle erfolgt über die von open62541 bereitgestellte Methode \lstinline{UA_Server_setVariableNode_dataSource()}. Vor dem Lesen und nach dem Schreiben dieser Variable werden von nun an die entsprechenden Callbacks aufgerufen.
     
\begin{lstlisting}[language={c},firstnumber=168,caption={Auszug des Callbacks \lstinline{writeDataSourceVariable} in \lstinline{variables.c}\label{lst:4-writeDataSourceVariable}}]  
extern UA_StatusCode
 writeDataSourceVariable(UA_Server *server,
            const UA_NodeId *sessionId, void *sessionContext,
            const UA_NodeId *nodeId, void *nodeContext,
            const UA_NumericRange *range, const UA_DataValue *dataValue) {

    UA_StatusCode retval  = UA_STATUSCODE_GOOD;
    UA_NodeId *nameNodeId = UA_malloc(sizeof(UA_NodeId));
    UA_QualifiedName nameQN = UA_QUALIFIEDNAME(1, "Name");
    UA_Variant nameVar;
    UA_Boolean bit;

    retval |= findSiblingByBrowsename(server, nodeId, &nameQN, nameNodeId);
    retval |= UA_Server_readValue(server, *nameNodeId, &nameVar);
    retval |= UA_Boolean_copy(dataValue->value.data, &bit);

    |>\tikzmarkin[set border color=martinired]{writeIO}<|PI_writeSingleIO(String_fromUA_String(nameVar.data), &bit, false);                                                 |>\tikzmarkend{writeIO}<|

    free(nameNodeId);
    return retval;
 }
\end{lstlisting}

Listing~\ref{lst:4-writeDataSourceVariable} zeigt die Callback-Methode, welche nach dem Schreiben einer Variablen auf dem OPC-Server aufgerufen wird.
Dieser Methode wird neben der NodeId der mit ihr verknüpften Variablen auch der Wert dieser in Form eines Zeigers auf ein Struct vom Typ \lstinline{UA_DataValue} übergeben.

Die Gestaltung des hier verwendeten Nodesets sieht vor, dass in einer OPC-Variablen \lstinline{"Name"} der Bezeichner des zu schreibenden Digitalausgangs hinterlegt ist, siehe Abbildung~\ref{fig:opc-do}. Dies erlaubt eine Rekonfiguration der Ein- und Ausgänge der SPS ohne Änderungen im Programmcode des OPC-Servers vornehmen zu müssen.
Es ist daher erforderlich, nach jedem Schreiben einer mit einem Digitalausgang verknüpften Variablen, hier \lstinline{"Value"}, dessen Bezeichner \lstinline{"Name"} abzufragen. 
Dies geschieht in den Zeilen 180 und 181.
Anschließend wird dieser Bezeichner sowie der zu schreibende Wert der Methode \lstinline{PI_writeSingleIO()} übergeben, welche wiederum die Interaktion mit piControl übernimmt (vgl. Abschnitt \ref{sec:4-picontrol}).
 
\subsection{Integration von piControl%
     \label{sec:4-picontrol}}
In Abschnitt~\ref{sec:2-io} wurde die Anbindung der IO-Module des Revolution Pi sowie die Funktionsweise von piControl aus Anwendersicht beschrieben. Die verfügbare Literatur beschränkt sich auch auf lediglich diese Sicht; eine weiterführende Dokumentation für Entwickler gibt es, neben der in Abschnitt~\ref{sec:3-anbindung} vorgestellten Manpage, nicht. 
In diesem Abschnitt soll daher der Quellcode von piControl sowie dessen Verwendung im Projekt genauer betrachtet werden.
Hierzu wird exemplarisch die in Abschnitt~\ref{sec:4-open62541} eingeführte Methode \lstinline{PI_writeSingleIO()} untersucht.
Diese Methode ermöglicht das Setzen eines einzelnen Bits im Prozessabbild der SPS, und damit das Schalten eines digitalen Ausgangs auf einem IO-Modul.
Die äquivalente Methode \lstinline{int piControlGetBitValue(SPIValue *pSpiValue)} zum Lesen eines Bits bzw. Eingangs funktioniert analog und soll daher an dieser Stelle nicht dediziert erörtert werden.

\begin{lstlisting}[language={c},firstnumber=97,
                   caption={Setzen eines phsikalischen, digitalen Ausgangs in \lstinline{revpi.c}
                   \label{lst:4-PI_writeSingleIO}}]
extern void PI_writeSingleIO(char *pszVariableName, bool *bit, bool verbose)
{
	int rc;
	SPIVariable sPiVariable;
	SPIValue sPIValue;

	strncpy(sPiVariable.strVarName, pszVariableName, sizeof(sPiVariable.strVarName));
	rc = piControlGetVariableInfo(&sPiVariable);
	if (rc < 0) {
		printf("Cannot find variable '%s'\n", pszVariableName);
		return;
	}

		sPIValue.i16uAddress = sPiVariable.i16uAddress;
		sPIValue.i8uBit = sPiVariable.i8uBit;
		sPIValue.i8uValue = *bit;
		rc = |>\tikzmarkin[set border color=martinired]{setBitValue}<|piControlSetBitValue(&sPIValue)|>\tikzmarkend{setBitValue}<|;
		if (rc < 0)
			printf("Set bit error %s\n", getWriteError(rc));
		else if (verbose)
			printf("Set bit %d on byte at offset %d. Value %d\n", sPIValue.i8uBit, sPIValue.i16uAddress,
			       sPIValue.i8uValue);
}
\end{lstlisting}

Der Programmcode in Listing~\ref{lst:4-PI_writeSingleIO} ist Teil des implementierten OPC-Servers. In diesem wird auf zwei Funktionen des piControl-Treibers zugegriffen. 
Beiden Methoden wird als Argument ein Zeiger auf ein Struct vom Typ \lstinline{SPIValue} übergeben. Der im Struct abgelegte Name wird mittels \lstinline{piControlGetVariableInfo(&sPIValue)} zu einer Adresse im Prozessabbild aufgelöst. Diese wird in \lstinline{sPIValue.i16uAdress} gespeichert. Der Wert der Variablen wird anschließend mittels \lstinline{piControlSetBitValue(&sPIValue)} an dieser Adresse in das Prozessabbild geschrieben.

\begin{lstlisting}[language={c},firstnumber=309,caption={Methode \lstinline{piControlSetBitValue} in \lstinline{piControlIf.c}\label{lst:4-piControlSetBitValue}}]
int |>\tikzmarkin[set border color=martiniblue]{setBitValueFcn}<|piControlSetBitValue(SPIValue *pSpiValue)|>\tikzmarkend{setBitValueFcn}<|
{
    piControlOpen();

    if (PiControlHandle_g < 0)
	    return -ENODEV;

    pSpiValue->i16uAddress += pSpiValue->i8uBit / 8;
    pSpiValue->i8uBit %= 8;

    if (|>\tikzmarkin[set border color=martinired]{ioctl}<|ioctl(PiControlHandle_g, KB_SET_VALUE, pSpiValue)|>\tikzmarkend{ioctl}<| < 0)
	    return errno;

    return 0;
}
\end{lstlisting}

Die in Listing~\ref{lst:4-piControlSetBitValue} dargestellte Methode \lstinline{piControlSetBitValue} ist lediglich eine Hüllfunktion (häufig auch als Wrapper-Funktion bezeichnet) für einen Aufruf des \lstinline{ioctl} Kernel-Moduls.
Folgende Parameter werden übergeben:
\lstinline{PiControlHandle_g} ist die Referenz auf die Geräte-Datei des piControl-Treibers. \lstinline{KB_SET_VALUE} ist das ioctl-Kommando zum Schreiben eines Bits in das Prozessabbild. Der Zeiger \lstinline{pSpiValue} verweist auf ein Struct des bereits vorgestellten Typs \lstinline{SPIValue}.

\begin{lstlisting}[language={c},firstnumber=80,caption={Methode \lstinline{piControlOpen} in \lstinline{piControlIf.c}\label{lst:4-piControlOpen}}]
void piControlOpen(void)
{
    /* open handle if needed */
    if (PiControlHandle_g < 0)
    {
	    |>\tikzmarkin[set border color=martiniblue]{PiControlHandle}<|PiControlHandle_g = open(PICONTROL_DEVICE, O_RDWR)|>\tikzmarkend{PiControlHandle}<|;
    }
}
\end{lstlisting}

Die in Listing~\ref{lst:4-piControlOpen} dargestellte Methode öffnet, sofern nicht bereits geschehen, die Geräte-Datei. Das Macro \lstinline{PICONTROL_DEVICE} verweist hierbei auf \lstinline{/dev/piControl0}.

\begin{lstlisting}[language={c},firstnumber=721,caption={Methode \lstinline{piControlIoctl} in \lstinline{piControlMain.c}\label{lst:4-piControlIoctl}}]
static long |>\tikzmarkin[set border color=martiniblue, below offset=0.9em]{piControlIoctl}<|piControlIoctl(struct file *file, unsigned int prg_nr, 
                           unsigned long usr_addr)                                      |>\tikzmarkend{piControlIoctl}<|
{
  int status = -EFAULT;
  tpiControlInst *priv;
  int timeout = 10000;	// ms

  if (prg_nr != KB_CONFIG_SEND && prg_nr != KB_CONFIG_START && !isRunning()) {
  	return -EAGAIN;
  }

  priv = (tpiControlInst *) file->private_data;

  if (prg_nr != KB_GET_LAST_MESSAGE) {
  	// clear old message
  	priv->pcErrorMessage[0] = 0;
  }

  switch (prg_nr) {|>\setcounter{lstnumber}{864}<|

    case |>\tikzmarkin[set border color=martiniblue]{KB_SET_VALUE}<|KB_SET_VALUE:|>\tikzmarkend{KB_SET_VALUE}<|
  		{
  			SPIValue *pValue = (SPIValue *) usr_addr;

  			if (!isRunning())
  				return -EFAULT;

  			if (pValue->i16uAddress >= KB_PI_LEN) {
  				status = -EFAULT;
  			} else {
  				INT8U i8uValue_l;
  				my_rt_mutex_lock(&piDev_g.lockPI);
  				i8uValue_l = piDev_g.ai8uPI[pValue->i16uAddress];

  				if (pValue->i8uBit >= 8) {
  					i8uValue_l = pValue->i8uValue;
  				} else {
  					if (pValue->i8uValue)
  						i8uValue_l |= (1 << pValue->i8uBit);
  					else
  						i8uValue_l &= ~(1 << pValue->i8uBit);
  				}

  				|>\tikzmarkin[set border color=martinired]{i8uValue}<|piDev_g.ai8uPI[pValue->i16uAddress] = i8uValue_l;|>\tikzmarkend{i8uValue}<|
  				rt_mutex_unlock(&piDev_g.lockPI);

  #ifdef VERBOSE
  				pr_info("piControlIoctl Addr=%u, bit=%u: %02x %02x\n", pValue->i16uAddress, pValue->i8uBit, pValue->i8uValue, i8uValue_l);
  #endif

  				status = 0;
  			}
  		}
  		break; |>\setcounter{lstnumber}{1314}<|

    default:
      pr_err("Invalid Ioctl");
      return (-EINVAL);
      break;

    }

    return status;
  }
\end{lstlisting}

Listing~\ref{lst:4-piControlIoctl} zeigt in Auszügen die ioctl-Methode des piControl Kernel-Treibers. Diese bekommt folgende Argumente übergeben: \lstinline{struct file *file} enthält den Verweis auf die Geräte-Datei, hier \lstinline{/dev/piControl0}. Der Wert von \lstinline{unsigned int prg_nr} beschreibt die Anfrage an den Treiber, in diesem Fall \lstinline{KB_SET_VALUE}. Das Argument \lstinline{unsigned long usr_addr} enthält einen typ-agnostischen Pointer. Dieser verweist auf einen Speicherbereich, in welchem die zur Bearbeitung der Anfrage notwendigen Daten abgelegt sind. Hier können auch vom Treiber empfangene Daten dem Anwendungsprogramm bereitgestellt werden. 

Die switch-case-Anweisung führt die über das Argument \lstinline{prg_nr} spezifizierte Aktion aus. Hier betrachten wir \lstinline{KB_SET_VALUE}:
Zunächst wird in Zeile 868 der übergebene Zeiger \lstinline{usr_addr} mittels explizitem Typecast zu einem Zeiger des Typs \lstinline{SPIValue *} konvertiert. Da dieser auf Daten im Userspace verweist, ist beim Zugriff durch den Kernel-Treiber besondere Vorsicht geboten.
In Zeile 877 wird mittels Mutex das Prozessabbild \lstinline{piDev_g} für den Zugriff durch andere Threads oder Prozesse gesperrt.
\lstinline{my_rt_mutex_lock} verweist hierbei auf die Funktion \lstinline{rt_mutex_lock} aus \lstinline{linux/sched.h}\footnote{Offenbar wurde hier auch eine alternative Implementierung vorgesehen, siehe revpi\_common.h}

In Zeile 889 wird das Byte \lstinline{i8uValue_l}, welches den zu schreibenden Wert enthält in das Prozessabbild übertragen. Anschließend wird die Mutex auf \lstinline{piDev_g} wieder entsperrt.
\newpage

\begin{lstlisting}[language={c},firstnumber=62,caption={Auszug des Struct \lstinline{spiControlDev} in \lstinline{piControlMain.h}\label{lst:4-spiControlDev}}]
|>\tikzmarkin[set border color=martiniblue]{spiControlDev}<|typedef struct spiControlDev|>\tikzmarkend{spiControlDev}<| {
	// device driver stuff
	int init_step;
	enum revpi_machine machine_type;
	void *machine;
	struct cdev cdev;	// Char device structure
	struct device *dev;
	struct thermal_zone_device *thermal_zone;

	|>\tikzmarkin[set border color=martiniblue]{processImage}<|// process image stuff
	INT8U ai8uPI[KB_PI_LEN];
	INT8U ai8uPIDefault|>\tikzmarkin[set border color=martinired]{KB_PI_LEN_0}<|[KB_PI_LEN]|>\tikzmarkend{KB_PI_LEN_0}<|;
	struct rt_mutex lockPI;        |>\tikzmarkend{processImage}<|
	bool stopIO;
	piDevices *devs; |>\setcounter{lstnumber}{94}<|
} tpiControlDev;
\end{lstlisting}

Das Prozessabbild ist als Byte-Array der Länge \lstinline{KB_PI_LEN} in Listing~\ref{lst:4-spiControlDev} definiert. Konfigurationsparameter wie \lstinline{KB_PI_LEN} oder die Zykluszeit für den Datenaustausch zwischen SPS und IO-Modulen sind im folgenden Listing~\ref{lst:4-process} definiert.

\begin{lstlisting}[language={c},firstnumber=119,caption={Konfigurationsparameter des Prozessabbildes in project.h\label{lst:4-process}}]
#define INTERVAL_PI_GATE (5*1000*1000)  // 5 ms piGateCommunication |>\setcounter{lstnumber}{128}<|

#define INTERVAL_IO_COM (5*1000*1000)  // 5 ms piIoComm |>\setcounter{lstnumber}{132}<|

#define KB_PD_LEN       512
|>\tikzmarkin[set border color=martiniblue]{KB_PI_LEN_1}<|#define KB_PI_LEN       4096|>\tikzmarkend{KB_PI_LEN_1}<|
\end{lstlisting}

Das zu setzende Bit wurde zu diesem Zeitpunkt erfolgreich in das Prozessabbild der SPS geschrieben.
Es stellt sich die Frage, wie dieses nun an das IO-Modul kommuniziert wird.
Die Kommunikation mit allen angebundenen Modulen ist ebenfalls Aufgabe des piControl-Treibers.

\begin{lstlisting}[language={c},firstnumber=256,caption={Auszug der Methode \lstinline{piIoThread} in \lstinline{revpi_core.c}\label{lst:4-piIoThread}}]
static int piIoThread(void *data)
{
	//TODO int value = 0;
	ktime_t time;
	ktime_t now;
	s64 tDiff;

	hrtimer_init(&piCore_g.ioTimer, CLOCK_MONOTONIC, HRTIMER_MODE_ABS);
	piCore_g.ioTimer.function = piIoTimer;

	pr_info("piIO thread started\n");

	now = hrtimer_cb_get_time(&piCore_g.ioTimer);

	PiBridgeMaster_Reset();

	while (!kthread_should_stop()) {
		if (|>\tikzmarkin[set border color=martinired]{PiBridgeMaster}<|PiBridgeMaster_Run()|>\tikzmarkend{PiBridgeMaster}<| < 0)
			break;
	}

	RevPiDevice_finish();

	pr_info("piIO exit\n");
	return 0;
}
\end{lstlisting}

Der Kernel-Thread \lstinline{piIoThread} ist verantwortlich für den zyklischen Datenaustausch mit den IO-Modulen. In diesem wird fortlaufend die Methode \lstinline{PiBridgeMaster_Run()} aufgerufen, siehe Listing~\ref{lst:4-piIoThread}.

\begin{lstlisting}[language={c},firstnumber=262,caption={Auszug der Methode \lstinline{PiBridgeMaster_Run(void)} in \lstinline{RevPiDevice.c}\label{lst:4-PiBridgeMaster_Run}}]
int PiBridgeMaster_Run(void)
{
	static kbUT_Timer tTimeoutTimer_s;
	static kbUT_Timer tConfigTimeoutTimer_s;
	static int error_cnt;
	static INT8U last_led;
	static unsigned long last_update;
	int ret = 0;
	int i;

	my_rt_mutex_lock(&piCore_g.lockBridgeState);
	if (piCore_g.eBridgeState != piBridgeStop) {
		switch (eRunStatus_s) { |>\setcounter{lstnumber}{514}<|
		    case enPiBridgeMasterStatus_EndOfConfig:|>\setcounter{lstnumber}{621}<|
		    if (|>\tikzmarkin[set border color=martinired]{RevPiDevice}<|RevPiDevice_run()|>\tikzmarkend{RevPiDevice}<|) {
				// an error occured, check error limits |>\setcounter{lstnumber}{641}<|
			} else {
				ret = 1;
			}
			piCore_g.image.drv.i16uRS485ErrorCnt = RevPiDevice_getErrCnt();
			break;
\end{lstlisting}

Die in Listing~\ref{lst:4-PiBridgeMaster_Run} dargestellte Methode ist eine sog. State-Machine. Ist die Konfiguration der IO-Module erfolgreich abgeschlossen, so führt sie bei Aufruf lediglich die Methode \lstinline{RevPiDevice_run()} aus.

\begin{lstlisting}[language={c},firstnumber=140,caption={Auszug der Methode \lstinline{RevPiDevice_run(void)} in \lstinline{RevPiDevice.c}\label{lst:4-RevPiDevice_run}}]
int RevPiDevice_run(void)
{
	INT8U i8uDevice = 0;
	INT32U r;
	int retval = 0;

	RevPiDevices_s.i16uErrorCnt = 0;

	for (i8uDevice = 0; i8uDevice < RevPiDevice_getDevCnt(); i8uDevice++) {
		if (RevPiDevice_getDev(i8uDevice)->i8uActive) {
			switch (RevPiDevice_getDev(i8uDevice)->sId.i16uModulType) {
			case KUNBUS_FW_DESCR_TYP_PI_DIO_14:
			case KUNBUS_FW_DESCR_TYP_PI_DI_16:
			case KUNBUS_FW_DESCR_TYP_PI_DO_16:
				r = |>\tikzmarkin[set border color=martinired]{sendCyclicTelegram}<|piDIOComm_sendCyclicTelegram(i8uDevice)|>\tikzmarkend{sendCyclicTelegram}\setcounter{lstnumber}{166} <|;

				break; |>\setcounter{lstnumber}{216}<|
			}
		}
	} |>\setcounter{lstnumber}{227}<|
	return retval;
}
\end{lstlisting}

Diese iteriert wie in Listing~\ref{lst:4-RevPiDevice_run} abgebildete durch alle gegenwärtig in der SPS konfigurierten Module. Ist das aktuelle Modul als aktiv markiert, so wird anhand eines sog. Firmware-Descriptors entschieden, welche Methode für die Ansteuerung des Moduls aufzurufen ist.

\begin{lstlisting}[language={c},firstnumber=161,caption={Auszug der Methode \lstinline{piDIOComm_sendCyclicTelegram} in \lstinline{piDIOComm.c}\label{lst:4-sendCyclicTelegram}}]
INT32U piDIOComm_sendCyclicTelegram(INT8U i8uDevice_p)
{
	INT32U i32uRv_l = 0;
	SIOGeneric sRequest_l;
	SIOGeneric sResponse_l;
	INT8U len_l, data_out[18], i, p, data_in[70];
	INT8U i8uAddress;
	int ret; |>\setcounter{lstnumber}{239}<|
	
    |>\tikzmarkin[set border color=martinired]{piIoComm}<|ret = piIoComm_send((INT8U *) & sRequest_l, IOPROTOCOL_HEADER_LENGTH + len_l + 1);  |>\tikzmarkend{piIoComm}\setcounter{lstnumber}{298}<|
}
\end{lstlisting}

Im Falle des hier verwendeten DO-Moduls wird die in Listing~\ref{lst:4-sendCyclicTelegram} abgebildete Methode \lstinline{piDIOComm_sendCyclicTelegram()} aufgerufen. Dieser wird ein Zeiger auf das zu schreibende Gerät übergeben. 
Zunächst wird das Prozessabbild mittels eines proprietären, jedoch im Quellcode offen nachvollziehbaren Protokolls in ein \lstinline{sRequest_l} genanntes Byte-Array umgewandelt. Dieser Schritt ist in Listing~\ref{lst:4-sendCyclicTelegram} nicht abgebildet. Anschließend wird \lstinline{piIoComm_send()} ein Zeiger auf die so generierte Schreib-Anfrage übergeben.

\begin{lstlisting}[language={c},firstnumber=220,caption={Auszug der Methode \lstinline{piIOComm_send} in \lstinline{piIOComm.c}\label{lst:4-piIOComm_send}}]
int piIoComm_send(INT8U * buf_p, INT16U i16uLen_p)
{
	ssize_t write_l = 0;
	INT16U i16uSent_l = 0;|>\setcounter{lstnumber}{249}<|

	while (i16uSent_l < i16uLen_p) {
		write_l = vfs_write(piIoComm_fd_m, buf_p + i16uSent_l, i16uLen_p - i16uSent_l, &piIoComm_fd_m->f_pos);
		if (write_l < 0) {
			pr_info_serial("write error %d\n", (int)write_l);
			return -1;
		} 
		i16uSent_l += write_l;|>\setcounter{lstnumber}{263}<|
	}
	clear();
	vfs_fsync(piIoComm_fd_m, 1);
	return 0;
}
\end{lstlisting}

Listing~\ref{lst:4-piIOComm_send} zeigt die Implementierung von \lstinline{piIoComm_send()}. Diese Methode ist für das Schreiben der oben generierten Anfrage auf die seriellen Schnittstelle verantwortlich. Realisiert wird dies mittels der Methode \lstinline{vfs_write()}. Diese ist in \lstinline{<linux/fs.h>} definiert. Sie ermöglicht das Schreiben einer Datei im Userspace aus dem Kernel heraus. Geschrieben wird hier die Datei mit dem Deskriptor \lstinline{piIoComm_fd_m}.
Da die Funktion \lstinline{vfs_write()} durch andere Kernel-Tasks unterbrochen werden kann, ist nicht gewährleistet, dass die gesamte Anfrage mit nur einem Aufruf geschrieben wird. Die oben abgebildete while-Schleife stellt das vollständige Senden der Anfrage sicher.

\begin{lstlisting}[language={c},firstnumber=157,caption={Auszug der Methode \lstinline{piIOComm_open_serial} in \lstinline{piIOComm.c}\label{lst:4-piIOComm_open_serial}}]
int piIoComm_open_serial(void)
{   |>\setcounter{lstnumber}{167}<|
	struct file *fd;	/* Filedeskriptor */
	struct termios newtio;	/* Schnittstellenoptionen */

	|>\tikzmarkin[set border color=martiniblue]{fd}<|/* Port oeffnen - read/write, kein "controlling tty", 
	    Status von DCD ignorieren */
	fd = filp_open(|>\tikzmarkin[set border color=martinired]{tty}<|REV_PI_TTY_DEVICE|>\tikzmarkend{tty}<|, O_RDWR | O_NOCTTY, 0); |>\setcounter{lstnumber}{208}<|
	
	piIoComm_fd_m = fd;                                                      |>\tikzmarkend{fd}\setcounter{lstnumber}{217}<|

	return 0;
}
\end{lstlisting}

Der zum Schreiben auf die serielle Schnittstelle verwendete Datei-Deskriptor wird von der in Listing~\ref{lst:4-piIOComm_open_serial} abgebildeten Methode \lstinline{piIoComm_open_serial()} generiert. 

\begin{lstlisting}[language={c},firstnumber=45,caption={Definition der seriellen Schnittstelle in \lstinline{piIOComm.h}\label{lst:4-REV_PI_TTY_DEVICE}}]
#define REV_PI_TTY_DEVICE	"/dev/ttyAMA0"
\end{lstlisting}

Das in Listing~\ref{lst:4-REV_PI_TTY_DEVICE} definierte Macro verweist auf eine der seriellen Schnittstellen des RaspberryPi.
Die Implementierung des zugehörigen Schnittstellentreibers soll hier nicht weiter untersucht werden. Somit ist an dieser Stelle die Kette vom Setzen einer Variablen auf dem OPC-Server bis hin zur Aktualisierung des Prozessabbilds der IO-Module geschlossen.

% \begin{lstlisting}[language={c},firstnumber={226},caption={Setzen der Scheduler-Priorität auf SCHED\_FIFO in 
% revpi\_common.c\label{lst:2-sched_priority}}]
% param.sched_priority = ktprio->prio;
% ret = sched_setscheduler(child, SCHED_FIFO, &param);
% \end{lstlisting}
% % % Imports nur für Referenzenauflösung während des Schreibens! Vorm Kompilieren auskommentieren!
% \bibliography{0_hauptdatei}
% \input{1_einleitung}
% \input{2_grundlagen}
% \input{3_konzeption}
% \input{4_implementierung}
% \input{5_tests}
% \input{6_zusammenfassung}
% % Ende Imports

\section{Test des OPC-Servers im Gesamtsystem%
  \label{sec:5-tests}}

% % % Imports nur für Referenzenauflösung während des schreibens! Vorm Kompilieren auskommentieren!
% \bibliography{0_hauptdatei}
% \input{1_einleitung}
% \input{2_grundlagen}
% \input{3_konzeption}
% \input{4_implementierung}
% \input{5_tests}
% \input{6_zusammenfassung}
% % Ende Imports

\section{Zusammenfassung und Ausblick%
  \label{sec:6-fazit}}
Der folgende Abschnitt~\ref{sec:6-zusammenfassung} fasst die gewonnenen Erkenntnisse und den Stand der Implementierung zusammen.
Den Abschluss dieser Arbeit bildet der Ausblick in Abschnitt~\ref{sec:6-ausblick}.

\subsection{Zusammenfassung%
     \label{sec:6-zusammenfassung}}

\subsection{Ausblick%
     \label{sec:6-ausblick}}

% % Ende Imports

\section{Zusammenfassung und Ausblick%
  \label{sec:6-fazit}}
Der folgende Abschnitt~\ref{sec:6-zusammenfassung} fasst die gewonnenen Erkenntnisse und den Stand der Implementierung zusammen.
Den Abschluss dieser Arbeit bildet der Ausblick in Abschnitt~\ref{sec:6-ausblick}.

\subsection{Zusammenfassung%
     \label{sec:6-zusammenfassung}}

\subsection{Ausblick%
     \label{sec:6-ausblick}}

% \input{anhang}
% % Ende Imports

\section{Implementierung%
  \label{sec:4-implementierung}}
Das folgende Kapitel stellt in Auszügen die Implementierung des OPC-Servers sowie die Anbindung an die IO-Module
der SPS dar. Der Schwerpunkt liegt hierbei auf der Funktionsweise des piControl-Treibers und dessen Integration in das Projekt. Abschnitt~\ref{sec:4-picontrol} erklärt die zum Schreibens eines Bits verwendeten Funktionsaufrufe.
Zuvor soll jedoch in Abschnitt~\ref{sec:4-open62541} der Teil des OPC-Servers vorgestellt werden, welcher auf besagten Treiber zugreift. 

\subsection{Implementierung des OPC-Servers%
     \label{sec:4-open62541}}
Wie im vorangegangenen Abschnitt~\ref{sec:3-integration} begründet, soll die Verknüpfung zwischen dem Prozessabbild der SPS und den auf dem OPC-Server bereitgestellten Werten über sog.\,Datenquellen erfolgen. Hierzu ist zunächst eine Callback-Methode zu implementieren, welche bei einem Lese- oder Schreibzugriff auf eine Variable aufgerufen wird. Die Verknüpfung zwischen Callback-Methode und Variable muss manuell erfolgen.

\begin{lstlisting}[language={c},firstnumber=237,caption={Auszug der Methode \lstinline{linkDataSourceVariable} in \lstinline{variables.c}\label{lst:4-linkDataSourceVariable}}]
extern UA_StatusCode
 linkDataSourceVariable(UA_Server *server, UA_NodeId nodeId) {
     bool readonly = false;
     UA_DataSource dataSourceVariable;
     UA_StatusCode rc; |>\setcounter{lstnumber}{254}<|

     dataSourceVariable.read = readDataSourceVariable;
     if (!readonly)
        dataSourceVariable.write = writeDataSourceVariable;
     else
        dataSourceVariable.write = writeReadonlyDataSourceVariable;

     return UA_Server_setVariableNode_dataSource(server, nodeId, dataSourceVariable);
 }
\end{lstlisting}

\begin{figure}[h]
    \centering
    \includegraphics[width=0.42\textwidth]{doc/img/OPC_RevPiDO.pdf}
    \caption{Auszug des verwendeten Nodesets, hier Digitalausgang 1 des Versuchsaufbaus
      \label{fig:opc-do}}
\end{figure}

Die in Listing~\ref{lst:4-linkDataSourceVariable} abgebildete Methode \lstinline{linkDataSourceVariable()} erzeugt ein Struct vom Typ \lstinline{UA_DataSource}. In diesem werden dem Lesen und Schreiben einer OPC-Variablen entsprechende Callback-Methoden zugewiesen. Die Verknüpfung einer OPC-Variable, genauer ihrer NodeId, mit der zuvor definierten Datenquelle erfolgt über die von open62541 bereitgestellte Methode \lstinline{UA_Server_setVariableNode_dataSource()}. Vor dem Lesen und nach dem Schreiben dieser Variable werden von nun an die entsprechenden Callbacks aufgerufen.
     
\begin{lstlisting}[language={c},firstnumber=168,caption={Auszug des Callbacks \lstinline{writeDataSourceVariable} in \lstinline{variables.c}\label{lst:4-writeDataSourceVariable}}]  
extern UA_StatusCode
 writeDataSourceVariable(UA_Server *server,
            const UA_NodeId *sessionId, void *sessionContext,
            const UA_NodeId *nodeId, void *nodeContext,
            const UA_NumericRange *range, const UA_DataValue *dataValue) {

    UA_StatusCode retval  = UA_STATUSCODE_GOOD;
    UA_NodeId *nameNodeId = UA_malloc(sizeof(UA_NodeId));
    UA_QualifiedName nameQN = UA_QUALIFIEDNAME(1, "Name");
    UA_Variant nameVar;
    UA_Boolean bit;

    retval |= findSiblingByBrowsename(server, nodeId, &nameQN, nameNodeId);
    retval |= UA_Server_readValue(server, *nameNodeId, &nameVar);
    retval |= UA_Boolean_copy(dataValue->value.data, &bit);

    |>\tikzmarkin[set border color=martinired]{writeIO}<|PI_writeSingleIO(String_fromUA_String(nameVar.data), &bit, false);                                                 |>\tikzmarkend{writeIO}<|

    free(nameNodeId);
    return retval;
 }
\end{lstlisting}

Listing~\ref{lst:4-writeDataSourceVariable} zeigt die Callback-Methode, welche nach dem Schreiben einer Variablen auf dem OPC-Server aufgerufen wird.
Dieser Methode wird neben der NodeId der mit ihr verknüpften Variablen auch der Wert dieser in Form eines Zeigers auf ein Struct vom Typ \lstinline{UA_DataValue} übergeben.

Die Gestaltung des hier verwendeten Nodesets sieht vor, dass in einer OPC-Variablen \lstinline{"Name"} der Bezeichner des zu schreibenden Digitalausgangs hinterlegt ist, siehe Abbildung~\ref{fig:opc-do}. Dies erlaubt eine Rekonfiguration der Ein- und Ausgänge der SPS ohne Änderungen im Programmcode des OPC-Servers vornehmen zu müssen.
Es ist daher erforderlich, nach jedem Schreiben einer mit einem Digitalausgang verknüpften Variablen, hier \lstinline{"Value"}, dessen Bezeichner \lstinline{"Name"} abzufragen. 
Dies geschieht in den Zeilen 180 und 181.
Anschließend wird dieser Bezeichner sowie der zu schreibende Wert der Methode \lstinline{PI_writeSingleIO()} übergeben, welche wiederum die Interaktion mit piControl übernimmt (vgl. Abschnitt \ref{sec:4-picontrol}).
 
\subsection{Integration von piControl%
     \label{sec:4-picontrol}}
In Abschnitt~\ref{sec:2-io} wurde die Anbindung der IO-Module des Revolution Pi sowie die Funktionsweise von piControl aus Anwendersicht beschrieben. Die verfügbare Literatur beschränkt sich auch auf lediglich diese Sicht; eine weiterführende Dokumentation für Entwickler gibt es, neben der in Abschnitt~\ref{sec:3-anbindung} vorgestellten Manpage, nicht. 
In diesem Abschnitt soll daher der Quellcode von piControl sowie dessen Verwendung im Projekt genauer betrachtet werden.
Hierzu wird exemplarisch die in Abschnitt~\ref{sec:4-open62541} eingeführte Methode \lstinline{PI_writeSingleIO()} untersucht.
Diese Methode ermöglicht das Setzen eines einzelnen Bits im Prozessabbild der SPS, und damit das Schalten eines digitalen Ausgangs auf einem IO-Modul.
Die äquivalente Methode \lstinline{int piControlGetBitValue(SPIValue *pSpiValue)} zum Lesen eines Bits bzw. Eingangs funktioniert analog und soll daher an dieser Stelle nicht dediziert erörtert werden.

\begin{lstlisting}[language={c},firstnumber=97,
                   caption={Setzen eines phsikalischen, digitalen Ausgangs in \lstinline{revpi.c}
                   \label{lst:4-PI_writeSingleIO}}]
extern void PI_writeSingleIO(char *pszVariableName, bool *bit, bool verbose)
{
	int rc;
	SPIVariable sPiVariable;
	SPIValue sPIValue;

	strncpy(sPiVariable.strVarName, pszVariableName, sizeof(sPiVariable.strVarName));
	rc = piControlGetVariableInfo(&sPiVariable);
	if (rc < 0) {
		printf("Cannot find variable '%s'\n", pszVariableName);
		return;
	}

		sPIValue.i16uAddress = sPiVariable.i16uAddress;
		sPIValue.i8uBit = sPiVariable.i8uBit;
		sPIValue.i8uValue = *bit;
		rc = |>\tikzmarkin[set border color=martinired]{setBitValue}<|piControlSetBitValue(&sPIValue)|>\tikzmarkend{setBitValue}<|;
		if (rc < 0)
			printf("Set bit error %s\n", getWriteError(rc));
		else if (verbose)
			printf("Set bit %d on byte at offset %d. Value %d\n", sPIValue.i8uBit, sPIValue.i16uAddress,
			       sPIValue.i8uValue);
}
\end{lstlisting}

Der Programmcode in Listing~\ref{lst:4-PI_writeSingleIO} ist Teil des implementierten OPC-Servers. In diesem wird auf zwei Funktionen des piControl-Treibers zugegriffen. 
Beiden Methoden wird als Argument ein Zeiger auf ein Struct vom Typ \lstinline{SPIValue} übergeben. Der im Struct abgelegte Name wird mittels \lstinline{piControlGetVariableInfo(&sPIValue)} zu einer Adresse im Prozessabbild aufgelöst. Diese wird in \lstinline{sPIValue.i16uAdress} gespeichert. Der Wert der Variablen wird anschließend mittels \lstinline{piControlSetBitValue(&sPIValue)} an dieser Adresse in das Prozessabbild geschrieben.

\begin{lstlisting}[language={c},firstnumber=309,caption={Methode \lstinline{piControlSetBitValue} in \lstinline{piControlIf.c}\label{lst:4-piControlSetBitValue}}]
int |>\tikzmarkin[set border color=martiniblue]{setBitValueFcn}<|piControlSetBitValue(SPIValue *pSpiValue)|>\tikzmarkend{setBitValueFcn}<|
{
    piControlOpen();

    if (PiControlHandle_g < 0)
	    return -ENODEV;

    pSpiValue->i16uAddress += pSpiValue->i8uBit / 8;
    pSpiValue->i8uBit %= 8;

    if (|>\tikzmarkin[set border color=martinired]{ioctl}<|ioctl(PiControlHandle_g, KB_SET_VALUE, pSpiValue)|>\tikzmarkend{ioctl}<| < 0)
	    return errno;

    return 0;
}
\end{lstlisting}

Die in Listing~\ref{lst:4-piControlSetBitValue} dargestellte Methode \lstinline{piControlSetBitValue} ist lediglich eine Hüllfunktion (häufig auch als Wrapper-Funktion bezeichnet) für einen Aufruf des \lstinline{ioctl} Kernel-Moduls.
Folgende Parameter werden übergeben:
\lstinline{PiControlHandle_g} ist die Referenz auf die Geräte-Datei des piControl-Treibers. \lstinline{KB_SET_VALUE} ist das ioctl-Kommando zum Schreiben eines Bits in das Prozessabbild. Der Zeiger \lstinline{pSpiValue} verweist auf ein Struct des bereits vorgestellten Typs \lstinline{SPIValue}.

\begin{lstlisting}[language={c},firstnumber=80,caption={Methode \lstinline{piControlOpen} in \lstinline{piControlIf.c}\label{lst:4-piControlOpen}}]
void piControlOpen(void)
{
    /* open handle if needed */
    if (PiControlHandle_g < 0)
    {
	    |>\tikzmarkin[set border color=martiniblue]{PiControlHandle}<|PiControlHandle_g = open(PICONTROL_DEVICE, O_RDWR)|>\tikzmarkend{PiControlHandle}<|;
    }
}
\end{lstlisting}

Die in Listing~\ref{lst:4-piControlOpen} dargestellte Methode öffnet, sofern nicht bereits geschehen, die Geräte-Datei. Das Macro \lstinline{PICONTROL_DEVICE} verweist hierbei auf \lstinline{/dev/piControl0}.

\begin{lstlisting}[language={c},firstnumber=721,caption={Methode \lstinline{piControlIoctl} in \lstinline{piControlMain.c}\label{lst:4-piControlIoctl}}]
static long |>\tikzmarkin[set border color=martiniblue, below offset=0.9em]{piControlIoctl}<|piControlIoctl(struct file *file, unsigned int prg_nr, 
                           unsigned long usr_addr)                                      |>\tikzmarkend{piControlIoctl}<|
{
  int status = -EFAULT;
  tpiControlInst *priv;
  int timeout = 10000;	// ms

  if (prg_nr != KB_CONFIG_SEND && prg_nr != KB_CONFIG_START && !isRunning()) {
  	return -EAGAIN;
  }

  priv = (tpiControlInst *) file->private_data;

  if (prg_nr != KB_GET_LAST_MESSAGE) {
  	// clear old message
  	priv->pcErrorMessage[0] = 0;
  }

  switch (prg_nr) {|>\setcounter{lstnumber}{864}<|

    case |>\tikzmarkin[set border color=martiniblue]{KB_SET_VALUE}<|KB_SET_VALUE:|>\tikzmarkend{KB_SET_VALUE}<|
  		{
  			SPIValue *pValue = (SPIValue *) usr_addr;

  			if (!isRunning())
  				return -EFAULT;

  			if (pValue->i16uAddress >= KB_PI_LEN) {
  				status = -EFAULT;
  			} else {
  				INT8U i8uValue_l;
  				my_rt_mutex_lock(&piDev_g.lockPI);
  				i8uValue_l = piDev_g.ai8uPI[pValue->i16uAddress];

  				if (pValue->i8uBit >= 8) {
  					i8uValue_l = pValue->i8uValue;
  				} else {
  					if (pValue->i8uValue)
  						i8uValue_l |= (1 << pValue->i8uBit);
  					else
  						i8uValue_l &= ~(1 << pValue->i8uBit);
  				}

  				|>\tikzmarkin[set border color=martinired]{i8uValue}<|piDev_g.ai8uPI[pValue->i16uAddress] = i8uValue_l;|>\tikzmarkend{i8uValue}<|
  				rt_mutex_unlock(&piDev_g.lockPI);

  #ifdef VERBOSE
  				pr_info("piControlIoctl Addr=%u, bit=%u: %02x %02x\n", pValue->i16uAddress, pValue->i8uBit, pValue->i8uValue, i8uValue_l);
  #endif

  				status = 0;
  			}
  		}
  		break; |>\setcounter{lstnumber}{1314}<|

    default:
      pr_err("Invalid Ioctl");
      return (-EINVAL);
      break;

    }

    return status;
  }
\end{lstlisting}

Listing~\ref{lst:4-piControlIoctl} zeigt in Auszügen die ioctl-Methode des piControl Kernel-Treibers. Diese bekommt folgende Argumente übergeben: \lstinline{struct file *file} enthält den Verweis auf die Geräte-Datei, hier \lstinline{/dev/piControl0}. Der Wert von \lstinline{unsigned int prg_nr} beschreibt die Anfrage an den Treiber, in diesem Fall \lstinline{KB_SET_VALUE}. Das Argument \lstinline{unsigned long usr_addr} enthält einen typ-agnostischen Pointer. Dieser verweist auf einen Speicherbereich, in welchem die zur Bearbeitung der Anfrage notwendigen Daten abgelegt sind. Hier können auch vom Treiber empfangene Daten dem Anwendungsprogramm bereitgestellt werden. 

Die switch-case-Anweisung führt die über das Argument \lstinline{prg_nr} spezifizierte Aktion aus. Hier betrachten wir \lstinline{KB_SET_VALUE}:
Zunächst wird in Zeile 868 der übergebene Zeiger \lstinline{usr_addr} mittels explizitem Typecast zu einem Zeiger des Typs \lstinline{SPIValue *} konvertiert. Da dieser auf Daten im Userspace verweist, ist beim Zugriff durch den Kernel-Treiber besondere Vorsicht geboten.
In Zeile 877 wird mittels Mutex das Prozessabbild \lstinline{piDev_g} für den Zugriff durch andere Threads oder Prozesse gesperrt.
\lstinline{my_rt_mutex_lock} verweist hierbei auf die Funktion \lstinline{rt_mutex_lock} aus \lstinline{linux/sched.h}\footnote{Offenbar wurde hier auch eine alternative Implementierung vorgesehen, siehe revpi\_common.h}

In Zeile 889 wird das Byte \lstinline{i8uValue_l}, welches den zu schreibenden Wert enthält in das Prozessabbild übertragen. Anschließend wird die Mutex auf \lstinline{piDev_g} wieder entsperrt.
\newpage

\begin{lstlisting}[language={c},firstnumber=62,caption={Auszug des Struct \lstinline{spiControlDev} in \lstinline{piControlMain.h}\label{lst:4-spiControlDev}}]
|>\tikzmarkin[set border color=martiniblue]{spiControlDev}<|typedef struct spiControlDev|>\tikzmarkend{spiControlDev}<| {
	// device driver stuff
	int init_step;
	enum revpi_machine machine_type;
	void *machine;
	struct cdev cdev;	// Char device structure
	struct device *dev;
	struct thermal_zone_device *thermal_zone;

	|>\tikzmarkin[set border color=martiniblue]{processImage}<|// process image stuff
	INT8U ai8uPI[KB_PI_LEN];
	INT8U ai8uPIDefault|>\tikzmarkin[set border color=martinired]{KB_PI_LEN_0}<|[KB_PI_LEN]|>\tikzmarkend{KB_PI_LEN_0}<|;
	struct rt_mutex lockPI;        |>\tikzmarkend{processImage}<|
	bool stopIO;
	piDevices *devs; |>\setcounter{lstnumber}{94}<|
} tpiControlDev;
\end{lstlisting}

Das Prozessabbild ist als Byte-Array der Länge \lstinline{KB_PI_LEN} in Listing~\ref{lst:4-spiControlDev} definiert. Konfigurationsparameter wie \lstinline{KB_PI_LEN} oder die Zykluszeit für den Datenaustausch zwischen SPS und IO-Modulen sind im folgenden Listing~\ref{lst:4-process} definiert.

\begin{lstlisting}[language={c},firstnumber=119,caption={Konfigurationsparameter des Prozessabbildes in project.h\label{lst:4-process}}]
#define INTERVAL_PI_GATE (5*1000*1000)  // 5 ms piGateCommunication |>\setcounter{lstnumber}{128}<|

#define INTERVAL_IO_COM (5*1000*1000)  // 5 ms piIoComm |>\setcounter{lstnumber}{132}<|

#define KB_PD_LEN       512
|>\tikzmarkin[set border color=martiniblue]{KB_PI_LEN_1}<|#define KB_PI_LEN       4096|>\tikzmarkend{KB_PI_LEN_1}<|
\end{lstlisting}

Das zu setzende Bit wurde zu diesem Zeitpunkt erfolgreich in das Prozessabbild der SPS geschrieben.
Es stellt sich die Frage, wie dieses nun an das IO-Modul kommuniziert wird.
Die Kommunikation mit allen angebundenen Modulen ist ebenfalls Aufgabe des piControl-Treibers.

\begin{lstlisting}[language={c},firstnumber=256,caption={Auszug der Methode \lstinline{piIoThread} in \lstinline{revpi_core.c}\label{lst:4-piIoThread}}]
static int piIoThread(void *data)
{
	//TODO int value = 0;
	ktime_t time;
	ktime_t now;
	s64 tDiff;

	hrtimer_init(&piCore_g.ioTimer, CLOCK_MONOTONIC, HRTIMER_MODE_ABS);
	piCore_g.ioTimer.function = piIoTimer;

	pr_info("piIO thread started\n");

	now = hrtimer_cb_get_time(&piCore_g.ioTimer);

	PiBridgeMaster_Reset();

	while (!kthread_should_stop()) {
		if (|>\tikzmarkin[set border color=martinired]{PiBridgeMaster}<|PiBridgeMaster_Run()|>\tikzmarkend{PiBridgeMaster}<| < 0)
			break;
	}

	RevPiDevice_finish();

	pr_info("piIO exit\n");
	return 0;
}
\end{lstlisting}

Der Kernel-Thread \lstinline{piIoThread} ist verantwortlich für den zyklischen Datenaustausch mit den IO-Modulen. In diesem wird fortlaufend die Methode \lstinline{PiBridgeMaster_Run()} aufgerufen, siehe Listing~\ref{lst:4-piIoThread}.

\begin{lstlisting}[language={c},firstnumber=262,caption={Auszug der Methode \lstinline{PiBridgeMaster_Run(void)} in \lstinline{RevPiDevice.c}\label{lst:4-PiBridgeMaster_Run}}]
int PiBridgeMaster_Run(void)
{
	static kbUT_Timer tTimeoutTimer_s;
	static kbUT_Timer tConfigTimeoutTimer_s;
	static int error_cnt;
	static INT8U last_led;
	static unsigned long last_update;
	int ret = 0;
	int i;

	my_rt_mutex_lock(&piCore_g.lockBridgeState);
	if (piCore_g.eBridgeState != piBridgeStop) {
		switch (eRunStatus_s) { |>\setcounter{lstnumber}{514}<|
		    case enPiBridgeMasterStatus_EndOfConfig:|>\setcounter{lstnumber}{621}<|
		    if (|>\tikzmarkin[set border color=martinired]{RevPiDevice}<|RevPiDevice_run()|>\tikzmarkend{RevPiDevice}<|) {
				// an error occured, check error limits |>\setcounter{lstnumber}{641}<|
			} else {
				ret = 1;
			}
			piCore_g.image.drv.i16uRS485ErrorCnt = RevPiDevice_getErrCnt();
			break;
\end{lstlisting}

Die in Listing~\ref{lst:4-PiBridgeMaster_Run} dargestellte Methode ist eine sog. State-Machine. Ist die Konfiguration der IO-Module erfolgreich abgeschlossen, so führt sie bei Aufruf lediglich die Methode \lstinline{RevPiDevice_run()} aus.

\begin{lstlisting}[language={c},firstnumber=140,caption={Auszug der Methode \lstinline{RevPiDevice_run(void)} in \lstinline{RevPiDevice.c}\label{lst:4-RevPiDevice_run}}]
int RevPiDevice_run(void)
{
	INT8U i8uDevice = 0;
	INT32U r;
	int retval = 0;

	RevPiDevices_s.i16uErrorCnt = 0;

	for (i8uDevice = 0; i8uDevice < RevPiDevice_getDevCnt(); i8uDevice++) {
		if (RevPiDevice_getDev(i8uDevice)->i8uActive) {
			switch (RevPiDevice_getDev(i8uDevice)->sId.i16uModulType) {
			case KUNBUS_FW_DESCR_TYP_PI_DIO_14:
			case KUNBUS_FW_DESCR_TYP_PI_DI_16:
			case KUNBUS_FW_DESCR_TYP_PI_DO_16:
				r = |>\tikzmarkin[set border color=martinired]{sendCyclicTelegram}<|piDIOComm_sendCyclicTelegram(i8uDevice)|>\tikzmarkend{sendCyclicTelegram}\setcounter{lstnumber}{166} <|;

				break; |>\setcounter{lstnumber}{216}<|
			}
		}
	} |>\setcounter{lstnumber}{227}<|
	return retval;
}
\end{lstlisting}

Diese iteriert wie in Listing~\ref{lst:4-RevPiDevice_run} abgebildete durch alle gegenwärtig in der SPS konfigurierten Module. Ist das aktuelle Modul als aktiv markiert, so wird anhand eines sog. Firmware-Descriptors entschieden, welche Methode für die Ansteuerung des Moduls aufzurufen ist.

\begin{lstlisting}[language={c},firstnumber=161,caption={Auszug der Methode \lstinline{piDIOComm_sendCyclicTelegram} in \lstinline{piDIOComm.c}\label{lst:4-sendCyclicTelegram}}]
INT32U piDIOComm_sendCyclicTelegram(INT8U i8uDevice_p)
{
	INT32U i32uRv_l = 0;
	SIOGeneric sRequest_l;
	SIOGeneric sResponse_l;
	INT8U len_l, data_out[18], i, p, data_in[70];
	INT8U i8uAddress;
	int ret; |>\setcounter{lstnumber}{239}<|
	
    |>\tikzmarkin[set border color=martinired]{piIoComm}<|ret = piIoComm_send((INT8U *) & sRequest_l, IOPROTOCOL_HEADER_LENGTH + len_l + 1);  |>\tikzmarkend{piIoComm}\setcounter{lstnumber}{298}<|
}
\end{lstlisting}

Im Falle des hier verwendeten DO-Moduls wird die in Listing~\ref{lst:4-sendCyclicTelegram} abgebildete Methode \lstinline{piDIOComm_sendCyclicTelegram()} aufgerufen. Dieser wird ein Zeiger auf das zu schreibende Gerät übergeben. 
Zunächst wird das Prozessabbild mittels eines proprietären, jedoch im Quellcode offen nachvollziehbaren Protokolls in ein \lstinline{sRequest_l} genanntes Byte-Array umgewandelt. Dieser Schritt ist in Listing~\ref{lst:4-sendCyclicTelegram} nicht abgebildet. Anschließend wird \lstinline{piIoComm_send()} ein Zeiger auf die so generierte Schreib-Anfrage übergeben.

\begin{lstlisting}[language={c},firstnumber=220,caption={Auszug der Methode \lstinline{piIOComm_send} in \lstinline{piIOComm.c}\label{lst:4-piIOComm_send}}]
int piIoComm_send(INT8U * buf_p, INT16U i16uLen_p)
{
	ssize_t write_l = 0;
	INT16U i16uSent_l = 0;|>\setcounter{lstnumber}{249}<|

	while (i16uSent_l < i16uLen_p) {
		write_l = vfs_write(piIoComm_fd_m, buf_p + i16uSent_l, i16uLen_p - i16uSent_l, &piIoComm_fd_m->f_pos);
		if (write_l < 0) {
			pr_info_serial("write error %d\n", (int)write_l);
			return -1;
		} 
		i16uSent_l += write_l;|>\setcounter{lstnumber}{263}<|
	}
	clear();
	vfs_fsync(piIoComm_fd_m, 1);
	return 0;
}
\end{lstlisting}

Listing~\ref{lst:4-piIOComm_send} zeigt die Implementierung von \lstinline{piIoComm_send()}. Diese Methode ist für das Schreiben der oben generierten Anfrage auf die seriellen Schnittstelle verantwortlich. Realisiert wird dies mittels der Methode \lstinline{vfs_write()}. Diese ist in \lstinline{<linux/fs.h>} definiert. Sie ermöglicht das Schreiben einer Datei im Userspace aus dem Kernel heraus. Geschrieben wird hier die Datei mit dem Deskriptor \lstinline{piIoComm_fd_m}.
Da die Funktion \lstinline{vfs_write()} durch andere Kernel-Tasks unterbrochen werden kann, ist nicht gewährleistet, dass die gesamte Anfrage mit nur einem Aufruf geschrieben wird. Die oben abgebildete while-Schleife stellt das vollständige Senden der Anfrage sicher.

\begin{lstlisting}[language={c},firstnumber=157,caption={Auszug der Methode \lstinline{piIOComm_open_serial} in \lstinline{piIOComm.c}\label{lst:4-piIOComm_open_serial}}]
int piIoComm_open_serial(void)
{   |>\setcounter{lstnumber}{167}<|
	struct file *fd;	/* Filedeskriptor */
	struct termios newtio;	/* Schnittstellenoptionen */

	|>\tikzmarkin[set border color=martiniblue]{fd}<|/* Port oeffnen - read/write, kein "controlling tty", 
	    Status von DCD ignorieren */
	fd = filp_open(|>\tikzmarkin[set border color=martinired]{tty}<|REV_PI_TTY_DEVICE|>\tikzmarkend{tty}<|, O_RDWR | O_NOCTTY, 0); |>\setcounter{lstnumber}{208}<|
	
	piIoComm_fd_m = fd;                                                      |>\tikzmarkend{fd}\setcounter{lstnumber}{217}<|

	return 0;
}
\end{lstlisting}

Der zum Schreiben auf die serielle Schnittstelle verwendete Datei-Deskriptor wird von der in Listing~\ref{lst:4-piIOComm_open_serial} abgebildeten Methode \lstinline{piIoComm_open_serial()} generiert. 

\begin{lstlisting}[language={c},firstnumber=45,caption={Definition der seriellen Schnittstelle in \lstinline{piIOComm.h}\label{lst:4-REV_PI_TTY_DEVICE}}]
#define REV_PI_TTY_DEVICE	"/dev/ttyAMA0"
\end{lstlisting}

Das in Listing~\ref{lst:4-REV_PI_TTY_DEVICE} definierte Macro verweist auf eine der seriellen Schnittstellen des RaspberryPi.
Die Implementierung des zugehörigen Schnittstellentreibers soll hier nicht weiter untersucht werden. Somit ist an dieser Stelle die Kette vom Setzen einer Variablen auf dem OPC-Server bis hin zur Aktualisierung des Prozessabbilds der IO-Module geschlossen.

% \begin{lstlisting}[language={c},firstnumber={226},caption={Setzen der Scheduler-Priorität auf SCHED\_FIFO in 
% revpi\_common.c\label{lst:2-sched_priority}}]
% param.sched_priority = ktprio->prio;
% ret = sched_setscheduler(child, SCHED_FIFO, &param);
% \end{lstlisting}
% % % Imports nur für Referenzenauflösung während des Schreibens! Vorm Kompilieren auskommentieren!
% \bibliography{0_hauptdatei}
% % Mit \section{...} eröffnen wir einen neuen Abschnitt.
% Der Befehl setzt nicht nur den Text in einer größeren,
% fetten Schrift, sondern sorgt außerdem dafür, daß er im
% Inhaltsverzeichnis erscheint.
%
% Mit \label{...} erzeugen wir einen Bezeichner, mit dessen Hilfe
% wir später auf die Nummer des Abschnitts verweisen können (nämlich
% mit~\ref{...}).
%
% Das Kommentarzeichen hinter „Übersicht“ dient dazu, ein
% Leerzeichen zwischen „Übersicht“ und dem \label-Befehl
% zu vermeiden, das andernfalls sichtbar würde – z.B. im
% Inhaltsverzeichnis.
%

% % Imports nur für Referenzenauflösung während des Schreibens! Vorm Kompilieren auskommentieren!
% \bibliography{0_hauptdatei}
% % Mit \section{...} eröffnen wir einen neuen Abschnitt.
% Der Befehl setzt nicht nur den Text in einer größeren,
% fetten Schrift, sondern sorgt außerdem dafür, daß er im
% Inhaltsverzeichnis erscheint.
%
% Mit \label{...} erzeugen wir einen Bezeichner, mit dessen Hilfe
% wir später auf die Nummer des Abschnitts verweisen können (nämlich
% mit~\ref{...}).
%
% Das Kommentarzeichen hinter „Übersicht“ dient dazu, ein
% Leerzeichen zwischen „Übersicht“ und dem \label-Befehl
% zu vermeiden, das andernfalls sichtbar würde – z.B. im
% Inhaltsverzeichnis.
%

% % Imports nur für Referenzenauflösung während des Schreibens! Vorm Kompilieren auskommentieren!
% \bibliography{0_hauptdatei}
% \input{1_einleitung}
%\input{2_grundlagen}
%\input{3_konzeption}
%\input{4_implementierung}
%\input{5_tests}
%\input{6_zusammenfassung}
% % Ende Imports

\section{Einleitung und Motivation%
  \label{sec:1-einleitung}}
Ziel dieses Projektes ist die Integration eines OPC-Servers mit einer auf Linux
basierenden speicherprogrammierbaren Steuerung (SPS). Angeschlossen an diese SPS
ist jeweils ein digitales Ein-/\,bzw.~Ausgabemodul. Die von diesen bereitgestellten
Ein-/\, bzw.~Ausgänge (IO) sollen in der Datenstruktur des OPC-Servers abgebildet
und über diesen für OPC-Clients les-/\,und schreibar sein. Weiterhin sollen einige
Funktionen zur Überwachung und Steuerung der an die SPS angeschlossenen Aktoren
und Sensoren direkt im OPC-Server implementiert werden.
Hiermit stellt dieses Projekt eine der Grundlagen für ein übergeordnetes Projekt,
die cloudbasierte Steuerung eines miniaturisierten Produktions-Systems, dar.

Der hier verwendete OPC-Server ist Teil des sog. open62541 Projekts. Er ist in C
geschrieben und implementiert bereits einen großen Teil der im OPC-UA-Standard
spezifizierten Funktionen.
Als SPS findet ein Revolution Pi 3 der Firma Kunbus Verwendung. Dieser integriert
ein sog. Compute Module der Raspberry Pi Foundation in ein industrietaugliches
Gehäuse und erlaubt die Erweiterung mittels IO- oder Gateway-Modulen. Über diese
erfolgt die Kommunikation mit weiteren Komponenten der Automatisierungstechnik.

Motiviert ist dieses Projekt durch die Beobachtung, dass die Verbreitung offener
Standards sowie freier Software auch in der Automatisierungstechnik zunimmt.
Linux ist ein freies Betriebssystem, OPC-UA ein offen zugänglicher, aktiv gepflegter
und weit verbreiteter Standard. Der Raspberry Pi findet sowohl bei Hobby-Anwendern als
auch in den Bereichen Forschung und Entwicklung sowie bei industriellen Anwendern
Verwendung. Dieses Projekt stellt somit eine für unterschiedliche Anwender interessante
Entwicklung dar.

Im Anschluss an diese einleitende Übersicht im Abschnitt~\ref{sec:1-einleitung} folgt
die Darstellung der wichtigsten Grundlagen in Abschnitt~\ref{sec:2-grundlagen}.
Aufbauend auf diesen Grundlagen folgt die konzeptuelle Ausarbeitung im Abschnitt~\ref{sec:3-konzeption}.
Die Umsetzung wird im Abschnitt~\ref{sec:4-implementierung} erläutert.
Die Leistungsfähigkeit der Implementierung wird in Abschnitt~\ref{sec:5-tests} untersucht.
Eine Zusammenfassung und ein Ausblick schließen die Arbeit in
Abschnitt~\ref{sec:6-fazit} ab. Eventuell noch benötigte Anhänge
finden sich in den Anhängen [...] bis [...].

%% % Imports nur für Referenzenauflösung während des Schreibens! Vorm Kompilieren auskommentieren!
% \bibliography{0_hauptdatei}
% \input{1_einleitung}
% \input{2_grundlagen}
% \input{3_konzeption}
% \input{4_implementierung}
% \input{5_tests}
% \input{6_zusammenfassung}
% % Ende Imports

\section{Grundlagen%
  \label{sec:2-grundlagen}}

\subsection{Speicherprogrammierbare-Steuerung und Linux -- Revolution Pi%
     \label{sec:2-sps}}

\subsubsection{Kunbus RevolutionPi%
        \label{sec:2-revpi}}
Der RevolutionPi 3 ist eine speicherprogrammierbare Steuerung (SPS) des Herstellers
Kunbus GmbH. Kern dieser SPS ist das von der Raspberry Pi Foundation entwickelte
und vertriebene Raspberry Pi Compute Module 3. Dieses integriert ein Broadcom BCM2837
System-on-Chip (SoC) mit vier 1,2GHz Prozessorkernen, 1GB RAM, 4GB eMMC Anwendungsspeicher
und sonstige Peripherie in ein Modul im DDR2-SODIMM Formfaktor. Diese Spezifikationen
sind weitgehend identisch zu denen des ausgesprochen populären Raspberry Pi 3.
Der Revolution Pi profitiert daher von dem gleichen großen Angebot an Software
und Unterstützung wie der Raspberry Pi, ergänzt dessen Hardware jedoch um eine 24V
Spannungsversorgung, die Möglichkeit der Erweiterung durch mehrere industrietaugliche
Ein-/ Ausgabemodule und Gateways sowie ein Gehäuse zur Montage auf einer DIN-Schiene.
\begin{itemize}
  \item{Prozessor: BCM2837}
  \item{Taktfrequenz 1,2 GHz}
  \item{Anzahl Prozessorkerne: 4}
  \item{Arbeitsspeicher: 1 GByte}
  \item{eMMC Flash Speicher: 4 GByte}
  \item{Betriebssystem: Angepasstes Raspbian mit RT-Patch}
  \item{RTC mit 24h Pufferung über wartungsfreien Kondensator}
  \item{Treiber / API: Treiber schreibt zyklisch Prozessdaten in ein Prozessabbild, Zugriff auf Prozessabbild über Linux-Filesystem als API zu Fremdsoftware.}
  \item{Kommunikationsanschlüsse: 2 x USB 2.0 A (je 500 mA belastbar), 1 x Micro-USB, HDMI, Ethernet (RJ45) 10/100 Mbit/s}
  \item{Stromversorgung: min. 10,7 V, max. 28,8 V, maximal 10 Watt}
  \item{Zulässige Umgebungstemperatur: -40 bis +55 C}
  \item{Gehäuseabmessungen: (HxBxL) 96 mm x 22,5 mm x 110,5 mm (ohne gesteckte Stecker)}
  \item{ESD Schutz: 4 kV / 8 kV gemäß EN61131-2 und IEC 61000-6-2}
  \item{Surge / Burst Prüfungen: gemäß EN61131-2 und IEC 61000-6-2 eingekoppelt auf Versorgungsspannung, Ethernet und IO-Leitungen}
  \item{EMI Prüfungen: gemäß EN61131-2 und IEC 61000-6-2}
\end{itemize}

Kunbus bietet eine Auswahl an IO- und Gateway-Modulen zur Erweiterung des Revolution Pi an.
Gateways dienen der Kommunikation mit Systemen oder Komponenten der Automatisierungstechnik
über Protokolle wie PROFIBUS oder EtherCAT. IO-Module erlauben die Überwachung
und Steuerung von digitalen oder analogen Ein- und Ausgängen.

\subsubsection{Zugriff auf IO-Module%
        \label{sec:2-io}}
Der Zugriff auf die Ein- und Ausgänge der IO-Module erfolgt über ein Prozessabbild
und einen hierfür von Kunbus bereitgestellten Treiber, genannt piControl. Dieser
aktualisiert das Prozessabbild zyklisch. Die angestrebte Zykluszeit beträgt 5ms,
kann jedoch je nach Anzahl der angeschlossenen Module auch größer sein. Kunbus
garantiert bei drei IO-Modulen und zwei Gateway-Modulen eine Zykluszeit von 10 ms.
Jedes der IO-Module stellt ein eigenständiges eingebettetes System dar. Es verfügt
über einen Microcontroller, welcher die IOs bereitstellt und über einen RS485-Bus
mit dem Revolution Pi kommuniziert.
% https://revolution.kunbus.de/io-modul/

Lizenz: GPL
% https://github.com/RevolutionPi/piControl

\begin{lstlisting}[language={c},firstnumber={226},caption={Setzen der Scheduler-Priorität auf SCHED\_FIFO in revpi\_common.c\label{lst:2-sched_priority}}]
param.sched_priority = ktprio->prio;
ret = sched_setscheduler(child, SCHED_FIFO,
       &param);
\end{lstlisting}


\subsection{Echtzeit und Multithreading unter Linux -- preemptRT und posix%
     \label{sec:2-echtzeit}}


 Der Linux-Kernel verfügt über mehrere unterschiedliche Preemtion-Modelle:

\begin{itemize}
  \item No Forced Preemption (server):
  Ausgelegt auf maximal möglichen Durchsatz, lediglich Interrupts und
  System-Call-Returns bewirken Präemption.

  \item Voluntary Kernel Preemption (Desktop):
  Neben den implizit bevorrechtigten Interrupts und System-Call-Returns gibt es
  in diesem Modell weitere Abschnitte des Kernels in welchen Preämption explizit
  gestattet ist.

  \item Preemptible Kernel (Low-Latency Desktop):
  In diesem Modell ist der gesamte Kernel, mit Ausnahme sog.~kritischer Abschnitte
  präemptible. Nach jedem kritischen Abschnitt gibt es einen impliziten Präemptions-Punkt.

  \item Preemptible Kernel (Basic RT):
  Dieses Modell ist dem zuvor genannten sehr ähnlich, hier sind jedoch alle Interrupt-Handler
  als eigenständige Threads ausgeführt.

  \item Fully Preemptible Kernel (RT):
  Wie auch bei den beiden zuvor genannten Modellen ist hier der gesamte Kernel
  präemtible, die Anzahl und Dauer der nicht-präemtiblen kritischen Abschnitte
  ist auf ein notwendiges Minimum beschränkt. Alle Interrupt-Handler sind als
  eigenständige Threads ausgeführt, Spinlocks durch Sleeping-Spinlocks und Mutexe
  durch sog.~RT-Mutexe ersetzt.

\end{itemize}
\todo{Spinlocks und Mutexe sowie die RT-Varianten dieser erklären!}

Lediglich mit dem vollständig präemtiblen Kernel kann Echtzeit-Verhalten realisiert werden.

% https://wiki.linuxfoundation.org/realtime/documentation/technical_basics/preemption_models bzw kernel/Kconfig.preempt

\subsubsection{preemptRT%
        \label{sec:2-preemptRT}}
% https://wiki.linuxfoundation.org/realtime/documentation/technical_details/start
% https://wiki.linuxfoundation.org/realtime/documentation/technical_basics/start

Das dem PREEMPT RT Kernel zugrunde liegende Prinzip lässt sich in einer einfachen
Regel ausdrücken: Nur Code, welcher absolut nicht-präemtible sein darf, ist es
gestattet nicht-präemtible zu sein.
Das erklärte Ziel des PREEMPT\_RT Patches ist es folglich, die Menge des nicht-präemtiblen
Codes im Linux-Kernel auf das absolut notwendige Minimum zu reduzieren.

Dies wird durch Verwendung folgender Mechanismen erreicht:

\begin{itemize}
  \item Hochauflösende Timer
  \item Sleeping Spinlocks
  \item Threaded Interrupt Handlers
  \item rt\_mutex
  \item RCU
\end{itemize}


\subsubsection{posix%
        \label{sec:2-posix}}
Ist posix hier wirklich relevant? Debian bzw.~Raspbian sind weitgehend posix
kompatibel, aber wird es hier genutzt? -> JA, open62541 nutzt pthread.h
piControl nutzt kthread.h, und semaphore.h

\subsection{OPC-UA und open62541%
     \label{sec:2-opc}}

\subsubsection{OPC UA%
        \label{sec:2-opcua}}
Open Platform Communications (OPC) ist eine Familie von Standards zur herstellerunabhängigen
Kommunikation von Maschinen (M2M) in der Automatisierungstechnik. Die sog.~OPC Task Force, zu deren
Mitgliedern verschiedene große Firmen der Automatisierungsindustrie gehören, veröffentlichte
die OPC Specification Version 1.0 im August 1996.
Motiviert ist dieser offene Standard durch die Erkenntniss, dass die Anpassung der
zahlreichen Herstellerstandards an individuelle Infrastrukturen und Anlagen einen
großen Mehraufwand verursachen.
Die Wikipedia beschreibt das Anwendungsgebiet für OPC wie folgt:

\glqq{}OPC wird dort eingesetzt, wo Sensoren, Regler und Steuerungen verschiedener Hersteller
ein gemeinsames Netzwerk bilden. Ohne OPC benötigten zwei Geräte zum Datenaustausch
genaue Kenntnis über die Kommunikationsmöglichkeiten des Gegenübers. Erweiterungen
und Austausch gestalten sich entsprechend schwierig. Mit OPC genügt es, für jedes
Gerät genau einmal einen OPC-konformen Treiber zu schreiben. Idealerweise wird
dieser bereits vom Hersteller zur Verfügung gestellt. Ein OPC-Treiber lässt sich
ohne großen Anpassungsaufwand in beliebig große Steuer- und Überwachungssysteme
integrieren.

OPC unterteilt sich in verschiedene Unterstandards, die für den jeweiligen Anwendungsfall
unabhängig voneinander implementiert werden können. OPC lässt sich damit verwenden
für Echtzeitdaten (Überwachung), Datenarchivierung, Alarm-Meldungen und neuerdings
auch direkt zur Steuerung (Befehlsübermittlung).\grqq{}

OPC basiert in der ursprünglichen Spezifikation auf Microsofts DCOM-Spezifikation.
DCOM macht Funktionen und Objekte einer Anwendung anderen Anwendungen im Netzwerk
zugänglich. Der OPC-Standard definiert entsprechende DCOM-Objekte um mit anderen
OPC-Anwendungen Daten austauschen zu können. Die Verwendung von DCOM bindet Anwender
an Betriebssysteme von Microsoft. Die ursprüngliche OPC Spezifikation wird durch die
Entwicklung von OPC Unified Architecture (OPC UA) abgelöst.
OPC UA setzt auf einem eigenen Kommunikationionsstack auf, die Verwendung von DCOM
und damit die Bindung an Microsoft wurden aufgelöst.

Die OPC-UA-Architektur ist eine Service-orientierte Architektur (SOA), deren Struktur
aus mehreren Schichten besteht.

% Wikipedia
Das OPC-Informationsmodell ist nicht mehr nur eine Hierarchie aus Ordnern, Items
und Properties. Es ist ein sogenanntes Full-Mesh-Network aus Nodes, mit dem neben
den Nutzdaten eines Nodes auch Meta- und Diagnoseinformationen repräsentiert werden.
Ein Node ähnelt einem Objekt aus der objektorientierten Programmierung. Ein Node
kann Attribute besitzen, die gelesen werden können (Data Access (DA), Historical
Data Access (HDA)). Es ist möglich Methoden zu definieren und aufzurufen.
Eine Methode besitzt Aufrufargumente und Rückgabewerte. Sie wird durch ein Command
aufgerufen. Weiterhin werden Events unterstützt, die versendet werden können
(AE (Alarms \& Events), DA DataChange), um bestimmte Informationen zwischen Geräten
auszutauschen. Ein Event besitzt unter anderem einen Empfangszeitpunkt, eine Nachricht
und einen Schweregrad. Die o. g. Nodes werden sowohl für die Nutzdaten als auch
alle anderen Arten von Metadaten verwendet. Der damit modellierte OPC-Adressraum
beinhaltet nun auch ein Typmodell, mit dem sämtliche Datentypen spezifiziert werden.

% https://de.wikipedia.org/wiki/Open_Platform_Communications
% https://de.wikipedia.org/wiki/OPC_Unified_Architecture
% https://opcfoundation.org/developer-tools/specifications-unified-architecture
% Von Gerhard Gappmeier - ascolab GmbH, CC BY-SA 3.0, https://de.wikipedia.org/w/index.php?curid=1892069
\subsubsection{open62541%
        \label{sec:2-open62541}}
open62541 ist eine offene und freie Implementierung von OPC UA. Die in C geschriebene
Bibliothek stellt eine beständig zunehmende Anzahl der im OPC UA Standard definierten
Funktionen bereit. Sie kann sowohl zur Erstellung von OPC-Servern als auch -Clients
genutzt werden. Ergänzend zu der unter der Mozilla Public License v2.0 lizensierten
Bibliothek stellt das open62541 Projekt auch Beispielprogramme unter einer CC0 Lizenz
zur Verfügung.

Die Bibliothek eignet sich auch für die Entwicklung auf eingebetteten Systemen und
Microcontrollern. Je nach Umfang der gewünschten Funktionen und des OPC Informationsmodells
beträgt die Größe einer Server-Binary weniger als 100kb. %evtl. kürzen?

\todo{Nodes erklären! Evtl.~oben!}

Folgende Auswahl an Eigenschaften und Funktionen zeichnet die in dieser Arbeit verwendete
Version 0.3 von open62541 aus:
\begin{itemize}
  \item Kommunikationionsstack
  \begin{itemize}
      \item OPC UA Binär-Protokoll (HTTP oder SOAP werden gegenwärtig nicht unterstützt)
      \item Austauschbare Netzwerk-Schicht, welche die Verwendung eigener Netzwerk-APIs
      erlaubt.
      \item Verschlüsselte Kommunikationion
      \item Asynchrone Dienst-Anfragen im Client
  \end{itemize}
  \item Informationsmodell
  \begin{itemize}
    \item Unterstützung aller OPC UA Node-Typen, inkl.~Methoden
    \item Hinzufügen und Entfernen von Nodes und Referenzen zur Laufzeit.
    \item Vererbung und Instanziierung von Objekt- und Variablentypen
    \item Zugriffskontrolle auch für einzelne Nodes
  \end{itemize}
  \item Subscriptions
  \begin{itemize}
    \item Erlaubt die Überwachung (subscriptions / monitoreditems)
    \item Sehr geringer Ressourcenbedarf pro überwachtem Wert
  \end{itemize}
  \item Code-Generierung auf XML-Basis
  \begin{itemize}
    \item Erlaubt die Erstellung von Datentypen
    \item Erlaubt die Generierung des serverseitigen Informationsmodells
  \end{itemize}
\end{itemize}

% https://open62541.org/doc/0.3/


Mozilla Public License
CC0 Lizenz für Beispiele und Plugins

% https://open62541.org/doc/open62541-current.pdf
% https://open62541.org/

%% % Imports nur für Referenzenauflösung während des Schreibens! Vorm Kompilieren auskommentieren!
% \bibliography{0_hauptdatei}
% \input{1_einleitung}
% \input{2_grundlagen}
% \input{3_konzeption}
% \input{4_implementierung}
% \input{5_tests}
% \input{6_zusammenfassung}
% \input{anhang}
% % Ende Imports

\section{Systemkonzept%
  \label{sec:3-konzeption}}
Auf Basis der in Abschnitt \ref{sec:2-grundlagen} vorgestellten Möglichkeiten folgt nun die Ausarbeitung eines Konzepts.
In den folgenden Abschnitten soll näher auf zwei zentrale Aspekte eingegangen werden: Abschnitt~\ref{sec:3-anbindung} stellt Möglichkeiten zum Zugriff auf Variablen bzw.\,Werte im Prozessabbild des Revolution Pi vor; in Abschnitt~\ref{sec:3-integration} wird ein Konzept zur Bereitstellung dieser Variablen auf einem OPC-Server vorgestellt.

\subsection{Anbindung der IO an den OPC-Server%
     \label{sec:3-anbindung}}

Eine Webanwendung mit Bezeichnung PiCtory dient zur Konfiguration der I/O- und virtuellen Module des RevolutionPi. Die Konfiguration liegt im JSON-Format in der Datei \lstinline{/etc/revpi/config.rsc}. Der piControl-Treiber liest diese Datei beim Start. 
Der folgende Auszug aus der Manpage des piControl-Kernelmoduls beschreibt die von diesem zum Lesen und Schreiben einzelner Bits des Prozessabbildes bereitgestellten Funktionen~\citep[vgl.]{web-revpi-manpage}. Sie ist an dieser Stelle weitgehend ungekürzt zitiert, da sie die nutzbare Schnittstelle sehr kompakt beschreibt.

\begin{lstlisting}[breakindent=0pt, numbers=none, caption={Auszug aus der Revolution Pi Programmers Manual\label{lst:4-manpage}}]
KB_FIND_VARIABLE SPIVariable *argp
Find a variable in the process image by its name. A pointer to a structure of type SPIVariable must be passed as argument. [...]
The struct SPIVariable [...] is defined as 
typedef struct SPIVariableStr
{
    char strVarName[32]; // Variable name
    uint16_t i16uAddress; // Address of the byte in the process image
    uint8_t i8uBit; // 0-7 bit position, >= 8 whole byte
    uint16_t i16uLength; // length of the variable in bits.
    // Possible values are 1, 8, 16 and 32
} SPIVariable;

Set and get values of the process image
KB_GET_VALUE SPIValue *argp
[...]
KB_SET_VALUE SPIValue *argp
Write one bit or one byte to the process image [...].  This call is more efficient than the usual calls of seek and write because only one function call is necessary. If more than on application are writing bits in one output byte, this call is the only safe way to set a bit without overwriting the other bits because this call is doing a read-modify-write-cycle. 

The struct SPIValue used by this ioctl is defined as
typedef struct SPIValueStr
{
    uint16_t i16uAddress; // Address of the byte in the process image
    uint8_t i8uBit; // 0-7 bit position, >= 8 whole byte
    uint8_t i8uValue; // Value: 0/1 for bit access, whole byte otherwise
} SPIValue;
\end{lstlisting} 

Die oben beschriebenden Funtkionen \lstinline{KB_FIND_VARIABLE}, \lstinline{KB_GET_VALUE} und \lstinline{KB_SET_VALUE} ermöglichen einen einfachen und (lt.\,Manpage) effizienten Zugriff auf einzelne Bits des Prozessabbildes und damit auch auf die IO des RevolutionPi.
Der Zugriff des OPC-Servers auf das Prozessabbild soll daher mittels dieser Funktionen realisiert werden.
\lstinline{KB_FIND_VARIABLE} kann genutzt werden, um Adressen von Variablen im Prozessabbild mittels ihres Namens aufzulösen.
\lstinline{KB_GET_VALUE} und \lstinline{KB_SET_VALUE} ermöglichen den Zugriff auf die Werte dieser Variablen.


\subsection{Integration des OPC-Servers in das System%
     \label{sec:3-integration}}

open62541 bietet drei Möglichkeiten zum Abgleich von Variablen mit dem Prozessabbild~\citep[vgl.][Tutorials - Connecting a Variable with a Physical Process]{web-open62541}:
\begin{itemize}
    \item Manuelles oder zyklisches Aktualisieren
    \item Variable Value Callback
    \item Variable Datasource
\end{itemize}

Die zyklische Aktualisierung eines oder mehrerer Werte nimmt, abhängig von der Zykluszeit, viele Systemressourcen in Anspruch. Value Callbacks ermöglichen es, einen Variablenwert effizienter mit einer Ressource wie etwa einem Prozessabbild zu synchronisieren. An die Variable wird ein Callback angehängt, welches vor jedem Lesen und nach jedem Schreibvorgang ausgeführt wird.
Der Wert der Variablen wird weiterhin im Variablenknoten auf dem OPC-Server gespeichert, der Abgleich mit der verknüpften Ressource erfolgt durch die Callback-Methoden.

Sogenannte Datenquellen gehen noch einen Schritt weiter. Der Server leitet jede Lese- und Schreibanforderung direkt an eine Callback-Funktion weiter. Beim Lesen liefert der Rückruf eine Kopie des aktuellen Wertes. Die Datenquelle muss intern ein eigenes Speichermanagement implementieren.

Der Zugriff auf die Werte des Prozessabbildes erfolgt, wie in Abschnitt~\ref{sec:3-anbindung} beschrieben, über von piControl bereitgestellte Methoden. Um die durch open62541 gepflegte OPC-Datenstruktur und das durch piControl verwaltete Prozessabbild möglichst effektiv verknüpfen zu können, soll diese Interaktion mittels Datenquellen und den zugehörigen Callbacks implementiert werden.
%% % Imports nur für Referenzenauflösung während des Schreibens! Vorm Kompilieren auskommentieren!
% \bibliography{0_hauptdatei}
% \input{1_einleitung}
% \input{2_grundlagen}
% \input{3_konzeption}
% \input{4_implementierung}
% \input{5_tests}
% \input{6_zusammenfassung}
% \input{anhang}
% % Ende Imports

\section{Implementierung%
  \label{sec:4-implementierung}}
Das folgende Kapitel stellt in Auszügen die Implementierung des OPC-Servers sowie die Anbindung an die IO-Module
der SPS dar. Der Schwerpunkt liegt hierbei auf der Funktionsweise des piControl-Treibers und dessen Integration in das Projekt. Abschnitt~\ref{sec:4-picontrol} erklärt die zum Schreibens eines Bits verwendeten Funktionsaufrufe.
Zuvor soll jedoch in Abschnitt~\ref{sec:4-open62541} der Teil des OPC-Servers vorgestellt werden, welcher auf besagten Treiber zugreift. 

\subsection{Implementierung des OPC-Servers%
     \label{sec:4-open62541}}
Wie im vorangegangenen Abschnitt~\ref{sec:3-integration} begründet, soll die Verknüpfung zwischen dem Prozessabbild der SPS und den auf dem OPC-Server bereitgestellten Werten über sog.\,Datenquellen erfolgen. Hierzu ist zunächst eine Callback-Methode zu implementieren, welche bei einem Lese- oder Schreibzugriff auf eine Variable aufgerufen wird. Die Verknüpfung zwischen Callback-Methode und Variable muss manuell erfolgen.

\begin{lstlisting}[language={c},firstnumber=237,caption={Auszug der Methode \lstinline{linkDataSourceVariable} in \lstinline{variables.c}\label{lst:4-linkDataSourceVariable}}]
extern UA_StatusCode
 linkDataSourceVariable(UA_Server *server, UA_NodeId nodeId) {
     bool readonly = false;
     UA_DataSource dataSourceVariable;
     UA_StatusCode rc; |>\setcounter{lstnumber}{254}<|

     dataSourceVariable.read = readDataSourceVariable;
     if (!readonly)
        dataSourceVariable.write = writeDataSourceVariable;
     else
        dataSourceVariable.write = writeReadonlyDataSourceVariable;

     return UA_Server_setVariableNode_dataSource(server, nodeId, dataSourceVariable);
 }
\end{lstlisting}

\begin{figure}[h]
    \centering
    \includegraphics[width=0.42\textwidth]{doc/img/OPC_RevPiDO.pdf}
    \caption{Auszug des verwendeten Nodesets, hier Digitalausgang 1 des Versuchsaufbaus
      \label{fig:opc-do}}
\end{figure}

Die in Listing~\ref{lst:4-linkDataSourceVariable} abgebildete Methode \lstinline{linkDataSourceVariable()} erzeugt ein Struct vom Typ \lstinline{UA_DataSource}. In diesem werden dem Lesen und Schreiben einer OPC-Variablen entsprechende Callback-Methoden zugewiesen. Die Verknüpfung einer OPC-Variable, genauer ihrer NodeId, mit der zuvor definierten Datenquelle erfolgt über die von open62541 bereitgestellte Methode \lstinline{UA_Server_setVariableNode_dataSource()}. Vor dem Lesen und nach dem Schreiben dieser Variable werden von nun an die entsprechenden Callbacks aufgerufen.
     
\begin{lstlisting}[language={c},firstnumber=168,caption={Auszug des Callbacks \lstinline{writeDataSourceVariable} in \lstinline{variables.c}\label{lst:4-writeDataSourceVariable}}]  
extern UA_StatusCode
 writeDataSourceVariable(UA_Server *server,
            const UA_NodeId *sessionId, void *sessionContext,
            const UA_NodeId *nodeId, void *nodeContext,
            const UA_NumericRange *range, const UA_DataValue *dataValue) {

    UA_StatusCode retval  = UA_STATUSCODE_GOOD;
    UA_NodeId *nameNodeId = UA_malloc(sizeof(UA_NodeId));
    UA_QualifiedName nameQN = UA_QUALIFIEDNAME(1, "Name");
    UA_Variant nameVar;
    UA_Boolean bit;

    retval |= findSiblingByBrowsename(server, nodeId, &nameQN, nameNodeId);
    retval |= UA_Server_readValue(server, *nameNodeId, &nameVar);
    retval |= UA_Boolean_copy(dataValue->value.data, &bit);

    |>\tikzmarkin[set border color=martinired]{writeIO}<|PI_writeSingleIO(String_fromUA_String(nameVar.data), &bit, false);                                                 |>\tikzmarkend{writeIO}<|

    free(nameNodeId);
    return retval;
 }
\end{lstlisting}

Listing~\ref{lst:4-writeDataSourceVariable} zeigt die Callback-Methode, welche nach dem Schreiben einer Variablen auf dem OPC-Server aufgerufen wird.
Dieser Methode wird neben der NodeId der mit ihr verknüpften Variablen auch der Wert dieser in Form eines Zeigers auf ein Struct vom Typ \lstinline{UA_DataValue} übergeben.

Die Gestaltung des hier verwendeten Nodesets sieht vor, dass in einer OPC-Variablen \lstinline{"Name"} der Bezeichner des zu schreibenden Digitalausgangs hinterlegt ist, siehe Abbildung~\ref{fig:opc-do}. Dies erlaubt eine Rekonfiguration der Ein- und Ausgänge der SPS ohne Änderungen im Programmcode des OPC-Servers vornehmen zu müssen.
Es ist daher erforderlich, nach jedem Schreiben einer mit einem Digitalausgang verknüpften Variablen, hier \lstinline{"Value"}, dessen Bezeichner \lstinline{"Name"} abzufragen. 
Dies geschieht in den Zeilen 180 und 181.
Anschließend wird dieser Bezeichner sowie der zu schreibende Wert der Methode \lstinline{PI_writeSingleIO()} übergeben, welche wiederum die Interaktion mit piControl übernimmt (vgl. Abschnitt \ref{sec:4-picontrol}).
 
\subsection{Integration von piControl%
     \label{sec:4-picontrol}}
In Abschnitt~\ref{sec:2-io} wurde die Anbindung der IO-Module des Revolution Pi sowie die Funktionsweise von piControl aus Anwendersicht beschrieben. Die verfügbare Literatur beschränkt sich auch auf lediglich diese Sicht; eine weiterführende Dokumentation für Entwickler gibt es, neben der in Abschnitt~\ref{sec:3-anbindung} vorgestellten Manpage, nicht. 
In diesem Abschnitt soll daher der Quellcode von piControl sowie dessen Verwendung im Projekt genauer betrachtet werden.
Hierzu wird exemplarisch die in Abschnitt~\ref{sec:4-open62541} eingeführte Methode \lstinline{PI_writeSingleIO()} untersucht.
Diese Methode ermöglicht das Setzen eines einzelnen Bits im Prozessabbild der SPS, und damit das Schalten eines digitalen Ausgangs auf einem IO-Modul.
Die äquivalente Methode \lstinline{int piControlGetBitValue(SPIValue *pSpiValue)} zum Lesen eines Bits bzw. Eingangs funktioniert analog und soll daher an dieser Stelle nicht dediziert erörtert werden.

\begin{lstlisting}[language={c},firstnumber=97,
                   caption={Setzen eines phsikalischen, digitalen Ausgangs in \lstinline{revpi.c}
                   \label{lst:4-PI_writeSingleIO}}]
extern void PI_writeSingleIO(char *pszVariableName, bool *bit, bool verbose)
{
	int rc;
	SPIVariable sPiVariable;
	SPIValue sPIValue;

	strncpy(sPiVariable.strVarName, pszVariableName, sizeof(sPiVariable.strVarName));
	rc = piControlGetVariableInfo(&sPiVariable);
	if (rc < 0) {
		printf("Cannot find variable '%s'\n", pszVariableName);
		return;
	}

		sPIValue.i16uAddress = sPiVariable.i16uAddress;
		sPIValue.i8uBit = sPiVariable.i8uBit;
		sPIValue.i8uValue = *bit;
		rc = |>\tikzmarkin[set border color=martinired]{setBitValue}<|piControlSetBitValue(&sPIValue)|>\tikzmarkend{setBitValue}<|;
		if (rc < 0)
			printf("Set bit error %s\n", getWriteError(rc));
		else if (verbose)
			printf("Set bit %d on byte at offset %d. Value %d\n", sPIValue.i8uBit, sPIValue.i16uAddress,
			       sPIValue.i8uValue);
}
\end{lstlisting}

Der Programmcode in Listing~\ref{lst:4-PI_writeSingleIO} ist Teil des implementierten OPC-Servers. In diesem wird auf zwei Funktionen des piControl-Treibers zugegriffen. 
Beiden Methoden wird als Argument ein Zeiger auf ein Struct vom Typ \lstinline{SPIValue} übergeben. Der im Struct abgelegte Name wird mittels \lstinline{piControlGetVariableInfo(&sPIValue)} zu einer Adresse im Prozessabbild aufgelöst. Diese wird in \lstinline{sPIValue.i16uAdress} gespeichert. Der Wert der Variablen wird anschließend mittels \lstinline{piControlSetBitValue(&sPIValue)} an dieser Adresse in das Prozessabbild geschrieben.

\begin{lstlisting}[language={c},firstnumber=309,caption={Methode \lstinline{piControlSetBitValue} in \lstinline{piControlIf.c}\label{lst:4-piControlSetBitValue}}]
int |>\tikzmarkin[set border color=martiniblue]{setBitValueFcn}<|piControlSetBitValue(SPIValue *pSpiValue)|>\tikzmarkend{setBitValueFcn}<|
{
    piControlOpen();

    if (PiControlHandle_g < 0)
	    return -ENODEV;

    pSpiValue->i16uAddress += pSpiValue->i8uBit / 8;
    pSpiValue->i8uBit %= 8;

    if (|>\tikzmarkin[set border color=martinired]{ioctl}<|ioctl(PiControlHandle_g, KB_SET_VALUE, pSpiValue)|>\tikzmarkend{ioctl}<| < 0)
	    return errno;

    return 0;
}
\end{lstlisting}

Die in Listing~\ref{lst:4-piControlSetBitValue} dargestellte Methode \lstinline{piControlSetBitValue} ist lediglich eine Hüllfunktion (häufig auch als Wrapper-Funktion bezeichnet) für einen Aufruf des \lstinline{ioctl} Kernel-Moduls.
Folgende Parameter werden übergeben:
\lstinline{PiControlHandle_g} ist die Referenz auf die Geräte-Datei des piControl-Treibers. \lstinline{KB_SET_VALUE} ist das ioctl-Kommando zum Schreiben eines Bits in das Prozessabbild. Der Zeiger \lstinline{pSpiValue} verweist auf ein Struct des bereits vorgestellten Typs \lstinline{SPIValue}.

\begin{lstlisting}[language={c},firstnumber=80,caption={Methode \lstinline{piControlOpen} in \lstinline{piControlIf.c}\label{lst:4-piControlOpen}}]
void piControlOpen(void)
{
    /* open handle if needed */
    if (PiControlHandle_g < 0)
    {
	    |>\tikzmarkin[set border color=martiniblue]{PiControlHandle}<|PiControlHandle_g = open(PICONTROL_DEVICE, O_RDWR)|>\tikzmarkend{PiControlHandle}<|;
    }
}
\end{lstlisting}

Die in Listing~\ref{lst:4-piControlOpen} dargestellte Methode öffnet, sofern nicht bereits geschehen, die Geräte-Datei. Das Macro \lstinline{PICONTROL_DEVICE} verweist hierbei auf \lstinline{/dev/piControl0}.

\begin{lstlisting}[language={c},firstnumber=721,caption={Methode \lstinline{piControlIoctl} in \lstinline{piControlMain.c}\label{lst:4-piControlIoctl}}]
static long |>\tikzmarkin[set border color=martiniblue, below offset=0.9em]{piControlIoctl}<|piControlIoctl(struct file *file, unsigned int prg_nr, 
                           unsigned long usr_addr)                                      |>\tikzmarkend{piControlIoctl}<|
{
  int status = -EFAULT;
  tpiControlInst *priv;
  int timeout = 10000;	// ms

  if (prg_nr != KB_CONFIG_SEND && prg_nr != KB_CONFIG_START && !isRunning()) {
  	return -EAGAIN;
  }

  priv = (tpiControlInst *) file->private_data;

  if (prg_nr != KB_GET_LAST_MESSAGE) {
  	// clear old message
  	priv->pcErrorMessage[0] = 0;
  }

  switch (prg_nr) {|>\setcounter{lstnumber}{864}<|

    case |>\tikzmarkin[set border color=martiniblue]{KB_SET_VALUE}<|KB_SET_VALUE:|>\tikzmarkend{KB_SET_VALUE}<|
  		{
  			SPIValue *pValue = (SPIValue *) usr_addr;

  			if (!isRunning())
  				return -EFAULT;

  			if (pValue->i16uAddress >= KB_PI_LEN) {
  				status = -EFAULT;
  			} else {
  				INT8U i8uValue_l;
  				my_rt_mutex_lock(&piDev_g.lockPI);
  				i8uValue_l = piDev_g.ai8uPI[pValue->i16uAddress];

  				if (pValue->i8uBit >= 8) {
  					i8uValue_l = pValue->i8uValue;
  				} else {
  					if (pValue->i8uValue)
  						i8uValue_l |= (1 << pValue->i8uBit);
  					else
  						i8uValue_l &= ~(1 << pValue->i8uBit);
  				}

  				|>\tikzmarkin[set border color=martinired]{i8uValue}<|piDev_g.ai8uPI[pValue->i16uAddress] = i8uValue_l;|>\tikzmarkend{i8uValue}<|
  				rt_mutex_unlock(&piDev_g.lockPI);

  #ifdef VERBOSE
  				pr_info("piControlIoctl Addr=%u, bit=%u: %02x %02x\n", pValue->i16uAddress, pValue->i8uBit, pValue->i8uValue, i8uValue_l);
  #endif

  				status = 0;
  			}
  		}
  		break; |>\setcounter{lstnumber}{1314}<|

    default:
      pr_err("Invalid Ioctl");
      return (-EINVAL);
      break;

    }

    return status;
  }
\end{lstlisting}

Listing~\ref{lst:4-piControlIoctl} zeigt in Auszügen die ioctl-Methode des piControl Kernel-Treibers. Diese bekommt folgende Argumente übergeben: \lstinline{struct file *file} enthält den Verweis auf die Geräte-Datei, hier \lstinline{/dev/piControl0}. Der Wert von \lstinline{unsigned int prg_nr} beschreibt die Anfrage an den Treiber, in diesem Fall \lstinline{KB_SET_VALUE}. Das Argument \lstinline{unsigned long usr_addr} enthält einen typ-agnostischen Pointer. Dieser verweist auf einen Speicherbereich, in welchem die zur Bearbeitung der Anfrage notwendigen Daten abgelegt sind. Hier können auch vom Treiber empfangene Daten dem Anwendungsprogramm bereitgestellt werden. 

Die switch-case-Anweisung führt die über das Argument \lstinline{prg_nr} spezifizierte Aktion aus. Hier betrachten wir \lstinline{KB_SET_VALUE}:
Zunächst wird in Zeile 868 der übergebene Zeiger \lstinline{usr_addr} mittels explizitem Typecast zu einem Zeiger des Typs \lstinline{SPIValue *} konvertiert. Da dieser auf Daten im Userspace verweist, ist beim Zugriff durch den Kernel-Treiber besondere Vorsicht geboten.
In Zeile 877 wird mittels Mutex das Prozessabbild \lstinline{piDev_g} für den Zugriff durch andere Threads oder Prozesse gesperrt.
\lstinline{my_rt_mutex_lock} verweist hierbei auf die Funktion \lstinline{rt_mutex_lock} aus \lstinline{linux/sched.h}\footnote{Offenbar wurde hier auch eine alternative Implementierung vorgesehen, siehe revpi\_common.h}

In Zeile 889 wird das Byte \lstinline{i8uValue_l}, welches den zu schreibenden Wert enthält in das Prozessabbild übertragen. Anschließend wird die Mutex auf \lstinline{piDev_g} wieder entsperrt.
\newpage

\begin{lstlisting}[language={c},firstnumber=62,caption={Auszug des Struct \lstinline{spiControlDev} in \lstinline{piControlMain.h}\label{lst:4-spiControlDev}}]
|>\tikzmarkin[set border color=martiniblue]{spiControlDev}<|typedef struct spiControlDev|>\tikzmarkend{spiControlDev}<| {
	// device driver stuff
	int init_step;
	enum revpi_machine machine_type;
	void *machine;
	struct cdev cdev;	// Char device structure
	struct device *dev;
	struct thermal_zone_device *thermal_zone;

	|>\tikzmarkin[set border color=martiniblue]{processImage}<|// process image stuff
	INT8U ai8uPI[KB_PI_LEN];
	INT8U ai8uPIDefault|>\tikzmarkin[set border color=martinired]{KB_PI_LEN_0}<|[KB_PI_LEN]|>\tikzmarkend{KB_PI_LEN_0}<|;
	struct rt_mutex lockPI;        |>\tikzmarkend{processImage}<|
	bool stopIO;
	piDevices *devs; |>\setcounter{lstnumber}{94}<|
} tpiControlDev;
\end{lstlisting}

Das Prozessabbild ist als Byte-Array der Länge \lstinline{KB_PI_LEN} in Listing~\ref{lst:4-spiControlDev} definiert. Konfigurationsparameter wie \lstinline{KB_PI_LEN} oder die Zykluszeit für den Datenaustausch zwischen SPS und IO-Modulen sind im folgenden Listing~\ref{lst:4-process} definiert.

\begin{lstlisting}[language={c},firstnumber=119,caption={Konfigurationsparameter des Prozessabbildes in project.h\label{lst:4-process}}]
#define INTERVAL_PI_GATE (5*1000*1000)  // 5 ms piGateCommunication |>\setcounter{lstnumber}{128}<|

#define INTERVAL_IO_COM (5*1000*1000)  // 5 ms piIoComm |>\setcounter{lstnumber}{132}<|

#define KB_PD_LEN       512
|>\tikzmarkin[set border color=martiniblue]{KB_PI_LEN_1}<|#define KB_PI_LEN       4096|>\tikzmarkend{KB_PI_LEN_1}<|
\end{lstlisting}

Das zu setzende Bit wurde zu diesem Zeitpunkt erfolgreich in das Prozessabbild der SPS geschrieben.
Es stellt sich die Frage, wie dieses nun an das IO-Modul kommuniziert wird.
Die Kommunikation mit allen angebundenen Modulen ist ebenfalls Aufgabe des piControl-Treibers.

\begin{lstlisting}[language={c},firstnumber=256,caption={Auszug der Methode \lstinline{piIoThread} in \lstinline{revpi_core.c}\label{lst:4-piIoThread}}]
static int piIoThread(void *data)
{
	//TODO int value = 0;
	ktime_t time;
	ktime_t now;
	s64 tDiff;

	hrtimer_init(&piCore_g.ioTimer, CLOCK_MONOTONIC, HRTIMER_MODE_ABS);
	piCore_g.ioTimer.function = piIoTimer;

	pr_info("piIO thread started\n");

	now = hrtimer_cb_get_time(&piCore_g.ioTimer);

	PiBridgeMaster_Reset();

	while (!kthread_should_stop()) {
		if (|>\tikzmarkin[set border color=martinired]{PiBridgeMaster}<|PiBridgeMaster_Run()|>\tikzmarkend{PiBridgeMaster}<| < 0)
			break;
	}

	RevPiDevice_finish();

	pr_info("piIO exit\n");
	return 0;
}
\end{lstlisting}

Der Kernel-Thread \lstinline{piIoThread} ist verantwortlich für den zyklischen Datenaustausch mit den IO-Modulen. In diesem wird fortlaufend die Methode \lstinline{PiBridgeMaster_Run()} aufgerufen, siehe Listing~\ref{lst:4-piIoThread}.

\begin{lstlisting}[language={c},firstnumber=262,caption={Auszug der Methode \lstinline{PiBridgeMaster_Run(void)} in \lstinline{RevPiDevice.c}\label{lst:4-PiBridgeMaster_Run}}]
int PiBridgeMaster_Run(void)
{
	static kbUT_Timer tTimeoutTimer_s;
	static kbUT_Timer tConfigTimeoutTimer_s;
	static int error_cnt;
	static INT8U last_led;
	static unsigned long last_update;
	int ret = 0;
	int i;

	my_rt_mutex_lock(&piCore_g.lockBridgeState);
	if (piCore_g.eBridgeState != piBridgeStop) {
		switch (eRunStatus_s) { |>\setcounter{lstnumber}{514}<|
		    case enPiBridgeMasterStatus_EndOfConfig:|>\setcounter{lstnumber}{621}<|
		    if (|>\tikzmarkin[set border color=martinired]{RevPiDevice}<|RevPiDevice_run()|>\tikzmarkend{RevPiDevice}<|) {
				// an error occured, check error limits |>\setcounter{lstnumber}{641}<|
			} else {
				ret = 1;
			}
			piCore_g.image.drv.i16uRS485ErrorCnt = RevPiDevice_getErrCnt();
			break;
\end{lstlisting}

Die in Listing~\ref{lst:4-PiBridgeMaster_Run} dargestellte Methode ist eine sog. State-Machine. Ist die Konfiguration der IO-Module erfolgreich abgeschlossen, so führt sie bei Aufruf lediglich die Methode \lstinline{RevPiDevice_run()} aus.

\begin{lstlisting}[language={c},firstnumber=140,caption={Auszug der Methode \lstinline{RevPiDevice_run(void)} in \lstinline{RevPiDevice.c}\label{lst:4-RevPiDevice_run}}]
int RevPiDevice_run(void)
{
	INT8U i8uDevice = 0;
	INT32U r;
	int retval = 0;

	RevPiDevices_s.i16uErrorCnt = 0;

	for (i8uDevice = 0; i8uDevice < RevPiDevice_getDevCnt(); i8uDevice++) {
		if (RevPiDevice_getDev(i8uDevice)->i8uActive) {
			switch (RevPiDevice_getDev(i8uDevice)->sId.i16uModulType) {
			case KUNBUS_FW_DESCR_TYP_PI_DIO_14:
			case KUNBUS_FW_DESCR_TYP_PI_DI_16:
			case KUNBUS_FW_DESCR_TYP_PI_DO_16:
				r = |>\tikzmarkin[set border color=martinired]{sendCyclicTelegram}<|piDIOComm_sendCyclicTelegram(i8uDevice)|>\tikzmarkend{sendCyclicTelegram}\setcounter{lstnumber}{166} <|;

				break; |>\setcounter{lstnumber}{216}<|
			}
		}
	} |>\setcounter{lstnumber}{227}<|
	return retval;
}
\end{lstlisting}

Diese iteriert wie in Listing~\ref{lst:4-RevPiDevice_run} abgebildete durch alle gegenwärtig in der SPS konfigurierten Module. Ist das aktuelle Modul als aktiv markiert, so wird anhand eines sog. Firmware-Descriptors entschieden, welche Methode für die Ansteuerung des Moduls aufzurufen ist.

\begin{lstlisting}[language={c},firstnumber=161,caption={Auszug der Methode \lstinline{piDIOComm_sendCyclicTelegram} in \lstinline{piDIOComm.c}\label{lst:4-sendCyclicTelegram}}]
INT32U piDIOComm_sendCyclicTelegram(INT8U i8uDevice_p)
{
	INT32U i32uRv_l = 0;
	SIOGeneric sRequest_l;
	SIOGeneric sResponse_l;
	INT8U len_l, data_out[18], i, p, data_in[70];
	INT8U i8uAddress;
	int ret; |>\setcounter{lstnumber}{239}<|
	
    |>\tikzmarkin[set border color=martinired]{piIoComm}<|ret = piIoComm_send((INT8U *) & sRequest_l, IOPROTOCOL_HEADER_LENGTH + len_l + 1);  |>\tikzmarkend{piIoComm}\setcounter{lstnumber}{298}<|
}
\end{lstlisting}

Im Falle des hier verwendeten DO-Moduls wird die in Listing~\ref{lst:4-sendCyclicTelegram} abgebildete Methode \lstinline{piDIOComm_sendCyclicTelegram()} aufgerufen. Dieser wird ein Zeiger auf das zu schreibende Gerät übergeben. 
Zunächst wird das Prozessabbild mittels eines proprietären, jedoch im Quellcode offen nachvollziehbaren Protokolls in ein \lstinline{sRequest_l} genanntes Byte-Array umgewandelt. Dieser Schritt ist in Listing~\ref{lst:4-sendCyclicTelegram} nicht abgebildet. Anschließend wird \lstinline{piIoComm_send()} ein Zeiger auf die so generierte Schreib-Anfrage übergeben.

\begin{lstlisting}[language={c},firstnumber=220,caption={Auszug der Methode \lstinline{piIOComm_send} in \lstinline{piIOComm.c}\label{lst:4-piIOComm_send}}]
int piIoComm_send(INT8U * buf_p, INT16U i16uLen_p)
{
	ssize_t write_l = 0;
	INT16U i16uSent_l = 0;|>\setcounter{lstnumber}{249}<|

	while (i16uSent_l < i16uLen_p) {
		write_l = vfs_write(piIoComm_fd_m, buf_p + i16uSent_l, i16uLen_p - i16uSent_l, &piIoComm_fd_m->f_pos);
		if (write_l < 0) {
			pr_info_serial("write error %d\n", (int)write_l);
			return -1;
		} 
		i16uSent_l += write_l;|>\setcounter{lstnumber}{263}<|
	}
	clear();
	vfs_fsync(piIoComm_fd_m, 1);
	return 0;
}
\end{lstlisting}

Listing~\ref{lst:4-piIOComm_send} zeigt die Implementierung von \lstinline{piIoComm_send()}. Diese Methode ist für das Schreiben der oben generierten Anfrage auf die seriellen Schnittstelle verantwortlich. Realisiert wird dies mittels der Methode \lstinline{vfs_write()}. Diese ist in \lstinline{<linux/fs.h>} definiert. Sie ermöglicht das Schreiben einer Datei im Userspace aus dem Kernel heraus. Geschrieben wird hier die Datei mit dem Deskriptor \lstinline{piIoComm_fd_m}.
Da die Funktion \lstinline{vfs_write()} durch andere Kernel-Tasks unterbrochen werden kann, ist nicht gewährleistet, dass die gesamte Anfrage mit nur einem Aufruf geschrieben wird. Die oben abgebildete while-Schleife stellt das vollständige Senden der Anfrage sicher.

\begin{lstlisting}[language={c},firstnumber=157,caption={Auszug der Methode \lstinline{piIOComm_open_serial} in \lstinline{piIOComm.c}\label{lst:4-piIOComm_open_serial}}]
int piIoComm_open_serial(void)
{   |>\setcounter{lstnumber}{167}<|
	struct file *fd;	/* Filedeskriptor */
	struct termios newtio;	/* Schnittstellenoptionen */

	|>\tikzmarkin[set border color=martiniblue]{fd}<|/* Port oeffnen - read/write, kein "controlling tty", 
	    Status von DCD ignorieren */
	fd = filp_open(|>\tikzmarkin[set border color=martinired]{tty}<|REV_PI_TTY_DEVICE|>\tikzmarkend{tty}<|, O_RDWR | O_NOCTTY, 0); |>\setcounter{lstnumber}{208}<|
	
	piIoComm_fd_m = fd;                                                      |>\tikzmarkend{fd}\setcounter{lstnumber}{217}<|

	return 0;
}
\end{lstlisting}

Der zum Schreiben auf die serielle Schnittstelle verwendete Datei-Deskriptor wird von der in Listing~\ref{lst:4-piIOComm_open_serial} abgebildeten Methode \lstinline{piIoComm_open_serial()} generiert. 

\begin{lstlisting}[language={c},firstnumber=45,caption={Definition der seriellen Schnittstelle in \lstinline{piIOComm.h}\label{lst:4-REV_PI_TTY_DEVICE}}]
#define REV_PI_TTY_DEVICE	"/dev/ttyAMA0"
\end{lstlisting}

Das in Listing~\ref{lst:4-REV_PI_TTY_DEVICE} definierte Macro verweist auf eine der seriellen Schnittstellen des RaspberryPi.
Die Implementierung des zugehörigen Schnittstellentreibers soll hier nicht weiter untersucht werden. Somit ist an dieser Stelle die Kette vom Setzen einer Variablen auf dem OPC-Server bis hin zur Aktualisierung des Prozessabbilds der IO-Module geschlossen.

% \begin{lstlisting}[language={c},firstnumber={226},caption={Setzen der Scheduler-Priorität auf SCHED\_FIFO in 
% revpi\_common.c\label{lst:2-sched_priority}}]
% param.sched_priority = ktprio->prio;
% ret = sched_setscheduler(child, SCHED_FIFO, &param);
% \end{lstlisting}
%% % Imports nur für Referenzenauflösung während des Schreibens! Vorm Kompilieren auskommentieren!
% \bibliography{0_hauptdatei}
% \input{1_einleitung}
% \input{2_grundlagen}
% \input{3_konzeption}
% \input{4_implementierung}
% \input{5_tests}
% \input{6_zusammenfassung}
% % Ende Imports

\section{Test des OPC-Servers im Gesamtsystem%
  \label{sec:5-tests}}

%% % Imports nur für Referenzenauflösung während des schreibens! Vorm Kompilieren auskommentieren!
% \bibliography{0_hauptdatei}
% \input{1_einleitung}
% \input{2_grundlagen}
% \input{3_konzeption}
% \input{4_implementierung}
% \input{5_tests}
% \input{6_zusammenfassung}
% % Ende Imports

\section{Zusammenfassung und Ausblick%
  \label{sec:6-fazit}}
Der folgende Abschnitt~\ref{sec:6-zusammenfassung} fasst die gewonnenen Erkenntnisse und den Stand der Implementierung zusammen.
Den Abschluss dieser Arbeit bildet der Ausblick in Abschnitt~\ref{sec:6-ausblick}.

\subsection{Zusammenfassung%
     \label{sec:6-zusammenfassung}}

\subsection{Ausblick%
     \label{sec:6-ausblick}}

% % Ende Imports

\section{Einleitung und Motivation%
  \label{sec:1-einleitung}}
Ziel dieses Projektes ist die Integration eines OPC-Servers mit einer auf Linux
basierenden speicherprogrammierbaren Steuerung (SPS). Angeschlossen an diese SPS
ist jeweils ein digitales Ein-/\,bzw.~Ausgabemodul. Die von diesen bereitgestellten
Ein-/\, bzw.~Ausgänge (IO) sollen in der Datenstruktur des OPC-Servers abgebildet
und über diesen für OPC-Clients les-/\,und schreibar sein. Weiterhin sollen einige
Funktionen zur Überwachung und Steuerung der an die SPS angeschlossenen Aktoren
und Sensoren direkt im OPC-Server implementiert werden.
Hiermit stellt dieses Projekt eine der Grundlagen für ein übergeordnetes Projekt,
die cloudbasierte Steuerung eines miniaturisierten Produktions-Systems, dar.

Der hier verwendete OPC-Server ist Teil des sog. open62541 Projekts. Er ist in C
geschrieben und implementiert bereits einen großen Teil der im OPC-UA-Standard
spezifizierten Funktionen.
Als SPS findet ein Revolution Pi 3 der Firma Kunbus Verwendung. Dieser integriert
ein sog. Compute Module der Raspberry Pi Foundation in ein industrietaugliches
Gehäuse und erlaubt die Erweiterung mittels IO- oder Gateway-Modulen. Über diese
erfolgt die Kommunikation mit weiteren Komponenten der Automatisierungstechnik.

Motiviert ist dieses Projekt durch die Beobachtung, dass die Verbreitung offener
Standards sowie freier Software auch in der Automatisierungstechnik zunimmt.
Linux ist ein freies Betriebssystem, OPC-UA ein offen zugänglicher, aktiv gepflegter
und weit verbreiteter Standard. Der Raspberry Pi findet sowohl bei Hobby-Anwendern als
auch in den Bereichen Forschung und Entwicklung sowie bei industriellen Anwendern
Verwendung. Dieses Projekt stellt somit eine für unterschiedliche Anwender interessante
Entwicklung dar.

Im Anschluss an diese einleitende Übersicht im Abschnitt~\ref{sec:1-einleitung} folgt
die Darstellung der wichtigsten Grundlagen in Abschnitt~\ref{sec:2-grundlagen}.
Aufbauend auf diesen Grundlagen folgt die konzeptuelle Ausarbeitung im Abschnitt~\ref{sec:3-konzeption}.
Die Umsetzung wird im Abschnitt~\ref{sec:4-implementierung} erläutert.
Die Leistungsfähigkeit der Implementierung wird in Abschnitt~\ref{sec:5-tests} untersucht.
Eine Zusammenfassung und ein Ausblick schließen die Arbeit in
Abschnitt~\ref{sec:6-fazit} ab. Eventuell noch benötigte Anhänge
finden sich in den Anhängen [...] bis [...].

% % % Imports nur für Referenzenauflösung während des Schreibens! Vorm Kompilieren auskommentieren!
% \bibliography{0_hauptdatei}
% % Mit \section{...} eröffnen wir einen neuen Abschnitt.
% Der Befehl setzt nicht nur den Text in einer größeren,
% fetten Schrift, sondern sorgt außerdem dafür, daß er im
% Inhaltsverzeichnis erscheint.
%
% Mit \label{...} erzeugen wir einen Bezeichner, mit dessen Hilfe
% wir später auf die Nummer des Abschnitts verweisen können (nämlich
% mit~\ref{...}).
%
% Das Kommentarzeichen hinter „Übersicht“ dient dazu, ein
% Leerzeichen zwischen „Übersicht“ und dem \label-Befehl
% zu vermeiden, das andernfalls sichtbar würde – z.B. im
% Inhaltsverzeichnis.
%

% % Imports nur für Referenzenauflösung während des Schreibens! Vorm Kompilieren auskommentieren!
% \bibliography{0_hauptdatei}
% \input{1_einleitung}
%\input{2_grundlagen}
%\input{3_konzeption}
%\input{4_implementierung}
%\input{5_tests}
%\input{6_zusammenfassung}
% % Ende Imports

\section{Einleitung und Motivation%
  \label{sec:1-einleitung}}
Ziel dieses Projektes ist die Integration eines OPC-Servers mit einer auf Linux
basierenden speicherprogrammierbaren Steuerung (SPS). Angeschlossen an diese SPS
ist jeweils ein digitales Ein-/\,bzw.~Ausgabemodul. Die von diesen bereitgestellten
Ein-/\, bzw.~Ausgänge (IO) sollen in der Datenstruktur des OPC-Servers abgebildet
und über diesen für OPC-Clients les-/\,und schreibar sein. Weiterhin sollen einige
Funktionen zur Überwachung und Steuerung der an die SPS angeschlossenen Aktoren
und Sensoren direkt im OPC-Server implementiert werden.
Hiermit stellt dieses Projekt eine der Grundlagen für ein übergeordnetes Projekt,
die cloudbasierte Steuerung eines miniaturisierten Produktions-Systems, dar.

Der hier verwendete OPC-Server ist Teil des sog. open62541 Projekts. Er ist in C
geschrieben und implementiert bereits einen großen Teil der im OPC-UA-Standard
spezifizierten Funktionen.
Als SPS findet ein Revolution Pi 3 der Firma Kunbus Verwendung. Dieser integriert
ein sog. Compute Module der Raspberry Pi Foundation in ein industrietaugliches
Gehäuse und erlaubt die Erweiterung mittels IO- oder Gateway-Modulen. Über diese
erfolgt die Kommunikation mit weiteren Komponenten der Automatisierungstechnik.

Motiviert ist dieses Projekt durch die Beobachtung, dass die Verbreitung offener
Standards sowie freier Software auch in der Automatisierungstechnik zunimmt.
Linux ist ein freies Betriebssystem, OPC-UA ein offen zugänglicher, aktiv gepflegter
und weit verbreiteter Standard. Der Raspberry Pi findet sowohl bei Hobby-Anwendern als
auch in den Bereichen Forschung und Entwicklung sowie bei industriellen Anwendern
Verwendung. Dieses Projekt stellt somit eine für unterschiedliche Anwender interessante
Entwicklung dar.

Im Anschluss an diese einleitende Übersicht im Abschnitt~\ref{sec:1-einleitung} folgt
die Darstellung der wichtigsten Grundlagen in Abschnitt~\ref{sec:2-grundlagen}.
Aufbauend auf diesen Grundlagen folgt die konzeptuelle Ausarbeitung im Abschnitt~\ref{sec:3-konzeption}.
Die Umsetzung wird im Abschnitt~\ref{sec:4-implementierung} erläutert.
Die Leistungsfähigkeit der Implementierung wird in Abschnitt~\ref{sec:5-tests} untersucht.
Eine Zusammenfassung und ein Ausblick schließen die Arbeit in
Abschnitt~\ref{sec:6-fazit} ab. Eventuell noch benötigte Anhänge
finden sich in den Anhängen [...] bis [...].

% % % Imports nur für Referenzenauflösung während des Schreibens! Vorm Kompilieren auskommentieren!
% \bibliography{0_hauptdatei}
% \input{1_einleitung}
% \input{2_grundlagen}
% \input{3_konzeption}
% \input{4_implementierung}
% \input{5_tests}
% \input{6_zusammenfassung}
% % Ende Imports

\section{Grundlagen%
  \label{sec:2-grundlagen}}

\subsection{Speicherprogrammierbare-Steuerung und Linux -- Revolution Pi%
     \label{sec:2-sps}}

\subsubsection{Kunbus RevolutionPi%
        \label{sec:2-revpi}}
Der RevolutionPi 3 ist eine speicherprogrammierbare Steuerung (SPS) des Herstellers
Kunbus GmbH. Kern dieser SPS ist das von der Raspberry Pi Foundation entwickelte
und vertriebene Raspberry Pi Compute Module 3. Dieses integriert ein Broadcom BCM2837
System-on-Chip (SoC) mit vier 1,2GHz Prozessorkernen, 1GB RAM, 4GB eMMC Anwendungsspeicher
und sonstige Peripherie in ein Modul im DDR2-SODIMM Formfaktor. Diese Spezifikationen
sind weitgehend identisch zu denen des ausgesprochen populären Raspberry Pi 3.
Der Revolution Pi profitiert daher von dem gleichen großen Angebot an Software
und Unterstützung wie der Raspberry Pi, ergänzt dessen Hardware jedoch um eine 24V
Spannungsversorgung, die Möglichkeit der Erweiterung durch mehrere industrietaugliche
Ein-/ Ausgabemodule und Gateways sowie ein Gehäuse zur Montage auf einer DIN-Schiene.
\begin{itemize}
  \item{Prozessor: BCM2837}
  \item{Taktfrequenz 1,2 GHz}
  \item{Anzahl Prozessorkerne: 4}
  \item{Arbeitsspeicher: 1 GByte}
  \item{eMMC Flash Speicher: 4 GByte}
  \item{Betriebssystem: Angepasstes Raspbian mit RT-Patch}
  \item{RTC mit 24h Pufferung über wartungsfreien Kondensator}
  \item{Treiber / API: Treiber schreibt zyklisch Prozessdaten in ein Prozessabbild, Zugriff auf Prozessabbild über Linux-Filesystem als API zu Fremdsoftware.}
  \item{Kommunikationsanschlüsse: 2 x USB 2.0 A (je 500 mA belastbar), 1 x Micro-USB, HDMI, Ethernet (RJ45) 10/100 Mbit/s}
  \item{Stromversorgung: min. 10,7 V, max. 28,8 V, maximal 10 Watt}
  \item{Zulässige Umgebungstemperatur: -40 bis +55 C}
  \item{Gehäuseabmessungen: (HxBxL) 96 mm x 22,5 mm x 110,5 mm (ohne gesteckte Stecker)}
  \item{ESD Schutz: 4 kV / 8 kV gemäß EN61131-2 und IEC 61000-6-2}
  \item{Surge / Burst Prüfungen: gemäß EN61131-2 und IEC 61000-6-2 eingekoppelt auf Versorgungsspannung, Ethernet und IO-Leitungen}
  \item{EMI Prüfungen: gemäß EN61131-2 und IEC 61000-6-2}
\end{itemize}

Kunbus bietet eine Auswahl an IO- und Gateway-Modulen zur Erweiterung des Revolution Pi an.
Gateways dienen der Kommunikation mit Systemen oder Komponenten der Automatisierungstechnik
über Protokolle wie PROFIBUS oder EtherCAT. IO-Module erlauben die Überwachung
und Steuerung von digitalen oder analogen Ein- und Ausgängen.

\subsubsection{Zugriff auf IO-Module%
        \label{sec:2-io}}
Der Zugriff auf die Ein- und Ausgänge der IO-Module erfolgt über ein Prozessabbild
und einen hierfür von Kunbus bereitgestellten Treiber, genannt piControl. Dieser
aktualisiert das Prozessabbild zyklisch. Die angestrebte Zykluszeit beträgt 5ms,
kann jedoch je nach Anzahl der angeschlossenen Module auch größer sein. Kunbus
garantiert bei drei IO-Modulen und zwei Gateway-Modulen eine Zykluszeit von 10 ms.
Jedes der IO-Module stellt ein eigenständiges eingebettetes System dar. Es verfügt
über einen Microcontroller, welcher die IOs bereitstellt und über einen RS485-Bus
mit dem Revolution Pi kommuniziert.
% https://revolution.kunbus.de/io-modul/

Lizenz: GPL
% https://github.com/RevolutionPi/piControl

\begin{lstlisting}[language={c},firstnumber={226},caption={Setzen der Scheduler-Priorität auf SCHED\_FIFO in revpi\_common.c\label{lst:2-sched_priority}}]
param.sched_priority = ktprio->prio;
ret = sched_setscheduler(child, SCHED_FIFO,
       &param);
\end{lstlisting}


\subsection{Echtzeit und Multithreading unter Linux -- preemptRT und posix%
     \label{sec:2-echtzeit}}


 Der Linux-Kernel verfügt über mehrere unterschiedliche Preemtion-Modelle:

\begin{itemize}
  \item No Forced Preemption (server):
  Ausgelegt auf maximal möglichen Durchsatz, lediglich Interrupts und
  System-Call-Returns bewirken Präemption.

  \item Voluntary Kernel Preemption (Desktop):
  Neben den implizit bevorrechtigten Interrupts und System-Call-Returns gibt es
  in diesem Modell weitere Abschnitte des Kernels in welchen Preämption explizit
  gestattet ist.

  \item Preemptible Kernel (Low-Latency Desktop):
  In diesem Modell ist der gesamte Kernel, mit Ausnahme sog.~kritischer Abschnitte
  präemptible. Nach jedem kritischen Abschnitt gibt es einen impliziten Präemptions-Punkt.

  \item Preemptible Kernel (Basic RT):
  Dieses Modell ist dem zuvor genannten sehr ähnlich, hier sind jedoch alle Interrupt-Handler
  als eigenständige Threads ausgeführt.

  \item Fully Preemptible Kernel (RT):
  Wie auch bei den beiden zuvor genannten Modellen ist hier der gesamte Kernel
  präemtible, die Anzahl und Dauer der nicht-präemtiblen kritischen Abschnitte
  ist auf ein notwendiges Minimum beschränkt. Alle Interrupt-Handler sind als
  eigenständige Threads ausgeführt, Spinlocks durch Sleeping-Spinlocks und Mutexe
  durch sog.~RT-Mutexe ersetzt.

\end{itemize}
\todo{Spinlocks und Mutexe sowie die RT-Varianten dieser erklären!}

Lediglich mit dem vollständig präemtiblen Kernel kann Echtzeit-Verhalten realisiert werden.

% https://wiki.linuxfoundation.org/realtime/documentation/technical_basics/preemption_models bzw kernel/Kconfig.preempt

\subsubsection{preemptRT%
        \label{sec:2-preemptRT}}
% https://wiki.linuxfoundation.org/realtime/documentation/technical_details/start
% https://wiki.linuxfoundation.org/realtime/documentation/technical_basics/start

Das dem PREEMPT RT Kernel zugrunde liegende Prinzip lässt sich in einer einfachen
Regel ausdrücken: Nur Code, welcher absolut nicht-präemtible sein darf, ist es
gestattet nicht-präemtible zu sein.
Das erklärte Ziel des PREEMPT\_RT Patches ist es folglich, die Menge des nicht-präemtiblen
Codes im Linux-Kernel auf das absolut notwendige Minimum zu reduzieren.

Dies wird durch Verwendung folgender Mechanismen erreicht:

\begin{itemize}
  \item Hochauflösende Timer
  \item Sleeping Spinlocks
  \item Threaded Interrupt Handlers
  \item rt\_mutex
  \item RCU
\end{itemize}


\subsubsection{posix%
        \label{sec:2-posix}}
Ist posix hier wirklich relevant? Debian bzw.~Raspbian sind weitgehend posix
kompatibel, aber wird es hier genutzt? -> JA, open62541 nutzt pthread.h
piControl nutzt kthread.h, und semaphore.h

\subsection{OPC-UA und open62541%
     \label{sec:2-opc}}

\subsubsection{OPC UA%
        \label{sec:2-opcua}}
Open Platform Communications (OPC) ist eine Familie von Standards zur herstellerunabhängigen
Kommunikation von Maschinen (M2M) in der Automatisierungstechnik. Die sog.~OPC Task Force, zu deren
Mitgliedern verschiedene große Firmen der Automatisierungsindustrie gehören, veröffentlichte
die OPC Specification Version 1.0 im August 1996.
Motiviert ist dieser offene Standard durch die Erkenntniss, dass die Anpassung der
zahlreichen Herstellerstandards an individuelle Infrastrukturen und Anlagen einen
großen Mehraufwand verursachen.
Die Wikipedia beschreibt das Anwendungsgebiet für OPC wie folgt:

\glqq{}OPC wird dort eingesetzt, wo Sensoren, Regler und Steuerungen verschiedener Hersteller
ein gemeinsames Netzwerk bilden. Ohne OPC benötigten zwei Geräte zum Datenaustausch
genaue Kenntnis über die Kommunikationsmöglichkeiten des Gegenübers. Erweiterungen
und Austausch gestalten sich entsprechend schwierig. Mit OPC genügt es, für jedes
Gerät genau einmal einen OPC-konformen Treiber zu schreiben. Idealerweise wird
dieser bereits vom Hersteller zur Verfügung gestellt. Ein OPC-Treiber lässt sich
ohne großen Anpassungsaufwand in beliebig große Steuer- und Überwachungssysteme
integrieren.

OPC unterteilt sich in verschiedene Unterstandards, die für den jeweiligen Anwendungsfall
unabhängig voneinander implementiert werden können. OPC lässt sich damit verwenden
für Echtzeitdaten (Überwachung), Datenarchivierung, Alarm-Meldungen und neuerdings
auch direkt zur Steuerung (Befehlsübermittlung).\grqq{}

OPC basiert in der ursprünglichen Spezifikation auf Microsofts DCOM-Spezifikation.
DCOM macht Funktionen und Objekte einer Anwendung anderen Anwendungen im Netzwerk
zugänglich. Der OPC-Standard definiert entsprechende DCOM-Objekte um mit anderen
OPC-Anwendungen Daten austauschen zu können. Die Verwendung von DCOM bindet Anwender
an Betriebssysteme von Microsoft. Die ursprüngliche OPC Spezifikation wird durch die
Entwicklung von OPC Unified Architecture (OPC UA) abgelöst.
OPC UA setzt auf einem eigenen Kommunikationionsstack auf, die Verwendung von DCOM
und damit die Bindung an Microsoft wurden aufgelöst.

Die OPC-UA-Architektur ist eine Service-orientierte Architektur (SOA), deren Struktur
aus mehreren Schichten besteht.

% Wikipedia
Das OPC-Informationsmodell ist nicht mehr nur eine Hierarchie aus Ordnern, Items
und Properties. Es ist ein sogenanntes Full-Mesh-Network aus Nodes, mit dem neben
den Nutzdaten eines Nodes auch Meta- und Diagnoseinformationen repräsentiert werden.
Ein Node ähnelt einem Objekt aus der objektorientierten Programmierung. Ein Node
kann Attribute besitzen, die gelesen werden können (Data Access (DA), Historical
Data Access (HDA)). Es ist möglich Methoden zu definieren und aufzurufen.
Eine Methode besitzt Aufrufargumente und Rückgabewerte. Sie wird durch ein Command
aufgerufen. Weiterhin werden Events unterstützt, die versendet werden können
(AE (Alarms \& Events), DA DataChange), um bestimmte Informationen zwischen Geräten
auszutauschen. Ein Event besitzt unter anderem einen Empfangszeitpunkt, eine Nachricht
und einen Schweregrad. Die o. g. Nodes werden sowohl für die Nutzdaten als auch
alle anderen Arten von Metadaten verwendet. Der damit modellierte OPC-Adressraum
beinhaltet nun auch ein Typmodell, mit dem sämtliche Datentypen spezifiziert werden.

% https://de.wikipedia.org/wiki/Open_Platform_Communications
% https://de.wikipedia.org/wiki/OPC_Unified_Architecture
% https://opcfoundation.org/developer-tools/specifications-unified-architecture
% Von Gerhard Gappmeier - ascolab GmbH, CC BY-SA 3.0, https://de.wikipedia.org/w/index.php?curid=1892069
\subsubsection{open62541%
        \label{sec:2-open62541}}
open62541 ist eine offene und freie Implementierung von OPC UA. Die in C geschriebene
Bibliothek stellt eine beständig zunehmende Anzahl der im OPC UA Standard definierten
Funktionen bereit. Sie kann sowohl zur Erstellung von OPC-Servern als auch -Clients
genutzt werden. Ergänzend zu der unter der Mozilla Public License v2.0 lizensierten
Bibliothek stellt das open62541 Projekt auch Beispielprogramme unter einer CC0 Lizenz
zur Verfügung.

Die Bibliothek eignet sich auch für die Entwicklung auf eingebetteten Systemen und
Microcontrollern. Je nach Umfang der gewünschten Funktionen und des OPC Informationsmodells
beträgt die Größe einer Server-Binary weniger als 100kb. %evtl. kürzen?

\todo{Nodes erklären! Evtl.~oben!}

Folgende Auswahl an Eigenschaften und Funktionen zeichnet die in dieser Arbeit verwendete
Version 0.3 von open62541 aus:
\begin{itemize}
  \item Kommunikationionsstack
  \begin{itemize}
      \item OPC UA Binär-Protokoll (HTTP oder SOAP werden gegenwärtig nicht unterstützt)
      \item Austauschbare Netzwerk-Schicht, welche die Verwendung eigener Netzwerk-APIs
      erlaubt.
      \item Verschlüsselte Kommunikationion
      \item Asynchrone Dienst-Anfragen im Client
  \end{itemize}
  \item Informationsmodell
  \begin{itemize}
    \item Unterstützung aller OPC UA Node-Typen, inkl.~Methoden
    \item Hinzufügen und Entfernen von Nodes und Referenzen zur Laufzeit.
    \item Vererbung und Instanziierung von Objekt- und Variablentypen
    \item Zugriffskontrolle auch für einzelne Nodes
  \end{itemize}
  \item Subscriptions
  \begin{itemize}
    \item Erlaubt die Überwachung (subscriptions / monitoreditems)
    \item Sehr geringer Ressourcenbedarf pro überwachtem Wert
  \end{itemize}
  \item Code-Generierung auf XML-Basis
  \begin{itemize}
    \item Erlaubt die Erstellung von Datentypen
    \item Erlaubt die Generierung des serverseitigen Informationsmodells
  \end{itemize}
\end{itemize}

% https://open62541.org/doc/0.3/


Mozilla Public License
CC0 Lizenz für Beispiele und Plugins

% https://open62541.org/doc/open62541-current.pdf
% https://open62541.org/

% % % Imports nur für Referenzenauflösung während des Schreibens! Vorm Kompilieren auskommentieren!
% \bibliography{0_hauptdatei}
% \input{1_einleitung}
% \input{2_grundlagen}
% \input{3_konzeption}
% \input{4_implementierung}
% \input{5_tests}
% \input{6_zusammenfassung}
% \input{anhang}
% % Ende Imports

\section{Systemkonzept%
  \label{sec:3-konzeption}}
Auf Basis der in Abschnitt \ref{sec:2-grundlagen} vorgestellten Möglichkeiten folgt nun die Ausarbeitung eines Konzepts.
In den folgenden Abschnitten soll näher auf zwei zentrale Aspekte eingegangen werden: Abschnitt~\ref{sec:3-anbindung} stellt Möglichkeiten zum Zugriff auf Variablen bzw.\,Werte im Prozessabbild des Revolution Pi vor; in Abschnitt~\ref{sec:3-integration} wird ein Konzept zur Bereitstellung dieser Variablen auf einem OPC-Server vorgestellt.

\subsection{Anbindung der IO an den OPC-Server%
     \label{sec:3-anbindung}}

Eine Webanwendung mit Bezeichnung PiCtory dient zur Konfiguration der I/O- und virtuellen Module des RevolutionPi. Die Konfiguration liegt im JSON-Format in der Datei \lstinline{/etc/revpi/config.rsc}. Der piControl-Treiber liest diese Datei beim Start. 
Der folgende Auszug aus der Manpage des piControl-Kernelmoduls beschreibt die von diesem zum Lesen und Schreiben einzelner Bits des Prozessabbildes bereitgestellten Funktionen~\citep[vgl.]{web-revpi-manpage}. Sie ist an dieser Stelle weitgehend ungekürzt zitiert, da sie die nutzbare Schnittstelle sehr kompakt beschreibt.

\begin{lstlisting}[breakindent=0pt, numbers=none, caption={Auszug aus der Revolution Pi Programmers Manual\label{lst:4-manpage}}]
KB_FIND_VARIABLE SPIVariable *argp
Find a variable in the process image by its name. A pointer to a structure of type SPIVariable must be passed as argument. [...]
The struct SPIVariable [...] is defined as 
typedef struct SPIVariableStr
{
    char strVarName[32]; // Variable name
    uint16_t i16uAddress; // Address of the byte in the process image
    uint8_t i8uBit; // 0-7 bit position, >= 8 whole byte
    uint16_t i16uLength; // length of the variable in bits.
    // Possible values are 1, 8, 16 and 32
} SPIVariable;

Set and get values of the process image
KB_GET_VALUE SPIValue *argp
[...]
KB_SET_VALUE SPIValue *argp
Write one bit or one byte to the process image [...].  This call is more efficient than the usual calls of seek and write because only one function call is necessary. If more than on application are writing bits in one output byte, this call is the only safe way to set a bit without overwriting the other bits because this call is doing a read-modify-write-cycle. 

The struct SPIValue used by this ioctl is defined as
typedef struct SPIValueStr
{
    uint16_t i16uAddress; // Address of the byte in the process image
    uint8_t i8uBit; // 0-7 bit position, >= 8 whole byte
    uint8_t i8uValue; // Value: 0/1 for bit access, whole byte otherwise
} SPIValue;
\end{lstlisting} 

Die oben beschriebenden Funtkionen \lstinline{KB_FIND_VARIABLE}, \lstinline{KB_GET_VALUE} und \lstinline{KB_SET_VALUE} ermöglichen einen einfachen und (lt.\,Manpage) effizienten Zugriff auf einzelne Bits des Prozessabbildes und damit auch auf die IO des RevolutionPi.
Der Zugriff des OPC-Servers auf das Prozessabbild soll daher mittels dieser Funktionen realisiert werden.
\lstinline{KB_FIND_VARIABLE} kann genutzt werden, um Adressen von Variablen im Prozessabbild mittels ihres Namens aufzulösen.
\lstinline{KB_GET_VALUE} und \lstinline{KB_SET_VALUE} ermöglichen den Zugriff auf die Werte dieser Variablen.


\subsection{Integration des OPC-Servers in das System%
     \label{sec:3-integration}}

open62541 bietet drei Möglichkeiten zum Abgleich von Variablen mit dem Prozessabbild~\citep[vgl.][Tutorials - Connecting a Variable with a Physical Process]{web-open62541}:
\begin{itemize}
    \item Manuelles oder zyklisches Aktualisieren
    \item Variable Value Callback
    \item Variable Datasource
\end{itemize}

Die zyklische Aktualisierung eines oder mehrerer Werte nimmt, abhängig von der Zykluszeit, viele Systemressourcen in Anspruch. Value Callbacks ermöglichen es, einen Variablenwert effizienter mit einer Ressource wie etwa einem Prozessabbild zu synchronisieren. An die Variable wird ein Callback angehängt, welches vor jedem Lesen und nach jedem Schreibvorgang ausgeführt wird.
Der Wert der Variablen wird weiterhin im Variablenknoten auf dem OPC-Server gespeichert, der Abgleich mit der verknüpften Ressource erfolgt durch die Callback-Methoden.

Sogenannte Datenquellen gehen noch einen Schritt weiter. Der Server leitet jede Lese- und Schreibanforderung direkt an eine Callback-Funktion weiter. Beim Lesen liefert der Rückruf eine Kopie des aktuellen Wertes. Die Datenquelle muss intern ein eigenes Speichermanagement implementieren.

Der Zugriff auf die Werte des Prozessabbildes erfolgt, wie in Abschnitt~\ref{sec:3-anbindung} beschrieben, über von piControl bereitgestellte Methoden. Um die durch open62541 gepflegte OPC-Datenstruktur und das durch piControl verwaltete Prozessabbild möglichst effektiv verknüpfen zu können, soll diese Interaktion mittels Datenquellen und den zugehörigen Callbacks implementiert werden.
% % % Imports nur für Referenzenauflösung während des Schreibens! Vorm Kompilieren auskommentieren!
% \bibliography{0_hauptdatei}
% \input{1_einleitung}
% \input{2_grundlagen}
% \input{3_konzeption}
% \input{4_implementierung}
% \input{5_tests}
% \input{6_zusammenfassung}
% \input{anhang}
% % Ende Imports

\section{Implementierung%
  \label{sec:4-implementierung}}
Das folgende Kapitel stellt in Auszügen die Implementierung des OPC-Servers sowie die Anbindung an die IO-Module
der SPS dar. Der Schwerpunkt liegt hierbei auf der Funktionsweise des piControl-Treibers und dessen Integration in das Projekt. Abschnitt~\ref{sec:4-picontrol} erklärt die zum Schreibens eines Bits verwendeten Funktionsaufrufe.
Zuvor soll jedoch in Abschnitt~\ref{sec:4-open62541} der Teil des OPC-Servers vorgestellt werden, welcher auf besagten Treiber zugreift. 

\subsection{Implementierung des OPC-Servers%
     \label{sec:4-open62541}}
Wie im vorangegangenen Abschnitt~\ref{sec:3-integration} begründet, soll die Verknüpfung zwischen dem Prozessabbild der SPS und den auf dem OPC-Server bereitgestellten Werten über sog.\,Datenquellen erfolgen. Hierzu ist zunächst eine Callback-Methode zu implementieren, welche bei einem Lese- oder Schreibzugriff auf eine Variable aufgerufen wird. Die Verknüpfung zwischen Callback-Methode und Variable muss manuell erfolgen.

\begin{lstlisting}[language={c},firstnumber=237,caption={Auszug der Methode \lstinline{linkDataSourceVariable} in \lstinline{variables.c}\label{lst:4-linkDataSourceVariable}}]
extern UA_StatusCode
 linkDataSourceVariable(UA_Server *server, UA_NodeId nodeId) {
     bool readonly = false;
     UA_DataSource dataSourceVariable;
     UA_StatusCode rc; |>\setcounter{lstnumber}{254}<|

     dataSourceVariable.read = readDataSourceVariable;
     if (!readonly)
        dataSourceVariable.write = writeDataSourceVariable;
     else
        dataSourceVariable.write = writeReadonlyDataSourceVariable;

     return UA_Server_setVariableNode_dataSource(server, nodeId, dataSourceVariable);
 }
\end{lstlisting}

\begin{figure}[h]
    \centering
    \includegraphics[width=0.42\textwidth]{doc/img/OPC_RevPiDO.pdf}
    \caption{Auszug des verwendeten Nodesets, hier Digitalausgang 1 des Versuchsaufbaus
      \label{fig:opc-do}}
\end{figure}

Die in Listing~\ref{lst:4-linkDataSourceVariable} abgebildete Methode \lstinline{linkDataSourceVariable()} erzeugt ein Struct vom Typ \lstinline{UA_DataSource}. In diesem werden dem Lesen und Schreiben einer OPC-Variablen entsprechende Callback-Methoden zugewiesen. Die Verknüpfung einer OPC-Variable, genauer ihrer NodeId, mit der zuvor definierten Datenquelle erfolgt über die von open62541 bereitgestellte Methode \lstinline{UA_Server_setVariableNode_dataSource()}. Vor dem Lesen und nach dem Schreiben dieser Variable werden von nun an die entsprechenden Callbacks aufgerufen.
     
\begin{lstlisting}[language={c},firstnumber=168,caption={Auszug des Callbacks \lstinline{writeDataSourceVariable} in \lstinline{variables.c}\label{lst:4-writeDataSourceVariable}}]  
extern UA_StatusCode
 writeDataSourceVariable(UA_Server *server,
            const UA_NodeId *sessionId, void *sessionContext,
            const UA_NodeId *nodeId, void *nodeContext,
            const UA_NumericRange *range, const UA_DataValue *dataValue) {

    UA_StatusCode retval  = UA_STATUSCODE_GOOD;
    UA_NodeId *nameNodeId = UA_malloc(sizeof(UA_NodeId));
    UA_QualifiedName nameQN = UA_QUALIFIEDNAME(1, "Name");
    UA_Variant nameVar;
    UA_Boolean bit;

    retval |= findSiblingByBrowsename(server, nodeId, &nameQN, nameNodeId);
    retval |= UA_Server_readValue(server, *nameNodeId, &nameVar);
    retval |= UA_Boolean_copy(dataValue->value.data, &bit);

    |>\tikzmarkin[set border color=martinired]{writeIO}<|PI_writeSingleIO(String_fromUA_String(nameVar.data), &bit, false);                                                 |>\tikzmarkend{writeIO}<|

    free(nameNodeId);
    return retval;
 }
\end{lstlisting}

Listing~\ref{lst:4-writeDataSourceVariable} zeigt die Callback-Methode, welche nach dem Schreiben einer Variablen auf dem OPC-Server aufgerufen wird.
Dieser Methode wird neben der NodeId der mit ihr verknüpften Variablen auch der Wert dieser in Form eines Zeigers auf ein Struct vom Typ \lstinline{UA_DataValue} übergeben.

Die Gestaltung des hier verwendeten Nodesets sieht vor, dass in einer OPC-Variablen \lstinline{"Name"} der Bezeichner des zu schreibenden Digitalausgangs hinterlegt ist, siehe Abbildung~\ref{fig:opc-do}. Dies erlaubt eine Rekonfiguration der Ein- und Ausgänge der SPS ohne Änderungen im Programmcode des OPC-Servers vornehmen zu müssen.
Es ist daher erforderlich, nach jedem Schreiben einer mit einem Digitalausgang verknüpften Variablen, hier \lstinline{"Value"}, dessen Bezeichner \lstinline{"Name"} abzufragen. 
Dies geschieht in den Zeilen 180 und 181.
Anschließend wird dieser Bezeichner sowie der zu schreibende Wert der Methode \lstinline{PI_writeSingleIO()} übergeben, welche wiederum die Interaktion mit piControl übernimmt (vgl. Abschnitt \ref{sec:4-picontrol}).
 
\subsection{Integration von piControl%
     \label{sec:4-picontrol}}
In Abschnitt~\ref{sec:2-io} wurde die Anbindung der IO-Module des Revolution Pi sowie die Funktionsweise von piControl aus Anwendersicht beschrieben. Die verfügbare Literatur beschränkt sich auch auf lediglich diese Sicht; eine weiterführende Dokumentation für Entwickler gibt es, neben der in Abschnitt~\ref{sec:3-anbindung} vorgestellten Manpage, nicht. 
In diesem Abschnitt soll daher der Quellcode von piControl sowie dessen Verwendung im Projekt genauer betrachtet werden.
Hierzu wird exemplarisch die in Abschnitt~\ref{sec:4-open62541} eingeführte Methode \lstinline{PI_writeSingleIO()} untersucht.
Diese Methode ermöglicht das Setzen eines einzelnen Bits im Prozessabbild der SPS, und damit das Schalten eines digitalen Ausgangs auf einem IO-Modul.
Die äquivalente Methode \lstinline{int piControlGetBitValue(SPIValue *pSpiValue)} zum Lesen eines Bits bzw. Eingangs funktioniert analog und soll daher an dieser Stelle nicht dediziert erörtert werden.

\begin{lstlisting}[language={c},firstnumber=97,
                   caption={Setzen eines phsikalischen, digitalen Ausgangs in \lstinline{revpi.c}
                   \label{lst:4-PI_writeSingleIO}}]
extern void PI_writeSingleIO(char *pszVariableName, bool *bit, bool verbose)
{
	int rc;
	SPIVariable sPiVariable;
	SPIValue sPIValue;

	strncpy(sPiVariable.strVarName, pszVariableName, sizeof(sPiVariable.strVarName));
	rc = piControlGetVariableInfo(&sPiVariable);
	if (rc < 0) {
		printf("Cannot find variable '%s'\n", pszVariableName);
		return;
	}

		sPIValue.i16uAddress = sPiVariable.i16uAddress;
		sPIValue.i8uBit = sPiVariable.i8uBit;
		sPIValue.i8uValue = *bit;
		rc = |>\tikzmarkin[set border color=martinired]{setBitValue}<|piControlSetBitValue(&sPIValue)|>\tikzmarkend{setBitValue}<|;
		if (rc < 0)
			printf("Set bit error %s\n", getWriteError(rc));
		else if (verbose)
			printf("Set bit %d on byte at offset %d. Value %d\n", sPIValue.i8uBit, sPIValue.i16uAddress,
			       sPIValue.i8uValue);
}
\end{lstlisting}

Der Programmcode in Listing~\ref{lst:4-PI_writeSingleIO} ist Teil des implementierten OPC-Servers. In diesem wird auf zwei Funktionen des piControl-Treibers zugegriffen. 
Beiden Methoden wird als Argument ein Zeiger auf ein Struct vom Typ \lstinline{SPIValue} übergeben. Der im Struct abgelegte Name wird mittels \lstinline{piControlGetVariableInfo(&sPIValue)} zu einer Adresse im Prozessabbild aufgelöst. Diese wird in \lstinline{sPIValue.i16uAdress} gespeichert. Der Wert der Variablen wird anschließend mittels \lstinline{piControlSetBitValue(&sPIValue)} an dieser Adresse in das Prozessabbild geschrieben.

\begin{lstlisting}[language={c},firstnumber=309,caption={Methode \lstinline{piControlSetBitValue} in \lstinline{piControlIf.c}\label{lst:4-piControlSetBitValue}}]
int |>\tikzmarkin[set border color=martiniblue]{setBitValueFcn}<|piControlSetBitValue(SPIValue *pSpiValue)|>\tikzmarkend{setBitValueFcn}<|
{
    piControlOpen();

    if (PiControlHandle_g < 0)
	    return -ENODEV;

    pSpiValue->i16uAddress += pSpiValue->i8uBit / 8;
    pSpiValue->i8uBit %= 8;

    if (|>\tikzmarkin[set border color=martinired]{ioctl}<|ioctl(PiControlHandle_g, KB_SET_VALUE, pSpiValue)|>\tikzmarkend{ioctl}<| < 0)
	    return errno;

    return 0;
}
\end{lstlisting}

Die in Listing~\ref{lst:4-piControlSetBitValue} dargestellte Methode \lstinline{piControlSetBitValue} ist lediglich eine Hüllfunktion (häufig auch als Wrapper-Funktion bezeichnet) für einen Aufruf des \lstinline{ioctl} Kernel-Moduls.
Folgende Parameter werden übergeben:
\lstinline{PiControlHandle_g} ist die Referenz auf die Geräte-Datei des piControl-Treibers. \lstinline{KB_SET_VALUE} ist das ioctl-Kommando zum Schreiben eines Bits in das Prozessabbild. Der Zeiger \lstinline{pSpiValue} verweist auf ein Struct des bereits vorgestellten Typs \lstinline{SPIValue}.

\begin{lstlisting}[language={c},firstnumber=80,caption={Methode \lstinline{piControlOpen} in \lstinline{piControlIf.c}\label{lst:4-piControlOpen}}]
void piControlOpen(void)
{
    /* open handle if needed */
    if (PiControlHandle_g < 0)
    {
	    |>\tikzmarkin[set border color=martiniblue]{PiControlHandle}<|PiControlHandle_g = open(PICONTROL_DEVICE, O_RDWR)|>\tikzmarkend{PiControlHandle}<|;
    }
}
\end{lstlisting}

Die in Listing~\ref{lst:4-piControlOpen} dargestellte Methode öffnet, sofern nicht bereits geschehen, die Geräte-Datei. Das Macro \lstinline{PICONTROL_DEVICE} verweist hierbei auf \lstinline{/dev/piControl0}.

\begin{lstlisting}[language={c},firstnumber=721,caption={Methode \lstinline{piControlIoctl} in \lstinline{piControlMain.c}\label{lst:4-piControlIoctl}}]
static long |>\tikzmarkin[set border color=martiniblue, below offset=0.9em]{piControlIoctl}<|piControlIoctl(struct file *file, unsigned int prg_nr, 
                           unsigned long usr_addr)                                      |>\tikzmarkend{piControlIoctl}<|
{
  int status = -EFAULT;
  tpiControlInst *priv;
  int timeout = 10000;	// ms

  if (prg_nr != KB_CONFIG_SEND && prg_nr != KB_CONFIG_START && !isRunning()) {
  	return -EAGAIN;
  }

  priv = (tpiControlInst *) file->private_data;

  if (prg_nr != KB_GET_LAST_MESSAGE) {
  	// clear old message
  	priv->pcErrorMessage[0] = 0;
  }

  switch (prg_nr) {|>\setcounter{lstnumber}{864}<|

    case |>\tikzmarkin[set border color=martiniblue]{KB_SET_VALUE}<|KB_SET_VALUE:|>\tikzmarkend{KB_SET_VALUE}<|
  		{
  			SPIValue *pValue = (SPIValue *) usr_addr;

  			if (!isRunning())
  				return -EFAULT;

  			if (pValue->i16uAddress >= KB_PI_LEN) {
  				status = -EFAULT;
  			} else {
  				INT8U i8uValue_l;
  				my_rt_mutex_lock(&piDev_g.lockPI);
  				i8uValue_l = piDev_g.ai8uPI[pValue->i16uAddress];

  				if (pValue->i8uBit >= 8) {
  					i8uValue_l = pValue->i8uValue;
  				} else {
  					if (pValue->i8uValue)
  						i8uValue_l |= (1 << pValue->i8uBit);
  					else
  						i8uValue_l &= ~(1 << pValue->i8uBit);
  				}

  				|>\tikzmarkin[set border color=martinired]{i8uValue}<|piDev_g.ai8uPI[pValue->i16uAddress] = i8uValue_l;|>\tikzmarkend{i8uValue}<|
  				rt_mutex_unlock(&piDev_g.lockPI);

  #ifdef VERBOSE
  				pr_info("piControlIoctl Addr=%u, bit=%u: %02x %02x\n", pValue->i16uAddress, pValue->i8uBit, pValue->i8uValue, i8uValue_l);
  #endif

  				status = 0;
  			}
  		}
  		break; |>\setcounter{lstnumber}{1314}<|

    default:
      pr_err("Invalid Ioctl");
      return (-EINVAL);
      break;

    }

    return status;
  }
\end{lstlisting}

Listing~\ref{lst:4-piControlIoctl} zeigt in Auszügen die ioctl-Methode des piControl Kernel-Treibers. Diese bekommt folgende Argumente übergeben: \lstinline{struct file *file} enthält den Verweis auf die Geräte-Datei, hier \lstinline{/dev/piControl0}. Der Wert von \lstinline{unsigned int prg_nr} beschreibt die Anfrage an den Treiber, in diesem Fall \lstinline{KB_SET_VALUE}. Das Argument \lstinline{unsigned long usr_addr} enthält einen typ-agnostischen Pointer. Dieser verweist auf einen Speicherbereich, in welchem die zur Bearbeitung der Anfrage notwendigen Daten abgelegt sind. Hier können auch vom Treiber empfangene Daten dem Anwendungsprogramm bereitgestellt werden. 

Die switch-case-Anweisung führt die über das Argument \lstinline{prg_nr} spezifizierte Aktion aus. Hier betrachten wir \lstinline{KB_SET_VALUE}:
Zunächst wird in Zeile 868 der übergebene Zeiger \lstinline{usr_addr} mittels explizitem Typecast zu einem Zeiger des Typs \lstinline{SPIValue *} konvertiert. Da dieser auf Daten im Userspace verweist, ist beim Zugriff durch den Kernel-Treiber besondere Vorsicht geboten.
In Zeile 877 wird mittels Mutex das Prozessabbild \lstinline{piDev_g} für den Zugriff durch andere Threads oder Prozesse gesperrt.
\lstinline{my_rt_mutex_lock} verweist hierbei auf die Funktion \lstinline{rt_mutex_lock} aus \lstinline{linux/sched.h}\footnote{Offenbar wurde hier auch eine alternative Implementierung vorgesehen, siehe revpi\_common.h}

In Zeile 889 wird das Byte \lstinline{i8uValue_l}, welches den zu schreibenden Wert enthält in das Prozessabbild übertragen. Anschließend wird die Mutex auf \lstinline{piDev_g} wieder entsperrt.
\newpage

\begin{lstlisting}[language={c},firstnumber=62,caption={Auszug des Struct \lstinline{spiControlDev} in \lstinline{piControlMain.h}\label{lst:4-spiControlDev}}]
|>\tikzmarkin[set border color=martiniblue]{spiControlDev}<|typedef struct spiControlDev|>\tikzmarkend{spiControlDev}<| {
	// device driver stuff
	int init_step;
	enum revpi_machine machine_type;
	void *machine;
	struct cdev cdev;	// Char device structure
	struct device *dev;
	struct thermal_zone_device *thermal_zone;

	|>\tikzmarkin[set border color=martiniblue]{processImage}<|// process image stuff
	INT8U ai8uPI[KB_PI_LEN];
	INT8U ai8uPIDefault|>\tikzmarkin[set border color=martinired]{KB_PI_LEN_0}<|[KB_PI_LEN]|>\tikzmarkend{KB_PI_LEN_0}<|;
	struct rt_mutex lockPI;        |>\tikzmarkend{processImage}<|
	bool stopIO;
	piDevices *devs; |>\setcounter{lstnumber}{94}<|
} tpiControlDev;
\end{lstlisting}

Das Prozessabbild ist als Byte-Array der Länge \lstinline{KB_PI_LEN} in Listing~\ref{lst:4-spiControlDev} definiert. Konfigurationsparameter wie \lstinline{KB_PI_LEN} oder die Zykluszeit für den Datenaustausch zwischen SPS und IO-Modulen sind im folgenden Listing~\ref{lst:4-process} definiert.

\begin{lstlisting}[language={c},firstnumber=119,caption={Konfigurationsparameter des Prozessabbildes in project.h\label{lst:4-process}}]
#define INTERVAL_PI_GATE (5*1000*1000)  // 5 ms piGateCommunication |>\setcounter{lstnumber}{128}<|

#define INTERVAL_IO_COM (5*1000*1000)  // 5 ms piIoComm |>\setcounter{lstnumber}{132}<|

#define KB_PD_LEN       512
|>\tikzmarkin[set border color=martiniblue]{KB_PI_LEN_1}<|#define KB_PI_LEN       4096|>\tikzmarkend{KB_PI_LEN_1}<|
\end{lstlisting}

Das zu setzende Bit wurde zu diesem Zeitpunkt erfolgreich in das Prozessabbild der SPS geschrieben.
Es stellt sich die Frage, wie dieses nun an das IO-Modul kommuniziert wird.
Die Kommunikation mit allen angebundenen Modulen ist ebenfalls Aufgabe des piControl-Treibers.

\begin{lstlisting}[language={c},firstnumber=256,caption={Auszug der Methode \lstinline{piIoThread} in \lstinline{revpi_core.c}\label{lst:4-piIoThread}}]
static int piIoThread(void *data)
{
	//TODO int value = 0;
	ktime_t time;
	ktime_t now;
	s64 tDiff;

	hrtimer_init(&piCore_g.ioTimer, CLOCK_MONOTONIC, HRTIMER_MODE_ABS);
	piCore_g.ioTimer.function = piIoTimer;

	pr_info("piIO thread started\n");

	now = hrtimer_cb_get_time(&piCore_g.ioTimer);

	PiBridgeMaster_Reset();

	while (!kthread_should_stop()) {
		if (|>\tikzmarkin[set border color=martinired]{PiBridgeMaster}<|PiBridgeMaster_Run()|>\tikzmarkend{PiBridgeMaster}<| < 0)
			break;
	}

	RevPiDevice_finish();

	pr_info("piIO exit\n");
	return 0;
}
\end{lstlisting}

Der Kernel-Thread \lstinline{piIoThread} ist verantwortlich für den zyklischen Datenaustausch mit den IO-Modulen. In diesem wird fortlaufend die Methode \lstinline{PiBridgeMaster_Run()} aufgerufen, siehe Listing~\ref{lst:4-piIoThread}.

\begin{lstlisting}[language={c},firstnumber=262,caption={Auszug der Methode \lstinline{PiBridgeMaster_Run(void)} in \lstinline{RevPiDevice.c}\label{lst:4-PiBridgeMaster_Run}}]
int PiBridgeMaster_Run(void)
{
	static kbUT_Timer tTimeoutTimer_s;
	static kbUT_Timer tConfigTimeoutTimer_s;
	static int error_cnt;
	static INT8U last_led;
	static unsigned long last_update;
	int ret = 0;
	int i;

	my_rt_mutex_lock(&piCore_g.lockBridgeState);
	if (piCore_g.eBridgeState != piBridgeStop) {
		switch (eRunStatus_s) { |>\setcounter{lstnumber}{514}<|
		    case enPiBridgeMasterStatus_EndOfConfig:|>\setcounter{lstnumber}{621}<|
		    if (|>\tikzmarkin[set border color=martinired]{RevPiDevice}<|RevPiDevice_run()|>\tikzmarkend{RevPiDevice}<|) {
				// an error occured, check error limits |>\setcounter{lstnumber}{641}<|
			} else {
				ret = 1;
			}
			piCore_g.image.drv.i16uRS485ErrorCnt = RevPiDevice_getErrCnt();
			break;
\end{lstlisting}

Die in Listing~\ref{lst:4-PiBridgeMaster_Run} dargestellte Methode ist eine sog. State-Machine. Ist die Konfiguration der IO-Module erfolgreich abgeschlossen, so führt sie bei Aufruf lediglich die Methode \lstinline{RevPiDevice_run()} aus.

\begin{lstlisting}[language={c},firstnumber=140,caption={Auszug der Methode \lstinline{RevPiDevice_run(void)} in \lstinline{RevPiDevice.c}\label{lst:4-RevPiDevice_run}}]
int RevPiDevice_run(void)
{
	INT8U i8uDevice = 0;
	INT32U r;
	int retval = 0;

	RevPiDevices_s.i16uErrorCnt = 0;

	for (i8uDevice = 0; i8uDevice < RevPiDevice_getDevCnt(); i8uDevice++) {
		if (RevPiDevice_getDev(i8uDevice)->i8uActive) {
			switch (RevPiDevice_getDev(i8uDevice)->sId.i16uModulType) {
			case KUNBUS_FW_DESCR_TYP_PI_DIO_14:
			case KUNBUS_FW_DESCR_TYP_PI_DI_16:
			case KUNBUS_FW_DESCR_TYP_PI_DO_16:
				r = |>\tikzmarkin[set border color=martinired]{sendCyclicTelegram}<|piDIOComm_sendCyclicTelegram(i8uDevice)|>\tikzmarkend{sendCyclicTelegram}\setcounter{lstnumber}{166} <|;

				break; |>\setcounter{lstnumber}{216}<|
			}
		}
	} |>\setcounter{lstnumber}{227}<|
	return retval;
}
\end{lstlisting}

Diese iteriert wie in Listing~\ref{lst:4-RevPiDevice_run} abgebildete durch alle gegenwärtig in der SPS konfigurierten Module. Ist das aktuelle Modul als aktiv markiert, so wird anhand eines sog. Firmware-Descriptors entschieden, welche Methode für die Ansteuerung des Moduls aufzurufen ist.

\begin{lstlisting}[language={c},firstnumber=161,caption={Auszug der Methode \lstinline{piDIOComm_sendCyclicTelegram} in \lstinline{piDIOComm.c}\label{lst:4-sendCyclicTelegram}}]
INT32U piDIOComm_sendCyclicTelegram(INT8U i8uDevice_p)
{
	INT32U i32uRv_l = 0;
	SIOGeneric sRequest_l;
	SIOGeneric sResponse_l;
	INT8U len_l, data_out[18], i, p, data_in[70];
	INT8U i8uAddress;
	int ret; |>\setcounter{lstnumber}{239}<|
	
    |>\tikzmarkin[set border color=martinired]{piIoComm}<|ret = piIoComm_send((INT8U *) & sRequest_l, IOPROTOCOL_HEADER_LENGTH + len_l + 1);  |>\tikzmarkend{piIoComm}\setcounter{lstnumber}{298}<|
}
\end{lstlisting}

Im Falle des hier verwendeten DO-Moduls wird die in Listing~\ref{lst:4-sendCyclicTelegram} abgebildete Methode \lstinline{piDIOComm_sendCyclicTelegram()} aufgerufen. Dieser wird ein Zeiger auf das zu schreibende Gerät übergeben. 
Zunächst wird das Prozessabbild mittels eines proprietären, jedoch im Quellcode offen nachvollziehbaren Protokolls in ein \lstinline{sRequest_l} genanntes Byte-Array umgewandelt. Dieser Schritt ist in Listing~\ref{lst:4-sendCyclicTelegram} nicht abgebildet. Anschließend wird \lstinline{piIoComm_send()} ein Zeiger auf die so generierte Schreib-Anfrage übergeben.

\begin{lstlisting}[language={c},firstnumber=220,caption={Auszug der Methode \lstinline{piIOComm_send} in \lstinline{piIOComm.c}\label{lst:4-piIOComm_send}}]
int piIoComm_send(INT8U * buf_p, INT16U i16uLen_p)
{
	ssize_t write_l = 0;
	INT16U i16uSent_l = 0;|>\setcounter{lstnumber}{249}<|

	while (i16uSent_l < i16uLen_p) {
		write_l = vfs_write(piIoComm_fd_m, buf_p + i16uSent_l, i16uLen_p - i16uSent_l, &piIoComm_fd_m->f_pos);
		if (write_l < 0) {
			pr_info_serial("write error %d\n", (int)write_l);
			return -1;
		} 
		i16uSent_l += write_l;|>\setcounter{lstnumber}{263}<|
	}
	clear();
	vfs_fsync(piIoComm_fd_m, 1);
	return 0;
}
\end{lstlisting}

Listing~\ref{lst:4-piIOComm_send} zeigt die Implementierung von \lstinline{piIoComm_send()}. Diese Methode ist für das Schreiben der oben generierten Anfrage auf die seriellen Schnittstelle verantwortlich. Realisiert wird dies mittels der Methode \lstinline{vfs_write()}. Diese ist in \lstinline{<linux/fs.h>} definiert. Sie ermöglicht das Schreiben einer Datei im Userspace aus dem Kernel heraus. Geschrieben wird hier die Datei mit dem Deskriptor \lstinline{piIoComm_fd_m}.
Da die Funktion \lstinline{vfs_write()} durch andere Kernel-Tasks unterbrochen werden kann, ist nicht gewährleistet, dass die gesamte Anfrage mit nur einem Aufruf geschrieben wird. Die oben abgebildete while-Schleife stellt das vollständige Senden der Anfrage sicher.

\begin{lstlisting}[language={c},firstnumber=157,caption={Auszug der Methode \lstinline{piIOComm_open_serial} in \lstinline{piIOComm.c}\label{lst:4-piIOComm_open_serial}}]
int piIoComm_open_serial(void)
{   |>\setcounter{lstnumber}{167}<|
	struct file *fd;	/* Filedeskriptor */
	struct termios newtio;	/* Schnittstellenoptionen */

	|>\tikzmarkin[set border color=martiniblue]{fd}<|/* Port oeffnen - read/write, kein "controlling tty", 
	    Status von DCD ignorieren */
	fd = filp_open(|>\tikzmarkin[set border color=martinired]{tty}<|REV_PI_TTY_DEVICE|>\tikzmarkend{tty}<|, O_RDWR | O_NOCTTY, 0); |>\setcounter{lstnumber}{208}<|
	
	piIoComm_fd_m = fd;                                                      |>\tikzmarkend{fd}\setcounter{lstnumber}{217}<|

	return 0;
}
\end{lstlisting}

Der zum Schreiben auf die serielle Schnittstelle verwendete Datei-Deskriptor wird von der in Listing~\ref{lst:4-piIOComm_open_serial} abgebildeten Methode \lstinline{piIoComm_open_serial()} generiert. 

\begin{lstlisting}[language={c},firstnumber=45,caption={Definition der seriellen Schnittstelle in \lstinline{piIOComm.h}\label{lst:4-REV_PI_TTY_DEVICE}}]
#define REV_PI_TTY_DEVICE	"/dev/ttyAMA0"
\end{lstlisting}

Das in Listing~\ref{lst:4-REV_PI_TTY_DEVICE} definierte Macro verweist auf eine der seriellen Schnittstellen des RaspberryPi.
Die Implementierung des zugehörigen Schnittstellentreibers soll hier nicht weiter untersucht werden. Somit ist an dieser Stelle die Kette vom Setzen einer Variablen auf dem OPC-Server bis hin zur Aktualisierung des Prozessabbilds der IO-Module geschlossen.

% \begin{lstlisting}[language={c},firstnumber={226},caption={Setzen der Scheduler-Priorität auf SCHED\_FIFO in 
% revpi\_common.c\label{lst:2-sched_priority}}]
% param.sched_priority = ktprio->prio;
% ret = sched_setscheduler(child, SCHED_FIFO, &param);
% \end{lstlisting}
% % % Imports nur für Referenzenauflösung während des Schreibens! Vorm Kompilieren auskommentieren!
% \bibliography{0_hauptdatei}
% \input{1_einleitung}
% \input{2_grundlagen}
% \input{3_konzeption}
% \input{4_implementierung}
% \input{5_tests}
% \input{6_zusammenfassung}
% % Ende Imports

\section{Test des OPC-Servers im Gesamtsystem%
  \label{sec:5-tests}}

% % % Imports nur für Referenzenauflösung während des schreibens! Vorm Kompilieren auskommentieren!
% \bibliography{0_hauptdatei}
% \input{1_einleitung}
% \input{2_grundlagen}
% \input{3_konzeption}
% \input{4_implementierung}
% \input{5_tests}
% \input{6_zusammenfassung}
% % Ende Imports

\section{Zusammenfassung und Ausblick%
  \label{sec:6-fazit}}
Der folgende Abschnitt~\ref{sec:6-zusammenfassung} fasst die gewonnenen Erkenntnisse und den Stand der Implementierung zusammen.
Den Abschluss dieser Arbeit bildet der Ausblick in Abschnitt~\ref{sec:6-ausblick}.

\subsection{Zusammenfassung%
     \label{sec:6-zusammenfassung}}

\subsection{Ausblick%
     \label{sec:6-ausblick}}

% % Ende Imports

\section{Grundlagen%
  \label{sec:2-grundlagen}}

\subsection{Speicherprogrammierbare-Steuerung und Linux -- Revolution Pi%
     \label{sec:2-sps}}

\subsubsection{Kunbus RevolutionPi%
        \label{sec:2-revpi}}
Der RevolutionPi 3 ist eine speicherprogrammierbare Steuerung (SPS) des Herstellers
Kunbus GmbH. Kern dieser SPS ist das von der Raspberry Pi Foundation entwickelte
und vertriebene Raspberry Pi Compute Module 3. Dieses integriert ein Broadcom BCM2837
System-on-Chip (SoC) mit vier 1,2GHz Prozessorkernen, 1GB RAM, 4GB eMMC Anwendungsspeicher
und sonstige Peripherie in ein Modul im DDR2-SODIMM Formfaktor. Diese Spezifikationen
sind weitgehend identisch zu denen des ausgesprochen populären Raspberry Pi 3.
Der Revolution Pi profitiert daher von dem gleichen großen Angebot an Software
und Unterstützung wie der Raspberry Pi, ergänzt dessen Hardware jedoch um eine 24V
Spannungsversorgung, die Möglichkeit der Erweiterung durch mehrere industrietaugliche
Ein-/ Ausgabemodule und Gateways sowie ein Gehäuse zur Montage auf einer DIN-Schiene.
\begin{itemize}
  \item{Prozessor: BCM2837}
  \item{Taktfrequenz 1,2 GHz}
  \item{Anzahl Prozessorkerne: 4}
  \item{Arbeitsspeicher: 1 GByte}
  \item{eMMC Flash Speicher: 4 GByte}
  \item{Betriebssystem: Angepasstes Raspbian mit RT-Patch}
  \item{RTC mit 24h Pufferung über wartungsfreien Kondensator}
  \item{Treiber / API: Treiber schreibt zyklisch Prozessdaten in ein Prozessabbild, Zugriff auf Prozessabbild über Linux-Filesystem als API zu Fremdsoftware.}
  \item{Kommunikationsanschlüsse: 2 x USB 2.0 A (je 500 mA belastbar), 1 x Micro-USB, HDMI, Ethernet (RJ45) 10/100 Mbit/s}
  \item{Stromversorgung: min. 10,7 V, max. 28,8 V, maximal 10 Watt}
  \item{Zulässige Umgebungstemperatur: -40 bis +55 C}
  \item{Gehäuseabmessungen: (HxBxL) 96 mm x 22,5 mm x 110,5 mm (ohne gesteckte Stecker)}
  \item{ESD Schutz: 4 kV / 8 kV gemäß EN61131-2 und IEC 61000-6-2}
  \item{Surge / Burst Prüfungen: gemäß EN61131-2 und IEC 61000-6-2 eingekoppelt auf Versorgungsspannung, Ethernet und IO-Leitungen}
  \item{EMI Prüfungen: gemäß EN61131-2 und IEC 61000-6-2}
\end{itemize}

Kunbus bietet eine Auswahl an IO- und Gateway-Modulen zur Erweiterung des Revolution Pi an.
Gateways dienen der Kommunikation mit Systemen oder Komponenten der Automatisierungstechnik
über Protokolle wie PROFIBUS oder EtherCAT. IO-Module erlauben die Überwachung
und Steuerung von digitalen oder analogen Ein- und Ausgängen.

\subsubsection{Zugriff auf IO-Module%
        \label{sec:2-io}}
Der Zugriff auf die Ein- und Ausgänge der IO-Module erfolgt über ein Prozessabbild
und einen hierfür von Kunbus bereitgestellten Treiber, genannt piControl. Dieser
aktualisiert das Prozessabbild zyklisch. Die angestrebte Zykluszeit beträgt 5ms,
kann jedoch je nach Anzahl der angeschlossenen Module auch größer sein. Kunbus
garantiert bei drei IO-Modulen und zwei Gateway-Modulen eine Zykluszeit von 10 ms.
Jedes der IO-Module stellt ein eigenständiges eingebettetes System dar. Es verfügt
über einen Microcontroller, welcher die IOs bereitstellt und über einen RS485-Bus
mit dem Revolution Pi kommuniziert.
% https://revolution.kunbus.de/io-modul/

Lizenz: GPL
% https://github.com/RevolutionPi/piControl

\begin{lstlisting}[language={c},firstnumber={226},caption={Setzen der Scheduler-Priorität auf SCHED\_FIFO in revpi\_common.c\label{lst:2-sched_priority}}]
param.sched_priority = ktprio->prio;
ret = sched_setscheduler(child, SCHED_FIFO,
       &param);
\end{lstlisting}


\subsection{Echtzeit und Multithreading unter Linux -- preemptRT und posix%
     \label{sec:2-echtzeit}}


 Der Linux-Kernel verfügt über mehrere unterschiedliche Preemtion-Modelle:

\begin{itemize}
  \item No Forced Preemption (server):
  Ausgelegt auf maximal möglichen Durchsatz, lediglich Interrupts und
  System-Call-Returns bewirken Präemption.

  \item Voluntary Kernel Preemption (Desktop):
  Neben den implizit bevorrechtigten Interrupts und System-Call-Returns gibt es
  in diesem Modell weitere Abschnitte des Kernels in welchen Preämption explizit
  gestattet ist.

  \item Preemptible Kernel (Low-Latency Desktop):
  In diesem Modell ist der gesamte Kernel, mit Ausnahme sog.~kritischer Abschnitte
  präemptible. Nach jedem kritischen Abschnitt gibt es einen impliziten Präemptions-Punkt.

  \item Preemptible Kernel (Basic RT):
  Dieses Modell ist dem zuvor genannten sehr ähnlich, hier sind jedoch alle Interrupt-Handler
  als eigenständige Threads ausgeführt.

  \item Fully Preemptible Kernel (RT):
  Wie auch bei den beiden zuvor genannten Modellen ist hier der gesamte Kernel
  präemtible, die Anzahl und Dauer der nicht-präemtiblen kritischen Abschnitte
  ist auf ein notwendiges Minimum beschränkt. Alle Interrupt-Handler sind als
  eigenständige Threads ausgeführt, Spinlocks durch Sleeping-Spinlocks und Mutexe
  durch sog.~RT-Mutexe ersetzt.

\end{itemize}
\todo{Spinlocks und Mutexe sowie die RT-Varianten dieser erklären!}

Lediglich mit dem vollständig präemtiblen Kernel kann Echtzeit-Verhalten realisiert werden.

% https://wiki.linuxfoundation.org/realtime/documentation/technical_basics/preemption_models bzw kernel/Kconfig.preempt

\subsubsection{preemptRT%
        \label{sec:2-preemptRT}}
% https://wiki.linuxfoundation.org/realtime/documentation/technical_details/start
% https://wiki.linuxfoundation.org/realtime/documentation/technical_basics/start

Das dem PREEMPT RT Kernel zugrunde liegende Prinzip lässt sich in einer einfachen
Regel ausdrücken: Nur Code, welcher absolut nicht-präemtible sein darf, ist es
gestattet nicht-präemtible zu sein.
Das erklärte Ziel des PREEMPT\_RT Patches ist es folglich, die Menge des nicht-präemtiblen
Codes im Linux-Kernel auf das absolut notwendige Minimum zu reduzieren.

Dies wird durch Verwendung folgender Mechanismen erreicht:

\begin{itemize}
  \item Hochauflösende Timer
  \item Sleeping Spinlocks
  \item Threaded Interrupt Handlers
  \item rt\_mutex
  \item RCU
\end{itemize}


\subsubsection{posix%
        \label{sec:2-posix}}
Ist posix hier wirklich relevant? Debian bzw.~Raspbian sind weitgehend posix
kompatibel, aber wird es hier genutzt? -> JA, open62541 nutzt pthread.h
piControl nutzt kthread.h, und semaphore.h

\subsection{OPC-UA und open62541%
     \label{sec:2-opc}}

\subsubsection{OPC UA%
        \label{sec:2-opcua}}
Open Platform Communications (OPC) ist eine Familie von Standards zur herstellerunabhängigen
Kommunikation von Maschinen (M2M) in der Automatisierungstechnik. Die sog.~OPC Task Force, zu deren
Mitgliedern verschiedene große Firmen der Automatisierungsindustrie gehören, veröffentlichte
die OPC Specification Version 1.0 im August 1996.
Motiviert ist dieser offene Standard durch die Erkenntniss, dass die Anpassung der
zahlreichen Herstellerstandards an individuelle Infrastrukturen und Anlagen einen
großen Mehraufwand verursachen.
Die Wikipedia beschreibt das Anwendungsgebiet für OPC wie folgt:

\glqq{}OPC wird dort eingesetzt, wo Sensoren, Regler und Steuerungen verschiedener Hersteller
ein gemeinsames Netzwerk bilden. Ohne OPC benötigten zwei Geräte zum Datenaustausch
genaue Kenntnis über die Kommunikationsmöglichkeiten des Gegenübers. Erweiterungen
und Austausch gestalten sich entsprechend schwierig. Mit OPC genügt es, für jedes
Gerät genau einmal einen OPC-konformen Treiber zu schreiben. Idealerweise wird
dieser bereits vom Hersteller zur Verfügung gestellt. Ein OPC-Treiber lässt sich
ohne großen Anpassungsaufwand in beliebig große Steuer- und Überwachungssysteme
integrieren.

OPC unterteilt sich in verschiedene Unterstandards, die für den jeweiligen Anwendungsfall
unabhängig voneinander implementiert werden können. OPC lässt sich damit verwenden
für Echtzeitdaten (Überwachung), Datenarchivierung, Alarm-Meldungen und neuerdings
auch direkt zur Steuerung (Befehlsübermittlung).\grqq{}

OPC basiert in der ursprünglichen Spezifikation auf Microsofts DCOM-Spezifikation.
DCOM macht Funktionen und Objekte einer Anwendung anderen Anwendungen im Netzwerk
zugänglich. Der OPC-Standard definiert entsprechende DCOM-Objekte um mit anderen
OPC-Anwendungen Daten austauschen zu können. Die Verwendung von DCOM bindet Anwender
an Betriebssysteme von Microsoft. Die ursprüngliche OPC Spezifikation wird durch die
Entwicklung von OPC Unified Architecture (OPC UA) abgelöst.
OPC UA setzt auf einem eigenen Kommunikationionsstack auf, die Verwendung von DCOM
und damit die Bindung an Microsoft wurden aufgelöst.

Die OPC-UA-Architektur ist eine Service-orientierte Architektur (SOA), deren Struktur
aus mehreren Schichten besteht.

% Wikipedia
Das OPC-Informationsmodell ist nicht mehr nur eine Hierarchie aus Ordnern, Items
und Properties. Es ist ein sogenanntes Full-Mesh-Network aus Nodes, mit dem neben
den Nutzdaten eines Nodes auch Meta- und Diagnoseinformationen repräsentiert werden.
Ein Node ähnelt einem Objekt aus der objektorientierten Programmierung. Ein Node
kann Attribute besitzen, die gelesen werden können (Data Access (DA), Historical
Data Access (HDA)). Es ist möglich Methoden zu definieren und aufzurufen.
Eine Methode besitzt Aufrufargumente und Rückgabewerte. Sie wird durch ein Command
aufgerufen. Weiterhin werden Events unterstützt, die versendet werden können
(AE (Alarms \& Events), DA DataChange), um bestimmte Informationen zwischen Geräten
auszutauschen. Ein Event besitzt unter anderem einen Empfangszeitpunkt, eine Nachricht
und einen Schweregrad. Die o. g. Nodes werden sowohl für die Nutzdaten als auch
alle anderen Arten von Metadaten verwendet. Der damit modellierte OPC-Adressraum
beinhaltet nun auch ein Typmodell, mit dem sämtliche Datentypen spezifiziert werden.

% https://de.wikipedia.org/wiki/Open_Platform_Communications
% https://de.wikipedia.org/wiki/OPC_Unified_Architecture
% https://opcfoundation.org/developer-tools/specifications-unified-architecture
% Von Gerhard Gappmeier - ascolab GmbH, CC BY-SA 3.0, https://de.wikipedia.org/w/index.php?curid=1892069
\subsubsection{open62541%
        \label{sec:2-open62541}}
open62541 ist eine offene und freie Implementierung von OPC UA. Die in C geschriebene
Bibliothek stellt eine beständig zunehmende Anzahl der im OPC UA Standard definierten
Funktionen bereit. Sie kann sowohl zur Erstellung von OPC-Servern als auch -Clients
genutzt werden. Ergänzend zu der unter der Mozilla Public License v2.0 lizensierten
Bibliothek stellt das open62541 Projekt auch Beispielprogramme unter einer CC0 Lizenz
zur Verfügung.

Die Bibliothek eignet sich auch für die Entwicklung auf eingebetteten Systemen und
Microcontrollern. Je nach Umfang der gewünschten Funktionen und des OPC Informationsmodells
beträgt die Größe einer Server-Binary weniger als 100kb. %evtl. kürzen?

\todo{Nodes erklären! Evtl.~oben!}

Folgende Auswahl an Eigenschaften und Funktionen zeichnet die in dieser Arbeit verwendete
Version 0.3 von open62541 aus:
\begin{itemize}
  \item Kommunikationionsstack
  \begin{itemize}
      \item OPC UA Binär-Protokoll (HTTP oder SOAP werden gegenwärtig nicht unterstützt)
      \item Austauschbare Netzwerk-Schicht, welche die Verwendung eigener Netzwerk-APIs
      erlaubt.
      \item Verschlüsselte Kommunikationion
      \item Asynchrone Dienst-Anfragen im Client
  \end{itemize}
  \item Informationsmodell
  \begin{itemize}
    \item Unterstützung aller OPC UA Node-Typen, inkl.~Methoden
    \item Hinzufügen und Entfernen von Nodes und Referenzen zur Laufzeit.
    \item Vererbung und Instanziierung von Objekt- und Variablentypen
    \item Zugriffskontrolle auch für einzelne Nodes
  \end{itemize}
  \item Subscriptions
  \begin{itemize}
    \item Erlaubt die Überwachung (subscriptions / monitoreditems)
    \item Sehr geringer Ressourcenbedarf pro überwachtem Wert
  \end{itemize}
  \item Code-Generierung auf XML-Basis
  \begin{itemize}
    \item Erlaubt die Erstellung von Datentypen
    \item Erlaubt die Generierung des serverseitigen Informationsmodells
  \end{itemize}
\end{itemize}

% https://open62541.org/doc/0.3/


Mozilla Public License
CC0 Lizenz für Beispiele und Plugins

% https://open62541.org/doc/open62541-current.pdf
% https://open62541.org/

% % % Imports nur für Referenzenauflösung während des Schreibens! Vorm Kompilieren auskommentieren!
% \bibliography{0_hauptdatei}
% % Mit \section{...} eröffnen wir einen neuen Abschnitt.
% Der Befehl setzt nicht nur den Text in einer größeren,
% fetten Schrift, sondern sorgt außerdem dafür, daß er im
% Inhaltsverzeichnis erscheint.
%
% Mit \label{...} erzeugen wir einen Bezeichner, mit dessen Hilfe
% wir später auf die Nummer des Abschnitts verweisen können (nämlich
% mit~\ref{...}).
%
% Das Kommentarzeichen hinter „Übersicht“ dient dazu, ein
% Leerzeichen zwischen „Übersicht“ und dem \label-Befehl
% zu vermeiden, das andernfalls sichtbar würde – z.B. im
% Inhaltsverzeichnis.
%

% % Imports nur für Referenzenauflösung während des Schreibens! Vorm Kompilieren auskommentieren!
% \bibliography{0_hauptdatei}
% \input{1_einleitung}
%\input{2_grundlagen}
%\input{3_konzeption}
%\input{4_implementierung}
%\input{5_tests}
%\input{6_zusammenfassung}
% % Ende Imports

\section{Einleitung und Motivation%
  \label{sec:1-einleitung}}
Ziel dieses Projektes ist die Integration eines OPC-Servers mit einer auf Linux
basierenden speicherprogrammierbaren Steuerung (SPS). Angeschlossen an diese SPS
ist jeweils ein digitales Ein-/\,bzw.~Ausgabemodul. Die von diesen bereitgestellten
Ein-/\, bzw.~Ausgänge (IO) sollen in der Datenstruktur des OPC-Servers abgebildet
und über diesen für OPC-Clients les-/\,und schreibar sein. Weiterhin sollen einige
Funktionen zur Überwachung und Steuerung der an die SPS angeschlossenen Aktoren
und Sensoren direkt im OPC-Server implementiert werden.
Hiermit stellt dieses Projekt eine der Grundlagen für ein übergeordnetes Projekt,
die cloudbasierte Steuerung eines miniaturisierten Produktions-Systems, dar.

Der hier verwendete OPC-Server ist Teil des sog. open62541 Projekts. Er ist in C
geschrieben und implementiert bereits einen großen Teil der im OPC-UA-Standard
spezifizierten Funktionen.
Als SPS findet ein Revolution Pi 3 der Firma Kunbus Verwendung. Dieser integriert
ein sog. Compute Module der Raspberry Pi Foundation in ein industrietaugliches
Gehäuse und erlaubt die Erweiterung mittels IO- oder Gateway-Modulen. Über diese
erfolgt die Kommunikation mit weiteren Komponenten der Automatisierungstechnik.

Motiviert ist dieses Projekt durch die Beobachtung, dass die Verbreitung offener
Standards sowie freier Software auch in der Automatisierungstechnik zunimmt.
Linux ist ein freies Betriebssystem, OPC-UA ein offen zugänglicher, aktiv gepflegter
und weit verbreiteter Standard. Der Raspberry Pi findet sowohl bei Hobby-Anwendern als
auch in den Bereichen Forschung und Entwicklung sowie bei industriellen Anwendern
Verwendung. Dieses Projekt stellt somit eine für unterschiedliche Anwender interessante
Entwicklung dar.

Im Anschluss an diese einleitende Übersicht im Abschnitt~\ref{sec:1-einleitung} folgt
die Darstellung der wichtigsten Grundlagen in Abschnitt~\ref{sec:2-grundlagen}.
Aufbauend auf diesen Grundlagen folgt die konzeptuelle Ausarbeitung im Abschnitt~\ref{sec:3-konzeption}.
Die Umsetzung wird im Abschnitt~\ref{sec:4-implementierung} erläutert.
Die Leistungsfähigkeit der Implementierung wird in Abschnitt~\ref{sec:5-tests} untersucht.
Eine Zusammenfassung und ein Ausblick schließen die Arbeit in
Abschnitt~\ref{sec:6-fazit} ab. Eventuell noch benötigte Anhänge
finden sich in den Anhängen [...] bis [...].

% % % Imports nur für Referenzenauflösung während des Schreibens! Vorm Kompilieren auskommentieren!
% \bibliography{0_hauptdatei}
% \input{1_einleitung}
% \input{2_grundlagen}
% \input{3_konzeption}
% \input{4_implementierung}
% \input{5_tests}
% \input{6_zusammenfassung}
% % Ende Imports

\section{Grundlagen%
  \label{sec:2-grundlagen}}

\subsection{Speicherprogrammierbare-Steuerung und Linux -- Revolution Pi%
     \label{sec:2-sps}}

\subsubsection{Kunbus RevolutionPi%
        \label{sec:2-revpi}}
Der RevolutionPi 3 ist eine speicherprogrammierbare Steuerung (SPS) des Herstellers
Kunbus GmbH. Kern dieser SPS ist das von der Raspberry Pi Foundation entwickelte
und vertriebene Raspberry Pi Compute Module 3. Dieses integriert ein Broadcom BCM2837
System-on-Chip (SoC) mit vier 1,2GHz Prozessorkernen, 1GB RAM, 4GB eMMC Anwendungsspeicher
und sonstige Peripherie in ein Modul im DDR2-SODIMM Formfaktor. Diese Spezifikationen
sind weitgehend identisch zu denen des ausgesprochen populären Raspberry Pi 3.
Der Revolution Pi profitiert daher von dem gleichen großen Angebot an Software
und Unterstützung wie der Raspberry Pi, ergänzt dessen Hardware jedoch um eine 24V
Spannungsversorgung, die Möglichkeit der Erweiterung durch mehrere industrietaugliche
Ein-/ Ausgabemodule und Gateways sowie ein Gehäuse zur Montage auf einer DIN-Schiene.
\begin{itemize}
  \item{Prozessor: BCM2837}
  \item{Taktfrequenz 1,2 GHz}
  \item{Anzahl Prozessorkerne: 4}
  \item{Arbeitsspeicher: 1 GByte}
  \item{eMMC Flash Speicher: 4 GByte}
  \item{Betriebssystem: Angepasstes Raspbian mit RT-Patch}
  \item{RTC mit 24h Pufferung über wartungsfreien Kondensator}
  \item{Treiber / API: Treiber schreibt zyklisch Prozessdaten in ein Prozessabbild, Zugriff auf Prozessabbild über Linux-Filesystem als API zu Fremdsoftware.}
  \item{Kommunikationsanschlüsse: 2 x USB 2.0 A (je 500 mA belastbar), 1 x Micro-USB, HDMI, Ethernet (RJ45) 10/100 Mbit/s}
  \item{Stromversorgung: min. 10,7 V, max. 28,8 V, maximal 10 Watt}
  \item{Zulässige Umgebungstemperatur: -40 bis +55 C}
  \item{Gehäuseabmessungen: (HxBxL) 96 mm x 22,5 mm x 110,5 mm (ohne gesteckte Stecker)}
  \item{ESD Schutz: 4 kV / 8 kV gemäß EN61131-2 und IEC 61000-6-2}
  \item{Surge / Burst Prüfungen: gemäß EN61131-2 und IEC 61000-6-2 eingekoppelt auf Versorgungsspannung, Ethernet und IO-Leitungen}
  \item{EMI Prüfungen: gemäß EN61131-2 und IEC 61000-6-2}
\end{itemize}

Kunbus bietet eine Auswahl an IO- und Gateway-Modulen zur Erweiterung des Revolution Pi an.
Gateways dienen der Kommunikation mit Systemen oder Komponenten der Automatisierungstechnik
über Protokolle wie PROFIBUS oder EtherCAT. IO-Module erlauben die Überwachung
und Steuerung von digitalen oder analogen Ein- und Ausgängen.

\subsubsection{Zugriff auf IO-Module%
        \label{sec:2-io}}
Der Zugriff auf die Ein- und Ausgänge der IO-Module erfolgt über ein Prozessabbild
und einen hierfür von Kunbus bereitgestellten Treiber, genannt piControl. Dieser
aktualisiert das Prozessabbild zyklisch. Die angestrebte Zykluszeit beträgt 5ms,
kann jedoch je nach Anzahl der angeschlossenen Module auch größer sein. Kunbus
garantiert bei drei IO-Modulen und zwei Gateway-Modulen eine Zykluszeit von 10 ms.
Jedes der IO-Module stellt ein eigenständiges eingebettetes System dar. Es verfügt
über einen Microcontroller, welcher die IOs bereitstellt und über einen RS485-Bus
mit dem Revolution Pi kommuniziert.
% https://revolution.kunbus.de/io-modul/

Lizenz: GPL
% https://github.com/RevolutionPi/piControl

\begin{lstlisting}[language={c},firstnumber={226},caption={Setzen der Scheduler-Priorität auf SCHED\_FIFO in revpi\_common.c\label{lst:2-sched_priority}}]
param.sched_priority = ktprio->prio;
ret = sched_setscheduler(child, SCHED_FIFO,
       &param);
\end{lstlisting}


\subsection{Echtzeit und Multithreading unter Linux -- preemptRT und posix%
     \label{sec:2-echtzeit}}


 Der Linux-Kernel verfügt über mehrere unterschiedliche Preemtion-Modelle:

\begin{itemize}
  \item No Forced Preemption (server):
  Ausgelegt auf maximal möglichen Durchsatz, lediglich Interrupts und
  System-Call-Returns bewirken Präemption.

  \item Voluntary Kernel Preemption (Desktop):
  Neben den implizit bevorrechtigten Interrupts und System-Call-Returns gibt es
  in diesem Modell weitere Abschnitte des Kernels in welchen Preämption explizit
  gestattet ist.

  \item Preemptible Kernel (Low-Latency Desktop):
  In diesem Modell ist der gesamte Kernel, mit Ausnahme sog.~kritischer Abschnitte
  präemptible. Nach jedem kritischen Abschnitt gibt es einen impliziten Präemptions-Punkt.

  \item Preemptible Kernel (Basic RT):
  Dieses Modell ist dem zuvor genannten sehr ähnlich, hier sind jedoch alle Interrupt-Handler
  als eigenständige Threads ausgeführt.

  \item Fully Preemptible Kernel (RT):
  Wie auch bei den beiden zuvor genannten Modellen ist hier der gesamte Kernel
  präemtible, die Anzahl und Dauer der nicht-präemtiblen kritischen Abschnitte
  ist auf ein notwendiges Minimum beschränkt. Alle Interrupt-Handler sind als
  eigenständige Threads ausgeführt, Spinlocks durch Sleeping-Spinlocks und Mutexe
  durch sog.~RT-Mutexe ersetzt.

\end{itemize}
\todo{Spinlocks und Mutexe sowie die RT-Varianten dieser erklären!}

Lediglich mit dem vollständig präemtiblen Kernel kann Echtzeit-Verhalten realisiert werden.

% https://wiki.linuxfoundation.org/realtime/documentation/technical_basics/preemption_models bzw kernel/Kconfig.preempt

\subsubsection{preemptRT%
        \label{sec:2-preemptRT}}
% https://wiki.linuxfoundation.org/realtime/documentation/technical_details/start
% https://wiki.linuxfoundation.org/realtime/documentation/technical_basics/start

Das dem PREEMPT RT Kernel zugrunde liegende Prinzip lässt sich in einer einfachen
Regel ausdrücken: Nur Code, welcher absolut nicht-präemtible sein darf, ist es
gestattet nicht-präemtible zu sein.
Das erklärte Ziel des PREEMPT\_RT Patches ist es folglich, die Menge des nicht-präemtiblen
Codes im Linux-Kernel auf das absolut notwendige Minimum zu reduzieren.

Dies wird durch Verwendung folgender Mechanismen erreicht:

\begin{itemize}
  \item Hochauflösende Timer
  \item Sleeping Spinlocks
  \item Threaded Interrupt Handlers
  \item rt\_mutex
  \item RCU
\end{itemize}


\subsubsection{posix%
        \label{sec:2-posix}}
Ist posix hier wirklich relevant? Debian bzw.~Raspbian sind weitgehend posix
kompatibel, aber wird es hier genutzt? -> JA, open62541 nutzt pthread.h
piControl nutzt kthread.h, und semaphore.h

\subsection{OPC-UA und open62541%
     \label{sec:2-opc}}

\subsubsection{OPC UA%
        \label{sec:2-opcua}}
Open Platform Communications (OPC) ist eine Familie von Standards zur herstellerunabhängigen
Kommunikation von Maschinen (M2M) in der Automatisierungstechnik. Die sog.~OPC Task Force, zu deren
Mitgliedern verschiedene große Firmen der Automatisierungsindustrie gehören, veröffentlichte
die OPC Specification Version 1.0 im August 1996.
Motiviert ist dieser offene Standard durch die Erkenntniss, dass die Anpassung der
zahlreichen Herstellerstandards an individuelle Infrastrukturen und Anlagen einen
großen Mehraufwand verursachen.
Die Wikipedia beschreibt das Anwendungsgebiet für OPC wie folgt:

\glqq{}OPC wird dort eingesetzt, wo Sensoren, Regler und Steuerungen verschiedener Hersteller
ein gemeinsames Netzwerk bilden. Ohne OPC benötigten zwei Geräte zum Datenaustausch
genaue Kenntnis über die Kommunikationsmöglichkeiten des Gegenübers. Erweiterungen
und Austausch gestalten sich entsprechend schwierig. Mit OPC genügt es, für jedes
Gerät genau einmal einen OPC-konformen Treiber zu schreiben. Idealerweise wird
dieser bereits vom Hersteller zur Verfügung gestellt. Ein OPC-Treiber lässt sich
ohne großen Anpassungsaufwand in beliebig große Steuer- und Überwachungssysteme
integrieren.

OPC unterteilt sich in verschiedene Unterstandards, die für den jeweiligen Anwendungsfall
unabhängig voneinander implementiert werden können. OPC lässt sich damit verwenden
für Echtzeitdaten (Überwachung), Datenarchivierung, Alarm-Meldungen und neuerdings
auch direkt zur Steuerung (Befehlsübermittlung).\grqq{}

OPC basiert in der ursprünglichen Spezifikation auf Microsofts DCOM-Spezifikation.
DCOM macht Funktionen und Objekte einer Anwendung anderen Anwendungen im Netzwerk
zugänglich. Der OPC-Standard definiert entsprechende DCOM-Objekte um mit anderen
OPC-Anwendungen Daten austauschen zu können. Die Verwendung von DCOM bindet Anwender
an Betriebssysteme von Microsoft. Die ursprüngliche OPC Spezifikation wird durch die
Entwicklung von OPC Unified Architecture (OPC UA) abgelöst.
OPC UA setzt auf einem eigenen Kommunikationionsstack auf, die Verwendung von DCOM
und damit die Bindung an Microsoft wurden aufgelöst.

Die OPC-UA-Architektur ist eine Service-orientierte Architektur (SOA), deren Struktur
aus mehreren Schichten besteht.

% Wikipedia
Das OPC-Informationsmodell ist nicht mehr nur eine Hierarchie aus Ordnern, Items
und Properties. Es ist ein sogenanntes Full-Mesh-Network aus Nodes, mit dem neben
den Nutzdaten eines Nodes auch Meta- und Diagnoseinformationen repräsentiert werden.
Ein Node ähnelt einem Objekt aus der objektorientierten Programmierung. Ein Node
kann Attribute besitzen, die gelesen werden können (Data Access (DA), Historical
Data Access (HDA)). Es ist möglich Methoden zu definieren und aufzurufen.
Eine Methode besitzt Aufrufargumente und Rückgabewerte. Sie wird durch ein Command
aufgerufen. Weiterhin werden Events unterstützt, die versendet werden können
(AE (Alarms \& Events), DA DataChange), um bestimmte Informationen zwischen Geräten
auszutauschen. Ein Event besitzt unter anderem einen Empfangszeitpunkt, eine Nachricht
und einen Schweregrad. Die o. g. Nodes werden sowohl für die Nutzdaten als auch
alle anderen Arten von Metadaten verwendet. Der damit modellierte OPC-Adressraum
beinhaltet nun auch ein Typmodell, mit dem sämtliche Datentypen spezifiziert werden.

% https://de.wikipedia.org/wiki/Open_Platform_Communications
% https://de.wikipedia.org/wiki/OPC_Unified_Architecture
% https://opcfoundation.org/developer-tools/specifications-unified-architecture
% Von Gerhard Gappmeier - ascolab GmbH, CC BY-SA 3.0, https://de.wikipedia.org/w/index.php?curid=1892069
\subsubsection{open62541%
        \label{sec:2-open62541}}
open62541 ist eine offene und freie Implementierung von OPC UA. Die in C geschriebene
Bibliothek stellt eine beständig zunehmende Anzahl der im OPC UA Standard definierten
Funktionen bereit. Sie kann sowohl zur Erstellung von OPC-Servern als auch -Clients
genutzt werden. Ergänzend zu der unter der Mozilla Public License v2.0 lizensierten
Bibliothek stellt das open62541 Projekt auch Beispielprogramme unter einer CC0 Lizenz
zur Verfügung.

Die Bibliothek eignet sich auch für die Entwicklung auf eingebetteten Systemen und
Microcontrollern. Je nach Umfang der gewünschten Funktionen und des OPC Informationsmodells
beträgt die Größe einer Server-Binary weniger als 100kb. %evtl. kürzen?

\todo{Nodes erklären! Evtl.~oben!}

Folgende Auswahl an Eigenschaften und Funktionen zeichnet die in dieser Arbeit verwendete
Version 0.3 von open62541 aus:
\begin{itemize}
  \item Kommunikationionsstack
  \begin{itemize}
      \item OPC UA Binär-Protokoll (HTTP oder SOAP werden gegenwärtig nicht unterstützt)
      \item Austauschbare Netzwerk-Schicht, welche die Verwendung eigener Netzwerk-APIs
      erlaubt.
      \item Verschlüsselte Kommunikationion
      \item Asynchrone Dienst-Anfragen im Client
  \end{itemize}
  \item Informationsmodell
  \begin{itemize}
    \item Unterstützung aller OPC UA Node-Typen, inkl.~Methoden
    \item Hinzufügen und Entfernen von Nodes und Referenzen zur Laufzeit.
    \item Vererbung und Instanziierung von Objekt- und Variablentypen
    \item Zugriffskontrolle auch für einzelne Nodes
  \end{itemize}
  \item Subscriptions
  \begin{itemize}
    \item Erlaubt die Überwachung (subscriptions / monitoreditems)
    \item Sehr geringer Ressourcenbedarf pro überwachtem Wert
  \end{itemize}
  \item Code-Generierung auf XML-Basis
  \begin{itemize}
    \item Erlaubt die Erstellung von Datentypen
    \item Erlaubt die Generierung des serverseitigen Informationsmodells
  \end{itemize}
\end{itemize}

% https://open62541.org/doc/0.3/


Mozilla Public License
CC0 Lizenz für Beispiele und Plugins

% https://open62541.org/doc/open62541-current.pdf
% https://open62541.org/

% % % Imports nur für Referenzenauflösung während des Schreibens! Vorm Kompilieren auskommentieren!
% \bibliography{0_hauptdatei}
% \input{1_einleitung}
% \input{2_grundlagen}
% \input{3_konzeption}
% \input{4_implementierung}
% \input{5_tests}
% \input{6_zusammenfassung}
% \input{anhang}
% % Ende Imports

\section{Systemkonzept%
  \label{sec:3-konzeption}}
Auf Basis der in Abschnitt \ref{sec:2-grundlagen} vorgestellten Möglichkeiten folgt nun die Ausarbeitung eines Konzepts.
In den folgenden Abschnitten soll näher auf zwei zentrale Aspekte eingegangen werden: Abschnitt~\ref{sec:3-anbindung} stellt Möglichkeiten zum Zugriff auf Variablen bzw.\,Werte im Prozessabbild des Revolution Pi vor; in Abschnitt~\ref{sec:3-integration} wird ein Konzept zur Bereitstellung dieser Variablen auf einem OPC-Server vorgestellt.

\subsection{Anbindung der IO an den OPC-Server%
     \label{sec:3-anbindung}}

Eine Webanwendung mit Bezeichnung PiCtory dient zur Konfiguration der I/O- und virtuellen Module des RevolutionPi. Die Konfiguration liegt im JSON-Format in der Datei \lstinline{/etc/revpi/config.rsc}. Der piControl-Treiber liest diese Datei beim Start. 
Der folgende Auszug aus der Manpage des piControl-Kernelmoduls beschreibt die von diesem zum Lesen und Schreiben einzelner Bits des Prozessabbildes bereitgestellten Funktionen~\citep[vgl.]{web-revpi-manpage}. Sie ist an dieser Stelle weitgehend ungekürzt zitiert, da sie die nutzbare Schnittstelle sehr kompakt beschreibt.

\begin{lstlisting}[breakindent=0pt, numbers=none, caption={Auszug aus der Revolution Pi Programmers Manual\label{lst:4-manpage}}]
KB_FIND_VARIABLE SPIVariable *argp
Find a variable in the process image by its name. A pointer to a structure of type SPIVariable must be passed as argument. [...]
The struct SPIVariable [...] is defined as 
typedef struct SPIVariableStr
{
    char strVarName[32]; // Variable name
    uint16_t i16uAddress; // Address of the byte in the process image
    uint8_t i8uBit; // 0-7 bit position, >= 8 whole byte
    uint16_t i16uLength; // length of the variable in bits.
    // Possible values are 1, 8, 16 and 32
} SPIVariable;

Set and get values of the process image
KB_GET_VALUE SPIValue *argp
[...]
KB_SET_VALUE SPIValue *argp
Write one bit or one byte to the process image [...].  This call is more efficient than the usual calls of seek and write because only one function call is necessary. If more than on application are writing bits in one output byte, this call is the only safe way to set a bit without overwriting the other bits because this call is doing a read-modify-write-cycle. 

The struct SPIValue used by this ioctl is defined as
typedef struct SPIValueStr
{
    uint16_t i16uAddress; // Address of the byte in the process image
    uint8_t i8uBit; // 0-7 bit position, >= 8 whole byte
    uint8_t i8uValue; // Value: 0/1 for bit access, whole byte otherwise
} SPIValue;
\end{lstlisting} 

Die oben beschriebenden Funtkionen \lstinline{KB_FIND_VARIABLE}, \lstinline{KB_GET_VALUE} und \lstinline{KB_SET_VALUE} ermöglichen einen einfachen und (lt.\,Manpage) effizienten Zugriff auf einzelne Bits des Prozessabbildes und damit auch auf die IO des RevolutionPi.
Der Zugriff des OPC-Servers auf das Prozessabbild soll daher mittels dieser Funktionen realisiert werden.
\lstinline{KB_FIND_VARIABLE} kann genutzt werden, um Adressen von Variablen im Prozessabbild mittels ihres Namens aufzulösen.
\lstinline{KB_GET_VALUE} und \lstinline{KB_SET_VALUE} ermöglichen den Zugriff auf die Werte dieser Variablen.


\subsection{Integration des OPC-Servers in das System%
     \label{sec:3-integration}}

open62541 bietet drei Möglichkeiten zum Abgleich von Variablen mit dem Prozessabbild~\citep[vgl.][Tutorials - Connecting a Variable with a Physical Process]{web-open62541}:
\begin{itemize}
    \item Manuelles oder zyklisches Aktualisieren
    \item Variable Value Callback
    \item Variable Datasource
\end{itemize}

Die zyklische Aktualisierung eines oder mehrerer Werte nimmt, abhängig von der Zykluszeit, viele Systemressourcen in Anspruch. Value Callbacks ermöglichen es, einen Variablenwert effizienter mit einer Ressource wie etwa einem Prozessabbild zu synchronisieren. An die Variable wird ein Callback angehängt, welches vor jedem Lesen und nach jedem Schreibvorgang ausgeführt wird.
Der Wert der Variablen wird weiterhin im Variablenknoten auf dem OPC-Server gespeichert, der Abgleich mit der verknüpften Ressource erfolgt durch die Callback-Methoden.

Sogenannte Datenquellen gehen noch einen Schritt weiter. Der Server leitet jede Lese- und Schreibanforderung direkt an eine Callback-Funktion weiter. Beim Lesen liefert der Rückruf eine Kopie des aktuellen Wertes. Die Datenquelle muss intern ein eigenes Speichermanagement implementieren.

Der Zugriff auf die Werte des Prozessabbildes erfolgt, wie in Abschnitt~\ref{sec:3-anbindung} beschrieben, über von piControl bereitgestellte Methoden. Um die durch open62541 gepflegte OPC-Datenstruktur und das durch piControl verwaltete Prozessabbild möglichst effektiv verknüpfen zu können, soll diese Interaktion mittels Datenquellen und den zugehörigen Callbacks implementiert werden.
% % % Imports nur für Referenzenauflösung während des Schreibens! Vorm Kompilieren auskommentieren!
% \bibliography{0_hauptdatei}
% \input{1_einleitung}
% \input{2_grundlagen}
% \input{3_konzeption}
% \input{4_implementierung}
% \input{5_tests}
% \input{6_zusammenfassung}
% \input{anhang}
% % Ende Imports

\section{Implementierung%
  \label{sec:4-implementierung}}
Das folgende Kapitel stellt in Auszügen die Implementierung des OPC-Servers sowie die Anbindung an die IO-Module
der SPS dar. Der Schwerpunkt liegt hierbei auf der Funktionsweise des piControl-Treibers und dessen Integration in das Projekt. Abschnitt~\ref{sec:4-picontrol} erklärt die zum Schreibens eines Bits verwendeten Funktionsaufrufe.
Zuvor soll jedoch in Abschnitt~\ref{sec:4-open62541} der Teil des OPC-Servers vorgestellt werden, welcher auf besagten Treiber zugreift. 

\subsection{Implementierung des OPC-Servers%
     \label{sec:4-open62541}}
Wie im vorangegangenen Abschnitt~\ref{sec:3-integration} begründet, soll die Verknüpfung zwischen dem Prozessabbild der SPS und den auf dem OPC-Server bereitgestellten Werten über sog.\,Datenquellen erfolgen. Hierzu ist zunächst eine Callback-Methode zu implementieren, welche bei einem Lese- oder Schreibzugriff auf eine Variable aufgerufen wird. Die Verknüpfung zwischen Callback-Methode und Variable muss manuell erfolgen.

\begin{lstlisting}[language={c},firstnumber=237,caption={Auszug der Methode \lstinline{linkDataSourceVariable} in \lstinline{variables.c}\label{lst:4-linkDataSourceVariable}}]
extern UA_StatusCode
 linkDataSourceVariable(UA_Server *server, UA_NodeId nodeId) {
     bool readonly = false;
     UA_DataSource dataSourceVariable;
     UA_StatusCode rc; |>\setcounter{lstnumber}{254}<|

     dataSourceVariable.read = readDataSourceVariable;
     if (!readonly)
        dataSourceVariable.write = writeDataSourceVariable;
     else
        dataSourceVariable.write = writeReadonlyDataSourceVariable;

     return UA_Server_setVariableNode_dataSource(server, nodeId, dataSourceVariable);
 }
\end{lstlisting}

\begin{figure}[h]
    \centering
    \includegraphics[width=0.42\textwidth]{doc/img/OPC_RevPiDO.pdf}
    \caption{Auszug des verwendeten Nodesets, hier Digitalausgang 1 des Versuchsaufbaus
      \label{fig:opc-do}}
\end{figure}

Die in Listing~\ref{lst:4-linkDataSourceVariable} abgebildete Methode \lstinline{linkDataSourceVariable()} erzeugt ein Struct vom Typ \lstinline{UA_DataSource}. In diesem werden dem Lesen und Schreiben einer OPC-Variablen entsprechende Callback-Methoden zugewiesen. Die Verknüpfung einer OPC-Variable, genauer ihrer NodeId, mit der zuvor definierten Datenquelle erfolgt über die von open62541 bereitgestellte Methode \lstinline{UA_Server_setVariableNode_dataSource()}. Vor dem Lesen und nach dem Schreiben dieser Variable werden von nun an die entsprechenden Callbacks aufgerufen.
     
\begin{lstlisting}[language={c},firstnumber=168,caption={Auszug des Callbacks \lstinline{writeDataSourceVariable} in \lstinline{variables.c}\label{lst:4-writeDataSourceVariable}}]  
extern UA_StatusCode
 writeDataSourceVariable(UA_Server *server,
            const UA_NodeId *sessionId, void *sessionContext,
            const UA_NodeId *nodeId, void *nodeContext,
            const UA_NumericRange *range, const UA_DataValue *dataValue) {

    UA_StatusCode retval  = UA_STATUSCODE_GOOD;
    UA_NodeId *nameNodeId = UA_malloc(sizeof(UA_NodeId));
    UA_QualifiedName nameQN = UA_QUALIFIEDNAME(1, "Name");
    UA_Variant nameVar;
    UA_Boolean bit;

    retval |= findSiblingByBrowsename(server, nodeId, &nameQN, nameNodeId);
    retval |= UA_Server_readValue(server, *nameNodeId, &nameVar);
    retval |= UA_Boolean_copy(dataValue->value.data, &bit);

    |>\tikzmarkin[set border color=martinired]{writeIO}<|PI_writeSingleIO(String_fromUA_String(nameVar.data), &bit, false);                                                 |>\tikzmarkend{writeIO}<|

    free(nameNodeId);
    return retval;
 }
\end{lstlisting}

Listing~\ref{lst:4-writeDataSourceVariable} zeigt die Callback-Methode, welche nach dem Schreiben einer Variablen auf dem OPC-Server aufgerufen wird.
Dieser Methode wird neben der NodeId der mit ihr verknüpften Variablen auch der Wert dieser in Form eines Zeigers auf ein Struct vom Typ \lstinline{UA_DataValue} übergeben.

Die Gestaltung des hier verwendeten Nodesets sieht vor, dass in einer OPC-Variablen \lstinline{"Name"} der Bezeichner des zu schreibenden Digitalausgangs hinterlegt ist, siehe Abbildung~\ref{fig:opc-do}. Dies erlaubt eine Rekonfiguration der Ein- und Ausgänge der SPS ohne Änderungen im Programmcode des OPC-Servers vornehmen zu müssen.
Es ist daher erforderlich, nach jedem Schreiben einer mit einem Digitalausgang verknüpften Variablen, hier \lstinline{"Value"}, dessen Bezeichner \lstinline{"Name"} abzufragen. 
Dies geschieht in den Zeilen 180 und 181.
Anschließend wird dieser Bezeichner sowie der zu schreibende Wert der Methode \lstinline{PI_writeSingleIO()} übergeben, welche wiederum die Interaktion mit piControl übernimmt (vgl. Abschnitt \ref{sec:4-picontrol}).
 
\subsection{Integration von piControl%
     \label{sec:4-picontrol}}
In Abschnitt~\ref{sec:2-io} wurde die Anbindung der IO-Module des Revolution Pi sowie die Funktionsweise von piControl aus Anwendersicht beschrieben. Die verfügbare Literatur beschränkt sich auch auf lediglich diese Sicht; eine weiterführende Dokumentation für Entwickler gibt es, neben der in Abschnitt~\ref{sec:3-anbindung} vorgestellten Manpage, nicht. 
In diesem Abschnitt soll daher der Quellcode von piControl sowie dessen Verwendung im Projekt genauer betrachtet werden.
Hierzu wird exemplarisch die in Abschnitt~\ref{sec:4-open62541} eingeführte Methode \lstinline{PI_writeSingleIO()} untersucht.
Diese Methode ermöglicht das Setzen eines einzelnen Bits im Prozessabbild der SPS, und damit das Schalten eines digitalen Ausgangs auf einem IO-Modul.
Die äquivalente Methode \lstinline{int piControlGetBitValue(SPIValue *pSpiValue)} zum Lesen eines Bits bzw. Eingangs funktioniert analog und soll daher an dieser Stelle nicht dediziert erörtert werden.

\begin{lstlisting}[language={c},firstnumber=97,
                   caption={Setzen eines phsikalischen, digitalen Ausgangs in \lstinline{revpi.c}
                   \label{lst:4-PI_writeSingleIO}}]
extern void PI_writeSingleIO(char *pszVariableName, bool *bit, bool verbose)
{
	int rc;
	SPIVariable sPiVariable;
	SPIValue sPIValue;

	strncpy(sPiVariable.strVarName, pszVariableName, sizeof(sPiVariable.strVarName));
	rc = piControlGetVariableInfo(&sPiVariable);
	if (rc < 0) {
		printf("Cannot find variable '%s'\n", pszVariableName);
		return;
	}

		sPIValue.i16uAddress = sPiVariable.i16uAddress;
		sPIValue.i8uBit = sPiVariable.i8uBit;
		sPIValue.i8uValue = *bit;
		rc = |>\tikzmarkin[set border color=martinired]{setBitValue}<|piControlSetBitValue(&sPIValue)|>\tikzmarkend{setBitValue}<|;
		if (rc < 0)
			printf("Set bit error %s\n", getWriteError(rc));
		else if (verbose)
			printf("Set bit %d on byte at offset %d. Value %d\n", sPIValue.i8uBit, sPIValue.i16uAddress,
			       sPIValue.i8uValue);
}
\end{lstlisting}

Der Programmcode in Listing~\ref{lst:4-PI_writeSingleIO} ist Teil des implementierten OPC-Servers. In diesem wird auf zwei Funktionen des piControl-Treibers zugegriffen. 
Beiden Methoden wird als Argument ein Zeiger auf ein Struct vom Typ \lstinline{SPIValue} übergeben. Der im Struct abgelegte Name wird mittels \lstinline{piControlGetVariableInfo(&sPIValue)} zu einer Adresse im Prozessabbild aufgelöst. Diese wird in \lstinline{sPIValue.i16uAdress} gespeichert. Der Wert der Variablen wird anschließend mittels \lstinline{piControlSetBitValue(&sPIValue)} an dieser Adresse in das Prozessabbild geschrieben.

\begin{lstlisting}[language={c},firstnumber=309,caption={Methode \lstinline{piControlSetBitValue} in \lstinline{piControlIf.c}\label{lst:4-piControlSetBitValue}}]
int |>\tikzmarkin[set border color=martiniblue]{setBitValueFcn}<|piControlSetBitValue(SPIValue *pSpiValue)|>\tikzmarkend{setBitValueFcn}<|
{
    piControlOpen();

    if (PiControlHandle_g < 0)
	    return -ENODEV;

    pSpiValue->i16uAddress += pSpiValue->i8uBit / 8;
    pSpiValue->i8uBit %= 8;

    if (|>\tikzmarkin[set border color=martinired]{ioctl}<|ioctl(PiControlHandle_g, KB_SET_VALUE, pSpiValue)|>\tikzmarkend{ioctl}<| < 0)
	    return errno;

    return 0;
}
\end{lstlisting}

Die in Listing~\ref{lst:4-piControlSetBitValue} dargestellte Methode \lstinline{piControlSetBitValue} ist lediglich eine Hüllfunktion (häufig auch als Wrapper-Funktion bezeichnet) für einen Aufruf des \lstinline{ioctl} Kernel-Moduls.
Folgende Parameter werden übergeben:
\lstinline{PiControlHandle_g} ist die Referenz auf die Geräte-Datei des piControl-Treibers. \lstinline{KB_SET_VALUE} ist das ioctl-Kommando zum Schreiben eines Bits in das Prozessabbild. Der Zeiger \lstinline{pSpiValue} verweist auf ein Struct des bereits vorgestellten Typs \lstinline{SPIValue}.

\begin{lstlisting}[language={c},firstnumber=80,caption={Methode \lstinline{piControlOpen} in \lstinline{piControlIf.c}\label{lst:4-piControlOpen}}]
void piControlOpen(void)
{
    /* open handle if needed */
    if (PiControlHandle_g < 0)
    {
	    |>\tikzmarkin[set border color=martiniblue]{PiControlHandle}<|PiControlHandle_g = open(PICONTROL_DEVICE, O_RDWR)|>\tikzmarkend{PiControlHandle}<|;
    }
}
\end{lstlisting}

Die in Listing~\ref{lst:4-piControlOpen} dargestellte Methode öffnet, sofern nicht bereits geschehen, die Geräte-Datei. Das Macro \lstinline{PICONTROL_DEVICE} verweist hierbei auf \lstinline{/dev/piControl0}.

\begin{lstlisting}[language={c},firstnumber=721,caption={Methode \lstinline{piControlIoctl} in \lstinline{piControlMain.c}\label{lst:4-piControlIoctl}}]
static long |>\tikzmarkin[set border color=martiniblue, below offset=0.9em]{piControlIoctl}<|piControlIoctl(struct file *file, unsigned int prg_nr, 
                           unsigned long usr_addr)                                      |>\tikzmarkend{piControlIoctl}<|
{
  int status = -EFAULT;
  tpiControlInst *priv;
  int timeout = 10000;	// ms

  if (prg_nr != KB_CONFIG_SEND && prg_nr != KB_CONFIG_START && !isRunning()) {
  	return -EAGAIN;
  }

  priv = (tpiControlInst *) file->private_data;

  if (prg_nr != KB_GET_LAST_MESSAGE) {
  	// clear old message
  	priv->pcErrorMessage[0] = 0;
  }

  switch (prg_nr) {|>\setcounter{lstnumber}{864}<|

    case |>\tikzmarkin[set border color=martiniblue]{KB_SET_VALUE}<|KB_SET_VALUE:|>\tikzmarkend{KB_SET_VALUE}<|
  		{
  			SPIValue *pValue = (SPIValue *) usr_addr;

  			if (!isRunning())
  				return -EFAULT;

  			if (pValue->i16uAddress >= KB_PI_LEN) {
  				status = -EFAULT;
  			} else {
  				INT8U i8uValue_l;
  				my_rt_mutex_lock(&piDev_g.lockPI);
  				i8uValue_l = piDev_g.ai8uPI[pValue->i16uAddress];

  				if (pValue->i8uBit >= 8) {
  					i8uValue_l = pValue->i8uValue;
  				} else {
  					if (pValue->i8uValue)
  						i8uValue_l |= (1 << pValue->i8uBit);
  					else
  						i8uValue_l &= ~(1 << pValue->i8uBit);
  				}

  				|>\tikzmarkin[set border color=martinired]{i8uValue}<|piDev_g.ai8uPI[pValue->i16uAddress] = i8uValue_l;|>\tikzmarkend{i8uValue}<|
  				rt_mutex_unlock(&piDev_g.lockPI);

  #ifdef VERBOSE
  				pr_info("piControlIoctl Addr=%u, bit=%u: %02x %02x\n", pValue->i16uAddress, pValue->i8uBit, pValue->i8uValue, i8uValue_l);
  #endif

  				status = 0;
  			}
  		}
  		break; |>\setcounter{lstnumber}{1314}<|

    default:
      pr_err("Invalid Ioctl");
      return (-EINVAL);
      break;

    }

    return status;
  }
\end{lstlisting}

Listing~\ref{lst:4-piControlIoctl} zeigt in Auszügen die ioctl-Methode des piControl Kernel-Treibers. Diese bekommt folgende Argumente übergeben: \lstinline{struct file *file} enthält den Verweis auf die Geräte-Datei, hier \lstinline{/dev/piControl0}. Der Wert von \lstinline{unsigned int prg_nr} beschreibt die Anfrage an den Treiber, in diesem Fall \lstinline{KB_SET_VALUE}. Das Argument \lstinline{unsigned long usr_addr} enthält einen typ-agnostischen Pointer. Dieser verweist auf einen Speicherbereich, in welchem die zur Bearbeitung der Anfrage notwendigen Daten abgelegt sind. Hier können auch vom Treiber empfangene Daten dem Anwendungsprogramm bereitgestellt werden. 

Die switch-case-Anweisung führt die über das Argument \lstinline{prg_nr} spezifizierte Aktion aus. Hier betrachten wir \lstinline{KB_SET_VALUE}:
Zunächst wird in Zeile 868 der übergebene Zeiger \lstinline{usr_addr} mittels explizitem Typecast zu einem Zeiger des Typs \lstinline{SPIValue *} konvertiert. Da dieser auf Daten im Userspace verweist, ist beim Zugriff durch den Kernel-Treiber besondere Vorsicht geboten.
In Zeile 877 wird mittels Mutex das Prozessabbild \lstinline{piDev_g} für den Zugriff durch andere Threads oder Prozesse gesperrt.
\lstinline{my_rt_mutex_lock} verweist hierbei auf die Funktion \lstinline{rt_mutex_lock} aus \lstinline{linux/sched.h}\footnote{Offenbar wurde hier auch eine alternative Implementierung vorgesehen, siehe revpi\_common.h}

In Zeile 889 wird das Byte \lstinline{i8uValue_l}, welches den zu schreibenden Wert enthält in das Prozessabbild übertragen. Anschließend wird die Mutex auf \lstinline{piDev_g} wieder entsperrt.
\newpage

\begin{lstlisting}[language={c},firstnumber=62,caption={Auszug des Struct \lstinline{spiControlDev} in \lstinline{piControlMain.h}\label{lst:4-spiControlDev}}]
|>\tikzmarkin[set border color=martiniblue]{spiControlDev}<|typedef struct spiControlDev|>\tikzmarkend{spiControlDev}<| {
	// device driver stuff
	int init_step;
	enum revpi_machine machine_type;
	void *machine;
	struct cdev cdev;	// Char device structure
	struct device *dev;
	struct thermal_zone_device *thermal_zone;

	|>\tikzmarkin[set border color=martiniblue]{processImage}<|// process image stuff
	INT8U ai8uPI[KB_PI_LEN];
	INT8U ai8uPIDefault|>\tikzmarkin[set border color=martinired]{KB_PI_LEN_0}<|[KB_PI_LEN]|>\tikzmarkend{KB_PI_LEN_0}<|;
	struct rt_mutex lockPI;        |>\tikzmarkend{processImage}<|
	bool stopIO;
	piDevices *devs; |>\setcounter{lstnumber}{94}<|
} tpiControlDev;
\end{lstlisting}

Das Prozessabbild ist als Byte-Array der Länge \lstinline{KB_PI_LEN} in Listing~\ref{lst:4-spiControlDev} definiert. Konfigurationsparameter wie \lstinline{KB_PI_LEN} oder die Zykluszeit für den Datenaustausch zwischen SPS und IO-Modulen sind im folgenden Listing~\ref{lst:4-process} definiert.

\begin{lstlisting}[language={c},firstnumber=119,caption={Konfigurationsparameter des Prozessabbildes in project.h\label{lst:4-process}}]
#define INTERVAL_PI_GATE (5*1000*1000)  // 5 ms piGateCommunication |>\setcounter{lstnumber}{128}<|

#define INTERVAL_IO_COM (5*1000*1000)  // 5 ms piIoComm |>\setcounter{lstnumber}{132}<|

#define KB_PD_LEN       512
|>\tikzmarkin[set border color=martiniblue]{KB_PI_LEN_1}<|#define KB_PI_LEN       4096|>\tikzmarkend{KB_PI_LEN_1}<|
\end{lstlisting}

Das zu setzende Bit wurde zu diesem Zeitpunkt erfolgreich in das Prozessabbild der SPS geschrieben.
Es stellt sich die Frage, wie dieses nun an das IO-Modul kommuniziert wird.
Die Kommunikation mit allen angebundenen Modulen ist ebenfalls Aufgabe des piControl-Treibers.

\begin{lstlisting}[language={c},firstnumber=256,caption={Auszug der Methode \lstinline{piIoThread} in \lstinline{revpi_core.c}\label{lst:4-piIoThread}}]
static int piIoThread(void *data)
{
	//TODO int value = 0;
	ktime_t time;
	ktime_t now;
	s64 tDiff;

	hrtimer_init(&piCore_g.ioTimer, CLOCK_MONOTONIC, HRTIMER_MODE_ABS);
	piCore_g.ioTimer.function = piIoTimer;

	pr_info("piIO thread started\n");

	now = hrtimer_cb_get_time(&piCore_g.ioTimer);

	PiBridgeMaster_Reset();

	while (!kthread_should_stop()) {
		if (|>\tikzmarkin[set border color=martinired]{PiBridgeMaster}<|PiBridgeMaster_Run()|>\tikzmarkend{PiBridgeMaster}<| < 0)
			break;
	}

	RevPiDevice_finish();

	pr_info("piIO exit\n");
	return 0;
}
\end{lstlisting}

Der Kernel-Thread \lstinline{piIoThread} ist verantwortlich für den zyklischen Datenaustausch mit den IO-Modulen. In diesem wird fortlaufend die Methode \lstinline{PiBridgeMaster_Run()} aufgerufen, siehe Listing~\ref{lst:4-piIoThread}.

\begin{lstlisting}[language={c},firstnumber=262,caption={Auszug der Methode \lstinline{PiBridgeMaster_Run(void)} in \lstinline{RevPiDevice.c}\label{lst:4-PiBridgeMaster_Run}}]
int PiBridgeMaster_Run(void)
{
	static kbUT_Timer tTimeoutTimer_s;
	static kbUT_Timer tConfigTimeoutTimer_s;
	static int error_cnt;
	static INT8U last_led;
	static unsigned long last_update;
	int ret = 0;
	int i;

	my_rt_mutex_lock(&piCore_g.lockBridgeState);
	if (piCore_g.eBridgeState != piBridgeStop) {
		switch (eRunStatus_s) { |>\setcounter{lstnumber}{514}<|
		    case enPiBridgeMasterStatus_EndOfConfig:|>\setcounter{lstnumber}{621}<|
		    if (|>\tikzmarkin[set border color=martinired]{RevPiDevice}<|RevPiDevice_run()|>\tikzmarkend{RevPiDevice}<|) {
				// an error occured, check error limits |>\setcounter{lstnumber}{641}<|
			} else {
				ret = 1;
			}
			piCore_g.image.drv.i16uRS485ErrorCnt = RevPiDevice_getErrCnt();
			break;
\end{lstlisting}

Die in Listing~\ref{lst:4-PiBridgeMaster_Run} dargestellte Methode ist eine sog. State-Machine. Ist die Konfiguration der IO-Module erfolgreich abgeschlossen, so führt sie bei Aufruf lediglich die Methode \lstinline{RevPiDevice_run()} aus.

\begin{lstlisting}[language={c},firstnumber=140,caption={Auszug der Methode \lstinline{RevPiDevice_run(void)} in \lstinline{RevPiDevice.c}\label{lst:4-RevPiDevice_run}}]
int RevPiDevice_run(void)
{
	INT8U i8uDevice = 0;
	INT32U r;
	int retval = 0;

	RevPiDevices_s.i16uErrorCnt = 0;

	for (i8uDevice = 0; i8uDevice < RevPiDevice_getDevCnt(); i8uDevice++) {
		if (RevPiDevice_getDev(i8uDevice)->i8uActive) {
			switch (RevPiDevice_getDev(i8uDevice)->sId.i16uModulType) {
			case KUNBUS_FW_DESCR_TYP_PI_DIO_14:
			case KUNBUS_FW_DESCR_TYP_PI_DI_16:
			case KUNBUS_FW_DESCR_TYP_PI_DO_16:
				r = |>\tikzmarkin[set border color=martinired]{sendCyclicTelegram}<|piDIOComm_sendCyclicTelegram(i8uDevice)|>\tikzmarkend{sendCyclicTelegram}\setcounter{lstnumber}{166} <|;

				break; |>\setcounter{lstnumber}{216}<|
			}
		}
	} |>\setcounter{lstnumber}{227}<|
	return retval;
}
\end{lstlisting}

Diese iteriert wie in Listing~\ref{lst:4-RevPiDevice_run} abgebildete durch alle gegenwärtig in der SPS konfigurierten Module. Ist das aktuelle Modul als aktiv markiert, so wird anhand eines sog. Firmware-Descriptors entschieden, welche Methode für die Ansteuerung des Moduls aufzurufen ist.

\begin{lstlisting}[language={c},firstnumber=161,caption={Auszug der Methode \lstinline{piDIOComm_sendCyclicTelegram} in \lstinline{piDIOComm.c}\label{lst:4-sendCyclicTelegram}}]
INT32U piDIOComm_sendCyclicTelegram(INT8U i8uDevice_p)
{
	INT32U i32uRv_l = 0;
	SIOGeneric sRequest_l;
	SIOGeneric sResponse_l;
	INT8U len_l, data_out[18], i, p, data_in[70];
	INT8U i8uAddress;
	int ret; |>\setcounter{lstnumber}{239}<|
	
    |>\tikzmarkin[set border color=martinired]{piIoComm}<|ret = piIoComm_send((INT8U *) & sRequest_l, IOPROTOCOL_HEADER_LENGTH + len_l + 1);  |>\tikzmarkend{piIoComm}\setcounter{lstnumber}{298}<|
}
\end{lstlisting}

Im Falle des hier verwendeten DO-Moduls wird die in Listing~\ref{lst:4-sendCyclicTelegram} abgebildete Methode \lstinline{piDIOComm_sendCyclicTelegram()} aufgerufen. Dieser wird ein Zeiger auf das zu schreibende Gerät übergeben. 
Zunächst wird das Prozessabbild mittels eines proprietären, jedoch im Quellcode offen nachvollziehbaren Protokolls in ein \lstinline{sRequest_l} genanntes Byte-Array umgewandelt. Dieser Schritt ist in Listing~\ref{lst:4-sendCyclicTelegram} nicht abgebildet. Anschließend wird \lstinline{piIoComm_send()} ein Zeiger auf die so generierte Schreib-Anfrage übergeben.

\begin{lstlisting}[language={c},firstnumber=220,caption={Auszug der Methode \lstinline{piIOComm_send} in \lstinline{piIOComm.c}\label{lst:4-piIOComm_send}}]
int piIoComm_send(INT8U * buf_p, INT16U i16uLen_p)
{
	ssize_t write_l = 0;
	INT16U i16uSent_l = 0;|>\setcounter{lstnumber}{249}<|

	while (i16uSent_l < i16uLen_p) {
		write_l = vfs_write(piIoComm_fd_m, buf_p + i16uSent_l, i16uLen_p - i16uSent_l, &piIoComm_fd_m->f_pos);
		if (write_l < 0) {
			pr_info_serial("write error %d\n", (int)write_l);
			return -1;
		} 
		i16uSent_l += write_l;|>\setcounter{lstnumber}{263}<|
	}
	clear();
	vfs_fsync(piIoComm_fd_m, 1);
	return 0;
}
\end{lstlisting}

Listing~\ref{lst:4-piIOComm_send} zeigt die Implementierung von \lstinline{piIoComm_send()}. Diese Methode ist für das Schreiben der oben generierten Anfrage auf die seriellen Schnittstelle verantwortlich. Realisiert wird dies mittels der Methode \lstinline{vfs_write()}. Diese ist in \lstinline{<linux/fs.h>} definiert. Sie ermöglicht das Schreiben einer Datei im Userspace aus dem Kernel heraus. Geschrieben wird hier die Datei mit dem Deskriptor \lstinline{piIoComm_fd_m}.
Da die Funktion \lstinline{vfs_write()} durch andere Kernel-Tasks unterbrochen werden kann, ist nicht gewährleistet, dass die gesamte Anfrage mit nur einem Aufruf geschrieben wird. Die oben abgebildete while-Schleife stellt das vollständige Senden der Anfrage sicher.

\begin{lstlisting}[language={c},firstnumber=157,caption={Auszug der Methode \lstinline{piIOComm_open_serial} in \lstinline{piIOComm.c}\label{lst:4-piIOComm_open_serial}}]
int piIoComm_open_serial(void)
{   |>\setcounter{lstnumber}{167}<|
	struct file *fd;	/* Filedeskriptor */
	struct termios newtio;	/* Schnittstellenoptionen */

	|>\tikzmarkin[set border color=martiniblue]{fd}<|/* Port oeffnen - read/write, kein "controlling tty", 
	    Status von DCD ignorieren */
	fd = filp_open(|>\tikzmarkin[set border color=martinired]{tty}<|REV_PI_TTY_DEVICE|>\tikzmarkend{tty}<|, O_RDWR | O_NOCTTY, 0); |>\setcounter{lstnumber}{208}<|
	
	piIoComm_fd_m = fd;                                                      |>\tikzmarkend{fd}\setcounter{lstnumber}{217}<|

	return 0;
}
\end{lstlisting}

Der zum Schreiben auf die serielle Schnittstelle verwendete Datei-Deskriptor wird von der in Listing~\ref{lst:4-piIOComm_open_serial} abgebildeten Methode \lstinline{piIoComm_open_serial()} generiert. 

\begin{lstlisting}[language={c},firstnumber=45,caption={Definition der seriellen Schnittstelle in \lstinline{piIOComm.h}\label{lst:4-REV_PI_TTY_DEVICE}}]
#define REV_PI_TTY_DEVICE	"/dev/ttyAMA0"
\end{lstlisting}

Das in Listing~\ref{lst:4-REV_PI_TTY_DEVICE} definierte Macro verweist auf eine der seriellen Schnittstellen des RaspberryPi.
Die Implementierung des zugehörigen Schnittstellentreibers soll hier nicht weiter untersucht werden. Somit ist an dieser Stelle die Kette vom Setzen einer Variablen auf dem OPC-Server bis hin zur Aktualisierung des Prozessabbilds der IO-Module geschlossen.

% \begin{lstlisting}[language={c},firstnumber={226},caption={Setzen der Scheduler-Priorität auf SCHED\_FIFO in 
% revpi\_common.c\label{lst:2-sched_priority}}]
% param.sched_priority = ktprio->prio;
% ret = sched_setscheduler(child, SCHED_FIFO, &param);
% \end{lstlisting}
% % % Imports nur für Referenzenauflösung während des Schreibens! Vorm Kompilieren auskommentieren!
% \bibliography{0_hauptdatei}
% \input{1_einleitung}
% \input{2_grundlagen}
% \input{3_konzeption}
% \input{4_implementierung}
% \input{5_tests}
% \input{6_zusammenfassung}
% % Ende Imports

\section{Test des OPC-Servers im Gesamtsystem%
  \label{sec:5-tests}}

% % % Imports nur für Referenzenauflösung während des schreibens! Vorm Kompilieren auskommentieren!
% \bibliography{0_hauptdatei}
% \input{1_einleitung}
% \input{2_grundlagen}
% \input{3_konzeption}
% \input{4_implementierung}
% \input{5_tests}
% \input{6_zusammenfassung}
% % Ende Imports

\section{Zusammenfassung und Ausblick%
  \label{sec:6-fazit}}
Der folgende Abschnitt~\ref{sec:6-zusammenfassung} fasst die gewonnenen Erkenntnisse und den Stand der Implementierung zusammen.
Den Abschluss dieser Arbeit bildet der Ausblick in Abschnitt~\ref{sec:6-ausblick}.

\subsection{Zusammenfassung%
     \label{sec:6-zusammenfassung}}

\subsection{Ausblick%
     \label{sec:6-ausblick}}

% \input{anhang}
% % Ende Imports

\section{Systemkonzept%
  \label{sec:3-konzeption}}
Auf Basis der in Abschnitt \ref{sec:2-grundlagen} vorgestellten Möglichkeiten folgt nun die Ausarbeitung eines Konzepts.
In den folgenden Abschnitten soll näher auf zwei zentrale Aspekte eingegangen werden: Abschnitt~\ref{sec:3-anbindung} stellt Möglichkeiten zum Zugriff auf Variablen bzw.\,Werte im Prozessabbild des Revolution Pi vor; in Abschnitt~\ref{sec:3-integration} wird ein Konzept zur Bereitstellung dieser Variablen auf einem OPC-Server vorgestellt.

\subsection{Anbindung der IO an den OPC-Server%
     \label{sec:3-anbindung}}

Eine Webanwendung mit Bezeichnung PiCtory dient zur Konfiguration der I/O- und virtuellen Module des RevolutionPi. Die Konfiguration liegt im JSON-Format in der Datei \lstinline{/etc/revpi/config.rsc}. Der piControl-Treiber liest diese Datei beim Start. 
Der folgende Auszug aus der Manpage des piControl-Kernelmoduls beschreibt die von diesem zum Lesen und Schreiben einzelner Bits des Prozessabbildes bereitgestellten Funktionen~\citep[vgl.]{web-revpi-manpage}. Sie ist an dieser Stelle weitgehend ungekürzt zitiert, da sie die nutzbare Schnittstelle sehr kompakt beschreibt.

\begin{lstlisting}[breakindent=0pt, numbers=none, caption={Auszug aus der Revolution Pi Programmers Manual\label{lst:4-manpage}}]
KB_FIND_VARIABLE SPIVariable *argp
Find a variable in the process image by its name. A pointer to a structure of type SPIVariable must be passed as argument. [...]
The struct SPIVariable [...] is defined as 
typedef struct SPIVariableStr
{
    char strVarName[32]; // Variable name
    uint16_t i16uAddress; // Address of the byte in the process image
    uint8_t i8uBit; // 0-7 bit position, >= 8 whole byte
    uint16_t i16uLength; // length of the variable in bits.
    // Possible values are 1, 8, 16 and 32
} SPIVariable;

Set and get values of the process image
KB_GET_VALUE SPIValue *argp
[...]
KB_SET_VALUE SPIValue *argp
Write one bit or one byte to the process image [...].  This call is more efficient than the usual calls of seek and write because only one function call is necessary. If more than on application are writing bits in one output byte, this call is the only safe way to set a bit without overwriting the other bits because this call is doing a read-modify-write-cycle. 

The struct SPIValue used by this ioctl is defined as
typedef struct SPIValueStr
{
    uint16_t i16uAddress; // Address of the byte in the process image
    uint8_t i8uBit; // 0-7 bit position, >= 8 whole byte
    uint8_t i8uValue; // Value: 0/1 for bit access, whole byte otherwise
} SPIValue;
\end{lstlisting} 

Die oben beschriebenden Funtkionen \lstinline{KB_FIND_VARIABLE}, \lstinline{KB_GET_VALUE} und \lstinline{KB_SET_VALUE} ermöglichen einen einfachen und (lt.\,Manpage) effizienten Zugriff auf einzelne Bits des Prozessabbildes und damit auch auf die IO des RevolutionPi.
Der Zugriff des OPC-Servers auf das Prozessabbild soll daher mittels dieser Funktionen realisiert werden.
\lstinline{KB_FIND_VARIABLE} kann genutzt werden, um Adressen von Variablen im Prozessabbild mittels ihres Namens aufzulösen.
\lstinline{KB_GET_VALUE} und \lstinline{KB_SET_VALUE} ermöglichen den Zugriff auf die Werte dieser Variablen.


\subsection{Integration des OPC-Servers in das System%
     \label{sec:3-integration}}

open62541 bietet drei Möglichkeiten zum Abgleich von Variablen mit dem Prozessabbild~\citep[vgl.][Tutorials - Connecting a Variable with a Physical Process]{web-open62541}:
\begin{itemize}
    \item Manuelles oder zyklisches Aktualisieren
    \item Variable Value Callback
    \item Variable Datasource
\end{itemize}

Die zyklische Aktualisierung eines oder mehrerer Werte nimmt, abhängig von der Zykluszeit, viele Systemressourcen in Anspruch. Value Callbacks ermöglichen es, einen Variablenwert effizienter mit einer Ressource wie etwa einem Prozessabbild zu synchronisieren. An die Variable wird ein Callback angehängt, welches vor jedem Lesen und nach jedem Schreibvorgang ausgeführt wird.
Der Wert der Variablen wird weiterhin im Variablenknoten auf dem OPC-Server gespeichert, der Abgleich mit der verknüpften Ressource erfolgt durch die Callback-Methoden.

Sogenannte Datenquellen gehen noch einen Schritt weiter. Der Server leitet jede Lese- und Schreibanforderung direkt an eine Callback-Funktion weiter. Beim Lesen liefert der Rückruf eine Kopie des aktuellen Wertes. Die Datenquelle muss intern ein eigenes Speichermanagement implementieren.

Der Zugriff auf die Werte des Prozessabbildes erfolgt, wie in Abschnitt~\ref{sec:3-anbindung} beschrieben, über von piControl bereitgestellte Methoden. Um die durch open62541 gepflegte OPC-Datenstruktur und das durch piControl verwaltete Prozessabbild möglichst effektiv verknüpfen zu können, soll diese Interaktion mittels Datenquellen und den zugehörigen Callbacks implementiert werden.
% % % Imports nur für Referenzenauflösung während des Schreibens! Vorm Kompilieren auskommentieren!
% \bibliography{0_hauptdatei}
% % Mit \section{...} eröffnen wir einen neuen Abschnitt.
% Der Befehl setzt nicht nur den Text in einer größeren,
% fetten Schrift, sondern sorgt außerdem dafür, daß er im
% Inhaltsverzeichnis erscheint.
%
% Mit \label{...} erzeugen wir einen Bezeichner, mit dessen Hilfe
% wir später auf die Nummer des Abschnitts verweisen können (nämlich
% mit~\ref{...}).
%
% Das Kommentarzeichen hinter „Übersicht“ dient dazu, ein
% Leerzeichen zwischen „Übersicht“ und dem \label-Befehl
% zu vermeiden, das andernfalls sichtbar würde – z.B. im
% Inhaltsverzeichnis.
%

% % Imports nur für Referenzenauflösung während des Schreibens! Vorm Kompilieren auskommentieren!
% \bibliography{0_hauptdatei}
% \input{1_einleitung}
%\input{2_grundlagen}
%\input{3_konzeption}
%\input{4_implementierung}
%\input{5_tests}
%\input{6_zusammenfassung}
% % Ende Imports

\section{Einleitung und Motivation%
  \label{sec:1-einleitung}}
Ziel dieses Projektes ist die Integration eines OPC-Servers mit einer auf Linux
basierenden speicherprogrammierbaren Steuerung (SPS). Angeschlossen an diese SPS
ist jeweils ein digitales Ein-/\,bzw.~Ausgabemodul. Die von diesen bereitgestellten
Ein-/\, bzw.~Ausgänge (IO) sollen in der Datenstruktur des OPC-Servers abgebildet
und über diesen für OPC-Clients les-/\,und schreibar sein. Weiterhin sollen einige
Funktionen zur Überwachung und Steuerung der an die SPS angeschlossenen Aktoren
und Sensoren direkt im OPC-Server implementiert werden.
Hiermit stellt dieses Projekt eine der Grundlagen für ein übergeordnetes Projekt,
die cloudbasierte Steuerung eines miniaturisierten Produktions-Systems, dar.

Der hier verwendete OPC-Server ist Teil des sog. open62541 Projekts. Er ist in C
geschrieben und implementiert bereits einen großen Teil der im OPC-UA-Standard
spezifizierten Funktionen.
Als SPS findet ein Revolution Pi 3 der Firma Kunbus Verwendung. Dieser integriert
ein sog. Compute Module der Raspberry Pi Foundation in ein industrietaugliches
Gehäuse und erlaubt die Erweiterung mittels IO- oder Gateway-Modulen. Über diese
erfolgt die Kommunikation mit weiteren Komponenten der Automatisierungstechnik.

Motiviert ist dieses Projekt durch die Beobachtung, dass die Verbreitung offener
Standards sowie freier Software auch in der Automatisierungstechnik zunimmt.
Linux ist ein freies Betriebssystem, OPC-UA ein offen zugänglicher, aktiv gepflegter
und weit verbreiteter Standard. Der Raspberry Pi findet sowohl bei Hobby-Anwendern als
auch in den Bereichen Forschung und Entwicklung sowie bei industriellen Anwendern
Verwendung. Dieses Projekt stellt somit eine für unterschiedliche Anwender interessante
Entwicklung dar.

Im Anschluss an diese einleitende Übersicht im Abschnitt~\ref{sec:1-einleitung} folgt
die Darstellung der wichtigsten Grundlagen in Abschnitt~\ref{sec:2-grundlagen}.
Aufbauend auf diesen Grundlagen folgt die konzeptuelle Ausarbeitung im Abschnitt~\ref{sec:3-konzeption}.
Die Umsetzung wird im Abschnitt~\ref{sec:4-implementierung} erläutert.
Die Leistungsfähigkeit der Implementierung wird in Abschnitt~\ref{sec:5-tests} untersucht.
Eine Zusammenfassung und ein Ausblick schließen die Arbeit in
Abschnitt~\ref{sec:6-fazit} ab. Eventuell noch benötigte Anhänge
finden sich in den Anhängen [...] bis [...].

% % % Imports nur für Referenzenauflösung während des Schreibens! Vorm Kompilieren auskommentieren!
% \bibliography{0_hauptdatei}
% \input{1_einleitung}
% \input{2_grundlagen}
% \input{3_konzeption}
% \input{4_implementierung}
% \input{5_tests}
% \input{6_zusammenfassung}
% % Ende Imports

\section{Grundlagen%
  \label{sec:2-grundlagen}}

\subsection{Speicherprogrammierbare-Steuerung und Linux -- Revolution Pi%
     \label{sec:2-sps}}

\subsubsection{Kunbus RevolutionPi%
        \label{sec:2-revpi}}
Der RevolutionPi 3 ist eine speicherprogrammierbare Steuerung (SPS) des Herstellers
Kunbus GmbH. Kern dieser SPS ist das von der Raspberry Pi Foundation entwickelte
und vertriebene Raspberry Pi Compute Module 3. Dieses integriert ein Broadcom BCM2837
System-on-Chip (SoC) mit vier 1,2GHz Prozessorkernen, 1GB RAM, 4GB eMMC Anwendungsspeicher
und sonstige Peripherie in ein Modul im DDR2-SODIMM Formfaktor. Diese Spezifikationen
sind weitgehend identisch zu denen des ausgesprochen populären Raspberry Pi 3.
Der Revolution Pi profitiert daher von dem gleichen großen Angebot an Software
und Unterstützung wie der Raspberry Pi, ergänzt dessen Hardware jedoch um eine 24V
Spannungsversorgung, die Möglichkeit der Erweiterung durch mehrere industrietaugliche
Ein-/ Ausgabemodule und Gateways sowie ein Gehäuse zur Montage auf einer DIN-Schiene.
\begin{itemize}
  \item{Prozessor: BCM2837}
  \item{Taktfrequenz 1,2 GHz}
  \item{Anzahl Prozessorkerne: 4}
  \item{Arbeitsspeicher: 1 GByte}
  \item{eMMC Flash Speicher: 4 GByte}
  \item{Betriebssystem: Angepasstes Raspbian mit RT-Patch}
  \item{RTC mit 24h Pufferung über wartungsfreien Kondensator}
  \item{Treiber / API: Treiber schreibt zyklisch Prozessdaten in ein Prozessabbild, Zugriff auf Prozessabbild über Linux-Filesystem als API zu Fremdsoftware.}
  \item{Kommunikationsanschlüsse: 2 x USB 2.0 A (je 500 mA belastbar), 1 x Micro-USB, HDMI, Ethernet (RJ45) 10/100 Mbit/s}
  \item{Stromversorgung: min. 10,7 V, max. 28,8 V, maximal 10 Watt}
  \item{Zulässige Umgebungstemperatur: -40 bis +55 C}
  \item{Gehäuseabmessungen: (HxBxL) 96 mm x 22,5 mm x 110,5 mm (ohne gesteckte Stecker)}
  \item{ESD Schutz: 4 kV / 8 kV gemäß EN61131-2 und IEC 61000-6-2}
  \item{Surge / Burst Prüfungen: gemäß EN61131-2 und IEC 61000-6-2 eingekoppelt auf Versorgungsspannung, Ethernet und IO-Leitungen}
  \item{EMI Prüfungen: gemäß EN61131-2 und IEC 61000-6-2}
\end{itemize}

Kunbus bietet eine Auswahl an IO- und Gateway-Modulen zur Erweiterung des Revolution Pi an.
Gateways dienen der Kommunikation mit Systemen oder Komponenten der Automatisierungstechnik
über Protokolle wie PROFIBUS oder EtherCAT. IO-Module erlauben die Überwachung
und Steuerung von digitalen oder analogen Ein- und Ausgängen.

\subsubsection{Zugriff auf IO-Module%
        \label{sec:2-io}}
Der Zugriff auf die Ein- und Ausgänge der IO-Module erfolgt über ein Prozessabbild
und einen hierfür von Kunbus bereitgestellten Treiber, genannt piControl. Dieser
aktualisiert das Prozessabbild zyklisch. Die angestrebte Zykluszeit beträgt 5ms,
kann jedoch je nach Anzahl der angeschlossenen Module auch größer sein. Kunbus
garantiert bei drei IO-Modulen und zwei Gateway-Modulen eine Zykluszeit von 10 ms.
Jedes der IO-Module stellt ein eigenständiges eingebettetes System dar. Es verfügt
über einen Microcontroller, welcher die IOs bereitstellt und über einen RS485-Bus
mit dem Revolution Pi kommuniziert.
% https://revolution.kunbus.de/io-modul/

Lizenz: GPL
% https://github.com/RevolutionPi/piControl

\begin{lstlisting}[language={c},firstnumber={226},caption={Setzen der Scheduler-Priorität auf SCHED\_FIFO in revpi\_common.c\label{lst:2-sched_priority}}]
param.sched_priority = ktprio->prio;
ret = sched_setscheduler(child, SCHED_FIFO,
       &param);
\end{lstlisting}


\subsection{Echtzeit und Multithreading unter Linux -- preemptRT und posix%
     \label{sec:2-echtzeit}}


 Der Linux-Kernel verfügt über mehrere unterschiedliche Preemtion-Modelle:

\begin{itemize}
  \item No Forced Preemption (server):
  Ausgelegt auf maximal möglichen Durchsatz, lediglich Interrupts und
  System-Call-Returns bewirken Präemption.

  \item Voluntary Kernel Preemption (Desktop):
  Neben den implizit bevorrechtigten Interrupts und System-Call-Returns gibt es
  in diesem Modell weitere Abschnitte des Kernels in welchen Preämption explizit
  gestattet ist.

  \item Preemptible Kernel (Low-Latency Desktop):
  In diesem Modell ist der gesamte Kernel, mit Ausnahme sog.~kritischer Abschnitte
  präemptible. Nach jedem kritischen Abschnitt gibt es einen impliziten Präemptions-Punkt.

  \item Preemptible Kernel (Basic RT):
  Dieses Modell ist dem zuvor genannten sehr ähnlich, hier sind jedoch alle Interrupt-Handler
  als eigenständige Threads ausgeführt.

  \item Fully Preemptible Kernel (RT):
  Wie auch bei den beiden zuvor genannten Modellen ist hier der gesamte Kernel
  präemtible, die Anzahl und Dauer der nicht-präemtiblen kritischen Abschnitte
  ist auf ein notwendiges Minimum beschränkt. Alle Interrupt-Handler sind als
  eigenständige Threads ausgeführt, Spinlocks durch Sleeping-Spinlocks und Mutexe
  durch sog.~RT-Mutexe ersetzt.

\end{itemize}
\todo{Spinlocks und Mutexe sowie die RT-Varianten dieser erklären!}

Lediglich mit dem vollständig präemtiblen Kernel kann Echtzeit-Verhalten realisiert werden.

% https://wiki.linuxfoundation.org/realtime/documentation/technical_basics/preemption_models bzw kernel/Kconfig.preempt

\subsubsection{preemptRT%
        \label{sec:2-preemptRT}}
% https://wiki.linuxfoundation.org/realtime/documentation/technical_details/start
% https://wiki.linuxfoundation.org/realtime/documentation/technical_basics/start

Das dem PREEMPT RT Kernel zugrunde liegende Prinzip lässt sich in einer einfachen
Regel ausdrücken: Nur Code, welcher absolut nicht-präemtible sein darf, ist es
gestattet nicht-präemtible zu sein.
Das erklärte Ziel des PREEMPT\_RT Patches ist es folglich, die Menge des nicht-präemtiblen
Codes im Linux-Kernel auf das absolut notwendige Minimum zu reduzieren.

Dies wird durch Verwendung folgender Mechanismen erreicht:

\begin{itemize}
  \item Hochauflösende Timer
  \item Sleeping Spinlocks
  \item Threaded Interrupt Handlers
  \item rt\_mutex
  \item RCU
\end{itemize}


\subsubsection{posix%
        \label{sec:2-posix}}
Ist posix hier wirklich relevant? Debian bzw.~Raspbian sind weitgehend posix
kompatibel, aber wird es hier genutzt? -> JA, open62541 nutzt pthread.h
piControl nutzt kthread.h, und semaphore.h

\subsection{OPC-UA und open62541%
     \label{sec:2-opc}}

\subsubsection{OPC UA%
        \label{sec:2-opcua}}
Open Platform Communications (OPC) ist eine Familie von Standards zur herstellerunabhängigen
Kommunikation von Maschinen (M2M) in der Automatisierungstechnik. Die sog.~OPC Task Force, zu deren
Mitgliedern verschiedene große Firmen der Automatisierungsindustrie gehören, veröffentlichte
die OPC Specification Version 1.0 im August 1996.
Motiviert ist dieser offene Standard durch die Erkenntniss, dass die Anpassung der
zahlreichen Herstellerstandards an individuelle Infrastrukturen und Anlagen einen
großen Mehraufwand verursachen.
Die Wikipedia beschreibt das Anwendungsgebiet für OPC wie folgt:

\glqq{}OPC wird dort eingesetzt, wo Sensoren, Regler und Steuerungen verschiedener Hersteller
ein gemeinsames Netzwerk bilden. Ohne OPC benötigten zwei Geräte zum Datenaustausch
genaue Kenntnis über die Kommunikationsmöglichkeiten des Gegenübers. Erweiterungen
und Austausch gestalten sich entsprechend schwierig. Mit OPC genügt es, für jedes
Gerät genau einmal einen OPC-konformen Treiber zu schreiben. Idealerweise wird
dieser bereits vom Hersteller zur Verfügung gestellt. Ein OPC-Treiber lässt sich
ohne großen Anpassungsaufwand in beliebig große Steuer- und Überwachungssysteme
integrieren.

OPC unterteilt sich in verschiedene Unterstandards, die für den jeweiligen Anwendungsfall
unabhängig voneinander implementiert werden können. OPC lässt sich damit verwenden
für Echtzeitdaten (Überwachung), Datenarchivierung, Alarm-Meldungen und neuerdings
auch direkt zur Steuerung (Befehlsübermittlung).\grqq{}

OPC basiert in der ursprünglichen Spezifikation auf Microsofts DCOM-Spezifikation.
DCOM macht Funktionen und Objekte einer Anwendung anderen Anwendungen im Netzwerk
zugänglich. Der OPC-Standard definiert entsprechende DCOM-Objekte um mit anderen
OPC-Anwendungen Daten austauschen zu können. Die Verwendung von DCOM bindet Anwender
an Betriebssysteme von Microsoft. Die ursprüngliche OPC Spezifikation wird durch die
Entwicklung von OPC Unified Architecture (OPC UA) abgelöst.
OPC UA setzt auf einem eigenen Kommunikationionsstack auf, die Verwendung von DCOM
und damit die Bindung an Microsoft wurden aufgelöst.

Die OPC-UA-Architektur ist eine Service-orientierte Architektur (SOA), deren Struktur
aus mehreren Schichten besteht.

% Wikipedia
Das OPC-Informationsmodell ist nicht mehr nur eine Hierarchie aus Ordnern, Items
und Properties. Es ist ein sogenanntes Full-Mesh-Network aus Nodes, mit dem neben
den Nutzdaten eines Nodes auch Meta- und Diagnoseinformationen repräsentiert werden.
Ein Node ähnelt einem Objekt aus der objektorientierten Programmierung. Ein Node
kann Attribute besitzen, die gelesen werden können (Data Access (DA), Historical
Data Access (HDA)). Es ist möglich Methoden zu definieren und aufzurufen.
Eine Methode besitzt Aufrufargumente und Rückgabewerte. Sie wird durch ein Command
aufgerufen. Weiterhin werden Events unterstützt, die versendet werden können
(AE (Alarms \& Events), DA DataChange), um bestimmte Informationen zwischen Geräten
auszutauschen. Ein Event besitzt unter anderem einen Empfangszeitpunkt, eine Nachricht
und einen Schweregrad. Die o. g. Nodes werden sowohl für die Nutzdaten als auch
alle anderen Arten von Metadaten verwendet. Der damit modellierte OPC-Adressraum
beinhaltet nun auch ein Typmodell, mit dem sämtliche Datentypen spezifiziert werden.

% https://de.wikipedia.org/wiki/Open_Platform_Communications
% https://de.wikipedia.org/wiki/OPC_Unified_Architecture
% https://opcfoundation.org/developer-tools/specifications-unified-architecture
% Von Gerhard Gappmeier - ascolab GmbH, CC BY-SA 3.0, https://de.wikipedia.org/w/index.php?curid=1892069
\subsubsection{open62541%
        \label{sec:2-open62541}}
open62541 ist eine offene und freie Implementierung von OPC UA. Die in C geschriebene
Bibliothek stellt eine beständig zunehmende Anzahl der im OPC UA Standard definierten
Funktionen bereit. Sie kann sowohl zur Erstellung von OPC-Servern als auch -Clients
genutzt werden. Ergänzend zu der unter der Mozilla Public License v2.0 lizensierten
Bibliothek stellt das open62541 Projekt auch Beispielprogramme unter einer CC0 Lizenz
zur Verfügung.

Die Bibliothek eignet sich auch für die Entwicklung auf eingebetteten Systemen und
Microcontrollern. Je nach Umfang der gewünschten Funktionen und des OPC Informationsmodells
beträgt die Größe einer Server-Binary weniger als 100kb. %evtl. kürzen?

\todo{Nodes erklären! Evtl.~oben!}

Folgende Auswahl an Eigenschaften und Funktionen zeichnet die in dieser Arbeit verwendete
Version 0.3 von open62541 aus:
\begin{itemize}
  \item Kommunikationionsstack
  \begin{itemize}
      \item OPC UA Binär-Protokoll (HTTP oder SOAP werden gegenwärtig nicht unterstützt)
      \item Austauschbare Netzwerk-Schicht, welche die Verwendung eigener Netzwerk-APIs
      erlaubt.
      \item Verschlüsselte Kommunikationion
      \item Asynchrone Dienst-Anfragen im Client
  \end{itemize}
  \item Informationsmodell
  \begin{itemize}
    \item Unterstützung aller OPC UA Node-Typen, inkl.~Methoden
    \item Hinzufügen und Entfernen von Nodes und Referenzen zur Laufzeit.
    \item Vererbung und Instanziierung von Objekt- und Variablentypen
    \item Zugriffskontrolle auch für einzelne Nodes
  \end{itemize}
  \item Subscriptions
  \begin{itemize}
    \item Erlaubt die Überwachung (subscriptions / monitoreditems)
    \item Sehr geringer Ressourcenbedarf pro überwachtem Wert
  \end{itemize}
  \item Code-Generierung auf XML-Basis
  \begin{itemize}
    \item Erlaubt die Erstellung von Datentypen
    \item Erlaubt die Generierung des serverseitigen Informationsmodells
  \end{itemize}
\end{itemize}

% https://open62541.org/doc/0.3/


Mozilla Public License
CC0 Lizenz für Beispiele und Plugins

% https://open62541.org/doc/open62541-current.pdf
% https://open62541.org/

% % % Imports nur für Referenzenauflösung während des Schreibens! Vorm Kompilieren auskommentieren!
% \bibliography{0_hauptdatei}
% \input{1_einleitung}
% \input{2_grundlagen}
% \input{3_konzeption}
% \input{4_implementierung}
% \input{5_tests}
% \input{6_zusammenfassung}
% \input{anhang}
% % Ende Imports

\section{Systemkonzept%
  \label{sec:3-konzeption}}
Auf Basis der in Abschnitt \ref{sec:2-grundlagen} vorgestellten Möglichkeiten folgt nun die Ausarbeitung eines Konzepts.
In den folgenden Abschnitten soll näher auf zwei zentrale Aspekte eingegangen werden: Abschnitt~\ref{sec:3-anbindung} stellt Möglichkeiten zum Zugriff auf Variablen bzw.\,Werte im Prozessabbild des Revolution Pi vor; in Abschnitt~\ref{sec:3-integration} wird ein Konzept zur Bereitstellung dieser Variablen auf einem OPC-Server vorgestellt.

\subsection{Anbindung der IO an den OPC-Server%
     \label{sec:3-anbindung}}

Eine Webanwendung mit Bezeichnung PiCtory dient zur Konfiguration der I/O- und virtuellen Module des RevolutionPi. Die Konfiguration liegt im JSON-Format in der Datei \lstinline{/etc/revpi/config.rsc}. Der piControl-Treiber liest diese Datei beim Start. 
Der folgende Auszug aus der Manpage des piControl-Kernelmoduls beschreibt die von diesem zum Lesen und Schreiben einzelner Bits des Prozessabbildes bereitgestellten Funktionen~\citep[vgl.]{web-revpi-manpage}. Sie ist an dieser Stelle weitgehend ungekürzt zitiert, da sie die nutzbare Schnittstelle sehr kompakt beschreibt.

\begin{lstlisting}[breakindent=0pt, numbers=none, caption={Auszug aus der Revolution Pi Programmers Manual\label{lst:4-manpage}}]
KB_FIND_VARIABLE SPIVariable *argp
Find a variable in the process image by its name. A pointer to a structure of type SPIVariable must be passed as argument. [...]
The struct SPIVariable [...] is defined as 
typedef struct SPIVariableStr
{
    char strVarName[32]; // Variable name
    uint16_t i16uAddress; // Address of the byte in the process image
    uint8_t i8uBit; // 0-7 bit position, >= 8 whole byte
    uint16_t i16uLength; // length of the variable in bits.
    // Possible values are 1, 8, 16 and 32
} SPIVariable;

Set and get values of the process image
KB_GET_VALUE SPIValue *argp
[...]
KB_SET_VALUE SPIValue *argp
Write one bit or one byte to the process image [...].  This call is more efficient than the usual calls of seek and write because only one function call is necessary. If more than on application are writing bits in one output byte, this call is the only safe way to set a bit without overwriting the other bits because this call is doing a read-modify-write-cycle. 

The struct SPIValue used by this ioctl is defined as
typedef struct SPIValueStr
{
    uint16_t i16uAddress; // Address of the byte in the process image
    uint8_t i8uBit; // 0-7 bit position, >= 8 whole byte
    uint8_t i8uValue; // Value: 0/1 for bit access, whole byte otherwise
} SPIValue;
\end{lstlisting} 

Die oben beschriebenden Funtkionen \lstinline{KB_FIND_VARIABLE}, \lstinline{KB_GET_VALUE} und \lstinline{KB_SET_VALUE} ermöglichen einen einfachen und (lt.\,Manpage) effizienten Zugriff auf einzelne Bits des Prozessabbildes und damit auch auf die IO des RevolutionPi.
Der Zugriff des OPC-Servers auf das Prozessabbild soll daher mittels dieser Funktionen realisiert werden.
\lstinline{KB_FIND_VARIABLE} kann genutzt werden, um Adressen von Variablen im Prozessabbild mittels ihres Namens aufzulösen.
\lstinline{KB_GET_VALUE} und \lstinline{KB_SET_VALUE} ermöglichen den Zugriff auf die Werte dieser Variablen.


\subsection{Integration des OPC-Servers in das System%
     \label{sec:3-integration}}

open62541 bietet drei Möglichkeiten zum Abgleich von Variablen mit dem Prozessabbild~\citep[vgl.][Tutorials - Connecting a Variable with a Physical Process]{web-open62541}:
\begin{itemize}
    \item Manuelles oder zyklisches Aktualisieren
    \item Variable Value Callback
    \item Variable Datasource
\end{itemize}

Die zyklische Aktualisierung eines oder mehrerer Werte nimmt, abhängig von der Zykluszeit, viele Systemressourcen in Anspruch. Value Callbacks ermöglichen es, einen Variablenwert effizienter mit einer Ressource wie etwa einem Prozessabbild zu synchronisieren. An die Variable wird ein Callback angehängt, welches vor jedem Lesen und nach jedem Schreibvorgang ausgeführt wird.
Der Wert der Variablen wird weiterhin im Variablenknoten auf dem OPC-Server gespeichert, der Abgleich mit der verknüpften Ressource erfolgt durch die Callback-Methoden.

Sogenannte Datenquellen gehen noch einen Schritt weiter. Der Server leitet jede Lese- und Schreibanforderung direkt an eine Callback-Funktion weiter. Beim Lesen liefert der Rückruf eine Kopie des aktuellen Wertes. Die Datenquelle muss intern ein eigenes Speichermanagement implementieren.

Der Zugriff auf die Werte des Prozessabbildes erfolgt, wie in Abschnitt~\ref{sec:3-anbindung} beschrieben, über von piControl bereitgestellte Methoden. Um die durch open62541 gepflegte OPC-Datenstruktur und das durch piControl verwaltete Prozessabbild möglichst effektiv verknüpfen zu können, soll diese Interaktion mittels Datenquellen und den zugehörigen Callbacks implementiert werden.
% % % Imports nur für Referenzenauflösung während des Schreibens! Vorm Kompilieren auskommentieren!
% \bibliography{0_hauptdatei}
% \input{1_einleitung}
% \input{2_grundlagen}
% \input{3_konzeption}
% \input{4_implementierung}
% \input{5_tests}
% \input{6_zusammenfassung}
% \input{anhang}
% % Ende Imports

\section{Implementierung%
  \label{sec:4-implementierung}}
Das folgende Kapitel stellt in Auszügen die Implementierung des OPC-Servers sowie die Anbindung an die IO-Module
der SPS dar. Der Schwerpunkt liegt hierbei auf der Funktionsweise des piControl-Treibers und dessen Integration in das Projekt. Abschnitt~\ref{sec:4-picontrol} erklärt die zum Schreibens eines Bits verwendeten Funktionsaufrufe.
Zuvor soll jedoch in Abschnitt~\ref{sec:4-open62541} der Teil des OPC-Servers vorgestellt werden, welcher auf besagten Treiber zugreift. 

\subsection{Implementierung des OPC-Servers%
     \label{sec:4-open62541}}
Wie im vorangegangenen Abschnitt~\ref{sec:3-integration} begründet, soll die Verknüpfung zwischen dem Prozessabbild der SPS und den auf dem OPC-Server bereitgestellten Werten über sog.\,Datenquellen erfolgen. Hierzu ist zunächst eine Callback-Methode zu implementieren, welche bei einem Lese- oder Schreibzugriff auf eine Variable aufgerufen wird. Die Verknüpfung zwischen Callback-Methode und Variable muss manuell erfolgen.

\begin{lstlisting}[language={c},firstnumber=237,caption={Auszug der Methode \lstinline{linkDataSourceVariable} in \lstinline{variables.c}\label{lst:4-linkDataSourceVariable}}]
extern UA_StatusCode
 linkDataSourceVariable(UA_Server *server, UA_NodeId nodeId) {
     bool readonly = false;
     UA_DataSource dataSourceVariable;
     UA_StatusCode rc; |>\setcounter{lstnumber}{254}<|

     dataSourceVariable.read = readDataSourceVariable;
     if (!readonly)
        dataSourceVariable.write = writeDataSourceVariable;
     else
        dataSourceVariable.write = writeReadonlyDataSourceVariable;

     return UA_Server_setVariableNode_dataSource(server, nodeId, dataSourceVariable);
 }
\end{lstlisting}

\begin{figure}[h]
    \centering
    \includegraphics[width=0.42\textwidth]{doc/img/OPC_RevPiDO.pdf}
    \caption{Auszug des verwendeten Nodesets, hier Digitalausgang 1 des Versuchsaufbaus
      \label{fig:opc-do}}
\end{figure}

Die in Listing~\ref{lst:4-linkDataSourceVariable} abgebildete Methode \lstinline{linkDataSourceVariable()} erzeugt ein Struct vom Typ \lstinline{UA_DataSource}. In diesem werden dem Lesen und Schreiben einer OPC-Variablen entsprechende Callback-Methoden zugewiesen. Die Verknüpfung einer OPC-Variable, genauer ihrer NodeId, mit der zuvor definierten Datenquelle erfolgt über die von open62541 bereitgestellte Methode \lstinline{UA_Server_setVariableNode_dataSource()}. Vor dem Lesen und nach dem Schreiben dieser Variable werden von nun an die entsprechenden Callbacks aufgerufen.
     
\begin{lstlisting}[language={c},firstnumber=168,caption={Auszug des Callbacks \lstinline{writeDataSourceVariable} in \lstinline{variables.c}\label{lst:4-writeDataSourceVariable}}]  
extern UA_StatusCode
 writeDataSourceVariable(UA_Server *server,
            const UA_NodeId *sessionId, void *sessionContext,
            const UA_NodeId *nodeId, void *nodeContext,
            const UA_NumericRange *range, const UA_DataValue *dataValue) {

    UA_StatusCode retval  = UA_STATUSCODE_GOOD;
    UA_NodeId *nameNodeId = UA_malloc(sizeof(UA_NodeId));
    UA_QualifiedName nameQN = UA_QUALIFIEDNAME(1, "Name");
    UA_Variant nameVar;
    UA_Boolean bit;

    retval |= findSiblingByBrowsename(server, nodeId, &nameQN, nameNodeId);
    retval |= UA_Server_readValue(server, *nameNodeId, &nameVar);
    retval |= UA_Boolean_copy(dataValue->value.data, &bit);

    |>\tikzmarkin[set border color=martinired]{writeIO}<|PI_writeSingleIO(String_fromUA_String(nameVar.data), &bit, false);                                                 |>\tikzmarkend{writeIO}<|

    free(nameNodeId);
    return retval;
 }
\end{lstlisting}

Listing~\ref{lst:4-writeDataSourceVariable} zeigt die Callback-Methode, welche nach dem Schreiben einer Variablen auf dem OPC-Server aufgerufen wird.
Dieser Methode wird neben der NodeId der mit ihr verknüpften Variablen auch der Wert dieser in Form eines Zeigers auf ein Struct vom Typ \lstinline{UA_DataValue} übergeben.

Die Gestaltung des hier verwendeten Nodesets sieht vor, dass in einer OPC-Variablen \lstinline{"Name"} der Bezeichner des zu schreibenden Digitalausgangs hinterlegt ist, siehe Abbildung~\ref{fig:opc-do}. Dies erlaubt eine Rekonfiguration der Ein- und Ausgänge der SPS ohne Änderungen im Programmcode des OPC-Servers vornehmen zu müssen.
Es ist daher erforderlich, nach jedem Schreiben einer mit einem Digitalausgang verknüpften Variablen, hier \lstinline{"Value"}, dessen Bezeichner \lstinline{"Name"} abzufragen. 
Dies geschieht in den Zeilen 180 und 181.
Anschließend wird dieser Bezeichner sowie der zu schreibende Wert der Methode \lstinline{PI_writeSingleIO()} übergeben, welche wiederum die Interaktion mit piControl übernimmt (vgl. Abschnitt \ref{sec:4-picontrol}).
 
\subsection{Integration von piControl%
     \label{sec:4-picontrol}}
In Abschnitt~\ref{sec:2-io} wurde die Anbindung der IO-Module des Revolution Pi sowie die Funktionsweise von piControl aus Anwendersicht beschrieben. Die verfügbare Literatur beschränkt sich auch auf lediglich diese Sicht; eine weiterführende Dokumentation für Entwickler gibt es, neben der in Abschnitt~\ref{sec:3-anbindung} vorgestellten Manpage, nicht. 
In diesem Abschnitt soll daher der Quellcode von piControl sowie dessen Verwendung im Projekt genauer betrachtet werden.
Hierzu wird exemplarisch die in Abschnitt~\ref{sec:4-open62541} eingeführte Methode \lstinline{PI_writeSingleIO()} untersucht.
Diese Methode ermöglicht das Setzen eines einzelnen Bits im Prozessabbild der SPS, und damit das Schalten eines digitalen Ausgangs auf einem IO-Modul.
Die äquivalente Methode \lstinline{int piControlGetBitValue(SPIValue *pSpiValue)} zum Lesen eines Bits bzw. Eingangs funktioniert analog und soll daher an dieser Stelle nicht dediziert erörtert werden.

\begin{lstlisting}[language={c},firstnumber=97,
                   caption={Setzen eines phsikalischen, digitalen Ausgangs in \lstinline{revpi.c}
                   \label{lst:4-PI_writeSingleIO}}]
extern void PI_writeSingleIO(char *pszVariableName, bool *bit, bool verbose)
{
	int rc;
	SPIVariable sPiVariable;
	SPIValue sPIValue;

	strncpy(sPiVariable.strVarName, pszVariableName, sizeof(sPiVariable.strVarName));
	rc = piControlGetVariableInfo(&sPiVariable);
	if (rc < 0) {
		printf("Cannot find variable '%s'\n", pszVariableName);
		return;
	}

		sPIValue.i16uAddress = sPiVariable.i16uAddress;
		sPIValue.i8uBit = sPiVariable.i8uBit;
		sPIValue.i8uValue = *bit;
		rc = |>\tikzmarkin[set border color=martinired]{setBitValue}<|piControlSetBitValue(&sPIValue)|>\tikzmarkend{setBitValue}<|;
		if (rc < 0)
			printf("Set bit error %s\n", getWriteError(rc));
		else if (verbose)
			printf("Set bit %d on byte at offset %d. Value %d\n", sPIValue.i8uBit, sPIValue.i16uAddress,
			       sPIValue.i8uValue);
}
\end{lstlisting}

Der Programmcode in Listing~\ref{lst:4-PI_writeSingleIO} ist Teil des implementierten OPC-Servers. In diesem wird auf zwei Funktionen des piControl-Treibers zugegriffen. 
Beiden Methoden wird als Argument ein Zeiger auf ein Struct vom Typ \lstinline{SPIValue} übergeben. Der im Struct abgelegte Name wird mittels \lstinline{piControlGetVariableInfo(&sPIValue)} zu einer Adresse im Prozessabbild aufgelöst. Diese wird in \lstinline{sPIValue.i16uAdress} gespeichert. Der Wert der Variablen wird anschließend mittels \lstinline{piControlSetBitValue(&sPIValue)} an dieser Adresse in das Prozessabbild geschrieben.

\begin{lstlisting}[language={c},firstnumber=309,caption={Methode \lstinline{piControlSetBitValue} in \lstinline{piControlIf.c}\label{lst:4-piControlSetBitValue}}]
int |>\tikzmarkin[set border color=martiniblue]{setBitValueFcn}<|piControlSetBitValue(SPIValue *pSpiValue)|>\tikzmarkend{setBitValueFcn}<|
{
    piControlOpen();

    if (PiControlHandle_g < 0)
	    return -ENODEV;

    pSpiValue->i16uAddress += pSpiValue->i8uBit / 8;
    pSpiValue->i8uBit %= 8;

    if (|>\tikzmarkin[set border color=martinired]{ioctl}<|ioctl(PiControlHandle_g, KB_SET_VALUE, pSpiValue)|>\tikzmarkend{ioctl}<| < 0)
	    return errno;

    return 0;
}
\end{lstlisting}

Die in Listing~\ref{lst:4-piControlSetBitValue} dargestellte Methode \lstinline{piControlSetBitValue} ist lediglich eine Hüllfunktion (häufig auch als Wrapper-Funktion bezeichnet) für einen Aufruf des \lstinline{ioctl} Kernel-Moduls.
Folgende Parameter werden übergeben:
\lstinline{PiControlHandle_g} ist die Referenz auf die Geräte-Datei des piControl-Treibers. \lstinline{KB_SET_VALUE} ist das ioctl-Kommando zum Schreiben eines Bits in das Prozessabbild. Der Zeiger \lstinline{pSpiValue} verweist auf ein Struct des bereits vorgestellten Typs \lstinline{SPIValue}.

\begin{lstlisting}[language={c},firstnumber=80,caption={Methode \lstinline{piControlOpen} in \lstinline{piControlIf.c}\label{lst:4-piControlOpen}}]
void piControlOpen(void)
{
    /* open handle if needed */
    if (PiControlHandle_g < 0)
    {
	    |>\tikzmarkin[set border color=martiniblue]{PiControlHandle}<|PiControlHandle_g = open(PICONTROL_DEVICE, O_RDWR)|>\tikzmarkend{PiControlHandle}<|;
    }
}
\end{lstlisting}

Die in Listing~\ref{lst:4-piControlOpen} dargestellte Methode öffnet, sofern nicht bereits geschehen, die Geräte-Datei. Das Macro \lstinline{PICONTROL_DEVICE} verweist hierbei auf \lstinline{/dev/piControl0}.

\begin{lstlisting}[language={c},firstnumber=721,caption={Methode \lstinline{piControlIoctl} in \lstinline{piControlMain.c}\label{lst:4-piControlIoctl}}]
static long |>\tikzmarkin[set border color=martiniblue, below offset=0.9em]{piControlIoctl}<|piControlIoctl(struct file *file, unsigned int prg_nr, 
                           unsigned long usr_addr)                                      |>\tikzmarkend{piControlIoctl}<|
{
  int status = -EFAULT;
  tpiControlInst *priv;
  int timeout = 10000;	// ms

  if (prg_nr != KB_CONFIG_SEND && prg_nr != KB_CONFIG_START && !isRunning()) {
  	return -EAGAIN;
  }

  priv = (tpiControlInst *) file->private_data;

  if (prg_nr != KB_GET_LAST_MESSAGE) {
  	// clear old message
  	priv->pcErrorMessage[0] = 0;
  }

  switch (prg_nr) {|>\setcounter{lstnumber}{864}<|

    case |>\tikzmarkin[set border color=martiniblue]{KB_SET_VALUE}<|KB_SET_VALUE:|>\tikzmarkend{KB_SET_VALUE}<|
  		{
  			SPIValue *pValue = (SPIValue *) usr_addr;

  			if (!isRunning())
  				return -EFAULT;

  			if (pValue->i16uAddress >= KB_PI_LEN) {
  				status = -EFAULT;
  			} else {
  				INT8U i8uValue_l;
  				my_rt_mutex_lock(&piDev_g.lockPI);
  				i8uValue_l = piDev_g.ai8uPI[pValue->i16uAddress];

  				if (pValue->i8uBit >= 8) {
  					i8uValue_l = pValue->i8uValue;
  				} else {
  					if (pValue->i8uValue)
  						i8uValue_l |= (1 << pValue->i8uBit);
  					else
  						i8uValue_l &= ~(1 << pValue->i8uBit);
  				}

  				|>\tikzmarkin[set border color=martinired]{i8uValue}<|piDev_g.ai8uPI[pValue->i16uAddress] = i8uValue_l;|>\tikzmarkend{i8uValue}<|
  				rt_mutex_unlock(&piDev_g.lockPI);

  #ifdef VERBOSE
  				pr_info("piControlIoctl Addr=%u, bit=%u: %02x %02x\n", pValue->i16uAddress, pValue->i8uBit, pValue->i8uValue, i8uValue_l);
  #endif

  				status = 0;
  			}
  		}
  		break; |>\setcounter{lstnumber}{1314}<|

    default:
      pr_err("Invalid Ioctl");
      return (-EINVAL);
      break;

    }

    return status;
  }
\end{lstlisting}

Listing~\ref{lst:4-piControlIoctl} zeigt in Auszügen die ioctl-Methode des piControl Kernel-Treibers. Diese bekommt folgende Argumente übergeben: \lstinline{struct file *file} enthält den Verweis auf die Geräte-Datei, hier \lstinline{/dev/piControl0}. Der Wert von \lstinline{unsigned int prg_nr} beschreibt die Anfrage an den Treiber, in diesem Fall \lstinline{KB_SET_VALUE}. Das Argument \lstinline{unsigned long usr_addr} enthält einen typ-agnostischen Pointer. Dieser verweist auf einen Speicherbereich, in welchem die zur Bearbeitung der Anfrage notwendigen Daten abgelegt sind. Hier können auch vom Treiber empfangene Daten dem Anwendungsprogramm bereitgestellt werden. 

Die switch-case-Anweisung führt die über das Argument \lstinline{prg_nr} spezifizierte Aktion aus. Hier betrachten wir \lstinline{KB_SET_VALUE}:
Zunächst wird in Zeile 868 der übergebene Zeiger \lstinline{usr_addr} mittels explizitem Typecast zu einem Zeiger des Typs \lstinline{SPIValue *} konvertiert. Da dieser auf Daten im Userspace verweist, ist beim Zugriff durch den Kernel-Treiber besondere Vorsicht geboten.
In Zeile 877 wird mittels Mutex das Prozessabbild \lstinline{piDev_g} für den Zugriff durch andere Threads oder Prozesse gesperrt.
\lstinline{my_rt_mutex_lock} verweist hierbei auf die Funktion \lstinline{rt_mutex_lock} aus \lstinline{linux/sched.h}\footnote{Offenbar wurde hier auch eine alternative Implementierung vorgesehen, siehe revpi\_common.h}

In Zeile 889 wird das Byte \lstinline{i8uValue_l}, welches den zu schreibenden Wert enthält in das Prozessabbild übertragen. Anschließend wird die Mutex auf \lstinline{piDev_g} wieder entsperrt.
\newpage

\begin{lstlisting}[language={c},firstnumber=62,caption={Auszug des Struct \lstinline{spiControlDev} in \lstinline{piControlMain.h}\label{lst:4-spiControlDev}}]
|>\tikzmarkin[set border color=martiniblue]{spiControlDev}<|typedef struct spiControlDev|>\tikzmarkend{spiControlDev}<| {
	// device driver stuff
	int init_step;
	enum revpi_machine machine_type;
	void *machine;
	struct cdev cdev;	// Char device structure
	struct device *dev;
	struct thermal_zone_device *thermal_zone;

	|>\tikzmarkin[set border color=martiniblue]{processImage}<|// process image stuff
	INT8U ai8uPI[KB_PI_LEN];
	INT8U ai8uPIDefault|>\tikzmarkin[set border color=martinired]{KB_PI_LEN_0}<|[KB_PI_LEN]|>\tikzmarkend{KB_PI_LEN_0}<|;
	struct rt_mutex lockPI;        |>\tikzmarkend{processImage}<|
	bool stopIO;
	piDevices *devs; |>\setcounter{lstnumber}{94}<|
} tpiControlDev;
\end{lstlisting}

Das Prozessabbild ist als Byte-Array der Länge \lstinline{KB_PI_LEN} in Listing~\ref{lst:4-spiControlDev} definiert. Konfigurationsparameter wie \lstinline{KB_PI_LEN} oder die Zykluszeit für den Datenaustausch zwischen SPS und IO-Modulen sind im folgenden Listing~\ref{lst:4-process} definiert.

\begin{lstlisting}[language={c},firstnumber=119,caption={Konfigurationsparameter des Prozessabbildes in project.h\label{lst:4-process}}]
#define INTERVAL_PI_GATE (5*1000*1000)  // 5 ms piGateCommunication |>\setcounter{lstnumber}{128}<|

#define INTERVAL_IO_COM (5*1000*1000)  // 5 ms piIoComm |>\setcounter{lstnumber}{132}<|

#define KB_PD_LEN       512
|>\tikzmarkin[set border color=martiniblue]{KB_PI_LEN_1}<|#define KB_PI_LEN       4096|>\tikzmarkend{KB_PI_LEN_1}<|
\end{lstlisting}

Das zu setzende Bit wurde zu diesem Zeitpunkt erfolgreich in das Prozessabbild der SPS geschrieben.
Es stellt sich die Frage, wie dieses nun an das IO-Modul kommuniziert wird.
Die Kommunikation mit allen angebundenen Modulen ist ebenfalls Aufgabe des piControl-Treibers.

\begin{lstlisting}[language={c},firstnumber=256,caption={Auszug der Methode \lstinline{piIoThread} in \lstinline{revpi_core.c}\label{lst:4-piIoThread}}]
static int piIoThread(void *data)
{
	//TODO int value = 0;
	ktime_t time;
	ktime_t now;
	s64 tDiff;

	hrtimer_init(&piCore_g.ioTimer, CLOCK_MONOTONIC, HRTIMER_MODE_ABS);
	piCore_g.ioTimer.function = piIoTimer;

	pr_info("piIO thread started\n");

	now = hrtimer_cb_get_time(&piCore_g.ioTimer);

	PiBridgeMaster_Reset();

	while (!kthread_should_stop()) {
		if (|>\tikzmarkin[set border color=martinired]{PiBridgeMaster}<|PiBridgeMaster_Run()|>\tikzmarkend{PiBridgeMaster}<| < 0)
			break;
	}

	RevPiDevice_finish();

	pr_info("piIO exit\n");
	return 0;
}
\end{lstlisting}

Der Kernel-Thread \lstinline{piIoThread} ist verantwortlich für den zyklischen Datenaustausch mit den IO-Modulen. In diesem wird fortlaufend die Methode \lstinline{PiBridgeMaster_Run()} aufgerufen, siehe Listing~\ref{lst:4-piIoThread}.

\begin{lstlisting}[language={c},firstnumber=262,caption={Auszug der Methode \lstinline{PiBridgeMaster_Run(void)} in \lstinline{RevPiDevice.c}\label{lst:4-PiBridgeMaster_Run}}]
int PiBridgeMaster_Run(void)
{
	static kbUT_Timer tTimeoutTimer_s;
	static kbUT_Timer tConfigTimeoutTimer_s;
	static int error_cnt;
	static INT8U last_led;
	static unsigned long last_update;
	int ret = 0;
	int i;

	my_rt_mutex_lock(&piCore_g.lockBridgeState);
	if (piCore_g.eBridgeState != piBridgeStop) {
		switch (eRunStatus_s) { |>\setcounter{lstnumber}{514}<|
		    case enPiBridgeMasterStatus_EndOfConfig:|>\setcounter{lstnumber}{621}<|
		    if (|>\tikzmarkin[set border color=martinired]{RevPiDevice}<|RevPiDevice_run()|>\tikzmarkend{RevPiDevice}<|) {
				// an error occured, check error limits |>\setcounter{lstnumber}{641}<|
			} else {
				ret = 1;
			}
			piCore_g.image.drv.i16uRS485ErrorCnt = RevPiDevice_getErrCnt();
			break;
\end{lstlisting}

Die in Listing~\ref{lst:4-PiBridgeMaster_Run} dargestellte Methode ist eine sog. State-Machine. Ist die Konfiguration der IO-Module erfolgreich abgeschlossen, so führt sie bei Aufruf lediglich die Methode \lstinline{RevPiDevice_run()} aus.

\begin{lstlisting}[language={c},firstnumber=140,caption={Auszug der Methode \lstinline{RevPiDevice_run(void)} in \lstinline{RevPiDevice.c}\label{lst:4-RevPiDevice_run}}]
int RevPiDevice_run(void)
{
	INT8U i8uDevice = 0;
	INT32U r;
	int retval = 0;

	RevPiDevices_s.i16uErrorCnt = 0;

	for (i8uDevice = 0; i8uDevice < RevPiDevice_getDevCnt(); i8uDevice++) {
		if (RevPiDevice_getDev(i8uDevice)->i8uActive) {
			switch (RevPiDevice_getDev(i8uDevice)->sId.i16uModulType) {
			case KUNBUS_FW_DESCR_TYP_PI_DIO_14:
			case KUNBUS_FW_DESCR_TYP_PI_DI_16:
			case KUNBUS_FW_DESCR_TYP_PI_DO_16:
				r = |>\tikzmarkin[set border color=martinired]{sendCyclicTelegram}<|piDIOComm_sendCyclicTelegram(i8uDevice)|>\tikzmarkend{sendCyclicTelegram}\setcounter{lstnumber}{166} <|;

				break; |>\setcounter{lstnumber}{216}<|
			}
		}
	} |>\setcounter{lstnumber}{227}<|
	return retval;
}
\end{lstlisting}

Diese iteriert wie in Listing~\ref{lst:4-RevPiDevice_run} abgebildete durch alle gegenwärtig in der SPS konfigurierten Module. Ist das aktuelle Modul als aktiv markiert, so wird anhand eines sog. Firmware-Descriptors entschieden, welche Methode für die Ansteuerung des Moduls aufzurufen ist.

\begin{lstlisting}[language={c},firstnumber=161,caption={Auszug der Methode \lstinline{piDIOComm_sendCyclicTelegram} in \lstinline{piDIOComm.c}\label{lst:4-sendCyclicTelegram}}]
INT32U piDIOComm_sendCyclicTelegram(INT8U i8uDevice_p)
{
	INT32U i32uRv_l = 0;
	SIOGeneric sRequest_l;
	SIOGeneric sResponse_l;
	INT8U len_l, data_out[18], i, p, data_in[70];
	INT8U i8uAddress;
	int ret; |>\setcounter{lstnumber}{239}<|
	
    |>\tikzmarkin[set border color=martinired]{piIoComm}<|ret = piIoComm_send((INT8U *) & sRequest_l, IOPROTOCOL_HEADER_LENGTH + len_l + 1);  |>\tikzmarkend{piIoComm}\setcounter{lstnumber}{298}<|
}
\end{lstlisting}

Im Falle des hier verwendeten DO-Moduls wird die in Listing~\ref{lst:4-sendCyclicTelegram} abgebildete Methode \lstinline{piDIOComm_sendCyclicTelegram()} aufgerufen. Dieser wird ein Zeiger auf das zu schreibende Gerät übergeben. 
Zunächst wird das Prozessabbild mittels eines proprietären, jedoch im Quellcode offen nachvollziehbaren Protokolls in ein \lstinline{sRequest_l} genanntes Byte-Array umgewandelt. Dieser Schritt ist in Listing~\ref{lst:4-sendCyclicTelegram} nicht abgebildet. Anschließend wird \lstinline{piIoComm_send()} ein Zeiger auf die so generierte Schreib-Anfrage übergeben.

\begin{lstlisting}[language={c},firstnumber=220,caption={Auszug der Methode \lstinline{piIOComm_send} in \lstinline{piIOComm.c}\label{lst:4-piIOComm_send}}]
int piIoComm_send(INT8U * buf_p, INT16U i16uLen_p)
{
	ssize_t write_l = 0;
	INT16U i16uSent_l = 0;|>\setcounter{lstnumber}{249}<|

	while (i16uSent_l < i16uLen_p) {
		write_l = vfs_write(piIoComm_fd_m, buf_p + i16uSent_l, i16uLen_p - i16uSent_l, &piIoComm_fd_m->f_pos);
		if (write_l < 0) {
			pr_info_serial("write error %d\n", (int)write_l);
			return -1;
		} 
		i16uSent_l += write_l;|>\setcounter{lstnumber}{263}<|
	}
	clear();
	vfs_fsync(piIoComm_fd_m, 1);
	return 0;
}
\end{lstlisting}

Listing~\ref{lst:4-piIOComm_send} zeigt die Implementierung von \lstinline{piIoComm_send()}. Diese Methode ist für das Schreiben der oben generierten Anfrage auf die seriellen Schnittstelle verantwortlich. Realisiert wird dies mittels der Methode \lstinline{vfs_write()}. Diese ist in \lstinline{<linux/fs.h>} definiert. Sie ermöglicht das Schreiben einer Datei im Userspace aus dem Kernel heraus. Geschrieben wird hier die Datei mit dem Deskriptor \lstinline{piIoComm_fd_m}.
Da die Funktion \lstinline{vfs_write()} durch andere Kernel-Tasks unterbrochen werden kann, ist nicht gewährleistet, dass die gesamte Anfrage mit nur einem Aufruf geschrieben wird. Die oben abgebildete while-Schleife stellt das vollständige Senden der Anfrage sicher.

\begin{lstlisting}[language={c},firstnumber=157,caption={Auszug der Methode \lstinline{piIOComm_open_serial} in \lstinline{piIOComm.c}\label{lst:4-piIOComm_open_serial}}]
int piIoComm_open_serial(void)
{   |>\setcounter{lstnumber}{167}<|
	struct file *fd;	/* Filedeskriptor */
	struct termios newtio;	/* Schnittstellenoptionen */

	|>\tikzmarkin[set border color=martiniblue]{fd}<|/* Port oeffnen - read/write, kein "controlling tty", 
	    Status von DCD ignorieren */
	fd = filp_open(|>\tikzmarkin[set border color=martinired]{tty}<|REV_PI_TTY_DEVICE|>\tikzmarkend{tty}<|, O_RDWR | O_NOCTTY, 0); |>\setcounter{lstnumber}{208}<|
	
	piIoComm_fd_m = fd;                                                      |>\tikzmarkend{fd}\setcounter{lstnumber}{217}<|

	return 0;
}
\end{lstlisting}

Der zum Schreiben auf die serielle Schnittstelle verwendete Datei-Deskriptor wird von der in Listing~\ref{lst:4-piIOComm_open_serial} abgebildeten Methode \lstinline{piIoComm_open_serial()} generiert. 

\begin{lstlisting}[language={c},firstnumber=45,caption={Definition der seriellen Schnittstelle in \lstinline{piIOComm.h}\label{lst:4-REV_PI_TTY_DEVICE}}]
#define REV_PI_TTY_DEVICE	"/dev/ttyAMA0"
\end{lstlisting}

Das in Listing~\ref{lst:4-REV_PI_TTY_DEVICE} definierte Macro verweist auf eine der seriellen Schnittstellen des RaspberryPi.
Die Implementierung des zugehörigen Schnittstellentreibers soll hier nicht weiter untersucht werden. Somit ist an dieser Stelle die Kette vom Setzen einer Variablen auf dem OPC-Server bis hin zur Aktualisierung des Prozessabbilds der IO-Module geschlossen.

% \begin{lstlisting}[language={c},firstnumber={226},caption={Setzen der Scheduler-Priorität auf SCHED\_FIFO in 
% revpi\_common.c\label{lst:2-sched_priority}}]
% param.sched_priority = ktprio->prio;
% ret = sched_setscheduler(child, SCHED_FIFO, &param);
% \end{lstlisting}
% % % Imports nur für Referenzenauflösung während des Schreibens! Vorm Kompilieren auskommentieren!
% \bibliography{0_hauptdatei}
% \input{1_einleitung}
% \input{2_grundlagen}
% \input{3_konzeption}
% \input{4_implementierung}
% \input{5_tests}
% \input{6_zusammenfassung}
% % Ende Imports

\section{Test des OPC-Servers im Gesamtsystem%
  \label{sec:5-tests}}

% % % Imports nur für Referenzenauflösung während des schreibens! Vorm Kompilieren auskommentieren!
% \bibliography{0_hauptdatei}
% \input{1_einleitung}
% \input{2_grundlagen}
% \input{3_konzeption}
% \input{4_implementierung}
% \input{5_tests}
% \input{6_zusammenfassung}
% % Ende Imports

\section{Zusammenfassung und Ausblick%
  \label{sec:6-fazit}}
Der folgende Abschnitt~\ref{sec:6-zusammenfassung} fasst die gewonnenen Erkenntnisse und den Stand der Implementierung zusammen.
Den Abschluss dieser Arbeit bildet der Ausblick in Abschnitt~\ref{sec:6-ausblick}.

\subsection{Zusammenfassung%
     \label{sec:6-zusammenfassung}}

\subsection{Ausblick%
     \label{sec:6-ausblick}}

% \input{anhang}
% % Ende Imports

\section{Implementierung%
  \label{sec:4-implementierung}}
Das folgende Kapitel stellt in Auszügen die Implementierung des OPC-Servers sowie die Anbindung an die IO-Module
der SPS dar. Der Schwerpunkt liegt hierbei auf der Funktionsweise des piControl-Treibers und dessen Integration in das Projekt. Abschnitt~\ref{sec:4-picontrol} erklärt die zum Schreibens eines Bits verwendeten Funktionsaufrufe.
Zuvor soll jedoch in Abschnitt~\ref{sec:4-open62541} der Teil des OPC-Servers vorgestellt werden, welcher auf besagten Treiber zugreift. 

\subsection{Implementierung des OPC-Servers%
     \label{sec:4-open62541}}
Wie im vorangegangenen Abschnitt~\ref{sec:3-integration} begründet, soll die Verknüpfung zwischen dem Prozessabbild der SPS und den auf dem OPC-Server bereitgestellten Werten über sog.\,Datenquellen erfolgen. Hierzu ist zunächst eine Callback-Methode zu implementieren, welche bei einem Lese- oder Schreibzugriff auf eine Variable aufgerufen wird. Die Verknüpfung zwischen Callback-Methode und Variable muss manuell erfolgen.

\begin{lstlisting}[language={c},firstnumber=237,caption={Auszug der Methode \lstinline{linkDataSourceVariable} in \lstinline{variables.c}\label{lst:4-linkDataSourceVariable}}]
extern UA_StatusCode
 linkDataSourceVariable(UA_Server *server, UA_NodeId nodeId) {
     bool readonly = false;
     UA_DataSource dataSourceVariable;
     UA_StatusCode rc; |>\setcounter{lstnumber}{254}<|

     dataSourceVariable.read = readDataSourceVariable;
     if (!readonly)
        dataSourceVariable.write = writeDataSourceVariable;
     else
        dataSourceVariable.write = writeReadonlyDataSourceVariable;

     return UA_Server_setVariableNode_dataSource(server, nodeId, dataSourceVariable);
 }
\end{lstlisting}

\begin{figure}[h]
    \centering
    \includegraphics[width=0.42\textwidth]{doc/img/OPC_RevPiDO.pdf}
    \caption{Auszug des verwendeten Nodesets, hier Digitalausgang 1 des Versuchsaufbaus
      \label{fig:opc-do}}
\end{figure}

Die in Listing~\ref{lst:4-linkDataSourceVariable} abgebildete Methode \lstinline{linkDataSourceVariable()} erzeugt ein Struct vom Typ \lstinline{UA_DataSource}. In diesem werden dem Lesen und Schreiben einer OPC-Variablen entsprechende Callback-Methoden zugewiesen. Die Verknüpfung einer OPC-Variable, genauer ihrer NodeId, mit der zuvor definierten Datenquelle erfolgt über die von open62541 bereitgestellte Methode \lstinline{UA_Server_setVariableNode_dataSource()}. Vor dem Lesen und nach dem Schreiben dieser Variable werden von nun an die entsprechenden Callbacks aufgerufen.
     
\begin{lstlisting}[language={c},firstnumber=168,caption={Auszug des Callbacks \lstinline{writeDataSourceVariable} in \lstinline{variables.c}\label{lst:4-writeDataSourceVariable}}]  
extern UA_StatusCode
 writeDataSourceVariable(UA_Server *server,
            const UA_NodeId *sessionId, void *sessionContext,
            const UA_NodeId *nodeId, void *nodeContext,
            const UA_NumericRange *range, const UA_DataValue *dataValue) {

    UA_StatusCode retval  = UA_STATUSCODE_GOOD;
    UA_NodeId *nameNodeId = UA_malloc(sizeof(UA_NodeId));
    UA_QualifiedName nameQN = UA_QUALIFIEDNAME(1, "Name");
    UA_Variant nameVar;
    UA_Boolean bit;

    retval |= findSiblingByBrowsename(server, nodeId, &nameQN, nameNodeId);
    retval |= UA_Server_readValue(server, *nameNodeId, &nameVar);
    retval |= UA_Boolean_copy(dataValue->value.data, &bit);

    |>\tikzmarkin[set border color=martinired]{writeIO}<|PI_writeSingleIO(String_fromUA_String(nameVar.data), &bit, false);                                                 |>\tikzmarkend{writeIO}<|

    free(nameNodeId);
    return retval;
 }
\end{lstlisting}

Listing~\ref{lst:4-writeDataSourceVariable} zeigt die Callback-Methode, welche nach dem Schreiben einer Variablen auf dem OPC-Server aufgerufen wird.
Dieser Methode wird neben der NodeId der mit ihr verknüpften Variablen auch der Wert dieser in Form eines Zeigers auf ein Struct vom Typ \lstinline{UA_DataValue} übergeben.

Die Gestaltung des hier verwendeten Nodesets sieht vor, dass in einer OPC-Variablen \lstinline{"Name"} der Bezeichner des zu schreibenden Digitalausgangs hinterlegt ist, siehe Abbildung~\ref{fig:opc-do}. Dies erlaubt eine Rekonfiguration der Ein- und Ausgänge der SPS ohne Änderungen im Programmcode des OPC-Servers vornehmen zu müssen.
Es ist daher erforderlich, nach jedem Schreiben einer mit einem Digitalausgang verknüpften Variablen, hier \lstinline{"Value"}, dessen Bezeichner \lstinline{"Name"} abzufragen. 
Dies geschieht in den Zeilen 180 und 181.
Anschließend wird dieser Bezeichner sowie der zu schreibende Wert der Methode \lstinline{PI_writeSingleIO()} übergeben, welche wiederum die Interaktion mit piControl übernimmt (vgl. Abschnitt \ref{sec:4-picontrol}).
 
\subsection{Integration von piControl%
     \label{sec:4-picontrol}}
In Abschnitt~\ref{sec:2-io} wurde die Anbindung der IO-Module des Revolution Pi sowie die Funktionsweise von piControl aus Anwendersicht beschrieben. Die verfügbare Literatur beschränkt sich auch auf lediglich diese Sicht; eine weiterführende Dokumentation für Entwickler gibt es, neben der in Abschnitt~\ref{sec:3-anbindung} vorgestellten Manpage, nicht. 
In diesem Abschnitt soll daher der Quellcode von piControl sowie dessen Verwendung im Projekt genauer betrachtet werden.
Hierzu wird exemplarisch die in Abschnitt~\ref{sec:4-open62541} eingeführte Methode \lstinline{PI_writeSingleIO()} untersucht.
Diese Methode ermöglicht das Setzen eines einzelnen Bits im Prozessabbild der SPS, und damit das Schalten eines digitalen Ausgangs auf einem IO-Modul.
Die äquivalente Methode \lstinline{int piControlGetBitValue(SPIValue *pSpiValue)} zum Lesen eines Bits bzw. Eingangs funktioniert analog und soll daher an dieser Stelle nicht dediziert erörtert werden.

\begin{lstlisting}[language={c},firstnumber=97,
                   caption={Setzen eines phsikalischen, digitalen Ausgangs in \lstinline{revpi.c}
                   \label{lst:4-PI_writeSingleIO}}]
extern void PI_writeSingleIO(char *pszVariableName, bool *bit, bool verbose)
{
	int rc;
	SPIVariable sPiVariable;
	SPIValue sPIValue;

	strncpy(sPiVariable.strVarName, pszVariableName, sizeof(sPiVariable.strVarName));
	rc = piControlGetVariableInfo(&sPiVariable);
	if (rc < 0) {
		printf("Cannot find variable '%s'\n", pszVariableName);
		return;
	}

		sPIValue.i16uAddress = sPiVariable.i16uAddress;
		sPIValue.i8uBit = sPiVariable.i8uBit;
		sPIValue.i8uValue = *bit;
		rc = |>\tikzmarkin[set border color=martinired]{setBitValue}<|piControlSetBitValue(&sPIValue)|>\tikzmarkend{setBitValue}<|;
		if (rc < 0)
			printf("Set bit error %s\n", getWriteError(rc));
		else if (verbose)
			printf("Set bit %d on byte at offset %d. Value %d\n", sPIValue.i8uBit, sPIValue.i16uAddress,
			       sPIValue.i8uValue);
}
\end{lstlisting}

Der Programmcode in Listing~\ref{lst:4-PI_writeSingleIO} ist Teil des implementierten OPC-Servers. In diesem wird auf zwei Funktionen des piControl-Treibers zugegriffen. 
Beiden Methoden wird als Argument ein Zeiger auf ein Struct vom Typ \lstinline{SPIValue} übergeben. Der im Struct abgelegte Name wird mittels \lstinline{piControlGetVariableInfo(&sPIValue)} zu einer Adresse im Prozessabbild aufgelöst. Diese wird in \lstinline{sPIValue.i16uAdress} gespeichert. Der Wert der Variablen wird anschließend mittels \lstinline{piControlSetBitValue(&sPIValue)} an dieser Adresse in das Prozessabbild geschrieben.

\begin{lstlisting}[language={c},firstnumber=309,caption={Methode \lstinline{piControlSetBitValue} in \lstinline{piControlIf.c}\label{lst:4-piControlSetBitValue}}]
int |>\tikzmarkin[set border color=martiniblue]{setBitValueFcn}<|piControlSetBitValue(SPIValue *pSpiValue)|>\tikzmarkend{setBitValueFcn}<|
{
    piControlOpen();

    if (PiControlHandle_g < 0)
	    return -ENODEV;

    pSpiValue->i16uAddress += pSpiValue->i8uBit / 8;
    pSpiValue->i8uBit %= 8;

    if (|>\tikzmarkin[set border color=martinired]{ioctl}<|ioctl(PiControlHandle_g, KB_SET_VALUE, pSpiValue)|>\tikzmarkend{ioctl}<| < 0)
	    return errno;

    return 0;
}
\end{lstlisting}

Die in Listing~\ref{lst:4-piControlSetBitValue} dargestellte Methode \lstinline{piControlSetBitValue} ist lediglich eine Hüllfunktion (häufig auch als Wrapper-Funktion bezeichnet) für einen Aufruf des \lstinline{ioctl} Kernel-Moduls.
Folgende Parameter werden übergeben:
\lstinline{PiControlHandle_g} ist die Referenz auf die Geräte-Datei des piControl-Treibers. \lstinline{KB_SET_VALUE} ist das ioctl-Kommando zum Schreiben eines Bits in das Prozessabbild. Der Zeiger \lstinline{pSpiValue} verweist auf ein Struct des bereits vorgestellten Typs \lstinline{SPIValue}.

\begin{lstlisting}[language={c},firstnumber=80,caption={Methode \lstinline{piControlOpen} in \lstinline{piControlIf.c}\label{lst:4-piControlOpen}}]
void piControlOpen(void)
{
    /* open handle if needed */
    if (PiControlHandle_g < 0)
    {
	    |>\tikzmarkin[set border color=martiniblue]{PiControlHandle}<|PiControlHandle_g = open(PICONTROL_DEVICE, O_RDWR)|>\tikzmarkend{PiControlHandle}<|;
    }
}
\end{lstlisting}

Die in Listing~\ref{lst:4-piControlOpen} dargestellte Methode öffnet, sofern nicht bereits geschehen, die Geräte-Datei. Das Macro \lstinline{PICONTROL_DEVICE} verweist hierbei auf \lstinline{/dev/piControl0}.

\begin{lstlisting}[language={c},firstnumber=721,caption={Methode \lstinline{piControlIoctl} in \lstinline{piControlMain.c}\label{lst:4-piControlIoctl}}]
static long |>\tikzmarkin[set border color=martiniblue, below offset=0.9em]{piControlIoctl}<|piControlIoctl(struct file *file, unsigned int prg_nr, 
                           unsigned long usr_addr)                                      |>\tikzmarkend{piControlIoctl}<|
{
  int status = -EFAULT;
  tpiControlInst *priv;
  int timeout = 10000;	// ms

  if (prg_nr != KB_CONFIG_SEND && prg_nr != KB_CONFIG_START && !isRunning()) {
  	return -EAGAIN;
  }

  priv = (tpiControlInst *) file->private_data;

  if (prg_nr != KB_GET_LAST_MESSAGE) {
  	// clear old message
  	priv->pcErrorMessage[0] = 0;
  }

  switch (prg_nr) {|>\setcounter{lstnumber}{864}<|

    case |>\tikzmarkin[set border color=martiniblue]{KB_SET_VALUE}<|KB_SET_VALUE:|>\tikzmarkend{KB_SET_VALUE}<|
  		{
  			SPIValue *pValue = (SPIValue *) usr_addr;

  			if (!isRunning())
  				return -EFAULT;

  			if (pValue->i16uAddress >= KB_PI_LEN) {
  				status = -EFAULT;
  			} else {
  				INT8U i8uValue_l;
  				my_rt_mutex_lock(&piDev_g.lockPI);
  				i8uValue_l = piDev_g.ai8uPI[pValue->i16uAddress];

  				if (pValue->i8uBit >= 8) {
  					i8uValue_l = pValue->i8uValue;
  				} else {
  					if (pValue->i8uValue)
  						i8uValue_l |= (1 << pValue->i8uBit);
  					else
  						i8uValue_l &= ~(1 << pValue->i8uBit);
  				}

  				|>\tikzmarkin[set border color=martinired]{i8uValue}<|piDev_g.ai8uPI[pValue->i16uAddress] = i8uValue_l;|>\tikzmarkend{i8uValue}<|
  				rt_mutex_unlock(&piDev_g.lockPI);

  #ifdef VERBOSE
  				pr_info("piControlIoctl Addr=%u, bit=%u: %02x %02x\n", pValue->i16uAddress, pValue->i8uBit, pValue->i8uValue, i8uValue_l);
  #endif

  				status = 0;
  			}
  		}
  		break; |>\setcounter{lstnumber}{1314}<|

    default:
      pr_err("Invalid Ioctl");
      return (-EINVAL);
      break;

    }

    return status;
  }
\end{lstlisting}

Listing~\ref{lst:4-piControlIoctl} zeigt in Auszügen die ioctl-Methode des piControl Kernel-Treibers. Diese bekommt folgende Argumente übergeben: \lstinline{struct file *file} enthält den Verweis auf die Geräte-Datei, hier \lstinline{/dev/piControl0}. Der Wert von \lstinline{unsigned int prg_nr} beschreibt die Anfrage an den Treiber, in diesem Fall \lstinline{KB_SET_VALUE}. Das Argument \lstinline{unsigned long usr_addr} enthält einen typ-agnostischen Pointer. Dieser verweist auf einen Speicherbereich, in welchem die zur Bearbeitung der Anfrage notwendigen Daten abgelegt sind. Hier können auch vom Treiber empfangene Daten dem Anwendungsprogramm bereitgestellt werden. 

Die switch-case-Anweisung führt die über das Argument \lstinline{prg_nr} spezifizierte Aktion aus. Hier betrachten wir \lstinline{KB_SET_VALUE}:
Zunächst wird in Zeile 868 der übergebene Zeiger \lstinline{usr_addr} mittels explizitem Typecast zu einem Zeiger des Typs \lstinline{SPIValue *} konvertiert. Da dieser auf Daten im Userspace verweist, ist beim Zugriff durch den Kernel-Treiber besondere Vorsicht geboten.
In Zeile 877 wird mittels Mutex das Prozessabbild \lstinline{piDev_g} für den Zugriff durch andere Threads oder Prozesse gesperrt.
\lstinline{my_rt_mutex_lock} verweist hierbei auf die Funktion \lstinline{rt_mutex_lock} aus \lstinline{linux/sched.h}\footnote{Offenbar wurde hier auch eine alternative Implementierung vorgesehen, siehe revpi\_common.h}

In Zeile 889 wird das Byte \lstinline{i8uValue_l}, welches den zu schreibenden Wert enthält in das Prozessabbild übertragen. Anschließend wird die Mutex auf \lstinline{piDev_g} wieder entsperrt.
\newpage

\begin{lstlisting}[language={c},firstnumber=62,caption={Auszug des Struct \lstinline{spiControlDev} in \lstinline{piControlMain.h}\label{lst:4-spiControlDev}}]
|>\tikzmarkin[set border color=martiniblue]{spiControlDev}<|typedef struct spiControlDev|>\tikzmarkend{spiControlDev}<| {
	// device driver stuff
	int init_step;
	enum revpi_machine machine_type;
	void *machine;
	struct cdev cdev;	// Char device structure
	struct device *dev;
	struct thermal_zone_device *thermal_zone;

	|>\tikzmarkin[set border color=martiniblue]{processImage}<|// process image stuff
	INT8U ai8uPI[KB_PI_LEN];
	INT8U ai8uPIDefault|>\tikzmarkin[set border color=martinired]{KB_PI_LEN_0}<|[KB_PI_LEN]|>\tikzmarkend{KB_PI_LEN_0}<|;
	struct rt_mutex lockPI;        |>\tikzmarkend{processImage}<|
	bool stopIO;
	piDevices *devs; |>\setcounter{lstnumber}{94}<|
} tpiControlDev;
\end{lstlisting}

Das Prozessabbild ist als Byte-Array der Länge \lstinline{KB_PI_LEN} in Listing~\ref{lst:4-spiControlDev} definiert. Konfigurationsparameter wie \lstinline{KB_PI_LEN} oder die Zykluszeit für den Datenaustausch zwischen SPS und IO-Modulen sind im folgenden Listing~\ref{lst:4-process} definiert.

\begin{lstlisting}[language={c},firstnumber=119,caption={Konfigurationsparameter des Prozessabbildes in project.h\label{lst:4-process}}]
#define INTERVAL_PI_GATE (5*1000*1000)  // 5 ms piGateCommunication |>\setcounter{lstnumber}{128}<|

#define INTERVAL_IO_COM (5*1000*1000)  // 5 ms piIoComm |>\setcounter{lstnumber}{132}<|

#define KB_PD_LEN       512
|>\tikzmarkin[set border color=martiniblue]{KB_PI_LEN_1}<|#define KB_PI_LEN       4096|>\tikzmarkend{KB_PI_LEN_1}<|
\end{lstlisting}

Das zu setzende Bit wurde zu diesem Zeitpunkt erfolgreich in das Prozessabbild der SPS geschrieben.
Es stellt sich die Frage, wie dieses nun an das IO-Modul kommuniziert wird.
Die Kommunikation mit allen angebundenen Modulen ist ebenfalls Aufgabe des piControl-Treibers.

\begin{lstlisting}[language={c},firstnumber=256,caption={Auszug der Methode \lstinline{piIoThread} in \lstinline{revpi_core.c}\label{lst:4-piIoThread}}]
static int piIoThread(void *data)
{
	//TODO int value = 0;
	ktime_t time;
	ktime_t now;
	s64 tDiff;

	hrtimer_init(&piCore_g.ioTimer, CLOCK_MONOTONIC, HRTIMER_MODE_ABS);
	piCore_g.ioTimer.function = piIoTimer;

	pr_info("piIO thread started\n");

	now = hrtimer_cb_get_time(&piCore_g.ioTimer);

	PiBridgeMaster_Reset();

	while (!kthread_should_stop()) {
		if (|>\tikzmarkin[set border color=martinired]{PiBridgeMaster}<|PiBridgeMaster_Run()|>\tikzmarkend{PiBridgeMaster}<| < 0)
			break;
	}

	RevPiDevice_finish();

	pr_info("piIO exit\n");
	return 0;
}
\end{lstlisting}

Der Kernel-Thread \lstinline{piIoThread} ist verantwortlich für den zyklischen Datenaustausch mit den IO-Modulen. In diesem wird fortlaufend die Methode \lstinline{PiBridgeMaster_Run()} aufgerufen, siehe Listing~\ref{lst:4-piIoThread}.

\begin{lstlisting}[language={c},firstnumber=262,caption={Auszug der Methode \lstinline{PiBridgeMaster_Run(void)} in \lstinline{RevPiDevice.c}\label{lst:4-PiBridgeMaster_Run}}]
int PiBridgeMaster_Run(void)
{
	static kbUT_Timer tTimeoutTimer_s;
	static kbUT_Timer tConfigTimeoutTimer_s;
	static int error_cnt;
	static INT8U last_led;
	static unsigned long last_update;
	int ret = 0;
	int i;

	my_rt_mutex_lock(&piCore_g.lockBridgeState);
	if (piCore_g.eBridgeState != piBridgeStop) {
		switch (eRunStatus_s) { |>\setcounter{lstnumber}{514}<|
		    case enPiBridgeMasterStatus_EndOfConfig:|>\setcounter{lstnumber}{621}<|
		    if (|>\tikzmarkin[set border color=martinired]{RevPiDevice}<|RevPiDevice_run()|>\tikzmarkend{RevPiDevice}<|) {
				// an error occured, check error limits |>\setcounter{lstnumber}{641}<|
			} else {
				ret = 1;
			}
			piCore_g.image.drv.i16uRS485ErrorCnt = RevPiDevice_getErrCnt();
			break;
\end{lstlisting}

Die in Listing~\ref{lst:4-PiBridgeMaster_Run} dargestellte Methode ist eine sog. State-Machine. Ist die Konfiguration der IO-Module erfolgreich abgeschlossen, so führt sie bei Aufruf lediglich die Methode \lstinline{RevPiDevice_run()} aus.

\begin{lstlisting}[language={c},firstnumber=140,caption={Auszug der Methode \lstinline{RevPiDevice_run(void)} in \lstinline{RevPiDevice.c}\label{lst:4-RevPiDevice_run}}]
int RevPiDevice_run(void)
{
	INT8U i8uDevice = 0;
	INT32U r;
	int retval = 0;

	RevPiDevices_s.i16uErrorCnt = 0;

	for (i8uDevice = 0; i8uDevice < RevPiDevice_getDevCnt(); i8uDevice++) {
		if (RevPiDevice_getDev(i8uDevice)->i8uActive) {
			switch (RevPiDevice_getDev(i8uDevice)->sId.i16uModulType) {
			case KUNBUS_FW_DESCR_TYP_PI_DIO_14:
			case KUNBUS_FW_DESCR_TYP_PI_DI_16:
			case KUNBUS_FW_DESCR_TYP_PI_DO_16:
				r = |>\tikzmarkin[set border color=martinired]{sendCyclicTelegram}<|piDIOComm_sendCyclicTelegram(i8uDevice)|>\tikzmarkend{sendCyclicTelegram}\setcounter{lstnumber}{166} <|;

				break; |>\setcounter{lstnumber}{216}<|
			}
		}
	} |>\setcounter{lstnumber}{227}<|
	return retval;
}
\end{lstlisting}

Diese iteriert wie in Listing~\ref{lst:4-RevPiDevice_run} abgebildete durch alle gegenwärtig in der SPS konfigurierten Module. Ist das aktuelle Modul als aktiv markiert, so wird anhand eines sog. Firmware-Descriptors entschieden, welche Methode für die Ansteuerung des Moduls aufzurufen ist.

\begin{lstlisting}[language={c},firstnumber=161,caption={Auszug der Methode \lstinline{piDIOComm_sendCyclicTelegram} in \lstinline{piDIOComm.c}\label{lst:4-sendCyclicTelegram}}]
INT32U piDIOComm_sendCyclicTelegram(INT8U i8uDevice_p)
{
	INT32U i32uRv_l = 0;
	SIOGeneric sRequest_l;
	SIOGeneric sResponse_l;
	INT8U len_l, data_out[18], i, p, data_in[70];
	INT8U i8uAddress;
	int ret; |>\setcounter{lstnumber}{239}<|
	
    |>\tikzmarkin[set border color=martinired]{piIoComm}<|ret = piIoComm_send((INT8U *) & sRequest_l, IOPROTOCOL_HEADER_LENGTH + len_l + 1);  |>\tikzmarkend{piIoComm}\setcounter{lstnumber}{298}<|
}
\end{lstlisting}

Im Falle des hier verwendeten DO-Moduls wird die in Listing~\ref{lst:4-sendCyclicTelegram} abgebildete Methode \lstinline{piDIOComm_sendCyclicTelegram()} aufgerufen. Dieser wird ein Zeiger auf das zu schreibende Gerät übergeben. 
Zunächst wird das Prozessabbild mittels eines proprietären, jedoch im Quellcode offen nachvollziehbaren Protokolls in ein \lstinline{sRequest_l} genanntes Byte-Array umgewandelt. Dieser Schritt ist in Listing~\ref{lst:4-sendCyclicTelegram} nicht abgebildet. Anschließend wird \lstinline{piIoComm_send()} ein Zeiger auf die so generierte Schreib-Anfrage übergeben.

\begin{lstlisting}[language={c},firstnumber=220,caption={Auszug der Methode \lstinline{piIOComm_send} in \lstinline{piIOComm.c}\label{lst:4-piIOComm_send}}]
int piIoComm_send(INT8U * buf_p, INT16U i16uLen_p)
{
	ssize_t write_l = 0;
	INT16U i16uSent_l = 0;|>\setcounter{lstnumber}{249}<|

	while (i16uSent_l < i16uLen_p) {
		write_l = vfs_write(piIoComm_fd_m, buf_p + i16uSent_l, i16uLen_p - i16uSent_l, &piIoComm_fd_m->f_pos);
		if (write_l < 0) {
			pr_info_serial("write error %d\n", (int)write_l);
			return -1;
		} 
		i16uSent_l += write_l;|>\setcounter{lstnumber}{263}<|
	}
	clear();
	vfs_fsync(piIoComm_fd_m, 1);
	return 0;
}
\end{lstlisting}

Listing~\ref{lst:4-piIOComm_send} zeigt die Implementierung von \lstinline{piIoComm_send()}. Diese Methode ist für das Schreiben der oben generierten Anfrage auf die seriellen Schnittstelle verantwortlich. Realisiert wird dies mittels der Methode \lstinline{vfs_write()}. Diese ist in \lstinline{<linux/fs.h>} definiert. Sie ermöglicht das Schreiben einer Datei im Userspace aus dem Kernel heraus. Geschrieben wird hier die Datei mit dem Deskriptor \lstinline{piIoComm_fd_m}.
Da die Funktion \lstinline{vfs_write()} durch andere Kernel-Tasks unterbrochen werden kann, ist nicht gewährleistet, dass die gesamte Anfrage mit nur einem Aufruf geschrieben wird. Die oben abgebildete while-Schleife stellt das vollständige Senden der Anfrage sicher.

\begin{lstlisting}[language={c},firstnumber=157,caption={Auszug der Methode \lstinline{piIOComm_open_serial} in \lstinline{piIOComm.c}\label{lst:4-piIOComm_open_serial}}]
int piIoComm_open_serial(void)
{   |>\setcounter{lstnumber}{167}<|
	struct file *fd;	/* Filedeskriptor */
	struct termios newtio;	/* Schnittstellenoptionen */

	|>\tikzmarkin[set border color=martiniblue]{fd}<|/* Port oeffnen - read/write, kein "controlling tty", 
	    Status von DCD ignorieren */
	fd = filp_open(|>\tikzmarkin[set border color=martinired]{tty}<|REV_PI_TTY_DEVICE|>\tikzmarkend{tty}<|, O_RDWR | O_NOCTTY, 0); |>\setcounter{lstnumber}{208}<|
	
	piIoComm_fd_m = fd;                                                      |>\tikzmarkend{fd}\setcounter{lstnumber}{217}<|

	return 0;
}
\end{lstlisting}

Der zum Schreiben auf die serielle Schnittstelle verwendete Datei-Deskriptor wird von der in Listing~\ref{lst:4-piIOComm_open_serial} abgebildeten Methode \lstinline{piIoComm_open_serial()} generiert. 

\begin{lstlisting}[language={c},firstnumber=45,caption={Definition der seriellen Schnittstelle in \lstinline{piIOComm.h}\label{lst:4-REV_PI_TTY_DEVICE}}]
#define REV_PI_TTY_DEVICE	"/dev/ttyAMA0"
\end{lstlisting}

Das in Listing~\ref{lst:4-REV_PI_TTY_DEVICE} definierte Macro verweist auf eine der seriellen Schnittstellen des RaspberryPi.
Die Implementierung des zugehörigen Schnittstellentreibers soll hier nicht weiter untersucht werden. Somit ist an dieser Stelle die Kette vom Setzen einer Variablen auf dem OPC-Server bis hin zur Aktualisierung des Prozessabbilds der IO-Module geschlossen.

% \begin{lstlisting}[language={c},firstnumber={226},caption={Setzen der Scheduler-Priorität auf SCHED\_FIFO in 
% revpi\_common.c\label{lst:2-sched_priority}}]
% param.sched_priority = ktprio->prio;
% ret = sched_setscheduler(child, SCHED_FIFO, &param);
% \end{lstlisting}
% % % Imports nur für Referenzenauflösung während des Schreibens! Vorm Kompilieren auskommentieren!
% \bibliography{0_hauptdatei}
% % Mit \section{...} eröffnen wir einen neuen Abschnitt.
% Der Befehl setzt nicht nur den Text in einer größeren,
% fetten Schrift, sondern sorgt außerdem dafür, daß er im
% Inhaltsverzeichnis erscheint.
%
% Mit \label{...} erzeugen wir einen Bezeichner, mit dessen Hilfe
% wir später auf die Nummer des Abschnitts verweisen können (nämlich
% mit~\ref{...}).
%
% Das Kommentarzeichen hinter „Übersicht“ dient dazu, ein
% Leerzeichen zwischen „Übersicht“ und dem \label-Befehl
% zu vermeiden, das andernfalls sichtbar würde – z.B. im
% Inhaltsverzeichnis.
%

% % Imports nur für Referenzenauflösung während des Schreibens! Vorm Kompilieren auskommentieren!
% \bibliography{0_hauptdatei}
% \input{1_einleitung}
%\input{2_grundlagen}
%\input{3_konzeption}
%\input{4_implementierung}
%\input{5_tests}
%\input{6_zusammenfassung}
% % Ende Imports

\section{Einleitung und Motivation%
  \label{sec:1-einleitung}}
Ziel dieses Projektes ist die Integration eines OPC-Servers mit einer auf Linux
basierenden speicherprogrammierbaren Steuerung (SPS). Angeschlossen an diese SPS
ist jeweils ein digitales Ein-/\,bzw.~Ausgabemodul. Die von diesen bereitgestellten
Ein-/\, bzw.~Ausgänge (IO) sollen in der Datenstruktur des OPC-Servers abgebildet
und über diesen für OPC-Clients les-/\,und schreibar sein. Weiterhin sollen einige
Funktionen zur Überwachung und Steuerung der an die SPS angeschlossenen Aktoren
und Sensoren direkt im OPC-Server implementiert werden.
Hiermit stellt dieses Projekt eine der Grundlagen für ein übergeordnetes Projekt,
die cloudbasierte Steuerung eines miniaturisierten Produktions-Systems, dar.

Der hier verwendete OPC-Server ist Teil des sog. open62541 Projekts. Er ist in C
geschrieben und implementiert bereits einen großen Teil der im OPC-UA-Standard
spezifizierten Funktionen.
Als SPS findet ein Revolution Pi 3 der Firma Kunbus Verwendung. Dieser integriert
ein sog. Compute Module der Raspberry Pi Foundation in ein industrietaugliches
Gehäuse und erlaubt die Erweiterung mittels IO- oder Gateway-Modulen. Über diese
erfolgt die Kommunikation mit weiteren Komponenten der Automatisierungstechnik.

Motiviert ist dieses Projekt durch die Beobachtung, dass die Verbreitung offener
Standards sowie freier Software auch in der Automatisierungstechnik zunimmt.
Linux ist ein freies Betriebssystem, OPC-UA ein offen zugänglicher, aktiv gepflegter
und weit verbreiteter Standard. Der Raspberry Pi findet sowohl bei Hobby-Anwendern als
auch in den Bereichen Forschung und Entwicklung sowie bei industriellen Anwendern
Verwendung. Dieses Projekt stellt somit eine für unterschiedliche Anwender interessante
Entwicklung dar.

Im Anschluss an diese einleitende Übersicht im Abschnitt~\ref{sec:1-einleitung} folgt
die Darstellung der wichtigsten Grundlagen in Abschnitt~\ref{sec:2-grundlagen}.
Aufbauend auf diesen Grundlagen folgt die konzeptuelle Ausarbeitung im Abschnitt~\ref{sec:3-konzeption}.
Die Umsetzung wird im Abschnitt~\ref{sec:4-implementierung} erläutert.
Die Leistungsfähigkeit der Implementierung wird in Abschnitt~\ref{sec:5-tests} untersucht.
Eine Zusammenfassung und ein Ausblick schließen die Arbeit in
Abschnitt~\ref{sec:6-fazit} ab. Eventuell noch benötigte Anhänge
finden sich in den Anhängen [...] bis [...].

% % % Imports nur für Referenzenauflösung während des Schreibens! Vorm Kompilieren auskommentieren!
% \bibliography{0_hauptdatei}
% \input{1_einleitung}
% \input{2_grundlagen}
% \input{3_konzeption}
% \input{4_implementierung}
% \input{5_tests}
% \input{6_zusammenfassung}
% % Ende Imports

\section{Grundlagen%
  \label{sec:2-grundlagen}}

\subsection{Speicherprogrammierbare-Steuerung und Linux -- Revolution Pi%
     \label{sec:2-sps}}

\subsubsection{Kunbus RevolutionPi%
        \label{sec:2-revpi}}
Der RevolutionPi 3 ist eine speicherprogrammierbare Steuerung (SPS) des Herstellers
Kunbus GmbH. Kern dieser SPS ist das von der Raspberry Pi Foundation entwickelte
und vertriebene Raspberry Pi Compute Module 3. Dieses integriert ein Broadcom BCM2837
System-on-Chip (SoC) mit vier 1,2GHz Prozessorkernen, 1GB RAM, 4GB eMMC Anwendungsspeicher
und sonstige Peripherie in ein Modul im DDR2-SODIMM Formfaktor. Diese Spezifikationen
sind weitgehend identisch zu denen des ausgesprochen populären Raspberry Pi 3.
Der Revolution Pi profitiert daher von dem gleichen großen Angebot an Software
und Unterstützung wie der Raspberry Pi, ergänzt dessen Hardware jedoch um eine 24V
Spannungsversorgung, die Möglichkeit der Erweiterung durch mehrere industrietaugliche
Ein-/ Ausgabemodule und Gateways sowie ein Gehäuse zur Montage auf einer DIN-Schiene.
\begin{itemize}
  \item{Prozessor: BCM2837}
  \item{Taktfrequenz 1,2 GHz}
  \item{Anzahl Prozessorkerne: 4}
  \item{Arbeitsspeicher: 1 GByte}
  \item{eMMC Flash Speicher: 4 GByte}
  \item{Betriebssystem: Angepasstes Raspbian mit RT-Patch}
  \item{RTC mit 24h Pufferung über wartungsfreien Kondensator}
  \item{Treiber / API: Treiber schreibt zyklisch Prozessdaten in ein Prozessabbild, Zugriff auf Prozessabbild über Linux-Filesystem als API zu Fremdsoftware.}
  \item{Kommunikationsanschlüsse: 2 x USB 2.0 A (je 500 mA belastbar), 1 x Micro-USB, HDMI, Ethernet (RJ45) 10/100 Mbit/s}
  \item{Stromversorgung: min. 10,7 V, max. 28,8 V, maximal 10 Watt}
  \item{Zulässige Umgebungstemperatur: -40 bis +55 C}
  \item{Gehäuseabmessungen: (HxBxL) 96 mm x 22,5 mm x 110,5 mm (ohne gesteckte Stecker)}
  \item{ESD Schutz: 4 kV / 8 kV gemäß EN61131-2 und IEC 61000-6-2}
  \item{Surge / Burst Prüfungen: gemäß EN61131-2 und IEC 61000-6-2 eingekoppelt auf Versorgungsspannung, Ethernet und IO-Leitungen}
  \item{EMI Prüfungen: gemäß EN61131-2 und IEC 61000-6-2}
\end{itemize}

Kunbus bietet eine Auswahl an IO- und Gateway-Modulen zur Erweiterung des Revolution Pi an.
Gateways dienen der Kommunikation mit Systemen oder Komponenten der Automatisierungstechnik
über Protokolle wie PROFIBUS oder EtherCAT. IO-Module erlauben die Überwachung
und Steuerung von digitalen oder analogen Ein- und Ausgängen.

\subsubsection{Zugriff auf IO-Module%
        \label{sec:2-io}}
Der Zugriff auf die Ein- und Ausgänge der IO-Module erfolgt über ein Prozessabbild
und einen hierfür von Kunbus bereitgestellten Treiber, genannt piControl. Dieser
aktualisiert das Prozessabbild zyklisch. Die angestrebte Zykluszeit beträgt 5ms,
kann jedoch je nach Anzahl der angeschlossenen Module auch größer sein. Kunbus
garantiert bei drei IO-Modulen und zwei Gateway-Modulen eine Zykluszeit von 10 ms.
Jedes der IO-Module stellt ein eigenständiges eingebettetes System dar. Es verfügt
über einen Microcontroller, welcher die IOs bereitstellt und über einen RS485-Bus
mit dem Revolution Pi kommuniziert.
% https://revolution.kunbus.de/io-modul/

Lizenz: GPL
% https://github.com/RevolutionPi/piControl

\begin{lstlisting}[language={c},firstnumber={226},caption={Setzen der Scheduler-Priorität auf SCHED\_FIFO in revpi\_common.c\label{lst:2-sched_priority}}]
param.sched_priority = ktprio->prio;
ret = sched_setscheduler(child, SCHED_FIFO,
       &param);
\end{lstlisting}


\subsection{Echtzeit und Multithreading unter Linux -- preemptRT und posix%
     \label{sec:2-echtzeit}}


 Der Linux-Kernel verfügt über mehrere unterschiedliche Preemtion-Modelle:

\begin{itemize}
  \item No Forced Preemption (server):
  Ausgelegt auf maximal möglichen Durchsatz, lediglich Interrupts und
  System-Call-Returns bewirken Präemption.

  \item Voluntary Kernel Preemption (Desktop):
  Neben den implizit bevorrechtigten Interrupts und System-Call-Returns gibt es
  in diesem Modell weitere Abschnitte des Kernels in welchen Preämption explizit
  gestattet ist.

  \item Preemptible Kernel (Low-Latency Desktop):
  In diesem Modell ist der gesamte Kernel, mit Ausnahme sog.~kritischer Abschnitte
  präemptible. Nach jedem kritischen Abschnitt gibt es einen impliziten Präemptions-Punkt.

  \item Preemptible Kernel (Basic RT):
  Dieses Modell ist dem zuvor genannten sehr ähnlich, hier sind jedoch alle Interrupt-Handler
  als eigenständige Threads ausgeführt.

  \item Fully Preemptible Kernel (RT):
  Wie auch bei den beiden zuvor genannten Modellen ist hier der gesamte Kernel
  präemtible, die Anzahl und Dauer der nicht-präemtiblen kritischen Abschnitte
  ist auf ein notwendiges Minimum beschränkt. Alle Interrupt-Handler sind als
  eigenständige Threads ausgeführt, Spinlocks durch Sleeping-Spinlocks und Mutexe
  durch sog.~RT-Mutexe ersetzt.

\end{itemize}
\todo{Spinlocks und Mutexe sowie die RT-Varianten dieser erklären!}

Lediglich mit dem vollständig präemtiblen Kernel kann Echtzeit-Verhalten realisiert werden.

% https://wiki.linuxfoundation.org/realtime/documentation/technical_basics/preemption_models bzw kernel/Kconfig.preempt

\subsubsection{preemptRT%
        \label{sec:2-preemptRT}}
% https://wiki.linuxfoundation.org/realtime/documentation/technical_details/start
% https://wiki.linuxfoundation.org/realtime/documentation/technical_basics/start

Das dem PREEMPT RT Kernel zugrunde liegende Prinzip lässt sich in einer einfachen
Regel ausdrücken: Nur Code, welcher absolut nicht-präemtible sein darf, ist es
gestattet nicht-präemtible zu sein.
Das erklärte Ziel des PREEMPT\_RT Patches ist es folglich, die Menge des nicht-präemtiblen
Codes im Linux-Kernel auf das absolut notwendige Minimum zu reduzieren.

Dies wird durch Verwendung folgender Mechanismen erreicht:

\begin{itemize}
  \item Hochauflösende Timer
  \item Sleeping Spinlocks
  \item Threaded Interrupt Handlers
  \item rt\_mutex
  \item RCU
\end{itemize}


\subsubsection{posix%
        \label{sec:2-posix}}
Ist posix hier wirklich relevant? Debian bzw.~Raspbian sind weitgehend posix
kompatibel, aber wird es hier genutzt? -> JA, open62541 nutzt pthread.h
piControl nutzt kthread.h, und semaphore.h

\subsection{OPC-UA und open62541%
     \label{sec:2-opc}}

\subsubsection{OPC UA%
        \label{sec:2-opcua}}
Open Platform Communications (OPC) ist eine Familie von Standards zur herstellerunabhängigen
Kommunikation von Maschinen (M2M) in der Automatisierungstechnik. Die sog.~OPC Task Force, zu deren
Mitgliedern verschiedene große Firmen der Automatisierungsindustrie gehören, veröffentlichte
die OPC Specification Version 1.0 im August 1996.
Motiviert ist dieser offene Standard durch die Erkenntniss, dass die Anpassung der
zahlreichen Herstellerstandards an individuelle Infrastrukturen und Anlagen einen
großen Mehraufwand verursachen.
Die Wikipedia beschreibt das Anwendungsgebiet für OPC wie folgt:

\glqq{}OPC wird dort eingesetzt, wo Sensoren, Regler und Steuerungen verschiedener Hersteller
ein gemeinsames Netzwerk bilden. Ohne OPC benötigten zwei Geräte zum Datenaustausch
genaue Kenntnis über die Kommunikationsmöglichkeiten des Gegenübers. Erweiterungen
und Austausch gestalten sich entsprechend schwierig. Mit OPC genügt es, für jedes
Gerät genau einmal einen OPC-konformen Treiber zu schreiben. Idealerweise wird
dieser bereits vom Hersteller zur Verfügung gestellt. Ein OPC-Treiber lässt sich
ohne großen Anpassungsaufwand in beliebig große Steuer- und Überwachungssysteme
integrieren.

OPC unterteilt sich in verschiedene Unterstandards, die für den jeweiligen Anwendungsfall
unabhängig voneinander implementiert werden können. OPC lässt sich damit verwenden
für Echtzeitdaten (Überwachung), Datenarchivierung, Alarm-Meldungen und neuerdings
auch direkt zur Steuerung (Befehlsübermittlung).\grqq{}

OPC basiert in der ursprünglichen Spezifikation auf Microsofts DCOM-Spezifikation.
DCOM macht Funktionen und Objekte einer Anwendung anderen Anwendungen im Netzwerk
zugänglich. Der OPC-Standard definiert entsprechende DCOM-Objekte um mit anderen
OPC-Anwendungen Daten austauschen zu können. Die Verwendung von DCOM bindet Anwender
an Betriebssysteme von Microsoft. Die ursprüngliche OPC Spezifikation wird durch die
Entwicklung von OPC Unified Architecture (OPC UA) abgelöst.
OPC UA setzt auf einem eigenen Kommunikationionsstack auf, die Verwendung von DCOM
und damit die Bindung an Microsoft wurden aufgelöst.

Die OPC-UA-Architektur ist eine Service-orientierte Architektur (SOA), deren Struktur
aus mehreren Schichten besteht.

% Wikipedia
Das OPC-Informationsmodell ist nicht mehr nur eine Hierarchie aus Ordnern, Items
und Properties. Es ist ein sogenanntes Full-Mesh-Network aus Nodes, mit dem neben
den Nutzdaten eines Nodes auch Meta- und Diagnoseinformationen repräsentiert werden.
Ein Node ähnelt einem Objekt aus der objektorientierten Programmierung. Ein Node
kann Attribute besitzen, die gelesen werden können (Data Access (DA), Historical
Data Access (HDA)). Es ist möglich Methoden zu definieren und aufzurufen.
Eine Methode besitzt Aufrufargumente und Rückgabewerte. Sie wird durch ein Command
aufgerufen. Weiterhin werden Events unterstützt, die versendet werden können
(AE (Alarms \& Events), DA DataChange), um bestimmte Informationen zwischen Geräten
auszutauschen. Ein Event besitzt unter anderem einen Empfangszeitpunkt, eine Nachricht
und einen Schweregrad. Die o. g. Nodes werden sowohl für die Nutzdaten als auch
alle anderen Arten von Metadaten verwendet. Der damit modellierte OPC-Adressraum
beinhaltet nun auch ein Typmodell, mit dem sämtliche Datentypen spezifiziert werden.

% https://de.wikipedia.org/wiki/Open_Platform_Communications
% https://de.wikipedia.org/wiki/OPC_Unified_Architecture
% https://opcfoundation.org/developer-tools/specifications-unified-architecture
% Von Gerhard Gappmeier - ascolab GmbH, CC BY-SA 3.0, https://de.wikipedia.org/w/index.php?curid=1892069
\subsubsection{open62541%
        \label{sec:2-open62541}}
open62541 ist eine offene und freie Implementierung von OPC UA. Die in C geschriebene
Bibliothek stellt eine beständig zunehmende Anzahl der im OPC UA Standard definierten
Funktionen bereit. Sie kann sowohl zur Erstellung von OPC-Servern als auch -Clients
genutzt werden. Ergänzend zu der unter der Mozilla Public License v2.0 lizensierten
Bibliothek stellt das open62541 Projekt auch Beispielprogramme unter einer CC0 Lizenz
zur Verfügung.

Die Bibliothek eignet sich auch für die Entwicklung auf eingebetteten Systemen und
Microcontrollern. Je nach Umfang der gewünschten Funktionen und des OPC Informationsmodells
beträgt die Größe einer Server-Binary weniger als 100kb. %evtl. kürzen?

\todo{Nodes erklären! Evtl.~oben!}

Folgende Auswahl an Eigenschaften und Funktionen zeichnet die in dieser Arbeit verwendete
Version 0.3 von open62541 aus:
\begin{itemize}
  \item Kommunikationionsstack
  \begin{itemize}
      \item OPC UA Binär-Protokoll (HTTP oder SOAP werden gegenwärtig nicht unterstützt)
      \item Austauschbare Netzwerk-Schicht, welche die Verwendung eigener Netzwerk-APIs
      erlaubt.
      \item Verschlüsselte Kommunikationion
      \item Asynchrone Dienst-Anfragen im Client
  \end{itemize}
  \item Informationsmodell
  \begin{itemize}
    \item Unterstützung aller OPC UA Node-Typen, inkl.~Methoden
    \item Hinzufügen und Entfernen von Nodes und Referenzen zur Laufzeit.
    \item Vererbung und Instanziierung von Objekt- und Variablentypen
    \item Zugriffskontrolle auch für einzelne Nodes
  \end{itemize}
  \item Subscriptions
  \begin{itemize}
    \item Erlaubt die Überwachung (subscriptions / monitoreditems)
    \item Sehr geringer Ressourcenbedarf pro überwachtem Wert
  \end{itemize}
  \item Code-Generierung auf XML-Basis
  \begin{itemize}
    \item Erlaubt die Erstellung von Datentypen
    \item Erlaubt die Generierung des serverseitigen Informationsmodells
  \end{itemize}
\end{itemize}

% https://open62541.org/doc/0.3/


Mozilla Public License
CC0 Lizenz für Beispiele und Plugins

% https://open62541.org/doc/open62541-current.pdf
% https://open62541.org/

% % % Imports nur für Referenzenauflösung während des Schreibens! Vorm Kompilieren auskommentieren!
% \bibliography{0_hauptdatei}
% \input{1_einleitung}
% \input{2_grundlagen}
% \input{3_konzeption}
% \input{4_implementierung}
% \input{5_tests}
% \input{6_zusammenfassung}
% \input{anhang}
% % Ende Imports

\section{Systemkonzept%
  \label{sec:3-konzeption}}
Auf Basis der in Abschnitt \ref{sec:2-grundlagen} vorgestellten Möglichkeiten folgt nun die Ausarbeitung eines Konzepts.
In den folgenden Abschnitten soll näher auf zwei zentrale Aspekte eingegangen werden: Abschnitt~\ref{sec:3-anbindung} stellt Möglichkeiten zum Zugriff auf Variablen bzw.\,Werte im Prozessabbild des Revolution Pi vor; in Abschnitt~\ref{sec:3-integration} wird ein Konzept zur Bereitstellung dieser Variablen auf einem OPC-Server vorgestellt.

\subsection{Anbindung der IO an den OPC-Server%
     \label{sec:3-anbindung}}

Eine Webanwendung mit Bezeichnung PiCtory dient zur Konfiguration der I/O- und virtuellen Module des RevolutionPi. Die Konfiguration liegt im JSON-Format in der Datei \lstinline{/etc/revpi/config.rsc}. Der piControl-Treiber liest diese Datei beim Start. 
Der folgende Auszug aus der Manpage des piControl-Kernelmoduls beschreibt die von diesem zum Lesen und Schreiben einzelner Bits des Prozessabbildes bereitgestellten Funktionen~\citep[vgl.]{web-revpi-manpage}. Sie ist an dieser Stelle weitgehend ungekürzt zitiert, da sie die nutzbare Schnittstelle sehr kompakt beschreibt.

\begin{lstlisting}[breakindent=0pt, numbers=none, caption={Auszug aus der Revolution Pi Programmers Manual\label{lst:4-manpage}}]
KB_FIND_VARIABLE SPIVariable *argp
Find a variable in the process image by its name. A pointer to a structure of type SPIVariable must be passed as argument. [...]
The struct SPIVariable [...] is defined as 
typedef struct SPIVariableStr
{
    char strVarName[32]; // Variable name
    uint16_t i16uAddress; // Address of the byte in the process image
    uint8_t i8uBit; // 0-7 bit position, >= 8 whole byte
    uint16_t i16uLength; // length of the variable in bits.
    // Possible values are 1, 8, 16 and 32
} SPIVariable;

Set and get values of the process image
KB_GET_VALUE SPIValue *argp
[...]
KB_SET_VALUE SPIValue *argp
Write one bit or one byte to the process image [...].  This call is more efficient than the usual calls of seek and write because only one function call is necessary. If more than on application are writing bits in one output byte, this call is the only safe way to set a bit without overwriting the other bits because this call is doing a read-modify-write-cycle. 

The struct SPIValue used by this ioctl is defined as
typedef struct SPIValueStr
{
    uint16_t i16uAddress; // Address of the byte in the process image
    uint8_t i8uBit; // 0-7 bit position, >= 8 whole byte
    uint8_t i8uValue; // Value: 0/1 for bit access, whole byte otherwise
} SPIValue;
\end{lstlisting} 

Die oben beschriebenden Funtkionen \lstinline{KB_FIND_VARIABLE}, \lstinline{KB_GET_VALUE} und \lstinline{KB_SET_VALUE} ermöglichen einen einfachen und (lt.\,Manpage) effizienten Zugriff auf einzelne Bits des Prozessabbildes und damit auch auf die IO des RevolutionPi.
Der Zugriff des OPC-Servers auf das Prozessabbild soll daher mittels dieser Funktionen realisiert werden.
\lstinline{KB_FIND_VARIABLE} kann genutzt werden, um Adressen von Variablen im Prozessabbild mittels ihres Namens aufzulösen.
\lstinline{KB_GET_VALUE} und \lstinline{KB_SET_VALUE} ermöglichen den Zugriff auf die Werte dieser Variablen.


\subsection{Integration des OPC-Servers in das System%
     \label{sec:3-integration}}

open62541 bietet drei Möglichkeiten zum Abgleich von Variablen mit dem Prozessabbild~\citep[vgl.][Tutorials - Connecting a Variable with a Physical Process]{web-open62541}:
\begin{itemize}
    \item Manuelles oder zyklisches Aktualisieren
    \item Variable Value Callback
    \item Variable Datasource
\end{itemize}

Die zyklische Aktualisierung eines oder mehrerer Werte nimmt, abhängig von der Zykluszeit, viele Systemressourcen in Anspruch. Value Callbacks ermöglichen es, einen Variablenwert effizienter mit einer Ressource wie etwa einem Prozessabbild zu synchronisieren. An die Variable wird ein Callback angehängt, welches vor jedem Lesen und nach jedem Schreibvorgang ausgeführt wird.
Der Wert der Variablen wird weiterhin im Variablenknoten auf dem OPC-Server gespeichert, der Abgleich mit der verknüpften Ressource erfolgt durch die Callback-Methoden.

Sogenannte Datenquellen gehen noch einen Schritt weiter. Der Server leitet jede Lese- und Schreibanforderung direkt an eine Callback-Funktion weiter. Beim Lesen liefert der Rückruf eine Kopie des aktuellen Wertes. Die Datenquelle muss intern ein eigenes Speichermanagement implementieren.

Der Zugriff auf die Werte des Prozessabbildes erfolgt, wie in Abschnitt~\ref{sec:3-anbindung} beschrieben, über von piControl bereitgestellte Methoden. Um die durch open62541 gepflegte OPC-Datenstruktur und das durch piControl verwaltete Prozessabbild möglichst effektiv verknüpfen zu können, soll diese Interaktion mittels Datenquellen und den zugehörigen Callbacks implementiert werden.
% % % Imports nur für Referenzenauflösung während des Schreibens! Vorm Kompilieren auskommentieren!
% \bibliography{0_hauptdatei}
% \input{1_einleitung}
% \input{2_grundlagen}
% \input{3_konzeption}
% \input{4_implementierung}
% \input{5_tests}
% \input{6_zusammenfassung}
% \input{anhang}
% % Ende Imports

\section{Implementierung%
  \label{sec:4-implementierung}}
Das folgende Kapitel stellt in Auszügen die Implementierung des OPC-Servers sowie die Anbindung an die IO-Module
der SPS dar. Der Schwerpunkt liegt hierbei auf der Funktionsweise des piControl-Treibers und dessen Integration in das Projekt. Abschnitt~\ref{sec:4-picontrol} erklärt die zum Schreibens eines Bits verwendeten Funktionsaufrufe.
Zuvor soll jedoch in Abschnitt~\ref{sec:4-open62541} der Teil des OPC-Servers vorgestellt werden, welcher auf besagten Treiber zugreift. 

\subsection{Implementierung des OPC-Servers%
     \label{sec:4-open62541}}
Wie im vorangegangenen Abschnitt~\ref{sec:3-integration} begründet, soll die Verknüpfung zwischen dem Prozessabbild der SPS und den auf dem OPC-Server bereitgestellten Werten über sog.\,Datenquellen erfolgen. Hierzu ist zunächst eine Callback-Methode zu implementieren, welche bei einem Lese- oder Schreibzugriff auf eine Variable aufgerufen wird. Die Verknüpfung zwischen Callback-Methode und Variable muss manuell erfolgen.

\begin{lstlisting}[language={c},firstnumber=237,caption={Auszug der Methode \lstinline{linkDataSourceVariable} in \lstinline{variables.c}\label{lst:4-linkDataSourceVariable}}]
extern UA_StatusCode
 linkDataSourceVariable(UA_Server *server, UA_NodeId nodeId) {
     bool readonly = false;
     UA_DataSource dataSourceVariable;
     UA_StatusCode rc; |>\setcounter{lstnumber}{254}<|

     dataSourceVariable.read = readDataSourceVariable;
     if (!readonly)
        dataSourceVariable.write = writeDataSourceVariable;
     else
        dataSourceVariable.write = writeReadonlyDataSourceVariable;

     return UA_Server_setVariableNode_dataSource(server, nodeId, dataSourceVariable);
 }
\end{lstlisting}

\begin{figure}[h]
    \centering
    \includegraphics[width=0.42\textwidth]{doc/img/OPC_RevPiDO.pdf}
    \caption{Auszug des verwendeten Nodesets, hier Digitalausgang 1 des Versuchsaufbaus
      \label{fig:opc-do}}
\end{figure}

Die in Listing~\ref{lst:4-linkDataSourceVariable} abgebildete Methode \lstinline{linkDataSourceVariable()} erzeugt ein Struct vom Typ \lstinline{UA_DataSource}. In diesem werden dem Lesen und Schreiben einer OPC-Variablen entsprechende Callback-Methoden zugewiesen. Die Verknüpfung einer OPC-Variable, genauer ihrer NodeId, mit der zuvor definierten Datenquelle erfolgt über die von open62541 bereitgestellte Methode \lstinline{UA_Server_setVariableNode_dataSource()}. Vor dem Lesen und nach dem Schreiben dieser Variable werden von nun an die entsprechenden Callbacks aufgerufen.
     
\begin{lstlisting}[language={c},firstnumber=168,caption={Auszug des Callbacks \lstinline{writeDataSourceVariable} in \lstinline{variables.c}\label{lst:4-writeDataSourceVariable}}]  
extern UA_StatusCode
 writeDataSourceVariable(UA_Server *server,
            const UA_NodeId *sessionId, void *sessionContext,
            const UA_NodeId *nodeId, void *nodeContext,
            const UA_NumericRange *range, const UA_DataValue *dataValue) {

    UA_StatusCode retval  = UA_STATUSCODE_GOOD;
    UA_NodeId *nameNodeId = UA_malloc(sizeof(UA_NodeId));
    UA_QualifiedName nameQN = UA_QUALIFIEDNAME(1, "Name");
    UA_Variant nameVar;
    UA_Boolean bit;

    retval |= findSiblingByBrowsename(server, nodeId, &nameQN, nameNodeId);
    retval |= UA_Server_readValue(server, *nameNodeId, &nameVar);
    retval |= UA_Boolean_copy(dataValue->value.data, &bit);

    |>\tikzmarkin[set border color=martinired]{writeIO}<|PI_writeSingleIO(String_fromUA_String(nameVar.data), &bit, false);                                                 |>\tikzmarkend{writeIO}<|

    free(nameNodeId);
    return retval;
 }
\end{lstlisting}

Listing~\ref{lst:4-writeDataSourceVariable} zeigt die Callback-Methode, welche nach dem Schreiben einer Variablen auf dem OPC-Server aufgerufen wird.
Dieser Methode wird neben der NodeId der mit ihr verknüpften Variablen auch der Wert dieser in Form eines Zeigers auf ein Struct vom Typ \lstinline{UA_DataValue} übergeben.

Die Gestaltung des hier verwendeten Nodesets sieht vor, dass in einer OPC-Variablen \lstinline{"Name"} der Bezeichner des zu schreibenden Digitalausgangs hinterlegt ist, siehe Abbildung~\ref{fig:opc-do}. Dies erlaubt eine Rekonfiguration der Ein- und Ausgänge der SPS ohne Änderungen im Programmcode des OPC-Servers vornehmen zu müssen.
Es ist daher erforderlich, nach jedem Schreiben einer mit einem Digitalausgang verknüpften Variablen, hier \lstinline{"Value"}, dessen Bezeichner \lstinline{"Name"} abzufragen. 
Dies geschieht in den Zeilen 180 und 181.
Anschließend wird dieser Bezeichner sowie der zu schreibende Wert der Methode \lstinline{PI_writeSingleIO()} übergeben, welche wiederum die Interaktion mit piControl übernimmt (vgl. Abschnitt \ref{sec:4-picontrol}).
 
\subsection{Integration von piControl%
     \label{sec:4-picontrol}}
In Abschnitt~\ref{sec:2-io} wurde die Anbindung der IO-Module des Revolution Pi sowie die Funktionsweise von piControl aus Anwendersicht beschrieben. Die verfügbare Literatur beschränkt sich auch auf lediglich diese Sicht; eine weiterführende Dokumentation für Entwickler gibt es, neben der in Abschnitt~\ref{sec:3-anbindung} vorgestellten Manpage, nicht. 
In diesem Abschnitt soll daher der Quellcode von piControl sowie dessen Verwendung im Projekt genauer betrachtet werden.
Hierzu wird exemplarisch die in Abschnitt~\ref{sec:4-open62541} eingeführte Methode \lstinline{PI_writeSingleIO()} untersucht.
Diese Methode ermöglicht das Setzen eines einzelnen Bits im Prozessabbild der SPS, und damit das Schalten eines digitalen Ausgangs auf einem IO-Modul.
Die äquivalente Methode \lstinline{int piControlGetBitValue(SPIValue *pSpiValue)} zum Lesen eines Bits bzw. Eingangs funktioniert analog und soll daher an dieser Stelle nicht dediziert erörtert werden.

\begin{lstlisting}[language={c},firstnumber=97,
                   caption={Setzen eines phsikalischen, digitalen Ausgangs in \lstinline{revpi.c}
                   \label{lst:4-PI_writeSingleIO}}]
extern void PI_writeSingleIO(char *pszVariableName, bool *bit, bool verbose)
{
	int rc;
	SPIVariable sPiVariable;
	SPIValue sPIValue;

	strncpy(sPiVariable.strVarName, pszVariableName, sizeof(sPiVariable.strVarName));
	rc = piControlGetVariableInfo(&sPiVariable);
	if (rc < 0) {
		printf("Cannot find variable '%s'\n", pszVariableName);
		return;
	}

		sPIValue.i16uAddress = sPiVariable.i16uAddress;
		sPIValue.i8uBit = sPiVariable.i8uBit;
		sPIValue.i8uValue = *bit;
		rc = |>\tikzmarkin[set border color=martinired]{setBitValue}<|piControlSetBitValue(&sPIValue)|>\tikzmarkend{setBitValue}<|;
		if (rc < 0)
			printf("Set bit error %s\n", getWriteError(rc));
		else if (verbose)
			printf("Set bit %d on byte at offset %d. Value %d\n", sPIValue.i8uBit, sPIValue.i16uAddress,
			       sPIValue.i8uValue);
}
\end{lstlisting}

Der Programmcode in Listing~\ref{lst:4-PI_writeSingleIO} ist Teil des implementierten OPC-Servers. In diesem wird auf zwei Funktionen des piControl-Treibers zugegriffen. 
Beiden Methoden wird als Argument ein Zeiger auf ein Struct vom Typ \lstinline{SPIValue} übergeben. Der im Struct abgelegte Name wird mittels \lstinline{piControlGetVariableInfo(&sPIValue)} zu einer Adresse im Prozessabbild aufgelöst. Diese wird in \lstinline{sPIValue.i16uAdress} gespeichert. Der Wert der Variablen wird anschließend mittels \lstinline{piControlSetBitValue(&sPIValue)} an dieser Adresse in das Prozessabbild geschrieben.

\begin{lstlisting}[language={c},firstnumber=309,caption={Methode \lstinline{piControlSetBitValue} in \lstinline{piControlIf.c}\label{lst:4-piControlSetBitValue}}]
int |>\tikzmarkin[set border color=martiniblue]{setBitValueFcn}<|piControlSetBitValue(SPIValue *pSpiValue)|>\tikzmarkend{setBitValueFcn}<|
{
    piControlOpen();

    if (PiControlHandle_g < 0)
	    return -ENODEV;

    pSpiValue->i16uAddress += pSpiValue->i8uBit / 8;
    pSpiValue->i8uBit %= 8;

    if (|>\tikzmarkin[set border color=martinired]{ioctl}<|ioctl(PiControlHandle_g, KB_SET_VALUE, pSpiValue)|>\tikzmarkend{ioctl}<| < 0)
	    return errno;

    return 0;
}
\end{lstlisting}

Die in Listing~\ref{lst:4-piControlSetBitValue} dargestellte Methode \lstinline{piControlSetBitValue} ist lediglich eine Hüllfunktion (häufig auch als Wrapper-Funktion bezeichnet) für einen Aufruf des \lstinline{ioctl} Kernel-Moduls.
Folgende Parameter werden übergeben:
\lstinline{PiControlHandle_g} ist die Referenz auf die Geräte-Datei des piControl-Treibers. \lstinline{KB_SET_VALUE} ist das ioctl-Kommando zum Schreiben eines Bits in das Prozessabbild. Der Zeiger \lstinline{pSpiValue} verweist auf ein Struct des bereits vorgestellten Typs \lstinline{SPIValue}.

\begin{lstlisting}[language={c},firstnumber=80,caption={Methode \lstinline{piControlOpen} in \lstinline{piControlIf.c}\label{lst:4-piControlOpen}}]
void piControlOpen(void)
{
    /* open handle if needed */
    if (PiControlHandle_g < 0)
    {
	    |>\tikzmarkin[set border color=martiniblue]{PiControlHandle}<|PiControlHandle_g = open(PICONTROL_DEVICE, O_RDWR)|>\tikzmarkend{PiControlHandle}<|;
    }
}
\end{lstlisting}

Die in Listing~\ref{lst:4-piControlOpen} dargestellte Methode öffnet, sofern nicht bereits geschehen, die Geräte-Datei. Das Macro \lstinline{PICONTROL_DEVICE} verweist hierbei auf \lstinline{/dev/piControl0}.

\begin{lstlisting}[language={c},firstnumber=721,caption={Methode \lstinline{piControlIoctl} in \lstinline{piControlMain.c}\label{lst:4-piControlIoctl}}]
static long |>\tikzmarkin[set border color=martiniblue, below offset=0.9em]{piControlIoctl}<|piControlIoctl(struct file *file, unsigned int prg_nr, 
                           unsigned long usr_addr)                                      |>\tikzmarkend{piControlIoctl}<|
{
  int status = -EFAULT;
  tpiControlInst *priv;
  int timeout = 10000;	// ms

  if (prg_nr != KB_CONFIG_SEND && prg_nr != KB_CONFIG_START && !isRunning()) {
  	return -EAGAIN;
  }

  priv = (tpiControlInst *) file->private_data;

  if (prg_nr != KB_GET_LAST_MESSAGE) {
  	// clear old message
  	priv->pcErrorMessage[0] = 0;
  }

  switch (prg_nr) {|>\setcounter{lstnumber}{864}<|

    case |>\tikzmarkin[set border color=martiniblue]{KB_SET_VALUE}<|KB_SET_VALUE:|>\tikzmarkend{KB_SET_VALUE}<|
  		{
  			SPIValue *pValue = (SPIValue *) usr_addr;

  			if (!isRunning())
  				return -EFAULT;

  			if (pValue->i16uAddress >= KB_PI_LEN) {
  				status = -EFAULT;
  			} else {
  				INT8U i8uValue_l;
  				my_rt_mutex_lock(&piDev_g.lockPI);
  				i8uValue_l = piDev_g.ai8uPI[pValue->i16uAddress];

  				if (pValue->i8uBit >= 8) {
  					i8uValue_l = pValue->i8uValue;
  				} else {
  					if (pValue->i8uValue)
  						i8uValue_l |= (1 << pValue->i8uBit);
  					else
  						i8uValue_l &= ~(1 << pValue->i8uBit);
  				}

  				|>\tikzmarkin[set border color=martinired]{i8uValue}<|piDev_g.ai8uPI[pValue->i16uAddress] = i8uValue_l;|>\tikzmarkend{i8uValue}<|
  				rt_mutex_unlock(&piDev_g.lockPI);

  #ifdef VERBOSE
  				pr_info("piControlIoctl Addr=%u, bit=%u: %02x %02x\n", pValue->i16uAddress, pValue->i8uBit, pValue->i8uValue, i8uValue_l);
  #endif

  				status = 0;
  			}
  		}
  		break; |>\setcounter{lstnumber}{1314}<|

    default:
      pr_err("Invalid Ioctl");
      return (-EINVAL);
      break;

    }

    return status;
  }
\end{lstlisting}

Listing~\ref{lst:4-piControlIoctl} zeigt in Auszügen die ioctl-Methode des piControl Kernel-Treibers. Diese bekommt folgende Argumente übergeben: \lstinline{struct file *file} enthält den Verweis auf die Geräte-Datei, hier \lstinline{/dev/piControl0}. Der Wert von \lstinline{unsigned int prg_nr} beschreibt die Anfrage an den Treiber, in diesem Fall \lstinline{KB_SET_VALUE}. Das Argument \lstinline{unsigned long usr_addr} enthält einen typ-agnostischen Pointer. Dieser verweist auf einen Speicherbereich, in welchem die zur Bearbeitung der Anfrage notwendigen Daten abgelegt sind. Hier können auch vom Treiber empfangene Daten dem Anwendungsprogramm bereitgestellt werden. 

Die switch-case-Anweisung führt die über das Argument \lstinline{prg_nr} spezifizierte Aktion aus. Hier betrachten wir \lstinline{KB_SET_VALUE}:
Zunächst wird in Zeile 868 der übergebene Zeiger \lstinline{usr_addr} mittels explizitem Typecast zu einem Zeiger des Typs \lstinline{SPIValue *} konvertiert. Da dieser auf Daten im Userspace verweist, ist beim Zugriff durch den Kernel-Treiber besondere Vorsicht geboten.
In Zeile 877 wird mittels Mutex das Prozessabbild \lstinline{piDev_g} für den Zugriff durch andere Threads oder Prozesse gesperrt.
\lstinline{my_rt_mutex_lock} verweist hierbei auf die Funktion \lstinline{rt_mutex_lock} aus \lstinline{linux/sched.h}\footnote{Offenbar wurde hier auch eine alternative Implementierung vorgesehen, siehe revpi\_common.h}

In Zeile 889 wird das Byte \lstinline{i8uValue_l}, welches den zu schreibenden Wert enthält in das Prozessabbild übertragen. Anschließend wird die Mutex auf \lstinline{piDev_g} wieder entsperrt.
\newpage

\begin{lstlisting}[language={c},firstnumber=62,caption={Auszug des Struct \lstinline{spiControlDev} in \lstinline{piControlMain.h}\label{lst:4-spiControlDev}}]
|>\tikzmarkin[set border color=martiniblue]{spiControlDev}<|typedef struct spiControlDev|>\tikzmarkend{spiControlDev}<| {
	// device driver stuff
	int init_step;
	enum revpi_machine machine_type;
	void *machine;
	struct cdev cdev;	// Char device structure
	struct device *dev;
	struct thermal_zone_device *thermal_zone;

	|>\tikzmarkin[set border color=martiniblue]{processImage}<|// process image stuff
	INT8U ai8uPI[KB_PI_LEN];
	INT8U ai8uPIDefault|>\tikzmarkin[set border color=martinired]{KB_PI_LEN_0}<|[KB_PI_LEN]|>\tikzmarkend{KB_PI_LEN_0}<|;
	struct rt_mutex lockPI;        |>\tikzmarkend{processImage}<|
	bool stopIO;
	piDevices *devs; |>\setcounter{lstnumber}{94}<|
} tpiControlDev;
\end{lstlisting}

Das Prozessabbild ist als Byte-Array der Länge \lstinline{KB_PI_LEN} in Listing~\ref{lst:4-spiControlDev} definiert. Konfigurationsparameter wie \lstinline{KB_PI_LEN} oder die Zykluszeit für den Datenaustausch zwischen SPS und IO-Modulen sind im folgenden Listing~\ref{lst:4-process} definiert.

\begin{lstlisting}[language={c},firstnumber=119,caption={Konfigurationsparameter des Prozessabbildes in project.h\label{lst:4-process}}]
#define INTERVAL_PI_GATE (5*1000*1000)  // 5 ms piGateCommunication |>\setcounter{lstnumber}{128}<|

#define INTERVAL_IO_COM (5*1000*1000)  // 5 ms piIoComm |>\setcounter{lstnumber}{132}<|

#define KB_PD_LEN       512
|>\tikzmarkin[set border color=martiniblue]{KB_PI_LEN_1}<|#define KB_PI_LEN       4096|>\tikzmarkend{KB_PI_LEN_1}<|
\end{lstlisting}

Das zu setzende Bit wurde zu diesem Zeitpunkt erfolgreich in das Prozessabbild der SPS geschrieben.
Es stellt sich die Frage, wie dieses nun an das IO-Modul kommuniziert wird.
Die Kommunikation mit allen angebundenen Modulen ist ebenfalls Aufgabe des piControl-Treibers.

\begin{lstlisting}[language={c},firstnumber=256,caption={Auszug der Methode \lstinline{piIoThread} in \lstinline{revpi_core.c}\label{lst:4-piIoThread}}]
static int piIoThread(void *data)
{
	//TODO int value = 0;
	ktime_t time;
	ktime_t now;
	s64 tDiff;

	hrtimer_init(&piCore_g.ioTimer, CLOCK_MONOTONIC, HRTIMER_MODE_ABS);
	piCore_g.ioTimer.function = piIoTimer;

	pr_info("piIO thread started\n");

	now = hrtimer_cb_get_time(&piCore_g.ioTimer);

	PiBridgeMaster_Reset();

	while (!kthread_should_stop()) {
		if (|>\tikzmarkin[set border color=martinired]{PiBridgeMaster}<|PiBridgeMaster_Run()|>\tikzmarkend{PiBridgeMaster}<| < 0)
			break;
	}

	RevPiDevice_finish();

	pr_info("piIO exit\n");
	return 0;
}
\end{lstlisting}

Der Kernel-Thread \lstinline{piIoThread} ist verantwortlich für den zyklischen Datenaustausch mit den IO-Modulen. In diesem wird fortlaufend die Methode \lstinline{PiBridgeMaster_Run()} aufgerufen, siehe Listing~\ref{lst:4-piIoThread}.

\begin{lstlisting}[language={c},firstnumber=262,caption={Auszug der Methode \lstinline{PiBridgeMaster_Run(void)} in \lstinline{RevPiDevice.c}\label{lst:4-PiBridgeMaster_Run}}]
int PiBridgeMaster_Run(void)
{
	static kbUT_Timer tTimeoutTimer_s;
	static kbUT_Timer tConfigTimeoutTimer_s;
	static int error_cnt;
	static INT8U last_led;
	static unsigned long last_update;
	int ret = 0;
	int i;

	my_rt_mutex_lock(&piCore_g.lockBridgeState);
	if (piCore_g.eBridgeState != piBridgeStop) {
		switch (eRunStatus_s) { |>\setcounter{lstnumber}{514}<|
		    case enPiBridgeMasterStatus_EndOfConfig:|>\setcounter{lstnumber}{621}<|
		    if (|>\tikzmarkin[set border color=martinired]{RevPiDevice}<|RevPiDevice_run()|>\tikzmarkend{RevPiDevice}<|) {
				// an error occured, check error limits |>\setcounter{lstnumber}{641}<|
			} else {
				ret = 1;
			}
			piCore_g.image.drv.i16uRS485ErrorCnt = RevPiDevice_getErrCnt();
			break;
\end{lstlisting}

Die in Listing~\ref{lst:4-PiBridgeMaster_Run} dargestellte Methode ist eine sog. State-Machine. Ist die Konfiguration der IO-Module erfolgreich abgeschlossen, so führt sie bei Aufruf lediglich die Methode \lstinline{RevPiDevice_run()} aus.

\begin{lstlisting}[language={c},firstnumber=140,caption={Auszug der Methode \lstinline{RevPiDevice_run(void)} in \lstinline{RevPiDevice.c}\label{lst:4-RevPiDevice_run}}]
int RevPiDevice_run(void)
{
	INT8U i8uDevice = 0;
	INT32U r;
	int retval = 0;

	RevPiDevices_s.i16uErrorCnt = 0;

	for (i8uDevice = 0; i8uDevice < RevPiDevice_getDevCnt(); i8uDevice++) {
		if (RevPiDevice_getDev(i8uDevice)->i8uActive) {
			switch (RevPiDevice_getDev(i8uDevice)->sId.i16uModulType) {
			case KUNBUS_FW_DESCR_TYP_PI_DIO_14:
			case KUNBUS_FW_DESCR_TYP_PI_DI_16:
			case KUNBUS_FW_DESCR_TYP_PI_DO_16:
				r = |>\tikzmarkin[set border color=martinired]{sendCyclicTelegram}<|piDIOComm_sendCyclicTelegram(i8uDevice)|>\tikzmarkend{sendCyclicTelegram}\setcounter{lstnumber}{166} <|;

				break; |>\setcounter{lstnumber}{216}<|
			}
		}
	} |>\setcounter{lstnumber}{227}<|
	return retval;
}
\end{lstlisting}

Diese iteriert wie in Listing~\ref{lst:4-RevPiDevice_run} abgebildete durch alle gegenwärtig in der SPS konfigurierten Module. Ist das aktuelle Modul als aktiv markiert, so wird anhand eines sog. Firmware-Descriptors entschieden, welche Methode für die Ansteuerung des Moduls aufzurufen ist.

\begin{lstlisting}[language={c},firstnumber=161,caption={Auszug der Methode \lstinline{piDIOComm_sendCyclicTelegram} in \lstinline{piDIOComm.c}\label{lst:4-sendCyclicTelegram}}]
INT32U piDIOComm_sendCyclicTelegram(INT8U i8uDevice_p)
{
	INT32U i32uRv_l = 0;
	SIOGeneric sRequest_l;
	SIOGeneric sResponse_l;
	INT8U len_l, data_out[18], i, p, data_in[70];
	INT8U i8uAddress;
	int ret; |>\setcounter{lstnumber}{239}<|
	
    |>\tikzmarkin[set border color=martinired]{piIoComm}<|ret = piIoComm_send((INT8U *) & sRequest_l, IOPROTOCOL_HEADER_LENGTH + len_l + 1);  |>\tikzmarkend{piIoComm}\setcounter{lstnumber}{298}<|
}
\end{lstlisting}

Im Falle des hier verwendeten DO-Moduls wird die in Listing~\ref{lst:4-sendCyclicTelegram} abgebildete Methode \lstinline{piDIOComm_sendCyclicTelegram()} aufgerufen. Dieser wird ein Zeiger auf das zu schreibende Gerät übergeben. 
Zunächst wird das Prozessabbild mittels eines proprietären, jedoch im Quellcode offen nachvollziehbaren Protokolls in ein \lstinline{sRequest_l} genanntes Byte-Array umgewandelt. Dieser Schritt ist in Listing~\ref{lst:4-sendCyclicTelegram} nicht abgebildet. Anschließend wird \lstinline{piIoComm_send()} ein Zeiger auf die so generierte Schreib-Anfrage übergeben.

\begin{lstlisting}[language={c},firstnumber=220,caption={Auszug der Methode \lstinline{piIOComm_send} in \lstinline{piIOComm.c}\label{lst:4-piIOComm_send}}]
int piIoComm_send(INT8U * buf_p, INT16U i16uLen_p)
{
	ssize_t write_l = 0;
	INT16U i16uSent_l = 0;|>\setcounter{lstnumber}{249}<|

	while (i16uSent_l < i16uLen_p) {
		write_l = vfs_write(piIoComm_fd_m, buf_p + i16uSent_l, i16uLen_p - i16uSent_l, &piIoComm_fd_m->f_pos);
		if (write_l < 0) {
			pr_info_serial("write error %d\n", (int)write_l);
			return -1;
		} 
		i16uSent_l += write_l;|>\setcounter{lstnumber}{263}<|
	}
	clear();
	vfs_fsync(piIoComm_fd_m, 1);
	return 0;
}
\end{lstlisting}

Listing~\ref{lst:4-piIOComm_send} zeigt die Implementierung von \lstinline{piIoComm_send()}. Diese Methode ist für das Schreiben der oben generierten Anfrage auf die seriellen Schnittstelle verantwortlich. Realisiert wird dies mittels der Methode \lstinline{vfs_write()}. Diese ist in \lstinline{<linux/fs.h>} definiert. Sie ermöglicht das Schreiben einer Datei im Userspace aus dem Kernel heraus. Geschrieben wird hier die Datei mit dem Deskriptor \lstinline{piIoComm_fd_m}.
Da die Funktion \lstinline{vfs_write()} durch andere Kernel-Tasks unterbrochen werden kann, ist nicht gewährleistet, dass die gesamte Anfrage mit nur einem Aufruf geschrieben wird. Die oben abgebildete while-Schleife stellt das vollständige Senden der Anfrage sicher.

\begin{lstlisting}[language={c},firstnumber=157,caption={Auszug der Methode \lstinline{piIOComm_open_serial} in \lstinline{piIOComm.c}\label{lst:4-piIOComm_open_serial}}]
int piIoComm_open_serial(void)
{   |>\setcounter{lstnumber}{167}<|
	struct file *fd;	/* Filedeskriptor */
	struct termios newtio;	/* Schnittstellenoptionen */

	|>\tikzmarkin[set border color=martiniblue]{fd}<|/* Port oeffnen - read/write, kein "controlling tty", 
	    Status von DCD ignorieren */
	fd = filp_open(|>\tikzmarkin[set border color=martinired]{tty}<|REV_PI_TTY_DEVICE|>\tikzmarkend{tty}<|, O_RDWR | O_NOCTTY, 0); |>\setcounter{lstnumber}{208}<|
	
	piIoComm_fd_m = fd;                                                      |>\tikzmarkend{fd}\setcounter{lstnumber}{217}<|

	return 0;
}
\end{lstlisting}

Der zum Schreiben auf die serielle Schnittstelle verwendete Datei-Deskriptor wird von der in Listing~\ref{lst:4-piIOComm_open_serial} abgebildeten Methode \lstinline{piIoComm_open_serial()} generiert. 

\begin{lstlisting}[language={c},firstnumber=45,caption={Definition der seriellen Schnittstelle in \lstinline{piIOComm.h}\label{lst:4-REV_PI_TTY_DEVICE}}]
#define REV_PI_TTY_DEVICE	"/dev/ttyAMA0"
\end{lstlisting}

Das in Listing~\ref{lst:4-REV_PI_TTY_DEVICE} definierte Macro verweist auf eine der seriellen Schnittstellen des RaspberryPi.
Die Implementierung des zugehörigen Schnittstellentreibers soll hier nicht weiter untersucht werden. Somit ist an dieser Stelle die Kette vom Setzen einer Variablen auf dem OPC-Server bis hin zur Aktualisierung des Prozessabbilds der IO-Module geschlossen.

% \begin{lstlisting}[language={c},firstnumber={226},caption={Setzen der Scheduler-Priorität auf SCHED\_FIFO in 
% revpi\_common.c\label{lst:2-sched_priority}}]
% param.sched_priority = ktprio->prio;
% ret = sched_setscheduler(child, SCHED_FIFO, &param);
% \end{lstlisting}
% % % Imports nur für Referenzenauflösung während des Schreibens! Vorm Kompilieren auskommentieren!
% \bibliography{0_hauptdatei}
% \input{1_einleitung}
% \input{2_grundlagen}
% \input{3_konzeption}
% \input{4_implementierung}
% \input{5_tests}
% \input{6_zusammenfassung}
% % Ende Imports

\section{Test des OPC-Servers im Gesamtsystem%
  \label{sec:5-tests}}

% % % Imports nur für Referenzenauflösung während des schreibens! Vorm Kompilieren auskommentieren!
% \bibliography{0_hauptdatei}
% \input{1_einleitung}
% \input{2_grundlagen}
% \input{3_konzeption}
% \input{4_implementierung}
% \input{5_tests}
% \input{6_zusammenfassung}
% % Ende Imports

\section{Zusammenfassung und Ausblick%
  \label{sec:6-fazit}}
Der folgende Abschnitt~\ref{sec:6-zusammenfassung} fasst die gewonnenen Erkenntnisse und den Stand der Implementierung zusammen.
Den Abschluss dieser Arbeit bildet der Ausblick in Abschnitt~\ref{sec:6-ausblick}.

\subsection{Zusammenfassung%
     \label{sec:6-zusammenfassung}}

\subsection{Ausblick%
     \label{sec:6-ausblick}}

% % Ende Imports

\section{Test des OPC-Servers im Gesamtsystem%
  \label{sec:5-tests}}

% % % Imports nur für Referenzenauflösung während des schreibens! Vorm Kompilieren auskommentieren!
% \bibliography{0_hauptdatei}
% % Mit \section{...} eröffnen wir einen neuen Abschnitt.
% Der Befehl setzt nicht nur den Text in einer größeren,
% fetten Schrift, sondern sorgt außerdem dafür, daß er im
% Inhaltsverzeichnis erscheint.
%
% Mit \label{...} erzeugen wir einen Bezeichner, mit dessen Hilfe
% wir später auf die Nummer des Abschnitts verweisen können (nämlich
% mit~\ref{...}).
%
% Das Kommentarzeichen hinter „Übersicht“ dient dazu, ein
% Leerzeichen zwischen „Übersicht“ und dem \label-Befehl
% zu vermeiden, das andernfalls sichtbar würde – z.B. im
% Inhaltsverzeichnis.
%

% % Imports nur für Referenzenauflösung während des Schreibens! Vorm Kompilieren auskommentieren!
% \bibliography{0_hauptdatei}
% \input{1_einleitung}
%\input{2_grundlagen}
%\input{3_konzeption}
%\input{4_implementierung}
%\input{5_tests}
%\input{6_zusammenfassung}
% % Ende Imports

\section{Einleitung und Motivation%
  \label{sec:1-einleitung}}
Ziel dieses Projektes ist die Integration eines OPC-Servers mit einer auf Linux
basierenden speicherprogrammierbaren Steuerung (SPS). Angeschlossen an diese SPS
ist jeweils ein digitales Ein-/\,bzw.~Ausgabemodul. Die von diesen bereitgestellten
Ein-/\, bzw.~Ausgänge (IO) sollen in der Datenstruktur des OPC-Servers abgebildet
und über diesen für OPC-Clients les-/\,und schreibar sein. Weiterhin sollen einige
Funktionen zur Überwachung und Steuerung der an die SPS angeschlossenen Aktoren
und Sensoren direkt im OPC-Server implementiert werden.
Hiermit stellt dieses Projekt eine der Grundlagen für ein übergeordnetes Projekt,
die cloudbasierte Steuerung eines miniaturisierten Produktions-Systems, dar.

Der hier verwendete OPC-Server ist Teil des sog. open62541 Projekts. Er ist in C
geschrieben und implementiert bereits einen großen Teil der im OPC-UA-Standard
spezifizierten Funktionen.
Als SPS findet ein Revolution Pi 3 der Firma Kunbus Verwendung. Dieser integriert
ein sog. Compute Module der Raspberry Pi Foundation in ein industrietaugliches
Gehäuse und erlaubt die Erweiterung mittels IO- oder Gateway-Modulen. Über diese
erfolgt die Kommunikation mit weiteren Komponenten der Automatisierungstechnik.

Motiviert ist dieses Projekt durch die Beobachtung, dass die Verbreitung offener
Standards sowie freier Software auch in der Automatisierungstechnik zunimmt.
Linux ist ein freies Betriebssystem, OPC-UA ein offen zugänglicher, aktiv gepflegter
und weit verbreiteter Standard. Der Raspberry Pi findet sowohl bei Hobby-Anwendern als
auch in den Bereichen Forschung und Entwicklung sowie bei industriellen Anwendern
Verwendung. Dieses Projekt stellt somit eine für unterschiedliche Anwender interessante
Entwicklung dar.

Im Anschluss an diese einleitende Übersicht im Abschnitt~\ref{sec:1-einleitung} folgt
die Darstellung der wichtigsten Grundlagen in Abschnitt~\ref{sec:2-grundlagen}.
Aufbauend auf diesen Grundlagen folgt die konzeptuelle Ausarbeitung im Abschnitt~\ref{sec:3-konzeption}.
Die Umsetzung wird im Abschnitt~\ref{sec:4-implementierung} erläutert.
Die Leistungsfähigkeit der Implementierung wird in Abschnitt~\ref{sec:5-tests} untersucht.
Eine Zusammenfassung und ein Ausblick schließen die Arbeit in
Abschnitt~\ref{sec:6-fazit} ab. Eventuell noch benötigte Anhänge
finden sich in den Anhängen [...] bis [...].

% % % Imports nur für Referenzenauflösung während des Schreibens! Vorm Kompilieren auskommentieren!
% \bibliography{0_hauptdatei}
% \input{1_einleitung}
% \input{2_grundlagen}
% \input{3_konzeption}
% \input{4_implementierung}
% \input{5_tests}
% \input{6_zusammenfassung}
% % Ende Imports

\section{Grundlagen%
  \label{sec:2-grundlagen}}

\subsection{Speicherprogrammierbare-Steuerung und Linux -- Revolution Pi%
     \label{sec:2-sps}}

\subsubsection{Kunbus RevolutionPi%
        \label{sec:2-revpi}}
Der RevolutionPi 3 ist eine speicherprogrammierbare Steuerung (SPS) des Herstellers
Kunbus GmbH. Kern dieser SPS ist das von der Raspberry Pi Foundation entwickelte
und vertriebene Raspberry Pi Compute Module 3. Dieses integriert ein Broadcom BCM2837
System-on-Chip (SoC) mit vier 1,2GHz Prozessorkernen, 1GB RAM, 4GB eMMC Anwendungsspeicher
und sonstige Peripherie in ein Modul im DDR2-SODIMM Formfaktor. Diese Spezifikationen
sind weitgehend identisch zu denen des ausgesprochen populären Raspberry Pi 3.
Der Revolution Pi profitiert daher von dem gleichen großen Angebot an Software
und Unterstützung wie der Raspberry Pi, ergänzt dessen Hardware jedoch um eine 24V
Spannungsversorgung, die Möglichkeit der Erweiterung durch mehrere industrietaugliche
Ein-/ Ausgabemodule und Gateways sowie ein Gehäuse zur Montage auf einer DIN-Schiene.
\begin{itemize}
  \item{Prozessor: BCM2837}
  \item{Taktfrequenz 1,2 GHz}
  \item{Anzahl Prozessorkerne: 4}
  \item{Arbeitsspeicher: 1 GByte}
  \item{eMMC Flash Speicher: 4 GByte}
  \item{Betriebssystem: Angepasstes Raspbian mit RT-Patch}
  \item{RTC mit 24h Pufferung über wartungsfreien Kondensator}
  \item{Treiber / API: Treiber schreibt zyklisch Prozessdaten in ein Prozessabbild, Zugriff auf Prozessabbild über Linux-Filesystem als API zu Fremdsoftware.}
  \item{Kommunikationsanschlüsse: 2 x USB 2.0 A (je 500 mA belastbar), 1 x Micro-USB, HDMI, Ethernet (RJ45) 10/100 Mbit/s}
  \item{Stromversorgung: min. 10,7 V, max. 28,8 V, maximal 10 Watt}
  \item{Zulässige Umgebungstemperatur: -40 bis +55 C}
  \item{Gehäuseabmessungen: (HxBxL) 96 mm x 22,5 mm x 110,5 mm (ohne gesteckte Stecker)}
  \item{ESD Schutz: 4 kV / 8 kV gemäß EN61131-2 und IEC 61000-6-2}
  \item{Surge / Burst Prüfungen: gemäß EN61131-2 und IEC 61000-6-2 eingekoppelt auf Versorgungsspannung, Ethernet und IO-Leitungen}
  \item{EMI Prüfungen: gemäß EN61131-2 und IEC 61000-6-2}
\end{itemize}

Kunbus bietet eine Auswahl an IO- und Gateway-Modulen zur Erweiterung des Revolution Pi an.
Gateways dienen der Kommunikation mit Systemen oder Komponenten der Automatisierungstechnik
über Protokolle wie PROFIBUS oder EtherCAT. IO-Module erlauben die Überwachung
und Steuerung von digitalen oder analogen Ein- und Ausgängen.

\subsubsection{Zugriff auf IO-Module%
        \label{sec:2-io}}
Der Zugriff auf die Ein- und Ausgänge der IO-Module erfolgt über ein Prozessabbild
und einen hierfür von Kunbus bereitgestellten Treiber, genannt piControl. Dieser
aktualisiert das Prozessabbild zyklisch. Die angestrebte Zykluszeit beträgt 5ms,
kann jedoch je nach Anzahl der angeschlossenen Module auch größer sein. Kunbus
garantiert bei drei IO-Modulen und zwei Gateway-Modulen eine Zykluszeit von 10 ms.
Jedes der IO-Module stellt ein eigenständiges eingebettetes System dar. Es verfügt
über einen Microcontroller, welcher die IOs bereitstellt und über einen RS485-Bus
mit dem Revolution Pi kommuniziert.
% https://revolution.kunbus.de/io-modul/

Lizenz: GPL
% https://github.com/RevolutionPi/piControl

\begin{lstlisting}[language={c},firstnumber={226},caption={Setzen der Scheduler-Priorität auf SCHED\_FIFO in revpi\_common.c\label{lst:2-sched_priority}}]
param.sched_priority = ktprio->prio;
ret = sched_setscheduler(child, SCHED_FIFO,
       &param);
\end{lstlisting}


\subsection{Echtzeit und Multithreading unter Linux -- preemptRT und posix%
     \label{sec:2-echtzeit}}


 Der Linux-Kernel verfügt über mehrere unterschiedliche Preemtion-Modelle:

\begin{itemize}
  \item No Forced Preemption (server):
  Ausgelegt auf maximal möglichen Durchsatz, lediglich Interrupts und
  System-Call-Returns bewirken Präemption.

  \item Voluntary Kernel Preemption (Desktop):
  Neben den implizit bevorrechtigten Interrupts und System-Call-Returns gibt es
  in diesem Modell weitere Abschnitte des Kernels in welchen Preämption explizit
  gestattet ist.

  \item Preemptible Kernel (Low-Latency Desktop):
  In diesem Modell ist der gesamte Kernel, mit Ausnahme sog.~kritischer Abschnitte
  präemptible. Nach jedem kritischen Abschnitt gibt es einen impliziten Präemptions-Punkt.

  \item Preemptible Kernel (Basic RT):
  Dieses Modell ist dem zuvor genannten sehr ähnlich, hier sind jedoch alle Interrupt-Handler
  als eigenständige Threads ausgeführt.

  \item Fully Preemptible Kernel (RT):
  Wie auch bei den beiden zuvor genannten Modellen ist hier der gesamte Kernel
  präemtible, die Anzahl und Dauer der nicht-präemtiblen kritischen Abschnitte
  ist auf ein notwendiges Minimum beschränkt. Alle Interrupt-Handler sind als
  eigenständige Threads ausgeführt, Spinlocks durch Sleeping-Spinlocks und Mutexe
  durch sog.~RT-Mutexe ersetzt.

\end{itemize}
\todo{Spinlocks und Mutexe sowie die RT-Varianten dieser erklären!}

Lediglich mit dem vollständig präemtiblen Kernel kann Echtzeit-Verhalten realisiert werden.

% https://wiki.linuxfoundation.org/realtime/documentation/technical_basics/preemption_models bzw kernel/Kconfig.preempt

\subsubsection{preemptRT%
        \label{sec:2-preemptRT}}
% https://wiki.linuxfoundation.org/realtime/documentation/technical_details/start
% https://wiki.linuxfoundation.org/realtime/documentation/technical_basics/start

Das dem PREEMPT RT Kernel zugrunde liegende Prinzip lässt sich in einer einfachen
Regel ausdrücken: Nur Code, welcher absolut nicht-präemtible sein darf, ist es
gestattet nicht-präemtible zu sein.
Das erklärte Ziel des PREEMPT\_RT Patches ist es folglich, die Menge des nicht-präemtiblen
Codes im Linux-Kernel auf das absolut notwendige Minimum zu reduzieren.

Dies wird durch Verwendung folgender Mechanismen erreicht:

\begin{itemize}
  \item Hochauflösende Timer
  \item Sleeping Spinlocks
  \item Threaded Interrupt Handlers
  \item rt\_mutex
  \item RCU
\end{itemize}


\subsubsection{posix%
        \label{sec:2-posix}}
Ist posix hier wirklich relevant? Debian bzw.~Raspbian sind weitgehend posix
kompatibel, aber wird es hier genutzt? -> JA, open62541 nutzt pthread.h
piControl nutzt kthread.h, und semaphore.h

\subsection{OPC-UA und open62541%
     \label{sec:2-opc}}

\subsubsection{OPC UA%
        \label{sec:2-opcua}}
Open Platform Communications (OPC) ist eine Familie von Standards zur herstellerunabhängigen
Kommunikation von Maschinen (M2M) in der Automatisierungstechnik. Die sog.~OPC Task Force, zu deren
Mitgliedern verschiedene große Firmen der Automatisierungsindustrie gehören, veröffentlichte
die OPC Specification Version 1.0 im August 1996.
Motiviert ist dieser offene Standard durch die Erkenntniss, dass die Anpassung der
zahlreichen Herstellerstandards an individuelle Infrastrukturen und Anlagen einen
großen Mehraufwand verursachen.
Die Wikipedia beschreibt das Anwendungsgebiet für OPC wie folgt:

\glqq{}OPC wird dort eingesetzt, wo Sensoren, Regler und Steuerungen verschiedener Hersteller
ein gemeinsames Netzwerk bilden. Ohne OPC benötigten zwei Geräte zum Datenaustausch
genaue Kenntnis über die Kommunikationsmöglichkeiten des Gegenübers. Erweiterungen
und Austausch gestalten sich entsprechend schwierig. Mit OPC genügt es, für jedes
Gerät genau einmal einen OPC-konformen Treiber zu schreiben. Idealerweise wird
dieser bereits vom Hersteller zur Verfügung gestellt. Ein OPC-Treiber lässt sich
ohne großen Anpassungsaufwand in beliebig große Steuer- und Überwachungssysteme
integrieren.

OPC unterteilt sich in verschiedene Unterstandards, die für den jeweiligen Anwendungsfall
unabhängig voneinander implementiert werden können. OPC lässt sich damit verwenden
für Echtzeitdaten (Überwachung), Datenarchivierung, Alarm-Meldungen und neuerdings
auch direkt zur Steuerung (Befehlsübermittlung).\grqq{}

OPC basiert in der ursprünglichen Spezifikation auf Microsofts DCOM-Spezifikation.
DCOM macht Funktionen und Objekte einer Anwendung anderen Anwendungen im Netzwerk
zugänglich. Der OPC-Standard definiert entsprechende DCOM-Objekte um mit anderen
OPC-Anwendungen Daten austauschen zu können. Die Verwendung von DCOM bindet Anwender
an Betriebssysteme von Microsoft. Die ursprüngliche OPC Spezifikation wird durch die
Entwicklung von OPC Unified Architecture (OPC UA) abgelöst.
OPC UA setzt auf einem eigenen Kommunikationionsstack auf, die Verwendung von DCOM
und damit die Bindung an Microsoft wurden aufgelöst.

Die OPC-UA-Architektur ist eine Service-orientierte Architektur (SOA), deren Struktur
aus mehreren Schichten besteht.

% Wikipedia
Das OPC-Informationsmodell ist nicht mehr nur eine Hierarchie aus Ordnern, Items
und Properties. Es ist ein sogenanntes Full-Mesh-Network aus Nodes, mit dem neben
den Nutzdaten eines Nodes auch Meta- und Diagnoseinformationen repräsentiert werden.
Ein Node ähnelt einem Objekt aus der objektorientierten Programmierung. Ein Node
kann Attribute besitzen, die gelesen werden können (Data Access (DA), Historical
Data Access (HDA)). Es ist möglich Methoden zu definieren und aufzurufen.
Eine Methode besitzt Aufrufargumente und Rückgabewerte. Sie wird durch ein Command
aufgerufen. Weiterhin werden Events unterstützt, die versendet werden können
(AE (Alarms \& Events), DA DataChange), um bestimmte Informationen zwischen Geräten
auszutauschen. Ein Event besitzt unter anderem einen Empfangszeitpunkt, eine Nachricht
und einen Schweregrad. Die o. g. Nodes werden sowohl für die Nutzdaten als auch
alle anderen Arten von Metadaten verwendet. Der damit modellierte OPC-Adressraum
beinhaltet nun auch ein Typmodell, mit dem sämtliche Datentypen spezifiziert werden.

% https://de.wikipedia.org/wiki/Open_Platform_Communications
% https://de.wikipedia.org/wiki/OPC_Unified_Architecture
% https://opcfoundation.org/developer-tools/specifications-unified-architecture
% Von Gerhard Gappmeier - ascolab GmbH, CC BY-SA 3.0, https://de.wikipedia.org/w/index.php?curid=1892069
\subsubsection{open62541%
        \label{sec:2-open62541}}
open62541 ist eine offene und freie Implementierung von OPC UA. Die in C geschriebene
Bibliothek stellt eine beständig zunehmende Anzahl der im OPC UA Standard definierten
Funktionen bereit. Sie kann sowohl zur Erstellung von OPC-Servern als auch -Clients
genutzt werden. Ergänzend zu der unter der Mozilla Public License v2.0 lizensierten
Bibliothek stellt das open62541 Projekt auch Beispielprogramme unter einer CC0 Lizenz
zur Verfügung.

Die Bibliothek eignet sich auch für die Entwicklung auf eingebetteten Systemen und
Microcontrollern. Je nach Umfang der gewünschten Funktionen und des OPC Informationsmodells
beträgt die Größe einer Server-Binary weniger als 100kb. %evtl. kürzen?

\todo{Nodes erklären! Evtl.~oben!}

Folgende Auswahl an Eigenschaften und Funktionen zeichnet die in dieser Arbeit verwendete
Version 0.3 von open62541 aus:
\begin{itemize}
  \item Kommunikationionsstack
  \begin{itemize}
      \item OPC UA Binär-Protokoll (HTTP oder SOAP werden gegenwärtig nicht unterstützt)
      \item Austauschbare Netzwerk-Schicht, welche die Verwendung eigener Netzwerk-APIs
      erlaubt.
      \item Verschlüsselte Kommunikationion
      \item Asynchrone Dienst-Anfragen im Client
  \end{itemize}
  \item Informationsmodell
  \begin{itemize}
    \item Unterstützung aller OPC UA Node-Typen, inkl.~Methoden
    \item Hinzufügen und Entfernen von Nodes und Referenzen zur Laufzeit.
    \item Vererbung und Instanziierung von Objekt- und Variablentypen
    \item Zugriffskontrolle auch für einzelne Nodes
  \end{itemize}
  \item Subscriptions
  \begin{itemize}
    \item Erlaubt die Überwachung (subscriptions / monitoreditems)
    \item Sehr geringer Ressourcenbedarf pro überwachtem Wert
  \end{itemize}
  \item Code-Generierung auf XML-Basis
  \begin{itemize}
    \item Erlaubt die Erstellung von Datentypen
    \item Erlaubt die Generierung des serverseitigen Informationsmodells
  \end{itemize}
\end{itemize}

% https://open62541.org/doc/0.3/


Mozilla Public License
CC0 Lizenz für Beispiele und Plugins

% https://open62541.org/doc/open62541-current.pdf
% https://open62541.org/

% % % Imports nur für Referenzenauflösung während des Schreibens! Vorm Kompilieren auskommentieren!
% \bibliography{0_hauptdatei}
% \input{1_einleitung}
% \input{2_grundlagen}
% \input{3_konzeption}
% \input{4_implementierung}
% \input{5_tests}
% \input{6_zusammenfassung}
% \input{anhang}
% % Ende Imports

\section{Systemkonzept%
  \label{sec:3-konzeption}}
Auf Basis der in Abschnitt \ref{sec:2-grundlagen} vorgestellten Möglichkeiten folgt nun die Ausarbeitung eines Konzepts.
In den folgenden Abschnitten soll näher auf zwei zentrale Aspekte eingegangen werden: Abschnitt~\ref{sec:3-anbindung} stellt Möglichkeiten zum Zugriff auf Variablen bzw.\,Werte im Prozessabbild des Revolution Pi vor; in Abschnitt~\ref{sec:3-integration} wird ein Konzept zur Bereitstellung dieser Variablen auf einem OPC-Server vorgestellt.

\subsection{Anbindung der IO an den OPC-Server%
     \label{sec:3-anbindung}}

Eine Webanwendung mit Bezeichnung PiCtory dient zur Konfiguration der I/O- und virtuellen Module des RevolutionPi. Die Konfiguration liegt im JSON-Format in der Datei \lstinline{/etc/revpi/config.rsc}. Der piControl-Treiber liest diese Datei beim Start. 
Der folgende Auszug aus der Manpage des piControl-Kernelmoduls beschreibt die von diesem zum Lesen und Schreiben einzelner Bits des Prozessabbildes bereitgestellten Funktionen~\citep[vgl.]{web-revpi-manpage}. Sie ist an dieser Stelle weitgehend ungekürzt zitiert, da sie die nutzbare Schnittstelle sehr kompakt beschreibt.

\begin{lstlisting}[breakindent=0pt, numbers=none, caption={Auszug aus der Revolution Pi Programmers Manual\label{lst:4-manpage}}]
KB_FIND_VARIABLE SPIVariable *argp
Find a variable in the process image by its name. A pointer to a structure of type SPIVariable must be passed as argument. [...]
The struct SPIVariable [...] is defined as 
typedef struct SPIVariableStr
{
    char strVarName[32]; // Variable name
    uint16_t i16uAddress; // Address of the byte in the process image
    uint8_t i8uBit; // 0-7 bit position, >= 8 whole byte
    uint16_t i16uLength; // length of the variable in bits.
    // Possible values are 1, 8, 16 and 32
} SPIVariable;

Set and get values of the process image
KB_GET_VALUE SPIValue *argp
[...]
KB_SET_VALUE SPIValue *argp
Write one bit or one byte to the process image [...].  This call is more efficient than the usual calls of seek and write because only one function call is necessary. If more than on application are writing bits in one output byte, this call is the only safe way to set a bit without overwriting the other bits because this call is doing a read-modify-write-cycle. 

The struct SPIValue used by this ioctl is defined as
typedef struct SPIValueStr
{
    uint16_t i16uAddress; // Address of the byte in the process image
    uint8_t i8uBit; // 0-7 bit position, >= 8 whole byte
    uint8_t i8uValue; // Value: 0/1 for bit access, whole byte otherwise
} SPIValue;
\end{lstlisting} 

Die oben beschriebenden Funtkionen \lstinline{KB_FIND_VARIABLE}, \lstinline{KB_GET_VALUE} und \lstinline{KB_SET_VALUE} ermöglichen einen einfachen und (lt.\,Manpage) effizienten Zugriff auf einzelne Bits des Prozessabbildes und damit auch auf die IO des RevolutionPi.
Der Zugriff des OPC-Servers auf das Prozessabbild soll daher mittels dieser Funktionen realisiert werden.
\lstinline{KB_FIND_VARIABLE} kann genutzt werden, um Adressen von Variablen im Prozessabbild mittels ihres Namens aufzulösen.
\lstinline{KB_GET_VALUE} und \lstinline{KB_SET_VALUE} ermöglichen den Zugriff auf die Werte dieser Variablen.


\subsection{Integration des OPC-Servers in das System%
     \label{sec:3-integration}}

open62541 bietet drei Möglichkeiten zum Abgleich von Variablen mit dem Prozessabbild~\citep[vgl.][Tutorials - Connecting a Variable with a Physical Process]{web-open62541}:
\begin{itemize}
    \item Manuelles oder zyklisches Aktualisieren
    \item Variable Value Callback
    \item Variable Datasource
\end{itemize}

Die zyklische Aktualisierung eines oder mehrerer Werte nimmt, abhängig von der Zykluszeit, viele Systemressourcen in Anspruch. Value Callbacks ermöglichen es, einen Variablenwert effizienter mit einer Ressource wie etwa einem Prozessabbild zu synchronisieren. An die Variable wird ein Callback angehängt, welches vor jedem Lesen und nach jedem Schreibvorgang ausgeführt wird.
Der Wert der Variablen wird weiterhin im Variablenknoten auf dem OPC-Server gespeichert, der Abgleich mit der verknüpften Ressource erfolgt durch die Callback-Methoden.

Sogenannte Datenquellen gehen noch einen Schritt weiter. Der Server leitet jede Lese- und Schreibanforderung direkt an eine Callback-Funktion weiter. Beim Lesen liefert der Rückruf eine Kopie des aktuellen Wertes. Die Datenquelle muss intern ein eigenes Speichermanagement implementieren.

Der Zugriff auf die Werte des Prozessabbildes erfolgt, wie in Abschnitt~\ref{sec:3-anbindung} beschrieben, über von piControl bereitgestellte Methoden. Um die durch open62541 gepflegte OPC-Datenstruktur und das durch piControl verwaltete Prozessabbild möglichst effektiv verknüpfen zu können, soll diese Interaktion mittels Datenquellen und den zugehörigen Callbacks implementiert werden.
% % % Imports nur für Referenzenauflösung während des Schreibens! Vorm Kompilieren auskommentieren!
% \bibliography{0_hauptdatei}
% \input{1_einleitung}
% \input{2_grundlagen}
% \input{3_konzeption}
% \input{4_implementierung}
% \input{5_tests}
% \input{6_zusammenfassung}
% \input{anhang}
% % Ende Imports

\section{Implementierung%
  \label{sec:4-implementierung}}
Das folgende Kapitel stellt in Auszügen die Implementierung des OPC-Servers sowie die Anbindung an die IO-Module
der SPS dar. Der Schwerpunkt liegt hierbei auf der Funktionsweise des piControl-Treibers und dessen Integration in das Projekt. Abschnitt~\ref{sec:4-picontrol} erklärt die zum Schreibens eines Bits verwendeten Funktionsaufrufe.
Zuvor soll jedoch in Abschnitt~\ref{sec:4-open62541} der Teil des OPC-Servers vorgestellt werden, welcher auf besagten Treiber zugreift. 

\subsection{Implementierung des OPC-Servers%
     \label{sec:4-open62541}}
Wie im vorangegangenen Abschnitt~\ref{sec:3-integration} begründet, soll die Verknüpfung zwischen dem Prozessabbild der SPS und den auf dem OPC-Server bereitgestellten Werten über sog.\,Datenquellen erfolgen. Hierzu ist zunächst eine Callback-Methode zu implementieren, welche bei einem Lese- oder Schreibzugriff auf eine Variable aufgerufen wird. Die Verknüpfung zwischen Callback-Methode und Variable muss manuell erfolgen.

\begin{lstlisting}[language={c},firstnumber=237,caption={Auszug der Methode \lstinline{linkDataSourceVariable} in \lstinline{variables.c}\label{lst:4-linkDataSourceVariable}}]
extern UA_StatusCode
 linkDataSourceVariable(UA_Server *server, UA_NodeId nodeId) {
     bool readonly = false;
     UA_DataSource dataSourceVariable;
     UA_StatusCode rc; |>\setcounter{lstnumber}{254}<|

     dataSourceVariable.read = readDataSourceVariable;
     if (!readonly)
        dataSourceVariable.write = writeDataSourceVariable;
     else
        dataSourceVariable.write = writeReadonlyDataSourceVariable;

     return UA_Server_setVariableNode_dataSource(server, nodeId, dataSourceVariable);
 }
\end{lstlisting}

\begin{figure}[h]
    \centering
    \includegraphics[width=0.42\textwidth]{doc/img/OPC_RevPiDO.pdf}
    \caption{Auszug des verwendeten Nodesets, hier Digitalausgang 1 des Versuchsaufbaus
      \label{fig:opc-do}}
\end{figure}

Die in Listing~\ref{lst:4-linkDataSourceVariable} abgebildete Methode \lstinline{linkDataSourceVariable()} erzeugt ein Struct vom Typ \lstinline{UA_DataSource}. In diesem werden dem Lesen und Schreiben einer OPC-Variablen entsprechende Callback-Methoden zugewiesen. Die Verknüpfung einer OPC-Variable, genauer ihrer NodeId, mit der zuvor definierten Datenquelle erfolgt über die von open62541 bereitgestellte Methode \lstinline{UA_Server_setVariableNode_dataSource()}. Vor dem Lesen und nach dem Schreiben dieser Variable werden von nun an die entsprechenden Callbacks aufgerufen.
     
\begin{lstlisting}[language={c},firstnumber=168,caption={Auszug des Callbacks \lstinline{writeDataSourceVariable} in \lstinline{variables.c}\label{lst:4-writeDataSourceVariable}}]  
extern UA_StatusCode
 writeDataSourceVariable(UA_Server *server,
            const UA_NodeId *sessionId, void *sessionContext,
            const UA_NodeId *nodeId, void *nodeContext,
            const UA_NumericRange *range, const UA_DataValue *dataValue) {

    UA_StatusCode retval  = UA_STATUSCODE_GOOD;
    UA_NodeId *nameNodeId = UA_malloc(sizeof(UA_NodeId));
    UA_QualifiedName nameQN = UA_QUALIFIEDNAME(1, "Name");
    UA_Variant nameVar;
    UA_Boolean bit;

    retval |= findSiblingByBrowsename(server, nodeId, &nameQN, nameNodeId);
    retval |= UA_Server_readValue(server, *nameNodeId, &nameVar);
    retval |= UA_Boolean_copy(dataValue->value.data, &bit);

    |>\tikzmarkin[set border color=martinired]{writeIO}<|PI_writeSingleIO(String_fromUA_String(nameVar.data), &bit, false);                                                 |>\tikzmarkend{writeIO}<|

    free(nameNodeId);
    return retval;
 }
\end{lstlisting}

Listing~\ref{lst:4-writeDataSourceVariable} zeigt die Callback-Methode, welche nach dem Schreiben einer Variablen auf dem OPC-Server aufgerufen wird.
Dieser Methode wird neben der NodeId der mit ihr verknüpften Variablen auch der Wert dieser in Form eines Zeigers auf ein Struct vom Typ \lstinline{UA_DataValue} übergeben.

Die Gestaltung des hier verwendeten Nodesets sieht vor, dass in einer OPC-Variablen \lstinline{"Name"} der Bezeichner des zu schreibenden Digitalausgangs hinterlegt ist, siehe Abbildung~\ref{fig:opc-do}. Dies erlaubt eine Rekonfiguration der Ein- und Ausgänge der SPS ohne Änderungen im Programmcode des OPC-Servers vornehmen zu müssen.
Es ist daher erforderlich, nach jedem Schreiben einer mit einem Digitalausgang verknüpften Variablen, hier \lstinline{"Value"}, dessen Bezeichner \lstinline{"Name"} abzufragen. 
Dies geschieht in den Zeilen 180 und 181.
Anschließend wird dieser Bezeichner sowie der zu schreibende Wert der Methode \lstinline{PI_writeSingleIO()} übergeben, welche wiederum die Interaktion mit piControl übernimmt (vgl. Abschnitt \ref{sec:4-picontrol}).
 
\subsection{Integration von piControl%
     \label{sec:4-picontrol}}
In Abschnitt~\ref{sec:2-io} wurde die Anbindung der IO-Module des Revolution Pi sowie die Funktionsweise von piControl aus Anwendersicht beschrieben. Die verfügbare Literatur beschränkt sich auch auf lediglich diese Sicht; eine weiterführende Dokumentation für Entwickler gibt es, neben der in Abschnitt~\ref{sec:3-anbindung} vorgestellten Manpage, nicht. 
In diesem Abschnitt soll daher der Quellcode von piControl sowie dessen Verwendung im Projekt genauer betrachtet werden.
Hierzu wird exemplarisch die in Abschnitt~\ref{sec:4-open62541} eingeführte Methode \lstinline{PI_writeSingleIO()} untersucht.
Diese Methode ermöglicht das Setzen eines einzelnen Bits im Prozessabbild der SPS, und damit das Schalten eines digitalen Ausgangs auf einem IO-Modul.
Die äquivalente Methode \lstinline{int piControlGetBitValue(SPIValue *pSpiValue)} zum Lesen eines Bits bzw. Eingangs funktioniert analog und soll daher an dieser Stelle nicht dediziert erörtert werden.

\begin{lstlisting}[language={c},firstnumber=97,
                   caption={Setzen eines phsikalischen, digitalen Ausgangs in \lstinline{revpi.c}
                   \label{lst:4-PI_writeSingleIO}}]
extern void PI_writeSingleIO(char *pszVariableName, bool *bit, bool verbose)
{
	int rc;
	SPIVariable sPiVariable;
	SPIValue sPIValue;

	strncpy(sPiVariable.strVarName, pszVariableName, sizeof(sPiVariable.strVarName));
	rc = piControlGetVariableInfo(&sPiVariable);
	if (rc < 0) {
		printf("Cannot find variable '%s'\n", pszVariableName);
		return;
	}

		sPIValue.i16uAddress = sPiVariable.i16uAddress;
		sPIValue.i8uBit = sPiVariable.i8uBit;
		sPIValue.i8uValue = *bit;
		rc = |>\tikzmarkin[set border color=martinired]{setBitValue}<|piControlSetBitValue(&sPIValue)|>\tikzmarkend{setBitValue}<|;
		if (rc < 0)
			printf("Set bit error %s\n", getWriteError(rc));
		else if (verbose)
			printf("Set bit %d on byte at offset %d. Value %d\n", sPIValue.i8uBit, sPIValue.i16uAddress,
			       sPIValue.i8uValue);
}
\end{lstlisting}

Der Programmcode in Listing~\ref{lst:4-PI_writeSingleIO} ist Teil des implementierten OPC-Servers. In diesem wird auf zwei Funktionen des piControl-Treibers zugegriffen. 
Beiden Methoden wird als Argument ein Zeiger auf ein Struct vom Typ \lstinline{SPIValue} übergeben. Der im Struct abgelegte Name wird mittels \lstinline{piControlGetVariableInfo(&sPIValue)} zu einer Adresse im Prozessabbild aufgelöst. Diese wird in \lstinline{sPIValue.i16uAdress} gespeichert. Der Wert der Variablen wird anschließend mittels \lstinline{piControlSetBitValue(&sPIValue)} an dieser Adresse in das Prozessabbild geschrieben.

\begin{lstlisting}[language={c},firstnumber=309,caption={Methode \lstinline{piControlSetBitValue} in \lstinline{piControlIf.c}\label{lst:4-piControlSetBitValue}}]
int |>\tikzmarkin[set border color=martiniblue]{setBitValueFcn}<|piControlSetBitValue(SPIValue *pSpiValue)|>\tikzmarkend{setBitValueFcn}<|
{
    piControlOpen();

    if (PiControlHandle_g < 0)
	    return -ENODEV;

    pSpiValue->i16uAddress += pSpiValue->i8uBit / 8;
    pSpiValue->i8uBit %= 8;

    if (|>\tikzmarkin[set border color=martinired]{ioctl}<|ioctl(PiControlHandle_g, KB_SET_VALUE, pSpiValue)|>\tikzmarkend{ioctl}<| < 0)
	    return errno;

    return 0;
}
\end{lstlisting}

Die in Listing~\ref{lst:4-piControlSetBitValue} dargestellte Methode \lstinline{piControlSetBitValue} ist lediglich eine Hüllfunktion (häufig auch als Wrapper-Funktion bezeichnet) für einen Aufruf des \lstinline{ioctl} Kernel-Moduls.
Folgende Parameter werden übergeben:
\lstinline{PiControlHandle_g} ist die Referenz auf die Geräte-Datei des piControl-Treibers. \lstinline{KB_SET_VALUE} ist das ioctl-Kommando zum Schreiben eines Bits in das Prozessabbild. Der Zeiger \lstinline{pSpiValue} verweist auf ein Struct des bereits vorgestellten Typs \lstinline{SPIValue}.

\begin{lstlisting}[language={c},firstnumber=80,caption={Methode \lstinline{piControlOpen} in \lstinline{piControlIf.c}\label{lst:4-piControlOpen}}]
void piControlOpen(void)
{
    /* open handle if needed */
    if (PiControlHandle_g < 0)
    {
	    |>\tikzmarkin[set border color=martiniblue]{PiControlHandle}<|PiControlHandle_g = open(PICONTROL_DEVICE, O_RDWR)|>\tikzmarkend{PiControlHandle}<|;
    }
}
\end{lstlisting}

Die in Listing~\ref{lst:4-piControlOpen} dargestellte Methode öffnet, sofern nicht bereits geschehen, die Geräte-Datei. Das Macro \lstinline{PICONTROL_DEVICE} verweist hierbei auf \lstinline{/dev/piControl0}.

\begin{lstlisting}[language={c},firstnumber=721,caption={Methode \lstinline{piControlIoctl} in \lstinline{piControlMain.c}\label{lst:4-piControlIoctl}}]
static long |>\tikzmarkin[set border color=martiniblue, below offset=0.9em]{piControlIoctl}<|piControlIoctl(struct file *file, unsigned int prg_nr, 
                           unsigned long usr_addr)                                      |>\tikzmarkend{piControlIoctl}<|
{
  int status = -EFAULT;
  tpiControlInst *priv;
  int timeout = 10000;	// ms

  if (prg_nr != KB_CONFIG_SEND && prg_nr != KB_CONFIG_START && !isRunning()) {
  	return -EAGAIN;
  }

  priv = (tpiControlInst *) file->private_data;

  if (prg_nr != KB_GET_LAST_MESSAGE) {
  	// clear old message
  	priv->pcErrorMessage[0] = 0;
  }

  switch (prg_nr) {|>\setcounter{lstnumber}{864}<|

    case |>\tikzmarkin[set border color=martiniblue]{KB_SET_VALUE}<|KB_SET_VALUE:|>\tikzmarkend{KB_SET_VALUE}<|
  		{
  			SPIValue *pValue = (SPIValue *) usr_addr;

  			if (!isRunning())
  				return -EFAULT;

  			if (pValue->i16uAddress >= KB_PI_LEN) {
  				status = -EFAULT;
  			} else {
  				INT8U i8uValue_l;
  				my_rt_mutex_lock(&piDev_g.lockPI);
  				i8uValue_l = piDev_g.ai8uPI[pValue->i16uAddress];

  				if (pValue->i8uBit >= 8) {
  					i8uValue_l = pValue->i8uValue;
  				} else {
  					if (pValue->i8uValue)
  						i8uValue_l |= (1 << pValue->i8uBit);
  					else
  						i8uValue_l &= ~(1 << pValue->i8uBit);
  				}

  				|>\tikzmarkin[set border color=martinired]{i8uValue}<|piDev_g.ai8uPI[pValue->i16uAddress] = i8uValue_l;|>\tikzmarkend{i8uValue}<|
  				rt_mutex_unlock(&piDev_g.lockPI);

  #ifdef VERBOSE
  				pr_info("piControlIoctl Addr=%u, bit=%u: %02x %02x\n", pValue->i16uAddress, pValue->i8uBit, pValue->i8uValue, i8uValue_l);
  #endif

  				status = 0;
  			}
  		}
  		break; |>\setcounter{lstnumber}{1314}<|

    default:
      pr_err("Invalid Ioctl");
      return (-EINVAL);
      break;

    }

    return status;
  }
\end{lstlisting}

Listing~\ref{lst:4-piControlIoctl} zeigt in Auszügen die ioctl-Methode des piControl Kernel-Treibers. Diese bekommt folgende Argumente übergeben: \lstinline{struct file *file} enthält den Verweis auf die Geräte-Datei, hier \lstinline{/dev/piControl0}. Der Wert von \lstinline{unsigned int prg_nr} beschreibt die Anfrage an den Treiber, in diesem Fall \lstinline{KB_SET_VALUE}. Das Argument \lstinline{unsigned long usr_addr} enthält einen typ-agnostischen Pointer. Dieser verweist auf einen Speicherbereich, in welchem die zur Bearbeitung der Anfrage notwendigen Daten abgelegt sind. Hier können auch vom Treiber empfangene Daten dem Anwendungsprogramm bereitgestellt werden. 

Die switch-case-Anweisung führt die über das Argument \lstinline{prg_nr} spezifizierte Aktion aus. Hier betrachten wir \lstinline{KB_SET_VALUE}:
Zunächst wird in Zeile 868 der übergebene Zeiger \lstinline{usr_addr} mittels explizitem Typecast zu einem Zeiger des Typs \lstinline{SPIValue *} konvertiert. Da dieser auf Daten im Userspace verweist, ist beim Zugriff durch den Kernel-Treiber besondere Vorsicht geboten.
In Zeile 877 wird mittels Mutex das Prozessabbild \lstinline{piDev_g} für den Zugriff durch andere Threads oder Prozesse gesperrt.
\lstinline{my_rt_mutex_lock} verweist hierbei auf die Funktion \lstinline{rt_mutex_lock} aus \lstinline{linux/sched.h}\footnote{Offenbar wurde hier auch eine alternative Implementierung vorgesehen, siehe revpi\_common.h}

In Zeile 889 wird das Byte \lstinline{i8uValue_l}, welches den zu schreibenden Wert enthält in das Prozessabbild übertragen. Anschließend wird die Mutex auf \lstinline{piDev_g} wieder entsperrt.
\newpage

\begin{lstlisting}[language={c},firstnumber=62,caption={Auszug des Struct \lstinline{spiControlDev} in \lstinline{piControlMain.h}\label{lst:4-spiControlDev}}]
|>\tikzmarkin[set border color=martiniblue]{spiControlDev}<|typedef struct spiControlDev|>\tikzmarkend{spiControlDev}<| {
	// device driver stuff
	int init_step;
	enum revpi_machine machine_type;
	void *machine;
	struct cdev cdev;	// Char device structure
	struct device *dev;
	struct thermal_zone_device *thermal_zone;

	|>\tikzmarkin[set border color=martiniblue]{processImage}<|// process image stuff
	INT8U ai8uPI[KB_PI_LEN];
	INT8U ai8uPIDefault|>\tikzmarkin[set border color=martinired]{KB_PI_LEN_0}<|[KB_PI_LEN]|>\tikzmarkend{KB_PI_LEN_0}<|;
	struct rt_mutex lockPI;        |>\tikzmarkend{processImage}<|
	bool stopIO;
	piDevices *devs; |>\setcounter{lstnumber}{94}<|
} tpiControlDev;
\end{lstlisting}

Das Prozessabbild ist als Byte-Array der Länge \lstinline{KB_PI_LEN} in Listing~\ref{lst:4-spiControlDev} definiert. Konfigurationsparameter wie \lstinline{KB_PI_LEN} oder die Zykluszeit für den Datenaustausch zwischen SPS und IO-Modulen sind im folgenden Listing~\ref{lst:4-process} definiert.

\begin{lstlisting}[language={c},firstnumber=119,caption={Konfigurationsparameter des Prozessabbildes in project.h\label{lst:4-process}}]
#define INTERVAL_PI_GATE (5*1000*1000)  // 5 ms piGateCommunication |>\setcounter{lstnumber}{128}<|

#define INTERVAL_IO_COM (5*1000*1000)  // 5 ms piIoComm |>\setcounter{lstnumber}{132}<|

#define KB_PD_LEN       512
|>\tikzmarkin[set border color=martiniblue]{KB_PI_LEN_1}<|#define KB_PI_LEN       4096|>\tikzmarkend{KB_PI_LEN_1}<|
\end{lstlisting}

Das zu setzende Bit wurde zu diesem Zeitpunkt erfolgreich in das Prozessabbild der SPS geschrieben.
Es stellt sich die Frage, wie dieses nun an das IO-Modul kommuniziert wird.
Die Kommunikation mit allen angebundenen Modulen ist ebenfalls Aufgabe des piControl-Treibers.

\begin{lstlisting}[language={c},firstnumber=256,caption={Auszug der Methode \lstinline{piIoThread} in \lstinline{revpi_core.c}\label{lst:4-piIoThread}}]
static int piIoThread(void *data)
{
	//TODO int value = 0;
	ktime_t time;
	ktime_t now;
	s64 tDiff;

	hrtimer_init(&piCore_g.ioTimer, CLOCK_MONOTONIC, HRTIMER_MODE_ABS);
	piCore_g.ioTimer.function = piIoTimer;

	pr_info("piIO thread started\n");

	now = hrtimer_cb_get_time(&piCore_g.ioTimer);

	PiBridgeMaster_Reset();

	while (!kthread_should_stop()) {
		if (|>\tikzmarkin[set border color=martinired]{PiBridgeMaster}<|PiBridgeMaster_Run()|>\tikzmarkend{PiBridgeMaster}<| < 0)
			break;
	}

	RevPiDevice_finish();

	pr_info("piIO exit\n");
	return 0;
}
\end{lstlisting}

Der Kernel-Thread \lstinline{piIoThread} ist verantwortlich für den zyklischen Datenaustausch mit den IO-Modulen. In diesem wird fortlaufend die Methode \lstinline{PiBridgeMaster_Run()} aufgerufen, siehe Listing~\ref{lst:4-piIoThread}.

\begin{lstlisting}[language={c},firstnumber=262,caption={Auszug der Methode \lstinline{PiBridgeMaster_Run(void)} in \lstinline{RevPiDevice.c}\label{lst:4-PiBridgeMaster_Run}}]
int PiBridgeMaster_Run(void)
{
	static kbUT_Timer tTimeoutTimer_s;
	static kbUT_Timer tConfigTimeoutTimer_s;
	static int error_cnt;
	static INT8U last_led;
	static unsigned long last_update;
	int ret = 0;
	int i;

	my_rt_mutex_lock(&piCore_g.lockBridgeState);
	if (piCore_g.eBridgeState != piBridgeStop) {
		switch (eRunStatus_s) { |>\setcounter{lstnumber}{514}<|
		    case enPiBridgeMasterStatus_EndOfConfig:|>\setcounter{lstnumber}{621}<|
		    if (|>\tikzmarkin[set border color=martinired]{RevPiDevice}<|RevPiDevice_run()|>\tikzmarkend{RevPiDevice}<|) {
				// an error occured, check error limits |>\setcounter{lstnumber}{641}<|
			} else {
				ret = 1;
			}
			piCore_g.image.drv.i16uRS485ErrorCnt = RevPiDevice_getErrCnt();
			break;
\end{lstlisting}

Die in Listing~\ref{lst:4-PiBridgeMaster_Run} dargestellte Methode ist eine sog. State-Machine. Ist die Konfiguration der IO-Module erfolgreich abgeschlossen, so führt sie bei Aufruf lediglich die Methode \lstinline{RevPiDevice_run()} aus.

\begin{lstlisting}[language={c},firstnumber=140,caption={Auszug der Methode \lstinline{RevPiDevice_run(void)} in \lstinline{RevPiDevice.c}\label{lst:4-RevPiDevice_run}}]
int RevPiDevice_run(void)
{
	INT8U i8uDevice = 0;
	INT32U r;
	int retval = 0;

	RevPiDevices_s.i16uErrorCnt = 0;

	for (i8uDevice = 0; i8uDevice < RevPiDevice_getDevCnt(); i8uDevice++) {
		if (RevPiDevice_getDev(i8uDevice)->i8uActive) {
			switch (RevPiDevice_getDev(i8uDevice)->sId.i16uModulType) {
			case KUNBUS_FW_DESCR_TYP_PI_DIO_14:
			case KUNBUS_FW_DESCR_TYP_PI_DI_16:
			case KUNBUS_FW_DESCR_TYP_PI_DO_16:
				r = |>\tikzmarkin[set border color=martinired]{sendCyclicTelegram}<|piDIOComm_sendCyclicTelegram(i8uDevice)|>\tikzmarkend{sendCyclicTelegram}\setcounter{lstnumber}{166} <|;

				break; |>\setcounter{lstnumber}{216}<|
			}
		}
	} |>\setcounter{lstnumber}{227}<|
	return retval;
}
\end{lstlisting}

Diese iteriert wie in Listing~\ref{lst:4-RevPiDevice_run} abgebildete durch alle gegenwärtig in der SPS konfigurierten Module. Ist das aktuelle Modul als aktiv markiert, so wird anhand eines sog. Firmware-Descriptors entschieden, welche Methode für die Ansteuerung des Moduls aufzurufen ist.

\begin{lstlisting}[language={c},firstnumber=161,caption={Auszug der Methode \lstinline{piDIOComm_sendCyclicTelegram} in \lstinline{piDIOComm.c}\label{lst:4-sendCyclicTelegram}}]
INT32U piDIOComm_sendCyclicTelegram(INT8U i8uDevice_p)
{
	INT32U i32uRv_l = 0;
	SIOGeneric sRequest_l;
	SIOGeneric sResponse_l;
	INT8U len_l, data_out[18], i, p, data_in[70];
	INT8U i8uAddress;
	int ret; |>\setcounter{lstnumber}{239}<|
	
    |>\tikzmarkin[set border color=martinired]{piIoComm}<|ret = piIoComm_send((INT8U *) & sRequest_l, IOPROTOCOL_HEADER_LENGTH + len_l + 1);  |>\tikzmarkend{piIoComm}\setcounter{lstnumber}{298}<|
}
\end{lstlisting}

Im Falle des hier verwendeten DO-Moduls wird die in Listing~\ref{lst:4-sendCyclicTelegram} abgebildete Methode \lstinline{piDIOComm_sendCyclicTelegram()} aufgerufen. Dieser wird ein Zeiger auf das zu schreibende Gerät übergeben. 
Zunächst wird das Prozessabbild mittels eines proprietären, jedoch im Quellcode offen nachvollziehbaren Protokolls in ein \lstinline{sRequest_l} genanntes Byte-Array umgewandelt. Dieser Schritt ist in Listing~\ref{lst:4-sendCyclicTelegram} nicht abgebildet. Anschließend wird \lstinline{piIoComm_send()} ein Zeiger auf die so generierte Schreib-Anfrage übergeben.

\begin{lstlisting}[language={c},firstnumber=220,caption={Auszug der Methode \lstinline{piIOComm_send} in \lstinline{piIOComm.c}\label{lst:4-piIOComm_send}}]
int piIoComm_send(INT8U * buf_p, INT16U i16uLen_p)
{
	ssize_t write_l = 0;
	INT16U i16uSent_l = 0;|>\setcounter{lstnumber}{249}<|

	while (i16uSent_l < i16uLen_p) {
		write_l = vfs_write(piIoComm_fd_m, buf_p + i16uSent_l, i16uLen_p - i16uSent_l, &piIoComm_fd_m->f_pos);
		if (write_l < 0) {
			pr_info_serial("write error %d\n", (int)write_l);
			return -1;
		} 
		i16uSent_l += write_l;|>\setcounter{lstnumber}{263}<|
	}
	clear();
	vfs_fsync(piIoComm_fd_m, 1);
	return 0;
}
\end{lstlisting}

Listing~\ref{lst:4-piIOComm_send} zeigt die Implementierung von \lstinline{piIoComm_send()}. Diese Methode ist für das Schreiben der oben generierten Anfrage auf die seriellen Schnittstelle verantwortlich. Realisiert wird dies mittels der Methode \lstinline{vfs_write()}. Diese ist in \lstinline{<linux/fs.h>} definiert. Sie ermöglicht das Schreiben einer Datei im Userspace aus dem Kernel heraus. Geschrieben wird hier die Datei mit dem Deskriptor \lstinline{piIoComm_fd_m}.
Da die Funktion \lstinline{vfs_write()} durch andere Kernel-Tasks unterbrochen werden kann, ist nicht gewährleistet, dass die gesamte Anfrage mit nur einem Aufruf geschrieben wird. Die oben abgebildete while-Schleife stellt das vollständige Senden der Anfrage sicher.

\begin{lstlisting}[language={c},firstnumber=157,caption={Auszug der Methode \lstinline{piIOComm_open_serial} in \lstinline{piIOComm.c}\label{lst:4-piIOComm_open_serial}}]
int piIoComm_open_serial(void)
{   |>\setcounter{lstnumber}{167}<|
	struct file *fd;	/* Filedeskriptor */
	struct termios newtio;	/* Schnittstellenoptionen */

	|>\tikzmarkin[set border color=martiniblue]{fd}<|/* Port oeffnen - read/write, kein "controlling tty", 
	    Status von DCD ignorieren */
	fd = filp_open(|>\tikzmarkin[set border color=martinired]{tty}<|REV_PI_TTY_DEVICE|>\tikzmarkend{tty}<|, O_RDWR | O_NOCTTY, 0); |>\setcounter{lstnumber}{208}<|
	
	piIoComm_fd_m = fd;                                                      |>\tikzmarkend{fd}\setcounter{lstnumber}{217}<|

	return 0;
}
\end{lstlisting}

Der zum Schreiben auf die serielle Schnittstelle verwendete Datei-Deskriptor wird von der in Listing~\ref{lst:4-piIOComm_open_serial} abgebildeten Methode \lstinline{piIoComm_open_serial()} generiert. 

\begin{lstlisting}[language={c},firstnumber=45,caption={Definition der seriellen Schnittstelle in \lstinline{piIOComm.h}\label{lst:4-REV_PI_TTY_DEVICE}}]
#define REV_PI_TTY_DEVICE	"/dev/ttyAMA0"
\end{lstlisting}

Das in Listing~\ref{lst:4-REV_PI_TTY_DEVICE} definierte Macro verweist auf eine der seriellen Schnittstellen des RaspberryPi.
Die Implementierung des zugehörigen Schnittstellentreibers soll hier nicht weiter untersucht werden. Somit ist an dieser Stelle die Kette vom Setzen einer Variablen auf dem OPC-Server bis hin zur Aktualisierung des Prozessabbilds der IO-Module geschlossen.

% \begin{lstlisting}[language={c},firstnumber={226},caption={Setzen der Scheduler-Priorität auf SCHED\_FIFO in 
% revpi\_common.c\label{lst:2-sched_priority}}]
% param.sched_priority = ktprio->prio;
% ret = sched_setscheduler(child, SCHED_FIFO, &param);
% \end{lstlisting}
% % % Imports nur für Referenzenauflösung während des Schreibens! Vorm Kompilieren auskommentieren!
% \bibliography{0_hauptdatei}
% \input{1_einleitung}
% \input{2_grundlagen}
% \input{3_konzeption}
% \input{4_implementierung}
% \input{5_tests}
% \input{6_zusammenfassung}
% % Ende Imports

\section{Test des OPC-Servers im Gesamtsystem%
  \label{sec:5-tests}}

% % % Imports nur für Referenzenauflösung während des schreibens! Vorm Kompilieren auskommentieren!
% \bibliography{0_hauptdatei}
% \input{1_einleitung}
% \input{2_grundlagen}
% \input{3_konzeption}
% \input{4_implementierung}
% \input{5_tests}
% \input{6_zusammenfassung}
% % Ende Imports

\section{Zusammenfassung und Ausblick%
  \label{sec:6-fazit}}
Der folgende Abschnitt~\ref{sec:6-zusammenfassung} fasst die gewonnenen Erkenntnisse und den Stand der Implementierung zusammen.
Den Abschluss dieser Arbeit bildet der Ausblick in Abschnitt~\ref{sec:6-ausblick}.

\subsection{Zusammenfassung%
     \label{sec:6-zusammenfassung}}

\subsection{Ausblick%
     \label{sec:6-ausblick}}

% % Ende Imports

\section{Zusammenfassung und Ausblick%
  \label{sec:6-fazit}}
Der folgende Abschnitt~\ref{sec:6-zusammenfassung} fasst die gewonnenen Erkenntnisse und den Stand der Implementierung zusammen.
Den Abschluss dieser Arbeit bildet der Ausblick in Abschnitt~\ref{sec:6-ausblick}.

\subsection{Zusammenfassung%
     \label{sec:6-zusammenfassung}}

\subsection{Ausblick%
     \label{sec:6-ausblick}}

% % Ende Imports

\section{Test des OPC-Servers im Gesamtsystem%
  \label{sec:5-tests}}

% % % Imports nur für Referenzenauflösung während des schreibens! Vorm Kompilieren auskommentieren!
% \bibliography{0_hauptdatei}
% % Mit \section{...} eröffnen wir einen neuen Abschnitt.
% Der Befehl setzt nicht nur den Text in einer größeren,
% fetten Schrift, sondern sorgt außerdem dafür, daß er im
% Inhaltsverzeichnis erscheint.
%
% Mit \label{...} erzeugen wir einen Bezeichner, mit dessen Hilfe
% wir später auf die Nummer des Abschnitts verweisen können (nämlich
% mit~\ref{...}).
%
% Das Kommentarzeichen hinter „Übersicht“ dient dazu, ein
% Leerzeichen zwischen „Übersicht“ und dem \label-Befehl
% zu vermeiden, das andernfalls sichtbar würde – z.B. im
% Inhaltsverzeichnis.
%

% % Imports nur für Referenzenauflösung während des Schreibens! Vorm Kompilieren auskommentieren!
% \bibliography{0_hauptdatei}
% % Mit \section{...} eröffnen wir einen neuen Abschnitt.
% Der Befehl setzt nicht nur den Text in einer größeren,
% fetten Schrift, sondern sorgt außerdem dafür, daß er im
% Inhaltsverzeichnis erscheint.
%
% Mit \label{...} erzeugen wir einen Bezeichner, mit dessen Hilfe
% wir später auf die Nummer des Abschnitts verweisen können (nämlich
% mit~\ref{...}).
%
% Das Kommentarzeichen hinter „Übersicht“ dient dazu, ein
% Leerzeichen zwischen „Übersicht“ und dem \label-Befehl
% zu vermeiden, das andernfalls sichtbar würde – z.B. im
% Inhaltsverzeichnis.
%

% % Imports nur für Referenzenauflösung während des Schreibens! Vorm Kompilieren auskommentieren!
% \bibliography{0_hauptdatei}
% \input{1_einleitung}
%\input{2_grundlagen}
%\input{3_konzeption}
%\input{4_implementierung}
%\input{5_tests}
%\input{6_zusammenfassung}
% % Ende Imports

\section{Einleitung und Motivation%
  \label{sec:1-einleitung}}
Ziel dieses Projektes ist die Integration eines OPC-Servers mit einer auf Linux
basierenden speicherprogrammierbaren Steuerung (SPS). Angeschlossen an diese SPS
ist jeweils ein digitales Ein-/\,bzw.~Ausgabemodul. Die von diesen bereitgestellten
Ein-/\, bzw.~Ausgänge (IO) sollen in der Datenstruktur des OPC-Servers abgebildet
und über diesen für OPC-Clients les-/\,und schreibar sein. Weiterhin sollen einige
Funktionen zur Überwachung und Steuerung der an die SPS angeschlossenen Aktoren
und Sensoren direkt im OPC-Server implementiert werden.
Hiermit stellt dieses Projekt eine der Grundlagen für ein übergeordnetes Projekt,
die cloudbasierte Steuerung eines miniaturisierten Produktions-Systems, dar.

Der hier verwendete OPC-Server ist Teil des sog. open62541 Projekts. Er ist in C
geschrieben und implementiert bereits einen großen Teil der im OPC-UA-Standard
spezifizierten Funktionen.
Als SPS findet ein Revolution Pi 3 der Firma Kunbus Verwendung. Dieser integriert
ein sog. Compute Module der Raspberry Pi Foundation in ein industrietaugliches
Gehäuse und erlaubt die Erweiterung mittels IO- oder Gateway-Modulen. Über diese
erfolgt die Kommunikation mit weiteren Komponenten der Automatisierungstechnik.

Motiviert ist dieses Projekt durch die Beobachtung, dass die Verbreitung offener
Standards sowie freier Software auch in der Automatisierungstechnik zunimmt.
Linux ist ein freies Betriebssystem, OPC-UA ein offen zugänglicher, aktiv gepflegter
und weit verbreiteter Standard. Der Raspberry Pi findet sowohl bei Hobby-Anwendern als
auch in den Bereichen Forschung und Entwicklung sowie bei industriellen Anwendern
Verwendung. Dieses Projekt stellt somit eine für unterschiedliche Anwender interessante
Entwicklung dar.

Im Anschluss an diese einleitende Übersicht im Abschnitt~\ref{sec:1-einleitung} folgt
die Darstellung der wichtigsten Grundlagen in Abschnitt~\ref{sec:2-grundlagen}.
Aufbauend auf diesen Grundlagen folgt die konzeptuelle Ausarbeitung im Abschnitt~\ref{sec:3-konzeption}.
Die Umsetzung wird im Abschnitt~\ref{sec:4-implementierung} erläutert.
Die Leistungsfähigkeit der Implementierung wird in Abschnitt~\ref{sec:5-tests} untersucht.
Eine Zusammenfassung und ein Ausblick schließen die Arbeit in
Abschnitt~\ref{sec:6-fazit} ab. Eventuell noch benötigte Anhänge
finden sich in den Anhängen [...] bis [...].

%% % Imports nur für Referenzenauflösung während des Schreibens! Vorm Kompilieren auskommentieren!
% \bibliography{0_hauptdatei}
% \input{1_einleitung}
% \input{2_grundlagen}
% \input{3_konzeption}
% \input{4_implementierung}
% \input{5_tests}
% \input{6_zusammenfassung}
% % Ende Imports

\section{Grundlagen%
  \label{sec:2-grundlagen}}

\subsection{Speicherprogrammierbare-Steuerung und Linux -- Revolution Pi%
     \label{sec:2-sps}}

\subsubsection{Kunbus RevolutionPi%
        \label{sec:2-revpi}}
Der RevolutionPi 3 ist eine speicherprogrammierbare Steuerung (SPS) des Herstellers
Kunbus GmbH. Kern dieser SPS ist das von der Raspberry Pi Foundation entwickelte
und vertriebene Raspberry Pi Compute Module 3. Dieses integriert ein Broadcom BCM2837
System-on-Chip (SoC) mit vier 1,2GHz Prozessorkernen, 1GB RAM, 4GB eMMC Anwendungsspeicher
und sonstige Peripherie in ein Modul im DDR2-SODIMM Formfaktor. Diese Spezifikationen
sind weitgehend identisch zu denen des ausgesprochen populären Raspberry Pi 3.
Der Revolution Pi profitiert daher von dem gleichen großen Angebot an Software
und Unterstützung wie der Raspberry Pi, ergänzt dessen Hardware jedoch um eine 24V
Spannungsversorgung, die Möglichkeit der Erweiterung durch mehrere industrietaugliche
Ein-/ Ausgabemodule und Gateways sowie ein Gehäuse zur Montage auf einer DIN-Schiene.
\begin{itemize}
  \item{Prozessor: BCM2837}
  \item{Taktfrequenz 1,2 GHz}
  \item{Anzahl Prozessorkerne: 4}
  \item{Arbeitsspeicher: 1 GByte}
  \item{eMMC Flash Speicher: 4 GByte}
  \item{Betriebssystem: Angepasstes Raspbian mit RT-Patch}
  \item{RTC mit 24h Pufferung über wartungsfreien Kondensator}
  \item{Treiber / API: Treiber schreibt zyklisch Prozessdaten in ein Prozessabbild, Zugriff auf Prozessabbild über Linux-Filesystem als API zu Fremdsoftware.}
  \item{Kommunikationsanschlüsse: 2 x USB 2.0 A (je 500 mA belastbar), 1 x Micro-USB, HDMI, Ethernet (RJ45) 10/100 Mbit/s}
  \item{Stromversorgung: min. 10,7 V, max. 28,8 V, maximal 10 Watt}
  \item{Zulässige Umgebungstemperatur: -40 bis +55 C}
  \item{Gehäuseabmessungen: (HxBxL) 96 mm x 22,5 mm x 110,5 mm (ohne gesteckte Stecker)}
  \item{ESD Schutz: 4 kV / 8 kV gemäß EN61131-2 und IEC 61000-6-2}
  \item{Surge / Burst Prüfungen: gemäß EN61131-2 und IEC 61000-6-2 eingekoppelt auf Versorgungsspannung, Ethernet und IO-Leitungen}
  \item{EMI Prüfungen: gemäß EN61131-2 und IEC 61000-6-2}
\end{itemize}

Kunbus bietet eine Auswahl an IO- und Gateway-Modulen zur Erweiterung des Revolution Pi an.
Gateways dienen der Kommunikation mit Systemen oder Komponenten der Automatisierungstechnik
über Protokolle wie PROFIBUS oder EtherCAT. IO-Module erlauben die Überwachung
und Steuerung von digitalen oder analogen Ein- und Ausgängen.

\subsubsection{Zugriff auf IO-Module%
        \label{sec:2-io}}
Der Zugriff auf die Ein- und Ausgänge der IO-Module erfolgt über ein Prozessabbild
und einen hierfür von Kunbus bereitgestellten Treiber, genannt piControl. Dieser
aktualisiert das Prozessabbild zyklisch. Die angestrebte Zykluszeit beträgt 5ms,
kann jedoch je nach Anzahl der angeschlossenen Module auch größer sein. Kunbus
garantiert bei drei IO-Modulen und zwei Gateway-Modulen eine Zykluszeit von 10 ms.
Jedes der IO-Module stellt ein eigenständiges eingebettetes System dar. Es verfügt
über einen Microcontroller, welcher die IOs bereitstellt und über einen RS485-Bus
mit dem Revolution Pi kommuniziert.
% https://revolution.kunbus.de/io-modul/

Lizenz: GPL
% https://github.com/RevolutionPi/piControl

\begin{lstlisting}[language={c},firstnumber={226},caption={Setzen der Scheduler-Priorität auf SCHED\_FIFO in revpi\_common.c\label{lst:2-sched_priority}}]
param.sched_priority = ktprio->prio;
ret = sched_setscheduler(child, SCHED_FIFO,
       &param);
\end{lstlisting}


\subsection{Echtzeit und Multithreading unter Linux -- preemptRT und posix%
     \label{sec:2-echtzeit}}


 Der Linux-Kernel verfügt über mehrere unterschiedliche Preemtion-Modelle:

\begin{itemize}
  \item No Forced Preemption (server):
  Ausgelegt auf maximal möglichen Durchsatz, lediglich Interrupts und
  System-Call-Returns bewirken Präemption.

  \item Voluntary Kernel Preemption (Desktop):
  Neben den implizit bevorrechtigten Interrupts und System-Call-Returns gibt es
  in diesem Modell weitere Abschnitte des Kernels in welchen Preämption explizit
  gestattet ist.

  \item Preemptible Kernel (Low-Latency Desktop):
  In diesem Modell ist der gesamte Kernel, mit Ausnahme sog.~kritischer Abschnitte
  präemptible. Nach jedem kritischen Abschnitt gibt es einen impliziten Präemptions-Punkt.

  \item Preemptible Kernel (Basic RT):
  Dieses Modell ist dem zuvor genannten sehr ähnlich, hier sind jedoch alle Interrupt-Handler
  als eigenständige Threads ausgeführt.

  \item Fully Preemptible Kernel (RT):
  Wie auch bei den beiden zuvor genannten Modellen ist hier der gesamte Kernel
  präemtible, die Anzahl und Dauer der nicht-präemtiblen kritischen Abschnitte
  ist auf ein notwendiges Minimum beschränkt. Alle Interrupt-Handler sind als
  eigenständige Threads ausgeführt, Spinlocks durch Sleeping-Spinlocks und Mutexe
  durch sog.~RT-Mutexe ersetzt.

\end{itemize}
\todo{Spinlocks und Mutexe sowie die RT-Varianten dieser erklären!}

Lediglich mit dem vollständig präemtiblen Kernel kann Echtzeit-Verhalten realisiert werden.

% https://wiki.linuxfoundation.org/realtime/documentation/technical_basics/preemption_models bzw kernel/Kconfig.preempt

\subsubsection{preemptRT%
        \label{sec:2-preemptRT}}
% https://wiki.linuxfoundation.org/realtime/documentation/technical_details/start
% https://wiki.linuxfoundation.org/realtime/documentation/technical_basics/start

Das dem PREEMPT RT Kernel zugrunde liegende Prinzip lässt sich in einer einfachen
Regel ausdrücken: Nur Code, welcher absolut nicht-präemtible sein darf, ist es
gestattet nicht-präemtible zu sein.
Das erklärte Ziel des PREEMPT\_RT Patches ist es folglich, die Menge des nicht-präemtiblen
Codes im Linux-Kernel auf das absolut notwendige Minimum zu reduzieren.

Dies wird durch Verwendung folgender Mechanismen erreicht:

\begin{itemize}
  \item Hochauflösende Timer
  \item Sleeping Spinlocks
  \item Threaded Interrupt Handlers
  \item rt\_mutex
  \item RCU
\end{itemize}


\subsubsection{posix%
        \label{sec:2-posix}}
Ist posix hier wirklich relevant? Debian bzw.~Raspbian sind weitgehend posix
kompatibel, aber wird es hier genutzt? -> JA, open62541 nutzt pthread.h
piControl nutzt kthread.h, und semaphore.h

\subsection{OPC-UA und open62541%
     \label{sec:2-opc}}

\subsubsection{OPC UA%
        \label{sec:2-opcua}}
Open Platform Communications (OPC) ist eine Familie von Standards zur herstellerunabhängigen
Kommunikation von Maschinen (M2M) in der Automatisierungstechnik. Die sog.~OPC Task Force, zu deren
Mitgliedern verschiedene große Firmen der Automatisierungsindustrie gehören, veröffentlichte
die OPC Specification Version 1.0 im August 1996.
Motiviert ist dieser offene Standard durch die Erkenntniss, dass die Anpassung der
zahlreichen Herstellerstandards an individuelle Infrastrukturen und Anlagen einen
großen Mehraufwand verursachen.
Die Wikipedia beschreibt das Anwendungsgebiet für OPC wie folgt:

\glqq{}OPC wird dort eingesetzt, wo Sensoren, Regler und Steuerungen verschiedener Hersteller
ein gemeinsames Netzwerk bilden. Ohne OPC benötigten zwei Geräte zum Datenaustausch
genaue Kenntnis über die Kommunikationsmöglichkeiten des Gegenübers. Erweiterungen
und Austausch gestalten sich entsprechend schwierig. Mit OPC genügt es, für jedes
Gerät genau einmal einen OPC-konformen Treiber zu schreiben. Idealerweise wird
dieser bereits vom Hersteller zur Verfügung gestellt. Ein OPC-Treiber lässt sich
ohne großen Anpassungsaufwand in beliebig große Steuer- und Überwachungssysteme
integrieren.

OPC unterteilt sich in verschiedene Unterstandards, die für den jeweiligen Anwendungsfall
unabhängig voneinander implementiert werden können. OPC lässt sich damit verwenden
für Echtzeitdaten (Überwachung), Datenarchivierung, Alarm-Meldungen und neuerdings
auch direkt zur Steuerung (Befehlsübermittlung).\grqq{}

OPC basiert in der ursprünglichen Spezifikation auf Microsofts DCOM-Spezifikation.
DCOM macht Funktionen und Objekte einer Anwendung anderen Anwendungen im Netzwerk
zugänglich. Der OPC-Standard definiert entsprechende DCOM-Objekte um mit anderen
OPC-Anwendungen Daten austauschen zu können. Die Verwendung von DCOM bindet Anwender
an Betriebssysteme von Microsoft. Die ursprüngliche OPC Spezifikation wird durch die
Entwicklung von OPC Unified Architecture (OPC UA) abgelöst.
OPC UA setzt auf einem eigenen Kommunikationionsstack auf, die Verwendung von DCOM
und damit die Bindung an Microsoft wurden aufgelöst.

Die OPC-UA-Architektur ist eine Service-orientierte Architektur (SOA), deren Struktur
aus mehreren Schichten besteht.

% Wikipedia
Das OPC-Informationsmodell ist nicht mehr nur eine Hierarchie aus Ordnern, Items
und Properties. Es ist ein sogenanntes Full-Mesh-Network aus Nodes, mit dem neben
den Nutzdaten eines Nodes auch Meta- und Diagnoseinformationen repräsentiert werden.
Ein Node ähnelt einem Objekt aus der objektorientierten Programmierung. Ein Node
kann Attribute besitzen, die gelesen werden können (Data Access (DA), Historical
Data Access (HDA)). Es ist möglich Methoden zu definieren und aufzurufen.
Eine Methode besitzt Aufrufargumente und Rückgabewerte. Sie wird durch ein Command
aufgerufen. Weiterhin werden Events unterstützt, die versendet werden können
(AE (Alarms \& Events), DA DataChange), um bestimmte Informationen zwischen Geräten
auszutauschen. Ein Event besitzt unter anderem einen Empfangszeitpunkt, eine Nachricht
und einen Schweregrad. Die o. g. Nodes werden sowohl für die Nutzdaten als auch
alle anderen Arten von Metadaten verwendet. Der damit modellierte OPC-Adressraum
beinhaltet nun auch ein Typmodell, mit dem sämtliche Datentypen spezifiziert werden.

% https://de.wikipedia.org/wiki/Open_Platform_Communications
% https://de.wikipedia.org/wiki/OPC_Unified_Architecture
% https://opcfoundation.org/developer-tools/specifications-unified-architecture
% Von Gerhard Gappmeier - ascolab GmbH, CC BY-SA 3.0, https://de.wikipedia.org/w/index.php?curid=1892069
\subsubsection{open62541%
        \label{sec:2-open62541}}
open62541 ist eine offene und freie Implementierung von OPC UA. Die in C geschriebene
Bibliothek stellt eine beständig zunehmende Anzahl der im OPC UA Standard definierten
Funktionen bereit. Sie kann sowohl zur Erstellung von OPC-Servern als auch -Clients
genutzt werden. Ergänzend zu der unter der Mozilla Public License v2.0 lizensierten
Bibliothek stellt das open62541 Projekt auch Beispielprogramme unter einer CC0 Lizenz
zur Verfügung.

Die Bibliothek eignet sich auch für die Entwicklung auf eingebetteten Systemen und
Microcontrollern. Je nach Umfang der gewünschten Funktionen und des OPC Informationsmodells
beträgt die Größe einer Server-Binary weniger als 100kb. %evtl. kürzen?

\todo{Nodes erklären! Evtl.~oben!}

Folgende Auswahl an Eigenschaften und Funktionen zeichnet die in dieser Arbeit verwendete
Version 0.3 von open62541 aus:
\begin{itemize}
  \item Kommunikationionsstack
  \begin{itemize}
      \item OPC UA Binär-Protokoll (HTTP oder SOAP werden gegenwärtig nicht unterstützt)
      \item Austauschbare Netzwerk-Schicht, welche die Verwendung eigener Netzwerk-APIs
      erlaubt.
      \item Verschlüsselte Kommunikationion
      \item Asynchrone Dienst-Anfragen im Client
  \end{itemize}
  \item Informationsmodell
  \begin{itemize}
    \item Unterstützung aller OPC UA Node-Typen, inkl.~Methoden
    \item Hinzufügen und Entfernen von Nodes und Referenzen zur Laufzeit.
    \item Vererbung und Instanziierung von Objekt- und Variablentypen
    \item Zugriffskontrolle auch für einzelne Nodes
  \end{itemize}
  \item Subscriptions
  \begin{itemize}
    \item Erlaubt die Überwachung (subscriptions / monitoreditems)
    \item Sehr geringer Ressourcenbedarf pro überwachtem Wert
  \end{itemize}
  \item Code-Generierung auf XML-Basis
  \begin{itemize}
    \item Erlaubt die Erstellung von Datentypen
    \item Erlaubt die Generierung des serverseitigen Informationsmodells
  \end{itemize}
\end{itemize}

% https://open62541.org/doc/0.3/


Mozilla Public License
CC0 Lizenz für Beispiele und Plugins

% https://open62541.org/doc/open62541-current.pdf
% https://open62541.org/

%% % Imports nur für Referenzenauflösung während des Schreibens! Vorm Kompilieren auskommentieren!
% \bibliography{0_hauptdatei}
% \input{1_einleitung}
% \input{2_grundlagen}
% \input{3_konzeption}
% \input{4_implementierung}
% \input{5_tests}
% \input{6_zusammenfassung}
% \input{anhang}
% % Ende Imports

\section{Systemkonzept%
  \label{sec:3-konzeption}}
Auf Basis der in Abschnitt \ref{sec:2-grundlagen} vorgestellten Möglichkeiten folgt nun die Ausarbeitung eines Konzepts.
In den folgenden Abschnitten soll näher auf zwei zentrale Aspekte eingegangen werden: Abschnitt~\ref{sec:3-anbindung} stellt Möglichkeiten zum Zugriff auf Variablen bzw.\,Werte im Prozessabbild des Revolution Pi vor; in Abschnitt~\ref{sec:3-integration} wird ein Konzept zur Bereitstellung dieser Variablen auf einem OPC-Server vorgestellt.

\subsection{Anbindung der IO an den OPC-Server%
     \label{sec:3-anbindung}}

Eine Webanwendung mit Bezeichnung PiCtory dient zur Konfiguration der I/O- und virtuellen Module des RevolutionPi. Die Konfiguration liegt im JSON-Format in der Datei \lstinline{/etc/revpi/config.rsc}. Der piControl-Treiber liest diese Datei beim Start. 
Der folgende Auszug aus der Manpage des piControl-Kernelmoduls beschreibt die von diesem zum Lesen und Schreiben einzelner Bits des Prozessabbildes bereitgestellten Funktionen~\citep[vgl.]{web-revpi-manpage}. Sie ist an dieser Stelle weitgehend ungekürzt zitiert, da sie die nutzbare Schnittstelle sehr kompakt beschreibt.

\begin{lstlisting}[breakindent=0pt, numbers=none, caption={Auszug aus der Revolution Pi Programmers Manual\label{lst:4-manpage}}]
KB_FIND_VARIABLE SPIVariable *argp
Find a variable in the process image by its name. A pointer to a structure of type SPIVariable must be passed as argument. [...]
The struct SPIVariable [...] is defined as 
typedef struct SPIVariableStr
{
    char strVarName[32]; // Variable name
    uint16_t i16uAddress; // Address of the byte in the process image
    uint8_t i8uBit; // 0-7 bit position, >= 8 whole byte
    uint16_t i16uLength; // length of the variable in bits.
    // Possible values are 1, 8, 16 and 32
} SPIVariable;

Set and get values of the process image
KB_GET_VALUE SPIValue *argp
[...]
KB_SET_VALUE SPIValue *argp
Write one bit or one byte to the process image [...].  This call is more efficient than the usual calls of seek and write because only one function call is necessary. If more than on application are writing bits in one output byte, this call is the only safe way to set a bit without overwriting the other bits because this call is doing a read-modify-write-cycle. 

The struct SPIValue used by this ioctl is defined as
typedef struct SPIValueStr
{
    uint16_t i16uAddress; // Address of the byte in the process image
    uint8_t i8uBit; // 0-7 bit position, >= 8 whole byte
    uint8_t i8uValue; // Value: 0/1 for bit access, whole byte otherwise
} SPIValue;
\end{lstlisting} 

Die oben beschriebenden Funtkionen \lstinline{KB_FIND_VARIABLE}, \lstinline{KB_GET_VALUE} und \lstinline{KB_SET_VALUE} ermöglichen einen einfachen und (lt.\,Manpage) effizienten Zugriff auf einzelne Bits des Prozessabbildes und damit auch auf die IO des RevolutionPi.
Der Zugriff des OPC-Servers auf das Prozessabbild soll daher mittels dieser Funktionen realisiert werden.
\lstinline{KB_FIND_VARIABLE} kann genutzt werden, um Adressen von Variablen im Prozessabbild mittels ihres Namens aufzulösen.
\lstinline{KB_GET_VALUE} und \lstinline{KB_SET_VALUE} ermöglichen den Zugriff auf die Werte dieser Variablen.


\subsection{Integration des OPC-Servers in das System%
     \label{sec:3-integration}}

open62541 bietet drei Möglichkeiten zum Abgleich von Variablen mit dem Prozessabbild~\citep[vgl.][Tutorials - Connecting a Variable with a Physical Process]{web-open62541}:
\begin{itemize}
    \item Manuelles oder zyklisches Aktualisieren
    \item Variable Value Callback
    \item Variable Datasource
\end{itemize}

Die zyklische Aktualisierung eines oder mehrerer Werte nimmt, abhängig von der Zykluszeit, viele Systemressourcen in Anspruch. Value Callbacks ermöglichen es, einen Variablenwert effizienter mit einer Ressource wie etwa einem Prozessabbild zu synchronisieren. An die Variable wird ein Callback angehängt, welches vor jedem Lesen und nach jedem Schreibvorgang ausgeführt wird.
Der Wert der Variablen wird weiterhin im Variablenknoten auf dem OPC-Server gespeichert, der Abgleich mit der verknüpften Ressource erfolgt durch die Callback-Methoden.

Sogenannte Datenquellen gehen noch einen Schritt weiter. Der Server leitet jede Lese- und Schreibanforderung direkt an eine Callback-Funktion weiter. Beim Lesen liefert der Rückruf eine Kopie des aktuellen Wertes. Die Datenquelle muss intern ein eigenes Speichermanagement implementieren.

Der Zugriff auf die Werte des Prozessabbildes erfolgt, wie in Abschnitt~\ref{sec:3-anbindung} beschrieben, über von piControl bereitgestellte Methoden. Um die durch open62541 gepflegte OPC-Datenstruktur und das durch piControl verwaltete Prozessabbild möglichst effektiv verknüpfen zu können, soll diese Interaktion mittels Datenquellen und den zugehörigen Callbacks implementiert werden.
%% % Imports nur für Referenzenauflösung während des Schreibens! Vorm Kompilieren auskommentieren!
% \bibliography{0_hauptdatei}
% \input{1_einleitung}
% \input{2_grundlagen}
% \input{3_konzeption}
% \input{4_implementierung}
% \input{5_tests}
% \input{6_zusammenfassung}
% \input{anhang}
% % Ende Imports

\section{Implementierung%
  \label{sec:4-implementierung}}
Das folgende Kapitel stellt in Auszügen die Implementierung des OPC-Servers sowie die Anbindung an die IO-Module
der SPS dar. Der Schwerpunkt liegt hierbei auf der Funktionsweise des piControl-Treibers und dessen Integration in das Projekt. Abschnitt~\ref{sec:4-picontrol} erklärt die zum Schreibens eines Bits verwendeten Funktionsaufrufe.
Zuvor soll jedoch in Abschnitt~\ref{sec:4-open62541} der Teil des OPC-Servers vorgestellt werden, welcher auf besagten Treiber zugreift. 

\subsection{Implementierung des OPC-Servers%
     \label{sec:4-open62541}}
Wie im vorangegangenen Abschnitt~\ref{sec:3-integration} begründet, soll die Verknüpfung zwischen dem Prozessabbild der SPS und den auf dem OPC-Server bereitgestellten Werten über sog.\,Datenquellen erfolgen. Hierzu ist zunächst eine Callback-Methode zu implementieren, welche bei einem Lese- oder Schreibzugriff auf eine Variable aufgerufen wird. Die Verknüpfung zwischen Callback-Methode und Variable muss manuell erfolgen.

\begin{lstlisting}[language={c},firstnumber=237,caption={Auszug der Methode \lstinline{linkDataSourceVariable} in \lstinline{variables.c}\label{lst:4-linkDataSourceVariable}}]
extern UA_StatusCode
 linkDataSourceVariable(UA_Server *server, UA_NodeId nodeId) {
     bool readonly = false;
     UA_DataSource dataSourceVariable;
     UA_StatusCode rc; |>\setcounter{lstnumber}{254}<|

     dataSourceVariable.read = readDataSourceVariable;
     if (!readonly)
        dataSourceVariable.write = writeDataSourceVariable;
     else
        dataSourceVariable.write = writeReadonlyDataSourceVariable;

     return UA_Server_setVariableNode_dataSource(server, nodeId, dataSourceVariable);
 }
\end{lstlisting}

\begin{figure}[h]
    \centering
    \includegraphics[width=0.42\textwidth]{doc/img/OPC_RevPiDO.pdf}
    \caption{Auszug des verwendeten Nodesets, hier Digitalausgang 1 des Versuchsaufbaus
      \label{fig:opc-do}}
\end{figure}

Die in Listing~\ref{lst:4-linkDataSourceVariable} abgebildete Methode \lstinline{linkDataSourceVariable()} erzeugt ein Struct vom Typ \lstinline{UA_DataSource}. In diesem werden dem Lesen und Schreiben einer OPC-Variablen entsprechende Callback-Methoden zugewiesen. Die Verknüpfung einer OPC-Variable, genauer ihrer NodeId, mit der zuvor definierten Datenquelle erfolgt über die von open62541 bereitgestellte Methode \lstinline{UA_Server_setVariableNode_dataSource()}. Vor dem Lesen und nach dem Schreiben dieser Variable werden von nun an die entsprechenden Callbacks aufgerufen.
     
\begin{lstlisting}[language={c},firstnumber=168,caption={Auszug des Callbacks \lstinline{writeDataSourceVariable} in \lstinline{variables.c}\label{lst:4-writeDataSourceVariable}}]  
extern UA_StatusCode
 writeDataSourceVariable(UA_Server *server,
            const UA_NodeId *sessionId, void *sessionContext,
            const UA_NodeId *nodeId, void *nodeContext,
            const UA_NumericRange *range, const UA_DataValue *dataValue) {

    UA_StatusCode retval  = UA_STATUSCODE_GOOD;
    UA_NodeId *nameNodeId = UA_malloc(sizeof(UA_NodeId));
    UA_QualifiedName nameQN = UA_QUALIFIEDNAME(1, "Name");
    UA_Variant nameVar;
    UA_Boolean bit;

    retval |= findSiblingByBrowsename(server, nodeId, &nameQN, nameNodeId);
    retval |= UA_Server_readValue(server, *nameNodeId, &nameVar);
    retval |= UA_Boolean_copy(dataValue->value.data, &bit);

    |>\tikzmarkin[set border color=martinired]{writeIO}<|PI_writeSingleIO(String_fromUA_String(nameVar.data), &bit, false);                                                 |>\tikzmarkend{writeIO}<|

    free(nameNodeId);
    return retval;
 }
\end{lstlisting}

Listing~\ref{lst:4-writeDataSourceVariable} zeigt die Callback-Methode, welche nach dem Schreiben einer Variablen auf dem OPC-Server aufgerufen wird.
Dieser Methode wird neben der NodeId der mit ihr verknüpften Variablen auch der Wert dieser in Form eines Zeigers auf ein Struct vom Typ \lstinline{UA_DataValue} übergeben.

Die Gestaltung des hier verwendeten Nodesets sieht vor, dass in einer OPC-Variablen \lstinline{"Name"} der Bezeichner des zu schreibenden Digitalausgangs hinterlegt ist, siehe Abbildung~\ref{fig:opc-do}. Dies erlaubt eine Rekonfiguration der Ein- und Ausgänge der SPS ohne Änderungen im Programmcode des OPC-Servers vornehmen zu müssen.
Es ist daher erforderlich, nach jedem Schreiben einer mit einem Digitalausgang verknüpften Variablen, hier \lstinline{"Value"}, dessen Bezeichner \lstinline{"Name"} abzufragen. 
Dies geschieht in den Zeilen 180 und 181.
Anschließend wird dieser Bezeichner sowie der zu schreibende Wert der Methode \lstinline{PI_writeSingleIO()} übergeben, welche wiederum die Interaktion mit piControl übernimmt (vgl. Abschnitt \ref{sec:4-picontrol}).
 
\subsection{Integration von piControl%
     \label{sec:4-picontrol}}
In Abschnitt~\ref{sec:2-io} wurde die Anbindung der IO-Module des Revolution Pi sowie die Funktionsweise von piControl aus Anwendersicht beschrieben. Die verfügbare Literatur beschränkt sich auch auf lediglich diese Sicht; eine weiterführende Dokumentation für Entwickler gibt es, neben der in Abschnitt~\ref{sec:3-anbindung} vorgestellten Manpage, nicht. 
In diesem Abschnitt soll daher der Quellcode von piControl sowie dessen Verwendung im Projekt genauer betrachtet werden.
Hierzu wird exemplarisch die in Abschnitt~\ref{sec:4-open62541} eingeführte Methode \lstinline{PI_writeSingleIO()} untersucht.
Diese Methode ermöglicht das Setzen eines einzelnen Bits im Prozessabbild der SPS, und damit das Schalten eines digitalen Ausgangs auf einem IO-Modul.
Die äquivalente Methode \lstinline{int piControlGetBitValue(SPIValue *pSpiValue)} zum Lesen eines Bits bzw. Eingangs funktioniert analog und soll daher an dieser Stelle nicht dediziert erörtert werden.

\begin{lstlisting}[language={c},firstnumber=97,
                   caption={Setzen eines phsikalischen, digitalen Ausgangs in \lstinline{revpi.c}
                   \label{lst:4-PI_writeSingleIO}}]
extern void PI_writeSingleIO(char *pszVariableName, bool *bit, bool verbose)
{
	int rc;
	SPIVariable sPiVariable;
	SPIValue sPIValue;

	strncpy(sPiVariable.strVarName, pszVariableName, sizeof(sPiVariable.strVarName));
	rc = piControlGetVariableInfo(&sPiVariable);
	if (rc < 0) {
		printf("Cannot find variable '%s'\n", pszVariableName);
		return;
	}

		sPIValue.i16uAddress = sPiVariable.i16uAddress;
		sPIValue.i8uBit = sPiVariable.i8uBit;
		sPIValue.i8uValue = *bit;
		rc = |>\tikzmarkin[set border color=martinired]{setBitValue}<|piControlSetBitValue(&sPIValue)|>\tikzmarkend{setBitValue}<|;
		if (rc < 0)
			printf("Set bit error %s\n", getWriteError(rc));
		else if (verbose)
			printf("Set bit %d on byte at offset %d. Value %d\n", sPIValue.i8uBit, sPIValue.i16uAddress,
			       sPIValue.i8uValue);
}
\end{lstlisting}

Der Programmcode in Listing~\ref{lst:4-PI_writeSingleIO} ist Teil des implementierten OPC-Servers. In diesem wird auf zwei Funktionen des piControl-Treibers zugegriffen. 
Beiden Methoden wird als Argument ein Zeiger auf ein Struct vom Typ \lstinline{SPIValue} übergeben. Der im Struct abgelegte Name wird mittels \lstinline{piControlGetVariableInfo(&sPIValue)} zu einer Adresse im Prozessabbild aufgelöst. Diese wird in \lstinline{sPIValue.i16uAdress} gespeichert. Der Wert der Variablen wird anschließend mittels \lstinline{piControlSetBitValue(&sPIValue)} an dieser Adresse in das Prozessabbild geschrieben.

\begin{lstlisting}[language={c},firstnumber=309,caption={Methode \lstinline{piControlSetBitValue} in \lstinline{piControlIf.c}\label{lst:4-piControlSetBitValue}}]
int |>\tikzmarkin[set border color=martiniblue]{setBitValueFcn}<|piControlSetBitValue(SPIValue *pSpiValue)|>\tikzmarkend{setBitValueFcn}<|
{
    piControlOpen();

    if (PiControlHandle_g < 0)
	    return -ENODEV;

    pSpiValue->i16uAddress += pSpiValue->i8uBit / 8;
    pSpiValue->i8uBit %= 8;

    if (|>\tikzmarkin[set border color=martinired]{ioctl}<|ioctl(PiControlHandle_g, KB_SET_VALUE, pSpiValue)|>\tikzmarkend{ioctl}<| < 0)
	    return errno;

    return 0;
}
\end{lstlisting}

Die in Listing~\ref{lst:4-piControlSetBitValue} dargestellte Methode \lstinline{piControlSetBitValue} ist lediglich eine Hüllfunktion (häufig auch als Wrapper-Funktion bezeichnet) für einen Aufruf des \lstinline{ioctl} Kernel-Moduls.
Folgende Parameter werden übergeben:
\lstinline{PiControlHandle_g} ist die Referenz auf die Geräte-Datei des piControl-Treibers. \lstinline{KB_SET_VALUE} ist das ioctl-Kommando zum Schreiben eines Bits in das Prozessabbild. Der Zeiger \lstinline{pSpiValue} verweist auf ein Struct des bereits vorgestellten Typs \lstinline{SPIValue}.

\begin{lstlisting}[language={c},firstnumber=80,caption={Methode \lstinline{piControlOpen} in \lstinline{piControlIf.c}\label{lst:4-piControlOpen}}]
void piControlOpen(void)
{
    /* open handle if needed */
    if (PiControlHandle_g < 0)
    {
	    |>\tikzmarkin[set border color=martiniblue]{PiControlHandle}<|PiControlHandle_g = open(PICONTROL_DEVICE, O_RDWR)|>\tikzmarkend{PiControlHandle}<|;
    }
}
\end{lstlisting}

Die in Listing~\ref{lst:4-piControlOpen} dargestellte Methode öffnet, sofern nicht bereits geschehen, die Geräte-Datei. Das Macro \lstinline{PICONTROL_DEVICE} verweist hierbei auf \lstinline{/dev/piControl0}.

\begin{lstlisting}[language={c},firstnumber=721,caption={Methode \lstinline{piControlIoctl} in \lstinline{piControlMain.c}\label{lst:4-piControlIoctl}}]
static long |>\tikzmarkin[set border color=martiniblue, below offset=0.9em]{piControlIoctl}<|piControlIoctl(struct file *file, unsigned int prg_nr, 
                           unsigned long usr_addr)                                      |>\tikzmarkend{piControlIoctl}<|
{
  int status = -EFAULT;
  tpiControlInst *priv;
  int timeout = 10000;	// ms

  if (prg_nr != KB_CONFIG_SEND && prg_nr != KB_CONFIG_START && !isRunning()) {
  	return -EAGAIN;
  }

  priv = (tpiControlInst *) file->private_data;

  if (prg_nr != KB_GET_LAST_MESSAGE) {
  	// clear old message
  	priv->pcErrorMessage[0] = 0;
  }

  switch (prg_nr) {|>\setcounter{lstnumber}{864}<|

    case |>\tikzmarkin[set border color=martiniblue]{KB_SET_VALUE}<|KB_SET_VALUE:|>\tikzmarkend{KB_SET_VALUE}<|
  		{
  			SPIValue *pValue = (SPIValue *) usr_addr;

  			if (!isRunning())
  				return -EFAULT;

  			if (pValue->i16uAddress >= KB_PI_LEN) {
  				status = -EFAULT;
  			} else {
  				INT8U i8uValue_l;
  				my_rt_mutex_lock(&piDev_g.lockPI);
  				i8uValue_l = piDev_g.ai8uPI[pValue->i16uAddress];

  				if (pValue->i8uBit >= 8) {
  					i8uValue_l = pValue->i8uValue;
  				} else {
  					if (pValue->i8uValue)
  						i8uValue_l |= (1 << pValue->i8uBit);
  					else
  						i8uValue_l &= ~(1 << pValue->i8uBit);
  				}

  				|>\tikzmarkin[set border color=martinired]{i8uValue}<|piDev_g.ai8uPI[pValue->i16uAddress] = i8uValue_l;|>\tikzmarkend{i8uValue}<|
  				rt_mutex_unlock(&piDev_g.lockPI);

  #ifdef VERBOSE
  				pr_info("piControlIoctl Addr=%u, bit=%u: %02x %02x\n", pValue->i16uAddress, pValue->i8uBit, pValue->i8uValue, i8uValue_l);
  #endif

  				status = 0;
  			}
  		}
  		break; |>\setcounter{lstnumber}{1314}<|

    default:
      pr_err("Invalid Ioctl");
      return (-EINVAL);
      break;

    }

    return status;
  }
\end{lstlisting}

Listing~\ref{lst:4-piControlIoctl} zeigt in Auszügen die ioctl-Methode des piControl Kernel-Treibers. Diese bekommt folgende Argumente übergeben: \lstinline{struct file *file} enthält den Verweis auf die Geräte-Datei, hier \lstinline{/dev/piControl0}. Der Wert von \lstinline{unsigned int prg_nr} beschreibt die Anfrage an den Treiber, in diesem Fall \lstinline{KB_SET_VALUE}. Das Argument \lstinline{unsigned long usr_addr} enthält einen typ-agnostischen Pointer. Dieser verweist auf einen Speicherbereich, in welchem die zur Bearbeitung der Anfrage notwendigen Daten abgelegt sind. Hier können auch vom Treiber empfangene Daten dem Anwendungsprogramm bereitgestellt werden. 

Die switch-case-Anweisung führt die über das Argument \lstinline{prg_nr} spezifizierte Aktion aus. Hier betrachten wir \lstinline{KB_SET_VALUE}:
Zunächst wird in Zeile 868 der übergebene Zeiger \lstinline{usr_addr} mittels explizitem Typecast zu einem Zeiger des Typs \lstinline{SPIValue *} konvertiert. Da dieser auf Daten im Userspace verweist, ist beim Zugriff durch den Kernel-Treiber besondere Vorsicht geboten.
In Zeile 877 wird mittels Mutex das Prozessabbild \lstinline{piDev_g} für den Zugriff durch andere Threads oder Prozesse gesperrt.
\lstinline{my_rt_mutex_lock} verweist hierbei auf die Funktion \lstinline{rt_mutex_lock} aus \lstinline{linux/sched.h}\footnote{Offenbar wurde hier auch eine alternative Implementierung vorgesehen, siehe revpi\_common.h}

In Zeile 889 wird das Byte \lstinline{i8uValue_l}, welches den zu schreibenden Wert enthält in das Prozessabbild übertragen. Anschließend wird die Mutex auf \lstinline{piDev_g} wieder entsperrt.
\newpage

\begin{lstlisting}[language={c},firstnumber=62,caption={Auszug des Struct \lstinline{spiControlDev} in \lstinline{piControlMain.h}\label{lst:4-spiControlDev}}]
|>\tikzmarkin[set border color=martiniblue]{spiControlDev}<|typedef struct spiControlDev|>\tikzmarkend{spiControlDev}<| {
	// device driver stuff
	int init_step;
	enum revpi_machine machine_type;
	void *machine;
	struct cdev cdev;	// Char device structure
	struct device *dev;
	struct thermal_zone_device *thermal_zone;

	|>\tikzmarkin[set border color=martiniblue]{processImage}<|// process image stuff
	INT8U ai8uPI[KB_PI_LEN];
	INT8U ai8uPIDefault|>\tikzmarkin[set border color=martinired]{KB_PI_LEN_0}<|[KB_PI_LEN]|>\tikzmarkend{KB_PI_LEN_0}<|;
	struct rt_mutex lockPI;        |>\tikzmarkend{processImage}<|
	bool stopIO;
	piDevices *devs; |>\setcounter{lstnumber}{94}<|
} tpiControlDev;
\end{lstlisting}

Das Prozessabbild ist als Byte-Array der Länge \lstinline{KB_PI_LEN} in Listing~\ref{lst:4-spiControlDev} definiert. Konfigurationsparameter wie \lstinline{KB_PI_LEN} oder die Zykluszeit für den Datenaustausch zwischen SPS und IO-Modulen sind im folgenden Listing~\ref{lst:4-process} definiert.

\begin{lstlisting}[language={c},firstnumber=119,caption={Konfigurationsparameter des Prozessabbildes in project.h\label{lst:4-process}}]
#define INTERVAL_PI_GATE (5*1000*1000)  // 5 ms piGateCommunication |>\setcounter{lstnumber}{128}<|

#define INTERVAL_IO_COM (5*1000*1000)  // 5 ms piIoComm |>\setcounter{lstnumber}{132}<|

#define KB_PD_LEN       512
|>\tikzmarkin[set border color=martiniblue]{KB_PI_LEN_1}<|#define KB_PI_LEN       4096|>\tikzmarkend{KB_PI_LEN_1}<|
\end{lstlisting}

Das zu setzende Bit wurde zu diesem Zeitpunkt erfolgreich in das Prozessabbild der SPS geschrieben.
Es stellt sich die Frage, wie dieses nun an das IO-Modul kommuniziert wird.
Die Kommunikation mit allen angebundenen Modulen ist ebenfalls Aufgabe des piControl-Treibers.

\begin{lstlisting}[language={c},firstnumber=256,caption={Auszug der Methode \lstinline{piIoThread} in \lstinline{revpi_core.c}\label{lst:4-piIoThread}}]
static int piIoThread(void *data)
{
	//TODO int value = 0;
	ktime_t time;
	ktime_t now;
	s64 tDiff;

	hrtimer_init(&piCore_g.ioTimer, CLOCK_MONOTONIC, HRTIMER_MODE_ABS);
	piCore_g.ioTimer.function = piIoTimer;

	pr_info("piIO thread started\n");

	now = hrtimer_cb_get_time(&piCore_g.ioTimer);

	PiBridgeMaster_Reset();

	while (!kthread_should_stop()) {
		if (|>\tikzmarkin[set border color=martinired]{PiBridgeMaster}<|PiBridgeMaster_Run()|>\tikzmarkend{PiBridgeMaster}<| < 0)
			break;
	}

	RevPiDevice_finish();

	pr_info("piIO exit\n");
	return 0;
}
\end{lstlisting}

Der Kernel-Thread \lstinline{piIoThread} ist verantwortlich für den zyklischen Datenaustausch mit den IO-Modulen. In diesem wird fortlaufend die Methode \lstinline{PiBridgeMaster_Run()} aufgerufen, siehe Listing~\ref{lst:4-piIoThread}.

\begin{lstlisting}[language={c},firstnumber=262,caption={Auszug der Methode \lstinline{PiBridgeMaster_Run(void)} in \lstinline{RevPiDevice.c}\label{lst:4-PiBridgeMaster_Run}}]
int PiBridgeMaster_Run(void)
{
	static kbUT_Timer tTimeoutTimer_s;
	static kbUT_Timer tConfigTimeoutTimer_s;
	static int error_cnt;
	static INT8U last_led;
	static unsigned long last_update;
	int ret = 0;
	int i;

	my_rt_mutex_lock(&piCore_g.lockBridgeState);
	if (piCore_g.eBridgeState != piBridgeStop) {
		switch (eRunStatus_s) { |>\setcounter{lstnumber}{514}<|
		    case enPiBridgeMasterStatus_EndOfConfig:|>\setcounter{lstnumber}{621}<|
		    if (|>\tikzmarkin[set border color=martinired]{RevPiDevice}<|RevPiDevice_run()|>\tikzmarkend{RevPiDevice}<|) {
				// an error occured, check error limits |>\setcounter{lstnumber}{641}<|
			} else {
				ret = 1;
			}
			piCore_g.image.drv.i16uRS485ErrorCnt = RevPiDevice_getErrCnt();
			break;
\end{lstlisting}

Die in Listing~\ref{lst:4-PiBridgeMaster_Run} dargestellte Methode ist eine sog. State-Machine. Ist die Konfiguration der IO-Module erfolgreich abgeschlossen, so führt sie bei Aufruf lediglich die Methode \lstinline{RevPiDevice_run()} aus.

\begin{lstlisting}[language={c},firstnumber=140,caption={Auszug der Methode \lstinline{RevPiDevice_run(void)} in \lstinline{RevPiDevice.c}\label{lst:4-RevPiDevice_run}}]
int RevPiDevice_run(void)
{
	INT8U i8uDevice = 0;
	INT32U r;
	int retval = 0;

	RevPiDevices_s.i16uErrorCnt = 0;

	for (i8uDevice = 0; i8uDevice < RevPiDevice_getDevCnt(); i8uDevice++) {
		if (RevPiDevice_getDev(i8uDevice)->i8uActive) {
			switch (RevPiDevice_getDev(i8uDevice)->sId.i16uModulType) {
			case KUNBUS_FW_DESCR_TYP_PI_DIO_14:
			case KUNBUS_FW_DESCR_TYP_PI_DI_16:
			case KUNBUS_FW_DESCR_TYP_PI_DO_16:
				r = |>\tikzmarkin[set border color=martinired]{sendCyclicTelegram}<|piDIOComm_sendCyclicTelegram(i8uDevice)|>\tikzmarkend{sendCyclicTelegram}\setcounter{lstnumber}{166} <|;

				break; |>\setcounter{lstnumber}{216}<|
			}
		}
	} |>\setcounter{lstnumber}{227}<|
	return retval;
}
\end{lstlisting}

Diese iteriert wie in Listing~\ref{lst:4-RevPiDevice_run} abgebildete durch alle gegenwärtig in der SPS konfigurierten Module. Ist das aktuelle Modul als aktiv markiert, so wird anhand eines sog. Firmware-Descriptors entschieden, welche Methode für die Ansteuerung des Moduls aufzurufen ist.

\begin{lstlisting}[language={c},firstnumber=161,caption={Auszug der Methode \lstinline{piDIOComm_sendCyclicTelegram} in \lstinline{piDIOComm.c}\label{lst:4-sendCyclicTelegram}}]
INT32U piDIOComm_sendCyclicTelegram(INT8U i8uDevice_p)
{
	INT32U i32uRv_l = 0;
	SIOGeneric sRequest_l;
	SIOGeneric sResponse_l;
	INT8U len_l, data_out[18], i, p, data_in[70];
	INT8U i8uAddress;
	int ret; |>\setcounter{lstnumber}{239}<|
	
    |>\tikzmarkin[set border color=martinired]{piIoComm}<|ret = piIoComm_send((INT8U *) & sRequest_l, IOPROTOCOL_HEADER_LENGTH + len_l + 1);  |>\tikzmarkend{piIoComm}\setcounter{lstnumber}{298}<|
}
\end{lstlisting}

Im Falle des hier verwendeten DO-Moduls wird die in Listing~\ref{lst:4-sendCyclicTelegram} abgebildete Methode \lstinline{piDIOComm_sendCyclicTelegram()} aufgerufen. Dieser wird ein Zeiger auf das zu schreibende Gerät übergeben. 
Zunächst wird das Prozessabbild mittels eines proprietären, jedoch im Quellcode offen nachvollziehbaren Protokolls in ein \lstinline{sRequest_l} genanntes Byte-Array umgewandelt. Dieser Schritt ist in Listing~\ref{lst:4-sendCyclicTelegram} nicht abgebildet. Anschließend wird \lstinline{piIoComm_send()} ein Zeiger auf die so generierte Schreib-Anfrage übergeben.

\begin{lstlisting}[language={c},firstnumber=220,caption={Auszug der Methode \lstinline{piIOComm_send} in \lstinline{piIOComm.c}\label{lst:4-piIOComm_send}}]
int piIoComm_send(INT8U * buf_p, INT16U i16uLen_p)
{
	ssize_t write_l = 0;
	INT16U i16uSent_l = 0;|>\setcounter{lstnumber}{249}<|

	while (i16uSent_l < i16uLen_p) {
		write_l = vfs_write(piIoComm_fd_m, buf_p + i16uSent_l, i16uLen_p - i16uSent_l, &piIoComm_fd_m->f_pos);
		if (write_l < 0) {
			pr_info_serial("write error %d\n", (int)write_l);
			return -1;
		} 
		i16uSent_l += write_l;|>\setcounter{lstnumber}{263}<|
	}
	clear();
	vfs_fsync(piIoComm_fd_m, 1);
	return 0;
}
\end{lstlisting}

Listing~\ref{lst:4-piIOComm_send} zeigt die Implementierung von \lstinline{piIoComm_send()}. Diese Methode ist für das Schreiben der oben generierten Anfrage auf die seriellen Schnittstelle verantwortlich. Realisiert wird dies mittels der Methode \lstinline{vfs_write()}. Diese ist in \lstinline{<linux/fs.h>} definiert. Sie ermöglicht das Schreiben einer Datei im Userspace aus dem Kernel heraus. Geschrieben wird hier die Datei mit dem Deskriptor \lstinline{piIoComm_fd_m}.
Da die Funktion \lstinline{vfs_write()} durch andere Kernel-Tasks unterbrochen werden kann, ist nicht gewährleistet, dass die gesamte Anfrage mit nur einem Aufruf geschrieben wird. Die oben abgebildete while-Schleife stellt das vollständige Senden der Anfrage sicher.

\begin{lstlisting}[language={c},firstnumber=157,caption={Auszug der Methode \lstinline{piIOComm_open_serial} in \lstinline{piIOComm.c}\label{lst:4-piIOComm_open_serial}}]
int piIoComm_open_serial(void)
{   |>\setcounter{lstnumber}{167}<|
	struct file *fd;	/* Filedeskriptor */
	struct termios newtio;	/* Schnittstellenoptionen */

	|>\tikzmarkin[set border color=martiniblue]{fd}<|/* Port oeffnen - read/write, kein "controlling tty", 
	    Status von DCD ignorieren */
	fd = filp_open(|>\tikzmarkin[set border color=martinired]{tty}<|REV_PI_TTY_DEVICE|>\tikzmarkend{tty}<|, O_RDWR | O_NOCTTY, 0); |>\setcounter{lstnumber}{208}<|
	
	piIoComm_fd_m = fd;                                                      |>\tikzmarkend{fd}\setcounter{lstnumber}{217}<|

	return 0;
}
\end{lstlisting}

Der zum Schreiben auf die serielle Schnittstelle verwendete Datei-Deskriptor wird von der in Listing~\ref{lst:4-piIOComm_open_serial} abgebildeten Methode \lstinline{piIoComm_open_serial()} generiert. 

\begin{lstlisting}[language={c},firstnumber=45,caption={Definition der seriellen Schnittstelle in \lstinline{piIOComm.h}\label{lst:4-REV_PI_TTY_DEVICE}}]
#define REV_PI_TTY_DEVICE	"/dev/ttyAMA0"
\end{lstlisting}

Das in Listing~\ref{lst:4-REV_PI_TTY_DEVICE} definierte Macro verweist auf eine der seriellen Schnittstellen des RaspberryPi.
Die Implementierung des zugehörigen Schnittstellentreibers soll hier nicht weiter untersucht werden. Somit ist an dieser Stelle die Kette vom Setzen einer Variablen auf dem OPC-Server bis hin zur Aktualisierung des Prozessabbilds der IO-Module geschlossen.

% \begin{lstlisting}[language={c},firstnumber={226},caption={Setzen der Scheduler-Priorität auf SCHED\_FIFO in 
% revpi\_common.c\label{lst:2-sched_priority}}]
% param.sched_priority = ktprio->prio;
% ret = sched_setscheduler(child, SCHED_FIFO, &param);
% \end{lstlisting}
%% % Imports nur für Referenzenauflösung während des Schreibens! Vorm Kompilieren auskommentieren!
% \bibliography{0_hauptdatei}
% \input{1_einleitung}
% \input{2_grundlagen}
% \input{3_konzeption}
% \input{4_implementierung}
% \input{5_tests}
% \input{6_zusammenfassung}
% % Ende Imports

\section{Test des OPC-Servers im Gesamtsystem%
  \label{sec:5-tests}}

%% % Imports nur für Referenzenauflösung während des schreibens! Vorm Kompilieren auskommentieren!
% \bibliography{0_hauptdatei}
% \input{1_einleitung}
% \input{2_grundlagen}
% \input{3_konzeption}
% \input{4_implementierung}
% \input{5_tests}
% \input{6_zusammenfassung}
% % Ende Imports

\section{Zusammenfassung und Ausblick%
  \label{sec:6-fazit}}
Der folgende Abschnitt~\ref{sec:6-zusammenfassung} fasst die gewonnenen Erkenntnisse und den Stand der Implementierung zusammen.
Den Abschluss dieser Arbeit bildet der Ausblick in Abschnitt~\ref{sec:6-ausblick}.

\subsection{Zusammenfassung%
     \label{sec:6-zusammenfassung}}

\subsection{Ausblick%
     \label{sec:6-ausblick}}

% % Ende Imports

\section{Einleitung und Motivation%
  \label{sec:1-einleitung}}
Ziel dieses Projektes ist die Integration eines OPC-Servers mit einer auf Linux
basierenden speicherprogrammierbaren Steuerung (SPS). Angeschlossen an diese SPS
ist jeweils ein digitales Ein-/\,bzw.~Ausgabemodul. Die von diesen bereitgestellten
Ein-/\, bzw.~Ausgänge (IO) sollen in der Datenstruktur des OPC-Servers abgebildet
und über diesen für OPC-Clients les-/\,und schreibar sein. Weiterhin sollen einige
Funktionen zur Überwachung und Steuerung der an die SPS angeschlossenen Aktoren
und Sensoren direkt im OPC-Server implementiert werden.
Hiermit stellt dieses Projekt eine der Grundlagen für ein übergeordnetes Projekt,
die cloudbasierte Steuerung eines miniaturisierten Produktions-Systems, dar.

Der hier verwendete OPC-Server ist Teil des sog. open62541 Projekts. Er ist in C
geschrieben und implementiert bereits einen großen Teil der im OPC-UA-Standard
spezifizierten Funktionen.
Als SPS findet ein Revolution Pi 3 der Firma Kunbus Verwendung. Dieser integriert
ein sog. Compute Module der Raspberry Pi Foundation in ein industrietaugliches
Gehäuse und erlaubt die Erweiterung mittels IO- oder Gateway-Modulen. Über diese
erfolgt die Kommunikation mit weiteren Komponenten der Automatisierungstechnik.

Motiviert ist dieses Projekt durch die Beobachtung, dass die Verbreitung offener
Standards sowie freier Software auch in der Automatisierungstechnik zunimmt.
Linux ist ein freies Betriebssystem, OPC-UA ein offen zugänglicher, aktiv gepflegter
und weit verbreiteter Standard. Der Raspberry Pi findet sowohl bei Hobby-Anwendern als
auch in den Bereichen Forschung und Entwicklung sowie bei industriellen Anwendern
Verwendung. Dieses Projekt stellt somit eine für unterschiedliche Anwender interessante
Entwicklung dar.

Im Anschluss an diese einleitende Übersicht im Abschnitt~\ref{sec:1-einleitung} folgt
die Darstellung der wichtigsten Grundlagen in Abschnitt~\ref{sec:2-grundlagen}.
Aufbauend auf diesen Grundlagen folgt die konzeptuelle Ausarbeitung im Abschnitt~\ref{sec:3-konzeption}.
Die Umsetzung wird im Abschnitt~\ref{sec:4-implementierung} erläutert.
Die Leistungsfähigkeit der Implementierung wird in Abschnitt~\ref{sec:5-tests} untersucht.
Eine Zusammenfassung und ein Ausblick schließen die Arbeit in
Abschnitt~\ref{sec:6-fazit} ab. Eventuell noch benötigte Anhänge
finden sich in den Anhängen [...] bis [...].

% % % Imports nur für Referenzenauflösung während des Schreibens! Vorm Kompilieren auskommentieren!
% \bibliography{0_hauptdatei}
% % Mit \section{...} eröffnen wir einen neuen Abschnitt.
% Der Befehl setzt nicht nur den Text in einer größeren,
% fetten Schrift, sondern sorgt außerdem dafür, daß er im
% Inhaltsverzeichnis erscheint.
%
% Mit \label{...} erzeugen wir einen Bezeichner, mit dessen Hilfe
% wir später auf die Nummer des Abschnitts verweisen können (nämlich
% mit~\ref{...}).
%
% Das Kommentarzeichen hinter „Übersicht“ dient dazu, ein
% Leerzeichen zwischen „Übersicht“ und dem \label-Befehl
% zu vermeiden, das andernfalls sichtbar würde – z.B. im
% Inhaltsverzeichnis.
%

% % Imports nur für Referenzenauflösung während des Schreibens! Vorm Kompilieren auskommentieren!
% \bibliography{0_hauptdatei}
% \input{1_einleitung}
%\input{2_grundlagen}
%\input{3_konzeption}
%\input{4_implementierung}
%\input{5_tests}
%\input{6_zusammenfassung}
% % Ende Imports

\section{Einleitung und Motivation%
  \label{sec:1-einleitung}}
Ziel dieses Projektes ist die Integration eines OPC-Servers mit einer auf Linux
basierenden speicherprogrammierbaren Steuerung (SPS). Angeschlossen an diese SPS
ist jeweils ein digitales Ein-/\,bzw.~Ausgabemodul. Die von diesen bereitgestellten
Ein-/\, bzw.~Ausgänge (IO) sollen in der Datenstruktur des OPC-Servers abgebildet
und über diesen für OPC-Clients les-/\,und schreibar sein. Weiterhin sollen einige
Funktionen zur Überwachung und Steuerung der an die SPS angeschlossenen Aktoren
und Sensoren direkt im OPC-Server implementiert werden.
Hiermit stellt dieses Projekt eine der Grundlagen für ein übergeordnetes Projekt,
die cloudbasierte Steuerung eines miniaturisierten Produktions-Systems, dar.

Der hier verwendete OPC-Server ist Teil des sog. open62541 Projekts. Er ist in C
geschrieben und implementiert bereits einen großen Teil der im OPC-UA-Standard
spezifizierten Funktionen.
Als SPS findet ein Revolution Pi 3 der Firma Kunbus Verwendung. Dieser integriert
ein sog. Compute Module der Raspberry Pi Foundation in ein industrietaugliches
Gehäuse und erlaubt die Erweiterung mittels IO- oder Gateway-Modulen. Über diese
erfolgt die Kommunikation mit weiteren Komponenten der Automatisierungstechnik.

Motiviert ist dieses Projekt durch die Beobachtung, dass die Verbreitung offener
Standards sowie freier Software auch in der Automatisierungstechnik zunimmt.
Linux ist ein freies Betriebssystem, OPC-UA ein offen zugänglicher, aktiv gepflegter
und weit verbreiteter Standard. Der Raspberry Pi findet sowohl bei Hobby-Anwendern als
auch in den Bereichen Forschung und Entwicklung sowie bei industriellen Anwendern
Verwendung. Dieses Projekt stellt somit eine für unterschiedliche Anwender interessante
Entwicklung dar.

Im Anschluss an diese einleitende Übersicht im Abschnitt~\ref{sec:1-einleitung} folgt
die Darstellung der wichtigsten Grundlagen in Abschnitt~\ref{sec:2-grundlagen}.
Aufbauend auf diesen Grundlagen folgt die konzeptuelle Ausarbeitung im Abschnitt~\ref{sec:3-konzeption}.
Die Umsetzung wird im Abschnitt~\ref{sec:4-implementierung} erläutert.
Die Leistungsfähigkeit der Implementierung wird in Abschnitt~\ref{sec:5-tests} untersucht.
Eine Zusammenfassung und ein Ausblick schließen die Arbeit in
Abschnitt~\ref{sec:6-fazit} ab. Eventuell noch benötigte Anhänge
finden sich in den Anhängen [...] bis [...].

% % % Imports nur für Referenzenauflösung während des Schreibens! Vorm Kompilieren auskommentieren!
% \bibliography{0_hauptdatei}
% \input{1_einleitung}
% \input{2_grundlagen}
% \input{3_konzeption}
% \input{4_implementierung}
% \input{5_tests}
% \input{6_zusammenfassung}
% % Ende Imports

\section{Grundlagen%
  \label{sec:2-grundlagen}}

\subsection{Speicherprogrammierbare-Steuerung und Linux -- Revolution Pi%
     \label{sec:2-sps}}

\subsubsection{Kunbus RevolutionPi%
        \label{sec:2-revpi}}
Der RevolutionPi 3 ist eine speicherprogrammierbare Steuerung (SPS) des Herstellers
Kunbus GmbH. Kern dieser SPS ist das von der Raspberry Pi Foundation entwickelte
und vertriebene Raspberry Pi Compute Module 3. Dieses integriert ein Broadcom BCM2837
System-on-Chip (SoC) mit vier 1,2GHz Prozessorkernen, 1GB RAM, 4GB eMMC Anwendungsspeicher
und sonstige Peripherie in ein Modul im DDR2-SODIMM Formfaktor. Diese Spezifikationen
sind weitgehend identisch zu denen des ausgesprochen populären Raspberry Pi 3.
Der Revolution Pi profitiert daher von dem gleichen großen Angebot an Software
und Unterstützung wie der Raspberry Pi, ergänzt dessen Hardware jedoch um eine 24V
Spannungsversorgung, die Möglichkeit der Erweiterung durch mehrere industrietaugliche
Ein-/ Ausgabemodule und Gateways sowie ein Gehäuse zur Montage auf einer DIN-Schiene.
\begin{itemize}
  \item{Prozessor: BCM2837}
  \item{Taktfrequenz 1,2 GHz}
  \item{Anzahl Prozessorkerne: 4}
  \item{Arbeitsspeicher: 1 GByte}
  \item{eMMC Flash Speicher: 4 GByte}
  \item{Betriebssystem: Angepasstes Raspbian mit RT-Patch}
  \item{RTC mit 24h Pufferung über wartungsfreien Kondensator}
  \item{Treiber / API: Treiber schreibt zyklisch Prozessdaten in ein Prozessabbild, Zugriff auf Prozessabbild über Linux-Filesystem als API zu Fremdsoftware.}
  \item{Kommunikationsanschlüsse: 2 x USB 2.0 A (je 500 mA belastbar), 1 x Micro-USB, HDMI, Ethernet (RJ45) 10/100 Mbit/s}
  \item{Stromversorgung: min. 10,7 V, max. 28,8 V, maximal 10 Watt}
  \item{Zulässige Umgebungstemperatur: -40 bis +55 C}
  \item{Gehäuseabmessungen: (HxBxL) 96 mm x 22,5 mm x 110,5 mm (ohne gesteckte Stecker)}
  \item{ESD Schutz: 4 kV / 8 kV gemäß EN61131-2 und IEC 61000-6-2}
  \item{Surge / Burst Prüfungen: gemäß EN61131-2 und IEC 61000-6-2 eingekoppelt auf Versorgungsspannung, Ethernet und IO-Leitungen}
  \item{EMI Prüfungen: gemäß EN61131-2 und IEC 61000-6-2}
\end{itemize}

Kunbus bietet eine Auswahl an IO- und Gateway-Modulen zur Erweiterung des Revolution Pi an.
Gateways dienen der Kommunikation mit Systemen oder Komponenten der Automatisierungstechnik
über Protokolle wie PROFIBUS oder EtherCAT. IO-Module erlauben die Überwachung
und Steuerung von digitalen oder analogen Ein- und Ausgängen.

\subsubsection{Zugriff auf IO-Module%
        \label{sec:2-io}}
Der Zugriff auf die Ein- und Ausgänge der IO-Module erfolgt über ein Prozessabbild
und einen hierfür von Kunbus bereitgestellten Treiber, genannt piControl. Dieser
aktualisiert das Prozessabbild zyklisch. Die angestrebte Zykluszeit beträgt 5ms,
kann jedoch je nach Anzahl der angeschlossenen Module auch größer sein. Kunbus
garantiert bei drei IO-Modulen und zwei Gateway-Modulen eine Zykluszeit von 10 ms.
Jedes der IO-Module stellt ein eigenständiges eingebettetes System dar. Es verfügt
über einen Microcontroller, welcher die IOs bereitstellt und über einen RS485-Bus
mit dem Revolution Pi kommuniziert.
% https://revolution.kunbus.de/io-modul/

Lizenz: GPL
% https://github.com/RevolutionPi/piControl

\begin{lstlisting}[language={c},firstnumber={226},caption={Setzen der Scheduler-Priorität auf SCHED\_FIFO in revpi\_common.c\label{lst:2-sched_priority}}]
param.sched_priority = ktprio->prio;
ret = sched_setscheduler(child, SCHED_FIFO,
       &param);
\end{lstlisting}


\subsection{Echtzeit und Multithreading unter Linux -- preemptRT und posix%
     \label{sec:2-echtzeit}}


 Der Linux-Kernel verfügt über mehrere unterschiedliche Preemtion-Modelle:

\begin{itemize}
  \item No Forced Preemption (server):
  Ausgelegt auf maximal möglichen Durchsatz, lediglich Interrupts und
  System-Call-Returns bewirken Präemption.

  \item Voluntary Kernel Preemption (Desktop):
  Neben den implizit bevorrechtigten Interrupts und System-Call-Returns gibt es
  in diesem Modell weitere Abschnitte des Kernels in welchen Preämption explizit
  gestattet ist.

  \item Preemptible Kernel (Low-Latency Desktop):
  In diesem Modell ist der gesamte Kernel, mit Ausnahme sog.~kritischer Abschnitte
  präemptible. Nach jedem kritischen Abschnitt gibt es einen impliziten Präemptions-Punkt.

  \item Preemptible Kernel (Basic RT):
  Dieses Modell ist dem zuvor genannten sehr ähnlich, hier sind jedoch alle Interrupt-Handler
  als eigenständige Threads ausgeführt.

  \item Fully Preemptible Kernel (RT):
  Wie auch bei den beiden zuvor genannten Modellen ist hier der gesamte Kernel
  präemtible, die Anzahl und Dauer der nicht-präemtiblen kritischen Abschnitte
  ist auf ein notwendiges Minimum beschränkt. Alle Interrupt-Handler sind als
  eigenständige Threads ausgeführt, Spinlocks durch Sleeping-Spinlocks und Mutexe
  durch sog.~RT-Mutexe ersetzt.

\end{itemize}
\todo{Spinlocks und Mutexe sowie die RT-Varianten dieser erklären!}

Lediglich mit dem vollständig präemtiblen Kernel kann Echtzeit-Verhalten realisiert werden.

% https://wiki.linuxfoundation.org/realtime/documentation/technical_basics/preemption_models bzw kernel/Kconfig.preempt

\subsubsection{preemptRT%
        \label{sec:2-preemptRT}}
% https://wiki.linuxfoundation.org/realtime/documentation/technical_details/start
% https://wiki.linuxfoundation.org/realtime/documentation/technical_basics/start

Das dem PREEMPT RT Kernel zugrunde liegende Prinzip lässt sich in einer einfachen
Regel ausdrücken: Nur Code, welcher absolut nicht-präemtible sein darf, ist es
gestattet nicht-präemtible zu sein.
Das erklärte Ziel des PREEMPT\_RT Patches ist es folglich, die Menge des nicht-präemtiblen
Codes im Linux-Kernel auf das absolut notwendige Minimum zu reduzieren.

Dies wird durch Verwendung folgender Mechanismen erreicht:

\begin{itemize}
  \item Hochauflösende Timer
  \item Sleeping Spinlocks
  \item Threaded Interrupt Handlers
  \item rt\_mutex
  \item RCU
\end{itemize}


\subsubsection{posix%
        \label{sec:2-posix}}
Ist posix hier wirklich relevant? Debian bzw.~Raspbian sind weitgehend posix
kompatibel, aber wird es hier genutzt? -> JA, open62541 nutzt pthread.h
piControl nutzt kthread.h, und semaphore.h

\subsection{OPC-UA und open62541%
     \label{sec:2-opc}}

\subsubsection{OPC UA%
        \label{sec:2-opcua}}
Open Platform Communications (OPC) ist eine Familie von Standards zur herstellerunabhängigen
Kommunikation von Maschinen (M2M) in der Automatisierungstechnik. Die sog.~OPC Task Force, zu deren
Mitgliedern verschiedene große Firmen der Automatisierungsindustrie gehören, veröffentlichte
die OPC Specification Version 1.0 im August 1996.
Motiviert ist dieser offene Standard durch die Erkenntniss, dass die Anpassung der
zahlreichen Herstellerstandards an individuelle Infrastrukturen und Anlagen einen
großen Mehraufwand verursachen.
Die Wikipedia beschreibt das Anwendungsgebiet für OPC wie folgt:

\glqq{}OPC wird dort eingesetzt, wo Sensoren, Regler und Steuerungen verschiedener Hersteller
ein gemeinsames Netzwerk bilden. Ohne OPC benötigten zwei Geräte zum Datenaustausch
genaue Kenntnis über die Kommunikationsmöglichkeiten des Gegenübers. Erweiterungen
und Austausch gestalten sich entsprechend schwierig. Mit OPC genügt es, für jedes
Gerät genau einmal einen OPC-konformen Treiber zu schreiben. Idealerweise wird
dieser bereits vom Hersteller zur Verfügung gestellt. Ein OPC-Treiber lässt sich
ohne großen Anpassungsaufwand in beliebig große Steuer- und Überwachungssysteme
integrieren.

OPC unterteilt sich in verschiedene Unterstandards, die für den jeweiligen Anwendungsfall
unabhängig voneinander implementiert werden können. OPC lässt sich damit verwenden
für Echtzeitdaten (Überwachung), Datenarchivierung, Alarm-Meldungen und neuerdings
auch direkt zur Steuerung (Befehlsübermittlung).\grqq{}

OPC basiert in der ursprünglichen Spezifikation auf Microsofts DCOM-Spezifikation.
DCOM macht Funktionen und Objekte einer Anwendung anderen Anwendungen im Netzwerk
zugänglich. Der OPC-Standard definiert entsprechende DCOM-Objekte um mit anderen
OPC-Anwendungen Daten austauschen zu können. Die Verwendung von DCOM bindet Anwender
an Betriebssysteme von Microsoft. Die ursprüngliche OPC Spezifikation wird durch die
Entwicklung von OPC Unified Architecture (OPC UA) abgelöst.
OPC UA setzt auf einem eigenen Kommunikationionsstack auf, die Verwendung von DCOM
und damit die Bindung an Microsoft wurden aufgelöst.

Die OPC-UA-Architektur ist eine Service-orientierte Architektur (SOA), deren Struktur
aus mehreren Schichten besteht.

% Wikipedia
Das OPC-Informationsmodell ist nicht mehr nur eine Hierarchie aus Ordnern, Items
und Properties. Es ist ein sogenanntes Full-Mesh-Network aus Nodes, mit dem neben
den Nutzdaten eines Nodes auch Meta- und Diagnoseinformationen repräsentiert werden.
Ein Node ähnelt einem Objekt aus der objektorientierten Programmierung. Ein Node
kann Attribute besitzen, die gelesen werden können (Data Access (DA), Historical
Data Access (HDA)). Es ist möglich Methoden zu definieren und aufzurufen.
Eine Methode besitzt Aufrufargumente und Rückgabewerte. Sie wird durch ein Command
aufgerufen. Weiterhin werden Events unterstützt, die versendet werden können
(AE (Alarms \& Events), DA DataChange), um bestimmte Informationen zwischen Geräten
auszutauschen. Ein Event besitzt unter anderem einen Empfangszeitpunkt, eine Nachricht
und einen Schweregrad. Die o. g. Nodes werden sowohl für die Nutzdaten als auch
alle anderen Arten von Metadaten verwendet. Der damit modellierte OPC-Adressraum
beinhaltet nun auch ein Typmodell, mit dem sämtliche Datentypen spezifiziert werden.

% https://de.wikipedia.org/wiki/Open_Platform_Communications
% https://de.wikipedia.org/wiki/OPC_Unified_Architecture
% https://opcfoundation.org/developer-tools/specifications-unified-architecture
% Von Gerhard Gappmeier - ascolab GmbH, CC BY-SA 3.0, https://de.wikipedia.org/w/index.php?curid=1892069
\subsubsection{open62541%
        \label{sec:2-open62541}}
open62541 ist eine offene und freie Implementierung von OPC UA. Die in C geschriebene
Bibliothek stellt eine beständig zunehmende Anzahl der im OPC UA Standard definierten
Funktionen bereit. Sie kann sowohl zur Erstellung von OPC-Servern als auch -Clients
genutzt werden. Ergänzend zu der unter der Mozilla Public License v2.0 lizensierten
Bibliothek stellt das open62541 Projekt auch Beispielprogramme unter einer CC0 Lizenz
zur Verfügung.

Die Bibliothek eignet sich auch für die Entwicklung auf eingebetteten Systemen und
Microcontrollern. Je nach Umfang der gewünschten Funktionen und des OPC Informationsmodells
beträgt die Größe einer Server-Binary weniger als 100kb. %evtl. kürzen?

\todo{Nodes erklären! Evtl.~oben!}

Folgende Auswahl an Eigenschaften und Funktionen zeichnet die in dieser Arbeit verwendete
Version 0.3 von open62541 aus:
\begin{itemize}
  \item Kommunikationionsstack
  \begin{itemize}
      \item OPC UA Binär-Protokoll (HTTP oder SOAP werden gegenwärtig nicht unterstützt)
      \item Austauschbare Netzwerk-Schicht, welche die Verwendung eigener Netzwerk-APIs
      erlaubt.
      \item Verschlüsselte Kommunikationion
      \item Asynchrone Dienst-Anfragen im Client
  \end{itemize}
  \item Informationsmodell
  \begin{itemize}
    \item Unterstützung aller OPC UA Node-Typen, inkl.~Methoden
    \item Hinzufügen und Entfernen von Nodes und Referenzen zur Laufzeit.
    \item Vererbung und Instanziierung von Objekt- und Variablentypen
    \item Zugriffskontrolle auch für einzelne Nodes
  \end{itemize}
  \item Subscriptions
  \begin{itemize}
    \item Erlaubt die Überwachung (subscriptions / monitoreditems)
    \item Sehr geringer Ressourcenbedarf pro überwachtem Wert
  \end{itemize}
  \item Code-Generierung auf XML-Basis
  \begin{itemize}
    \item Erlaubt die Erstellung von Datentypen
    \item Erlaubt die Generierung des serverseitigen Informationsmodells
  \end{itemize}
\end{itemize}

% https://open62541.org/doc/0.3/


Mozilla Public License
CC0 Lizenz für Beispiele und Plugins

% https://open62541.org/doc/open62541-current.pdf
% https://open62541.org/

% % % Imports nur für Referenzenauflösung während des Schreibens! Vorm Kompilieren auskommentieren!
% \bibliography{0_hauptdatei}
% \input{1_einleitung}
% \input{2_grundlagen}
% \input{3_konzeption}
% \input{4_implementierung}
% \input{5_tests}
% \input{6_zusammenfassung}
% \input{anhang}
% % Ende Imports

\section{Systemkonzept%
  \label{sec:3-konzeption}}
Auf Basis der in Abschnitt \ref{sec:2-grundlagen} vorgestellten Möglichkeiten folgt nun die Ausarbeitung eines Konzepts.
In den folgenden Abschnitten soll näher auf zwei zentrale Aspekte eingegangen werden: Abschnitt~\ref{sec:3-anbindung} stellt Möglichkeiten zum Zugriff auf Variablen bzw.\,Werte im Prozessabbild des Revolution Pi vor; in Abschnitt~\ref{sec:3-integration} wird ein Konzept zur Bereitstellung dieser Variablen auf einem OPC-Server vorgestellt.

\subsection{Anbindung der IO an den OPC-Server%
     \label{sec:3-anbindung}}

Eine Webanwendung mit Bezeichnung PiCtory dient zur Konfiguration der I/O- und virtuellen Module des RevolutionPi. Die Konfiguration liegt im JSON-Format in der Datei \lstinline{/etc/revpi/config.rsc}. Der piControl-Treiber liest diese Datei beim Start. 
Der folgende Auszug aus der Manpage des piControl-Kernelmoduls beschreibt die von diesem zum Lesen und Schreiben einzelner Bits des Prozessabbildes bereitgestellten Funktionen~\citep[vgl.]{web-revpi-manpage}. Sie ist an dieser Stelle weitgehend ungekürzt zitiert, da sie die nutzbare Schnittstelle sehr kompakt beschreibt.

\begin{lstlisting}[breakindent=0pt, numbers=none, caption={Auszug aus der Revolution Pi Programmers Manual\label{lst:4-manpage}}]
KB_FIND_VARIABLE SPIVariable *argp
Find a variable in the process image by its name. A pointer to a structure of type SPIVariable must be passed as argument. [...]
The struct SPIVariable [...] is defined as 
typedef struct SPIVariableStr
{
    char strVarName[32]; // Variable name
    uint16_t i16uAddress; // Address of the byte in the process image
    uint8_t i8uBit; // 0-7 bit position, >= 8 whole byte
    uint16_t i16uLength; // length of the variable in bits.
    // Possible values are 1, 8, 16 and 32
} SPIVariable;

Set and get values of the process image
KB_GET_VALUE SPIValue *argp
[...]
KB_SET_VALUE SPIValue *argp
Write one bit or one byte to the process image [...].  This call is more efficient than the usual calls of seek and write because only one function call is necessary. If more than on application are writing bits in one output byte, this call is the only safe way to set a bit without overwriting the other bits because this call is doing a read-modify-write-cycle. 

The struct SPIValue used by this ioctl is defined as
typedef struct SPIValueStr
{
    uint16_t i16uAddress; // Address of the byte in the process image
    uint8_t i8uBit; // 0-7 bit position, >= 8 whole byte
    uint8_t i8uValue; // Value: 0/1 for bit access, whole byte otherwise
} SPIValue;
\end{lstlisting} 

Die oben beschriebenden Funtkionen \lstinline{KB_FIND_VARIABLE}, \lstinline{KB_GET_VALUE} und \lstinline{KB_SET_VALUE} ermöglichen einen einfachen und (lt.\,Manpage) effizienten Zugriff auf einzelne Bits des Prozessabbildes und damit auch auf die IO des RevolutionPi.
Der Zugriff des OPC-Servers auf das Prozessabbild soll daher mittels dieser Funktionen realisiert werden.
\lstinline{KB_FIND_VARIABLE} kann genutzt werden, um Adressen von Variablen im Prozessabbild mittels ihres Namens aufzulösen.
\lstinline{KB_GET_VALUE} und \lstinline{KB_SET_VALUE} ermöglichen den Zugriff auf die Werte dieser Variablen.


\subsection{Integration des OPC-Servers in das System%
     \label{sec:3-integration}}

open62541 bietet drei Möglichkeiten zum Abgleich von Variablen mit dem Prozessabbild~\citep[vgl.][Tutorials - Connecting a Variable with a Physical Process]{web-open62541}:
\begin{itemize}
    \item Manuelles oder zyklisches Aktualisieren
    \item Variable Value Callback
    \item Variable Datasource
\end{itemize}

Die zyklische Aktualisierung eines oder mehrerer Werte nimmt, abhängig von der Zykluszeit, viele Systemressourcen in Anspruch. Value Callbacks ermöglichen es, einen Variablenwert effizienter mit einer Ressource wie etwa einem Prozessabbild zu synchronisieren. An die Variable wird ein Callback angehängt, welches vor jedem Lesen und nach jedem Schreibvorgang ausgeführt wird.
Der Wert der Variablen wird weiterhin im Variablenknoten auf dem OPC-Server gespeichert, der Abgleich mit der verknüpften Ressource erfolgt durch die Callback-Methoden.

Sogenannte Datenquellen gehen noch einen Schritt weiter. Der Server leitet jede Lese- und Schreibanforderung direkt an eine Callback-Funktion weiter. Beim Lesen liefert der Rückruf eine Kopie des aktuellen Wertes. Die Datenquelle muss intern ein eigenes Speichermanagement implementieren.

Der Zugriff auf die Werte des Prozessabbildes erfolgt, wie in Abschnitt~\ref{sec:3-anbindung} beschrieben, über von piControl bereitgestellte Methoden. Um die durch open62541 gepflegte OPC-Datenstruktur und das durch piControl verwaltete Prozessabbild möglichst effektiv verknüpfen zu können, soll diese Interaktion mittels Datenquellen und den zugehörigen Callbacks implementiert werden.
% % % Imports nur für Referenzenauflösung während des Schreibens! Vorm Kompilieren auskommentieren!
% \bibliography{0_hauptdatei}
% \input{1_einleitung}
% \input{2_grundlagen}
% \input{3_konzeption}
% \input{4_implementierung}
% \input{5_tests}
% \input{6_zusammenfassung}
% \input{anhang}
% % Ende Imports

\section{Implementierung%
  \label{sec:4-implementierung}}
Das folgende Kapitel stellt in Auszügen die Implementierung des OPC-Servers sowie die Anbindung an die IO-Module
der SPS dar. Der Schwerpunkt liegt hierbei auf der Funktionsweise des piControl-Treibers und dessen Integration in das Projekt. Abschnitt~\ref{sec:4-picontrol} erklärt die zum Schreibens eines Bits verwendeten Funktionsaufrufe.
Zuvor soll jedoch in Abschnitt~\ref{sec:4-open62541} der Teil des OPC-Servers vorgestellt werden, welcher auf besagten Treiber zugreift. 

\subsection{Implementierung des OPC-Servers%
     \label{sec:4-open62541}}
Wie im vorangegangenen Abschnitt~\ref{sec:3-integration} begründet, soll die Verknüpfung zwischen dem Prozessabbild der SPS und den auf dem OPC-Server bereitgestellten Werten über sog.\,Datenquellen erfolgen. Hierzu ist zunächst eine Callback-Methode zu implementieren, welche bei einem Lese- oder Schreibzugriff auf eine Variable aufgerufen wird. Die Verknüpfung zwischen Callback-Methode und Variable muss manuell erfolgen.

\begin{lstlisting}[language={c},firstnumber=237,caption={Auszug der Methode \lstinline{linkDataSourceVariable} in \lstinline{variables.c}\label{lst:4-linkDataSourceVariable}}]
extern UA_StatusCode
 linkDataSourceVariable(UA_Server *server, UA_NodeId nodeId) {
     bool readonly = false;
     UA_DataSource dataSourceVariable;
     UA_StatusCode rc; |>\setcounter{lstnumber}{254}<|

     dataSourceVariable.read = readDataSourceVariable;
     if (!readonly)
        dataSourceVariable.write = writeDataSourceVariable;
     else
        dataSourceVariable.write = writeReadonlyDataSourceVariable;

     return UA_Server_setVariableNode_dataSource(server, nodeId, dataSourceVariable);
 }
\end{lstlisting}

\begin{figure}[h]
    \centering
    \includegraphics[width=0.42\textwidth]{doc/img/OPC_RevPiDO.pdf}
    \caption{Auszug des verwendeten Nodesets, hier Digitalausgang 1 des Versuchsaufbaus
      \label{fig:opc-do}}
\end{figure}

Die in Listing~\ref{lst:4-linkDataSourceVariable} abgebildete Methode \lstinline{linkDataSourceVariable()} erzeugt ein Struct vom Typ \lstinline{UA_DataSource}. In diesem werden dem Lesen und Schreiben einer OPC-Variablen entsprechende Callback-Methoden zugewiesen. Die Verknüpfung einer OPC-Variable, genauer ihrer NodeId, mit der zuvor definierten Datenquelle erfolgt über die von open62541 bereitgestellte Methode \lstinline{UA_Server_setVariableNode_dataSource()}. Vor dem Lesen und nach dem Schreiben dieser Variable werden von nun an die entsprechenden Callbacks aufgerufen.
     
\begin{lstlisting}[language={c},firstnumber=168,caption={Auszug des Callbacks \lstinline{writeDataSourceVariable} in \lstinline{variables.c}\label{lst:4-writeDataSourceVariable}}]  
extern UA_StatusCode
 writeDataSourceVariable(UA_Server *server,
            const UA_NodeId *sessionId, void *sessionContext,
            const UA_NodeId *nodeId, void *nodeContext,
            const UA_NumericRange *range, const UA_DataValue *dataValue) {

    UA_StatusCode retval  = UA_STATUSCODE_GOOD;
    UA_NodeId *nameNodeId = UA_malloc(sizeof(UA_NodeId));
    UA_QualifiedName nameQN = UA_QUALIFIEDNAME(1, "Name");
    UA_Variant nameVar;
    UA_Boolean bit;

    retval |= findSiblingByBrowsename(server, nodeId, &nameQN, nameNodeId);
    retval |= UA_Server_readValue(server, *nameNodeId, &nameVar);
    retval |= UA_Boolean_copy(dataValue->value.data, &bit);

    |>\tikzmarkin[set border color=martinired]{writeIO}<|PI_writeSingleIO(String_fromUA_String(nameVar.data), &bit, false);                                                 |>\tikzmarkend{writeIO}<|

    free(nameNodeId);
    return retval;
 }
\end{lstlisting}

Listing~\ref{lst:4-writeDataSourceVariable} zeigt die Callback-Methode, welche nach dem Schreiben einer Variablen auf dem OPC-Server aufgerufen wird.
Dieser Methode wird neben der NodeId der mit ihr verknüpften Variablen auch der Wert dieser in Form eines Zeigers auf ein Struct vom Typ \lstinline{UA_DataValue} übergeben.

Die Gestaltung des hier verwendeten Nodesets sieht vor, dass in einer OPC-Variablen \lstinline{"Name"} der Bezeichner des zu schreibenden Digitalausgangs hinterlegt ist, siehe Abbildung~\ref{fig:opc-do}. Dies erlaubt eine Rekonfiguration der Ein- und Ausgänge der SPS ohne Änderungen im Programmcode des OPC-Servers vornehmen zu müssen.
Es ist daher erforderlich, nach jedem Schreiben einer mit einem Digitalausgang verknüpften Variablen, hier \lstinline{"Value"}, dessen Bezeichner \lstinline{"Name"} abzufragen. 
Dies geschieht in den Zeilen 180 und 181.
Anschließend wird dieser Bezeichner sowie der zu schreibende Wert der Methode \lstinline{PI_writeSingleIO()} übergeben, welche wiederum die Interaktion mit piControl übernimmt (vgl. Abschnitt \ref{sec:4-picontrol}).
 
\subsection{Integration von piControl%
     \label{sec:4-picontrol}}
In Abschnitt~\ref{sec:2-io} wurde die Anbindung der IO-Module des Revolution Pi sowie die Funktionsweise von piControl aus Anwendersicht beschrieben. Die verfügbare Literatur beschränkt sich auch auf lediglich diese Sicht; eine weiterführende Dokumentation für Entwickler gibt es, neben der in Abschnitt~\ref{sec:3-anbindung} vorgestellten Manpage, nicht. 
In diesem Abschnitt soll daher der Quellcode von piControl sowie dessen Verwendung im Projekt genauer betrachtet werden.
Hierzu wird exemplarisch die in Abschnitt~\ref{sec:4-open62541} eingeführte Methode \lstinline{PI_writeSingleIO()} untersucht.
Diese Methode ermöglicht das Setzen eines einzelnen Bits im Prozessabbild der SPS, und damit das Schalten eines digitalen Ausgangs auf einem IO-Modul.
Die äquivalente Methode \lstinline{int piControlGetBitValue(SPIValue *pSpiValue)} zum Lesen eines Bits bzw. Eingangs funktioniert analog und soll daher an dieser Stelle nicht dediziert erörtert werden.

\begin{lstlisting}[language={c},firstnumber=97,
                   caption={Setzen eines phsikalischen, digitalen Ausgangs in \lstinline{revpi.c}
                   \label{lst:4-PI_writeSingleIO}}]
extern void PI_writeSingleIO(char *pszVariableName, bool *bit, bool verbose)
{
	int rc;
	SPIVariable sPiVariable;
	SPIValue sPIValue;

	strncpy(sPiVariable.strVarName, pszVariableName, sizeof(sPiVariable.strVarName));
	rc = piControlGetVariableInfo(&sPiVariable);
	if (rc < 0) {
		printf("Cannot find variable '%s'\n", pszVariableName);
		return;
	}

		sPIValue.i16uAddress = sPiVariable.i16uAddress;
		sPIValue.i8uBit = sPiVariable.i8uBit;
		sPIValue.i8uValue = *bit;
		rc = |>\tikzmarkin[set border color=martinired]{setBitValue}<|piControlSetBitValue(&sPIValue)|>\tikzmarkend{setBitValue}<|;
		if (rc < 0)
			printf("Set bit error %s\n", getWriteError(rc));
		else if (verbose)
			printf("Set bit %d on byte at offset %d. Value %d\n", sPIValue.i8uBit, sPIValue.i16uAddress,
			       sPIValue.i8uValue);
}
\end{lstlisting}

Der Programmcode in Listing~\ref{lst:4-PI_writeSingleIO} ist Teil des implementierten OPC-Servers. In diesem wird auf zwei Funktionen des piControl-Treibers zugegriffen. 
Beiden Methoden wird als Argument ein Zeiger auf ein Struct vom Typ \lstinline{SPIValue} übergeben. Der im Struct abgelegte Name wird mittels \lstinline{piControlGetVariableInfo(&sPIValue)} zu einer Adresse im Prozessabbild aufgelöst. Diese wird in \lstinline{sPIValue.i16uAdress} gespeichert. Der Wert der Variablen wird anschließend mittels \lstinline{piControlSetBitValue(&sPIValue)} an dieser Adresse in das Prozessabbild geschrieben.

\begin{lstlisting}[language={c},firstnumber=309,caption={Methode \lstinline{piControlSetBitValue} in \lstinline{piControlIf.c}\label{lst:4-piControlSetBitValue}}]
int |>\tikzmarkin[set border color=martiniblue]{setBitValueFcn}<|piControlSetBitValue(SPIValue *pSpiValue)|>\tikzmarkend{setBitValueFcn}<|
{
    piControlOpen();

    if (PiControlHandle_g < 0)
	    return -ENODEV;

    pSpiValue->i16uAddress += pSpiValue->i8uBit / 8;
    pSpiValue->i8uBit %= 8;

    if (|>\tikzmarkin[set border color=martinired]{ioctl}<|ioctl(PiControlHandle_g, KB_SET_VALUE, pSpiValue)|>\tikzmarkend{ioctl}<| < 0)
	    return errno;

    return 0;
}
\end{lstlisting}

Die in Listing~\ref{lst:4-piControlSetBitValue} dargestellte Methode \lstinline{piControlSetBitValue} ist lediglich eine Hüllfunktion (häufig auch als Wrapper-Funktion bezeichnet) für einen Aufruf des \lstinline{ioctl} Kernel-Moduls.
Folgende Parameter werden übergeben:
\lstinline{PiControlHandle_g} ist die Referenz auf die Geräte-Datei des piControl-Treibers. \lstinline{KB_SET_VALUE} ist das ioctl-Kommando zum Schreiben eines Bits in das Prozessabbild. Der Zeiger \lstinline{pSpiValue} verweist auf ein Struct des bereits vorgestellten Typs \lstinline{SPIValue}.

\begin{lstlisting}[language={c},firstnumber=80,caption={Methode \lstinline{piControlOpen} in \lstinline{piControlIf.c}\label{lst:4-piControlOpen}}]
void piControlOpen(void)
{
    /* open handle if needed */
    if (PiControlHandle_g < 0)
    {
	    |>\tikzmarkin[set border color=martiniblue]{PiControlHandle}<|PiControlHandle_g = open(PICONTROL_DEVICE, O_RDWR)|>\tikzmarkend{PiControlHandle}<|;
    }
}
\end{lstlisting}

Die in Listing~\ref{lst:4-piControlOpen} dargestellte Methode öffnet, sofern nicht bereits geschehen, die Geräte-Datei. Das Macro \lstinline{PICONTROL_DEVICE} verweist hierbei auf \lstinline{/dev/piControl0}.

\begin{lstlisting}[language={c},firstnumber=721,caption={Methode \lstinline{piControlIoctl} in \lstinline{piControlMain.c}\label{lst:4-piControlIoctl}}]
static long |>\tikzmarkin[set border color=martiniblue, below offset=0.9em]{piControlIoctl}<|piControlIoctl(struct file *file, unsigned int prg_nr, 
                           unsigned long usr_addr)                                      |>\tikzmarkend{piControlIoctl}<|
{
  int status = -EFAULT;
  tpiControlInst *priv;
  int timeout = 10000;	// ms

  if (prg_nr != KB_CONFIG_SEND && prg_nr != KB_CONFIG_START && !isRunning()) {
  	return -EAGAIN;
  }

  priv = (tpiControlInst *) file->private_data;

  if (prg_nr != KB_GET_LAST_MESSAGE) {
  	// clear old message
  	priv->pcErrorMessage[0] = 0;
  }

  switch (prg_nr) {|>\setcounter{lstnumber}{864}<|

    case |>\tikzmarkin[set border color=martiniblue]{KB_SET_VALUE}<|KB_SET_VALUE:|>\tikzmarkend{KB_SET_VALUE}<|
  		{
  			SPIValue *pValue = (SPIValue *) usr_addr;

  			if (!isRunning())
  				return -EFAULT;

  			if (pValue->i16uAddress >= KB_PI_LEN) {
  				status = -EFAULT;
  			} else {
  				INT8U i8uValue_l;
  				my_rt_mutex_lock(&piDev_g.lockPI);
  				i8uValue_l = piDev_g.ai8uPI[pValue->i16uAddress];

  				if (pValue->i8uBit >= 8) {
  					i8uValue_l = pValue->i8uValue;
  				} else {
  					if (pValue->i8uValue)
  						i8uValue_l |= (1 << pValue->i8uBit);
  					else
  						i8uValue_l &= ~(1 << pValue->i8uBit);
  				}

  				|>\tikzmarkin[set border color=martinired]{i8uValue}<|piDev_g.ai8uPI[pValue->i16uAddress] = i8uValue_l;|>\tikzmarkend{i8uValue}<|
  				rt_mutex_unlock(&piDev_g.lockPI);

  #ifdef VERBOSE
  				pr_info("piControlIoctl Addr=%u, bit=%u: %02x %02x\n", pValue->i16uAddress, pValue->i8uBit, pValue->i8uValue, i8uValue_l);
  #endif

  				status = 0;
  			}
  		}
  		break; |>\setcounter{lstnumber}{1314}<|

    default:
      pr_err("Invalid Ioctl");
      return (-EINVAL);
      break;

    }

    return status;
  }
\end{lstlisting}

Listing~\ref{lst:4-piControlIoctl} zeigt in Auszügen die ioctl-Methode des piControl Kernel-Treibers. Diese bekommt folgende Argumente übergeben: \lstinline{struct file *file} enthält den Verweis auf die Geräte-Datei, hier \lstinline{/dev/piControl0}. Der Wert von \lstinline{unsigned int prg_nr} beschreibt die Anfrage an den Treiber, in diesem Fall \lstinline{KB_SET_VALUE}. Das Argument \lstinline{unsigned long usr_addr} enthält einen typ-agnostischen Pointer. Dieser verweist auf einen Speicherbereich, in welchem die zur Bearbeitung der Anfrage notwendigen Daten abgelegt sind. Hier können auch vom Treiber empfangene Daten dem Anwendungsprogramm bereitgestellt werden. 

Die switch-case-Anweisung führt die über das Argument \lstinline{prg_nr} spezifizierte Aktion aus. Hier betrachten wir \lstinline{KB_SET_VALUE}:
Zunächst wird in Zeile 868 der übergebene Zeiger \lstinline{usr_addr} mittels explizitem Typecast zu einem Zeiger des Typs \lstinline{SPIValue *} konvertiert. Da dieser auf Daten im Userspace verweist, ist beim Zugriff durch den Kernel-Treiber besondere Vorsicht geboten.
In Zeile 877 wird mittels Mutex das Prozessabbild \lstinline{piDev_g} für den Zugriff durch andere Threads oder Prozesse gesperrt.
\lstinline{my_rt_mutex_lock} verweist hierbei auf die Funktion \lstinline{rt_mutex_lock} aus \lstinline{linux/sched.h}\footnote{Offenbar wurde hier auch eine alternative Implementierung vorgesehen, siehe revpi\_common.h}

In Zeile 889 wird das Byte \lstinline{i8uValue_l}, welches den zu schreibenden Wert enthält in das Prozessabbild übertragen. Anschließend wird die Mutex auf \lstinline{piDev_g} wieder entsperrt.
\newpage

\begin{lstlisting}[language={c},firstnumber=62,caption={Auszug des Struct \lstinline{spiControlDev} in \lstinline{piControlMain.h}\label{lst:4-spiControlDev}}]
|>\tikzmarkin[set border color=martiniblue]{spiControlDev}<|typedef struct spiControlDev|>\tikzmarkend{spiControlDev}<| {
	// device driver stuff
	int init_step;
	enum revpi_machine machine_type;
	void *machine;
	struct cdev cdev;	// Char device structure
	struct device *dev;
	struct thermal_zone_device *thermal_zone;

	|>\tikzmarkin[set border color=martiniblue]{processImage}<|// process image stuff
	INT8U ai8uPI[KB_PI_LEN];
	INT8U ai8uPIDefault|>\tikzmarkin[set border color=martinired]{KB_PI_LEN_0}<|[KB_PI_LEN]|>\tikzmarkend{KB_PI_LEN_0}<|;
	struct rt_mutex lockPI;        |>\tikzmarkend{processImage}<|
	bool stopIO;
	piDevices *devs; |>\setcounter{lstnumber}{94}<|
} tpiControlDev;
\end{lstlisting}

Das Prozessabbild ist als Byte-Array der Länge \lstinline{KB_PI_LEN} in Listing~\ref{lst:4-spiControlDev} definiert. Konfigurationsparameter wie \lstinline{KB_PI_LEN} oder die Zykluszeit für den Datenaustausch zwischen SPS und IO-Modulen sind im folgenden Listing~\ref{lst:4-process} definiert.

\begin{lstlisting}[language={c},firstnumber=119,caption={Konfigurationsparameter des Prozessabbildes in project.h\label{lst:4-process}}]
#define INTERVAL_PI_GATE (5*1000*1000)  // 5 ms piGateCommunication |>\setcounter{lstnumber}{128}<|

#define INTERVAL_IO_COM (5*1000*1000)  // 5 ms piIoComm |>\setcounter{lstnumber}{132}<|

#define KB_PD_LEN       512
|>\tikzmarkin[set border color=martiniblue]{KB_PI_LEN_1}<|#define KB_PI_LEN       4096|>\tikzmarkend{KB_PI_LEN_1}<|
\end{lstlisting}

Das zu setzende Bit wurde zu diesem Zeitpunkt erfolgreich in das Prozessabbild der SPS geschrieben.
Es stellt sich die Frage, wie dieses nun an das IO-Modul kommuniziert wird.
Die Kommunikation mit allen angebundenen Modulen ist ebenfalls Aufgabe des piControl-Treibers.

\begin{lstlisting}[language={c},firstnumber=256,caption={Auszug der Methode \lstinline{piIoThread} in \lstinline{revpi_core.c}\label{lst:4-piIoThread}}]
static int piIoThread(void *data)
{
	//TODO int value = 0;
	ktime_t time;
	ktime_t now;
	s64 tDiff;

	hrtimer_init(&piCore_g.ioTimer, CLOCK_MONOTONIC, HRTIMER_MODE_ABS);
	piCore_g.ioTimer.function = piIoTimer;

	pr_info("piIO thread started\n");

	now = hrtimer_cb_get_time(&piCore_g.ioTimer);

	PiBridgeMaster_Reset();

	while (!kthread_should_stop()) {
		if (|>\tikzmarkin[set border color=martinired]{PiBridgeMaster}<|PiBridgeMaster_Run()|>\tikzmarkend{PiBridgeMaster}<| < 0)
			break;
	}

	RevPiDevice_finish();

	pr_info("piIO exit\n");
	return 0;
}
\end{lstlisting}

Der Kernel-Thread \lstinline{piIoThread} ist verantwortlich für den zyklischen Datenaustausch mit den IO-Modulen. In diesem wird fortlaufend die Methode \lstinline{PiBridgeMaster_Run()} aufgerufen, siehe Listing~\ref{lst:4-piIoThread}.

\begin{lstlisting}[language={c},firstnumber=262,caption={Auszug der Methode \lstinline{PiBridgeMaster_Run(void)} in \lstinline{RevPiDevice.c}\label{lst:4-PiBridgeMaster_Run}}]
int PiBridgeMaster_Run(void)
{
	static kbUT_Timer tTimeoutTimer_s;
	static kbUT_Timer tConfigTimeoutTimer_s;
	static int error_cnt;
	static INT8U last_led;
	static unsigned long last_update;
	int ret = 0;
	int i;

	my_rt_mutex_lock(&piCore_g.lockBridgeState);
	if (piCore_g.eBridgeState != piBridgeStop) {
		switch (eRunStatus_s) { |>\setcounter{lstnumber}{514}<|
		    case enPiBridgeMasterStatus_EndOfConfig:|>\setcounter{lstnumber}{621}<|
		    if (|>\tikzmarkin[set border color=martinired]{RevPiDevice}<|RevPiDevice_run()|>\tikzmarkend{RevPiDevice}<|) {
				// an error occured, check error limits |>\setcounter{lstnumber}{641}<|
			} else {
				ret = 1;
			}
			piCore_g.image.drv.i16uRS485ErrorCnt = RevPiDevice_getErrCnt();
			break;
\end{lstlisting}

Die in Listing~\ref{lst:4-PiBridgeMaster_Run} dargestellte Methode ist eine sog. State-Machine. Ist die Konfiguration der IO-Module erfolgreich abgeschlossen, so führt sie bei Aufruf lediglich die Methode \lstinline{RevPiDevice_run()} aus.

\begin{lstlisting}[language={c},firstnumber=140,caption={Auszug der Methode \lstinline{RevPiDevice_run(void)} in \lstinline{RevPiDevice.c}\label{lst:4-RevPiDevice_run}}]
int RevPiDevice_run(void)
{
	INT8U i8uDevice = 0;
	INT32U r;
	int retval = 0;

	RevPiDevices_s.i16uErrorCnt = 0;

	for (i8uDevice = 0; i8uDevice < RevPiDevice_getDevCnt(); i8uDevice++) {
		if (RevPiDevice_getDev(i8uDevice)->i8uActive) {
			switch (RevPiDevice_getDev(i8uDevice)->sId.i16uModulType) {
			case KUNBUS_FW_DESCR_TYP_PI_DIO_14:
			case KUNBUS_FW_DESCR_TYP_PI_DI_16:
			case KUNBUS_FW_DESCR_TYP_PI_DO_16:
				r = |>\tikzmarkin[set border color=martinired]{sendCyclicTelegram}<|piDIOComm_sendCyclicTelegram(i8uDevice)|>\tikzmarkend{sendCyclicTelegram}\setcounter{lstnumber}{166} <|;

				break; |>\setcounter{lstnumber}{216}<|
			}
		}
	} |>\setcounter{lstnumber}{227}<|
	return retval;
}
\end{lstlisting}

Diese iteriert wie in Listing~\ref{lst:4-RevPiDevice_run} abgebildete durch alle gegenwärtig in der SPS konfigurierten Module. Ist das aktuelle Modul als aktiv markiert, so wird anhand eines sog. Firmware-Descriptors entschieden, welche Methode für die Ansteuerung des Moduls aufzurufen ist.

\begin{lstlisting}[language={c},firstnumber=161,caption={Auszug der Methode \lstinline{piDIOComm_sendCyclicTelegram} in \lstinline{piDIOComm.c}\label{lst:4-sendCyclicTelegram}}]
INT32U piDIOComm_sendCyclicTelegram(INT8U i8uDevice_p)
{
	INT32U i32uRv_l = 0;
	SIOGeneric sRequest_l;
	SIOGeneric sResponse_l;
	INT8U len_l, data_out[18], i, p, data_in[70];
	INT8U i8uAddress;
	int ret; |>\setcounter{lstnumber}{239}<|
	
    |>\tikzmarkin[set border color=martinired]{piIoComm}<|ret = piIoComm_send((INT8U *) & sRequest_l, IOPROTOCOL_HEADER_LENGTH + len_l + 1);  |>\tikzmarkend{piIoComm}\setcounter{lstnumber}{298}<|
}
\end{lstlisting}

Im Falle des hier verwendeten DO-Moduls wird die in Listing~\ref{lst:4-sendCyclicTelegram} abgebildete Methode \lstinline{piDIOComm_sendCyclicTelegram()} aufgerufen. Dieser wird ein Zeiger auf das zu schreibende Gerät übergeben. 
Zunächst wird das Prozessabbild mittels eines proprietären, jedoch im Quellcode offen nachvollziehbaren Protokolls in ein \lstinline{sRequest_l} genanntes Byte-Array umgewandelt. Dieser Schritt ist in Listing~\ref{lst:4-sendCyclicTelegram} nicht abgebildet. Anschließend wird \lstinline{piIoComm_send()} ein Zeiger auf die so generierte Schreib-Anfrage übergeben.

\begin{lstlisting}[language={c},firstnumber=220,caption={Auszug der Methode \lstinline{piIOComm_send} in \lstinline{piIOComm.c}\label{lst:4-piIOComm_send}}]
int piIoComm_send(INT8U * buf_p, INT16U i16uLen_p)
{
	ssize_t write_l = 0;
	INT16U i16uSent_l = 0;|>\setcounter{lstnumber}{249}<|

	while (i16uSent_l < i16uLen_p) {
		write_l = vfs_write(piIoComm_fd_m, buf_p + i16uSent_l, i16uLen_p - i16uSent_l, &piIoComm_fd_m->f_pos);
		if (write_l < 0) {
			pr_info_serial("write error %d\n", (int)write_l);
			return -1;
		} 
		i16uSent_l += write_l;|>\setcounter{lstnumber}{263}<|
	}
	clear();
	vfs_fsync(piIoComm_fd_m, 1);
	return 0;
}
\end{lstlisting}

Listing~\ref{lst:4-piIOComm_send} zeigt die Implementierung von \lstinline{piIoComm_send()}. Diese Methode ist für das Schreiben der oben generierten Anfrage auf die seriellen Schnittstelle verantwortlich. Realisiert wird dies mittels der Methode \lstinline{vfs_write()}. Diese ist in \lstinline{<linux/fs.h>} definiert. Sie ermöglicht das Schreiben einer Datei im Userspace aus dem Kernel heraus. Geschrieben wird hier die Datei mit dem Deskriptor \lstinline{piIoComm_fd_m}.
Da die Funktion \lstinline{vfs_write()} durch andere Kernel-Tasks unterbrochen werden kann, ist nicht gewährleistet, dass die gesamte Anfrage mit nur einem Aufruf geschrieben wird. Die oben abgebildete while-Schleife stellt das vollständige Senden der Anfrage sicher.

\begin{lstlisting}[language={c},firstnumber=157,caption={Auszug der Methode \lstinline{piIOComm_open_serial} in \lstinline{piIOComm.c}\label{lst:4-piIOComm_open_serial}}]
int piIoComm_open_serial(void)
{   |>\setcounter{lstnumber}{167}<|
	struct file *fd;	/* Filedeskriptor */
	struct termios newtio;	/* Schnittstellenoptionen */

	|>\tikzmarkin[set border color=martiniblue]{fd}<|/* Port oeffnen - read/write, kein "controlling tty", 
	    Status von DCD ignorieren */
	fd = filp_open(|>\tikzmarkin[set border color=martinired]{tty}<|REV_PI_TTY_DEVICE|>\tikzmarkend{tty}<|, O_RDWR | O_NOCTTY, 0); |>\setcounter{lstnumber}{208}<|
	
	piIoComm_fd_m = fd;                                                      |>\tikzmarkend{fd}\setcounter{lstnumber}{217}<|

	return 0;
}
\end{lstlisting}

Der zum Schreiben auf die serielle Schnittstelle verwendete Datei-Deskriptor wird von der in Listing~\ref{lst:4-piIOComm_open_serial} abgebildeten Methode \lstinline{piIoComm_open_serial()} generiert. 

\begin{lstlisting}[language={c},firstnumber=45,caption={Definition der seriellen Schnittstelle in \lstinline{piIOComm.h}\label{lst:4-REV_PI_TTY_DEVICE}}]
#define REV_PI_TTY_DEVICE	"/dev/ttyAMA0"
\end{lstlisting}

Das in Listing~\ref{lst:4-REV_PI_TTY_DEVICE} definierte Macro verweist auf eine der seriellen Schnittstellen des RaspberryPi.
Die Implementierung des zugehörigen Schnittstellentreibers soll hier nicht weiter untersucht werden. Somit ist an dieser Stelle die Kette vom Setzen einer Variablen auf dem OPC-Server bis hin zur Aktualisierung des Prozessabbilds der IO-Module geschlossen.

% \begin{lstlisting}[language={c},firstnumber={226},caption={Setzen der Scheduler-Priorität auf SCHED\_FIFO in 
% revpi\_common.c\label{lst:2-sched_priority}}]
% param.sched_priority = ktprio->prio;
% ret = sched_setscheduler(child, SCHED_FIFO, &param);
% \end{lstlisting}
% % % Imports nur für Referenzenauflösung während des Schreibens! Vorm Kompilieren auskommentieren!
% \bibliography{0_hauptdatei}
% \input{1_einleitung}
% \input{2_grundlagen}
% \input{3_konzeption}
% \input{4_implementierung}
% \input{5_tests}
% \input{6_zusammenfassung}
% % Ende Imports

\section{Test des OPC-Servers im Gesamtsystem%
  \label{sec:5-tests}}

% % % Imports nur für Referenzenauflösung während des schreibens! Vorm Kompilieren auskommentieren!
% \bibliography{0_hauptdatei}
% \input{1_einleitung}
% \input{2_grundlagen}
% \input{3_konzeption}
% \input{4_implementierung}
% \input{5_tests}
% \input{6_zusammenfassung}
% % Ende Imports

\section{Zusammenfassung und Ausblick%
  \label{sec:6-fazit}}
Der folgende Abschnitt~\ref{sec:6-zusammenfassung} fasst die gewonnenen Erkenntnisse und den Stand der Implementierung zusammen.
Den Abschluss dieser Arbeit bildet der Ausblick in Abschnitt~\ref{sec:6-ausblick}.

\subsection{Zusammenfassung%
     \label{sec:6-zusammenfassung}}

\subsection{Ausblick%
     \label{sec:6-ausblick}}

% % Ende Imports

\section{Grundlagen%
  \label{sec:2-grundlagen}}

\subsection{Speicherprogrammierbare-Steuerung und Linux -- Revolution Pi%
     \label{sec:2-sps}}

\subsubsection{Kunbus RevolutionPi%
        \label{sec:2-revpi}}
Der RevolutionPi 3 ist eine speicherprogrammierbare Steuerung (SPS) des Herstellers
Kunbus GmbH. Kern dieser SPS ist das von der Raspberry Pi Foundation entwickelte
und vertriebene Raspberry Pi Compute Module 3. Dieses integriert ein Broadcom BCM2837
System-on-Chip (SoC) mit vier 1,2GHz Prozessorkernen, 1GB RAM, 4GB eMMC Anwendungsspeicher
und sonstige Peripherie in ein Modul im DDR2-SODIMM Formfaktor. Diese Spezifikationen
sind weitgehend identisch zu denen des ausgesprochen populären Raspberry Pi 3.
Der Revolution Pi profitiert daher von dem gleichen großen Angebot an Software
und Unterstützung wie der Raspberry Pi, ergänzt dessen Hardware jedoch um eine 24V
Spannungsversorgung, die Möglichkeit der Erweiterung durch mehrere industrietaugliche
Ein-/ Ausgabemodule und Gateways sowie ein Gehäuse zur Montage auf einer DIN-Schiene.
\begin{itemize}
  \item{Prozessor: BCM2837}
  \item{Taktfrequenz 1,2 GHz}
  \item{Anzahl Prozessorkerne: 4}
  \item{Arbeitsspeicher: 1 GByte}
  \item{eMMC Flash Speicher: 4 GByte}
  \item{Betriebssystem: Angepasstes Raspbian mit RT-Patch}
  \item{RTC mit 24h Pufferung über wartungsfreien Kondensator}
  \item{Treiber / API: Treiber schreibt zyklisch Prozessdaten in ein Prozessabbild, Zugriff auf Prozessabbild über Linux-Filesystem als API zu Fremdsoftware.}
  \item{Kommunikationsanschlüsse: 2 x USB 2.0 A (je 500 mA belastbar), 1 x Micro-USB, HDMI, Ethernet (RJ45) 10/100 Mbit/s}
  \item{Stromversorgung: min. 10,7 V, max. 28,8 V, maximal 10 Watt}
  \item{Zulässige Umgebungstemperatur: -40 bis +55 C}
  \item{Gehäuseabmessungen: (HxBxL) 96 mm x 22,5 mm x 110,5 mm (ohne gesteckte Stecker)}
  \item{ESD Schutz: 4 kV / 8 kV gemäß EN61131-2 und IEC 61000-6-2}
  \item{Surge / Burst Prüfungen: gemäß EN61131-2 und IEC 61000-6-2 eingekoppelt auf Versorgungsspannung, Ethernet und IO-Leitungen}
  \item{EMI Prüfungen: gemäß EN61131-2 und IEC 61000-6-2}
\end{itemize}

Kunbus bietet eine Auswahl an IO- und Gateway-Modulen zur Erweiterung des Revolution Pi an.
Gateways dienen der Kommunikation mit Systemen oder Komponenten der Automatisierungstechnik
über Protokolle wie PROFIBUS oder EtherCAT. IO-Module erlauben die Überwachung
und Steuerung von digitalen oder analogen Ein- und Ausgängen.

\subsubsection{Zugriff auf IO-Module%
        \label{sec:2-io}}
Der Zugriff auf die Ein- und Ausgänge der IO-Module erfolgt über ein Prozessabbild
und einen hierfür von Kunbus bereitgestellten Treiber, genannt piControl. Dieser
aktualisiert das Prozessabbild zyklisch. Die angestrebte Zykluszeit beträgt 5ms,
kann jedoch je nach Anzahl der angeschlossenen Module auch größer sein. Kunbus
garantiert bei drei IO-Modulen und zwei Gateway-Modulen eine Zykluszeit von 10 ms.
Jedes der IO-Module stellt ein eigenständiges eingebettetes System dar. Es verfügt
über einen Microcontroller, welcher die IOs bereitstellt und über einen RS485-Bus
mit dem Revolution Pi kommuniziert.
% https://revolution.kunbus.de/io-modul/

Lizenz: GPL
% https://github.com/RevolutionPi/piControl

\begin{lstlisting}[language={c},firstnumber={226},caption={Setzen der Scheduler-Priorität auf SCHED\_FIFO in revpi\_common.c\label{lst:2-sched_priority}}]
param.sched_priority = ktprio->prio;
ret = sched_setscheduler(child, SCHED_FIFO,
       &param);
\end{lstlisting}


\subsection{Echtzeit und Multithreading unter Linux -- preemptRT und posix%
     \label{sec:2-echtzeit}}


 Der Linux-Kernel verfügt über mehrere unterschiedliche Preemtion-Modelle:

\begin{itemize}
  \item No Forced Preemption (server):
  Ausgelegt auf maximal möglichen Durchsatz, lediglich Interrupts und
  System-Call-Returns bewirken Präemption.

  \item Voluntary Kernel Preemption (Desktop):
  Neben den implizit bevorrechtigten Interrupts und System-Call-Returns gibt es
  in diesem Modell weitere Abschnitte des Kernels in welchen Preämption explizit
  gestattet ist.

  \item Preemptible Kernel (Low-Latency Desktop):
  In diesem Modell ist der gesamte Kernel, mit Ausnahme sog.~kritischer Abschnitte
  präemptible. Nach jedem kritischen Abschnitt gibt es einen impliziten Präemptions-Punkt.

  \item Preemptible Kernel (Basic RT):
  Dieses Modell ist dem zuvor genannten sehr ähnlich, hier sind jedoch alle Interrupt-Handler
  als eigenständige Threads ausgeführt.

  \item Fully Preemptible Kernel (RT):
  Wie auch bei den beiden zuvor genannten Modellen ist hier der gesamte Kernel
  präemtible, die Anzahl und Dauer der nicht-präemtiblen kritischen Abschnitte
  ist auf ein notwendiges Minimum beschränkt. Alle Interrupt-Handler sind als
  eigenständige Threads ausgeführt, Spinlocks durch Sleeping-Spinlocks und Mutexe
  durch sog.~RT-Mutexe ersetzt.

\end{itemize}
\todo{Spinlocks und Mutexe sowie die RT-Varianten dieser erklären!}

Lediglich mit dem vollständig präemtiblen Kernel kann Echtzeit-Verhalten realisiert werden.

% https://wiki.linuxfoundation.org/realtime/documentation/technical_basics/preemption_models bzw kernel/Kconfig.preempt

\subsubsection{preemptRT%
        \label{sec:2-preemptRT}}
% https://wiki.linuxfoundation.org/realtime/documentation/technical_details/start
% https://wiki.linuxfoundation.org/realtime/documentation/technical_basics/start

Das dem PREEMPT RT Kernel zugrunde liegende Prinzip lässt sich in einer einfachen
Regel ausdrücken: Nur Code, welcher absolut nicht-präemtible sein darf, ist es
gestattet nicht-präemtible zu sein.
Das erklärte Ziel des PREEMPT\_RT Patches ist es folglich, die Menge des nicht-präemtiblen
Codes im Linux-Kernel auf das absolut notwendige Minimum zu reduzieren.

Dies wird durch Verwendung folgender Mechanismen erreicht:

\begin{itemize}
  \item Hochauflösende Timer
  \item Sleeping Spinlocks
  \item Threaded Interrupt Handlers
  \item rt\_mutex
  \item RCU
\end{itemize}


\subsubsection{posix%
        \label{sec:2-posix}}
Ist posix hier wirklich relevant? Debian bzw.~Raspbian sind weitgehend posix
kompatibel, aber wird es hier genutzt? -> JA, open62541 nutzt pthread.h
piControl nutzt kthread.h, und semaphore.h

\subsection{OPC-UA und open62541%
     \label{sec:2-opc}}

\subsubsection{OPC UA%
        \label{sec:2-opcua}}
Open Platform Communications (OPC) ist eine Familie von Standards zur herstellerunabhängigen
Kommunikation von Maschinen (M2M) in der Automatisierungstechnik. Die sog.~OPC Task Force, zu deren
Mitgliedern verschiedene große Firmen der Automatisierungsindustrie gehören, veröffentlichte
die OPC Specification Version 1.0 im August 1996.
Motiviert ist dieser offene Standard durch die Erkenntniss, dass die Anpassung der
zahlreichen Herstellerstandards an individuelle Infrastrukturen und Anlagen einen
großen Mehraufwand verursachen.
Die Wikipedia beschreibt das Anwendungsgebiet für OPC wie folgt:

\glqq{}OPC wird dort eingesetzt, wo Sensoren, Regler und Steuerungen verschiedener Hersteller
ein gemeinsames Netzwerk bilden. Ohne OPC benötigten zwei Geräte zum Datenaustausch
genaue Kenntnis über die Kommunikationsmöglichkeiten des Gegenübers. Erweiterungen
und Austausch gestalten sich entsprechend schwierig. Mit OPC genügt es, für jedes
Gerät genau einmal einen OPC-konformen Treiber zu schreiben. Idealerweise wird
dieser bereits vom Hersteller zur Verfügung gestellt. Ein OPC-Treiber lässt sich
ohne großen Anpassungsaufwand in beliebig große Steuer- und Überwachungssysteme
integrieren.

OPC unterteilt sich in verschiedene Unterstandards, die für den jeweiligen Anwendungsfall
unabhängig voneinander implementiert werden können. OPC lässt sich damit verwenden
für Echtzeitdaten (Überwachung), Datenarchivierung, Alarm-Meldungen und neuerdings
auch direkt zur Steuerung (Befehlsübermittlung).\grqq{}

OPC basiert in der ursprünglichen Spezifikation auf Microsofts DCOM-Spezifikation.
DCOM macht Funktionen und Objekte einer Anwendung anderen Anwendungen im Netzwerk
zugänglich. Der OPC-Standard definiert entsprechende DCOM-Objekte um mit anderen
OPC-Anwendungen Daten austauschen zu können. Die Verwendung von DCOM bindet Anwender
an Betriebssysteme von Microsoft. Die ursprüngliche OPC Spezifikation wird durch die
Entwicklung von OPC Unified Architecture (OPC UA) abgelöst.
OPC UA setzt auf einem eigenen Kommunikationionsstack auf, die Verwendung von DCOM
und damit die Bindung an Microsoft wurden aufgelöst.

Die OPC-UA-Architektur ist eine Service-orientierte Architektur (SOA), deren Struktur
aus mehreren Schichten besteht.

% Wikipedia
Das OPC-Informationsmodell ist nicht mehr nur eine Hierarchie aus Ordnern, Items
und Properties. Es ist ein sogenanntes Full-Mesh-Network aus Nodes, mit dem neben
den Nutzdaten eines Nodes auch Meta- und Diagnoseinformationen repräsentiert werden.
Ein Node ähnelt einem Objekt aus der objektorientierten Programmierung. Ein Node
kann Attribute besitzen, die gelesen werden können (Data Access (DA), Historical
Data Access (HDA)). Es ist möglich Methoden zu definieren und aufzurufen.
Eine Methode besitzt Aufrufargumente und Rückgabewerte. Sie wird durch ein Command
aufgerufen. Weiterhin werden Events unterstützt, die versendet werden können
(AE (Alarms \& Events), DA DataChange), um bestimmte Informationen zwischen Geräten
auszutauschen. Ein Event besitzt unter anderem einen Empfangszeitpunkt, eine Nachricht
und einen Schweregrad. Die o. g. Nodes werden sowohl für die Nutzdaten als auch
alle anderen Arten von Metadaten verwendet. Der damit modellierte OPC-Adressraum
beinhaltet nun auch ein Typmodell, mit dem sämtliche Datentypen spezifiziert werden.

% https://de.wikipedia.org/wiki/Open_Platform_Communications
% https://de.wikipedia.org/wiki/OPC_Unified_Architecture
% https://opcfoundation.org/developer-tools/specifications-unified-architecture
% Von Gerhard Gappmeier - ascolab GmbH, CC BY-SA 3.0, https://de.wikipedia.org/w/index.php?curid=1892069
\subsubsection{open62541%
        \label{sec:2-open62541}}
open62541 ist eine offene und freie Implementierung von OPC UA. Die in C geschriebene
Bibliothek stellt eine beständig zunehmende Anzahl der im OPC UA Standard definierten
Funktionen bereit. Sie kann sowohl zur Erstellung von OPC-Servern als auch -Clients
genutzt werden. Ergänzend zu der unter der Mozilla Public License v2.0 lizensierten
Bibliothek stellt das open62541 Projekt auch Beispielprogramme unter einer CC0 Lizenz
zur Verfügung.

Die Bibliothek eignet sich auch für die Entwicklung auf eingebetteten Systemen und
Microcontrollern. Je nach Umfang der gewünschten Funktionen und des OPC Informationsmodells
beträgt die Größe einer Server-Binary weniger als 100kb. %evtl. kürzen?

\todo{Nodes erklären! Evtl.~oben!}

Folgende Auswahl an Eigenschaften und Funktionen zeichnet die in dieser Arbeit verwendete
Version 0.3 von open62541 aus:
\begin{itemize}
  \item Kommunikationionsstack
  \begin{itemize}
      \item OPC UA Binär-Protokoll (HTTP oder SOAP werden gegenwärtig nicht unterstützt)
      \item Austauschbare Netzwerk-Schicht, welche die Verwendung eigener Netzwerk-APIs
      erlaubt.
      \item Verschlüsselte Kommunikationion
      \item Asynchrone Dienst-Anfragen im Client
  \end{itemize}
  \item Informationsmodell
  \begin{itemize}
    \item Unterstützung aller OPC UA Node-Typen, inkl.~Methoden
    \item Hinzufügen und Entfernen von Nodes und Referenzen zur Laufzeit.
    \item Vererbung und Instanziierung von Objekt- und Variablentypen
    \item Zugriffskontrolle auch für einzelne Nodes
  \end{itemize}
  \item Subscriptions
  \begin{itemize}
    \item Erlaubt die Überwachung (subscriptions / monitoreditems)
    \item Sehr geringer Ressourcenbedarf pro überwachtem Wert
  \end{itemize}
  \item Code-Generierung auf XML-Basis
  \begin{itemize}
    \item Erlaubt die Erstellung von Datentypen
    \item Erlaubt die Generierung des serverseitigen Informationsmodells
  \end{itemize}
\end{itemize}

% https://open62541.org/doc/0.3/


Mozilla Public License
CC0 Lizenz für Beispiele und Plugins

% https://open62541.org/doc/open62541-current.pdf
% https://open62541.org/

% % % Imports nur für Referenzenauflösung während des Schreibens! Vorm Kompilieren auskommentieren!
% \bibliography{0_hauptdatei}
% % Mit \section{...} eröffnen wir einen neuen Abschnitt.
% Der Befehl setzt nicht nur den Text in einer größeren,
% fetten Schrift, sondern sorgt außerdem dafür, daß er im
% Inhaltsverzeichnis erscheint.
%
% Mit \label{...} erzeugen wir einen Bezeichner, mit dessen Hilfe
% wir später auf die Nummer des Abschnitts verweisen können (nämlich
% mit~\ref{...}).
%
% Das Kommentarzeichen hinter „Übersicht“ dient dazu, ein
% Leerzeichen zwischen „Übersicht“ und dem \label-Befehl
% zu vermeiden, das andernfalls sichtbar würde – z.B. im
% Inhaltsverzeichnis.
%

% % Imports nur für Referenzenauflösung während des Schreibens! Vorm Kompilieren auskommentieren!
% \bibliography{0_hauptdatei}
% \input{1_einleitung}
%\input{2_grundlagen}
%\input{3_konzeption}
%\input{4_implementierung}
%\input{5_tests}
%\input{6_zusammenfassung}
% % Ende Imports

\section{Einleitung und Motivation%
  \label{sec:1-einleitung}}
Ziel dieses Projektes ist die Integration eines OPC-Servers mit einer auf Linux
basierenden speicherprogrammierbaren Steuerung (SPS). Angeschlossen an diese SPS
ist jeweils ein digitales Ein-/\,bzw.~Ausgabemodul. Die von diesen bereitgestellten
Ein-/\, bzw.~Ausgänge (IO) sollen in der Datenstruktur des OPC-Servers abgebildet
und über diesen für OPC-Clients les-/\,und schreibar sein. Weiterhin sollen einige
Funktionen zur Überwachung und Steuerung der an die SPS angeschlossenen Aktoren
und Sensoren direkt im OPC-Server implementiert werden.
Hiermit stellt dieses Projekt eine der Grundlagen für ein übergeordnetes Projekt,
die cloudbasierte Steuerung eines miniaturisierten Produktions-Systems, dar.

Der hier verwendete OPC-Server ist Teil des sog. open62541 Projekts. Er ist in C
geschrieben und implementiert bereits einen großen Teil der im OPC-UA-Standard
spezifizierten Funktionen.
Als SPS findet ein Revolution Pi 3 der Firma Kunbus Verwendung. Dieser integriert
ein sog. Compute Module der Raspberry Pi Foundation in ein industrietaugliches
Gehäuse und erlaubt die Erweiterung mittels IO- oder Gateway-Modulen. Über diese
erfolgt die Kommunikation mit weiteren Komponenten der Automatisierungstechnik.

Motiviert ist dieses Projekt durch die Beobachtung, dass die Verbreitung offener
Standards sowie freier Software auch in der Automatisierungstechnik zunimmt.
Linux ist ein freies Betriebssystem, OPC-UA ein offen zugänglicher, aktiv gepflegter
und weit verbreiteter Standard. Der Raspberry Pi findet sowohl bei Hobby-Anwendern als
auch in den Bereichen Forschung und Entwicklung sowie bei industriellen Anwendern
Verwendung. Dieses Projekt stellt somit eine für unterschiedliche Anwender interessante
Entwicklung dar.

Im Anschluss an diese einleitende Übersicht im Abschnitt~\ref{sec:1-einleitung} folgt
die Darstellung der wichtigsten Grundlagen in Abschnitt~\ref{sec:2-grundlagen}.
Aufbauend auf diesen Grundlagen folgt die konzeptuelle Ausarbeitung im Abschnitt~\ref{sec:3-konzeption}.
Die Umsetzung wird im Abschnitt~\ref{sec:4-implementierung} erläutert.
Die Leistungsfähigkeit der Implementierung wird in Abschnitt~\ref{sec:5-tests} untersucht.
Eine Zusammenfassung und ein Ausblick schließen die Arbeit in
Abschnitt~\ref{sec:6-fazit} ab. Eventuell noch benötigte Anhänge
finden sich in den Anhängen [...] bis [...].

% % % Imports nur für Referenzenauflösung während des Schreibens! Vorm Kompilieren auskommentieren!
% \bibliography{0_hauptdatei}
% \input{1_einleitung}
% \input{2_grundlagen}
% \input{3_konzeption}
% \input{4_implementierung}
% \input{5_tests}
% \input{6_zusammenfassung}
% % Ende Imports

\section{Grundlagen%
  \label{sec:2-grundlagen}}

\subsection{Speicherprogrammierbare-Steuerung und Linux -- Revolution Pi%
     \label{sec:2-sps}}

\subsubsection{Kunbus RevolutionPi%
        \label{sec:2-revpi}}
Der RevolutionPi 3 ist eine speicherprogrammierbare Steuerung (SPS) des Herstellers
Kunbus GmbH. Kern dieser SPS ist das von der Raspberry Pi Foundation entwickelte
und vertriebene Raspberry Pi Compute Module 3. Dieses integriert ein Broadcom BCM2837
System-on-Chip (SoC) mit vier 1,2GHz Prozessorkernen, 1GB RAM, 4GB eMMC Anwendungsspeicher
und sonstige Peripherie in ein Modul im DDR2-SODIMM Formfaktor. Diese Spezifikationen
sind weitgehend identisch zu denen des ausgesprochen populären Raspberry Pi 3.
Der Revolution Pi profitiert daher von dem gleichen großen Angebot an Software
und Unterstützung wie der Raspberry Pi, ergänzt dessen Hardware jedoch um eine 24V
Spannungsversorgung, die Möglichkeit der Erweiterung durch mehrere industrietaugliche
Ein-/ Ausgabemodule und Gateways sowie ein Gehäuse zur Montage auf einer DIN-Schiene.
\begin{itemize}
  \item{Prozessor: BCM2837}
  \item{Taktfrequenz 1,2 GHz}
  \item{Anzahl Prozessorkerne: 4}
  \item{Arbeitsspeicher: 1 GByte}
  \item{eMMC Flash Speicher: 4 GByte}
  \item{Betriebssystem: Angepasstes Raspbian mit RT-Patch}
  \item{RTC mit 24h Pufferung über wartungsfreien Kondensator}
  \item{Treiber / API: Treiber schreibt zyklisch Prozessdaten in ein Prozessabbild, Zugriff auf Prozessabbild über Linux-Filesystem als API zu Fremdsoftware.}
  \item{Kommunikationsanschlüsse: 2 x USB 2.0 A (je 500 mA belastbar), 1 x Micro-USB, HDMI, Ethernet (RJ45) 10/100 Mbit/s}
  \item{Stromversorgung: min. 10,7 V, max. 28,8 V, maximal 10 Watt}
  \item{Zulässige Umgebungstemperatur: -40 bis +55 C}
  \item{Gehäuseabmessungen: (HxBxL) 96 mm x 22,5 mm x 110,5 mm (ohne gesteckte Stecker)}
  \item{ESD Schutz: 4 kV / 8 kV gemäß EN61131-2 und IEC 61000-6-2}
  \item{Surge / Burst Prüfungen: gemäß EN61131-2 und IEC 61000-6-2 eingekoppelt auf Versorgungsspannung, Ethernet und IO-Leitungen}
  \item{EMI Prüfungen: gemäß EN61131-2 und IEC 61000-6-2}
\end{itemize}

Kunbus bietet eine Auswahl an IO- und Gateway-Modulen zur Erweiterung des Revolution Pi an.
Gateways dienen der Kommunikation mit Systemen oder Komponenten der Automatisierungstechnik
über Protokolle wie PROFIBUS oder EtherCAT. IO-Module erlauben die Überwachung
und Steuerung von digitalen oder analogen Ein- und Ausgängen.

\subsubsection{Zugriff auf IO-Module%
        \label{sec:2-io}}
Der Zugriff auf die Ein- und Ausgänge der IO-Module erfolgt über ein Prozessabbild
und einen hierfür von Kunbus bereitgestellten Treiber, genannt piControl. Dieser
aktualisiert das Prozessabbild zyklisch. Die angestrebte Zykluszeit beträgt 5ms,
kann jedoch je nach Anzahl der angeschlossenen Module auch größer sein. Kunbus
garantiert bei drei IO-Modulen und zwei Gateway-Modulen eine Zykluszeit von 10 ms.
Jedes der IO-Module stellt ein eigenständiges eingebettetes System dar. Es verfügt
über einen Microcontroller, welcher die IOs bereitstellt und über einen RS485-Bus
mit dem Revolution Pi kommuniziert.
% https://revolution.kunbus.de/io-modul/

Lizenz: GPL
% https://github.com/RevolutionPi/piControl

\begin{lstlisting}[language={c},firstnumber={226},caption={Setzen der Scheduler-Priorität auf SCHED\_FIFO in revpi\_common.c\label{lst:2-sched_priority}}]
param.sched_priority = ktprio->prio;
ret = sched_setscheduler(child, SCHED_FIFO,
       &param);
\end{lstlisting}


\subsection{Echtzeit und Multithreading unter Linux -- preemptRT und posix%
     \label{sec:2-echtzeit}}


 Der Linux-Kernel verfügt über mehrere unterschiedliche Preemtion-Modelle:

\begin{itemize}
  \item No Forced Preemption (server):
  Ausgelegt auf maximal möglichen Durchsatz, lediglich Interrupts und
  System-Call-Returns bewirken Präemption.

  \item Voluntary Kernel Preemption (Desktop):
  Neben den implizit bevorrechtigten Interrupts und System-Call-Returns gibt es
  in diesem Modell weitere Abschnitte des Kernels in welchen Preämption explizit
  gestattet ist.

  \item Preemptible Kernel (Low-Latency Desktop):
  In diesem Modell ist der gesamte Kernel, mit Ausnahme sog.~kritischer Abschnitte
  präemptible. Nach jedem kritischen Abschnitt gibt es einen impliziten Präemptions-Punkt.

  \item Preemptible Kernel (Basic RT):
  Dieses Modell ist dem zuvor genannten sehr ähnlich, hier sind jedoch alle Interrupt-Handler
  als eigenständige Threads ausgeführt.

  \item Fully Preemptible Kernel (RT):
  Wie auch bei den beiden zuvor genannten Modellen ist hier der gesamte Kernel
  präemtible, die Anzahl und Dauer der nicht-präemtiblen kritischen Abschnitte
  ist auf ein notwendiges Minimum beschränkt. Alle Interrupt-Handler sind als
  eigenständige Threads ausgeführt, Spinlocks durch Sleeping-Spinlocks und Mutexe
  durch sog.~RT-Mutexe ersetzt.

\end{itemize}
\todo{Spinlocks und Mutexe sowie die RT-Varianten dieser erklären!}

Lediglich mit dem vollständig präemtiblen Kernel kann Echtzeit-Verhalten realisiert werden.

% https://wiki.linuxfoundation.org/realtime/documentation/technical_basics/preemption_models bzw kernel/Kconfig.preempt

\subsubsection{preemptRT%
        \label{sec:2-preemptRT}}
% https://wiki.linuxfoundation.org/realtime/documentation/technical_details/start
% https://wiki.linuxfoundation.org/realtime/documentation/technical_basics/start

Das dem PREEMPT RT Kernel zugrunde liegende Prinzip lässt sich in einer einfachen
Regel ausdrücken: Nur Code, welcher absolut nicht-präemtible sein darf, ist es
gestattet nicht-präemtible zu sein.
Das erklärte Ziel des PREEMPT\_RT Patches ist es folglich, die Menge des nicht-präemtiblen
Codes im Linux-Kernel auf das absolut notwendige Minimum zu reduzieren.

Dies wird durch Verwendung folgender Mechanismen erreicht:

\begin{itemize}
  \item Hochauflösende Timer
  \item Sleeping Spinlocks
  \item Threaded Interrupt Handlers
  \item rt\_mutex
  \item RCU
\end{itemize}


\subsubsection{posix%
        \label{sec:2-posix}}
Ist posix hier wirklich relevant? Debian bzw.~Raspbian sind weitgehend posix
kompatibel, aber wird es hier genutzt? -> JA, open62541 nutzt pthread.h
piControl nutzt kthread.h, und semaphore.h

\subsection{OPC-UA und open62541%
     \label{sec:2-opc}}

\subsubsection{OPC UA%
        \label{sec:2-opcua}}
Open Platform Communications (OPC) ist eine Familie von Standards zur herstellerunabhängigen
Kommunikation von Maschinen (M2M) in der Automatisierungstechnik. Die sog.~OPC Task Force, zu deren
Mitgliedern verschiedene große Firmen der Automatisierungsindustrie gehören, veröffentlichte
die OPC Specification Version 1.0 im August 1996.
Motiviert ist dieser offene Standard durch die Erkenntniss, dass die Anpassung der
zahlreichen Herstellerstandards an individuelle Infrastrukturen und Anlagen einen
großen Mehraufwand verursachen.
Die Wikipedia beschreibt das Anwendungsgebiet für OPC wie folgt:

\glqq{}OPC wird dort eingesetzt, wo Sensoren, Regler und Steuerungen verschiedener Hersteller
ein gemeinsames Netzwerk bilden. Ohne OPC benötigten zwei Geräte zum Datenaustausch
genaue Kenntnis über die Kommunikationsmöglichkeiten des Gegenübers. Erweiterungen
und Austausch gestalten sich entsprechend schwierig. Mit OPC genügt es, für jedes
Gerät genau einmal einen OPC-konformen Treiber zu schreiben. Idealerweise wird
dieser bereits vom Hersteller zur Verfügung gestellt. Ein OPC-Treiber lässt sich
ohne großen Anpassungsaufwand in beliebig große Steuer- und Überwachungssysteme
integrieren.

OPC unterteilt sich in verschiedene Unterstandards, die für den jeweiligen Anwendungsfall
unabhängig voneinander implementiert werden können. OPC lässt sich damit verwenden
für Echtzeitdaten (Überwachung), Datenarchivierung, Alarm-Meldungen und neuerdings
auch direkt zur Steuerung (Befehlsübermittlung).\grqq{}

OPC basiert in der ursprünglichen Spezifikation auf Microsofts DCOM-Spezifikation.
DCOM macht Funktionen und Objekte einer Anwendung anderen Anwendungen im Netzwerk
zugänglich. Der OPC-Standard definiert entsprechende DCOM-Objekte um mit anderen
OPC-Anwendungen Daten austauschen zu können. Die Verwendung von DCOM bindet Anwender
an Betriebssysteme von Microsoft. Die ursprüngliche OPC Spezifikation wird durch die
Entwicklung von OPC Unified Architecture (OPC UA) abgelöst.
OPC UA setzt auf einem eigenen Kommunikationionsstack auf, die Verwendung von DCOM
und damit die Bindung an Microsoft wurden aufgelöst.

Die OPC-UA-Architektur ist eine Service-orientierte Architektur (SOA), deren Struktur
aus mehreren Schichten besteht.

% Wikipedia
Das OPC-Informationsmodell ist nicht mehr nur eine Hierarchie aus Ordnern, Items
und Properties. Es ist ein sogenanntes Full-Mesh-Network aus Nodes, mit dem neben
den Nutzdaten eines Nodes auch Meta- und Diagnoseinformationen repräsentiert werden.
Ein Node ähnelt einem Objekt aus der objektorientierten Programmierung. Ein Node
kann Attribute besitzen, die gelesen werden können (Data Access (DA), Historical
Data Access (HDA)). Es ist möglich Methoden zu definieren und aufzurufen.
Eine Methode besitzt Aufrufargumente und Rückgabewerte. Sie wird durch ein Command
aufgerufen. Weiterhin werden Events unterstützt, die versendet werden können
(AE (Alarms \& Events), DA DataChange), um bestimmte Informationen zwischen Geräten
auszutauschen. Ein Event besitzt unter anderem einen Empfangszeitpunkt, eine Nachricht
und einen Schweregrad. Die o. g. Nodes werden sowohl für die Nutzdaten als auch
alle anderen Arten von Metadaten verwendet. Der damit modellierte OPC-Adressraum
beinhaltet nun auch ein Typmodell, mit dem sämtliche Datentypen spezifiziert werden.

% https://de.wikipedia.org/wiki/Open_Platform_Communications
% https://de.wikipedia.org/wiki/OPC_Unified_Architecture
% https://opcfoundation.org/developer-tools/specifications-unified-architecture
% Von Gerhard Gappmeier - ascolab GmbH, CC BY-SA 3.0, https://de.wikipedia.org/w/index.php?curid=1892069
\subsubsection{open62541%
        \label{sec:2-open62541}}
open62541 ist eine offene und freie Implementierung von OPC UA. Die in C geschriebene
Bibliothek stellt eine beständig zunehmende Anzahl der im OPC UA Standard definierten
Funktionen bereit. Sie kann sowohl zur Erstellung von OPC-Servern als auch -Clients
genutzt werden. Ergänzend zu der unter der Mozilla Public License v2.0 lizensierten
Bibliothek stellt das open62541 Projekt auch Beispielprogramme unter einer CC0 Lizenz
zur Verfügung.

Die Bibliothek eignet sich auch für die Entwicklung auf eingebetteten Systemen und
Microcontrollern. Je nach Umfang der gewünschten Funktionen und des OPC Informationsmodells
beträgt die Größe einer Server-Binary weniger als 100kb. %evtl. kürzen?

\todo{Nodes erklären! Evtl.~oben!}

Folgende Auswahl an Eigenschaften und Funktionen zeichnet die in dieser Arbeit verwendete
Version 0.3 von open62541 aus:
\begin{itemize}
  \item Kommunikationionsstack
  \begin{itemize}
      \item OPC UA Binär-Protokoll (HTTP oder SOAP werden gegenwärtig nicht unterstützt)
      \item Austauschbare Netzwerk-Schicht, welche die Verwendung eigener Netzwerk-APIs
      erlaubt.
      \item Verschlüsselte Kommunikationion
      \item Asynchrone Dienst-Anfragen im Client
  \end{itemize}
  \item Informationsmodell
  \begin{itemize}
    \item Unterstützung aller OPC UA Node-Typen, inkl.~Methoden
    \item Hinzufügen und Entfernen von Nodes und Referenzen zur Laufzeit.
    \item Vererbung und Instanziierung von Objekt- und Variablentypen
    \item Zugriffskontrolle auch für einzelne Nodes
  \end{itemize}
  \item Subscriptions
  \begin{itemize}
    \item Erlaubt die Überwachung (subscriptions / monitoreditems)
    \item Sehr geringer Ressourcenbedarf pro überwachtem Wert
  \end{itemize}
  \item Code-Generierung auf XML-Basis
  \begin{itemize}
    \item Erlaubt die Erstellung von Datentypen
    \item Erlaubt die Generierung des serverseitigen Informationsmodells
  \end{itemize}
\end{itemize}

% https://open62541.org/doc/0.3/


Mozilla Public License
CC0 Lizenz für Beispiele und Plugins

% https://open62541.org/doc/open62541-current.pdf
% https://open62541.org/

% % % Imports nur für Referenzenauflösung während des Schreibens! Vorm Kompilieren auskommentieren!
% \bibliography{0_hauptdatei}
% \input{1_einleitung}
% \input{2_grundlagen}
% \input{3_konzeption}
% \input{4_implementierung}
% \input{5_tests}
% \input{6_zusammenfassung}
% \input{anhang}
% % Ende Imports

\section{Systemkonzept%
  \label{sec:3-konzeption}}
Auf Basis der in Abschnitt \ref{sec:2-grundlagen} vorgestellten Möglichkeiten folgt nun die Ausarbeitung eines Konzepts.
In den folgenden Abschnitten soll näher auf zwei zentrale Aspekte eingegangen werden: Abschnitt~\ref{sec:3-anbindung} stellt Möglichkeiten zum Zugriff auf Variablen bzw.\,Werte im Prozessabbild des Revolution Pi vor; in Abschnitt~\ref{sec:3-integration} wird ein Konzept zur Bereitstellung dieser Variablen auf einem OPC-Server vorgestellt.

\subsection{Anbindung der IO an den OPC-Server%
     \label{sec:3-anbindung}}

Eine Webanwendung mit Bezeichnung PiCtory dient zur Konfiguration der I/O- und virtuellen Module des RevolutionPi. Die Konfiguration liegt im JSON-Format in der Datei \lstinline{/etc/revpi/config.rsc}. Der piControl-Treiber liest diese Datei beim Start. 
Der folgende Auszug aus der Manpage des piControl-Kernelmoduls beschreibt die von diesem zum Lesen und Schreiben einzelner Bits des Prozessabbildes bereitgestellten Funktionen~\citep[vgl.]{web-revpi-manpage}. Sie ist an dieser Stelle weitgehend ungekürzt zitiert, da sie die nutzbare Schnittstelle sehr kompakt beschreibt.

\begin{lstlisting}[breakindent=0pt, numbers=none, caption={Auszug aus der Revolution Pi Programmers Manual\label{lst:4-manpage}}]
KB_FIND_VARIABLE SPIVariable *argp
Find a variable in the process image by its name. A pointer to a structure of type SPIVariable must be passed as argument. [...]
The struct SPIVariable [...] is defined as 
typedef struct SPIVariableStr
{
    char strVarName[32]; // Variable name
    uint16_t i16uAddress; // Address of the byte in the process image
    uint8_t i8uBit; // 0-7 bit position, >= 8 whole byte
    uint16_t i16uLength; // length of the variable in bits.
    // Possible values are 1, 8, 16 and 32
} SPIVariable;

Set and get values of the process image
KB_GET_VALUE SPIValue *argp
[...]
KB_SET_VALUE SPIValue *argp
Write one bit or one byte to the process image [...].  This call is more efficient than the usual calls of seek and write because only one function call is necessary. If more than on application are writing bits in one output byte, this call is the only safe way to set a bit without overwriting the other bits because this call is doing a read-modify-write-cycle. 

The struct SPIValue used by this ioctl is defined as
typedef struct SPIValueStr
{
    uint16_t i16uAddress; // Address of the byte in the process image
    uint8_t i8uBit; // 0-7 bit position, >= 8 whole byte
    uint8_t i8uValue; // Value: 0/1 for bit access, whole byte otherwise
} SPIValue;
\end{lstlisting} 

Die oben beschriebenden Funtkionen \lstinline{KB_FIND_VARIABLE}, \lstinline{KB_GET_VALUE} und \lstinline{KB_SET_VALUE} ermöglichen einen einfachen und (lt.\,Manpage) effizienten Zugriff auf einzelne Bits des Prozessabbildes und damit auch auf die IO des RevolutionPi.
Der Zugriff des OPC-Servers auf das Prozessabbild soll daher mittels dieser Funktionen realisiert werden.
\lstinline{KB_FIND_VARIABLE} kann genutzt werden, um Adressen von Variablen im Prozessabbild mittels ihres Namens aufzulösen.
\lstinline{KB_GET_VALUE} und \lstinline{KB_SET_VALUE} ermöglichen den Zugriff auf die Werte dieser Variablen.


\subsection{Integration des OPC-Servers in das System%
     \label{sec:3-integration}}

open62541 bietet drei Möglichkeiten zum Abgleich von Variablen mit dem Prozessabbild~\citep[vgl.][Tutorials - Connecting a Variable with a Physical Process]{web-open62541}:
\begin{itemize}
    \item Manuelles oder zyklisches Aktualisieren
    \item Variable Value Callback
    \item Variable Datasource
\end{itemize}

Die zyklische Aktualisierung eines oder mehrerer Werte nimmt, abhängig von der Zykluszeit, viele Systemressourcen in Anspruch. Value Callbacks ermöglichen es, einen Variablenwert effizienter mit einer Ressource wie etwa einem Prozessabbild zu synchronisieren. An die Variable wird ein Callback angehängt, welches vor jedem Lesen und nach jedem Schreibvorgang ausgeführt wird.
Der Wert der Variablen wird weiterhin im Variablenknoten auf dem OPC-Server gespeichert, der Abgleich mit der verknüpften Ressource erfolgt durch die Callback-Methoden.

Sogenannte Datenquellen gehen noch einen Schritt weiter. Der Server leitet jede Lese- und Schreibanforderung direkt an eine Callback-Funktion weiter. Beim Lesen liefert der Rückruf eine Kopie des aktuellen Wertes. Die Datenquelle muss intern ein eigenes Speichermanagement implementieren.

Der Zugriff auf die Werte des Prozessabbildes erfolgt, wie in Abschnitt~\ref{sec:3-anbindung} beschrieben, über von piControl bereitgestellte Methoden. Um die durch open62541 gepflegte OPC-Datenstruktur und das durch piControl verwaltete Prozessabbild möglichst effektiv verknüpfen zu können, soll diese Interaktion mittels Datenquellen und den zugehörigen Callbacks implementiert werden.
% % % Imports nur für Referenzenauflösung während des Schreibens! Vorm Kompilieren auskommentieren!
% \bibliography{0_hauptdatei}
% \input{1_einleitung}
% \input{2_grundlagen}
% \input{3_konzeption}
% \input{4_implementierung}
% \input{5_tests}
% \input{6_zusammenfassung}
% \input{anhang}
% % Ende Imports

\section{Implementierung%
  \label{sec:4-implementierung}}
Das folgende Kapitel stellt in Auszügen die Implementierung des OPC-Servers sowie die Anbindung an die IO-Module
der SPS dar. Der Schwerpunkt liegt hierbei auf der Funktionsweise des piControl-Treibers und dessen Integration in das Projekt. Abschnitt~\ref{sec:4-picontrol} erklärt die zum Schreibens eines Bits verwendeten Funktionsaufrufe.
Zuvor soll jedoch in Abschnitt~\ref{sec:4-open62541} der Teil des OPC-Servers vorgestellt werden, welcher auf besagten Treiber zugreift. 

\subsection{Implementierung des OPC-Servers%
     \label{sec:4-open62541}}
Wie im vorangegangenen Abschnitt~\ref{sec:3-integration} begründet, soll die Verknüpfung zwischen dem Prozessabbild der SPS und den auf dem OPC-Server bereitgestellten Werten über sog.\,Datenquellen erfolgen. Hierzu ist zunächst eine Callback-Methode zu implementieren, welche bei einem Lese- oder Schreibzugriff auf eine Variable aufgerufen wird. Die Verknüpfung zwischen Callback-Methode und Variable muss manuell erfolgen.

\begin{lstlisting}[language={c},firstnumber=237,caption={Auszug der Methode \lstinline{linkDataSourceVariable} in \lstinline{variables.c}\label{lst:4-linkDataSourceVariable}}]
extern UA_StatusCode
 linkDataSourceVariable(UA_Server *server, UA_NodeId nodeId) {
     bool readonly = false;
     UA_DataSource dataSourceVariable;
     UA_StatusCode rc; |>\setcounter{lstnumber}{254}<|

     dataSourceVariable.read = readDataSourceVariable;
     if (!readonly)
        dataSourceVariable.write = writeDataSourceVariable;
     else
        dataSourceVariable.write = writeReadonlyDataSourceVariable;

     return UA_Server_setVariableNode_dataSource(server, nodeId, dataSourceVariable);
 }
\end{lstlisting}

\begin{figure}[h]
    \centering
    \includegraphics[width=0.42\textwidth]{doc/img/OPC_RevPiDO.pdf}
    \caption{Auszug des verwendeten Nodesets, hier Digitalausgang 1 des Versuchsaufbaus
      \label{fig:opc-do}}
\end{figure}

Die in Listing~\ref{lst:4-linkDataSourceVariable} abgebildete Methode \lstinline{linkDataSourceVariable()} erzeugt ein Struct vom Typ \lstinline{UA_DataSource}. In diesem werden dem Lesen und Schreiben einer OPC-Variablen entsprechende Callback-Methoden zugewiesen. Die Verknüpfung einer OPC-Variable, genauer ihrer NodeId, mit der zuvor definierten Datenquelle erfolgt über die von open62541 bereitgestellte Methode \lstinline{UA_Server_setVariableNode_dataSource()}. Vor dem Lesen und nach dem Schreiben dieser Variable werden von nun an die entsprechenden Callbacks aufgerufen.
     
\begin{lstlisting}[language={c},firstnumber=168,caption={Auszug des Callbacks \lstinline{writeDataSourceVariable} in \lstinline{variables.c}\label{lst:4-writeDataSourceVariable}}]  
extern UA_StatusCode
 writeDataSourceVariable(UA_Server *server,
            const UA_NodeId *sessionId, void *sessionContext,
            const UA_NodeId *nodeId, void *nodeContext,
            const UA_NumericRange *range, const UA_DataValue *dataValue) {

    UA_StatusCode retval  = UA_STATUSCODE_GOOD;
    UA_NodeId *nameNodeId = UA_malloc(sizeof(UA_NodeId));
    UA_QualifiedName nameQN = UA_QUALIFIEDNAME(1, "Name");
    UA_Variant nameVar;
    UA_Boolean bit;

    retval |= findSiblingByBrowsename(server, nodeId, &nameQN, nameNodeId);
    retval |= UA_Server_readValue(server, *nameNodeId, &nameVar);
    retval |= UA_Boolean_copy(dataValue->value.data, &bit);

    |>\tikzmarkin[set border color=martinired]{writeIO}<|PI_writeSingleIO(String_fromUA_String(nameVar.data), &bit, false);                                                 |>\tikzmarkend{writeIO}<|

    free(nameNodeId);
    return retval;
 }
\end{lstlisting}

Listing~\ref{lst:4-writeDataSourceVariable} zeigt die Callback-Methode, welche nach dem Schreiben einer Variablen auf dem OPC-Server aufgerufen wird.
Dieser Methode wird neben der NodeId der mit ihr verknüpften Variablen auch der Wert dieser in Form eines Zeigers auf ein Struct vom Typ \lstinline{UA_DataValue} übergeben.

Die Gestaltung des hier verwendeten Nodesets sieht vor, dass in einer OPC-Variablen \lstinline{"Name"} der Bezeichner des zu schreibenden Digitalausgangs hinterlegt ist, siehe Abbildung~\ref{fig:opc-do}. Dies erlaubt eine Rekonfiguration der Ein- und Ausgänge der SPS ohne Änderungen im Programmcode des OPC-Servers vornehmen zu müssen.
Es ist daher erforderlich, nach jedem Schreiben einer mit einem Digitalausgang verknüpften Variablen, hier \lstinline{"Value"}, dessen Bezeichner \lstinline{"Name"} abzufragen. 
Dies geschieht in den Zeilen 180 und 181.
Anschließend wird dieser Bezeichner sowie der zu schreibende Wert der Methode \lstinline{PI_writeSingleIO()} übergeben, welche wiederum die Interaktion mit piControl übernimmt (vgl. Abschnitt \ref{sec:4-picontrol}).
 
\subsection{Integration von piControl%
     \label{sec:4-picontrol}}
In Abschnitt~\ref{sec:2-io} wurde die Anbindung der IO-Module des Revolution Pi sowie die Funktionsweise von piControl aus Anwendersicht beschrieben. Die verfügbare Literatur beschränkt sich auch auf lediglich diese Sicht; eine weiterführende Dokumentation für Entwickler gibt es, neben der in Abschnitt~\ref{sec:3-anbindung} vorgestellten Manpage, nicht. 
In diesem Abschnitt soll daher der Quellcode von piControl sowie dessen Verwendung im Projekt genauer betrachtet werden.
Hierzu wird exemplarisch die in Abschnitt~\ref{sec:4-open62541} eingeführte Methode \lstinline{PI_writeSingleIO()} untersucht.
Diese Methode ermöglicht das Setzen eines einzelnen Bits im Prozessabbild der SPS, und damit das Schalten eines digitalen Ausgangs auf einem IO-Modul.
Die äquivalente Methode \lstinline{int piControlGetBitValue(SPIValue *pSpiValue)} zum Lesen eines Bits bzw. Eingangs funktioniert analog und soll daher an dieser Stelle nicht dediziert erörtert werden.

\begin{lstlisting}[language={c},firstnumber=97,
                   caption={Setzen eines phsikalischen, digitalen Ausgangs in \lstinline{revpi.c}
                   \label{lst:4-PI_writeSingleIO}}]
extern void PI_writeSingleIO(char *pszVariableName, bool *bit, bool verbose)
{
	int rc;
	SPIVariable sPiVariable;
	SPIValue sPIValue;

	strncpy(sPiVariable.strVarName, pszVariableName, sizeof(sPiVariable.strVarName));
	rc = piControlGetVariableInfo(&sPiVariable);
	if (rc < 0) {
		printf("Cannot find variable '%s'\n", pszVariableName);
		return;
	}

		sPIValue.i16uAddress = sPiVariable.i16uAddress;
		sPIValue.i8uBit = sPiVariable.i8uBit;
		sPIValue.i8uValue = *bit;
		rc = |>\tikzmarkin[set border color=martinired]{setBitValue}<|piControlSetBitValue(&sPIValue)|>\tikzmarkend{setBitValue}<|;
		if (rc < 0)
			printf("Set bit error %s\n", getWriteError(rc));
		else if (verbose)
			printf("Set bit %d on byte at offset %d. Value %d\n", sPIValue.i8uBit, sPIValue.i16uAddress,
			       sPIValue.i8uValue);
}
\end{lstlisting}

Der Programmcode in Listing~\ref{lst:4-PI_writeSingleIO} ist Teil des implementierten OPC-Servers. In diesem wird auf zwei Funktionen des piControl-Treibers zugegriffen. 
Beiden Methoden wird als Argument ein Zeiger auf ein Struct vom Typ \lstinline{SPIValue} übergeben. Der im Struct abgelegte Name wird mittels \lstinline{piControlGetVariableInfo(&sPIValue)} zu einer Adresse im Prozessabbild aufgelöst. Diese wird in \lstinline{sPIValue.i16uAdress} gespeichert. Der Wert der Variablen wird anschließend mittels \lstinline{piControlSetBitValue(&sPIValue)} an dieser Adresse in das Prozessabbild geschrieben.

\begin{lstlisting}[language={c},firstnumber=309,caption={Methode \lstinline{piControlSetBitValue} in \lstinline{piControlIf.c}\label{lst:4-piControlSetBitValue}}]
int |>\tikzmarkin[set border color=martiniblue]{setBitValueFcn}<|piControlSetBitValue(SPIValue *pSpiValue)|>\tikzmarkend{setBitValueFcn}<|
{
    piControlOpen();

    if (PiControlHandle_g < 0)
	    return -ENODEV;

    pSpiValue->i16uAddress += pSpiValue->i8uBit / 8;
    pSpiValue->i8uBit %= 8;

    if (|>\tikzmarkin[set border color=martinired]{ioctl}<|ioctl(PiControlHandle_g, KB_SET_VALUE, pSpiValue)|>\tikzmarkend{ioctl}<| < 0)
	    return errno;

    return 0;
}
\end{lstlisting}

Die in Listing~\ref{lst:4-piControlSetBitValue} dargestellte Methode \lstinline{piControlSetBitValue} ist lediglich eine Hüllfunktion (häufig auch als Wrapper-Funktion bezeichnet) für einen Aufruf des \lstinline{ioctl} Kernel-Moduls.
Folgende Parameter werden übergeben:
\lstinline{PiControlHandle_g} ist die Referenz auf die Geräte-Datei des piControl-Treibers. \lstinline{KB_SET_VALUE} ist das ioctl-Kommando zum Schreiben eines Bits in das Prozessabbild. Der Zeiger \lstinline{pSpiValue} verweist auf ein Struct des bereits vorgestellten Typs \lstinline{SPIValue}.

\begin{lstlisting}[language={c},firstnumber=80,caption={Methode \lstinline{piControlOpen} in \lstinline{piControlIf.c}\label{lst:4-piControlOpen}}]
void piControlOpen(void)
{
    /* open handle if needed */
    if (PiControlHandle_g < 0)
    {
	    |>\tikzmarkin[set border color=martiniblue]{PiControlHandle}<|PiControlHandle_g = open(PICONTROL_DEVICE, O_RDWR)|>\tikzmarkend{PiControlHandle}<|;
    }
}
\end{lstlisting}

Die in Listing~\ref{lst:4-piControlOpen} dargestellte Methode öffnet, sofern nicht bereits geschehen, die Geräte-Datei. Das Macro \lstinline{PICONTROL_DEVICE} verweist hierbei auf \lstinline{/dev/piControl0}.

\begin{lstlisting}[language={c},firstnumber=721,caption={Methode \lstinline{piControlIoctl} in \lstinline{piControlMain.c}\label{lst:4-piControlIoctl}}]
static long |>\tikzmarkin[set border color=martiniblue, below offset=0.9em]{piControlIoctl}<|piControlIoctl(struct file *file, unsigned int prg_nr, 
                           unsigned long usr_addr)                                      |>\tikzmarkend{piControlIoctl}<|
{
  int status = -EFAULT;
  tpiControlInst *priv;
  int timeout = 10000;	// ms

  if (prg_nr != KB_CONFIG_SEND && prg_nr != KB_CONFIG_START && !isRunning()) {
  	return -EAGAIN;
  }

  priv = (tpiControlInst *) file->private_data;

  if (prg_nr != KB_GET_LAST_MESSAGE) {
  	// clear old message
  	priv->pcErrorMessage[0] = 0;
  }

  switch (prg_nr) {|>\setcounter{lstnumber}{864}<|

    case |>\tikzmarkin[set border color=martiniblue]{KB_SET_VALUE}<|KB_SET_VALUE:|>\tikzmarkend{KB_SET_VALUE}<|
  		{
  			SPIValue *pValue = (SPIValue *) usr_addr;

  			if (!isRunning())
  				return -EFAULT;

  			if (pValue->i16uAddress >= KB_PI_LEN) {
  				status = -EFAULT;
  			} else {
  				INT8U i8uValue_l;
  				my_rt_mutex_lock(&piDev_g.lockPI);
  				i8uValue_l = piDev_g.ai8uPI[pValue->i16uAddress];

  				if (pValue->i8uBit >= 8) {
  					i8uValue_l = pValue->i8uValue;
  				} else {
  					if (pValue->i8uValue)
  						i8uValue_l |= (1 << pValue->i8uBit);
  					else
  						i8uValue_l &= ~(1 << pValue->i8uBit);
  				}

  				|>\tikzmarkin[set border color=martinired]{i8uValue}<|piDev_g.ai8uPI[pValue->i16uAddress] = i8uValue_l;|>\tikzmarkend{i8uValue}<|
  				rt_mutex_unlock(&piDev_g.lockPI);

  #ifdef VERBOSE
  				pr_info("piControlIoctl Addr=%u, bit=%u: %02x %02x\n", pValue->i16uAddress, pValue->i8uBit, pValue->i8uValue, i8uValue_l);
  #endif

  				status = 0;
  			}
  		}
  		break; |>\setcounter{lstnumber}{1314}<|

    default:
      pr_err("Invalid Ioctl");
      return (-EINVAL);
      break;

    }

    return status;
  }
\end{lstlisting}

Listing~\ref{lst:4-piControlIoctl} zeigt in Auszügen die ioctl-Methode des piControl Kernel-Treibers. Diese bekommt folgende Argumente übergeben: \lstinline{struct file *file} enthält den Verweis auf die Geräte-Datei, hier \lstinline{/dev/piControl0}. Der Wert von \lstinline{unsigned int prg_nr} beschreibt die Anfrage an den Treiber, in diesem Fall \lstinline{KB_SET_VALUE}. Das Argument \lstinline{unsigned long usr_addr} enthält einen typ-agnostischen Pointer. Dieser verweist auf einen Speicherbereich, in welchem die zur Bearbeitung der Anfrage notwendigen Daten abgelegt sind. Hier können auch vom Treiber empfangene Daten dem Anwendungsprogramm bereitgestellt werden. 

Die switch-case-Anweisung führt die über das Argument \lstinline{prg_nr} spezifizierte Aktion aus. Hier betrachten wir \lstinline{KB_SET_VALUE}:
Zunächst wird in Zeile 868 der übergebene Zeiger \lstinline{usr_addr} mittels explizitem Typecast zu einem Zeiger des Typs \lstinline{SPIValue *} konvertiert. Da dieser auf Daten im Userspace verweist, ist beim Zugriff durch den Kernel-Treiber besondere Vorsicht geboten.
In Zeile 877 wird mittels Mutex das Prozessabbild \lstinline{piDev_g} für den Zugriff durch andere Threads oder Prozesse gesperrt.
\lstinline{my_rt_mutex_lock} verweist hierbei auf die Funktion \lstinline{rt_mutex_lock} aus \lstinline{linux/sched.h}\footnote{Offenbar wurde hier auch eine alternative Implementierung vorgesehen, siehe revpi\_common.h}

In Zeile 889 wird das Byte \lstinline{i8uValue_l}, welches den zu schreibenden Wert enthält in das Prozessabbild übertragen. Anschließend wird die Mutex auf \lstinline{piDev_g} wieder entsperrt.
\newpage

\begin{lstlisting}[language={c},firstnumber=62,caption={Auszug des Struct \lstinline{spiControlDev} in \lstinline{piControlMain.h}\label{lst:4-spiControlDev}}]
|>\tikzmarkin[set border color=martiniblue]{spiControlDev}<|typedef struct spiControlDev|>\tikzmarkend{spiControlDev}<| {
	// device driver stuff
	int init_step;
	enum revpi_machine machine_type;
	void *machine;
	struct cdev cdev;	// Char device structure
	struct device *dev;
	struct thermal_zone_device *thermal_zone;

	|>\tikzmarkin[set border color=martiniblue]{processImage}<|// process image stuff
	INT8U ai8uPI[KB_PI_LEN];
	INT8U ai8uPIDefault|>\tikzmarkin[set border color=martinired]{KB_PI_LEN_0}<|[KB_PI_LEN]|>\tikzmarkend{KB_PI_LEN_0}<|;
	struct rt_mutex lockPI;        |>\tikzmarkend{processImage}<|
	bool stopIO;
	piDevices *devs; |>\setcounter{lstnumber}{94}<|
} tpiControlDev;
\end{lstlisting}

Das Prozessabbild ist als Byte-Array der Länge \lstinline{KB_PI_LEN} in Listing~\ref{lst:4-spiControlDev} definiert. Konfigurationsparameter wie \lstinline{KB_PI_LEN} oder die Zykluszeit für den Datenaustausch zwischen SPS und IO-Modulen sind im folgenden Listing~\ref{lst:4-process} definiert.

\begin{lstlisting}[language={c},firstnumber=119,caption={Konfigurationsparameter des Prozessabbildes in project.h\label{lst:4-process}}]
#define INTERVAL_PI_GATE (5*1000*1000)  // 5 ms piGateCommunication |>\setcounter{lstnumber}{128}<|

#define INTERVAL_IO_COM (5*1000*1000)  // 5 ms piIoComm |>\setcounter{lstnumber}{132}<|

#define KB_PD_LEN       512
|>\tikzmarkin[set border color=martiniblue]{KB_PI_LEN_1}<|#define KB_PI_LEN       4096|>\tikzmarkend{KB_PI_LEN_1}<|
\end{lstlisting}

Das zu setzende Bit wurde zu diesem Zeitpunkt erfolgreich in das Prozessabbild der SPS geschrieben.
Es stellt sich die Frage, wie dieses nun an das IO-Modul kommuniziert wird.
Die Kommunikation mit allen angebundenen Modulen ist ebenfalls Aufgabe des piControl-Treibers.

\begin{lstlisting}[language={c},firstnumber=256,caption={Auszug der Methode \lstinline{piIoThread} in \lstinline{revpi_core.c}\label{lst:4-piIoThread}}]
static int piIoThread(void *data)
{
	//TODO int value = 0;
	ktime_t time;
	ktime_t now;
	s64 tDiff;

	hrtimer_init(&piCore_g.ioTimer, CLOCK_MONOTONIC, HRTIMER_MODE_ABS);
	piCore_g.ioTimer.function = piIoTimer;

	pr_info("piIO thread started\n");

	now = hrtimer_cb_get_time(&piCore_g.ioTimer);

	PiBridgeMaster_Reset();

	while (!kthread_should_stop()) {
		if (|>\tikzmarkin[set border color=martinired]{PiBridgeMaster}<|PiBridgeMaster_Run()|>\tikzmarkend{PiBridgeMaster}<| < 0)
			break;
	}

	RevPiDevice_finish();

	pr_info("piIO exit\n");
	return 0;
}
\end{lstlisting}

Der Kernel-Thread \lstinline{piIoThread} ist verantwortlich für den zyklischen Datenaustausch mit den IO-Modulen. In diesem wird fortlaufend die Methode \lstinline{PiBridgeMaster_Run()} aufgerufen, siehe Listing~\ref{lst:4-piIoThread}.

\begin{lstlisting}[language={c},firstnumber=262,caption={Auszug der Methode \lstinline{PiBridgeMaster_Run(void)} in \lstinline{RevPiDevice.c}\label{lst:4-PiBridgeMaster_Run}}]
int PiBridgeMaster_Run(void)
{
	static kbUT_Timer tTimeoutTimer_s;
	static kbUT_Timer tConfigTimeoutTimer_s;
	static int error_cnt;
	static INT8U last_led;
	static unsigned long last_update;
	int ret = 0;
	int i;

	my_rt_mutex_lock(&piCore_g.lockBridgeState);
	if (piCore_g.eBridgeState != piBridgeStop) {
		switch (eRunStatus_s) { |>\setcounter{lstnumber}{514}<|
		    case enPiBridgeMasterStatus_EndOfConfig:|>\setcounter{lstnumber}{621}<|
		    if (|>\tikzmarkin[set border color=martinired]{RevPiDevice}<|RevPiDevice_run()|>\tikzmarkend{RevPiDevice}<|) {
				// an error occured, check error limits |>\setcounter{lstnumber}{641}<|
			} else {
				ret = 1;
			}
			piCore_g.image.drv.i16uRS485ErrorCnt = RevPiDevice_getErrCnt();
			break;
\end{lstlisting}

Die in Listing~\ref{lst:4-PiBridgeMaster_Run} dargestellte Methode ist eine sog. State-Machine. Ist die Konfiguration der IO-Module erfolgreich abgeschlossen, so führt sie bei Aufruf lediglich die Methode \lstinline{RevPiDevice_run()} aus.

\begin{lstlisting}[language={c},firstnumber=140,caption={Auszug der Methode \lstinline{RevPiDevice_run(void)} in \lstinline{RevPiDevice.c}\label{lst:4-RevPiDevice_run}}]
int RevPiDevice_run(void)
{
	INT8U i8uDevice = 0;
	INT32U r;
	int retval = 0;

	RevPiDevices_s.i16uErrorCnt = 0;

	for (i8uDevice = 0; i8uDevice < RevPiDevice_getDevCnt(); i8uDevice++) {
		if (RevPiDevice_getDev(i8uDevice)->i8uActive) {
			switch (RevPiDevice_getDev(i8uDevice)->sId.i16uModulType) {
			case KUNBUS_FW_DESCR_TYP_PI_DIO_14:
			case KUNBUS_FW_DESCR_TYP_PI_DI_16:
			case KUNBUS_FW_DESCR_TYP_PI_DO_16:
				r = |>\tikzmarkin[set border color=martinired]{sendCyclicTelegram}<|piDIOComm_sendCyclicTelegram(i8uDevice)|>\tikzmarkend{sendCyclicTelegram}\setcounter{lstnumber}{166} <|;

				break; |>\setcounter{lstnumber}{216}<|
			}
		}
	} |>\setcounter{lstnumber}{227}<|
	return retval;
}
\end{lstlisting}

Diese iteriert wie in Listing~\ref{lst:4-RevPiDevice_run} abgebildete durch alle gegenwärtig in der SPS konfigurierten Module. Ist das aktuelle Modul als aktiv markiert, so wird anhand eines sog. Firmware-Descriptors entschieden, welche Methode für die Ansteuerung des Moduls aufzurufen ist.

\begin{lstlisting}[language={c},firstnumber=161,caption={Auszug der Methode \lstinline{piDIOComm_sendCyclicTelegram} in \lstinline{piDIOComm.c}\label{lst:4-sendCyclicTelegram}}]
INT32U piDIOComm_sendCyclicTelegram(INT8U i8uDevice_p)
{
	INT32U i32uRv_l = 0;
	SIOGeneric sRequest_l;
	SIOGeneric sResponse_l;
	INT8U len_l, data_out[18], i, p, data_in[70];
	INT8U i8uAddress;
	int ret; |>\setcounter{lstnumber}{239}<|
	
    |>\tikzmarkin[set border color=martinired]{piIoComm}<|ret = piIoComm_send((INT8U *) & sRequest_l, IOPROTOCOL_HEADER_LENGTH + len_l + 1);  |>\tikzmarkend{piIoComm}\setcounter{lstnumber}{298}<|
}
\end{lstlisting}

Im Falle des hier verwendeten DO-Moduls wird die in Listing~\ref{lst:4-sendCyclicTelegram} abgebildete Methode \lstinline{piDIOComm_sendCyclicTelegram()} aufgerufen. Dieser wird ein Zeiger auf das zu schreibende Gerät übergeben. 
Zunächst wird das Prozessabbild mittels eines proprietären, jedoch im Quellcode offen nachvollziehbaren Protokolls in ein \lstinline{sRequest_l} genanntes Byte-Array umgewandelt. Dieser Schritt ist in Listing~\ref{lst:4-sendCyclicTelegram} nicht abgebildet. Anschließend wird \lstinline{piIoComm_send()} ein Zeiger auf die so generierte Schreib-Anfrage übergeben.

\begin{lstlisting}[language={c},firstnumber=220,caption={Auszug der Methode \lstinline{piIOComm_send} in \lstinline{piIOComm.c}\label{lst:4-piIOComm_send}}]
int piIoComm_send(INT8U * buf_p, INT16U i16uLen_p)
{
	ssize_t write_l = 0;
	INT16U i16uSent_l = 0;|>\setcounter{lstnumber}{249}<|

	while (i16uSent_l < i16uLen_p) {
		write_l = vfs_write(piIoComm_fd_m, buf_p + i16uSent_l, i16uLen_p - i16uSent_l, &piIoComm_fd_m->f_pos);
		if (write_l < 0) {
			pr_info_serial("write error %d\n", (int)write_l);
			return -1;
		} 
		i16uSent_l += write_l;|>\setcounter{lstnumber}{263}<|
	}
	clear();
	vfs_fsync(piIoComm_fd_m, 1);
	return 0;
}
\end{lstlisting}

Listing~\ref{lst:4-piIOComm_send} zeigt die Implementierung von \lstinline{piIoComm_send()}. Diese Methode ist für das Schreiben der oben generierten Anfrage auf die seriellen Schnittstelle verantwortlich. Realisiert wird dies mittels der Methode \lstinline{vfs_write()}. Diese ist in \lstinline{<linux/fs.h>} definiert. Sie ermöglicht das Schreiben einer Datei im Userspace aus dem Kernel heraus. Geschrieben wird hier die Datei mit dem Deskriptor \lstinline{piIoComm_fd_m}.
Da die Funktion \lstinline{vfs_write()} durch andere Kernel-Tasks unterbrochen werden kann, ist nicht gewährleistet, dass die gesamte Anfrage mit nur einem Aufruf geschrieben wird. Die oben abgebildete while-Schleife stellt das vollständige Senden der Anfrage sicher.

\begin{lstlisting}[language={c},firstnumber=157,caption={Auszug der Methode \lstinline{piIOComm_open_serial} in \lstinline{piIOComm.c}\label{lst:4-piIOComm_open_serial}}]
int piIoComm_open_serial(void)
{   |>\setcounter{lstnumber}{167}<|
	struct file *fd;	/* Filedeskriptor */
	struct termios newtio;	/* Schnittstellenoptionen */

	|>\tikzmarkin[set border color=martiniblue]{fd}<|/* Port oeffnen - read/write, kein "controlling tty", 
	    Status von DCD ignorieren */
	fd = filp_open(|>\tikzmarkin[set border color=martinired]{tty}<|REV_PI_TTY_DEVICE|>\tikzmarkend{tty}<|, O_RDWR | O_NOCTTY, 0); |>\setcounter{lstnumber}{208}<|
	
	piIoComm_fd_m = fd;                                                      |>\tikzmarkend{fd}\setcounter{lstnumber}{217}<|

	return 0;
}
\end{lstlisting}

Der zum Schreiben auf die serielle Schnittstelle verwendete Datei-Deskriptor wird von der in Listing~\ref{lst:4-piIOComm_open_serial} abgebildeten Methode \lstinline{piIoComm_open_serial()} generiert. 

\begin{lstlisting}[language={c},firstnumber=45,caption={Definition der seriellen Schnittstelle in \lstinline{piIOComm.h}\label{lst:4-REV_PI_TTY_DEVICE}}]
#define REV_PI_TTY_DEVICE	"/dev/ttyAMA0"
\end{lstlisting}

Das in Listing~\ref{lst:4-REV_PI_TTY_DEVICE} definierte Macro verweist auf eine der seriellen Schnittstellen des RaspberryPi.
Die Implementierung des zugehörigen Schnittstellentreibers soll hier nicht weiter untersucht werden. Somit ist an dieser Stelle die Kette vom Setzen einer Variablen auf dem OPC-Server bis hin zur Aktualisierung des Prozessabbilds der IO-Module geschlossen.

% \begin{lstlisting}[language={c},firstnumber={226},caption={Setzen der Scheduler-Priorität auf SCHED\_FIFO in 
% revpi\_common.c\label{lst:2-sched_priority}}]
% param.sched_priority = ktprio->prio;
% ret = sched_setscheduler(child, SCHED_FIFO, &param);
% \end{lstlisting}
% % % Imports nur für Referenzenauflösung während des Schreibens! Vorm Kompilieren auskommentieren!
% \bibliography{0_hauptdatei}
% \input{1_einleitung}
% \input{2_grundlagen}
% \input{3_konzeption}
% \input{4_implementierung}
% \input{5_tests}
% \input{6_zusammenfassung}
% % Ende Imports

\section{Test des OPC-Servers im Gesamtsystem%
  \label{sec:5-tests}}

% % % Imports nur für Referenzenauflösung während des schreibens! Vorm Kompilieren auskommentieren!
% \bibliography{0_hauptdatei}
% \input{1_einleitung}
% \input{2_grundlagen}
% \input{3_konzeption}
% \input{4_implementierung}
% \input{5_tests}
% \input{6_zusammenfassung}
% % Ende Imports

\section{Zusammenfassung und Ausblick%
  \label{sec:6-fazit}}
Der folgende Abschnitt~\ref{sec:6-zusammenfassung} fasst die gewonnenen Erkenntnisse und den Stand der Implementierung zusammen.
Den Abschluss dieser Arbeit bildet der Ausblick in Abschnitt~\ref{sec:6-ausblick}.

\subsection{Zusammenfassung%
     \label{sec:6-zusammenfassung}}

\subsection{Ausblick%
     \label{sec:6-ausblick}}

% \input{anhang}
% % Ende Imports

\section{Systemkonzept%
  \label{sec:3-konzeption}}
Auf Basis der in Abschnitt \ref{sec:2-grundlagen} vorgestellten Möglichkeiten folgt nun die Ausarbeitung eines Konzepts.
In den folgenden Abschnitten soll näher auf zwei zentrale Aspekte eingegangen werden: Abschnitt~\ref{sec:3-anbindung} stellt Möglichkeiten zum Zugriff auf Variablen bzw.\,Werte im Prozessabbild des Revolution Pi vor; in Abschnitt~\ref{sec:3-integration} wird ein Konzept zur Bereitstellung dieser Variablen auf einem OPC-Server vorgestellt.

\subsection{Anbindung der IO an den OPC-Server%
     \label{sec:3-anbindung}}

Eine Webanwendung mit Bezeichnung PiCtory dient zur Konfiguration der I/O- und virtuellen Module des RevolutionPi. Die Konfiguration liegt im JSON-Format in der Datei \lstinline{/etc/revpi/config.rsc}. Der piControl-Treiber liest diese Datei beim Start. 
Der folgende Auszug aus der Manpage des piControl-Kernelmoduls beschreibt die von diesem zum Lesen und Schreiben einzelner Bits des Prozessabbildes bereitgestellten Funktionen~\citep[vgl.]{web-revpi-manpage}. Sie ist an dieser Stelle weitgehend ungekürzt zitiert, da sie die nutzbare Schnittstelle sehr kompakt beschreibt.

\begin{lstlisting}[breakindent=0pt, numbers=none, caption={Auszug aus der Revolution Pi Programmers Manual\label{lst:4-manpage}}]
KB_FIND_VARIABLE SPIVariable *argp
Find a variable in the process image by its name. A pointer to a structure of type SPIVariable must be passed as argument. [...]
The struct SPIVariable [...] is defined as 
typedef struct SPIVariableStr
{
    char strVarName[32]; // Variable name
    uint16_t i16uAddress; // Address of the byte in the process image
    uint8_t i8uBit; // 0-7 bit position, >= 8 whole byte
    uint16_t i16uLength; // length of the variable in bits.
    // Possible values are 1, 8, 16 and 32
} SPIVariable;

Set and get values of the process image
KB_GET_VALUE SPIValue *argp
[...]
KB_SET_VALUE SPIValue *argp
Write one bit or one byte to the process image [...].  This call is more efficient than the usual calls of seek and write because only one function call is necessary. If more than on application are writing bits in one output byte, this call is the only safe way to set a bit without overwriting the other bits because this call is doing a read-modify-write-cycle. 

The struct SPIValue used by this ioctl is defined as
typedef struct SPIValueStr
{
    uint16_t i16uAddress; // Address of the byte in the process image
    uint8_t i8uBit; // 0-7 bit position, >= 8 whole byte
    uint8_t i8uValue; // Value: 0/1 for bit access, whole byte otherwise
} SPIValue;
\end{lstlisting} 

Die oben beschriebenden Funtkionen \lstinline{KB_FIND_VARIABLE}, \lstinline{KB_GET_VALUE} und \lstinline{KB_SET_VALUE} ermöglichen einen einfachen und (lt.\,Manpage) effizienten Zugriff auf einzelne Bits des Prozessabbildes und damit auch auf die IO des RevolutionPi.
Der Zugriff des OPC-Servers auf das Prozessabbild soll daher mittels dieser Funktionen realisiert werden.
\lstinline{KB_FIND_VARIABLE} kann genutzt werden, um Adressen von Variablen im Prozessabbild mittels ihres Namens aufzulösen.
\lstinline{KB_GET_VALUE} und \lstinline{KB_SET_VALUE} ermöglichen den Zugriff auf die Werte dieser Variablen.


\subsection{Integration des OPC-Servers in das System%
     \label{sec:3-integration}}

open62541 bietet drei Möglichkeiten zum Abgleich von Variablen mit dem Prozessabbild~\citep[vgl.][Tutorials - Connecting a Variable with a Physical Process]{web-open62541}:
\begin{itemize}
    \item Manuelles oder zyklisches Aktualisieren
    \item Variable Value Callback
    \item Variable Datasource
\end{itemize}

Die zyklische Aktualisierung eines oder mehrerer Werte nimmt, abhängig von der Zykluszeit, viele Systemressourcen in Anspruch. Value Callbacks ermöglichen es, einen Variablenwert effizienter mit einer Ressource wie etwa einem Prozessabbild zu synchronisieren. An die Variable wird ein Callback angehängt, welches vor jedem Lesen und nach jedem Schreibvorgang ausgeführt wird.
Der Wert der Variablen wird weiterhin im Variablenknoten auf dem OPC-Server gespeichert, der Abgleich mit der verknüpften Ressource erfolgt durch die Callback-Methoden.

Sogenannte Datenquellen gehen noch einen Schritt weiter. Der Server leitet jede Lese- und Schreibanforderung direkt an eine Callback-Funktion weiter. Beim Lesen liefert der Rückruf eine Kopie des aktuellen Wertes. Die Datenquelle muss intern ein eigenes Speichermanagement implementieren.

Der Zugriff auf die Werte des Prozessabbildes erfolgt, wie in Abschnitt~\ref{sec:3-anbindung} beschrieben, über von piControl bereitgestellte Methoden. Um die durch open62541 gepflegte OPC-Datenstruktur und das durch piControl verwaltete Prozessabbild möglichst effektiv verknüpfen zu können, soll diese Interaktion mittels Datenquellen und den zugehörigen Callbacks implementiert werden.
% % % Imports nur für Referenzenauflösung während des Schreibens! Vorm Kompilieren auskommentieren!
% \bibliography{0_hauptdatei}
% % Mit \section{...} eröffnen wir einen neuen Abschnitt.
% Der Befehl setzt nicht nur den Text in einer größeren,
% fetten Schrift, sondern sorgt außerdem dafür, daß er im
% Inhaltsverzeichnis erscheint.
%
% Mit \label{...} erzeugen wir einen Bezeichner, mit dessen Hilfe
% wir später auf die Nummer des Abschnitts verweisen können (nämlich
% mit~\ref{...}).
%
% Das Kommentarzeichen hinter „Übersicht“ dient dazu, ein
% Leerzeichen zwischen „Übersicht“ und dem \label-Befehl
% zu vermeiden, das andernfalls sichtbar würde – z.B. im
% Inhaltsverzeichnis.
%

% % Imports nur für Referenzenauflösung während des Schreibens! Vorm Kompilieren auskommentieren!
% \bibliography{0_hauptdatei}
% \input{1_einleitung}
%\input{2_grundlagen}
%\input{3_konzeption}
%\input{4_implementierung}
%\input{5_tests}
%\input{6_zusammenfassung}
% % Ende Imports

\section{Einleitung und Motivation%
  \label{sec:1-einleitung}}
Ziel dieses Projektes ist die Integration eines OPC-Servers mit einer auf Linux
basierenden speicherprogrammierbaren Steuerung (SPS). Angeschlossen an diese SPS
ist jeweils ein digitales Ein-/\,bzw.~Ausgabemodul. Die von diesen bereitgestellten
Ein-/\, bzw.~Ausgänge (IO) sollen in der Datenstruktur des OPC-Servers abgebildet
und über diesen für OPC-Clients les-/\,und schreibar sein. Weiterhin sollen einige
Funktionen zur Überwachung und Steuerung der an die SPS angeschlossenen Aktoren
und Sensoren direkt im OPC-Server implementiert werden.
Hiermit stellt dieses Projekt eine der Grundlagen für ein übergeordnetes Projekt,
die cloudbasierte Steuerung eines miniaturisierten Produktions-Systems, dar.

Der hier verwendete OPC-Server ist Teil des sog. open62541 Projekts. Er ist in C
geschrieben und implementiert bereits einen großen Teil der im OPC-UA-Standard
spezifizierten Funktionen.
Als SPS findet ein Revolution Pi 3 der Firma Kunbus Verwendung. Dieser integriert
ein sog. Compute Module der Raspberry Pi Foundation in ein industrietaugliches
Gehäuse und erlaubt die Erweiterung mittels IO- oder Gateway-Modulen. Über diese
erfolgt die Kommunikation mit weiteren Komponenten der Automatisierungstechnik.

Motiviert ist dieses Projekt durch die Beobachtung, dass die Verbreitung offener
Standards sowie freier Software auch in der Automatisierungstechnik zunimmt.
Linux ist ein freies Betriebssystem, OPC-UA ein offen zugänglicher, aktiv gepflegter
und weit verbreiteter Standard. Der Raspberry Pi findet sowohl bei Hobby-Anwendern als
auch in den Bereichen Forschung und Entwicklung sowie bei industriellen Anwendern
Verwendung. Dieses Projekt stellt somit eine für unterschiedliche Anwender interessante
Entwicklung dar.

Im Anschluss an diese einleitende Übersicht im Abschnitt~\ref{sec:1-einleitung} folgt
die Darstellung der wichtigsten Grundlagen in Abschnitt~\ref{sec:2-grundlagen}.
Aufbauend auf diesen Grundlagen folgt die konzeptuelle Ausarbeitung im Abschnitt~\ref{sec:3-konzeption}.
Die Umsetzung wird im Abschnitt~\ref{sec:4-implementierung} erläutert.
Die Leistungsfähigkeit der Implementierung wird in Abschnitt~\ref{sec:5-tests} untersucht.
Eine Zusammenfassung und ein Ausblick schließen die Arbeit in
Abschnitt~\ref{sec:6-fazit} ab. Eventuell noch benötigte Anhänge
finden sich in den Anhängen [...] bis [...].

% % % Imports nur für Referenzenauflösung während des Schreibens! Vorm Kompilieren auskommentieren!
% \bibliography{0_hauptdatei}
% \input{1_einleitung}
% \input{2_grundlagen}
% \input{3_konzeption}
% \input{4_implementierung}
% \input{5_tests}
% \input{6_zusammenfassung}
% % Ende Imports

\section{Grundlagen%
  \label{sec:2-grundlagen}}

\subsection{Speicherprogrammierbare-Steuerung und Linux -- Revolution Pi%
     \label{sec:2-sps}}

\subsubsection{Kunbus RevolutionPi%
        \label{sec:2-revpi}}
Der RevolutionPi 3 ist eine speicherprogrammierbare Steuerung (SPS) des Herstellers
Kunbus GmbH. Kern dieser SPS ist das von der Raspberry Pi Foundation entwickelte
und vertriebene Raspberry Pi Compute Module 3. Dieses integriert ein Broadcom BCM2837
System-on-Chip (SoC) mit vier 1,2GHz Prozessorkernen, 1GB RAM, 4GB eMMC Anwendungsspeicher
und sonstige Peripherie in ein Modul im DDR2-SODIMM Formfaktor. Diese Spezifikationen
sind weitgehend identisch zu denen des ausgesprochen populären Raspberry Pi 3.
Der Revolution Pi profitiert daher von dem gleichen großen Angebot an Software
und Unterstützung wie der Raspberry Pi, ergänzt dessen Hardware jedoch um eine 24V
Spannungsversorgung, die Möglichkeit der Erweiterung durch mehrere industrietaugliche
Ein-/ Ausgabemodule und Gateways sowie ein Gehäuse zur Montage auf einer DIN-Schiene.
\begin{itemize}
  \item{Prozessor: BCM2837}
  \item{Taktfrequenz 1,2 GHz}
  \item{Anzahl Prozessorkerne: 4}
  \item{Arbeitsspeicher: 1 GByte}
  \item{eMMC Flash Speicher: 4 GByte}
  \item{Betriebssystem: Angepasstes Raspbian mit RT-Patch}
  \item{RTC mit 24h Pufferung über wartungsfreien Kondensator}
  \item{Treiber / API: Treiber schreibt zyklisch Prozessdaten in ein Prozessabbild, Zugriff auf Prozessabbild über Linux-Filesystem als API zu Fremdsoftware.}
  \item{Kommunikationsanschlüsse: 2 x USB 2.0 A (je 500 mA belastbar), 1 x Micro-USB, HDMI, Ethernet (RJ45) 10/100 Mbit/s}
  \item{Stromversorgung: min. 10,7 V, max. 28,8 V, maximal 10 Watt}
  \item{Zulässige Umgebungstemperatur: -40 bis +55 C}
  \item{Gehäuseabmessungen: (HxBxL) 96 mm x 22,5 mm x 110,5 mm (ohne gesteckte Stecker)}
  \item{ESD Schutz: 4 kV / 8 kV gemäß EN61131-2 und IEC 61000-6-2}
  \item{Surge / Burst Prüfungen: gemäß EN61131-2 und IEC 61000-6-2 eingekoppelt auf Versorgungsspannung, Ethernet und IO-Leitungen}
  \item{EMI Prüfungen: gemäß EN61131-2 und IEC 61000-6-2}
\end{itemize}

Kunbus bietet eine Auswahl an IO- und Gateway-Modulen zur Erweiterung des Revolution Pi an.
Gateways dienen der Kommunikation mit Systemen oder Komponenten der Automatisierungstechnik
über Protokolle wie PROFIBUS oder EtherCAT. IO-Module erlauben die Überwachung
und Steuerung von digitalen oder analogen Ein- und Ausgängen.

\subsubsection{Zugriff auf IO-Module%
        \label{sec:2-io}}
Der Zugriff auf die Ein- und Ausgänge der IO-Module erfolgt über ein Prozessabbild
und einen hierfür von Kunbus bereitgestellten Treiber, genannt piControl. Dieser
aktualisiert das Prozessabbild zyklisch. Die angestrebte Zykluszeit beträgt 5ms,
kann jedoch je nach Anzahl der angeschlossenen Module auch größer sein. Kunbus
garantiert bei drei IO-Modulen und zwei Gateway-Modulen eine Zykluszeit von 10 ms.
Jedes der IO-Module stellt ein eigenständiges eingebettetes System dar. Es verfügt
über einen Microcontroller, welcher die IOs bereitstellt und über einen RS485-Bus
mit dem Revolution Pi kommuniziert.
% https://revolution.kunbus.de/io-modul/

Lizenz: GPL
% https://github.com/RevolutionPi/piControl

\begin{lstlisting}[language={c},firstnumber={226},caption={Setzen der Scheduler-Priorität auf SCHED\_FIFO in revpi\_common.c\label{lst:2-sched_priority}}]
param.sched_priority = ktprio->prio;
ret = sched_setscheduler(child, SCHED_FIFO,
       &param);
\end{lstlisting}


\subsection{Echtzeit und Multithreading unter Linux -- preemptRT und posix%
     \label{sec:2-echtzeit}}


 Der Linux-Kernel verfügt über mehrere unterschiedliche Preemtion-Modelle:

\begin{itemize}
  \item No Forced Preemption (server):
  Ausgelegt auf maximal möglichen Durchsatz, lediglich Interrupts und
  System-Call-Returns bewirken Präemption.

  \item Voluntary Kernel Preemption (Desktop):
  Neben den implizit bevorrechtigten Interrupts und System-Call-Returns gibt es
  in diesem Modell weitere Abschnitte des Kernels in welchen Preämption explizit
  gestattet ist.

  \item Preemptible Kernel (Low-Latency Desktop):
  In diesem Modell ist der gesamte Kernel, mit Ausnahme sog.~kritischer Abschnitte
  präemptible. Nach jedem kritischen Abschnitt gibt es einen impliziten Präemptions-Punkt.

  \item Preemptible Kernel (Basic RT):
  Dieses Modell ist dem zuvor genannten sehr ähnlich, hier sind jedoch alle Interrupt-Handler
  als eigenständige Threads ausgeführt.

  \item Fully Preemptible Kernel (RT):
  Wie auch bei den beiden zuvor genannten Modellen ist hier der gesamte Kernel
  präemtible, die Anzahl und Dauer der nicht-präemtiblen kritischen Abschnitte
  ist auf ein notwendiges Minimum beschränkt. Alle Interrupt-Handler sind als
  eigenständige Threads ausgeführt, Spinlocks durch Sleeping-Spinlocks und Mutexe
  durch sog.~RT-Mutexe ersetzt.

\end{itemize}
\todo{Spinlocks und Mutexe sowie die RT-Varianten dieser erklären!}

Lediglich mit dem vollständig präemtiblen Kernel kann Echtzeit-Verhalten realisiert werden.

% https://wiki.linuxfoundation.org/realtime/documentation/technical_basics/preemption_models bzw kernel/Kconfig.preempt

\subsubsection{preemptRT%
        \label{sec:2-preemptRT}}
% https://wiki.linuxfoundation.org/realtime/documentation/technical_details/start
% https://wiki.linuxfoundation.org/realtime/documentation/technical_basics/start

Das dem PREEMPT RT Kernel zugrunde liegende Prinzip lässt sich in einer einfachen
Regel ausdrücken: Nur Code, welcher absolut nicht-präemtible sein darf, ist es
gestattet nicht-präemtible zu sein.
Das erklärte Ziel des PREEMPT\_RT Patches ist es folglich, die Menge des nicht-präemtiblen
Codes im Linux-Kernel auf das absolut notwendige Minimum zu reduzieren.

Dies wird durch Verwendung folgender Mechanismen erreicht:

\begin{itemize}
  \item Hochauflösende Timer
  \item Sleeping Spinlocks
  \item Threaded Interrupt Handlers
  \item rt\_mutex
  \item RCU
\end{itemize}


\subsubsection{posix%
        \label{sec:2-posix}}
Ist posix hier wirklich relevant? Debian bzw.~Raspbian sind weitgehend posix
kompatibel, aber wird es hier genutzt? -> JA, open62541 nutzt pthread.h
piControl nutzt kthread.h, und semaphore.h

\subsection{OPC-UA und open62541%
     \label{sec:2-opc}}

\subsubsection{OPC UA%
        \label{sec:2-opcua}}
Open Platform Communications (OPC) ist eine Familie von Standards zur herstellerunabhängigen
Kommunikation von Maschinen (M2M) in der Automatisierungstechnik. Die sog.~OPC Task Force, zu deren
Mitgliedern verschiedene große Firmen der Automatisierungsindustrie gehören, veröffentlichte
die OPC Specification Version 1.0 im August 1996.
Motiviert ist dieser offene Standard durch die Erkenntniss, dass die Anpassung der
zahlreichen Herstellerstandards an individuelle Infrastrukturen und Anlagen einen
großen Mehraufwand verursachen.
Die Wikipedia beschreibt das Anwendungsgebiet für OPC wie folgt:

\glqq{}OPC wird dort eingesetzt, wo Sensoren, Regler und Steuerungen verschiedener Hersteller
ein gemeinsames Netzwerk bilden. Ohne OPC benötigten zwei Geräte zum Datenaustausch
genaue Kenntnis über die Kommunikationsmöglichkeiten des Gegenübers. Erweiterungen
und Austausch gestalten sich entsprechend schwierig. Mit OPC genügt es, für jedes
Gerät genau einmal einen OPC-konformen Treiber zu schreiben. Idealerweise wird
dieser bereits vom Hersteller zur Verfügung gestellt. Ein OPC-Treiber lässt sich
ohne großen Anpassungsaufwand in beliebig große Steuer- und Überwachungssysteme
integrieren.

OPC unterteilt sich in verschiedene Unterstandards, die für den jeweiligen Anwendungsfall
unabhängig voneinander implementiert werden können. OPC lässt sich damit verwenden
für Echtzeitdaten (Überwachung), Datenarchivierung, Alarm-Meldungen und neuerdings
auch direkt zur Steuerung (Befehlsübermittlung).\grqq{}

OPC basiert in der ursprünglichen Spezifikation auf Microsofts DCOM-Spezifikation.
DCOM macht Funktionen und Objekte einer Anwendung anderen Anwendungen im Netzwerk
zugänglich. Der OPC-Standard definiert entsprechende DCOM-Objekte um mit anderen
OPC-Anwendungen Daten austauschen zu können. Die Verwendung von DCOM bindet Anwender
an Betriebssysteme von Microsoft. Die ursprüngliche OPC Spezifikation wird durch die
Entwicklung von OPC Unified Architecture (OPC UA) abgelöst.
OPC UA setzt auf einem eigenen Kommunikationionsstack auf, die Verwendung von DCOM
und damit die Bindung an Microsoft wurden aufgelöst.

Die OPC-UA-Architektur ist eine Service-orientierte Architektur (SOA), deren Struktur
aus mehreren Schichten besteht.

% Wikipedia
Das OPC-Informationsmodell ist nicht mehr nur eine Hierarchie aus Ordnern, Items
und Properties. Es ist ein sogenanntes Full-Mesh-Network aus Nodes, mit dem neben
den Nutzdaten eines Nodes auch Meta- und Diagnoseinformationen repräsentiert werden.
Ein Node ähnelt einem Objekt aus der objektorientierten Programmierung. Ein Node
kann Attribute besitzen, die gelesen werden können (Data Access (DA), Historical
Data Access (HDA)). Es ist möglich Methoden zu definieren und aufzurufen.
Eine Methode besitzt Aufrufargumente und Rückgabewerte. Sie wird durch ein Command
aufgerufen. Weiterhin werden Events unterstützt, die versendet werden können
(AE (Alarms \& Events), DA DataChange), um bestimmte Informationen zwischen Geräten
auszutauschen. Ein Event besitzt unter anderem einen Empfangszeitpunkt, eine Nachricht
und einen Schweregrad. Die o. g. Nodes werden sowohl für die Nutzdaten als auch
alle anderen Arten von Metadaten verwendet. Der damit modellierte OPC-Adressraum
beinhaltet nun auch ein Typmodell, mit dem sämtliche Datentypen spezifiziert werden.

% https://de.wikipedia.org/wiki/Open_Platform_Communications
% https://de.wikipedia.org/wiki/OPC_Unified_Architecture
% https://opcfoundation.org/developer-tools/specifications-unified-architecture
% Von Gerhard Gappmeier - ascolab GmbH, CC BY-SA 3.0, https://de.wikipedia.org/w/index.php?curid=1892069
\subsubsection{open62541%
        \label{sec:2-open62541}}
open62541 ist eine offene und freie Implementierung von OPC UA. Die in C geschriebene
Bibliothek stellt eine beständig zunehmende Anzahl der im OPC UA Standard definierten
Funktionen bereit. Sie kann sowohl zur Erstellung von OPC-Servern als auch -Clients
genutzt werden. Ergänzend zu der unter der Mozilla Public License v2.0 lizensierten
Bibliothek stellt das open62541 Projekt auch Beispielprogramme unter einer CC0 Lizenz
zur Verfügung.

Die Bibliothek eignet sich auch für die Entwicklung auf eingebetteten Systemen und
Microcontrollern. Je nach Umfang der gewünschten Funktionen und des OPC Informationsmodells
beträgt die Größe einer Server-Binary weniger als 100kb. %evtl. kürzen?

\todo{Nodes erklären! Evtl.~oben!}

Folgende Auswahl an Eigenschaften und Funktionen zeichnet die in dieser Arbeit verwendete
Version 0.3 von open62541 aus:
\begin{itemize}
  \item Kommunikationionsstack
  \begin{itemize}
      \item OPC UA Binär-Protokoll (HTTP oder SOAP werden gegenwärtig nicht unterstützt)
      \item Austauschbare Netzwerk-Schicht, welche die Verwendung eigener Netzwerk-APIs
      erlaubt.
      \item Verschlüsselte Kommunikationion
      \item Asynchrone Dienst-Anfragen im Client
  \end{itemize}
  \item Informationsmodell
  \begin{itemize}
    \item Unterstützung aller OPC UA Node-Typen, inkl.~Methoden
    \item Hinzufügen und Entfernen von Nodes und Referenzen zur Laufzeit.
    \item Vererbung und Instanziierung von Objekt- und Variablentypen
    \item Zugriffskontrolle auch für einzelne Nodes
  \end{itemize}
  \item Subscriptions
  \begin{itemize}
    \item Erlaubt die Überwachung (subscriptions / monitoreditems)
    \item Sehr geringer Ressourcenbedarf pro überwachtem Wert
  \end{itemize}
  \item Code-Generierung auf XML-Basis
  \begin{itemize}
    \item Erlaubt die Erstellung von Datentypen
    \item Erlaubt die Generierung des serverseitigen Informationsmodells
  \end{itemize}
\end{itemize}

% https://open62541.org/doc/0.3/


Mozilla Public License
CC0 Lizenz für Beispiele und Plugins

% https://open62541.org/doc/open62541-current.pdf
% https://open62541.org/

% % % Imports nur für Referenzenauflösung während des Schreibens! Vorm Kompilieren auskommentieren!
% \bibliography{0_hauptdatei}
% \input{1_einleitung}
% \input{2_grundlagen}
% \input{3_konzeption}
% \input{4_implementierung}
% \input{5_tests}
% \input{6_zusammenfassung}
% \input{anhang}
% % Ende Imports

\section{Systemkonzept%
  \label{sec:3-konzeption}}
Auf Basis der in Abschnitt \ref{sec:2-grundlagen} vorgestellten Möglichkeiten folgt nun die Ausarbeitung eines Konzepts.
In den folgenden Abschnitten soll näher auf zwei zentrale Aspekte eingegangen werden: Abschnitt~\ref{sec:3-anbindung} stellt Möglichkeiten zum Zugriff auf Variablen bzw.\,Werte im Prozessabbild des Revolution Pi vor; in Abschnitt~\ref{sec:3-integration} wird ein Konzept zur Bereitstellung dieser Variablen auf einem OPC-Server vorgestellt.

\subsection{Anbindung der IO an den OPC-Server%
     \label{sec:3-anbindung}}

Eine Webanwendung mit Bezeichnung PiCtory dient zur Konfiguration der I/O- und virtuellen Module des RevolutionPi. Die Konfiguration liegt im JSON-Format in der Datei \lstinline{/etc/revpi/config.rsc}. Der piControl-Treiber liest diese Datei beim Start. 
Der folgende Auszug aus der Manpage des piControl-Kernelmoduls beschreibt die von diesem zum Lesen und Schreiben einzelner Bits des Prozessabbildes bereitgestellten Funktionen~\citep[vgl.]{web-revpi-manpage}. Sie ist an dieser Stelle weitgehend ungekürzt zitiert, da sie die nutzbare Schnittstelle sehr kompakt beschreibt.

\begin{lstlisting}[breakindent=0pt, numbers=none, caption={Auszug aus der Revolution Pi Programmers Manual\label{lst:4-manpage}}]
KB_FIND_VARIABLE SPIVariable *argp
Find a variable in the process image by its name. A pointer to a structure of type SPIVariable must be passed as argument. [...]
The struct SPIVariable [...] is defined as 
typedef struct SPIVariableStr
{
    char strVarName[32]; // Variable name
    uint16_t i16uAddress; // Address of the byte in the process image
    uint8_t i8uBit; // 0-7 bit position, >= 8 whole byte
    uint16_t i16uLength; // length of the variable in bits.
    // Possible values are 1, 8, 16 and 32
} SPIVariable;

Set and get values of the process image
KB_GET_VALUE SPIValue *argp
[...]
KB_SET_VALUE SPIValue *argp
Write one bit or one byte to the process image [...].  This call is more efficient than the usual calls of seek and write because only one function call is necessary. If more than on application are writing bits in one output byte, this call is the only safe way to set a bit without overwriting the other bits because this call is doing a read-modify-write-cycle. 

The struct SPIValue used by this ioctl is defined as
typedef struct SPIValueStr
{
    uint16_t i16uAddress; // Address of the byte in the process image
    uint8_t i8uBit; // 0-7 bit position, >= 8 whole byte
    uint8_t i8uValue; // Value: 0/1 for bit access, whole byte otherwise
} SPIValue;
\end{lstlisting} 

Die oben beschriebenden Funtkionen \lstinline{KB_FIND_VARIABLE}, \lstinline{KB_GET_VALUE} und \lstinline{KB_SET_VALUE} ermöglichen einen einfachen und (lt.\,Manpage) effizienten Zugriff auf einzelne Bits des Prozessabbildes und damit auch auf die IO des RevolutionPi.
Der Zugriff des OPC-Servers auf das Prozessabbild soll daher mittels dieser Funktionen realisiert werden.
\lstinline{KB_FIND_VARIABLE} kann genutzt werden, um Adressen von Variablen im Prozessabbild mittels ihres Namens aufzulösen.
\lstinline{KB_GET_VALUE} und \lstinline{KB_SET_VALUE} ermöglichen den Zugriff auf die Werte dieser Variablen.


\subsection{Integration des OPC-Servers in das System%
     \label{sec:3-integration}}

open62541 bietet drei Möglichkeiten zum Abgleich von Variablen mit dem Prozessabbild~\citep[vgl.][Tutorials - Connecting a Variable with a Physical Process]{web-open62541}:
\begin{itemize}
    \item Manuelles oder zyklisches Aktualisieren
    \item Variable Value Callback
    \item Variable Datasource
\end{itemize}

Die zyklische Aktualisierung eines oder mehrerer Werte nimmt, abhängig von der Zykluszeit, viele Systemressourcen in Anspruch. Value Callbacks ermöglichen es, einen Variablenwert effizienter mit einer Ressource wie etwa einem Prozessabbild zu synchronisieren. An die Variable wird ein Callback angehängt, welches vor jedem Lesen und nach jedem Schreibvorgang ausgeführt wird.
Der Wert der Variablen wird weiterhin im Variablenknoten auf dem OPC-Server gespeichert, der Abgleich mit der verknüpften Ressource erfolgt durch die Callback-Methoden.

Sogenannte Datenquellen gehen noch einen Schritt weiter. Der Server leitet jede Lese- und Schreibanforderung direkt an eine Callback-Funktion weiter. Beim Lesen liefert der Rückruf eine Kopie des aktuellen Wertes. Die Datenquelle muss intern ein eigenes Speichermanagement implementieren.

Der Zugriff auf die Werte des Prozessabbildes erfolgt, wie in Abschnitt~\ref{sec:3-anbindung} beschrieben, über von piControl bereitgestellte Methoden. Um die durch open62541 gepflegte OPC-Datenstruktur und das durch piControl verwaltete Prozessabbild möglichst effektiv verknüpfen zu können, soll diese Interaktion mittels Datenquellen und den zugehörigen Callbacks implementiert werden.
% % % Imports nur für Referenzenauflösung während des Schreibens! Vorm Kompilieren auskommentieren!
% \bibliography{0_hauptdatei}
% \input{1_einleitung}
% \input{2_grundlagen}
% \input{3_konzeption}
% \input{4_implementierung}
% \input{5_tests}
% \input{6_zusammenfassung}
% \input{anhang}
% % Ende Imports

\section{Implementierung%
  \label{sec:4-implementierung}}
Das folgende Kapitel stellt in Auszügen die Implementierung des OPC-Servers sowie die Anbindung an die IO-Module
der SPS dar. Der Schwerpunkt liegt hierbei auf der Funktionsweise des piControl-Treibers und dessen Integration in das Projekt. Abschnitt~\ref{sec:4-picontrol} erklärt die zum Schreibens eines Bits verwendeten Funktionsaufrufe.
Zuvor soll jedoch in Abschnitt~\ref{sec:4-open62541} der Teil des OPC-Servers vorgestellt werden, welcher auf besagten Treiber zugreift. 

\subsection{Implementierung des OPC-Servers%
     \label{sec:4-open62541}}
Wie im vorangegangenen Abschnitt~\ref{sec:3-integration} begründet, soll die Verknüpfung zwischen dem Prozessabbild der SPS und den auf dem OPC-Server bereitgestellten Werten über sog.\,Datenquellen erfolgen. Hierzu ist zunächst eine Callback-Methode zu implementieren, welche bei einem Lese- oder Schreibzugriff auf eine Variable aufgerufen wird. Die Verknüpfung zwischen Callback-Methode und Variable muss manuell erfolgen.

\begin{lstlisting}[language={c},firstnumber=237,caption={Auszug der Methode \lstinline{linkDataSourceVariable} in \lstinline{variables.c}\label{lst:4-linkDataSourceVariable}}]
extern UA_StatusCode
 linkDataSourceVariable(UA_Server *server, UA_NodeId nodeId) {
     bool readonly = false;
     UA_DataSource dataSourceVariable;
     UA_StatusCode rc; |>\setcounter{lstnumber}{254}<|

     dataSourceVariable.read = readDataSourceVariable;
     if (!readonly)
        dataSourceVariable.write = writeDataSourceVariable;
     else
        dataSourceVariable.write = writeReadonlyDataSourceVariable;

     return UA_Server_setVariableNode_dataSource(server, nodeId, dataSourceVariable);
 }
\end{lstlisting}

\begin{figure}[h]
    \centering
    \includegraphics[width=0.42\textwidth]{doc/img/OPC_RevPiDO.pdf}
    \caption{Auszug des verwendeten Nodesets, hier Digitalausgang 1 des Versuchsaufbaus
      \label{fig:opc-do}}
\end{figure}

Die in Listing~\ref{lst:4-linkDataSourceVariable} abgebildete Methode \lstinline{linkDataSourceVariable()} erzeugt ein Struct vom Typ \lstinline{UA_DataSource}. In diesem werden dem Lesen und Schreiben einer OPC-Variablen entsprechende Callback-Methoden zugewiesen. Die Verknüpfung einer OPC-Variable, genauer ihrer NodeId, mit der zuvor definierten Datenquelle erfolgt über die von open62541 bereitgestellte Methode \lstinline{UA_Server_setVariableNode_dataSource()}. Vor dem Lesen und nach dem Schreiben dieser Variable werden von nun an die entsprechenden Callbacks aufgerufen.
     
\begin{lstlisting}[language={c},firstnumber=168,caption={Auszug des Callbacks \lstinline{writeDataSourceVariable} in \lstinline{variables.c}\label{lst:4-writeDataSourceVariable}}]  
extern UA_StatusCode
 writeDataSourceVariable(UA_Server *server,
            const UA_NodeId *sessionId, void *sessionContext,
            const UA_NodeId *nodeId, void *nodeContext,
            const UA_NumericRange *range, const UA_DataValue *dataValue) {

    UA_StatusCode retval  = UA_STATUSCODE_GOOD;
    UA_NodeId *nameNodeId = UA_malloc(sizeof(UA_NodeId));
    UA_QualifiedName nameQN = UA_QUALIFIEDNAME(1, "Name");
    UA_Variant nameVar;
    UA_Boolean bit;

    retval |= findSiblingByBrowsename(server, nodeId, &nameQN, nameNodeId);
    retval |= UA_Server_readValue(server, *nameNodeId, &nameVar);
    retval |= UA_Boolean_copy(dataValue->value.data, &bit);

    |>\tikzmarkin[set border color=martinired]{writeIO}<|PI_writeSingleIO(String_fromUA_String(nameVar.data), &bit, false);                                                 |>\tikzmarkend{writeIO}<|

    free(nameNodeId);
    return retval;
 }
\end{lstlisting}

Listing~\ref{lst:4-writeDataSourceVariable} zeigt die Callback-Methode, welche nach dem Schreiben einer Variablen auf dem OPC-Server aufgerufen wird.
Dieser Methode wird neben der NodeId der mit ihr verknüpften Variablen auch der Wert dieser in Form eines Zeigers auf ein Struct vom Typ \lstinline{UA_DataValue} übergeben.

Die Gestaltung des hier verwendeten Nodesets sieht vor, dass in einer OPC-Variablen \lstinline{"Name"} der Bezeichner des zu schreibenden Digitalausgangs hinterlegt ist, siehe Abbildung~\ref{fig:opc-do}. Dies erlaubt eine Rekonfiguration der Ein- und Ausgänge der SPS ohne Änderungen im Programmcode des OPC-Servers vornehmen zu müssen.
Es ist daher erforderlich, nach jedem Schreiben einer mit einem Digitalausgang verknüpften Variablen, hier \lstinline{"Value"}, dessen Bezeichner \lstinline{"Name"} abzufragen. 
Dies geschieht in den Zeilen 180 und 181.
Anschließend wird dieser Bezeichner sowie der zu schreibende Wert der Methode \lstinline{PI_writeSingleIO()} übergeben, welche wiederum die Interaktion mit piControl übernimmt (vgl. Abschnitt \ref{sec:4-picontrol}).
 
\subsection{Integration von piControl%
     \label{sec:4-picontrol}}
In Abschnitt~\ref{sec:2-io} wurde die Anbindung der IO-Module des Revolution Pi sowie die Funktionsweise von piControl aus Anwendersicht beschrieben. Die verfügbare Literatur beschränkt sich auch auf lediglich diese Sicht; eine weiterführende Dokumentation für Entwickler gibt es, neben der in Abschnitt~\ref{sec:3-anbindung} vorgestellten Manpage, nicht. 
In diesem Abschnitt soll daher der Quellcode von piControl sowie dessen Verwendung im Projekt genauer betrachtet werden.
Hierzu wird exemplarisch die in Abschnitt~\ref{sec:4-open62541} eingeführte Methode \lstinline{PI_writeSingleIO()} untersucht.
Diese Methode ermöglicht das Setzen eines einzelnen Bits im Prozessabbild der SPS, und damit das Schalten eines digitalen Ausgangs auf einem IO-Modul.
Die äquivalente Methode \lstinline{int piControlGetBitValue(SPIValue *pSpiValue)} zum Lesen eines Bits bzw. Eingangs funktioniert analog und soll daher an dieser Stelle nicht dediziert erörtert werden.

\begin{lstlisting}[language={c},firstnumber=97,
                   caption={Setzen eines phsikalischen, digitalen Ausgangs in \lstinline{revpi.c}
                   \label{lst:4-PI_writeSingleIO}}]
extern void PI_writeSingleIO(char *pszVariableName, bool *bit, bool verbose)
{
	int rc;
	SPIVariable sPiVariable;
	SPIValue sPIValue;

	strncpy(sPiVariable.strVarName, pszVariableName, sizeof(sPiVariable.strVarName));
	rc = piControlGetVariableInfo(&sPiVariable);
	if (rc < 0) {
		printf("Cannot find variable '%s'\n", pszVariableName);
		return;
	}

		sPIValue.i16uAddress = sPiVariable.i16uAddress;
		sPIValue.i8uBit = sPiVariable.i8uBit;
		sPIValue.i8uValue = *bit;
		rc = |>\tikzmarkin[set border color=martinired]{setBitValue}<|piControlSetBitValue(&sPIValue)|>\tikzmarkend{setBitValue}<|;
		if (rc < 0)
			printf("Set bit error %s\n", getWriteError(rc));
		else if (verbose)
			printf("Set bit %d on byte at offset %d. Value %d\n", sPIValue.i8uBit, sPIValue.i16uAddress,
			       sPIValue.i8uValue);
}
\end{lstlisting}

Der Programmcode in Listing~\ref{lst:4-PI_writeSingleIO} ist Teil des implementierten OPC-Servers. In diesem wird auf zwei Funktionen des piControl-Treibers zugegriffen. 
Beiden Methoden wird als Argument ein Zeiger auf ein Struct vom Typ \lstinline{SPIValue} übergeben. Der im Struct abgelegte Name wird mittels \lstinline{piControlGetVariableInfo(&sPIValue)} zu einer Adresse im Prozessabbild aufgelöst. Diese wird in \lstinline{sPIValue.i16uAdress} gespeichert. Der Wert der Variablen wird anschließend mittels \lstinline{piControlSetBitValue(&sPIValue)} an dieser Adresse in das Prozessabbild geschrieben.

\begin{lstlisting}[language={c},firstnumber=309,caption={Methode \lstinline{piControlSetBitValue} in \lstinline{piControlIf.c}\label{lst:4-piControlSetBitValue}}]
int |>\tikzmarkin[set border color=martiniblue]{setBitValueFcn}<|piControlSetBitValue(SPIValue *pSpiValue)|>\tikzmarkend{setBitValueFcn}<|
{
    piControlOpen();

    if (PiControlHandle_g < 0)
	    return -ENODEV;

    pSpiValue->i16uAddress += pSpiValue->i8uBit / 8;
    pSpiValue->i8uBit %= 8;

    if (|>\tikzmarkin[set border color=martinired]{ioctl}<|ioctl(PiControlHandle_g, KB_SET_VALUE, pSpiValue)|>\tikzmarkend{ioctl}<| < 0)
	    return errno;

    return 0;
}
\end{lstlisting}

Die in Listing~\ref{lst:4-piControlSetBitValue} dargestellte Methode \lstinline{piControlSetBitValue} ist lediglich eine Hüllfunktion (häufig auch als Wrapper-Funktion bezeichnet) für einen Aufruf des \lstinline{ioctl} Kernel-Moduls.
Folgende Parameter werden übergeben:
\lstinline{PiControlHandle_g} ist die Referenz auf die Geräte-Datei des piControl-Treibers. \lstinline{KB_SET_VALUE} ist das ioctl-Kommando zum Schreiben eines Bits in das Prozessabbild. Der Zeiger \lstinline{pSpiValue} verweist auf ein Struct des bereits vorgestellten Typs \lstinline{SPIValue}.

\begin{lstlisting}[language={c},firstnumber=80,caption={Methode \lstinline{piControlOpen} in \lstinline{piControlIf.c}\label{lst:4-piControlOpen}}]
void piControlOpen(void)
{
    /* open handle if needed */
    if (PiControlHandle_g < 0)
    {
	    |>\tikzmarkin[set border color=martiniblue]{PiControlHandle}<|PiControlHandle_g = open(PICONTROL_DEVICE, O_RDWR)|>\tikzmarkend{PiControlHandle}<|;
    }
}
\end{lstlisting}

Die in Listing~\ref{lst:4-piControlOpen} dargestellte Methode öffnet, sofern nicht bereits geschehen, die Geräte-Datei. Das Macro \lstinline{PICONTROL_DEVICE} verweist hierbei auf \lstinline{/dev/piControl0}.

\begin{lstlisting}[language={c},firstnumber=721,caption={Methode \lstinline{piControlIoctl} in \lstinline{piControlMain.c}\label{lst:4-piControlIoctl}}]
static long |>\tikzmarkin[set border color=martiniblue, below offset=0.9em]{piControlIoctl}<|piControlIoctl(struct file *file, unsigned int prg_nr, 
                           unsigned long usr_addr)                                      |>\tikzmarkend{piControlIoctl}<|
{
  int status = -EFAULT;
  tpiControlInst *priv;
  int timeout = 10000;	// ms

  if (prg_nr != KB_CONFIG_SEND && prg_nr != KB_CONFIG_START && !isRunning()) {
  	return -EAGAIN;
  }

  priv = (tpiControlInst *) file->private_data;

  if (prg_nr != KB_GET_LAST_MESSAGE) {
  	// clear old message
  	priv->pcErrorMessage[0] = 0;
  }

  switch (prg_nr) {|>\setcounter{lstnumber}{864}<|

    case |>\tikzmarkin[set border color=martiniblue]{KB_SET_VALUE}<|KB_SET_VALUE:|>\tikzmarkend{KB_SET_VALUE}<|
  		{
  			SPIValue *pValue = (SPIValue *) usr_addr;

  			if (!isRunning())
  				return -EFAULT;

  			if (pValue->i16uAddress >= KB_PI_LEN) {
  				status = -EFAULT;
  			} else {
  				INT8U i8uValue_l;
  				my_rt_mutex_lock(&piDev_g.lockPI);
  				i8uValue_l = piDev_g.ai8uPI[pValue->i16uAddress];

  				if (pValue->i8uBit >= 8) {
  					i8uValue_l = pValue->i8uValue;
  				} else {
  					if (pValue->i8uValue)
  						i8uValue_l |= (1 << pValue->i8uBit);
  					else
  						i8uValue_l &= ~(1 << pValue->i8uBit);
  				}

  				|>\tikzmarkin[set border color=martinired]{i8uValue}<|piDev_g.ai8uPI[pValue->i16uAddress] = i8uValue_l;|>\tikzmarkend{i8uValue}<|
  				rt_mutex_unlock(&piDev_g.lockPI);

  #ifdef VERBOSE
  				pr_info("piControlIoctl Addr=%u, bit=%u: %02x %02x\n", pValue->i16uAddress, pValue->i8uBit, pValue->i8uValue, i8uValue_l);
  #endif

  				status = 0;
  			}
  		}
  		break; |>\setcounter{lstnumber}{1314}<|

    default:
      pr_err("Invalid Ioctl");
      return (-EINVAL);
      break;

    }

    return status;
  }
\end{lstlisting}

Listing~\ref{lst:4-piControlIoctl} zeigt in Auszügen die ioctl-Methode des piControl Kernel-Treibers. Diese bekommt folgende Argumente übergeben: \lstinline{struct file *file} enthält den Verweis auf die Geräte-Datei, hier \lstinline{/dev/piControl0}. Der Wert von \lstinline{unsigned int prg_nr} beschreibt die Anfrage an den Treiber, in diesem Fall \lstinline{KB_SET_VALUE}. Das Argument \lstinline{unsigned long usr_addr} enthält einen typ-agnostischen Pointer. Dieser verweist auf einen Speicherbereich, in welchem die zur Bearbeitung der Anfrage notwendigen Daten abgelegt sind. Hier können auch vom Treiber empfangene Daten dem Anwendungsprogramm bereitgestellt werden. 

Die switch-case-Anweisung führt die über das Argument \lstinline{prg_nr} spezifizierte Aktion aus. Hier betrachten wir \lstinline{KB_SET_VALUE}:
Zunächst wird in Zeile 868 der übergebene Zeiger \lstinline{usr_addr} mittels explizitem Typecast zu einem Zeiger des Typs \lstinline{SPIValue *} konvertiert. Da dieser auf Daten im Userspace verweist, ist beim Zugriff durch den Kernel-Treiber besondere Vorsicht geboten.
In Zeile 877 wird mittels Mutex das Prozessabbild \lstinline{piDev_g} für den Zugriff durch andere Threads oder Prozesse gesperrt.
\lstinline{my_rt_mutex_lock} verweist hierbei auf die Funktion \lstinline{rt_mutex_lock} aus \lstinline{linux/sched.h}\footnote{Offenbar wurde hier auch eine alternative Implementierung vorgesehen, siehe revpi\_common.h}

In Zeile 889 wird das Byte \lstinline{i8uValue_l}, welches den zu schreibenden Wert enthält in das Prozessabbild übertragen. Anschließend wird die Mutex auf \lstinline{piDev_g} wieder entsperrt.
\newpage

\begin{lstlisting}[language={c},firstnumber=62,caption={Auszug des Struct \lstinline{spiControlDev} in \lstinline{piControlMain.h}\label{lst:4-spiControlDev}}]
|>\tikzmarkin[set border color=martiniblue]{spiControlDev}<|typedef struct spiControlDev|>\tikzmarkend{spiControlDev}<| {
	// device driver stuff
	int init_step;
	enum revpi_machine machine_type;
	void *machine;
	struct cdev cdev;	// Char device structure
	struct device *dev;
	struct thermal_zone_device *thermal_zone;

	|>\tikzmarkin[set border color=martiniblue]{processImage}<|// process image stuff
	INT8U ai8uPI[KB_PI_LEN];
	INT8U ai8uPIDefault|>\tikzmarkin[set border color=martinired]{KB_PI_LEN_0}<|[KB_PI_LEN]|>\tikzmarkend{KB_PI_LEN_0}<|;
	struct rt_mutex lockPI;        |>\tikzmarkend{processImage}<|
	bool stopIO;
	piDevices *devs; |>\setcounter{lstnumber}{94}<|
} tpiControlDev;
\end{lstlisting}

Das Prozessabbild ist als Byte-Array der Länge \lstinline{KB_PI_LEN} in Listing~\ref{lst:4-spiControlDev} definiert. Konfigurationsparameter wie \lstinline{KB_PI_LEN} oder die Zykluszeit für den Datenaustausch zwischen SPS und IO-Modulen sind im folgenden Listing~\ref{lst:4-process} definiert.

\begin{lstlisting}[language={c},firstnumber=119,caption={Konfigurationsparameter des Prozessabbildes in project.h\label{lst:4-process}}]
#define INTERVAL_PI_GATE (5*1000*1000)  // 5 ms piGateCommunication |>\setcounter{lstnumber}{128}<|

#define INTERVAL_IO_COM (5*1000*1000)  // 5 ms piIoComm |>\setcounter{lstnumber}{132}<|

#define KB_PD_LEN       512
|>\tikzmarkin[set border color=martiniblue]{KB_PI_LEN_1}<|#define KB_PI_LEN       4096|>\tikzmarkend{KB_PI_LEN_1}<|
\end{lstlisting}

Das zu setzende Bit wurde zu diesem Zeitpunkt erfolgreich in das Prozessabbild der SPS geschrieben.
Es stellt sich die Frage, wie dieses nun an das IO-Modul kommuniziert wird.
Die Kommunikation mit allen angebundenen Modulen ist ebenfalls Aufgabe des piControl-Treibers.

\begin{lstlisting}[language={c},firstnumber=256,caption={Auszug der Methode \lstinline{piIoThread} in \lstinline{revpi_core.c}\label{lst:4-piIoThread}}]
static int piIoThread(void *data)
{
	//TODO int value = 0;
	ktime_t time;
	ktime_t now;
	s64 tDiff;

	hrtimer_init(&piCore_g.ioTimer, CLOCK_MONOTONIC, HRTIMER_MODE_ABS);
	piCore_g.ioTimer.function = piIoTimer;

	pr_info("piIO thread started\n");

	now = hrtimer_cb_get_time(&piCore_g.ioTimer);

	PiBridgeMaster_Reset();

	while (!kthread_should_stop()) {
		if (|>\tikzmarkin[set border color=martinired]{PiBridgeMaster}<|PiBridgeMaster_Run()|>\tikzmarkend{PiBridgeMaster}<| < 0)
			break;
	}

	RevPiDevice_finish();

	pr_info("piIO exit\n");
	return 0;
}
\end{lstlisting}

Der Kernel-Thread \lstinline{piIoThread} ist verantwortlich für den zyklischen Datenaustausch mit den IO-Modulen. In diesem wird fortlaufend die Methode \lstinline{PiBridgeMaster_Run()} aufgerufen, siehe Listing~\ref{lst:4-piIoThread}.

\begin{lstlisting}[language={c},firstnumber=262,caption={Auszug der Methode \lstinline{PiBridgeMaster_Run(void)} in \lstinline{RevPiDevice.c}\label{lst:4-PiBridgeMaster_Run}}]
int PiBridgeMaster_Run(void)
{
	static kbUT_Timer tTimeoutTimer_s;
	static kbUT_Timer tConfigTimeoutTimer_s;
	static int error_cnt;
	static INT8U last_led;
	static unsigned long last_update;
	int ret = 0;
	int i;

	my_rt_mutex_lock(&piCore_g.lockBridgeState);
	if (piCore_g.eBridgeState != piBridgeStop) {
		switch (eRunStatus_s) { |>\setcounter{lstnumber}{514}<|
		    case enPiBridgeMasterStatus_EndOfConfig:|>\setcounter{lstnumber}{621}<|
		    if (|>\tikzmarkin[set border color=martinired]{RevPiDevice}<|RevPiDevice_run()|>\tikzmarkend{RevPiDevice}<|) {
				// an error occured, check error limits |>\setcounter{lstnumber}{641}<|
			} else {
				ret = 1;
			}
			piCore_g.image.drv.i16uRS485ErrorCnt = RevPiDevice_getErrCnt();
			break;
\end{lstlisting}

Die in Listing~\ref{lst:4-PiBridgeMaster_Run} dargestellte Methode ist eine sog. State-Machine. Ist die Konfiguration der IO-Module erfolgreich abgeschlossen, so führt sie bei Aufruf lediglich die Methode \lstinline{RevPiDevice_run()} aus.

\begin{lstlisting}[language={c},firstnumber=140,caption={Auszug der Methode \lstinline{RevPiDevice_run(void)} in \lstinline{RevPiDevice.c}\label{lst:4-RevPiDevice_run}}]
int RevPiDevice_run(void)
{
	INT8U i8uDevice = 0;
	INT32U r;
	int retval = 0;

	RevPiDevices_s.i16uErrorCnt = 0;

	for (i8uDevice = 0; i8uDevice < RevPiDevice_getDevCnt(); i8uDevice++) {
		if (RevPiDevice_getDev(i8uDevice)->i8uActive) {
			switch (RevPiDevice_getDev(i8uDevice)->sId.i16uModulType) {
			case KUNBUS_FW_DESCR_TYP_PI_DIO_14:
			case KUNBUS_FW_DESCR_TYP_PI_DI_16:
			case KUNBUS_FW_DESCR_TYP_PI_DO_16:
				r = |>\tikzmarkin[set border color=martinired]{sendCyclicTelegram}<|piDIOComm_sendCyclicTelegram(i8uDevice)|>\tikzmarkend{sendCyclicTelegram}\setcounter{lstnumber}{166} <|;

				break; |>\setcounter{lstnumber}{216}<|
			}
		}
	} |>\setcounter{lstnumber}{227}<|
	return retval;
}
\end{lstlisting}

Diese iteriert wie in Listing~\ref{lst:4-RevPiDevice_run} abgebildete durch alle gegenwärtig in der SPS konfigurierten Module. Ist das aktuelle Modul als aktiv markiert, so wird anhand eines sog. Firmware-Descriptors entschieden, welche Methode für die Ansteuerung des Moduls aufzurufen ist.

\begin{lstlisting}[language={c},firstnumber=161,caption={Auszug der Methode \lstinline{piDIOComm_sendCyclicTelegram} in \lstinline{piDIOComm.c}\label{lst:4-sendCyclicTelegram}}]
INT32U piDIOComm_sendCyclicTelegram(INT8U i8uDevice_p)
{
	INT32U i32uRv_l = 0;
	SIOGeneric sRequest_l;
	SIOGeneric sResponse_l;
	INT8U len_l, data_out[18], i, p, data_in[70];
	INT8U i8uAddress;
	int ret; |>\setcounter{lstnumber}{239}<|
	
    |>\tikzmarkin[set border color=martinired]{piIoComm}<|ret = piIoComm_send((INT8U *) & sRequest_l, IOPROTOCOL_HEADER_LENGTH + len_l + 1);  |>\tikzmarkend{piIoComm}\setcounter{lstnumber}{298}<|
}
\end{lstlisting}

Im Falle des hier verwendeten DO-Moduls wird die in Listing~\ref{lst:4-sendCyclicTelegram} abgebildete Methode \lstinline{piDIOComm_sendCyclicTelegram()} aufgerufen. Dieser wird ein Zeiger auf das zu schreibende Gerät übergeben. 
Zunächst wird das Prozessabbild mittels eines proprietären, jedoch im Quellcode offen nachvollziehbaren Protokolls in ein \lstinline{sRequest_l} genanntes Byte-Array umgewandelt. Dieser Schritt ist in Listing~\ref{lst:4-sendCyclicTelegram} nicht abgebildet. Anschließend wird \lstinline{piIoComm_send()} ein Zeiger auf die so generierte Schreib-Anfrage übergeben.

\begin{lstlisting}[language={c},firstnumber=220,caption={Auszug der Methode \lstinline{piIOComm_send} in \lstinline{piIOComm.c}\label{lst:4-piIOComm_send}}]
int piIoComm_send(INT8U * buf_p, INT16U i16uLen_p)
{
	ssize_t write_l = 0;
	INT16U i16uSent_l = 0;|>\setcounter{lstnumber}{249}<|

	while (i16uSent_l < i16uLen_p) {
		write_l = vfs_write(piIoComm_fd_m, buf_p + i16uSent_l, i16uLen_p - i16uSent_l, &piIoComm_fd_m->f_pos);
		if (write_l < 0) {
			pr_info_serial("write error %d\n", (int)write_l);
			return -1;
		} 
		i16uSent_l += write_l;|>\setcounter{lstnumber}{263}<|
	}
	clear();
	vfs_fsync(piIoComm_fd_m, 1);
	return 0;
}
\end{lstlisting}

Listing~\ref{lst:4-piIOComm_send} zeigt die Implementierung von \lstinline{piIoComm_send()}. Diese Methode ist für das Schreiben der oben generierten Anfrage auf die seriellen Schnittstelle verantwortlich. Realisiert wird dies mittels der Methode \lstinline{vfs_write()}. Diese ist in \lstinline{<linux/fs.h>} definiert. Sie ermöglicht das Schreiben einer Datei im Userspace aus dem Kernel heraus. Geschrieben wird hier die Datei mit dem Deskriptor \lstinline{piIoComm_fd_m}.
Da die Funktion \lstinline{vfs_write()} durch andere Kernel-Tasks unterbrochen werden kann, ist nicht gewährleistet, dass die gesamte Anfrage mit nur einem Aufruf geschrieben wird. Die oben abgebildete while-Schleife stellt das vollständige Senden der Anfrage sicher.

\begin{lstlisting}[language={c},firstnumber=157,caption={Auszug der Methode \lstinline{piIOComm_open_serial} in \lstinline{piIOComm.c}\label{lst:4-piIOComm_open_serial}}]
int piIoComm_open_serial(void)
{   |>\setcounter{lstnumber}{167}<|
	struct file *fd;	/* Filedeskriptor */
	struct termios newtio;	/* Schnittstellenoptionen */

	|>\tikzmarkin[set border color=martiniblue]{fd}<|/* Port oeffnen - read/write, kein "controlling tty", 
	    Status von DCD ignorieren */
	fd = filp_open(|>\tikzmarkin[set border color=martinired]{tty}<|REV_PI_TTY_DEVICE|>\tikzmarkend{tty}<|, O_RDWR | O_NOCTTY, 0); |>\setcounter{lstnumber}{208}<|
	
	piIoComm_fd_m = fd;                                                      |>\tikzmarkend{fd}\setcounter{lstnumber}{217}<|

	return 0;
}
\end{lstlisting}

Der zum Schreiben auf die serielle Schnittstelle verwendete Datei-Deskriptor wird von der in Listing~\ref{lst:4-piIOComm_open_serial} abgebildeten Methode \lstinline{piIoComm_open_serial()} generiert. 

\begin{lstlisting}[language={c},firstnumber=45,caption={Definition der seriellen Schnittstelle in \lstinline{piIOComm.h}\label{lst:4-REV_PI_TTY_DEVICE}}]
#define REV_PI_TTY_DEVICE	"/dev/ttyAMA0"
\end{lstlisting}

Das in Listing~\ref{lst:4-REV_PI_TTY_DEVICE} definierte Macro verweist auf eine der seriellen Schnittstellen des RaspberryPi.
Die Implementierung des zugehörigen Schnittstellentreibers soll hier nicht weiter untersucht werden. Somit ist an dieser Stelle die Kette vom Setzen einer Variablen auf dem OPC-Server bis hin zur Aktualisierung des Prozessabbilds der IO-Module geschlossen.

% \begin{lstlisting}[language={c},firstnumber={226},caption={Setzen der Scheduler-Priorität auf SCHED\_FIFO in 
% revpi\_common.c\label{lst:2-sched_priority}}]
% param.sched_priority = ktprio->prio;
% ret = sched_setscheduler(child, SCHED_FIFO, &param);
% \end{lstlisting}
% % % Imports nur für Referenzenauflösung während des Schreibens! Vorm Kompilieren auskommentieren!
% \bibliography{0_hauptdatei}
% \input{1_einleitung}
% \input{2_grundlagen}
% \input{3_konzeption}
% \input{4_implementierung}
% \input{5_tests}
% \input{6_zusammenfassung}
% % Ende Imports

\section{Test des OPC-Servers im Gesamtsystem%
  \label{sec:5-tests}}

% % % Imports nur für Referenzenauflösung während des schreibens! Vorm Kompilieren auskommentieren!
% \bibliography{0_hauptdatei}
% \input{1_einleitung}
% \input{2_grundlagen}
% \input{3_konzeption}
% \input{4_implementierung}
% \input{5_tests}
% \input{6_zusammenfassung}
% % Ende Imports

\section{Zusammenfassung und Ausblick%
  \label{sec:6-fazit}}
Der folgende Abschnitt~\ref{sec:6-zusammenfassung} fasst die gewonnenen Erkenntnisse und den Stand der Implementierung zusammen.
Den Abschluss dieser Arbeit bildet der Ausblick in Abschnitt~\ref{sec:6-ausblick}.

\subsection{Zusammenfassung%
     \label{sec:6-zusammenfassung}}

\subsection{Ausblick%
     \label{sec:6-ausblick}}

% \input{anhang}
% % Ende Imports

\section{Implementierung%
  \label{sec:4-implementierung}}
Das folgende Kapitel stellt in Auszügen die Implementierung des OPC-Servers sowie die Anbindung an die IO-Module
der SPS dar. Der Schwerpunkt liegt hierbei auf der Funktionsweise des piControl-Treibers und dessen Integration in das Projekt. Abschnitt~\ref{sec:4-picontrol} erklärt die zum Schreibens eines Bits verwendeten Funktionsaufrufe.
Zuvor soll jedoch in Abschnitt~\ref{sec:4-open62541} der Teil des OPC-Servers vorgestellt werden, welcher auf besagten Treiber zugreift. 

\subsection{Implementierung des OPC-Servers%
     \label{sec:4-open62541}}
Wie im vorangegangenen Abschnitt~\ref{sec:3-integration} begründet, soll die Verknüpfung zwischen dem Prozessabbild der SPS und den auf dem OPC-Server bereitgestellten Werten über sog.\,Datenquellen erfolgen. Hierzu ist zunächst eine Callback-Methode zu implementieren, welche bei einem Lese- oder Schreibzugriff auf eine Variable aufgerufen wird. Die Verknüpfung zwischen Callback-Methode und Variable muss manuell erfolgen.

\begin{lstlisting}[language={c},firstnumber=237,caption={Auszug der Methode \lstinline{linkDataSourceVariable} in \lstinline{variables.c}\label{lst:4-linkDataSourceVariable}}]
extern UA_StatusCode
 linkDataSourceVariable(UA_Server *server, UA_NodeId nodeId) {
     bool readonly = false;
     UA_DataSource dataSourceVariable;
     UA_StatusCode rc; |>\setcounter{lstnumber}{254}<|

     dataSourceVariable.read = readDataSourceVariable;
     if (!readonly)
        dataSourceVariable.write = writeDataSourceVariable;
     else
        dataSourceVariable.write = writeReadonlyDataSourceVariable;

     return UA_Server_setVariableNode_dataSource(server, nodeId, dataSourceVariable);
 }
\end{lstlisting}

\begin{figure}[h]
    \centering
    \includegraphics[width=0.42\textwidth]{doc/img/OPC_RevPiDO.pdf}
    \caption{Auszug des verwendeten Nodesets, hier Digitalausgang 1 des Versuchsaufbaus
      \label{fig:opc-do}}
\end{figure}

Die in Listing~\ref{lst:4-linkDataSourceVariable} abgebildete Methode \lstinline{linkDataSourceVariable()} erzeugt ein Struct vom Typ \lstinline{UA_DataSource}. In diesem werden dem Lesen und Schreiben einer OPC-Variablen entsprechende Callback-Methoden zugewiesen. Die Verknüpfung einer OPC-Variable, genauer ihrer NodeId, mit der zuvor definierten Datenquelle erfolgt über die von open62541 bereitgestellte Methode \lstinline{UA_Server_setVariableNode_dataSource()}. Vor dem Lesen und nach dem Schreiben dieser Variable werden von nun an die entsprechenden Callbacks aufgerufen.
     
\begin{lstlisting}[language={c},firstnumber=168,caption={Auszug des Callbacks \lstinline{writeDataSourceVariable} in \lstinline{variables.c}\label{lst:4-writeDataSourceVariable}}]  
extern UA_StatusCode
 writeDataSourceVariable(UA_Server *server,
            const UA_NodeId *sessionId, void *sessionContext,
            const UA_NodeId *nodeId, void *nodeContext,
            const UA_NumericRange *range, const UA_DataValue *dataValue) {

    UA_StatusCode retval  = UA_STATUSCODE_GOOD;
    UA_NodeId *nameNodeId = UA_malloc(sizeof(UA_NodeId));
    UA_QualifiedName nameQN = UA_QUALIFIEDNAME(1, "Name");
    UA_Variant nameVar;
    UA_Boolean bit;

    retval |= findSiblingByBrowsename(server, nodeId, &nameQN, nameNodeId);
    retval |= UA_Server_readValue(server, *nameNodeId, &nameVar);
    retval |= UA_Boolean_copy(dataValue->value.data, &bit);

    |>\tikzmarkin[set border color=martinired]{writeIO}<|PI_writeSingleIO(String_fromUA_String(nameVar.data), &bit, false);                                                 |>\tikzmarkend{writeIO}<|

    free(nameNodeId);
    return retval;
 }
\end{lstlisting}

Listing~\ref{lst:4-writeDataSourceVariable} zeigt die Callback-Methode, welche nach dem Schreiben einer Variablen auf dem OPC-Server aufgerufen wird.
Dieser Methode wird neben der NodeId der mit ihr verknüpften Variablen auch der Wert dieser in Form eines Zeigers auf ein Struct vom Typ \lstinline{UA_DataValue} übergeben.

Die Gestaltung des hier verwendeten Nodesets sieht vor, dass in einer OPC-Variablen \lstinline{"Name"} der Bezeichner des zu schreibenden Digitalausgangs hinterlegt ist, siehe Abbildung~\ref{fig:opc-do}. Dies erlaubt eine Rekonfiguration der Ein- und Ausgänge der SPS ohne Änderungen im Programmcode des OPC-Servers vornehmen zu müssen.
Es ist daher erforderlich, nach jedem Schreiben einer mit einem Digitalausgang verknüpften Variablen, hier \lstinline{"Value"}, dessen Bezeichner \lstinline{"Name"} abzufragen. 
Dies geschieht in den Zeilen 180 und 181.
Anschließend wird dieser Bezeichner sowie der zu schreibende Wert der Methode \lstinline{PI_writeSingleIO()} übergeben, welche wiederum die Interaktion mit piControl übernimmt (vgl. Abschnitt \ref{sec:4-picontrol}).
 
\subsection{Integration von piControl%
     \label{sec:4-picontrol}}
In Abschnitt~\ref{sec:2-io} wurde die Anbindung der IO-Module des Revolution Pi sowie die Funktionsweise von piControl aus Anwendersicht beschrieben. Die verfügbare Literatur beschränkt sich auch auf lediglich diese Sicht; eine weiterführende Dokumentation für Entwickler gibt es, neben der in Abschnitt~\ref{sec:3-anbindung} vorgestellten Manpage, nicht. 
In diesem Abschnitt soll daher der Quellcode von piControl sowie dessen Verwendung im Projekt genauer betrachtet werden.
Hierzu wird exemplarisch die in Abschnitt~\ref{sec:4-open62541} eingeführte Methode \lstinline{PI_writeSingleIO()} untersucht.
Diese Methode ermöglicht das Setzen eines einzelnen Bits im Prozessabbild der SPS, und damit das Schalten eines digitalen Ausgangs auf einem IO-Modul.
Die äquivalente Methode \lstinline{int piControlGetBitValue(SPIValue *pSpiValue)} zum Lesen eines Bits bzw. Eingangs funktioniert analog und soll daher an dieser Stelle nicht dediziert erörtert werden.

\begin{lstlisting}[language={c},firstnumber=97,
                   caption={Setzen eines phsikalischen, digitalen Ausgangs in \lstinline{revpi.c}
                   \label{lst:4-PI_writeSingleIO}}]
extern void PI_writeSingleIO(char *pszVariableName, bool *bit, bool verbose)
{
	int rc;
	SPIVariable sPiVariable;
	SPIValue sPIValue;

	strncpy(sPiVariable.strVarName, pszVariableName, sizeof(sPiVariable.strVarName));
	rc = piControlGetVariableInfo(&sPiVariable);
	if (rc < 0) {
		printf("Cannot find variable '%s'\n", pszVariableName);
		return;
	}

		sPIValue.i16uAddress = sPiVariable.i16uAddress;
		sPIValue.i8uBit = sPiVariable.i8uBit;
		sPIValue.i8uValue = *bit;
		rc = |>\tikzmarkin[set border color=martinired]{setBitValue}<|piControlSetBitValue(&sPIValue)|>\tikzmarkend{setBitValue}<|;
		if (rc < 0)
			printf("Set bit error %s\n", getWriteError(rc));
		else if (verbose)
			printf("Set bit %d on byte at offset %d. Value %d\n", sPIValue.i8uBit, sPIValue.i16uAddress,
			       sPIValue.i8uValue);
}
\end{lstlisting}

Der Programmcode in Listing~\ref{lst:4-PI_writeSingleIO} ist Teil des implementierten OPC-Servers. In diesem wird auf zwei Funktionen des piControl-Treibers zugegriffen. 
Beiden Methoden wird als Argument ein Zeiger auf ein Struct vom Typ \lstinline{SPIValue} übergeben. Der im Struct abgelegte Name wird mittels \lstinline{piControlGetVariableInfo(&sPIValue)} zu einer Adresse im Prozessabbild aufgelöst. Diese wird in \lstinline{sPIValue.i16uAdress} gespeichert. Der Wert der Variablen wird anschließend mittels \lstinline{piControlSetBitValue(&sPIValue)} an dieser Adresse in das Prozessabbild geschrieben.

\begin{lstlisting}[language={c},firstnumber=309,caption={Methode \lstinline{piControlSetBitValue} in \lstinline{piControlIf.c}\label{lst:4-piControlSetBitValue}}]
int |>\tikzmarkin[set border color=martiniblue]{setBitValueFcn}<|piControlSetBitValue(SPIValue *pSpiValue)|>\tikzmarkend{setBitValueFcn}<|
{
    piControlOpen();

    if (PiControlHandle_g < 0)
	    return -ENODEV;

    pSpiValue->i16uAddress += pSpiValue->i8uBit / 8;
    pSpiValue->i8uBit %= 8;

    if (|>\tikzmarkin[set border color=martinired]{ioctl}<|ioctl(PiControlHandle_g, KB_SET_VALUE, pSpiValue)|>\tikzmarkend{ioctl}<| < 0)
	    return errno;

    return 0;
}
\end{lstlisting}

Die in Listing~\ref{lst:4-piControlSetBitValue} dargestellte Methode \lstinline{piControlSetBitValue} ist lediglich eine Hüllfunktion (häufig auch als Wrapper-Funktion bezeichnet) für einen Aufruf des \lstinline{ioctl} Kernel-Moduls.
Folgende Parameter werden übergeben:
\lstinline{PiControlHandle_g} ist die Referenz auf die Geräte-Datei des piControl-Treibers. \lstinline{KB_SET_VALUE} ist das ioctl-Kommando zum Schreiben eines Bits in das Prozessabbild. Der Zeiger \lstinline{pSpiValue} verweist auf ein Struct des bereits vorgestellten Typs \lstinline{SPIValue}.

\begin{lstlisting}[language={c},firstnumber=80,caption={Methode \lstinline{piControlOpen} in \lstinline{piControlIf.c}\label{lst:4-piControlOpen}}]
void piControlOpen(void)
{
    /* open handle if needed */
    if (PiControlHandle_g < 0)
    {
	    |>\tikzmarkin[set border color=martiniblue]{PiControlHandle}<|PiControlHandle_g = open(PICONTROL_DEVICE, O_RDWR)|>\tikzmarkend{PiControlHandle}<|;
    }
}
\end{lstlisting}

Die in Listing~\ref{lst:4-piControlOpen} dargestellte Methode öffnet, sofern nicht bereits geschehen, die Geräte-Datei. Das Macro \lstinline{PICONTROL_DEVICE} verweist hierbei auf \lstinline{/dev/piControl0}.

\begin{lstlisting}[language={c},firstnumber=721,caption={Methode \lstinline{piControlIoctl} in \lstinline{piControlMain.c}\label{lst:4-piControlIoctl}}]
static long |>\tikzmarkin[set border color=martiniblue, below offset=0.9em]{piControlIoctl}<|piControlIoctl(struct file *file, unsigned int prg_nr, 
                           unsigned long usr_addr)                                      |>\tikzmarkend{piControlIoctl}<|
{
  int status = -EFAULT;
  tpiControlInst *priv;
  int timeout = 10000;	// ms

  if (prg_nr != KB_CONFIG_SEND && prg_nr != KB_CONFIG_START && !isRunning()) {
  	return -EAGAIN;
  }

  priv = (tpiControlInst *) file->private_data;

  if (prg_nr != KB_GET_LAST_MESSAGE) {
  	// clear old message
  	priv->pcErrorMessage[0] = 0;
  }

  switch (prg_nr) {|>\setcounter{lstnumber}{864}<|

    case |>\tikzmarkin[set border color=martiniblue]{KB_SET_VALUE}<|KB_SET_VALUE:|>\tikzmarkend{KB_SET_VALUE}<|
  		{
  			SPIValue *pValue = (SPIValue *) usr_addr;

  			if (!isRunning())
  				return -EFAULT;

  			if (pValue->i16uAddress >= KB_PI_LEN) {
  				status = -EFAULT;
  			} else {
  				INT8U i8uValue_l;
  				my_rt_mutex_lock(&piDev_g.lockPI);
  				i8uValue_l = piDev_g.ai8uPI[pValue->i16uAddress];

  				if (pValue->i8uBit >= 8) {
  					i8uValue_l = pValue->i8uValue;
  				} else {
  					if (pValue->i8uValue)
  						i8uValue_l |= (1 << pValue->i8uBit);
  					else
  						i8uValue_l &= ~(1 << pValue->i8uBit);
  				}

  				|>\tikzmarkin[set border color=martinired]{i8uValue}<|piDev_g.ai8uPI[pValue->i16uAddress] = i8uValue_l;|>\tikzmarkend{i8uValue}<|
  				rt_mutex_unlock(&piDev_g.lockPI);

  #ifdef VERBOSE
  				pr_info("piControlIoctl Addr=%u, bit=%u: %02x %02x\n", pValue->i16uAddress, pValue->i8uBit, pValue->i8uValue, i8uValue_l);
  #endif

  				status = 0;
  			}
  		}
  		break; |>\setcounter{lstnumber}{1314}<|

    default:
      pr_err("Invalid Ioctl");
      return (-EINVAL);
      break;

    }

    return status;
  }
\end{lstlisting}

Listing~\ref{lst:4-piControlIoctl} zeigt in Auszügen die ioctl-Methode des piControl Kernel-Treibers. Diese bekommt folgende Argumente übergeben: \lstinline{struct file *file} enthält den Verweis auf die Geräte-Datei, hier \lstinline{/dev/piControl0}. Der Wert von \lstinline{unsigned int prg_nr} beschreibt die Anfrage an den Treiber, in diesem Fall \lstinline{KB_SET_VALUE}. Das Argument \lstinline{unsigned long usr_addr} enthält einen typ-agnostischen Pointer. Dieser verweist auf einen Speicherbereich, in welchem die zur Bearbeitung der Anfrage notwendigen Daten abgelegt sind. Hier können auch vom Treiber empfangene Daten dem Anwendungsprogramm bereitgestellt werden. 

Die switch-case-Anweisung führt die über das Argument \lstinline{prg_nr} spezifizierte Aktion aus. Hier betrachten wir \lstinline{KB_SET_VALUE}:
Zunächst wird in Zeile 868 der übergebene Zeiger \lstinline{usr_addr} mittels explizitem Typecast zu einem Zeiger des Typs \lstinline{SPIValue *} konvertiert. Da dieser auf Daten im Userspace verweist, ist beim Zugriff durch den Kernel-Treiber besondere Vorsicht geboten.
In Zeile 877 wird mittels Mutex das Prozessabbild \lstinline{piDev_g} für den Zugriff durch andere Threads oder Prozesse gesperrt.
\lstinline{my_rt_mutex_lock} verweist hierbei auf die Funktion \lstinline{rt_mutex_lock} aus \lstinline{linux/sched.h}\footnote{Offenbar wurde hier auch eine alternative Implementierung vorgesehen, siehe revpi\_common.h}

In Zeile 889 wird das Byte \lstinline{i8uValue_l}, welches den zu schreibenden Wert enthält in das Prozessabbild übertragen. Anschließend wird die Mutex auf \lstinline{piDev_g} wieder entsperrt.
\newpage

\begin{lstlisting}[language={c},firstnumber=62,caption={Auszug des Struct \lstinline{spiControlDev} in \lstinline{piControlMain.h}\label{lst:4-spiControlDev}}]
|>\tikzmarkin[set border color=martiniblue]{spiControlDev}<|typedef struct spiControlDev|>\tikzmarkend{spiControlDev}<| {
	// device driver stuff
	int init_step;
	enum revpi_machine machine_type;
	void *machine;
	struct cdev cdev;	// Char device structure
	struct device *dev;
	struct thermal_zone_device *thermal_zone;

	|>\tikzmarkin[set border color=martiniblue]{processImage}<|// process image stuff
	INT8U ai8uPI[KB_PI_LEN];
	INT8U ai8uPIDefault|>\tikzmarkin[set border color=martinired]{KB_PI_LEN_0}<|[KB_PI_LEN]|>\tikzmarkend{KB_PI_LEN_0}<|;
	struct rt_mutex lockPI;        |>\tikzmarkend{processImage}<|
	bool stopIO;
	piDevices *devs; |>\setcounter{lstnumber}{94}<|
} tpiControlDev;
\end{lstlisting}

Das Prozessabbild ist als Byte-Array der Länge \lstinline{KB_PI_LEN} in Listing~\ref{lst:4-spiControlDev} definiert. Konfigurationsparameter wie \lstinline{KB_PI_LEN} oder die Zykluszeit für den Datenaustausch zwischen SPS und IO-Modulen sind im folgenden Listing~\ref{lst:4-process} definiert.

\begin{lstlisting}[language={c},firstnumber=119,caption={Konfigurationsparameter des Prozessabbildes in project.h\label{lst:4-process}}]
#define INTERVAL_PI_GATE (5*1000*1000)  // 5 ms piGateCommunication |>\setcounter{lstnumber}{128}<|

#define INTERVAL_IO_COM (5*1000*1000)  // 5 ms piIoComm |>\setcounter{lstnumber}{132}<|

#define KB_PD_LEN       512
|>\tikzmarkin[set border color=martiniblue]{KB_PI_LEN_1}<|#define KB_PI_LEN       4096|>\tikzmarkend{KB_PI_LEN_1}<|
\end{lstlisting}

Das zu setzende Bit wurde zu diesem Zeitpunkt erfolgreich in das Prozessabbild der SPS geschrieben.
Es stellt sich die Frage, wie dieses nun an das IO-Modul kommuniziert wird.
Die Kommunikation mit allen angebundenen Modulen ist ebenfalls Aufgabe des piControl-Treibers.

\begin{lstlisting}[language={c},firstnumber=256,caption={Auszug der Methode \lstinline{piIoThread} in \lstinline{revpi_core.c}\label{lst:4-piIoThread}}]
static int piIoThread(void *data)
{
	//TODO int value = 0;
	ktime_t time;
	ktime_t now;
	s64 tDiff;

	hrtimer_init(&piCore_g.ioTimer, CLOCK_MONOTONIC, HRTIMER_MODE_ABS);
	piCore_g.ioTimer.function = piIoTimer;

	pr_info("piIO thread started\n");

	now = hrtimer_cb_get_time(&piCore_g.ioTimer);

	PiBridgeMaster_Reset();

	while (!kthread_should_stop()) {
		if (|>\tikzmarkin[set border color=martinired]{PiBridgeMaster}<|PiBridgeMaster_Run()|>\tikzmarkend{PiBridgeMaster}<| < 0)
			break;
	}

	RevPiDevice_finish();

	pr_info("piIO exit\n");
	return 0;
}
\end{lstlisting}

Der Kernel-Thread \lstinline{piIoThread} ist verantwortlich für den zyklischen Datenaustausch mit den IO-Modulen. In diesem wird fortlaufend die Methode \lstinline{PiBridgeMaster_Run()} aufgerufen, siehe Listing~\ref{lst:4-piIoThread}.

\begin{lstlisting}[language={c},firstnumber=262,caption={Auszug der Methode \lstinline{PiBridgeMaster_Run(void)} in \lstinline{RevPiDevice.c}\label{lst:4-PiBridgeMaster_Run}}]
int PiBridgeMaster_Run(void)
{
	static kbUT_Timer tTimeoutTimer_s;
	static kbUT_Timer tConfigTimeoutTimer_s;
	static int error_cnt;
	static INT8U last_led;
	static unsigned long last_update;
	int ret = 0;
	int i;

	my_rt_mutex_lock(&piCore_g.lockBridgeState);
	if (piCore_g.eBridgeState != piBridgeStop) {
		switch (eRunStatus_s) { |>\setcounter{lstnumber}{514}<|
		    case enPiBridgeMasterStatus_EndOfConfig:|>\setcounter{lstnumber}{621}<|
		    if (|>\tikzmarkin[set border color=martinired]{RevPiDevice}<|RevPiDevice_run()|>\tikzmarkend{RevPiDevice}<|) {
				// an error occured, check error limits |>\setcounter{lstnumber}{641}<|
			} else {
				ret = 1;
			}
			piCore_g.image.drv.i16uRS485ErrorCnt = RevPiDevice_getErrCnt();
			break;
\end{lstlisting}

Die in Listing~\ref{lst:4-PiBridgeMaster_Run} dargestellte Methode ist eine sog. State-Machine. Ist die Konfiguration der IO-Module erfolgreich abgeschlossen, so führt sie bei Aufruf lediglich die Methode \lstinline{RevPiDevice_run()} aus.

\begin{lstlisting}[language={c},firstnumber=140,caption={Auszug der Methode \lstinline{RevPiDevice_run(void)} in \lstinline{RevPiDevice.c}\label{lst:4-RevPiDevice_run}}]
int RevPiDevice_run(void)
{
	INT8U i8uDevice = 0;
	INT32U r;
	int retval = 0;

	RevPiDevices_s.i16uErrorCnt = 0;

	for (i8uDevice = 0; i8uDevice < RevPiDevice_getDevCnt(); i8uDevice++) {
		if (RevPiDevice_getDev(i8uDevice)->i8uActive) {
			switch (RevPiDevice_getDev(i8uDevice)->sId.i16uModulType) {
			case KUNBUS_FW_DESCR_TYP_PI_DIO_14:
			case KUNBUS_FW_DESCR_TYP_PI_DI_16:
			case KUNBUS_FW_DESCR_TYP_PI_DO_16:
				r = |>\tikzmarkin[set border color=martinired]{sendCyclicTelegram}<|piDIOComm_sendCyclicTelegram(i8uDevice)|>\tikzmarkend{sendCyclicTelegram}\setcounter{lstnumber}{166} <|;

				break; |>\setcounter{lstnumber}{216}<|
			}
		}
	} |>\setcounter{lstnumber}{227}<|
	return retval;
}
\end{lstlisting}

Diese iteriert wie in Listing~\ref{lst:4-RevPiDevice_run} abgebildete durch alle gegenwärtig in der SPS konfigurierten Module. Ist das aktuelle Modul als aktiv markiert, so wird anhand eines sog. Firmware-Descriptors entschieden, welche Methode für die Ansteuerung des Moduls aufzurufen ist.

\begin{lstlisting}[language={c},firstnumber=161,caption={Auszug der Methode \lstinline{piDIOComm_sendCyclicTelegram} in \lstinline{piDIOComm.c}\label{lst:4-sendCyclicTelegram}}]
INT32U piDIOComm_sendCyclicTelegram(INT8U i8uDevice_p)
{
	INT32U i32uRv_l = 0;
	SIOGeneric sRequest_l;
	SIOGeneric sResponse_l;
	INT8U len_l, data_out[18], i, p, data_in[70];
	INT8U i8uAddress;
	int ret; |>\setcounter{lstnumber}{239}<|
	
    |>\tikzmarkin[set border color=martinired]{piIoComm}<|ret = piIoComm_send((INT8U *) & sRequest_l, IOPROTOCOL_HEADER_LENGTH + len_l + 1);  |>\tikzmarkend{piIoComm}\setcounter{lstnumber}{298}<|
}
\end{lstlisting}

Im Falle des hier verwendeten DO-Moduls wird die in Listing~\ref{lst:4-sendCyclicTelegram} abgebildete Methode \lstinline{piDIOComm_sendCyclicTelegram()} aufgerufen. Dieser wird ein Zeiger auf das zu schreibende Gerät übergeben. 
Zunächst wird das Prozessabbild mittels eines proprietären, jedoch im Quellcode offen nachvollziehbaren Protokolls in ein \lstinline{sRequest_l} genanntes Byte-Array umgewandelt. Dieser Schritt ist in Listing~\ref{lst:4-sendCyclicTelegram} nicht abgebildet. Anschließend wird \lstinline{piIoComm_send()} ein Zeiger auf die so generierte Schreib-Anfrage übergeben.

\begin{lstlisting}[language={c},firstnumber=220,caption={Auszug der Methode \lstinline{piIOComm_send} in \lstinline{piIOComm.c}\label{lst:4-piIOComm_send}}]
int piIoComm_send(INT8U * buf_p, INT16U i16uLen_p)
{
	ssize_t write_l = 0;
	INT16U i16uSent_l = 0;|>\setcounter{lstnumber}{249}<|

	while (i16uSent_l < i16uLen_p) {
		write_l = vfs_write(piIoComm_fd_m, buf_p + i16uSent_l, i16uLen_p - i16uSent_l, &piIoComm_fd_m->f_pos);
		if (write_l < 0) {
			pr_info_serial("write error %d\n", (int)write_l);
			return -1;
		} 
		i16uSent_l += write_l;|>\setcounter{lstnumber}{263}<|
	}
	clear();
	vfs_fsync(piIoComm_fd_m, 1);
	return 0;
}
\end{lstlisting}

Listing~\ref{lst:4-piIOComm_send} zeigt die Implementierung von \lstinline{piIoComm_send()}. Diese Methode ist für das Schreiben der oben generierten Anfrage auf die seriellen Schnittstelle verantwortlich. Realisiert wird dies mittels der Methode \lstinline{vfs_write()}. Diese ist in \lstinline{<linux/fs.h>} definiert. Sie ermöglicht das Schreiben einer Datei im Userspace aus dem Kernel heraus. Geschrieben wird hier die Datei mit dem Deskriptor \lstinline{piIoComm_fd_m}.
Da die Funktion \lstinline{vfs_write()} durch andere Kernel-Tasks unterbrochen werden kann, ist nicht gewährleistet, dass die gesamte Anfrage mit nur einem Aufruf geschrieben wird. Die oben abgebildete while-Schleife stellt das vollständige Senden der Anfrage sicher.

\begin{lstlisting}[language={c},firstnumber=157,caption={Auszug der Methode \lstinline{piIOComm_open_serial} in \lstinline{piIOComm.c}\label{lst:4-piIOComm_open_serial}}]
int piIoComm_open_serial(void)
{   |>\setcounter{lstnumber}{167}<|
	struct file *fd;	/* Filedeskriptor */
	struct termios newtio;	/* Schnittstellenoptionen */

	|>\tikzmarkin[set border color=martiniblue]{fd}<|/* Port oeffnen - read/write, kein "controlling tty", 
	    Status von DCD ignorieren */
	fd = filp_open(|>\tikzmarkin[set border color=martinired]{tty}<|REV_PI_TTY_DEVICE|>\tikzmarkend{tty}<|, O_RDWR | O_NOCTTY, 0); |>\setcounter{lstnumber}{208}<|
	
	piIoComm_fd_m = fd;                                                      |>\tikzmarkend{fd}\setcounter{lstnumber}{217}<|

	return 0;
}
\end{lstlisting}

Der zum Schreiben auf die serielle Schnittstelle verwendete Datei-Deskriptor wird von der in Listing~\ref{lst:4-piIOComm_open_serial} abgebildeten Methode \lstinline{piIoComm_open_serial()} generiert. 

\begin{lstlisting}[language={c},firstnumber=45,caption={Definition der seriellen Schnittstelle in \lstinline{piIOComm.h}\label{lst:4-REV_PI_TTY_DEVICE}}]
#define REV_PI_TTY_DEVICE	"/dev/ttyAMA0"
\end{lstlisting}

Das in Listing~\ref{lst:4-REV_PI_TTY_DEVICE} definierte Macro verweist auf eine der seriellen Schnittstellen des RaspberryPi.
Die Implementierung des zugehörigen Schnittstellentreibers soll hier nicht weiter untersucht werden. Somit ist an dieser Stelle die Kette vom Setzen einer Variablen auf dem OPC-Server bis hin zur Aktualisierung des Prozessabbilds der IO-Module geschlossen.

% \begin{lstlisting}[language={c},firstnumber={226},caption={Setzen der Scheduler-Priorität auf SCHED\_FIFO in 
% revpi\_common.c\label{lst:2-sched_priority}}]
% param.sched_priority = ktprio->prio;
% ret = sched_setscheduler(child, SCHED_FIFO, &param);
% \end{lstlisting}
% % % Imports nur für Referenzenauflösung während des Schreibens! Vorm Kompilieren auskommentieren!
% \bibliography{0_hauptdatei}
% % Mit \section{...} eröffnen wir einen neuen Abschnitt.
% Der Befehl setzt nicht nur den Text in einer größeren,
% fetten Schrift, sondern sorgt außerdem dafür, daß er im
% Inhaltsverzeichnis erscheint.
%
% Mit \label{...} erzeugen wir einen Bezeichner, mit dessen Hilfe
% wir später auf die Nummer des Abschnitts verweisen können (nämlich
% mit~\ref{...}).
%
% Das Kommentarzeichen hinter „Übersicht“ dient dazu, ein
% Leerzeichen zwischen „Übersicht“ und dem \label-Befehl
% zu vermeiden, das andernfalls sichtbar würde – z.B. im
% Inhaltsverzeichnis.
%

% % Imports nur für Referenzenauflösung während des Schreibens! Vorm Kompilieren auskommentieren!
% \bibliography{0_hauptdatei}
% \input{1_einleitung}
%\input{2_grundlagen}
%\input{3_konzeption}
%\input{4_implementierung}
%\input{5_tests}
%\input{6_zusammenfassung}
% % Ende Imports

\section{Einleitung und Motivation%
  \label{sec:1-einleitung}}
Ziel dieses Projektes ist die Integration eines OPC-Servers mit einer auf Linux
basierenden speicherprogrammierbaren Steuerung (SPS). Angeschlossen an diese SPS
ist jeweils ein digitales Ein-/\,bzw.~Ausgabemodul. Die von diesen bereitgestellten
Ein-/\, bzw.~Ausgänge (IO) sollen in der Datenstruktur des OPC-Servers abgebildet
und über diesen für OPC-Clients les-/\,und schreibar sein. Weiterhin sollen einige
Funktionen zur Überwachung und Steuerung der an die SPS angeschlossenen Aktoren
und Sensoren direkt im OPC-Server implementiert werden.
Hiermit stellt dieses Projekt eine der Grundlagen für ein übergeordnetes Projekt,
die cloudbasierte Steuerung eines miniaturisierten Produktions-Systems, dar.

Der hier verwendete OPC-Server ist Teil des sog. open62541 Projekts. Er ist in C
geschrieben und implementiert bereits einen großen Teil der im OPC-UA-Standard
spezifizierten Funktionen.
Als SPS findet ein Revolution Pi 3 der Firma Kunbus Verwendung. Dieser integriert
ein sog. Compute Module der Raspberry Pi Foundation in ein industrietaugliches
Gehäuse und erlaubt die Erweiterung mittels IO- oder Gateway-Modulen. Über diese
erfolgt die Kommunikation mit weiteren Komponenten der Automatisierungstechnik.

Motiviert ist dieses Projekt durch die Beobachtung, dass die Verbreitung offener
Standards sowie freier Software auch in der Automatisierungstechnik zunimmt.
Linux ist ein freies Betriebssystem, OPC-UA ein offen zugänglicher, aktiv gepflegter
und weit verbreiteter Standard. Der Raspberry Pi findet sowohl bei Hobby-Anwendern als
auch in den Bereichen Forschung und Entwicklung sowie bei industriellen Anwendern
Verwendung. Dieses Projekt stellt somit eine für unterschiedliche Anwender interessante
Entwicklung dar.

Im Anschluss an diese einleitende Übersicht im Abschnitt~\ref{sec:1-einleitung} folgt
die Darstellung der wichtigsten Grundlagen in Abschnitt~\ref{sec:2-grundlagen}.
Aufbauend auf diesen Grundlagen folgt die konzeptuelle Ausarbeitung im Abschnitt~\ref{sec:3-konzeption}.
Die Umsetzung wird im Abschnitt~\ref{sec:4-implementierung} erläutert.
Die Leistungsfähigkeit der Implementierung wird in Abschnitt~\ref{sec:5-tests} untersucht.
Eine Zusammenfassung und ein Ausblick schließen die Arbeit in
Abschnitt~\ref{sec:6-fazit} ab. Eventuell noch benötigte Anhänge
finden sich in den Anhängen [...] bis [...].

% % % Imports nur für Referenzenauflösung während des Schreibens! Vorm Kompilieren auskommentieren!
% \bibliography{0_hauptdatei}
% \input{1_einleitung}
% \input{2_grundlagen}
% \input{3_konzeption}
% \input{4_implementierung}
% \input{5_tests}
% \input{6_zusammenfassung}
% % Ende Imports

\section{Grundlagen%
  \label{sec:2-grundlagen}}

\subsection{Speicherprogrammierbare-Steuerung und Linux -- Revolution Pi%
     \label{sec:2-sps}}

\subsubsection{Kunbus RevolutionPi%
        \label{sec:2-revpi}}
Der RevolutionPi 3 ist eine speicherprogrammierbare Steuerung (SPS) des Herstellers
Kunbus GmbH. Kern dieser SPS ist das von der Raspberry Pi Foundation entwickelte
und vertriebene Raspberry Pi Compute Module 3. Dieses integriert ein Broadcom BCM2837
System-on-Chip (SoC) mit vier 1,2GHz Prozessorkernen, 1GB RAM, 4GB eMMC Anwendungsspeicher
und sonstige Peripherie in ein Modul im DDR2-SODIMM Formfaktor. Diese Spezifikationen
sind weitgehend identisch zu denen des ausgesprochen populären Raspberry Pi 3.
Der Revolution Pi profitiert daher von dem gleichen großen Angebot an Software
und Unterstützung wie der Raspberry Pi, ergänzt dessen Hardware jedoch um eine 24V
Spannungsversorgung, die Möglichkeit der Erweiterung durch mehrere industrietaugliche
Ein-/ Ausgabemodule und Gateways sowie ein Gehäuse zur Montage auf einer DIN-Schiene.
\begin{itemize}
  \item{Prozessor: BCM2837}
  \item{Taktfrequenz 1,2 GHz}
  \item{Anzahl Prozessorkerne: 4}
  \item{Arbeitsspeicher: 1 GByte}
  \item{eMMC Flash Speicher: 4 GByte}
  \item{Betriebssystem: Angepasstes Raspbian mit RT-Patch}
  \item{RTC mit 24h Pufferung über wartungsfreien Kondensator}
  \item{Treiber / API: Treiber schreibt zyklisch Prozessdaten in ein Prozessabbild, Zugriff auf Prozessabbild über Linux-Filesystem als API zu Fremdsoftware.}
  \item{Kommunikationsanschlüsse: 2 x USB 2.0 A (je 500 mA belastbar), 1 x Micro-USB, HDMI, Ethernet (RJ45) 10/100 Mbit/s}
  \item{Stromversorgung: min. 10,7 V, max. 28,8 V, maximal 10 Watt}
  \item{Zulässige Umgebungstemperatur: -40 bis +55 C}
  \item{Gehäuseabmessungen: (HxBxL) 96 mm x 22,5 mm x 110,5 mm (ohne gesteckte Stecker)}
  \item{ESD Schutz: 4 kV / 8 kV gemäß EN61131-2 und IEC 61000-6-2}
  \item{Surge / Burst Prüfungen: gemäß EN61131-2 und IEC 61000-6-2 eingekoppelt auf Versorgungsspannung, Ethernet und IO-Leitungen}
  \item{EMI Prüfungen: gemäß EN61131-2 und IEC 61000-6-2}
\end{itemize}

Kunbus bietet eine Auswahl an IO- und Gateway-Modulen zur Erweiterung des Revolution Pi an.
Gateways dienen der Kommunikation mit Systemen oder Komponenten der Automatisierungstechnik
über Protokolle wie PROFIBUS oder EtherCAT. IO-Module erlauben die Überwachung
und Steuerung von digitalen oder analogen Ein- und Ausgängen.

\subsubsection{Zugriff auf IO-Module%
        \label{sec:2-io}}
Der Zugriff auf die Ein- und Ausgänge der IO-Module erfolgt über ein Prozessabbild
und einen hierfür von Kunbus bereitgestellten Treiber, genannt piControl. Dieser
aktualisiert das Prozessabbild zyklisch. Die angestrebte Zykluszeit beträgt 5ms,
kann jedoch je nach Anzahl der angeschlossenen Module auch größer sein. Kunbus
garantiert bei drei IO-Modulen und zwei Gateway-Modulen eine Zykluszeit von 10 ms.
Jedes der IO-Module stellt ein eigenständiges eingebettetes System dar. Es verfügt
über einen Microcontroller, welcher die IOs bereitstellt und über einen RS485-Bus
mit dem Revolution Pi kommuniziert.
% https://revolution.kunbus.de/io-modul/

Lizenz: GPL
% https://github.com/RevolutionPi/piControl

\begin{lstlisting}[language={c},firstnumber={226},caption={Setzen der Scheduler-Priorität auf SCHED\_FIFO in revpi\_common.c\label{lst:2-sched_priority}}]
param.sched_priority = ktprio->prio;
ret = sched_setscheduler(child, SCHED_FIFO,
       &param);
\end{lstlisting}


\subsection{Echtzeit und Multithreading unter Linux -- preemptRT und posix%
     \label{sec:2-echtzeit}}


 Der Linux-Kernel verfügt über mehrere unterschiedliche Preemtion-Modelle:

\begin{itemize}
  \item No Forced Preemption (server):
  Ausgelegt auf maximal möglichen Durchsatz, lediglich Interrupts und
  System-Call-Returns bewirken Präemption.

  \item Voluntary Kernel Preemption (Desktop):
  Neben den implizit bevorrechtigten Interrupts und System-Call-Returns gibt es
  in diesem Modell weitere Abschnitte des Kernels in welchen Preämption explizit
  gestattet ist.

  \item Preemptible Kernel (Low-Latency Desktop):
  In diesem Modell ist der gesamte Kernel, mit Ausnahme sog.~kritischer Abschnitte
  präemptible. Nach jedem kritischen Abschnitt gibt es einen impliziten Präemptions-Punkt.

  \item Preemptible Kernel (Basic RT):
  Dieses Modell ist dem zuvor genannten sehr ähnlich, hier sind jedoch alle Interrupt-Handler
  als eigenständige Threads ausgeführt.

  \item Fully Preemptible Kernel (RT):
  Wie auch bei den beiden zuvor genannten Modellen ist hier der gesamte Kernel
  präemtible, die Anzahl und Dauer der nicht-präemtiblen kritischen Abschnitte
  ist auf ein notwendiges Minimum beschränkt. Alle Interrupt-Handler sind als
  eigenständige Threads ausgeführt, Spinlocks durch Sleeping-Spinlocks und Mutexe
  durch sog.~RT-Mutexe ersetzt.

\end{itemize}
\todo{Spinlocks und Mutexe sowie die RT-Varianten dieser erklären!}

Lediglich mit dem vollständig präemtiblen Kernel kann Echtzeit-Verhalten realisiert werden.

% https://wiki.linuxfoundation.org/realtime/documentation/technical_basics/preemption_models bzw kernel/Kconfig.preempt

\subsubsection{preemptRT%
        \label{sec:2-preemptRT}}
% https://wiki.linuxfoundation.org/realtime/documentation/technical_details/start
% https://wiki.linuxfoundation.org/realtime/documentation/technical_basics/start

Das dem PREEMPT RT Kernel zugrunde liegende Prinzip lässt sich in einer einfachen
Regel ausdrücken: Nur Code, welcher absolut nicht-präemtible sein darf, ist es
gestattet nicht-präemtible zu sein.
Das erklärte Ziel des PREEMPT\_RT Patches ist es folglich, die Menge des nicht-präemtiblen
Codes im Linux-Kernel auf das absolut notwendige Minimum zu reduzieren.

Dies wird durch Verwendung folgender Mechanismen erreicht:

\begin{itemize}
  \item Hochauflösende Timer
  \item Sleeping Spinlocks
  \item Threaded Interrupt Handlers
  \item rt\_mutex
  \item RCU
\end{itemize}


\subsubsection{posix%
        \label{sec:2-posix}}
Ist posix hier wirklich relevant? Debian bzw.~Raspbian sind weitgehend posix
kompatibel, aber wird es hier genutzt? -> JA, open62541 nutzt pthread.h
piControl nutzt kthread.h, und semaphore.h

\subsection{OPC-UA und open62541%
     \label{sec:2-opc}}

\subsubsection{OPC UA%
        \label{sec:2-opcua}}
Open Platform Communications (OPC) ist eine Familie von Standards zur herstellerunabhängigen
Kommunikation von Maschinen (M2M) in der Automatisierungstechnik. Die sog.~OPC Task Force, zu deren
Mitgliedern verschiedene große Firmen der Automatisierungsindustrie gehören, veröffentlichte
die OPC Specification Version 1.0 im August 1996.
Motiviert ist dieser offene Standard durch die Erkenntniss, dass die Anpassung der
zahlreichen Herstellerstandards an individuelle Infrastrukturen und Anlagen einen
großen Mehraufwand verursachen.
Die Wikipedia beschreibt das Anwendungsgebiet für OPC wie folgt:

\glqq{}OPC wird dort eingesetzt, wo Sensoren, Regler und Steuerungen verschiedener Hersteller
ein gemeinsames Netzwerk bilden. Ohne OPC benötigten zwei Geräte zum Datenaustausch
genaue Kenntnis über die Kommunikationsmöglichkeiten des Gegenübers. Erweiterungen
und Austausch gestalten sich entsprechend schwierig. Mit OPC genügt es, für jedes
Gerät genau einmal einen OPC-konformen Treiber zu schreiben. Idealerweise wird
dieser bereits vom Hersteller zur Verfügung gestellt. Ein OPC-Treiber lässt sich
ohne großen Anpassungsaufwand in beliebig große Steuer- und Überwachungssysteme
integrieren.

OPC unterteilt sich in verschiedene Unterstandards, die für den jeweiligen Anwendungsfall
unabhängig voneinander implementiert werden können. OPC lässt sich damit verwenden
für Echtzeitdaten (Überwachung), Datenarchivierung, Alarm-Meldungen und neuerdings
auch direkt zur Steuerung (Befehlsübermittlung).\grqq{}

OPC basiert in der ursprünglichen Spezifikation auf Microsofts DCOM-Spezifikation.
DCOM macht Funktionen und Objekte einer Anwendung anderen Anwendungen im Netzwerk
zugänglich. Der OPC-Standard definiert entsprechende DCOM-Objekte um mit anderen
OPC-Anwendungen Daten austauschen zu können. Die Verwendung von DCOM bindet Anwender
an Betriebssysteme von Microsoft. Die ursprüngliche OPC Spezifikation wird durch die
Entwicklung von OPC Unified Architecture (OPC UA) abgelöst.
OPC UA setzt auf einem eigenen Kommunikationionsstack auf, die Verwendung von DCOM
und damit die Bindung an Microsoft wurden aufgelöst.

Die OPC-UA-Architektur ist eine Service-orientierte Architektur (SOA), deren Struktur
aus mehreren Schichten besteht.

% Wikipedia
Das OPC-Informationsmodell ist nicht mehr nur eine Hierarchie aus Ordnern, Items
und Properties. Es ist ein sogenanntes Full-Mesh-Network aus Nodes, mit dem neben
den Nutzdaten eines Nodes auch Meta- und Diagnoseinformationen repräsentiert werden.
Ein Node ähnelt einem Objekt aus der objektorientierten Programmierung. Ein Node
kann Attribute besitzen, die gelesen werden können (Data Access (DA), Historical
Data Access (HDA)). Es ist möglich Methoden zu definieren und aufzurufen.
Eine Methode besitzt Aufrufargumente und Rückgabewerte. Sie wird durch ein Command
aufgerufen. Weiterhin werden Events unterstützt, die versendet werden können
(AE (Alarms \& Events), DA DataChange), um bestimmte Informationen zwischen Geräten
auszutauschen. Ein Event besitzt unter anderem einen Empfangszeitpunkt, eine Nachricht
und einen Schweregrad. Die o. g. Nodes werden sowohl für die Nutzdaten als auch
alle anderen Arten von Metadaten verwendet. Der damit modellierte OPC-Adressraum
beinhaltet nun auch ein Typmodell, mit dem sämtliche Datentypen spezifiziert werden.

% https://de.wikipedia.org/wiki/Open_Platform_Communications
% https://de.wikipedia.org/wiki/OPC_Unified_Architecture
% https://opcfoundation.org/developer-tools/specifications-unified-architecture
% Von Gerhard Gappmeier - ascolab GmbH, CC BY-SA 3.0, https://de.wikipedia.org/w/index.php?curid=1892069
\subsubsection{open62541%
        \label{sec:2-open62541}}
open62541 ist eine offene und freie Implementierung von OPC UA. Die in C geschriebene
Bibliothek stellt eine beständig zunehmende Anzahl der im OPC UA Standard definierten
Funktionen bereit. Sie kann sowohl zur Erstellung von OPC-Servern als auch -Clients
genutzt werden. Ergänzend zu der unter der Mozilla Public License v2.0 lizensierten
Bibliothek stellt das open62541 Projekt auch Beispielprogramme unter einer CC0 Lizenz
zur Verfügung.

Die Bibliothek eignet sich auch für die Entwicklung auf eingebetteten Systemen und
Microcontrollern. Je nach Umfang der gewünschten Funktionen und des OPC Informationsmodells
beträgt die Größe einer Server-Binary weniger als 100kb. %evtl. kürzen?

\todo{Nodes erklären! Evtl.~oben!}

Folgende Auswahl an Eigenschaften und Funktionen zeichnet die in dieser Arbeit verwendete
Version 0.3 von open62541 aus:
\begin{itemize}
  \item Kommunikationionsstack
  \begin{itemize}
      \item OPC UA Binär-Protokoll (HTTP oder SOAP werden gegenwärtig nicht unterstützt)
      \item Austauschbare Netzwerk-Schicht, welche die Verwendung eigener Netzwerk-APIs
      erlaubt.
      \item Verschlüsselte Kommunikationion
      \item Asynchrone Dienst-Anfragen im Client
  \end{itemize}
  \item Informationsmodell
  \begin{itemize}
    \item Unterstützung aller OPC UA Node-Typen, inkl.~Methoden
    \item Hinzufügen und Entfernen von Nodes und Referenzen zur Laufzeit.
    \item Vererbung und Instanziierung von Objekt- und Variablentypen
    \item Zugriffskontrolle auch für einzelne Nodes
  \end{itemize}
  \item Subscriptions
  \begin{itemize}
    \item Erlaubt die Überwachung (subscriptions / monitoreditems)
    \item Sehr geringer Ressourcenbedarf pro überwachtem Wert
  \end{itemize}
  \item Code-Generierung auf XML-Basis
  \begin{itemize}
    \item Erlaubt die Erstellung von Datentypen
    \item Erlaubt die Generierung des serverseitigen Informationsmodells
  \end{itemize}
\end{itemize}

% https://open62541.org/doc/0.3/


Mozilla Public License
CC0 Lizenz für Beispiele und Plugins

% https://open62541.org/doc/open62541-current.pdf
% https://open62541.org/

% % % Imports nur für Referenzenauflösung während des Schreibens! Vorm Kompilieren auskommentieren!
% \bibliography{0_hauptdatei}
% \input{1_einleitung}
% \input{2_grundlagen}
% \input{3_konzeption}
% \input{4_implementierung}
% \input{5_tests}
% \input{6_zusammenfassung}
% \input{anhang}
% % Ende Imports

\section{Systemkonzept%
  \label{sec:3-konzeption}}
Auf Basis der in Abschnitt \ref{sec:2-grundlagen} vorgestellten Möglichkeiten folgt nun die Ausarbeitung eines Konzepts.
In den folgenden Abschnitten soll näher auf zwei zentrale Aspekte eingegangen werden: Abschnitt~\ref{sec:3-anbindung} stellt Möglichkeiten zum Zugriff auf Variablen bzw.\,Werte im Prozessabbild des Revolution Pi vor; in Abschnitt~\ref{sec:3-integration} wird ein Konzept zur Bereitstellung dieser Variablen auf einem OPC-Server vorgestellt.

\subsection{Anbindung der IO an den OPC-Server%
     \label{sec:3-anbindung}}

Eine Webanwendung mit Bezeichnung PiCtory dient zur Konfiguration der I/O- und virtuellen Module des RevolutionPi. Die Konfiguration liegt im JSON-Format in der Datei \lstinline{/etc/revpi/config.rsc}. Der piControl-Treiber liest diese Datei beim Start. 
Der folgende Auszug aus der Manpage des piControl-Kernelmoduls beschreibt die von diesem zum Lesen und Schreiben einzelner Bits des Prozessabbildes bereitgestellten Funktionen~\citep[vgl.]{web-revpi-manpage}. Sie ist an dieser Stelle weitgehend ungekürzt zitiert, da sie die nutzbare Schnittstelle sehr kompakt beschreibt.

\begin{lstlisting}[breakindent=0pt, numbers=none, caption={Auszug aus der Revolution Pi Programmers Manual\label{lst:4-manpage}}]
KB_FIND_VARIABLE SPIVariable *argp
Find a variable in the process image by its name. A pointer to a structure of type SPIVariable must be passed as argument. [...]
The struct SPIVariable [...] is defined as 
typedef struct SPIVariableStr
{
    char strVarName[32]; // Variable name
    uint16_t i16uAddress; // Address of the byte in the process image
    uint8_t i8uBit; // 0-7 bit position, >= 8 whole byte
    uint16_t i16uLength; // length of the variable in bits.
    // Possible values are 1, 8, 16 and 32
} SPIVariable;

Set and get values of the process image
KB_GET_VALUE SPIValue *argp
[...]
KB_SET_VALUE SPIValue *argp
Write one bit or one byte to the process image [...].  This call is more efficient than the usual calls of seek and write because only one function call is necessary. If more than on application are writing bits in one output byte, this call is the only safe way to set a bit without overwriting the other bits because this call is doing a read-modify-write-cycle. 

The struct SPIValue used by this ioctl is defined as
typedef struct SPIValueStr
{
    uint16_t i16uAddress; // Address of the byte in the process image
    uint8_t i8uBit; // 0-7 bit position, >= 8 whole byte
    uint8_t i8uValue; // Value: 0/1 for bit access, whole byte otherwise
} SPIValue;
\end{lstlisting} 

Die oben beschriebenden Funtkionen \lstinline{KB_FIND_VARIABLE}, \lstinline{KB_GET_VALUE} und \lstinline{KB_SET_VALUE} ermöglichen einen einfachen und (lt.\,Manpage) effizienten Zugriff auf einzelne Bits des Prozessabbildes und damit auch auf die IO des RevolutionPi.
Der Zugriff des OPC-Servers auf das Prozessabbild soll daher mittels dieser Funktionen realisiert werden.
\lstinline{KB_FIND_VARIABLE} kann genutzt werden, um Adressen von Variablen im Prozessabbild mittels ihres Namens aufzulösen.
\lstinline{KB_GET_VALUE} und \lstinline{KB_SET_VALUE} ermöglichen den Zugriff auf die Werte dieser Variablen.


\subsection{Integration des OPC-Servers in das System%
     \label{sec:3-integration}}

open62541 bietet drei Möglichkeiten zum Abgleich von Variablen mit dem Prozessabbild~\citep[vgl.][Tutorials - Connecting a Variable with a Physical Process]{web-open62541}:
\begin{itemize}
    \item Manuelles oder zyklisches Aktualisieren
    \item Variable Value Callback
    \item Variable Datasource
\end{itemize}

Die zyklische Aktualisierung eines oder mehrerer Werte nimmt, abhängig von der Zykluszeit, viele Systemressourcen in Anspruch. Value Callbacks ermöglichen es, einen Variablenwert effizienter mit einer Ressource wie etwa einem Prozessabbild zu synchronisieren. An die Variable wird ein Callback angehängt, welches vor jedem Lesen und nach jedem Schreibvorgang ausgeführt wird.
Der Wert der Variablen wird weiterhin im Variablenknoten auf dem OPC-Server gespeichert, der Abgleich mit der verknüpften Ressource erfolgt durch die Callback-Methoden.

Sogenannte Datenquellen gehen noch einen Schritt weiter. Der Server leitet jede Lese- und Schreibanforderung direkt an eine Callback-Funktion weiter. Beim Lesen liefert der Rückruf eine Kopie des aktuellen Wertes. Die Datenquelle muss intern ein eigenes Speichermanagement implementieren.

Der Zugriff auf die Werte des Prozessabbildes erfolgt, wie in Abschnitt~\ref{sec:3-anbindung} beschrieben, über von piControl bereitgestellte Methoden. Um die durch open62541 gepflegte OPC-Datenstruktur und das durch piControl verwaltete Prozessabbild möglichst effektiv verknüpfen zu können, soll diese Interaktion mittels Datenquellen und den zugehörigen Callbacks implementiert werden.
% % % Imports nur für Referenzenauflösung während des Schreibens! Vorm Kompilieren auskommentieren!
% \bibliography{0_hauptdatei}
% \input{1_einleitung}
% \input{2_grundlagen}
% \input{3_konzeption}
% \input{4_implementierung}
% \input{5_tests}
% \input{6_zusammenfassung}
% \input{anhang}
% % Ende Imports

\section{Implementierung%
  \label{sec:4-implementierung}}
Das folgende Kapitel stellt in Auszügen die Implementierung des OPC-Servers sowie die Anbindung an die IO-Module
der SPS dar. Der Schwerpunkt liegt hierbei auf der Funktionsweise des piControl-Treibers und dessen Integration in das Projekt. Abschnitt~\ref{sec:4-picontrol} erklärt die zum Schreibens eines Bits verwendeten Funktionsaufrufe.
Zuvor soll jedoch in Abschnitt~\ref{sec:4-open62541} der Teil des OPC-Servers vorgestellt werden, welcher auf besagten Treiber zugreift. 

\subsection{Implementierung des OPC-Servers%
     \label{sec:4-open62541}}
Wie im vorangegangenen Abschnitt~\ref{sec:3-integration} begründet, soll die Verknüpfung zwischen dem Prozessabbild der SPS und den auf dem OPC-Server bereitgestellten Werten über sog.\,Datenquellen erfolgen. Hierzu ist zunächst eine Callback-Methode zu implementieren, welche bei einem Lese- oder Schreibzugriff auf eine Variable aufgerufen wird. Die Verknüpfung zwischen Callback-Methode und Variable muss manuell erfolgen.

\begin{lstlisting}[language={c},firstnumber=237,caption={Auszug der Methode \lstinline{linkDataSourceVariable} in \lstinline{variables.c}\label{lst:4-linkDataSourceVariable}}]
extern UA_StatusCode
 linkDataSourceVariable(UA_Server *server, UA_NodeId nodeId) {
     bool readonly = false;
     UA_DataSource dataSourceVariable;
     UA_StatusCode rc; |>\setcounter{lstnumber}{254}<|

     dataSourceVariable.read = readDataSourceVariable;
     if (!readonly)
        dataSourceVariable.write = writeDataSourceVariable;
     else
        dataSourceVariable.write = writeReadonlyDataSourceVariable;

     return UA_Server_setVariableNode_dataSource(server, nodeId, dataSourceVariable);
 }
\end{lstlisting}

\begin{figure}[h]
    \centering
    \includegraphics[width=0.42\textwidth]{doc/img/OPC_RevPiDO.pdf}
    \caption{Auszug des verwendeten Nodesets, hier Digitalausgang 1 des Versuchsaufbaus
      \label{fig:opc-do}}
\end{figure}

Die in Listing~\ref{lst:4-linkDataSourceVariable} abgebildete Methode \lstinline{linkDataSourceVariable()} erzeugt ein Struct vom Typ \lstinline{UA_DataSource}. In diesem werden dem Lesen und Schreiben einer OPC-Variablen entsprechende Callback-Methoden zugewiesen. Die Verknüpfung einer OPC-Variable, genauer ihrer NodeId, mit der zuvor definierten Datenquelle erfolgt über die von open62541 bereitgestellte Methode \lstinline{UA_Server_setVariableNode_dataSource()}. Vor dem Lesen und nach dem Schreiben dieser Variable werden von nun an die entsprechenden Callbacks aufgerufen.
     
\begin{lstlisting}[language={c},firstnumber=168,caption={Auszug des Callbacks \lstinline{writeDataSourceVariable} in \lstinline{variables.c}\label{lst:4-writeDataSourceVariable}}]  
extern UA_StatusCode
 writeDataSourceVariable(UA_Server *server,
            const UA_NodeId *sessionId, void *sessionContext,
            const UA_NodeId *nodeId, void *nodeContext,
            const UA_NumericRange *range, const UA_DataValue *dataValue) {

    UA_StatusCode retval  = UA_STATUSCODE_GOOD;
    UA_NodeId *nameNodeId = UA_malloc(sizeof(UA_NodeId));
    UA_QualifiedName nameQN = UA_QUALIFIEDNAME(1, "Name");
    UA_Variant nameVar;
    UA_Boolean bit;

    retval |= findSiblingByBrowsename(server, nodeId, &nameQN, nameNodeId);
    retval |= UA_Server_readValue(server, *nameNodeId, &nameVar);
    retval |= UA_Boolean_copy(dataValue->value.data, &bit);

    |>\tikzmarkin[set border color=martinired]{writeIO}<|PI_writeSingleIO(String_fromUA_String(nameVar.data), &bit, false);                                                 |>\tikzmarkend{writeIO}<|

    free(nameNodeId);
    return retval;
 }
\end{lstlisting}

Listing~\ref{lst:4-writeDataSourceVariable} zeigt die Callback-Methode, welche nach dem Schreiben einer Variablen auf dem OPC-Server aufgerufen wird.
Dieser Methode wird neben der NodeId der mit ihr verknüpften Variablen auch der Wert dieser in Form eines Zeigers auf ein Struct vom Typ \lstinline{UA_DataValue} übergeben.

Die Gestaltung des hier verwendeten Nodesets sieht vor, dass in einer OPC-Variablen \lstinline{"Name"} der Bezeichner des zu schreibenden Digitalausgangs hinterlegt ist, siehe Abbildung~\ref{fig:opc-do}. Dies erlaubt eine Rekonfiguration der Ein- und Ausgänge der SPS ohne Änderungen im Programmcode des OPC-Servers vornehmen zu müssen.
Es ist daher erforderlich, nach jedem Schreiben einer mit einem Digitalausgang verknüpften Variablen, hier \lstinline{"Value"}, dessen Bezeichner \lstinline{"Name"} abzufragen. 
Dies geschieht in den Zeilen 180 und 181.
Anschließend wird dieser Bezeichner sowie der zu schreibende Wert der Methode \lstinline{PI_writeSingleIO()} übergeben, welche wiederum die Interaktion mit piControl übernimmt (vgl. Abschnitt \ref{sec:4-picontrol}).
 
\subsection{Integration von piControl%
     \label{sec:4-picontrol}}
In Abschnitt~\ref{sec:2-io} wurde die Anbindung der IO-Module des Revolution Pi sowie die Funktionsweise von piControl aus Anwendersicht beschrieben. Die verfügbare Literatur beschränkt sich auch auf lediglich diese Sicht; eine weiterführende Dokumentation für Entwickler gibt es, neben der in Abschnitt~\ref{sec:3-anbindung} vorgestellten Manpage, nicht. 
In diesem Abschnitt soll daher der Quellcode von piControl sowie dessen Verwendung im Projekt genauer betrachtet werden.
Hierzu wird exemplarisch die in Abschnitt~\ref{sec:4-open62541} eingeführte Methode \lstinline{PI_writeSingleIO()} untersucht.
Diese Methode ermöglicht das Setzen eines einzelnen Bits im Prozessabbild der SPS, und damit das Schalten eines digitalen Ausgangs auf einem IO-Modul.
Die äquivalente Methode \lstinline{int piControlGetBitValue(SPIValue *pSpiValue)} zum Lesen eines Bits bzw. Eingangs funktioniert analog und soll daher an dieser Stelle nicht dediziert erörtert werden.

\begin{lstlisting}[language={c},firstnumber=97,
                   caption={Setzen eines phsikalischen, digitalen Ausgangs in \lstinline{revpi.c}
                   \label{lst:4-PI_writeSingleIO}}]
extern void PI_writeSingleIO(char *pszVariableName, bool *bit, bool verbose)
{
	int rc;
	SPIVariable sPiVariable;
	SPIValue sPIValue;

	strncpy(sPiVariable.strVarName, pszVariableName, sizeof(sPiVariable.strVarName));
	rc = piControlGetVariableInfo(&sPiVariable);
	if (rc < 0) {
		printf("Cannot find variable '%s'\n", pszVariableName);
		return;
	}

		sPIValue.i16uAddress = sPiVariable.i16uAddress;
		sPIValue.i8uBit = sPiVariable.i8uBit;
		sPIValue.i8uValue = *bit;
		rc = |>\tikzmarkin[set border color=martinired]{setBitValue}<|piControlSetBitValue(&sPIValue)|>\tikzmarkend{setBitValue}<|;
		if (rc < 0)
			printf("Set bit error %s\n", getWriteError(rc));
		else if (verbose)
			printf("Set bit %d on byte at offset %d. Value %d\n", sPIValue.i8uBit, sPIValue.i16uAddress,
			       sPIValue.i8uValue);
}
\end{lstlisting}

Der Programmcode in Listing~\ref{lst:4-PI_writeSingleIO} ist Teil des implementierten OPC-Servers. In diesem wird auf zwei Funktionen des piControl-Treibers zugegriffen. 
Beiden Methoden wird als Argument ein Zeiger auf ein Struct vom Typ \lstinline{SPIValue} übergeben. Der im Struct abgelegte Name wird mittels \lstinline{piControlGetVariableInfo(&sPIValue)} zu einer Adresse im Prozessabbild aufgelöst. Diese wird in \lstinline{sPIValue.i16uAdress} gespeichert. Der Wert der Variablen wird anschließend mittels \lstinline{piControlSetBitValue(&sPIValue)} an dieser Adresse in das Prozessabbild geschrieben.

\begin{lstlisting}[language={c},firstnumber=309,caption={Methode \lstinline{piControlSetBitValue} in \lstinline{piControlIf.c}\label{lst:4-piControlSetBitValue}}]
int |>\tikzmarkin[set border color=martiniblue]{setBitValueFcn}<|piControlSetBitValue(SPIValue *pSpiValue)|>\tikzmarkend{setBitValueFcn}<|
{
    piControlOpen();

    if (PiControlHandle_g < 0)
	    return -ENODEV;

    pSpiValue->i16uAddress += pSpiValue->i8uBit / 8;
    pSpiValue->i8uBit %= 8;

    if (|>\tikzmarkin[set border color=martinired]{ioctl}<|ioctl(PiControlHandle_g, KB_SET_VALUE, pSpiValue)|>\tikzmarkend{ioctl}<| < 0)
	    return errno;

    return 0;
}
\end{lstlisting}

Die in Listing~\ref{lst:4-piControlSetBitValue} dargestellte Methode \lstinline{piControlSetBitValue} ist lediglich eine Hüllfunktion (häufig auch als Wrapper-Funktion bezeichnet) für einen Aufruf des \lstinline{ioctl} Kernel-Moduls.
Folgende Parameter werden übergeben:
\lstinline{PiControlHandle_g} ist die Referenz auf die Geräte-Datei des piControl-Treibers. \lstinline{KB_SET_VALUE} ist das ioctl-Kommando zum Schreiben eines Bits in das Prozessabbild. Der Zeiger \lstinline{pSpiValue} verweist auf ein Struct des bereits vorgestellten Typs \lstinline{SPIValue}.

\begin{lstlisting}[language={c},firstnumber=80,caption={Methode \lstinline{piControlOpen} in \lstinline{piControlIf.c}\label{lst:4-piControlOpen}}]
void piControlOpen(void)
{
    /* open handle if needed */
    if (PiControlHandle_g < 0)
    {
	    |>\tikzmarkin[set border color=martiniblue]{PiControlHandle}<|PiControlHandle_g = open(PICONTROL_DEVICE, O_RDWR)|>\tikzmarkend{PiControlHandle}<|;
    }
}
\end{lstlisting}

Die in Listing~\ref{lst:4-piControlOpen} dargestellte Methode öffnet, sofern nicht bereits geschehen, die Geräte-Datei. Das Macro \lstinline{PICONTROL_DEVICE} verweist hierbei auf \lstinline{/dev/piControl0}.

\begin{lstlisting}[language={c},firstnumber=721,caption={Methode \lstinline{piControlIoctl} in \lstinline{piControlMain.c}\label{lst:4-piControlIoctl}}]
static long |>\tikzmarkin[set border color=martiniblue, below offset=0.9em]{piControlIoctl}<|piControlIoctl(struct file *file, unsigned int prg_nr, 
                           unsigned long usr_addr)                                      |>\tikzmarkend{piControlIoctl}<|
{
  int status = -EFAULT;
  tpiControlInst *priv;
  int timeout = 10000;	// ms

  if (prg_nr != KB_CONFIG_SEND && prg_nr != KB_CONFIG_START && !isRunning()) {
  	return -EAGAIN;
  }

  priv = (tpiControlInst *) file->private_data;

  if (prg_nr != KB_GET_LAST_MESSAGE) {
  	// clear old message
  	priv->pcErrorMessage[0] = 0;
  }

  switch (prg_nr) {|>\setcounter{lstnumber}{864}<|

    case |>\tikzmarkin[set border color=martiniblue]{KB_SET_VALUE}<|KB_SET_VALUE:|>\tikzmarkend{KB_SET_VALUE}<|
  		{
  			SPIValue *pValue = (SPIValue *) usr_addr;

  			if (!isRunning())
  				return -EFAULT;

  			if (pValue->i16uAddress >= KB_PI_LEN) {
  				status = -EFAULT;
  			} else {
  				INT8U i8uValue_l;
  				my_rt_mutex_lock(&piDev_g.lockPI);
  				i8uValue_l = piDev_g.ai8uPI[pValue->i16uAddress];

  				if (pValue->i8uBit >= 8) {
  					i8uValue_l = pValue->i8uValue;
  				} else {
  					if (pValue->i8uValue)
  						i8uValue_l |= (1 << pValue->i8uBit);
  					else
  						i8uValue_l &= ~(1 << pValue->i8uBit);
  				}

  				|>\tikzmarkin[set border color=martinired]{i8uValue}<|piDev_g.ai8uPI[pValue->i16uAddress] = i8uValue_l;|>\tikzmarkend{i8uValue}<|
  				rt_mutex_unlock(&piDev_g.lockPI);

  #ifdef VERBOSE
  				pr_info("piControlIoctl Addr=%u, bit=%u: %02x %02x\n", pValue->i16uAddress, pValue->i8uBit, pValue->i8uValue, i8uValue_l);
  #endif

  				status = 0;
  			}
  		}
  		break; |>\setcounter{lstnumber}{1314}<|

    default:
      pr_err("Invalid Ioctl");
      return (-EINVAL);
      break;

    }

    return status;
  }
\end{lstlisting}

Listing~\ref{lst:4-piControlIoctl} zeigt in Auszügen die ioctl-Methode des piControl Kernel-Treibers. Diese bekommt folgende Argumente übergeben: \lstinline{struct file *file} enthält den Verweis auf die Geräte-Datei, hier \lstinline{/dev/piControl0}. Der Wert von \lstinline{unsigned int prg_nr} beschreibt die Anfrage an den Treiber, in diesem Fall \lstinline{KB_SET_VALUE}. Das Argument \lstinline{unsigned long usr_addr} enthält einen typ-agnostischen Pointer. Dieser verweist auf einen Speicherbereich, in welchem die zur Bearbeitung der Anfrage notwendigen Daten abgelegt sind. Hier können auch vom Treiber empfangene Daten dem Anwendungsprogramm bereitgestellt werden. 

Die switch-case-Anweisung führt die über das Argument \lstinline{prg_nr} spezifizierte Aktion aus. Hier betrachten wir \lstinline{KB_SET_VALUE}:
Zunächst wird in Zeile 868 der übergebene Zeiger \lstinline{usr_addr} mittels explizitem Typecast zu einem Zeiger des Typs \lstinline{SPIValue *} konvertiert. Da dieser auf Daten im Userspace verweist, ist beim Zugriff durch den Kernel-Treiber besondere Vorsicht geboten.
In Zeile 877 wird mittels Mutex das Prozessabbild \lstinline{piDev_g} für den Zugriff durch andere Threads oder Prozesse gesperrt.
\lstinline{my_rt_mutex_lock} verweist hierbei auf die Funktion \lstinline{rt_mutex_lock} aus \lstinline{linux/sched.h}\footnote{Offenbar wurde hier auch eine alternative Implementierung vorgesehen, siehe revpi\_common.h}

In Zeile 889 wird das Byte \lstinline{i8uValue_l}, welches den zu schreibenden Wert enthält in das Prozessabbild übertragen. Anschließend wird die Mutex auf \lstinline{piDev_g} wieder entsperrt.
\newpage

\begin{lstlisting}[language={c},firstnumber=62,caption={Auszug des Struct \lstinline{spiControlDev} in \lstinline{piControlMain.h}\label{lst:4-spiControlDev}}]
|>\tikzmarkin[set border color=martiniblue]{spiControlDev}<|typedef struct spiControlDev|>\tikzmarkend{spiControlDev}<| {
	// device driver stuff
	int init_step;
	enum revpi_machine machine_type;
	void *machine;
	struct cdev cdev;	// Char device structure
	struct device *dev;
	struct thermal_zone_device *thermal_zone;

	|>\tikzmarkin[set border color=martiniblue]{processImage}<|// process image stuff
	INT8U ai8uPI[KB_PI_LEN];
	INT8U ai8uPIDefault|>\tikzmarkin[set border color=martinired]{KB_PI_LEN_0}<|[KB_PI_LEN]|>\tikzmarkend{KB_PI_LEN_0}<|;
	struct rt_mutex lockPI;        |>\tikzmarkend{processImage}<|
	bool stopIO;
	piDevices *devs; |>\setcounter{lstnumber}{94}<|
} tpiControlDev;
\end{lstlisting}

Das Prozessabbild ist als Byte-Array der Länge \lstinline{KB_PI_LEN} in Listing~\ref{lst:4-spiControlDev} definiert. Konfigurationsparameter wie \lstinline{KB_PI_LEN} oder die Zykluszeit für den Datenaustausch zwischen SPS und IO-Modulen sind im folgenden Listing~\ref{lst:4-process} definiert.

\begin{lstlisting}[language={c},firstnumber=119,caption={Konfigurationsparameter des Prozessabbildes in project.h\label{lst:4-process}}]
#define INTERVAL_PI_GATE (5*1000*1000)  // 5 ms piGateCommunication |>\setcounter{lstnumber}{128}<|

#define INTERVAL_IO_COM (5*1000*1000)  // 5 ms piIoComm |>\setcounter{lstnumber}{132}<|

#define KB_PD_LEN       512
|>\tikzmarkin[set border color=martiniblue]{KB_PI_LEN_1}<|#define KB_PI_LEN       4096|>\tikzmarkend{KB_PI_LEN_1}<|
\end{lstlisting}

Das zu setzende Bit wurde zu diesem Zeitpunkt erfolgreich in das Prozessabbild der SPS geschrieben.
Es stellt sich die Frage, wie dieses nun an das IO-Modul kommuniziert wird.
Die Kommunikation mit allen angebundenen Modulen ist ebenfalls Aufgabe des piControl-Treibers.

\begin{lstlisting}[language={c},firstnumber=256,caption={Auszug der Methode \lstinline{piIoThread} in \lstinline{revpi_core.c}\label{lst:4-piIoThread}}]
static int piIoThread(void *data)
{
	//TODO int value = 0;
	ktime_t time;
	ktime_t now;
	s64 tDiff;

	hrtimer_init(&piCore_g.ioTimer, CLOCK_MONOTONIC, HRTIMER_MODE_ABS);
	piCore_g.ioTimer.function = piIoTimer;

	pr_info("piIO thread started\n");

	now = hrtimer_cb_get_time(&piCore_g.ioTimer);

	PiBridgeMaster_Reset();

	while (!kthread_should_stop()) {
		if (|>\tikzmarkin[set border color=martinired]{PiBridgeMaster}<|PiBridgeMaster_Run()|>\tikzmarkend{PiBridgeMaster}<| < 0)
			break;
	}

	RevPiDevice_finish();

	pr_info("piIO exit\n");
	return 0;
}
\end{lstlisting}

Der Kernel-Thread \lstinline{piIoThread} ist verantwortlich für den zyklischen Datenaustausch mit den IO-Modulen. In diesem wird fortlaufend die Methode \lstinline{PiBridgeMaster_Run()} aufgerufen, siehe Listing~\ref{lst:4-piIoThread}.

\begin{lstlisting}[language={c},firstnumber=262,caption={Auszug der Methode \lstinline{PiBridgeMaster_Run(void)} in \lstinline{RevPiDevice.c}\label{lst:4-PiBridgeMaster_Run}}]
int PiBridgeMaster_Run(void)
{
	static kbUT_Timer tTimeoutTimer_s;
	static kbUT_Timer tConfigTimeoutTimer_s;
	static int error_cnt;
	static INT8U last_led;
	static unsigned long last_update;
	int ret = 0;
	int i;

	my_rt_mutex_lock(&piCore_g.lockBridgeState);
	if (piCore_g.eBridgeState != piBridgeStop) {
		switch (eRunStatus_s) { |>\setcounter{lstnumber}{514}<|
		    case enPiBridgeMasterStatus_EndOfConfig:|>\setcounter{lstnumber}{621}<|
		    if (|>\tikzmarkin[set border color=martinired]{RevPiDevice}<|RevPiDevice_run()|>\tikzmarkend{RevPiDevice}<|) {
				// an error occured, check error limits |>\setcounter{lstnumber}{641}<|
			} else {
				ret = 1;
			}
			piCore_g.image.drv.i16uRS485ErrorCnt = RevPiDevice_getErrCnt();
			break;
\end{lstlisting}

Die in Listing~\ref{lst:4-PiBridgeMaster_Run} dargestellte Methode ist eine sog. State-Machine. Ist die Konfiguration der IO-Module erfolgreich abgeschlossen, so führt sie bei Aufruf lediglich die Methode \lstinline{RevPiDevice_run()} aus.

\begin{lstlisting}[language={c},firstnumber=140,caption={Auszug der Methode \lstinline{RevPiDevice_run(void)} in \lstinline{RevPiDevice.c}\label{lst:4-RevPiDevice_run}}]
int RevPiDevice_run(void)
{
	INT8U i8uDevice = 0;
	INT32U r;
	int retval = 0;

	RevPiDevices_s.i16uErrorCnt = 0;

	for (i8uDevice = 0; i8uDevice < RevPiDevice_getDevCnt(); i8uDevice++) {
		if (RevPiDevice_getDev(i8uDevice)->i8uActive) {
			switch (RevPiDevice_getDev(i8uDevice)->sId.i16uModulType) {
			case KUNBUS_FW_DESCR_TYP_PI_DIO_14:
			case KUNBUS_FW_DESCR_TYP_PI_DI_16:
			case KUNBUS_FW_DESCR_TYP_PI_DO_16:
				r = |>\tikzmarkin[set border color=martinired]{sendCyclicTelegram}<|piDIOComm_sendCyclicTelegram(i8uDevice)|>\tikzmarkend{sendCyclicTelegram}\setcounter{lstnumber}{166} <|;

				break; |>\setcounter{lstnumber}{216}<|
			}
		}
	} |>\setcounter{lstnumber}{227}<|
	return retval;
}
\end{lstlisting}

Diese iteriert wie in Listing~\ref{lst:4-RevPiDevice_run} abgebildete durch alle gegenwärtig in der SPS konfigurierten Module. Ist das aktuelle Modul als aktiv markiert, so wird anhand eines sog. Firmware-Descriptors entschieden, welche Methode für die Ansteuerung des Moduls aufzurufen ist.

\begin{lstlisting}[language={c},firstnumber=161,caption={Auszug der Methode \lstinline{piDIOComm_sendCyclicTelegram} in \lstinline{piDIOComm.c}\label{lst:4-sendCyclicTelegram}}]
INT32U piDIOComm_sendCyclicTelegram(INT8U i8uDevice_p)
{
	INT32U i32uRv_l = 0;
	SIOGeneric sRequest_l;
	SIOGeneric sResponse_l;
	INT8U len_l, data_out[18], i, p, data_in[70];
	INT8U i8uAddress;
	int ret; |>\setcounter{lstnumber}{239}<|
	
    |>\tikzmarkin[set border color=martinired]{piIoComm}<|ret = piIoComm_send((INT8U *) & sRequest_l, IOPROTOCOL_HEADER_LENGTH + len_l + 1);  |>\tikzmarkend{piIoComm}\setcounter{lstnumber}{298}<|
}
\end{lstlisting}

Im Falle des hier verwendeten DO-Moduls wird die in Listing~\ref{lst:4-sendCyclicTelegram} abgebildete Methode \lstinline{piDIOComm_sendCyclicTelegram()} aufgerufen. Dieser wird ein Zeiger auf das zu schreibende Gerät übergeben. 
Zunächst wird das Prozessabbild mittels eines proprietären, jedoch im Quellcode offen nachvollziehbaren Protokolls in ein \lstinline{sRequest_l} genanntes Byte-Array umgewandelt. Dieser Schritt ist in Listing~\ref{lst:4-sendCyclicTelegram} nicht abgebildet. Anschließend wird \lstinline{piIoComm_send()} ein Zeiger auf die so generierte Schreib-Anfrage übergeben.

\begin{lstlisting}[language={c},firstnumber=220,caption={Auszug der Methode \lstinline{piIOComm_send} in \lstinline{piIOComm.c}\label{lst:4-piIOComm_send}}]
int piIoComm_send(INT8U * buf_p, INT16U i16uLen_p)
{
	ssize_t write_l = 0;
	INT16U i16uSent_l = 0;|>\setcounter{lstnumber}{249}<|

	while (i16uSent_l < i16uLen_p) {
		write_l = vfs_write(piIoComm_fd_m, buf_p + i16uSent_l, i16uLen_p - i16uSent_l, &piIoComm_fd_m->f_pos);
		if (write_l < 0) {
			pr_info_serial("write error %d\n", (int)write_l);
			return -1;
		} 
		i16uSent_l += write_l;|>\setcounter{lstnumber}{263}<|
	}
	clear();
	vfs_fsync(piIoComm_fd_m, 1);
	return 0;
}
\end{lstlisting}

Listing~\ref{lst:4-piIOComm_send} zeigt die Implementierung von \lstinline{piIoComm_send()}. Diese Methode ist für das Schreiben der oben generierten Anfrage auf die seriellen Schnittstelle verantwortlich. Realisiert wird dies mittels der Methode \lstinline{vfs_write()}. Diese ist in \lstinline{<linux/fs.h>} definiert. Sie ermöglicht das Schreiben einer Datei im Userspace aus dem Kernel heraus. Geschrieben wird hier die Datei mit dem Deskriptor \lstinline{piIoComm_fd_m}.
Da die Funktion \lstinline{vfs_write()} durch andere Kernel-Tasks unterbrochen werden kann, ist nicht gewährleistet, dass die gesamte Anfrage mit nur einem Aufruf geschrieben wird. Die oben abgebildete while-Schleife stellt das vollständige Senden der Anfrage sicher.

\begin{lstlisting}[language={c},firstnumber=157,caption={Auszug der Methode \lstinline{piIOComm_open_serial} in \lstinline{piIOComm.c}\label{lst:4-piIOComm_open_serial}}]
int piIoComm_open_serial(void)
{   |>\setcounter{lstnumber}{167}<|
	struct file *fd;	/* Filedeskriptor */
	struct termios newtio;	/* Schnittstellenoptionen */

	|>\tikzmarkin[set border color=martiniblue]{fd}<|/* Port oeffnen - read/write, kein "controlling tty", 
	    Status von DCD ignorieren */
	fd = filp_open(|>\tikzmarkin[set border color=martinired]{tty}<|REV_PI_TTY_DEVICE|>\tikzmarkend{tty}<|, O_RDWR | O_NOCTTY, 0); |>\setcounter{lstnumber}{208}<|
	
	piIoComm_fd_m = fd;                                                      |>\tikzmarkend{fd}\setcounter{lstnumber}{217}<|

	return 0;
}
\end{lstlisting}

Der zum Schreiben auf die serielle Schnittstelle verwendete Datei-Deskriptor wird von der in Listing~\ref{lst:4-piIOComm_open_serial} abgebildeten Methode \lstinline{piIoComm_open_serial()} generiert. 

\begin{lstlisting}[language={c},firstnumber=45,caption={Definition der seriellen Schnittstelle in \lstinline{piIOComm.h}\label{lst:4-REV_PI_TTY_DEVICE}}]
#define REV_PI_TTY_DEVICE	"/dev/ttyAMA0"
\end{lstlisting}

Das in Listing~\ref{lst:4-REV_PI_TTY_DEVICE} definierte Macro verweist auf eine der seriellen Schnittstellen des RaspberryPi.
Die Implementierung des zugehörigen Schnittstellentreibers soll hier nicht weiter untersucht werden. Somit ist an dieser Stelle die Kette vom Setzen einer Variablen auf dem OPC-Server bis hin zur Aktualisierung des Prozessabbilds der IO-Module geschlossen.

% \begin{lstlisting}[language={c},firstnumber={226},caption={Setzen der Scheduler-Priorität auf SCHED\_FIFO in 
% revpi\_common.c\label{lst:2-sched_priority}}]
% param.sched_priority = ktprio->prio;
% ret = sched_setscheduler(child, SCHED_FIFO, &param);
% \end{lstlisting}
% % % Imports nur für Referenzenauflösung während des Schreibens! Vorm Kompilieren auskommentieren!
% \bibliography{0_hauptdatei}
% \input{1_einleitung}
% \input{2_grundlagen}
% \input{3_konzeption}
% \input{4_implementierung}
% \input{5_tests}
% \input{6_zusammenfassung}
% % Ende Imports

\section{Test des OPC-Servers im Gesamtsystem%
  \label{sec:5-tests}}

% % % Imports nur für Referenzenauflösung während des schreibens! Vorm Kompilieren auskommentieren!
% \bibliography{0_hauptdatei}
% \input{1_einleitung}
% \input{2_grundlagen}
% \input{3_konzeption}
% \input{4_implementierung}
% \input{5_tests}
% \input{6_zusammenfassung}
% % Ende Imports

\section{Zusammenfassung und Ausblick%
  \label{sec:6-fazit}}
Der folgende Abschnitt~\ref{sec:6-zusammenfassung} fasst die gewonnenen Erkenntnisse und den Stand der Implementierung zusammen.
Den Abschluss dieser Arbeit bildet der Ausblick in Abschnitt~\ref{sec:6-ausblick}.

\subsection{Zusammenfassung%
     \label{sec:6-zusammenfassung}}

\subsection{Ausblick%
     \label{sec:6-ausblick}}

% % Ende Imports

\section{Test des OPC-Servers im Gesamtsystem%
  \label{sec:5-tests}}

% % % Imports nur für Referenzenauflösung während des schreibens! Vorm Kompilieren auskommentieren!
% \bibliography{0_hauptdatei}
% % Mit \section{...} eröffnen wir einen neuen Abschnitt.
% Der Befehl setzt nicht nur den Text in einer größeren,
% fetten Schrift, sondern sorgt außerdem dafür, daß er im
% Inhaltsverzeichnis erscheint.
%
% Mit \label{...} erzeugen wir einen Bezeichner, mit dessen Hilfe
% wir später auf die Nummer des Abschnitts verweisen können (nämlich
% mit~\ref{...}).
%
% Das Kommentarzeichen hinter „Übersicht“ dient dazu, ein
% Leerzeichen zwischen „Übersicht“ und dem \label-Befehl
% zu vermeiden, das andernfalls sichtbar würde – z.B. im
% Inhaltsverzeichnis.
%

% % Imports nur für Referenzenauflösung während des Schreibens! Vorm Kompilieren auskommentieren!
% \bibliography{0_hauptdatei}
% \input{1_einleitung}
%\input{2_grundlagen}
%\input{3_konzeption}
%\input{4_implementierung}
%\input{5_tests}
%\input{6_zusammenfassung}
% % Ende Imports

\section{Einleitung und Motivation%
  \label{sec:1-einleitung}}
Ziel dieses Projektes ist die Integration eines OPC-Servers mit einer auf Linux
basierenden speicherprogrammierbaren Steuerung (SPS). Angeschlossen an diese SPS
ist jeweils ein digitales Ein-/\,bzw.~Ausgabemodul. Die von diesen bereitgestellten
Ein-/\, bzw.~Ausgänge (IO) sollen in der Datenstruktur des OPC-Servers abgebildet
und über diesen für OPC-Clients les-/\,und schreibar sein. Weiterhin sollen einige
Funktionen zur Überwachung und Steuerung der an die SPS angeschlossenen Aktoren
und Sensoren direkt im OPC-Server implementiert werden.
Hiermit stellt dieses Projekt eine der Grundlagen für ein übergeordnetes Projekt,
die cloudbasierte Steuerung eines miniaturisierten Produktions-Systems, dar.

Der hier verwendete OPC-Server ist Teil des sog. open62541 Projekts. Er ist in C
geschrieben und implementiert bereits einen großen Teil der im OPC-UA-Standard
spezifizierten Funktionen.
Als SPS findet ein Revolution Pi 3 der Firma Kunbus Verwendung. Dieser integriert
ein sog. Compute Module der Raspberry Pi Foundation in ein industrietaugliches
Gehäuse und erlaubt die Erweiterung mittels IO- oder Gateway-Modulen. Über diese
erfolgt die Kommunikation mit weiteren Komponenten der Automatisierungstechnik.

Motiviert ist dieses Projekt durch die Beobachtung, dass die Verbreitung offener
Standards sowie freier Software auch in der Automatisierungstechnik zunimmt.
Linux ist ein freies Betriebssystem, OPC-UA ein offen zugänglicher, aktiv gepflegter
und weit verbreiteter Standard. Der Raspberry Pi findet sowohl bei Hobby-Anwendern als
auch in den Bereichen Forschung und Entwicklung sowie bei industriellen Anwendern
Verwendung. Dieses Projekt stellt somit eine für unterschiedliche Anwender interessante
Entwicklung dar.

Im Anschluss an diese einleitende Übersicht im Abschnitt~\ref{sec:1-einleitung} folgt
die Darstellung der wichtigsten Grundlagen in Abschnitt~\ref{sec:2-grundlagen}.
Aufbauend auf diesen Grundlagen folgt die konzeptuelle Ausarbeitung im Abschnitt~\ref{sec:3-konzeption}.
Die Umsetzung wird im Abschnitt~\ref{sec:4-implementierung} erläutert.
Die Leistungsfähigkeit der Implementierung wird in Abschnitt~\ref{sec:5-tests} untersucht.
Eine Zusammenfassung und ein Ausblick schließen die Arbeit in
Abschnitt~\ref{sec:6-fazit} ab. Eventuell noch benötigte Anhänge
finden sich in den Anhängen [...] bis [...].

% % % Imports nur für Referenzenauflösung während des Schreibens! Vorm Kompilieren auskommentieren!
% \bibliography{0_hauptdatei}
% \input{1_einleitung}
% \input{2_grundlagen}
% \input{3_konzeption}
% \input{4_implementierung}
% \input{5_tests}
% \input{6_zusammenfassung}
% % Ende Imports

\section{Grundlagen%
  \label{sec:2-grundlagen}}

\subsection{Speicherprogrammierbare-Steuerung und Linux -- Revolution Pi%
     \label{sec:2-sps}}

\subsubsection{Kunbus RevolutionPi%
        \label{sec:2-revpi}}
Der RevolutionPi 3 ist eine speicherprogrammierbare Steuerung (SPS) des Herstellers
Kunbus GmbH. Kern dieser SPS ist das von der Raspberry Pi Foundation entwickelte
und vertriebene Raspberry Pi Compute Module 3. Dieses integriert ein Broadcom BCM2837
System-on-Chip (SoC) mit vier 1,2GHz Prozessorkernen, 1GB RAM, 4GB eMMC Anwendungsspeicher
und sonstige Peripherie in ein Modul im DDR2-SODIMM Formfaktor. Diese Spezifikationen
sind weitgehend identisch zu denen des ausgesprochen populären Raspberry Pi 3.
Der Revolution Pi profitiert daher von dem gleichen großen Angebot an Software
und Unterstützung wie der Raspberry Pi, ergänzt dessen Hardware jedoch um eine 24V
Spannungsversorgung, die Möglichkeit der Erweiterung durch mehrere industrietaugliche
Ein-/ Ausgabemodule und Gateways sowie ein Gehäuse zur Montage auf einer DIN-Schiene.
\begin{itemize}
  \item{Prozessor: BCM2837}
  \item{Taktfrequenz 1,2 GHz}
  \item{Anzahl Prozessorkerne: 4}
  \item{Arbeitsspeicher: 1 GByte}
  \item{eMMC Flash Speicher: 4 GByte}
  \item{Betriebssystem: Angepasstes Raspbian mit RT-Patch}
  \item{RTC mit 24h Pufferung über wartungsfreien Kondensator}
  \item{Treiber / API: Treiber schreibt zyklisch Prozessdaten in ein Prozessabbild, Zugriff auf Prozessabbild über Linux-Filesystem als API zu Fremdsoftware.}
  \item{Kommunikationsanschlüsse: 2 x USB 2.0 A (je 500 mA belastbar), 1 x Micro-USB, HDMI, Ethernet (RJ45) 10/100 Mbit/s}
  \item{Stromversorgung: min. 10,7 V, max. 28,8 V, maximal 10 Watt}
  \item{Zulässige Umgebungstemperatur: -40 bis +55 C}
  \item{Gehäuseabmessungen: (HxBxL) 96 mm x 22,5 mm x 110,5 mm (ohne gesteckte Stecker)}
  \item{ESD Schutz: 4 kV / 8 kV gemäß EN61131-2 und IEC 61000-6-2}
  \item{Surge / Burst Prüfungen: gemäß EN61131-2 und IEC 61000-6-2 eingekoppelt auf Versorgungsspannung, Ethernet und IO-Leitungen}
  \item{EMI Prüfungen: gemäß EN61131-2 und IEC 61000-6-2}
\end{itemize}

Kunbus bietet eine Auswahl an IO- und Gateway-Modulen zur Erweiterung des Revolution Pi an.
Gateways dienen der Kommunikation mit Systemen oder Komponenten der Automatisierungstechnik
über Protokolle wie PROFIBUS oder EtherCAT. IO-Module erlauben die Überwachung
und Steuerung von digitalen oder analogen Ein- und Ausgängen.

\subsubsection{Zugriff auf IO-Module%
        \label{sec:2-io}}
Der Zugriff auf die Ein- und Ausgänge der IO-Module erfolgt über ein Prozessabbild
und einen hierfür von Kunbus bereitgestellten Treiber, genannt piControl. Dieser
aktualisiert das Prozessabbild zyklisch. Die angestrebte Zykluszeit beträgt 5ms,
kann jedoch je nach Anzahl der angeschlossenen Module auch größer sein. Kunbus
garantiert bei drei IO-Modulen und zwei Gateway-Modulen eine Zykluszeit von 10 ms.
Jedes der IO-Module stellt ein eigenständiges eingebettetes System dar. Es verfügt
über einen Microcontroller, welcher die IOs bereitstellt und über einen RS485-Bus
mit dem Revolution Pi kommuniziert.
% https://revolution.kunbus.de/io-modul/

Lizenz: GPL
% https://github.com/RevolutionPi/piControl

\begin{lstlisting}[language={c},firstnumber={226},caption={Setzen der Scheduler-Priorität auf SCHED\_FIFO in revpi\_common.c\label{lst:2-sched_priority}}]
param.sched_priority = ktprio->prio;
ret = sched_setscheduler(child, SCHED_FIFO,
       &param);
\end{lstlisting}


\subsection{Echtzeit und Multithreading unter Linux -- preemptRT und posix%
     \label{sec:2-echtzeit}}


 Der Linux-Kernel verfügt über mehrere unterschiedliche Preemtion-Modelle:

\begin{itemize}
  \item No Forced Preemption (server):
  Ausgelegt auf maximal möglichen Durchsatz, lediglich Interrupts und
  System-Call-Returns bewirken Präemption.

  \item Voluntary Kernel Preemption (Desktop):
  Neben den implizit bevorrechtigten Interrupts und System-Call-Returns gibt es
  in diesem Modell weitere Abschnitte des Kernels in welchen Preämption explizit
  gestattet ist.

  \item Preemptible Kernel (Low-Latency Desktop):
  In diesem Modell ist der gesamte Kernel, mit Ausnahme sog.~kritischer Abschnitte
  präemptible. Nach jedem kritischen Abschnitt gibt es einen impliziten Präemptions-Punkt.

  \item Preemptible Kernel (Basic RT):
  Dieses Modell ist dem zuvor genannten sehr ähnlich, hier sind jedoch alle Interrupt-Handler
  als eigenständige Threads ausgeführt.

  \item Fully Preemptible Kernel (RT):
  Wie auch bei den beiden zuvor genannten Modellen ist hier der gesamte Kernel
  präemtible, die Anzahl und Dauer der nicht-präemtiblen kritischen Abschnitte
  ist auf ein notwendiges Minimum beschränkt. Alle Interrupt-Handler sind als
  eigenständige Threads ausgeführt, Spinlocks durch Sleeping-Spinlocks und Mutexe
  durch sog.~RT-Mutexe ersetzt.

\end{itemize}
\todo{Spinlocks und Mutexe sowie die RT-Varianten dieser erklären!}

Lediglich mit dem vollständig präemtiblen Kernel kann Echtzeit-Verhalten realisiert werden.

% https://wiki.linuxfoundation.org/realtime/documentation/technical_basics/preemption_models bzw kernel/Kconfig.preempt

\subsubsection{preemptRT%
        \label{sec:2-preemptRT}}
% https://wiki.linuxfoundation.org/realtime/documentation/technical_details/start
% https://wiki.linuxfoundation.org/realtime/documentation/technical_basics/start

Das dem PREEMPT RT Kernel zugrunde liegende Prinzip lässt sich in einer einfachen
Regel ausdrücken: Nur Code, welcher absolut nicht-präemtible sein darf, ist es
gestattet nicht-präemtible zu sein.
Das erklärte Ziel des PREEMPT\_RT Patches ist es folglich, die Menge des nicht-präemtiblen
Codes im Linux-Kernel auf das absolut notwendige Minimum zu reduzieren.

Dies wird durch Verwendung folgender Mechanismen erreicht:

\begin{itemize}
  \item Hochauflösende Timer
  \item Sleeping Spinlocks
  \item Threaded Interrupt Handlers
  \item rt\_mutex
  \item RCU
\end{itemize}


\subsubsection{posix%
        \label{sec:2-posix}}
Ist posix hier wirklich relevant? Debian bzw.~Raspbian sind weitgehend posix
kompatibel, aber wird es hier genutzt? -> JA, open62541 nutzt pthread.h
piControl nutzt kthread.h, und semaphore.h

\subsection{OPC-UA und open62541%
     \label{sec:2-opc}}

\subsubsection{OPC UA%
        \label{sec:2-opcua}}
Open Platform Communications (OPC) ist eine Familie von Standards zur herstellerunabhängigen
Kommunikation von Maschinen (M2M) in der Automatisierungstechnik. Die sog.~OPC Task Force, zu deren
Mitgliedern verschiedene große Firmen der Automatisierungsindustrie gehören, veröffentlichte
die OPC Specification Version 1.0 im August 1996.
Motiviert ist dieser offene Standard durch die Erkenntniss, dass die Anpassung der
zahlreichen Herstellerstandards an individuelle Infrastrukturen und Anlagen einen
großen Mehraufwand verursachen.
Die Wikipedia beschreibt das Anwendungsgebiet für OPC wie folgt:

\glqq{}OPC wird dort eingesetzt, wo Sensoren, Regler und Steuerungen verschiedener Hersteller
ein gemeinsames Netzwerk bilden. Ohne OPC benötigten zwei Geräte zum Datenaustausch
genaue Kenntnis über die Kommunikationsmöglichkeiten des Gegenübers. Erweiterungen
und Austausch gestalten sich entsprechend schwierig. Mit OPC genügt es, für jedes
Gerät genau einmal einen OPC-konformen Treiber zu schreiben. Idealerweise wird
dieser bereits vom Hersteller zur Verfügung gestellt. Ein OPC-Treiber lässt sich
ohne großen Anpassungsaufwand in beliebig große Steuer- und Überwachungssysteme
integrieren.

OPC unterteilt sich in verschiedene Unterstandards, die für den jeweiligen Anwendungsfall
unabhängig voneinander implementiert werden können. OPC lässt sich damit verwenden
für Echtzeitdaten (Überwachung), Datenarchivierung, Alarm-Meldungen und neuerdings
auch direkt zur Steuerung (Befehlsübermittlung).\grqq{}

OPC basiert in der ursprünglichen Spezifikation auf Microsofts DCOM-Spezifikation.
DCOM macht Funktionen und Objekte einer Anwendung anderen Anwendungen im Netzwerk
zugänglich. Der OPC-Standard definiert entsprechende DCOM-Objekte um mit anderen
OPC-Anwendungen Daten austauschen zu können. Die Verwendung von DCOM bindet Anwender
an Betriebssysteme von Microsoft. Die ursprüngliche OPC Spezifikation wird durch die
Entwicklung von OPC Unified Architecture (OPC UA) abgelöst.
OPC UA setzt auf einem eigenen Kommunikationionsstack auf, die Verwendung von DCOM
und damit die Bindung an Microsoft wurden aufgelöst.

Die OPC-UA-Architektur ist eine Service-orientierte Architektur (SOA), deren Struktur
aus mehreren Schichten besteht.

% Wikipedia
Das OPC-Informationsmodell ist nicht mehr nur eine Hierarchie aus Ordnern, Items
und Properties. Es ist ein sogenanntes Full-Mesh-Network aus Nodes, mit dem neben
den Nutzdaten eines Nodes auch Meta- und Diagnoseinformationen repräsentiert werden.
Ein Node ähnelt einem Objekt aus der objektorientierten Programmierung. Ein Node
kann Attribute besitzen, die gelesen werden können (Data Access (DA), Historical
Data Access (HDA)). Es ist möglich Methoden zu definieren und aufzurufen.
Eine Methode besitzt Aufrufargumente und Rückgabewerte. Sie wird durch ein Command
aufgerufen. Weiterhin werden Events unterstützt, die versendet werden können
(AE (Alarms \& Events), DA DataChange), um bestimmte Informationen zwischen Geräten
auszutauschen. Ein Event besitzt unter anderem einen Empfangszeitpunkt, eine Nachricht
und einen Schweregrad. Die o. g. Nodes werden sowohl für die Nutzdaten als auch
alle anderen Arten von Metadaten verwendet. Der damit modellierte OPC-Adressraum
beinhaltet nun auch ein Typmodell, mit dem sämtliche Datentypen spezifiziert werden.

% https://de.wikipedia.org/wiki/Open_Platform_Communications
% https://de.wikipedia.org/wiki/OPC_Unified_Architecture
% https://opcfoundation.org/developer-tools/specifications-unified-architecture
% Von Gerhard Gappmeier - ascolab GmbH, CC BY-SA 3.0, https://de.wikipedia.org/w/index.php?curid=1892069
\subsubsection{open62541%
        \label{sec:2-open62541}}
open62541 ist eine offene und freie Implementierung von OPC UA. Die in C geschriebene
Bibliothek stellt eine beständig zunehmende Anzahl der im OPC UA Standard definierten
Funktionen bereit. Sie kann sowohl zur Erstellung von OPC-Servern als auch -Clients
genutzt werden. Ergänzend zu der unter der Mozilla Public License v2.0 lizensierten
Bibliothek stellt das open62541 Projekt auch Beispielprogramme unter einer CC0 Lizenz
zur Verfügung.

Die Bibliothek eignet sich auch für die Entwicklung auf eingebetteten Systemen und
Microcontrollern. Je nach Umfang der gewünschten Funktionen und des OPC Informationsmodells
beträgt die Größe einer Server-Binary weniger als 100kb. %evtl. kürzen?

\todo{Nodes erklären! Evtl.~oben!}

Folgende Auswahl an Eigenschaften und Funktionen zeichnet die in dieser Arbeit verwendete
Version 0.3 von open62541 aus:
\begin{itemize}
  \item Kommunikationionsstack
  \begin{itemize}
      \item OPC UA Binär-Protokoll (HTTP oder SOAP werden gegenwärtig nicht unterstützt)
      \item Austauschbare Netzwerk-Schicht, welche die Verwendung eigener Netzwerk-APIs
      erlaubt.
      \item Verschlüsselte Kommunikationion
      \item Asynchrone Dienst-Anfragen im Client
  \end{itemize}
  \item Informationsmodell
  \begin{itemize}
    \item Unterstützung aller OPC UA Node-Typen, inkl.~Methoden
    \item Hinzufügen und Entfernen von Nodes und Referenzen zur Laufzeit.
    \item Vererbung und Instanziierung von Objekt- und Variablentypen
    \item Zugriffskontrolle auch für einzelne Nodes
  \end{itemize}
  \item Subscriptions
  \begin{itemize}
    \item Erlaubt die Überwachung (subscriptions / monitoreditems)
    \item Sehr geringer Ressourcenbedarf pro überwachtem Wert
  \end{itemize}
  \item Code-Generierung auf XML-Basis
  \begin{itemize}
    \item Erlaubt die Erstellung von Datentypen
    \item Erlaubt die Generierung des serverseitigen Informationsmodells
  \end{itemize}
\end{itemize}

% https://open62541.org/doc/0.3/


Mozilla Public License
CC0 Lizenz für Beispiele und Plugins

% https://open62541.org/doc/open62541-current.pdf
% https://open62541.org/

% % % Imports nur für Referenzenauflösung während des Schreibens! Vorm Kompilieren auskommentieren!
% \bibliography{0_hauptdatei}
% \input{1_einleitung}
% \input{2_grundlagen}
% \input{3_konzeption}
% \input{4_implementierung}
% \input{5_tests}
% \input{6_zusammenfassung}
% \input{anhang}
% % Ende Imports

\section{Systemkonzept%
  \label{sec:3-konzeption}}
Auf Basis der in Abschnitt \ref{sec:2-grundlagen} vorgestellten Möglichkeiten folgt nun die Ausarbeitung eines Konzepts.
In den folgenden Abschnitten soll näher auf zwei zentrale Aspekte eingegangen werden: Abschnitt~\ref{sec:3-anbindung} stellt Möglichkeiten zum Zugriff auf Variablen bzw.\,Werte im Prozessabbild des Revolution Pi vor; in Abschnitt~\ref{sec:3-integration} wird ein Konzept zur Bereitstellung dieser Variablen auf einem OPC-Server vorgestellt.

\subsection{Anbindung der IO an den OPC-Server%
     \label{sec:3-anbindung}}

Eine Webanwendung mit Bezeichnung PiCtory dient zur Konfiguration der I/O- und virtuellen Module des RevolutionPi. Die Konfiguration liegt im JSON-Format in der Datei \lstinline{/etc/revpi/config.rsc}. Der piControl-Treiber liest diese Datei beim Start. 
Der folgende Auszug aus der Manpage des piControl-Kernelmoduls beschreibt die von diesem zum Lesen und Schreiben einzelner Bits des Prozessabbildes bereitgestellten Funktionen~\citep[vgl.]{web-revpi-manpage}. Sie ist an dieser Stelle weitgehend ungekürzt zitiert, da sie die nutzbare Schnittstelle sehr kompakt beschreibt.

\begin{lstlisting}[breakindent=0pt, numbers=none, caption={Auszug aus der Revolution Pi Programmers Manual\label{lst:4-manpage}}]
KB_FIND_VARIABLE SPIVariable *argp
Find a variable in the process image by its name. A pointer to a structure of type SPIVariable must be passed as argument. [...]
The struct SPIVariable [...] is defined as 
typedef struct SPIVariableStr
{
    char strVarName[32]; // Variable name
    uint16_t i16uAddress; // Address of the byte in the process image
    uint8_t i8uBit; // 0-7 bit position, >= 8 whole byte
    uint16_t i16uLength; // length of the variable in bits.
    // Possible values are 1, 8, 16 and 32
} SPIVariable;

Set and get values of the process image
KB_GET_VALUE SPIValue *argp
[...]
KB_SET_VALUE SPIValue *argp
Write one bit or one byte to the process image [...].  This call is more efficient than the usual calls of seek and write because only one function call is necessary. If more than on application are writing bits in one output byte, this call is the only safe way to set a bit without overwriting the other bits because this call is doing a read-modify-write-cycle. 

The struct SPIValue used by this ioctl is defined as
typedef struct SPIValueStr
{
    uint16_t i16uAddress; // Address of the byte in the process image
    uint8_t i8uBit; // 0-7 bit position, >= 8 whole byte
    uint8_t i8uValue; // Value: 0/1 for bit access, whole byte otherwise
} SPIValue;
\end{lstlisting} 

Die oben beschriebenden Funtkionen \lstinline{KB_FIND_VARIABLE}, \lstinline{KB_GET_VALUE} und \lstinline{KB_SET_VALUE} ermöglichen einen einfachen und (lt.\,Manpage) effizienten Zugriff auf einzelne Bits des Prozessabbildes und damit auch auf die IO des RevolutionPi.
Der Zugriff des OPC-Servers auf das Prozessabbild soll daher mittels dieser Funktionen realisiert werden.
\lstinline{KB_FIND_VARIABLE} kann genutzt werden, um Adressen von Variablen im Prozessabbild mittels ihres Namens aufzulösen.
\lstinline{KB_GET_VALUE} und \lstinline{KB_SET_VALUE} ermöglichen den Zugriff auf die Werte dieser Variablen.


\subsection{Integration des OPC-Servers in das System%
     \label{sec:3-integration}}

open62541 bietet drei Möglichkeiten zum Abgleich von Variablen mit dem Prozessabbild~\citep[vgl.][Tutorials - Connecting a Variable with a Physical Process]{web-open62541}:
\begin{itemize}
    \item Manuelles oder zyklisches Aktualisieren
    \item Variable Value Callback
    \item Variable Datasource
\end{itemize}

Die zyklische Aktualisierung eines oder mehrerer Werte nimmt, abhängig von der Zykluszeit, viele Systemressourcen in Anspruch. Value Callbacks ermöglichen es, einen Variablenwert effizienter mit einer Ressource wie etwa einem Prozessabbild zu synchronisieren. An die Variable wird ein Callback angehängt, welches vor jedem Lesen und nach jedem Schreibvorgang ausgeführt wird.
Der Wert der Variablen wird weiterhin im Variablenknoten auf dem OPC-Server gespeichert, der Abgleich mit der verknüpften Ressource erfolgt durch die Callback-Methoden.

Sogenannte Datenquellen gehen noch einen Schritt weiter. Der Server leitet jede Lese- und Schreibanforderung direkt an eine Callback-Funktion weiter. Beim Lesen liefert der Rückruf eine Kopie des aktuellen Wertes. Die Datenquelle muss intern ein eigenes Speichermanagement implementieren.

Der Zugriff auf die Werte des Prozessabbildes erfolgt, wie in Abschnitt~\ref{sec:3-anbindung} beschrieben, über von piControl bereitgestellte Methoden. Um die durch open62541 gepflegte OPC-Datenstruktur und das durch piControl verwaltete Prozessabbild möglichst effektiv verknüpfen zu können, soll diese Interaktion mittels Datenquellen und den zugehörigen Callbacks implementiert werden.
% % % Imports nur für Referenzenauflösung während des Schreibens! Vorm Kompilieren auskommentieren!
% \bibliography{0_hauptdatei}
% \input{1_einleitung}
% \input{2_grundlagen}
% \input{3_konzeption}
% \input{4_implementierung}
% \input{5_tests}
% \input{6_zusammenfassung}
% \input{anhang}
% % Ende Imports

\section{Implementierung%
  \label{sec:4-implementierung}}
Das folgende Kapitel stellt in Auszügen die Implementierung des OPC-Servers sowie die Anbindung an die IO-Module
der SPS dar. Der Schwerpunkt liegt hierbei auf der Funktionsweise des piControl-Treibers und dessen Integration in das Projekt. Abschnitt~\ref{sec:4-picontrol} erklärt die zum Schreibens eines Bits verwendeten Funktionsaufrufe.
Zuvor soll jedoch in Abschnitt~\ref{sec:4-open62541} der Teil des OPC-Servers vorgestellt werden, welcher auf besagten Treiber zugreift. 

\subsection{Implementierung des OPC-Servers%
     \label{sec:4-open62541}}
Wie im vorangegangenen Abschnitt~\ref{sec:3-integration} begründet, soll die Verknüpfung zwischen dem Prozessabbild der SPS und den auf dem OPC-Server bereitgestellten Werten über sog.\,Datenquellen erfolgen. Hierzu ist zunächst eine Callback-Methode zu implementieren, welche bei einem Lese- oder Schreibzugriff auf eine Variable aufgerufen wird. Die Verknüpfung zwischen Callback-Methode und Variable muss manuell erfolgen.

\begin{lstlisting}[language={c},firstnumber=237,caption={Auszug der Methode \lstinline{linkDataSourceVariable} in \lstinline{variables.c}\label{lst:4-linkDataSourceVariable}}]
extern UA_StatusCode
 linkDataSourceVariable(UA_Server *server, UA_NodeId nodeId) {
     bool readonly = false;
     UA_DataSource dataSourceVariable;
     UA_StatusCode rc; |>\setcounter{lstnumber}{254}<|

     dataSourceVariable.read = readDataSourceVariable;
     if (!readonly)
        dataSourceVariable.write = writeDataSourceVariable;
     else
        dataSourceVariable.write = writeReadonlyDataSourceVariable;

     return UA_Server_setVariableNode_dataSource(server, nodeId, dataSourceVariable);
 }
\end{lstlisting}

\begin{figure}[h]
    \centering
    \includegraphics[width=0.42\textwidth]{doc/img/OPC_RevPiDO.pdf}
    \caption{Auszug des verwendeten Nodesets, hier Digitalausgang 1 des Versuchsaufbaus
      \label{fig:opc-do}}
\end{figure}

Die in Listing~\ref{lst:4-linkDataSourceVariable} abgebildete Methode \lstinline{linkDataSourceVariable()} erzeugt ein Struct vom Typ \lstinline{UA_DataSource}. In diesem werden dem Lesen und Schreiben einer OPC-Variablen entsprechende Callback-Methoden zugewiesen. Die Verknüpfung einer OPC-Variable, genauer ihrer NodeId, mit der zuvor definierten Datenquelle erfolgt über die von open62541 bereitgestellte Methode \lstinline{UA_Server_setVariableNode_dataSource()}. Vor dem Lesen und nach dem Schreiben dieser Variable werden von nun an die entsprechenden Callbacks aufgerufen.
     
\begin{lstlisting}[language={c},firstnumber=168,caption={Auszug des Callbacks \lstinline{writeDataSourceVariable} in \lstinline{variables.c}\label{lst:4-writeDataSourceVariable}}]  
extern UA_StatusCode
 writeDataSourceVariable(UA_Server *server,
            const UA_NodeId *sessionId, void *sessionContext,
            const UA_NodeId *nodeId, void *nodeContext,
            const UA_NumericRange *range, const UA_DataValue *dataValue) {

    UA_StatusCode retval  = UA_STATUSCODE_GOOD;
    UA_NodeId *nameNodeId = UA_malloc(sizeof(UA_NodeId));
    UA_QualifiedName nameQN = UA_QUALIFIEDNAME(1, "Name");
    UA_Variant nameVar;
    UA_Boolean bit;

    retval |= findSiblingByBrowsename(server, nodeId, &nameQN, nameNodeId);
    retval |= UA_Server_readValue(server, *nameNodeId, &nameVar);
    retval |= UA_Boolean_copy(dataValue->value.data, &bit);

    |>\tikzmarkin[set border color=martinired]{writeIO}<|PI_writeSingleIO(String_fromUA_String(nameVar.data), &bit, false);                                                 |>\tikzmarkend{writeIO}<|

    free(nameNodeId);
    return retval;
 }
\end{lstlisting}

Listing~\ref{lst:4-writeDataSourceVariable} zeigt die Callback-Methode, welche nach dem Schreiben einer Variablen auf dem OPC-Server aufgerufen wird.
Dieser Methode wird neben der NodeId der mit ihr verknüpften Variablen auch der Wert dieser in Form eines Zeigers auf ein Struct vom Typ \lstinline{UA_DataValue} übergeben.

Die Gestaltung des hier verwendeten Nodesets sieht vor, dass in einer OPC-Variablen \lstinline{"Name"} der Bezeichner des zu schreibenden Digitalausgangs hinterlegt ist, siehe Abbildung~\ref{fig:opc-do}. Dies erlaubt eine Rekonfiguration der Ein- und Ausgänge der SPS ohne Änderungen im Programmcode des OPC-Servers vornehmen zu müssen.
Es ist daher erforderlich, nach jedem Schreiben einer mit einem Digitalausgang verknüpften Variablen, hier \lstinline{"Value"}, dessen Bezeichner \lstinline{"Name"} abzufragen. 
Dies geschieht in den Zeilen 180 und 181.
Anschließend wird dieser Bezeichner sowie der zu schreibende Wert der Methode \lstinline{PI_writeSingleIO()} übergeben, welche wiederum die Interaktion mit piControl übernimmt (vgl. Abschnitt \ref{sec:4-picontrol}).
 
\subsection{Integration von piControl%
     \label{sec:4-picontrol}}
In Abschnitt~\ref{sec:2-io} wurde die Anbindung der IO-Module des Revolution Pi sowie die Funktionsweise von piControl aus Anwendersicht beschrieben. Die verfügbare Literatur beschränkt sich auch auf lediglich diese Sicht; eine weiterführende Dokumentation für Entwickler gibt es, neben der in Abschnitt~\ref{sec:3-anbindung} vorgestellten Manpage, nicht. 
In diesem Abschnitt soll daher der Quellcode von piControl sowie dessen Verwendung im Projekt genauer betrachtet werden.
Hierzu wird exemplarisch die in Abschnitt~\ref{sec:4-open62541} eingeführte Methode \lstinline{PI_writeSingleIO()} untersucht.
Diese Methode ermöglicht das Setzen eines einzelnen Bits im Prozessabbild der SPS, und damit das Schalten eines digitalen Ausgangs auf einem IO-Modul.
Die äquivalente Methode \lstinline{int piControlGetBitValue(SPIValue *pSpiValue)} zum Lesen eines Bits bzw. Eingangs funktioniert analog und soll daher an dieser Stelle nicht dediziert erörtert werden.

\begin{lstlisting}[language={c},firstnumber=97,
                   caption={Setzen eines phsikalischen, digitalen Ausgangs in \lstinline{revpi.c}
                   \label{lst:4-PI_writeSingleIO}}]
extern void PI_writeSingleIO(char *pszVariableName, bool *bit, bool verbose)
{
	int rc;
	SPIVariable sPiVariable;
	SPIValue sPIValue;

	strncpy(sPiVariable.strVarName, pszVariableName, sizeof(sPiVariable.strVarName));
	rc = piControlGetVariableInfo(&sPiVariable);
	if (rc < 0) {
		printf("Cannot find variable '%s'\n", pszVariableName);
		return;
	}

		sPIValue.i16uAddress = sPiVariable.i16uAddress;
		sPIValue.i8uBit = sPiVariable.i8uBit;
		sPIValue.i8uValue = *bit;
		rc = |>\tikzmarkin[set border color=martinired]{setBitValue}<|piControlSetBitValue(&sPIValue)|>\tikzmarkend{setBitValue}<|;
		if (rc < 0)
			printf("Set bit error %s\n", getWriteError(rc));
		else if (verbose)
			printf("Set bit %d on byte at offset %d. Value %d\n", sPIValue.i8uBit, sPIValue.i16uAddress,
			       sPIValue.i8uValue);
}
\end{lstlisting}

Der Programmcode in Listing~\ref{lst:4-PI_writeSingleIO} ist Teil des implementierten OPC-Servers. In diesem wird auf zwei Funktionen des piControl-Treibers zugegriffen. 
Beiden Methoden wird als Argument ein Zeiger auf ein Struct vom Typ \lstinline{SPIValue} übergeben. Der im Struct abgelegte Name wird mittels \lstinline{piControlGetVariableInfo(&sPIValue)} zu einer Adresse im Prozessabbild aufgelöst. Diese wird in \lstinline{sPIValue.i16uAdress} gespeichert. Der Wert der Variablen wird anschließend mittels \lstinline{piControlSetBitValue(&sPIValue)} an dieser Adresse in das Prozessabbild geschrieben.

\begin{lstlisting}[language={c},firstnumber=309,caption={Methode \lstinline{piControlSetBitValue} in \lstinline{piControlIf.c}\label{lst:4-piControlSetBitValue}}]
int |>\tikzmarkin[set border color=martiniblue]{setBitValueFcn}<|piControlSetBitValue(SPIValue *pSpiValue)|>\tikzmarkend{setBitValueFcn}<|
{
    piControlOpen();

    if (PiControlHandle_g < 0)
	    return -ENODEV;

    pSpiValue->i16uAddress += pSpiValue->i8uBit / 8;
    pSpiValue->i8uBit %= 8;

    if (|>\tikzmarkin[set border color=martinired]{ioctl}<|ioctl(PiControlHandle_g, KB_SET_VALUE, pSpiValue)|>\tikzmarkend{ioctl}<| < 0)
	    return errno;

    return 0;
}
\end{lstlisting}

Die in Listing~\ref{lst:4-piControlSetBitValue} dargestellte Methode \lstinline{piControlSetBitValue} ist lediglich eine Hüllfunktion (häufig auch als Wrapper-Funktion bezeichnet) für einen Aufruf des \lstinline{ioctl} Kernel-Moduls.
Folgende Parameter werden übergeben:
\lstinline{PiControlHandle_g} ist die Referenz auf die Geräte-Datei des piControl-Treibers. \lstinline{KB_SET_VALUE} ist das ioctl-Kommando zum Schreiben eines Bits in das Prozessabbild. Der Zeiger \lstinline{pSpiValue} verweist auf ein Struct des bereits vorgestellten Typs \lstinline{SPIValue}.

\begin{lstlisting}[language={c},firstnumber=80,caption={Methode \lstinline{piControlOpen} in \lstinline{piControlIf.c}\label{lst:4-piControlOpen}}]
void piControlOpen(void)
{
    /* open handle if needed */
    if (PiControlHandle_g < 0)
    {
	    |>\tikzmarkin[set border color=martiniblue]{PiControlHandle}<|PiControlHandle_g = open(PICONTROL_DEVICE, O_RDWR)|>\tikzmarkend{PiControlHandle}<|;
    }
}
\end{lstlisting}

Die in Listing~\ref{lst:4-piControlOpen} dargestellte Methode öffnet, sofern nicht bereits geschehen, die Geräte-Datei. Das Macro \lstinline{PICONTROL_DEVICE} verweist hierbei auf \lstinline{/dev/piControl0}.

\begin{lstlisting}[language={c},firstnumber=721,caption={Methode \lstinline{piControlIoctl} in \lstinline{piControlMain.c}\label{lst:4-piControlIoctl}}]
static long |>\tikzmarkin[set border color=martiniblue, below offset=0.9em]{piControlIoctl}<|piControlIoctl(struct file *file, unsigned int prg_nr, 
                           unsigned long usr_addr)                                      |>\tikzmarkend{piControlIoctl}<|
{
  int status = -EFAULT;
  tpiControlInst *priv;
  int timeout = 10000;	// ms

  if (prg_nr != KB_CONFIG_SEND && prg_nr != KB_CONFIG_START && !isRunning()) {
  	return -EAGAIN;
  }

  priv = (tpiControlInst *) file->private_data;

  if (prg_nr != KB_GET_LAST_MESSAGE) {
  	// clear old message
  	priv->pcErrorMessage[0] = 0;
  }

  switch (prg_nr) {|>\setcounter{lstnumber}{864}<|

    case |>\tikzmarkin[set border color=martiniblue]{KB_SET_VALUE}<|KB_SET_VALUE:|>\tikzmarkend{KB_SET_VALUE}<|
  		{
  			SPIValue *pValue = (SPIValue *) usr_addr;

  			if (!isRunning())
  				return -EFAULT;

  			if (pValue->i16uAddress >= KB_PI_LEN) {
  				status = -EFAULT;
  			} else {
  				INT8U i8uValue_l;
  				my_rt_mutex_lock(&piDev_g.lockPI);
  				i8uValue_l = piDev_g.ai8uPI[pValue->i16uAddress];

  				if (pValue->i8uBit >= 8) {
  					i8uValue_l = pValue->i8uValue;
  				} else {
  					if (pValue->i8uValue)
  						i8uValue_l |= (1 << pValue->i8uBit);
  					else
  						i8uValue_l &= ~(1 << pValue->i8uBit);
  				}

  				|>\tikzmarkin[set border color=martinired]{i8uValue}<|piDev_g.ai8uPI[pValue->i16uAddress] = i8uValue_l;|>\tikzmarkend{i8uValue}<|
  				rt_mutex_unlock(&piDev_g.lockPI);

  #ifdef VERBOSE
  				pr_info("piControlIoctl Addr=%u, bit=%u: %02x %02x\n", pValue->i16uAddress, pValue->i8uBit, pValue->i8uValue, i8uValue_l);
  #endif

  				status = 0;
  			}
  		}
  		break; |>\setcounter{lstnumber}{1314}<|

    default:
      pr_err("Invalid Ioctl");
      return (-EINVAL);
      break;

    }

    return status;
  }
\end{lstlisting}

Listing~\ref{lst:4-piControlIoctl} zeigt in Auszügen die ioctl-Methode des piControl Kernel-Treibers. Diese bekommt folgende Argumente übergeben: \lstinline{struct file *file} enthält den Verweis auf die Geräte-Datei, hier \lstinline{/dev/piControl0}. Der Wert von \lstinline{unsigned int prg_nr} beschreibt die Anfrage an den Treiber, in diesem Fall \lstinline{KB_SET_VALUE}. Das Argument \lstinline{unsigned long usr_addr} enthält einen typ-agnostischen Pointer. Dieser verweist auf einen Speicherbereich, in welchem die zur Bearbeitung der Anfrage notwendigen Daten abgelegt sind. Hier können auch vom Treiber empfangene Daten dem Anwendungsprogramm bereitgestellt werden. 

Die switch-case-Anweisung führt die über das Argument \lstinline{prg_nr} spezifizierte Aktion aus. Hier betrachten wir \lstinline{KB_SET_VALUE}:
Zunächst wird in Zeile 868 der übergebene Zeiger \lstinline{usr_addr} mittels explizitem Typecast zu einem Zeiger des Typs \lstinline{SPIValue *} konvertiert. Da dieser auf Daten im Userspace verweist, ist beim Zugriff durch den Kernel-Treiber besondere Vorsicht geboten.
In Zeile 877 wird mittels Mutex das Prozessabbild \lstinline{piDev_g} für den Zugriff durch andere Threads oder Prozesse gesperrt.
\lstinline{my_rt_mutex_lock} verweist hierbei auf die Funktion \lstinline{rt_mutex_lock} aus \lstinline{linux/sched.h}\footnote{Offenbar wurde hier auch eine alternative Implementierung vorgesehen, siehe revpi\_common.h}

In Zeile 889 wird das Byte \lstinline{i8uValue_l}, welches den zu schreibenden Wert enthält in das Prozessabbild übertragen. Anschließend wird die Mutex auf \lstinline{piDev_g} wieder entsperrt.
\newpage

\begin{lstlisting}[language={c},firstnumber=62,caption={Auszug des Struct \lstinline{spiControlDev} in \lstinline{piControlMain.h}\label{lst:4-spiControlDev}}]
|>\tikzmarkin[set border color=martiniblue]{spiControlDev}<|typedef struct spiControlDev|>\tikzmarkend{spiControlDev}<| {
	// device driver stuff
	int init_step;
	enum revpi_machine machine_type;
	void *machine;
	struct cdev cdev;	// Char device structure
	struct device *dev;
	struct thermal_zone_device *thermal_zone;

	|>\tikzmarkin[set border color=martiniblue]{processImage}<|// process image stuff
	INT8U ai8uPI[KB_PI_LEN];
	INT8U ai8uPIDefault|>\tikzmarkin[set border color=martinired]{KB_PI_LEN_0}<|[KB_PI_LEN]|>\tikzmarkend{KB_PI_LEN_0}<|;
	struct rt_mutex lockPI;        |>\tikzmarkend{processImage}<|
	bool stopIO;
	piDevices *devs; |>\setcounter{lstnumber}{94}<|
} tpiControlDev;
\end{lstlisting}

Das Prozessabbild ist als Byte-Array der Länge \lstinline{KB_PI_LEN} in Listing~\ref{lst:4-spiControlDev} definiert. Konfigurationsparameter wie \lstinline{KB_PI_LEN} oder die Zykluszeit für den Datenaustausch zwischen SPS und IO-Modulen sind im folgenden Listing~\ref{lst:4-process} definiert.

\begin{lstlisting}[language={c},firstnumber=119,caption={Konfigurationsparameter des Prozessabbildes in project.h\label{lst:4-process}}]
#define INTERVAL_PI_GATE (5*1000*1000)  // 5 ms piGateCommunication |>\setcounter{lstnumber}{128}<|

#define INTERVAL_IO_COM (5*1000*1000)  // 5 ms piIoComm |>\setcounter{lstnumber}{132}<|

#define KB_PD_LEN       512
|>\tikzmarkin[set border color=martiniblue]{KB_PI_LEN_1}<|#define KB_PI_LEN       4096|>\tikzmarkend{KB_PI_LEN_1}<|
\end{lstlisting}

Das zu setzende Bit wurde zu diesem Zeitpunkt erfolgreich in das Prozessabbild der SPS geschrieben.
Es stellt sich die Frage, wie dieses nun an das IO-Modul kommuniziert wird.
Die Kommunikation mit allen angebundenen Modulen ist ebenfalls Aufgabe des piControl-Treibers.

\begin{lstlisting}[language={c},firstnumber=256,caption={Auszug der Methode \lstinline{piIoThread} in \lstinline{revpi_core.c}\label{lst:4-piIoThread}}]
static int piIoThread(void *data)
{
	//TODO int value = 0;
	ktime_t time;
	ktime_t now;
	s64 tDiff;

	hrtimer_init(&piCore_g.ioTimer, CLOCK_MONOTONIC, HRTIMER_MODE_ABS);
	piCore_g.ioTimer.function = piIoTimer;

	pr_info("piIO thread started\n");

	now = hrtimer_cb_get_time(&piCore_g.ioTimer);

	PiBridgeMaster_Reset();

	while (!kthread_should_stop()) {
		if (|>\tikzmarkin[set border color=martinired]{PiBridgeMaster}<|PiBridgeMaster_Run()|>\tikzmarkend{PiBridgeMaster}<| < 0)
			break;
	}

	RevPiDevice_finish();

	pr_info("piIO exit\n");
	return 0;
}
\end{lstlisting}

Der Kernel-Thread \lstinline{piIoThread} ist verantwortlich für den zyklischen Datenaustausch mit den IO-Modulen. In diesem wird fortlaufend die Methode \lstinline{PiBridgeMaster_Run()} aufgerufen, siehe Listing~\ref{lst:4-piIoThread}.

\begin{lstlisting}[language={c},firstnumber=262,caption={Auszug der Methode \lstinline{PiBridgeMaster_Run(void)} in \lstinline{RevPiDevice.c}\label{lst:4-PiBridgeMaster_Run}}]
int PiBridgeMaster_Run(void)
{
	static kbUT_Timer tTimeoutTimer_s;
	static kbUT_Timer tConfigTimeoutTimer_s;
	static int error_cnt;
	static INT8U last_led;
	static unsigned long last_update;
	int ret = 0;
	int i;

	my_rt_mutex_lock(&piCore_g.lockBridgeState);
	if (piCore_g.eBridgeState != piBridgeStop) {
		switch (eRunStatus_s) { |>\setcounter{lstnumber}{514}<|
		    case enPiBridgeMasterStatus_EndOfConfig:|>\setcounter{lstnumber}{621}<|
		    if (|>\tikzmarkin[set border color=martinired]{RevPiDevice}<|RevPiDevice_run()|>\tikzmarkend{RevPiDevice}<|) {
				// an error occured, check error limits |>\setcounter{lstnumber}{641}<|
			} else {
				ret = 1;
			}
			piCore_g.image.drv.i16uRS485ErrorCnt = RevPiDevice_getErrCnt();
			break;
\end{lstlisting}

Die in Listing~\ref{lst:4-PiBridgeMaster_Run} dargestellte Methode ist eine sog. State-Machine. Ist die Konfiguration der IO-Module erfolgreich abgeschlossen, so führt sie bei Aufruf lediglich die Methode \lstinline{RevPiDevice_run()} aus.

\begin{lstlisting}[language={c},firstnumber=140,caption={Auszug der Methode \lstinline{RevPiDevice_run(void)} in \lstinline{RevPiDevice.c}\label{lst:4-RevPiDevice_run}}]
int RevPiDevice_run(void)
{
	INT8U i8uDevice = 0;
	INT32U r;
	int retval = 0;

	RevPiDevices_s.i16uErrorCnt = 0;

	for (i8uDevice = 0; i8uDevice < RevPiDevice_getDevCnt(); i8uDevice++) {
		if (RevPiDevice_getDev(i8uDevice)->i8uActive) {
			switch (RevPiDevice_getDev(i8uDevice)->sId.i16uModulType) {
			case KUNBUS_FW_DESCR_TYP_PI_DIO_14:
			case KUNBUS_FW_DESCR_TYP_PI_DI_16:
			case KUNBUS_FW_DESCR_TYP_PI_DO_16:
				r = |>\tikzmarkin[set border color=martinired]{sendCyclicTelegram}<|piDIOComm_sendCyclicTelegram(i8uDevice)|>\tikzmarkend{sendCyclicTelegram}\setcounter{lstnumber}{166} <|;

				break; |>\setcounter{lstnumber}{216}<|
			}
		}
	} |>\setcounter{lstnumber}{227}<|
	return retval;
}
\end{lstlisting}

Diese iteriert wie in Listing~\ref{lst:4-RevPiDevice_run} abgebildete durch alle gegenwärtig in der SPS konfigurierten Module. Ist das aktuelle Modul als aktiv markiert, so wird anhand eines sog. Firmware-Descriptors entschieden, welche Methode für die Ansteuerung des Moduls aufzurufen ist.

\begin{lstlisting}[language={c},firstnumber=161,caption={Auszug der Methode \lstinline{piDIOComm_sendCyclicTelegram} in \lstinline{piDIOComm.c}\label{lst:4-sendCyclicTelegram}}]
INT32U piDIOComm_sendCyclicTelegram(INT8U i8uDevice_p)
{
	INT32U i32uRv_l = 0;
	SIOGeneric sRequest_l;
	SIOGeneric sResponse_l;
	INT8U len_l, data_out[18], i, p, data_in[70];
	INT8U i8uAddress;
	int ret; |>\setcounter{lstnumber}{239}<|
	
    |>\tikzmarkin[set border color=martinired]{piIoComm}<|ret = piIoComm_send((INT8U *) & sRequest_l, IOPROTOCOL_HEADER_LENGTH + len_l + 1);  |>\tikzmarkend{piIoComm}\setcounter{lstnumber}{298}<|
}
\end{lstlisting}

Im Falle des hier verwendeten DO-Moduls wird die in Listing~\ref{lst:4-sendCyclicTelegram} abgebildete Methode \lstinline{piDIOComm_sendCyclicTelegram()} aufgerufen. Dieser wird ein Zeiger auf das zu schreibende Gerät übergeben. 
Zunächst wird das Prozessabbild mittels eines proprietären, jedoch im Quellcode offen nachvollziehbaren Protokolls in ein \lstinline{sRequest_l} genanntes Byte-Array umgewandelt. Dieser Schritt ist in Listing~\ref{lst:4-sendCyclicTelegram} nicht abgebildet. Anschließend wird \lstinline{piIoComm_send()} ein Zeiger auf die so generierte Schreib-Anfrage übergeben.

\begin{lstlisting}[language={c},firstnumber=220,caption={Auszug der Methode \lstinline{piIOComm_send} in \lstinline{piIOComm.c}\label{lst:4-piIOComm_send}}]
int piIoComm_send(INT8U * buf_p, INT16U i16uLen_p)
{
	ssize_t write_l = 0;
	INT16U i16uSent_l = 0;|>\setcounter{lstnumber}{249}<|

	while (i16uSent_l < i16uLen_p) {
		write_l = vfs_write(piIoComm_fd_m, buf_p + i16uSent_l, i16uLen_p - i16uSent_l, &piIoComm_fd_m->f_pos);
		if (write_l < 0) {
			pr_info_serial("write error %d\n", (int)write_l);
			return -1;
		} 
		i16uSent_l += write_l;|>\setcounter{lstnumber}{263}<|
	}
	clear();
	vfs_fsync(piIoComm_fd_m, 1);
	return 0;
}
\end{lstlisting}

Listing~\ref{lst:4-piIOComm_send} zeigt die Implementierung von \lstinline{piIoComm_send()}. Diese Methode ist für das Schreiben der oben generierten Anfrage auf die seriellen Schnittstelle verantwortlich. Realisiert wird dies mittels der Methode \lstinline{vfs_write()}. Diese ist in \lstinline{<linux/fs.h>} definiert. Sie ermöglicht das Schreiben einer Datei im Userspace aus dem Kernel heraus. Geschrieben wird hier die Datei mit dem Deskriptor \lstinline{piIoComm_fd_m}.
Da die Funktion \lstinline{vfs_write()} durch andere Kernel-Tasks unterbrochen werden kann, ist nicht gewährleistet, dass die gesamte Anfrage mit nur einem Aufruf geschrieben wird. Die oben abgebildete while-Schleife stellt das vollständige Senden der Anfrage sicher.

\begin{lstlisting}[language={c},firstnumber=157,caption={Auszug der Methode \lstinline{piIOComm_open_serial} in \lstinline{piIOComm.c}\label{lst:4-piIOComm_open_serial}}]
int piIoComm_open_serial(void)
{   |>\setcounter{lstnumber}{167}<|
	struct file *fd;	/* Filedeskriptor */
	struct termios newtio;	/* Schnittstellenoptionen */

	|>\tikzmarkin[set border color=martiniblue]{fd}<|/* Port oeffnen - read/write, kein "controlling tty", 
	    Status von DCD ignorieren */
	fd = filp_open(|>\tikzmarkin[set border color=martinired]{tty}<|REV_PI_TTY_DEVICE|>\tikzmarkend{tty}<|, O_RDWR | O_NOCTTY, 0); |>\setcounter{lstnumber}{208}<|
	
	piIoComm_fd_m = fd;                                                      |>\tikzmarkend{fd}\setcounter{lstnumber}{217}<|

	return 0;
}
\end{lstlisting}

Der zum Schreiben auf die serielle Schnittstelle verwendete Datei-Deskriptor wird von der in Listing~\ref{lst:4-piIOComm_open_serial} abgebildeten Methode \lstinline{piIoComm_open_serial()} generiert. 

\begin{lstlisting}[language={c},firstnumber=45,caption={Definition der seriellen Schnittstelle in \lstinline{piIOComm.h}\label{lst:4-REV_PI_TTY_DEVICE}}]
#define REV_PI_TTY_DEVICE	"/dev/ttyAMA0"
\end{lstlisting}

Das in Listing~\ref{lst:4-REV_PI_TTY_DEVICE} definierte Macro verweist auf eine der seriellen Schnittstellen des RaspberryPi.
Die Implementierung des zugehörigen Schnittstellentreibers soll hier nicht weiter untersucht werden. Somit ist an dieser Stelle die Kette vom Setzen einer Variablen auf dem OPC-Server bis hin zur Aktualisierung des Prozessabbilds der IO-Module geschlossen.

% \begin{lstlisting}[language={c},firstnumber={226},caption={Setzen der Scheduler-Priorität auf SCHED\_FIFO in 
% revpi\_common.c\label{lst:2-sched_priority}}]
% param.sched_priority = ktprio->prio;
% ret = sched_setscheduler(child, SCHED_FIFO, &param);
% \end{lstlisting}
% % % Imports nur für Referenzenauflösung während des Schreibens! Vorm Kompilieren auskommentieren!
% \bibliography{0_hauptdatei}
% \input{1_einleitung}
% \input{2_grundlagen}
% \input{3_konzeption}
% \input{4_implementierung}
% \input{5_tests}
% \input{6_zusammenfassung}
% % Ende Imports

\section{Test des OPC-Servers im Gesamtsystem%
  \label{sec:5-tests}}

% % % Imports nur für Referenzenauflösung während des schreibens! Vorm Kompilieren auskommentieren!
% \bibliography{0_hauptdatei}
% \input{1_einleitung}
% \input{2_grundlagen}
% \input{3_konzeption}
% \input{4_implementierung}
% \input{5_tests}
% \input{6_zusammenfassung}
% % Ende Imports

\section{Zusammenfassung und Ausblick%
  \label{sec:6-fazit}}
Der folgende Abschnitt~\ref{sec:6-zusammenfassung} fasst die gewonnenen Erkenntnisse und den Stand der Implementierung zusammen.
Den Abschluss dieser Arbeit bildet der Ausblick in Abschnitt~\ref{sec:6-ausblick}.

\subsection{Zusammenfassung%
     \label{sec:6-zusammenfassung}}

\subsection{Ausblick%
     \label{sec:6-ausblick}}

% % Ende Imports

\section{Zusammenfassung und Ausblick%
  \label{sec:6-fazit}}
Der folgende Abschnitt~\ref{sec:6-zusammenfassung} fasst die gewonnenen Erkenntnisse und den Stand der Implementierung zusammen.
Den Abschluss dieser Arbeit bildet der Ausblick in Abschnitt~\ref{sec:6-ausblick}.

\subsection{Zusammenfassung%
     \label{sec:6-zusammenfassung}}

\subsection{Ausblick%
     \label{sec:6-ausblick}}

% % Ende Imports

\section{Zusammenfassung und Ausblick%
  \label{sec:6-fazit}}
Der folgende Abschnitt~\ref{sec:6-zusammenfassung} fasst die gewonnenen Erkenntnisse und den Stand der Implementierung zusammen.
Den Abschluss dieser Arbeit bildet der Ausblick in Abschnitt~\ref{sec:6-ausblick}.

\subsection{Zusammenfassung%
     \label{sec:6-zusammenfassung}}

\subsection{Ausblick%
     \label{sec:6-ausblick}}

% \input{anhang}
% % Ende Imports

\section{Implementierung%
  \label{sec:4-implementierung}}

revpi.c - 58

\begin{lstlisting}[language={c},firstnumber=97,caption={Setzen eines phsikalischen, digitalen Ausgangs\label{lst:4-PI_writeSingleIO}}]
extern void PI_writeSingleIO(char *pszVariableName, bool *bit, bool verbose)
{
	int rc;
	SPIVariable sPiVariable;
	SPIValue sPIValue;

	strncpy(sPiVariable.strVarName, pszVariableName, sizeof(sPiVariable.strVarName));
	rc = piControlGetVariableInfo(&sPiVariable);
	if (rc < 0) {
		printf("Cannot find variable '%s'\n", pszVariableName);
		return;
	}

		sPIValue.i16uAddress = sPiVariable.i16uAddress;
		sPIValue.i8uBit = sPiVariable.i8uBit;
		sPIValue.i8uValue = *bit;
		rc = piControlSetBitValue(&sPIValue);
		if (rc < 0)
			printf("Set bit error %s\n", getWriteError(rc));
		else if (verbose)
			printf("Set bit %d on byte at offset %d. Value %d\n", sPIValue.i8uBit, sPIValue.i16uAddress,
			       sPIValue.i8uValue);
}
\end{lstlisting}

\begin{lstlisting}[language={c},firstnumber=301,caption={Methode piControlSetBitValue in piControlIf.c\label{lst:4-piControlSetBitValue(}}]
/***********************************************************************************/
/*!
 * @brief Set Bit Value
 *
 * Set the value of one bit in the process image.
 *
 * @param[in/out]   Pointer to SPIValue.
 *
 * @return 0 or error if negative
 *
 ************************************************************************************/
int piControlSetBitValue(SPIValue *pSpiValue)
{
    piControlOpen();

    if (PiControlHandle_g < 0)
	return -ENODEV;

    pSpiValue->i16uAddress += pSpiValue->i8uBit / 8;
    pSpiValue->i8uBit %= 8;

    if (ioctl(PiControlHandle_g, KB_SET_VALUE, pSpiValue) < 0)
	return errno;

    return 0;
}
\end{lstlisting}
\begin{lstlisting}[numbers=none,caption={Auszug aus der RevolutionPi Programmers Manual\label{lst:4-manpage}}]
KB_SET_VALUE SPIValue *argp
Write one bit or one byte to the process image. Before the call the elements i16uAddress and
i8uBit must be set to the address of the value. i8uValue must set to the value to write. This call is
more efficient than the usual calls of seek and write because only one function call is necessary. If
more than on application are writing bits in one output byte, this call is the only safe way to set a
bit without overwriting the other bits because this call is doning a read-modify-write-cycle.
The struct SPIValue used by this ioctl is defined as
typedef struct SPIValueStr
{
uint16_t i16uAddress; // Address of the byte in the process image
uint8_t i8uBit; // 0-7 bit position, >= 8 whole byte
uint8_t i8uValue; // Value: 0/1 for bit access, whole byte otherwise
} SPIValue;
\end{lstlisting}

\begin{lstlisting}[language={c},firstnumber=74,caption={Methode piControlOpen in piControlIf.c\label{lst:4-piControlOpen}}]
/***********************************************************************************/
/*!
 * @brief Open Pi Control Interface
 *
 * Initialize the Pi Control Interface
 *
 ************************************************************************************/
void piControlOpen(void)
{
    /* open handle if needed */
    if (PiControlHandle_g < 0)
    {
	PiControlHandle_g = open(PICONTROL_DEVICE, O_RDWR);
    }
}
\end{lstlisting}

\begin{lstlisting}[language={c},firstnumber=718,caption={Methode piControlIoctl in piControlMain.c\label{lst:4-piControlIoctl}}]

  /*****************************************************************************/
  /*    I O C T L                                                           */
  /*****************************************************************************/
  static long piControlIoctl(struct file *file, unsigned int prg_nr, unsigned long usr_addr) // <-
  {
  	int status = -EFAULT;
  	tpiControlInst *priv;
  	int timeout = 10000;	// ms

  	if (prg_nr != KB_CONFIG_SEND && prg_nr != KB_CONFIG_START && !isRunning()) {
  		return -EAGAIN;
  	}

  	priv = (tpiControlInst *) file->private_data;

  	if (prg_nr != KB_GET_LAST_MESSAGE) {
  		// clear old message
  		priv->pcErrorMessage[0] = 0;
  	}

  	switch (prg_nr) {

    [...]

    piControlMain.c - 865
    case KB_SET_VALUE:	// <-
  		{
  			SPIValue *pValue = (SPIValue *) usr_addr;

  			if (!isRunning())
  				return -EFAULT;

  			if (pValue->i16uAddress >= KB_PI_LEN) {
  				status = -EFAULT;
  			} else {
  				INT8U i8uValue_l;
  				my_rt_mutex_lock(&piDev_g.lockPI);
  				i8uValue_l = piDev_g.ai8uPI[pValue->i16uAddress];

  				if (pValue->i8uBit >= 8) {
  					i8uValue_l = pValue->i8uValue;
  				} else {
  					if (pValue->i8uValue)
  						i8uValue_l |= (1 << pValue->i8uBit);
  					else
  						i8uValue_l &= ~(1 << pValue->i8uBit);
  				}

  				piDev_g.ai8uPI[pValue->i16uAddress] = i8uValue_l;  // <-
  				rt_mutex_unlock(&piDev_g.lockPI);

  #ifdef VERBOSE
  				pr_info("piControlIoctl Addr=%u, bit=%u: %02x %02x\n", pValue->i16uAddress, pValue->i8uBit, pValue->i8uValue, i8uValue_l);
  #endif

  				status = 0;
  			}
  		}
  		break;

    [...]

    default:
      pr_err("Invalid Ioctl");
      return (-EINVAL);
      break;

    }

    return status;
  }
\end{lstlisting}

\begin{lstlisting}[language={c},firstnumber=62,caption={Definition des Struct spiControlDev in piControlMain.h\label{lst:4-spiControlDev}}]
typedef struct spiControlDev {
	// device driver stuff
	int init_step;
	enum revpi_machine machine_type;
	void *machine;
	struct cdev cdev;	// Char device structure   // <->  #include <linux/cdev.h>
	struct device *dev;                          // <->  #include <linux/device.h>
	struct thermal_zone_device *thermal_zone;

	// process image stuff
	INT8U ai8uPI[KB_PI_LEN];
	INT8U ai8uPIDefault[KB_PI_LEN];
	struct rt_mutex lockPI;
	bool stopIO;
	piDevices *devs;
	piEntries *ent;
	piCopylist *cl;
	piConnectionList *connl;
	ktime_t tLastOutput1, tLastOutput2;

	// handle open connections and notification
	u8 PnAppCon;		// counter of open connections
	struct list_head listCon;
	struct rt_mutex lockListCon;

	struct led_trigger power_red;
	struct led_trigger a1_green;
	struct led_trigger a1_red;
	struct led_trigger a2_green;
	struct led_trigger a2_red;
	struct led_trigger a3_green;
	struct led_trigger a3_red;
} tpiControlDev;
\end{lstlisting}
