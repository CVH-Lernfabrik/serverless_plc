\thispagestyle{empty}

\begin{minipage}{0.55\textwidth}
\institutname{}
%  Institut für Mechatronik\\
%  und Fahrzeugtechnik
\end{minipage}\hfill
%\includegraphics[height=2cm]{bilder/bologo} % Pixelgrafik-Version
\begin{picture}(0,0)(0,1.37) % Vektorgrafik-Version
  \put(-2.7,0.0){\includegraphics[scale=1.06]{layout/logo-hochschule-bochum-bo.pdf}}
  \put(-6.2,0.87){\includegraphics[scale=0.545]{layout/logo-hochschule-bochum-text.pdf}}
  \put(-6.2,-0.07){\includegraphics[scale=0.635]{layout/logo-hochschule-bochum-cvh-text.pdf}}
\end{picture}%


% freier Platz: 1/3 oben
\vfill

\begin{center}
  % Die äußeren geschweiften Klammern begrenzen die Wirkung von \Huge.
  % Das \par am Ende ist notwendig, damit zum einen der Zeilenabstand
  % des \Huge-Textes angepaßt wird; zum anderen arbeitet \bigskip nicht
  % innerhalb von Textabsätzen, sondern außerhalb.
  {\huge\textbf{\thetitle}\par}
  \bigskip\bigskip
  ~von\par
  \bigskip\bigskip
  \begin{tabular}{ c c c }
    {\Large\textbf{\firstAuthor}} & und & {\Large\textbf{\secondAuthor}} \\
    Matrikelnummer:~\firstStudentnumber & & Matrikelnummer:~\secondStudentnumber \\
  \end{tabular}\par
  \bigskip\bigskip\bigskip
  \dokumentname{}\\
  %\smallskip
  %\companyname\\
  %\smallskip
  \discipline\\
% zur Erlangung des akademischen Grades\\
% \abschluss{}
\end{center}

% freier Platz: 2/3 unten
\vfill

\begin{tabular}{ll}
  \textbf{Eingereicht am:} & \printdate{\finaldate}\\
  \\
  \textbf{1.~Prüfer:} & Prof.\,Dr.\,rer.\,nat.~Peter~Gerwinski \\
  % \textbf{2.~Prüfer:} & Prof.\,Dr.\,Ing.~Markus Lemmen
  % \textbf{Prüfer:} & Prof.\,Dr.\,rer.\,nat.~Jörg Frochte
\end{tabular}
